%%%%%%%%%%%%%%%%%%%%%%%%%%%%%%%%%%%%%%%%%
% University/School Laboratory Report
% LaTeX Template
% Version 3.1 (25/3/14)
%
% This template has been downloaded from:
% http://www.LaTeXTemplates.com
%
% Original author:
% Linux and Unix Users Group at Virginia Tech Wiki 
% (https://vtluug.org/wiki/Example_LaTeX_chem_lab_report)
%
% License:
% CC BY-NC-SA 3.0 (http://creativecommons.org/licenses/by-nc-sa/3.0/)
%
%%%%%%%%%%%%%%%%%%%%%%%%%%%%%%%%%%%%%%%%%

%----------------------------------------------------------------------------------------
%	PACKAGES AND DOCUMENT CONFIGURATIONS
%----------------------------------------------------------------------------------------

\documentclass[12pt]{article}

\usepackage{setspace}
\usepackage[version=3]{mhchem} % Package for chemical equation typesetting
\usepackage{siunitx} % Provides the \SI{}{} and \si{} command for typesetting SI units
\usepackage{graphicx} % Required for the inclusion of images
\usepackage{amsmath} % Required for some math elements 
%\usepackage[style=chem-acs]{biblatex}
%\usepackage[backend=bibtex,style=chem-acs,biblabel=dot]{biblatex}
\usepackage[backend=bibtex]{biblatex}
\addbibresource{biblatex-chem.bib}
\setlength\parindent{0pt} % Removes all indentation from paragraphs
\usepackage[top=0.8in, bottom= 0.8in, left= 0.8in, right= 0.8in]{geometry}
\usepackage{fancyhdr}

\pagestyle{fancy}
\renewcommand{\labelenumi}{\alph{enumi}.} % Make numbering in the enumerate environment by letter rather than number (e.g. section 6)

%\usepackage{times} % Uncomment to use the Times New Roman font

%----------------------------------------------------------------------------------------
%	DOCUMENT INFORMATION
%----------------------------------------------------------------------------------------

\title{Identification and Synthesis of \\  Unknown White Compound \\ \vspace{0.3 in} Chemisty 1065} % Title

\author{Cole \textsc{Nielsen}} % Author name
\date{Spring 2015} % Date for the report


\begin{document}

\maketitle % Insert the title, author and date

\begin{center}
\begin{tabular}{l r}
Dates Performed: & February 10, 17 \& 24, 2015 \\ % Date the experiment was performed
Partners: & Nickolas Johnson \\ % Partner names
& Erin Potter-Rank \\
& Isabel Eisenstadt \\
Instructor: & Che Wu % Instructor/supervisor
\end{tabular}
\end{center}

\pagebreak
% If you wish to include an abstract, uncomment the lines below
 \begin{abstract}\doublespacing
The purpose of this project was to identify a vial of unknown white compound (\(392_R\)) so that it could be properly disposed of. Additionally the chemical properties of the compound were determined and a sample of it was synthesized. The identity of the substance was determined to be magnesium sulfate (\(\ce{MgSO4}\)) using several lab techniques including flame tests, solubility tests, ion tests, a pH test, and several reactivity tests. The flame test yielded no color change in the flame, indicating the presence of magnesium or ammonium ions. The ion tests determined the compound to be a sulfate, eliminating halide, carbonate and ammonium containing compounds. The pH test determined the compound to be neutral pH. Finally, a series of reactivity tests involving sodium carbonate (\(\ce{Na2CO3}\)), magnesium chloride (\(\ce{MgCl2}\)), sodium acetate (\(\ce{NaC2H3O2}\)) and the unknown compound were performed. The reaction between the compound and sodium carbonate precipitated, indicating the cation to be magnesium. The other attempted reactions did not occur. The presence of magnesium and sulfate ions in the compound lead to the conclusion that the compound is magnesium sulfate. Lastly, it was attempted to synthesize 1 gram of magnesium sulfate by reacting magnesium metal with sulfuric acid. The yield of the synthesis was 1.200 g. The synthesized compound was then tested with the same techniques used on the original compound. All tests resulted in the same outcomes as the original compound, validating its identity and the UWC's identity. 
 \end{abstract}

%----------------------------------------------------------------------------------------
%	SECTION 1
%----------------------------------------------------------------------------------------

\nocite{*}

\printbibliography

\section{Introduction}\doublespacing

The objectives of this project were to identify an unknown compound, characterize it and then synthesize it. The synthesized compound was then characterized in order to verify the identity of the original compound, and the yield was determined. In this project, the compound was found to be magnesium sulfate, and this was proven via several experimental techniques described in the \textit{Experimental} section and analyzed in the \textit{Discussion} section. The overall outcomes of this experiment are discussed in the \textit{conclusion} section.  The importance of this work is that it shows experimental approaches to identify unknown chemical compounds so they can be correctly handled in situations such as their disposal. Similar experiments and work have been done for example in biology to identify unknown organic chemicals, such as amino acids, present in organisms[1] in order to better understand the organisms and biochemical mechanisms. Another similar example is the identification of unknown drugs in a person's body for use to produce forensic evidence for a crime being investigated by organizations such as the FBI[3].

% If you have more than one objective, uncomment the below:
%\begin{description}
%\item[First Objective] \hfill \\
%Objective 1 text
%\item[Second Objective] \hfill \\
%Objective 2 text
%\end{description}


%----------------------------------------------------------------------------------------
%	SECTION 2
%----------------------------------------------------------------------------------------

\section{Experimental}
This project included two major experimental parts, the first being the identification and characterization of the unknown white compound and the latter being the synthesis of this compound. The experimental process for each will be described below:

\subsection{Identification and Characterization}
\textbf{Flame test}
A flame test was performed to determine the cation of the UWC (unknown white compound). First, a solution containing 20 mL of deionized water and 0.200 grams of the UWC was prepared in a beaker. A bunsen burner was then connected to the gas line and ignited by match. A second beaker was filled with approximately 20 mL of hydrochloric acid (HCl). A nichrome wire was then cleaned by swirling it in the HCl. Finally, the nichrome wire was dipped into the UWC solution in order to coat it with ions. The ion coated wire was then placed into the top of the bBnsen burner flame to excite the ions. The color of the flame was recorded over three different trials of the experiment.

\vspace{6pt}\textbf{Solubility Test}
\\It was found from the flame test that the UWC is soluble in water as it dissociated into the deionized water.

\vspace{6pt}\textbf{Ion tests}
\\Four ion tests were performed to determine the presence of halide, ammonium, sulfate or carbonate ions in the UWC. A solution containing 10 mL of deionized water and 0.230 g of UWC in a 50 mL was prepared for these tests. Each ion test will be outlined below:
\vspace{6pt}\\\textit{Halides}: 1.0 mL of each 0.1M sodium nitrate (\ce{NaNO3}) 1 mL of 6M nitric acid (\ce{HNO3}) were measured using a 10 mL graduated cylinder and then were poured into a 50 mL beaker. Then 1.0 mL of the prepared UWC solution was measured out and placed into the beaker and was allowed to react
\vspace{10pt}\\
\textit{Ammonium}: 1 mL of 6M sodium hydroxide (\ce{NaOH}) was measured using a 10 mL graduated cylinder and was poured into a 50 mL beaker. 1 mL of the UWC solution was then measured and added to the 50 mL beaker and was allowed to react. 
\vspace{10pt}\\\textit{Sulfate}: 1 mL of each 6M of hydrochloric acid and 0.1M barium chloride (\ce{BaCl2}) were each measured out and placed in a 50 mL beaker. 1 mL of the UWC solution was again added to the beaker and was allowed to react.
\vspace{10pt}\\\textit{Carbonate}: 1 mL of the UWC was measured using a 10 mL graduated cylinder and was placed into a 50 mL beaker. Drops of 6M hydrochloric acid contained in a pipette were placed into the beaker. No reaction occurred.
\vspace{6pt}\\\textbf{pH Test}
\\A solution of 0.2 g of UWC and 30 mL of deionized water was prepared in a 50 mL beaker. Three trials were performed, each were a fresh pH strip was dipped into the solution, was allowed to change color and then was quantified against the pH scale on the pH strip tube. 
\vspace{6pt}\\\textbf{Reactivity Tests}
\\Three reactivity tests were performed on the UWC with other compounds to characterize it. The three compounds used were sodium carbonate (\ce{Na2CO3}), magnesium chloride (\ce{MgCl2}) and sodium acetate (\ce{NaC2H3O2}). Each of these will be outlined below
\vspace{6pt}\\\textit{Sodium carbonate}: A 20 mL solution of deionized water and 1.000 g of sodium carbonate was prepared in a 50 mL beaker. In a second 50 mL beaker a solution of 10 mL of deionized water and 0.100 g of UWC was prepared. The two solutions were mixed and were allowed to react.
\vspace{6pt}\\\textit{Magnesium chloride}: A 20 mL solution of deionized water and 0.500 g of magnesium chloride was prepared in a 50 mL beaker. In a second 50 mL beaker a solution of 20 mL of deionized water and 0.1 g of UWC was prepared. The two solutions were mixed and were allowed to react.
\vspace{6pt}\\\textit{Sodium acetate}: A 20 mL solution of deionized water and 0.500 g of sodium acetate was prepared in a 50 mL beaker. In a second 50 mL beaker a solution of 20 mL of deionized water and 0.100 g of UWC was prepared. The two solutions were mixed and allowed to react.
\subsection{Synthesis}
8.31 mL of 1M \ce{H2SO4} was measured out with a 10 mL graduated cylinder and was transferred to a 50 mL beaker. Approximately 0.5 g of magnesium metal strips was then obtained. The surface of each strip was then filed down to expose unoxidized magnesium. The strips were then cut up using scissors into small slivers. 0.202 g of the processed magnesium metal was then massed out and then poured into the beaker. The metal and acid were allowed to react until termination (all metal dissolved). The solution was then placed onto a hot plate on high and stirred until all the liquid evaporated. Next the product containing beaker was placed in the oven for 20 minutes to further remove water. The product containing beaker was then massed. The product was then scraped out of the beaker using a scoopula and was placed into a second beaker. The original beaker was then cleaned and re-massed. The mass of the product was found as the difference of the two masses. Lastly, the flame test, sulfate ion test and sodium carbonate reaction test were repeated on the synthesized compound. The same procedures were used as previously listed in the above \textit{Identification and Characterization} section.

\subsection{Disposal of Waste Chemicals}
After completing each test, non acidic and non basic solutions were poured into the appropriate disposal jugs. Acidic solutions were neutralized by slowly adding sodium bicarbonate until neutral (verified using pH strips). Basic solutions were neutralized by slowly adding hydrochloric acid until neutral. Those solutions were then poured into the proper disposal jugs. All glass wear was rinsed out with deionized water, dried with KIMTECH wipes and then returned to the proper drawers. Pipettes were disposed in the proper glass disposal bin.
\pagebreak
%----------------------------------------------------------------------------------------
%	SECTION 3
%----------------------------------------------------------------------------------------

\section{Results}\singlespacing
\subsection{Identification and Characterization}
\textbf{Flame test}
\\The compound yielded no change to the flame.
\begin{center}
\textit{Table 1.} Flame Test Results\\
\vspace{10pt}
\begin{tabular}{|c|c|}
\hline 
Trial  & Hue \\ 
\hline 
1 & No change in hue \\ 
\hline 
2 & No change in hue \\ 
\hline 
3 & No change in hue \\ 
\hline 
\end{tabular} 
\end{center}

\textbf{Solubility Test}
\\The UWC is soluble.
\vspace{6pt}\\
\textbf{Ion Tests}
\\Only the sulfate ion test resulted in a reaction, indicating the presence of \ce{SO4^2-} ions in the UWC.
\begin{center}
\textit{Table 2.} Ion Test Results\\
\vspace{10pt}
\begin{tabular}{|c|c|}
\hline 
Ion Test & Result \\ 
\hline 
Halides (\ce{Cl^-}) & No RXN \\ 
\hline 
Sulfate (\ce{SO4^2-}) & Precipitation \\ 
\hline 
Carbonate (\ce{CO3^2-}) & No RXN \\ 
\hline 
Ammonium (\ce{NH4^+}) & No RXN \\ 
\hline 
\end{tabular} 
\end{center}
\textbf{pH Test}
\\The compound is of neutral pH. $\mu$ = 6.33, $\sigma$ = 0.47.
\begin{center}
\textit{Table 3.} pH Test Results\\
\vspace{10pt}
\begin{tabular}{|c|c|}
\hline 
Trial & pH \\ 
\hline 
1 & 6 \\ 
\hline 
2 & 7 \\ 
\hline 
3 & 6 \\ 
\hline 
\end{tabular} 
\end{center}
\pagebreak
\textbf{Reaction Tests}
\\Only the reaction between the UWC and \ce{Na2CO3} occured.
\begin{center}
\textit{Table 4.} Ion Test Results\\
\vspace{10pt}
\begin{tabular}{|c|c|}
\hline 
Ion Test & Result \\ 
\hline 
UWC + \ce{Na2CO3} & Precipitation \\ 
\hline 
UWC + \ce{MgCl2} & No RXN \\ 
\hline 
UWC + \ce{NaC2H3O2} & No RXN \\ 
\hline 
\end{tabular} 
\end{center}
\subsection{Synthesis}
\textbf{Product Mass}
\begin{center}
\begin{tabular}{cc}
Measured Product Mass: & 1.200 g \\ 
\end{tabular} 
\end{center}
\textbf{Flame test}
\\No change in flame color. Same as UWC.
\begin{center}
\textit{Table 5.} Flame Test Results\\
\vspace{10pt}
\begin{tabular}{|c|c|}
\hline 
Trial  & Hue \\ 
\hline 
1 & No change in hue \\ 
\hline 
2 & No change in hue \\ 
\hline 
3 & No change in hue \\ 
\hline 
\end{tabular} 
\end{center}
\textbf{Reaction Test}
\\Reaction occured with \ce{Na2CO3}. Same as UWC.
\begin{center}
\textit{Table 6.} Reaction Test Results\\
\vspace{10pt}
\begin{tabular}{|c|c|}
\hline 
Ion Test & Result \\ 
\hline 
UWC + \ce{Na2CO3} & Precipitation \\ 
\hline 
\end{tabular} 
\end{center}
\textbf{Ion Tests}
\\Sulfate ion test reacted. Same as UWC.
\begin{center}
\textit{Table 7.} Ion Test Results\\
\vspace{10pt}
\begin{tabular}{|c|c|}
\hline 
Ion Test & Result \\ 
\hline 
Sulfate (\ce{SO4^2-}) & Precipitation \\ 
\hline 
\end{tabular} 
\end{center}
%----------------------------------------------------------------------------------------
%	SECTION 4
%----------------------------------------------------------------------------------------
\pagebreak
\section{Discussion}\doublespacing
The results of this experiment identify the UWC as magnesium sulfate (\ce{MgSO4}). The flame test first determined the cation of the UWC being magnesium or ammonium as no hue was observed[\textit{Table 1}]. This determination was made because the 3 other possible cations (sodium, potassium and calcium) emit easily observable hues[2] during the flame test, leaving magnesium and ammonium as the cation possibilities as they do not change a flame's color. There is a possibility that  the concentrations of the solutions used were too weak to give reliable results for this test. The cation was proved to be magnesium using other later discussed means, so this possibility is mostly insignificant, and these results actually support this identification.
\vspace{12pt}\\The identity of the UWC was further tested using four ion tests (sourced the Lab Archives Journal) designed to detect the presence of halide (\ce{Cl^-}), sulfate, carbonate and ammonium ions. These tests work on the premise that a reaction will only occur when the specified ion is present. For instance, the expected results for a positive sulfate or halide test is a precipitation, and the for an ammonium or carbonate test it is a gas evolution reaction, which will be observed as fizzing or an odor. Only the sulfate test on the UWC [\textit{Table 2}] reacted, forming a barium sulfate precipitate, which indicated the presence of \ce{SO4^2-} ions in the UWC. This result eliminated ammonium as a possible cation, suggesting magnesium as the cation. At this point the compound was first suspected to be magnesium sulfate. Th below equation describes the sulfate test reaction with the UWC (magnesium sulfate).
\begin{center}
\ce{MgSO4(aq) + BaCl2(aq) -> BaSO4(s) + MgCl2(aq)}
\end{center}
The identity of the UWC was then confirmed to be magnesium sulfate using reactivity tests with three different compounds. These tests are meant to confirm or deny the the suspected identity of the UWC by showing if the compound does or does not react as expected for that identity with a number of other compounds. The compounds used in these tests were sodium carbonate, magnesium chloride and sodium acetate. These compounds were chosen methodically, for example sodium carbonate as chosen because out of the possible compounds, only the magnesium containing compounds will react with it, and therefore prove or disprove the UWC as containing magnesium. This reaction occurs because magnesium carbonate forms, which is fairly insoluble in water (0.106 g per 100 mL) [4]. The equation for this reaction is below:
\begin{center}
\ce{MgSO4(aq) + Na2CO3(aq) -> MgSO4(s) + Na2SO4(aq)}
\end{center}
The other two compounds were chosen that they were not expected to react with magnesium sulfate. This was intended to verify the identity by affirming the reactions do not occur as expected. The only compound that reacted with the UWC was sodium carbonate [\textit{Table 4}], yielding a precipitation. This reaction confirmed the presence of magnesium as the cation in the UWC. Since the other two reactions did not occur, this indicates the UWC is not reactive with magnesium, chlorine and acetate ions found in the other compounds. No reaction was expected because all these ions are expected to remain soluble with added magnesium and sulfate ions from the UWC. This further validates the identity as magnesium sulfate. These reaction based tests are expected to be highly reliable as they were conducted with high purity chemicals of well composition, providing high certainty that these qualitative reactions were accurate.
\vspace{6pt}\\The pH and solubility test characterized the UWC and reaffirmed its identity as magnesium sulfate. The pH of the UWC was determined to be neutral [\textit{Table 3}], as expected for magnesium sulfate as it contains no hydroxide or hydrogen ions.
\vspace{6pt}\\Further tests to identify and and characterize the UWC, such as mass spectrometry[5], could be performed to determine the elemental make up of the UWC. The density of the compound could also be determined and used to determine the identity of the compound using pre-existing density tables. The results of the characterization and identification experiments may help others in approaching the process of compound identification by giving them an idea on techniques useful for compound identification. 
\vspace{6pt}\\Magnesium sulfate was synthesized in a reaction between magnesium metal and sulfuric acid, yielding hydrogen gas and aqueous magnesium sulfate:
\begin{center}
\ce{Mg(s) + H2SO4(aq) -> H2(g) + MgSO4(aq)}
\end{center}
The synthesis experiment was designed to produce 1.0 g (0.00831 mol) of magnesium sulfate, reacting 0.202 g of Mg (0.00831 mol) with 8.31 mL of 1M sulfuric acid (0.00831 mol). The final mass of the product, however, was measured as 1.200g. The amount of reactants were limited to only an amount for each (0.00831 mol) that would yield 1.0 gram of \ce{MgSO4} product (0.00831 mol), meaning that the excess mass is not due to excess compound formed. Also, the uncertainty of the equipment used is very small, being much less than the error (+20\%), so it is improbable that this discrepancy is due to it. Rather, it is logical that the excess mass is due to water molecules that were not removed from the compound during the dehydration process. The presence of water in the magnesium chloride despite attempted dehydration suggests it is hygroscopic, which is a chemical affinity to water that causes compounds to hold onto water molecules. Some quick research[6] yields that magnesium sulfate is hygroscopic, with a tendency to form heptahydrates with water, making it difficult to fully dehydrate it and accurately mass it. This hygroscopic nature of magnesium sulfate is a major limitation in determining the yield of this reaction, contributing to a significant error and level of uncertainty in the yield. The most probably way to ensure accurate yield determinations is to place the product in an oven for long periods to remove all water, however that was not feasible in this project's time frame.
\vspace{6pt}\\The synthesized compound shared the same characteristics previously determined with the UWC (magnesium chloride), seen in the \textit{Results} section. In particular, it was found that the compound emited no hue in the flame test, contained sulfates and reacted with sodium carbonate, which is identical to that found for the UWC. This helps to solidify the identity of the UWC as magnesium sulfate. 


%----------------------------------------------------------------------------------------
%	SECTION 5
%----------------------------------------------------------------------------------------

\section{Conclusion}

It was determined that the vial of unknown white compound, \(392_R\), is magnesium sulfate. This was determined using several lab techniques including flame tests, reaction tests, ion test and pH tests. The characteristics of the compound were also determined using these tests, such as the hygroscopic nature, neutral pH and reactivity with a number of compounds like sodium carbonate of magnesium sulfate. It was also determined that magnesium sulfate produces no color change in the flame ion test. These results carry significance not necessarily in the fact that they bring forth new, unknown data, but that they confirm and validate already known data, bringing higher certainty to what is already known.
\vspace{6pt}\\The most apparent source of uncertainty in this experiment is in the yield of the synthesis reaction, due to the hygoscopic nature of magnesium sulfate. This is because the compound is synthesized aqueously, and to obtain solid compound, the liquid must be boiled off. However, magnesium sulfate has a high affinity for water molecules, so it is difficult to remove all the water and the resulting solid will likely contain excess mass due to remaining water molecules, skewing the yield. The pH test was the only quantitative test performed, so it is the only set of data of which statistical uncertainty was determined (then mean $\mu$ was 6.33 and $\sigma$ was 0.47). All other performed tests were qualitative, so uncertainty is based of the potential of error in the experiments. These potential errors are believed to be insignificant as the chemicals used were of high purity and known nature, giving high certainty in the results of the performed reactions.

\section*{References} \nonumber
%----------------------------------------------------------------------------------------
%	BIBLIOGRAPHY
%----------------------------------------------------------------------------------------

%----------------------------------------------------------------------------------------
[1]\hspace{4ex} Donovan, S.; Stiefbold, S; Sprague, K. Chemical Properties of Amino Acids and 

\hspace{12ex}
Identification of Unknown Amino Acids. \textit{ABLE Vol 17.} \textbf{1996,} 35-70.
\vspace{6pt}\newline
[2]\hspace{4ex} Johnson, K.; Schreiner, R. A Dramatic Flame Test Demonstration. \textit{J. Chem. Ed.}

\hspace{12ex}\textit{Vol 78.} \textbf{2001,} 640-641.
\vspace{6pt}\newline
[3]\hspace{4ex} Al-Hetlani, E. Forensic drug analysis and microfluidics. \textit{Electrophoresis Vol 39.}

\hspace{12ex}\textbf{2013,} 1262-1272.

[4]\hspace{4ex} Lide, DR.\textit{ CRC Handbook of Chemistry and Physics;} Boca Raton, 1990-1991; 71st ed,

\hspace{12ex}pp 4-76.
\hspace{12ex}\textbf{2013,} 1262-1272.

[5]\hspace{4ex} The Mass Spectrometer. http://www.chemguide.co.uk/analysis/masspec/howitworks.html

\hspace{12ex}(accessed March 10, 2015).

[6]\hspace{4ex} The Mass Spectrometer. http://www.chemguide.co.uk/analysis/masspec/howitworks.html

\hspace{12ex}(accessed March 10, 2015).


\end{document}