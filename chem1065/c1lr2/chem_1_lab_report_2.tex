\documentclass[12pt]{article}

\usepackage{setspace}
\usepackage[version=3]{mhchem} % Package for chemical equation typesetting
\usepackage{siunitx} % Provides the \SI{}{} and \si{} command for typesetting SI units
\usepackage{graphicx} % Required for the inclusion of images
\usepackage{amsmath} % Required for some math elements 
%\usepackage[style=chem-acs]{biblatex}
%\usepackage[backend=bibtex,style=chem-acs,biblabel=dot]{biblatex}
\usepackage[backend=bibtex]{biblatex}
\addbibresource{biblatex-chem.bib}
\usepackage{subcaption}
\setlength\parindent{0pt} % Removes all indentation from paragraphs
\usepackage[top=0.8in, bottom= 0.8in, left= 0.8in, right= 0.8in]{geometry}
\usepackage{fancyhdr}
\usepackage{array}
\newcolumntype{L}{>{\centering\arraybackslash}m{3cm}}

\pagestyle{fancy}
\renewcommand{\labelenumi}{\alph{enumi}.} % Make numbering in the enumerate environment by letter rather than number (e.g. section 6)

%----------------------------------------------------------------------------------------
%	Cover Page
%----------------------------------------------------------------------------------------

\title{Dye Identification and Replication\\of an Artificially Colored Drink\\ Using Absorption Spectroscopy\\ \vspace{0.3 in} Chemisty 1065} % Title

\author{Author: Cole \textsc{Nielsen}} % Author name
\date{Spring 2015} % Date for the report


\begin{document}

\maketitle % Insert the title, author and date

\begin{center}
 \begin{tabular}{l r}
  Dates Performed: & March 10 \& 24, 2015 \\ % Date the experiment was performed
  Partners: & Nickolas Johnson \\ % Partner names
  & Erin Potter-Rank \\
  Instructor: & Che Wu % Instructor/supervisor
 \end{tabular}
\end{center}

\pagebreak
%-----------------------------------------------------------------------------------------
%   Abstract
%-----------------------------------------------------------------------------------------
\begin{abstract}\doublespacing
The food dye content of a purple artificially colored drink was determined to replicate the product's appearance. This was done first using a UV-Visible absorption spectrometer to determine the  absorption spectrum of the drink. Then, seven common FD\&C food dyes were tested in the same spectrometer, producing absorption spectra for each dye. The spectra of each dye was compared the drink's spectrum. It was found that the maximum absorption wavelength of Red 40 and Blue 1 matched the drink, indicating their presence in it. The absorption spectra for Red 40 and Blue 1 were measured at 20\%, 40\%, 60\%, and 80\% concentration of respective stock dyes solutions. A calibration curve for each dye was produced by plotting the absorption against concentration. From the calibration curve, the concentration of Red 40 in the original drink was determined to be $4.65 \times 10^{-5}$ M and Blue 1 was found to have a concentration of $8.69 \times 10^{-6}$ M. A 50 mL replica solution of the original drink was prepared by diluting and mixing together the given Red 40 and Blue 1 stock solution to those concentrations. Visual inspection yielded that the drink and replicated solution appeared identical. To confirm this, the replicated solution was tested in the spectrometer, producing a spectrum that closely matched the original drink, indicating replication to be an accurate copy of the drink within 8.7 \% of the original drink's absorption. 
\end{abstract}

%-----------------------------------------------------------------------------------------
%  Introduction
%-----------------------------------------------------------------------------------------
\section{Introduction}\doublespacing
The objective of this experiment was to replicate the the color of an artificially colored drink. In this experiment, the drink used was a bottle of purple grape flavored Kool-Aid. The dyes present in the drink were determined using a absorption spectrometry-based lab techniques described in the \textit{Experimental} section and analyzed in the \textit{Discussion} section. The process of replicating the drink is also discussed procedurally in the \textit{Experimental} section and analytically in the \textit{Discussion} section. In particular, it was predicted that dye concentration and light absorption of the dyes in this experiment are linearly correlated, following Beer's law$^4$. This hypothesis was used as the basis for determining the dye concentration and replicating the drink in this experiment. The outcomes and final conclusions are lastly discussed in the \textit{Conclusions} section. The significance of this work lies in one regard with the generic foods industry. In particular, the methodology used is invaluable in that it outlines the process to accurately replicate the color appearance of name brand products, so that convincing generics can be made and sold for profit. The technique of absorption spectrometry used also holds significance in that it can be used to analyze more than just dye content, for example elemental composition of a medium such as an atmosphere can be determined by analyzing the wavelengths absorbed by it$^{1}$. From those wavelengths, quantum mechanical principles (energy associated with the absorbed wavelengths and electron transitions) can be used to infer the constituent elements of the medium. Similar work to this experiment has been done by the Israeli Ministry of Health using paper chromatography$^{2}$ to determine dye content of foods. This was done by staining paper with different dye containing samples, and then allowing different solvents to run across the paper, causing different streaks to occur. Based on the streaks and the solvents, the dye content was determined understanding the solubility of common dyes.
%-----------------------------------------------------------------------------------------
%  Experimental
%-----------------------------------------------------------------------------------------
\section{Experimental}
This experiment was performed in two parts, being identification of the dyes in the drink and replication of the drink. These are discussed in this respective order.
\subsection*{Dye Identification}
To identify the dyes in the grape drink, the absorption spectra of the drink and several FD\&C dyes were determined. This was done using an Ocean Optics USB2000 UV/Vis absorption spectrometer interfaced by USB to a computer running Logger Pro. The spectrometer used was a small black and silver box approximately 4 inches wide, 8 inches long and an inch tall with a small slot to place a sample containing clear acrylic cuvette.  When a sample was loaded into the spectrometer in the cuvette, the absorption spectrum was obtained for that sample using the Logger Pro software. The grape drink was tested first, followed by the seven FD\&C dyes provided (Red 3, Red 40, Blue 1, Blue 2, Yellow 5, Yellow 6 and Green 3).\par
\pagebreak
\textbf{Grape Drink}\\
The spectrometer cuvette was rinsed out with deionized water and then filled with deionized water. The filled cuvette was placed in the spectrometer, and a software calibration was performed in Logger Pro to zero out the meter. Three 200 mL bottles of grape Kool-Aid were poured into a large beaker. The cuvette was then emptied out in the sink, rinsed with grape drink from the beaker, and filled to the top with the grape drink. The cuvette was again placed in the spectrometer. Logger Pro was used to obtain the absorption spectrum (light absorption vs. wavelength graph) for the drink, which was repeated three times for the sample. The wavelength and absorption value of the absorption peaks were recorded and averaged.

\vspace{6pt}\textbf{Food Dyes}\\
The procedure to determine the absorption spectrum of each dye tested was identical, so it will only be described once, but it should be understood that it is done separately for each dye (Red 3, Red 40, Blue 1, Blue 2, Yellow 5, Yellow 6 and Green 3). Initially, the cuvette was rinsed out with deionized water, then with the stock solution of the dye being tested, and was finally filled to the top with that dye. The cuvette was then placed in the Ocean Optics spectrometer. Using Logger Pro, the absorption spectrum of the dye was obtained. The absorption spectrum was obtained two additional times using the same process for that dye. The values of the wavelength and absorption for the peak of the absorption was noted for each trial, and were averaged.

\subsection*{Drink Replication}
To replicate the coloration of the drink, the concentration of the dyes in it were determined. The dyes have already been determined to be Red 40 and Blue 1 as future discussed in the \textit{Discussion} section. Both Red 40 and Blue 1 were tested with the absorption spectrometer at several different concentrations. The resulting data was used to produce the replicated solution. The procedure for concentration determination will be first discussed followed by the drink replication procedure.

\vspace{6pt}\textbf{Red 40 and Blue 1 Concentration}
Initially four 20 mL samples of both Red 40 and Blue 1 were prepared from stock solutions. These samples were made at 20\%, 40\%, 60\%, and 80\% of the concentration of the respective $9.0 \times 10^{-5}$ M Red 40 and $2.0 \times 10^{-5}$ M Blue 1 stock solutions. These samples were prepared by measuring out calculated volumes for each dilution of deionized water and dye using a 10 mL graduated cylinder into a 50 mL beaker. Each sample was thoroughly mixed by swirling the beaker around for approximately 10 seconds. The absorption spectrum of each dye concentration using the absorption spectrometer was then obtained. The procedure for obtaining the spectra of each dye was the same, so it will be described once. First, a cuvette was rinsed out with deionized water, then with the sample solution and was filled with the sample solution. The cuvette was placed in the UV/Vis absorption spectrometer, and the spectrum was recorded in Logger Pro. The spectrum was then recorded two additional times. The values for peak absorption and wavelength were noted, and averaged to produce a result. The process was repeated for the next sample.

\vspace{6pt}\textbf{Drink Replication}\\
25.8 mL of $9.0 \times 10^{-5}$ M Red 40 stock solution, 21.8 mL of $2.0 \times 10^{-5}$ M Blue 1 stock solution and 2.4 mL of deionized water were measured out using a 10 mL graduated cylinder and were combined together into a 150 mL beaker in that order, producing a 50 mL solution. This was calculated in the \textit{Discussion} section to have the same dye concentration as the original grape drink. The solution was thoroughly mixed by swirling the beaker around for approximately 10 seconds. Then, the spectrometer cuvette was rinsed out again with deionized water, then with the replicated solution and was filled with the replicated solution. The solution containing cuvette was placed into the spectrometer, and the absorption spectrum of it was collected. The spectrum was collected two more times, and the values for wavelength and absorption of the peak absorption were recorded and averaged.
\pagebreak
%-----------------------------------------------------------------------------------------
%  Results
%-----------------------------------------------------------------------------------------
\section{Results}\singlespacing
\subsection*{Original Drink}
\begin {figure}[htb!]
  \title{\textbf{Figure 1.} Two peak absorption spectrum of grape drink.}
  \begin{center}
    	\resizebox{0.6\textwidth}{!}{\input{original_drink.tex}}
  \end	{center}
\end {figure}
%
%\renewcommand*{\arraystretch}{1.5}
%\title{\textbf{Table 1.} Values for the peaks of the grape Kool-Aid absorption spectrum.}\vspace{-6pt}
%\begin{center}
%\begin{tabular}{|L|L|L|L|L|}
%\hline 
%• &Peak 1 Wavelength ($\lambda$) & Peak 1 Absorbance &Peak 2 Wavelength ($\lambda$) &Peak 2 Absorbance \\ 
%\hline 
%Mean ($\mu$) & 504.9 nm & 0.905 & 632.8 nm & 0.928 \\ 
%\hline 
%Standard Deviation ($\sigma$) & 1.01 nm & 0.00624 & 0.945 nm & 0.00458 \\ 
%\hline 
%\end{tabular} 
%\end{center}
The purple Kool-Aid was found to have two peaks of maximal absorption, as seen in \textit{Figure 1}. The first was determined to be at 504.9 nm $\pm$ 1.01 nm with an absorption of 0.905 $\pm$ 0.00624, and the second was at 632.8 $\pm$ 0.945 nm with a 0.928 $\pm$ 0.00458 absorption.\par
\subsection*{FD\&C Dyes}
This table contains the aggregated results for peak wavelength and absorption of the maxima of absorption each of the seven provided dyes. Each dye had a unique peak wavelength associated with it.\par\vspace{6pt}
\title{\textbf{Table 1.} Peak wavelength of 7 FD\&C dyes.}
\begin{center}
\begin{tabular}{|L|L|L|L|}
\hline 
Dye & Peak Wavelength & Absorbance & Concentration  \\ 
\hline 
Red 3 & 528.3 $\pm$ 0.46 nm & 0.740 & $1.6 \times 10^{-5}$ M \\
\hline 
Red 40 & 504.2 $\pm$ 0.2 nm & 1.719 & $9.0 \times 10^{-5}$ M \\ 
\hline 
Blue 1 & 632.4 $\pm$ 0.12 nm & 1.837 & $2.0 \times 10^{-5}$ M \\ 
\hline 
Blue 2 & 614.3 $\pm$ 1.0 nm & 1.195 & $9.0 \times 10^{-5}$ M \\ 
\hline 
Yellow 5 & 431.6 $\pm$ 1.6 nm & 0.646 & $3.0 \times 10^{-5}$ M  \\ 
\hline 
Yellow 6 & 485.9 $\pm$ 0.87 nm & 0.998 & $4.7 \times 10^{-5}$ M  \\ 
\hline 
Green 3 & 627.3 $\pm$ 0.76 nm & 0.262 & $2.8 \times 10^{-6}$ M  \\ 
\hline 
\end{tabular} 
\end{center}
%
\subsection*{Dyes Identified in Drink}
The maximum absorption peak values for wavelength and absorption for  Red 40 and Blue 1 at concentrations of 20\%, 40\%, 60\% and 80\% of the respective stock solution are shown in \textit{Table 2} and \textit{Table 3}. Absorption approximately follows a linear trend with respect to concentration.\par\vspace{6pt}
\title{\textbf{Table 2.} Concentration vs. Absorbance for Red 40.}
\begin{center}
\begin{tabular}{|c|c|c|}
\hline 
Dye Concentration & Absorbance & Wavelength \\ 
\hline 
$1.8 \time 10^-5$ M & 0.327 $\pm$ 0.0082 & 507.5 $\pm$ 1.1 nm \\ 
\hline 
$3.6 \time 10^-5$ M & 0.767 $\pm$ 0.0015 & 504.9 $\pm$ 0.12 nm \\ 
\hline 
$5.4 \time 10^-5$ M & 1.055 $\pm$ 0.026 & 503.7 $\pm$ 0.78 nm\\ 
\hline 
$7.2 \time 10^-5$ M & 1.374 $\pm$ 0.030 & 499.9 $\pm$ 0.90 nm\\ 
\hline 
\end{tabular} 
\end{center}
\title{\textbf{Table 3.} Concentration vs. Absorbance for Blue 1.}
\begin{center}
\begin{tabular}{|c|c|c|}
\hline 
Dye Concentration & Absorbance & Wavelength \\ 
\hline 
$4.0 \time 10^-6$ M & 0.358 $\pm$ 0.012 & 633.1 $\pm$ 0.68 nm \\ 
\hline 
$8.0 \time 10^-6$ M & 0.887 $\pm$ 0.024 & 632.2 $\pm$ 1.5 nm \\ 
\hline 
$1.2 \time 10^-5$ M & 1.273 $\pm$ 0.0091 & 632.2 $\pm$ 0.70 nm\\ 
\hline 
$1.6 \time 10^-5$ M & 1.753 $\pm$ 0.0066 & 632.0 $\pm$ 2.1 nm\\ 
\hline 
\end{tabular} 
\end{center}
\subsection*{Replicated Drink}
The replicated drink yielded two peaks, seen in \textit{Figure 2}, similar to the original drink.  The first peak was found to occur at a wavelength of 502.4 $\pm$ 0.231 nm with and absorption of 0.984 $\pm$ 0.00568 and the second occurring at a wavelength of 629.4 $\pm$ 1.10 nm with an absorption of 0.948 $\pm$ 0.00100.\par\vspace{6pt}
\title{\textbf{Figure 2.} Absorption spectrum of replicated drink.}
\begin {figure}[htb!]
  \begin{center}
    	\resizebox{0.6\textwidth}{!}{\input{replicated_drink.tex}}
  \end	{center}
\end {figure}
%
%\renewcommand*{\arraystretch}{1.5}
%\title{\textbf{Table 5.} Values for the peaks of the replicated drink's absorption spectrum.}\vspace{-6pt}
%\begin{center}
%\begin{tabular}{|L|L|L|L|L|}
%\hline 
%• &Peak 1 Wavelength ($\lambda$) & Peak 1 Absorbance &Peak 2 Wavelength ($\lambda$) &Peak 2 Absorbance \\ 
%\hline 
%Mean ($\mu$) & 502.4 nm & 0.984 & 629.4  nm & 0.948 \\ 
%\hline 
%Standard Deviation ($\sigma$) & 0.231 nm & 0.00568 & 1.10 nm & 0.00100 \\ 
%\hline 
%\end{tabular} 
%\end{center}
%-----------------------------------------------------------------------------------------
%  Discussion
%-----------------------------------------------------------------------------------------
\section{Discussion}\doublespacing
To replicate the color of the grape drink (known to be colored by FD\&C dyes), it was determined that the dyes present in the drink and their concentration must be found. Once known, replicating the drink is simply a matter of producing a solution containing the same concentration of the dyes. The first step in this process is identifying the dyes. This was done by testing the drink and seven common FD\&C dyes in a absorption spectrometer, which yielded a absorption spectrum for each sample. An absorption spectrum is a graph that relates the wavelength of light passing through some medium to the amount of light absorbed by it. In the case of this experiment, the drink and dyes were tested in the visible light spectrum (approximately 400 nm to 800 nm). It is expected that a given dye will have a unique spectrum associated with it, with the absorption peaking at a specific wavelength that is independent of concentration. It is also expected if a solution contains multiple dyes, like the grape drink, it will have multiple peaks of absorption of which peak wavelengths corresponds to each dye in it. Therefore, by looking at the peaks in the drink and those of several known dyes, it is possible to determine what dyes are in the drink. This is exactly what was done in this lab. In the \textit{Original Drink} subsection of the \textit{Results} section, the spectra of the grape drink tested is given in \textit{Figure 1}, which shows two peaks, one occurring at 504.9 $\pm$ 1.01 nm with an absorption of 0.905 $\pm$ 0.00624 and the second at 632.8 $\pm$ 0.945 nm with an absorption of 0.928 $\pm$ 0.00458. These values were determined as the average of three trials plus or minus the standard deviation. The peak wavelength and absorption for the seven FD\&C dyes (Red 3, Red 4, Blue, Blue 2, Yellow, Yellow 6, Green 3) found in the same manner for each dye are listed in \textit{Table 1}. It is apparent that the absorption of each dye is unique in the sense that each dye has a distinct peak absorption wavelength, which enable different dyes to effectively be identified.\par\vspace{6pt}
%
From inspection, it is apparent that the peak absorption wavelength of Red 40 (504.2 $\pm$ 0.2 nm) agrees with the value for the first peak of the drink (504.9 $\pm$ 1.0 nm), given the overlap due to the uncertainty (calculated as the standard deviation). This indicates the presence of Red 40 in the drink. The second peak of the drink (occurring at 632.8 $\pm$ 0.945 nm) agrees with Blue 1 (peak wavelength of 632.4 $\pm$ 0.12 nm), given the overlap in values indicates that the second dye is Blue 1. There are no other matching peaks, leading to the conclusion that the drink only contains Red 40 and Blue 1. A source of uncertainty in this determination is due to a small peak at approximately 410 nm, which could indicate another dye, however none of the stock solutions tested have a peak absorption that agrees with this wavelength. It is possible this is a result of other chemicals in the drink, or an extraneous peak caused by one  of the dyes. Another limitation of the data is that the peak absorption of Green 3 (627.3 nm) is very close to that of Blue 1 (632.4 nm), which may lead to error in the interpretation of the results. This does not seem to have affected the results of this experiment because the wavelength measured for the drink agrees with the wavelength with Blue 1 given the uncertainty, but it does not agree with Green 3's wavelength. A third source of uncertainty is due to contamination of test samples from using the same cuvette for all trials, however this was mitigated by rinsing out the cuvette with deionized water, and then with the test solution to minimize error from dilution and cross contamination between trials.
\vspace{12pt}\\
With the dye content of the drink known, it is possible to determine the concentration of each dye in it using a calibration curve. A calibration curve is simply a graph that correlates dye concentration to light absorption at a wavelength. A calibration curve for both Red 40 and Blue 1 one were determined experimentally by finding the peak absorption for both dyes at different concentration (20\%, 40\%, 60\% an 80\% of the stock solution concentration), shown in \textit{Table 2} for Red 40 and \textit{Table 3} for Blue 1. These values were plotted in a single graph for each dye, and a line of best fit (giving absorption as a function of concentration) was determined using a linear regression calculation. Linear regression was performed because a linear relationship in this case is expected due to Beer's law. Below are the calibration curves and the respective line of best fit for each dye.
\begin{tabular}{l r}
\end{tabular}
\begin{figure}[htb!]
  \begin{subfigure}[b]{0.5\textwidth}
  	\caption*{\textbf{Figure 3.} Red 40 Calibration Curve}
    	\resizebox{1\textwidth}{!}{% GNUPLOT: LaTeX picture
\setlength{\unitlength}{0.240900pt}
\ifx\plotpoint\undefined\newsavebox{\plotpoint}\fi
\begin{picture}(1500,900)(0,0)
\sbox{\plotpoint}{\rule[-0.200pt]{0.400pt}{0.400pt}}%
\put(171.0,131.0){\rule[-0.200pt]{4.818pt}{0.400pt}}
\put(151,131){\makebox(0,0)[r]{ 0}}
\put(1419.0,131.0){\rule[-0.200pt]{4.818pt}{0.400pt}}
\put(171.0,212.0){\rule[-0.200pt]{4.818pt}{0.400pt}}
\put(151,212){\makebox(0,0)[r]{ 0.2}}
\put(1419.0,212.0){\rule[-0.200pt]{4.818pt}{0.400pt}}
\put(171.0,292.0){\rule[-0.200pt]{4.818pt}{0.400pt}}
\put(151,292){\makebox(0,0)[r]{ 0.4}}
\put(1419.0,292.0){\rule[-0.200pt]{4.818pt}{0.400pt}}
\put(171.0,373.0){\rule[-0.200pt]{4.818pt}{0.400pt}}
\put(151,373){\makebox(0,0)[r]{ 0.6}}
\put(1419.0,373.0){\rule[-0.200pt]{4.818pt}{0.400pt}}
\put(171.0,454.0){\rule[-0.200pt]{4.818pt}{0.400pt}}
\put(151,454){\makebox(0,0)[r]{ 0.8}}
\put(1419.0,454.0){\rule[-0.200pt]{4.818pt}{0.400pt}}
\put(171.0,534.0){\rule[-0.200pt]{4.818pt}{0.400pt}}
\put(151,534){\makebox(0,0)[r]{ 1}}
\put(1419.0,534.0){\rule[-0.200pt]{4.818pt}{0.400pt}}
\put(171.0,615.0){\rule[-0.200pt]{4.818pt}{0.400pt}}
\put(151,615){\makebox(0,0)[r]{ 1.2}}
\put(1419.0,615.0){\rule[-0.200pt]{4.818pt}{0.400pt}}
\put(171.0,695.0){\rule[-0.200pt]{4.818pt}{0.400pt}}
\put(151,695){\makebox(0,0)[r]{ 1.4}}
\put(1419.0,695.0){\rule[-0.200pt]{4.818pt}{0.400pt}}
\put(171.0,776.0){\rule[-0.200pt]{4.818pt}{0.400pt}}
\put(151,776){\makebox(0,0)[r]{ 1.6}}
\put(1419.0,776.0){\rule[-0.200pt]{4.818pt}{0.400pt}}
\put(171.0,131.0){\rule[-0.200pt]{0.400pt}{4.818pt}}
\put(171,90){\makebox(0,0){0.00e+00}}
\put(171.0,756.0){\rule[-0.200pt]{0.400pt}{4.818pt}}
\put(425.0,131.0){\rule[-0.200pt]{0.400pt}{4.818pt}}
\put(425,90){\makebox(0,0){1.80e-05}}
\put(425.0,756.0){\rule[-0.200pt]{0.400pt}{4.818pt}}
\put(678.0,131.0){\rule[-0.200pt]{0.400pt}{4.818pt}}
\put(678,90){\makebox(0,0){3.60e-05}}
\put(678.0,756.0){\rule[-0.200pt]{0.400pt}{4.818pt}}
\put(932.0,131.0){\rule[-0.200pt]{0.400pt}{4.818pt}}
\put(932,90){\makebox(0,0){5.40e-05}}
\put(932.0,756.0){\rule[-0.200pt]{0.400pt}{4.818pt}}
\put(1185.0,131.0){\rule[-0.200pt]{0.400pt}{4.818pt}}
\put(1185,90){\makebox(0,0){7.20e-05}}
\put(1185.0,756.0){\rule[-0.200pt]{0.400pt}{4.818pt}}
\put(1439.0,131.0){\rule[-0.200pt]{0.400pt}{4.818pt}}
\put(1439,90){\makebox(0,0){9.00e-05}}
\put(1439.0,756.0){\rule[-0.200pt]{0.400pt}{4.818pt}}
\put(171.0,131.0){\rule[-0.200pt]{0.400pt}{155.380pt}}
\put(171.0,131.0){\rule[-0.200pt]{305.461pt}{0.400pt}}
\put(1439.0,131.0){\rule[-0.200pt]{0.400pt}{155.380pt}}
\put(171.0,776.0){\rule[-0.200pt]{305.461pt}{0.400pt}}
\put(30,453){\makebox(0,0){\hspace{-0.6in}Absorbance}}
\put(805,29){\makebox(0,0){Concentration (M)}}
\put(805,838){\makebox(0,0){Red 40 Calibration Curve}}
\put(425,263){\makebox(0,0){$+$}}
\put(678,440){\makebox(0,0){$+$}}
\put(932,556){\makebox(0,0){$+$}}
\put(1185,685){\makebox(0,0){$+$}}
\put(171,135){\usebox{\plotpoint}}
\multiput(171.00,135.59)(0.950,0.485){11}{\rule{0.843pt}{0.117pt}}
\multiput(171.00,134.17)(11.251,7.000){2}{\rule{0.421pt}{0.400pt}}
\multiput(184.00,142.59)(0.950,0.485){11}{\rule{0.843pt}{0.117pt}}
\multiput(184.00,141.17)(11.251,7.000){2}{\rule{0.421pt}{0.400pt}}
\multiput(197.00,149.59)(0.874,0.485){11}{\rule{0.786pt}{0.117pt}}
\multiput(197.00,148.17)(10.369,7.000){2}{\rule{0.393pt}{0.400pt}}
\multiput(209.00,156.59)(0.950,0.485){11}{\rule{0.843pt}{0.117pt}}
\multiput(209.00,155.17)(11.251,7.000){2}{\rule{0.421pt}{0.400pt}}
\multiput(222.00,163.59)(0.950,0.485){11}{\rule{0.843pt}{0.117pt}}
\multiput(222.00,162.17)(11.251,7.000){2}{\rule{0.421pt}{0.400pt}}
\multiput(235.00,170.59)(0.950,0.485){11}{\rule{0.843pt}{0.117pt}}
\multiput(235.00,169.17)(11.251,7.000){2}{\rule{0.421pt}{0.400pt}}
\multiput(248.00,177.59)(0.950,0.485){11}{\rule{0.843pt}{0.117pt}}
\multiput(248.00,176.17)(11.251,7.000){2}{\rule{0.421pt}{0.400pt}}
\multiput(261.00,184.59)(0.874,0.485){11}{\rule{0.786pt}{0.117pt}}
\multiput(261.00,183.17)(10.369,7.000){2}{\rule{0.393pt}{0.400pt}}
\multiput(273.00,191.59)(0.950,0.485){11}{\rule{0.843pt}{0.117pt}}
\multiput(273.00,190.17)(11.251,7.000){2}{\rule{0.421pt}{0.400pt}}
\multiput(286.00,198.59)(0.824,0.488){13}{\rule{0.750pt}{0.117pt}}
\multiput(286.00,197.17)(11.443,8.000){2}{\rule{0.375pt}{0.400pt}}
\multiput(299.00,206.59)(0.950,0.485){11}{\rule{0.843pt}{0.117pt}}
\multiput(299.00,205.17)(11.251,7.000){2}{\rule{0.421pt}{0.400pt}}
\multiput(312.00,213.59)(0.950,0.485){11}{\rule{0.843pt}{0.117pt}}
\multiput(312.00,212.17)(11.251,7.000){2}{\rule{0.421pt}{0.400pt}}
\multiput(325.00,220.59)(0.950,0.485){11}{\rule{0.843pt}{0.117pt}}
\multiput(325.00,219.17)(11.251,7.000){2}{\rule{0.421pt}{0.400pt}}
\multiput(338.00,227.59)(0.874,0.485){11}{\rule{0.786pt}{0.117pt}}
\multiput(338.00,226.17)(10.369,7.000){2}{\rule{0.393pt}{0.400pt}}
\multiput(350.00,234.59)(0.950,0.485){11}{\rule{0.843pt}{0.117pt}}
\multiput(350.00,233.17)(11.251,7.000){2}{\rule{0.421pt}{0.400pt}}
\multiput(363.00,241.59)(0.950,0.485){11}{\rule{0.843pt}{0.117pt}}
\multiput(363.00,240.17)(11.251,7.000){2}{\rule{0.421pt}{0.400pt}}
\multiput(376.00,248.59)(0.950,0.485){11}{\rule{0.843pt}{0.117pt}}
\multiput(376.00,247.17)(11.251,7.000){2}{\rule{0.421pt}{0.400pt}}
\multiput(389.00,255.59)(0.950,0.485){11}{\rule{0.843pt}{0.117pt}}
\multiput(389.00,254.17)(11.251,7.000){2}{\rule{0.421pt}{0.400pt}}
\multiput(402.00,262.59)(0.874,0.485){11}{\rule{0.786pt}{0.117pt}}
\multiput(402.00,261.17)(10.369,7.000){2}{\rule{0.393pt}{0.400pt}}
\multiput(414.00,269.59)(0.950,0.485){11}{\rule{0.843pt}{0.117pt}}
\multiput(414.00,268.17)(11.251,7.000){2}{\rule{0.421pt}{0.400pt}}
\multiput(427.00,276.59)(0.950,0.485){11}{\rule{0.843pt}{0.117pt}}
\multiput(427.00,275.17)(11.251,7.000){2}{\rule{0.421pt}{0.400pt}}
\multiput(440.00,283.59)(0.950,0.485){11}{\rule{0.843pt}{0.117pt}}
\multiput(440.00,282.17)(11.251,7.000){2}{\rule{0.421pt}{0.400pt}}
\multiput(453.00,290.59)(0.824,0.488){13}{\rule{0.750pt}{0.117pt}}
\multiput(453.00,289.17)(11.443,8.000){2}{\rule{0.375pt}{0.400pt}}
\multiput(466.00,298.59)(0.874,0.485){11}{\rule{0.786pt}{0.117pt}}
\multiput(466.00,297.17)(10.369,7.000){2}{\rule{0.393pt}{0.400pt}}
\multiput(478.00,305.59)(0.950,0.485){11}{\rule{0.843pt}{0.117pt}}
\multiput(478.00,304.17)(11.251,7.000){2}{\rule{0.421pt}{0.400pt}}
\multiput(491.00,312.59)(0.950,0.485){11}{\rule{0.843pt}{0.117pt}}
\multiput(491.00,311.17)(11.251,7.000){2}{\rule{0.421pt}{0.400pt}}
\multiput(504.00,319.59)(0.950,0.485){11}{\rule{0.843pt}{0.117pt}}
\multiput(504.00,318.17)(11.251,7.000){2}{\rule{0.421pt}{0.400pt}}
\multiput(517.00,326.59)(0.950,0.485){11}{\rule{0.843pt}{0.117pt}}
\multiput(517.00,325.17)(11.251,7.000){2}{\rule{0.421pt}{0.400pt}}
\multiput(530.00,333.59)(0.874,0.485){11}{\rule{0.786pt}{0.117pt}}
\multiput(530.00,332.17)(10.369,7.000){2}{\rule{0.393pt}{0.400pt}}
\multiput(542.00,340.59)(0.950,0.485){11}{\rule{0.843pt}{0.117pt}}
\multiput(542.00,339.17)(11.251,7.000){2}{\rule{0.421pt}{0.400pt}}
\multiput(555.00,347.59)(0.950,0.485){11}{\rule{0.843pt}{0.117pt}}
\multiput(555.00,346.17)(11.251,7.000){2}{\rule{0.421pt}{0.400pt}}
\multiput(568.00,354.59)(0.950,0.485){11}{\rule{0.843pt}{0.117pt}}
\multiput(568.00,353.17)(11.251,7.000){2}{\rule{0.421pt}{0.400pt}}
\multiput(581.00,361.59)(0.950,0.485){11}{\rule{0.843pt}{0.117pt}}
\multiput(581.00,360.17)(11.251,7.000){2}{\rule{0.421pt}{0.400pt}}
\multiput(594.00,368.59)(0.874,0.485){11}{\rule{0.786pt}{0.117pt}}
\multiput(594.00,367.17)(10.369,7.000){2}{\rule{0.393pt}{0.400pt}}
\multiput(606.00,375.59)(0.950,0.485){11}{\rule{0.843pt}{0.117pt}}
\multiput(606.00,374.17)(11.251,7.000){2}{\rule{0.421pt}{0.400pt}}
\multiput(619.00,382.59)(0.824,0.488){13}{\rule{0.750pt}{0.117pt}}
\multiput(619.00,381.17)(11.443,8.000){2}{\rule{0.375pt}{0.400pt}}
\multiput(632.00,390.59)(0.950,0.485){11}{\rule{0.843pt}{0.117pt}}
\multiput(632.00,389.17)(11.251,7.000){2}{\rule{0.421pt}{0.400pt}}
\multiput(645.00,397.59)(0.950,0.485){11}{\rule{0.843pt}{0.117pt}}
\multiput(645.00,396.17)(11.251,7.000){2}{\rule{0.421pt}{0.400pt}}
\multiput(658.00,404.59)(0.950,0.485){11}{\rule{0.843pt}{0.117pt}}
\multiput(658.00,403.17)(11.251,7.000){2}{\rule{0.421pt}{0.400pt}}
\multiput(671.00,411.59)(0.874,0.485){11}{\rule{0.786pt}{0.117pt}}
\multiput(671.00,410.17)(10.369,7.000){2}{\rule{0.393pt}{0.400pt}}
\multiput(683.00,418.59)(0.950,0.485){11}{\rule{0.843pt}{0.117pt}}
\multiput(683.00,417.17)(11.251,7.000){2}{\rule{0.421pt}{0.400pt}}
\multiput(696.00,425.59)(0.950,0.485){11}{\rule{0.843pt}{0.117pt}}
\multiput(696.00,424.17)(11.251,7.000){2}{\rule{0.421pt}{0.400pt}}
\multiput(709.00,432.59)(0.950,0.485){11}{\rule{0.843pt}{0.117pt}}
\multiput(709.00,431.17)(11.251,7.000){2}{\rule{0.421pt}{0.400pt}}
\multiput(722.00,439.59)(0.950,0.485){11}{\rule{0.843pt}{0.117pt}}
\multiput(722.00,438.17)(11.251,7.000){2}{\rule{0.421pt}{0.400pt}}
\multiput(735.00,446.59)(0.874,0.485){11}{\rule{0.786pt}{0.117pt}}
\multiput(735.00,445.17)(10.369,7.000){2}{\rule{0.393pt}{0.400pt}}
\multiput(747.00,453.59)(0.950,0.485){11}{\rule{0.843pt}{0.117pt}}
\multiput(747.00,452.17)(11.251,7.000){2}{\rule{0.421pt}{0.400pt}}
\multiput(760.00,460.59)(0.950,0.485){11}{\rule{0.843pt}{0.117pt}}
\multiput(760.00,459.17)(11.251,7.000){2}{\rule{0.421pt}{0.400pt}}
\multiput(773.00,467.59)(0.950,0.485){11}{\rule{0.843pt}{0.117pt}}
\multiput(773.00,466.17)(11.251,7.000){2}{\rule{0.421pt}{0.400pt}}
\multiput(786.00,474.59)(0.824,0.488){13}{\rule{0.750pt}{0.117pt}}
\multiput(786.00,473.17)(11.443,8.000){2}{\rule{0.375pt}{0.400pt}}
\multiput(799.00,482.59)(0.874,0.485){11}{\rule{0.786pt}{0.117pt}}
\multiput(799.00,481.17)(10.369,7.000){2}{\rule{0.393pt}{0.400pt}}
\multiput(811.00,489.59)(0.950,0.485){11}{\rule{0.843pt}{0.117pt}}
\multiput(811.00,488.17)(11.251,7.000){2}{\rule{0.421pt}{0.400pt}}
\multiput(824.00,496.59)(0.950,0.485){11}{\rule{0.843pt}{0.117pt}}
\multiput(824.00,495.17)(11.251,7.000){2}{\rule{0.421pt}{0.400pt}}
\multiput(837.00,503.59)(0.950,0.485){11}{\rule{0.843pt}{0.117pt}}
\multiput(837.00,502.17)(11.251,7.000){2}{\rule{0.421pt}{0.400pt}}
\multiput(850.00,510.59)(0.950,0.485){11}{\rule{0.843pt}{0.117pt}}
\multiput(850.00,509.17)(11.251,7.000){2}{\rule{0.421pt}{0.400pt}}
\multiput(863.00,517.59)(0.874,0.485){11}{\rule{0.786pt}{0.117pt}}
\multiput(863.00,516.17)(10.369,7.000){2}{\rule{0.393pt}{0.400pt}}
\multiput(875.00,524.59)(0.950,0.485){11}{\rule{0.843pt}{0.117pt}}
\multiput(875.00,523.17)(11.251,7.000){2}{\rule{0.421pt}{0.400pt}}
\multiput(888.00,531.59)(0.950,0.485){11}{\rule{0.843pt}{0.117pt}}
\multiput(888.00,530.17)(11.251,7.000){2}{\rule{0.421pt}{0.400pt}}
\multiput(901.00,538.59)(0.950,0.485){11}{\rule{0.843pt}{0.117pt}}
\multiput(901.00,537.17)(11.251,7.000){2}{\rule{0.421pt}{0.400pt}}
\multiput(914.00,545.59)(0.950,0.485){11}{\rule{0.843pt}{0.117pt}}
\multiput(914.00,544.17)(11.251,7.000){2}{\rule{0.421pt}{0.400pt}}
\multiput(927.00,552.59)(0.874,0.485){11}{\rule{0.786pt}{0.117pt}}
\multiput(927.00,551.17)(10.369,7.000){2}{\rule{0.393pt}{0.400pt}}
\multiput(939.00,559.59)(0.950,0.485){11}{\rule{0.843pt}{0.117pt}}
\multiput(939.00,558.17)(11.251,7.000){2}{\rule{0.421pt}{0.400pt}}
\multiput(952.00,566.59)(0.824,0.488){13}{\rule{0.750pt}{0.117pt}}
\multiput(952.00,565.17)(11.443,8.000){2}{\rule{0.375pt}{0.400pt}}
\multiput(965.00,574.59)(0.950,0.485){11}{\rule{0.843pt}{0.117pt}}
\multiput(965.00,573.17)(11.251,7.000){2}{\rule{0.421pt}{0.400pt}}
\multiput(978.00,581.59)(0.950,0.485){11}{\rule{0.843pt}{0.117pt}}
\multiput(978.00,580.17)(11.251,7.000){2}{\rule{0.421pt}{0.400pt}}
\multiput(991.00,588.59)(0.950,0.485){11}{\rule{0.843pt}{0.117pt}}
\multiput(991.00,587.17)(11.251,7.000){2}{\rule{0.421pt}{0.400pt}}
\multiput(1004.00,595.59)(0.874,0.485){11}{\rule{0.786pt}{0.117pt}}
\multiput(1004.00,594.17)(10.369,7.000){2}{\rule{0.393pt}{0.400pt}}
\multiput(1016.00,602.59)(0.950,0.485){11}{\rule{0.843pt}{0.117pt}}
\multiput(1016.00,601.17)(11.251,7.000){2}{\rule{0.421pt}{0.400pt}}
\multiput(1029.00,609.59)(0.950,0.485){11}{\rule{0.843pt}{0.117pt}}
\multiput(1029.00,608.17)(11.251,7.000){2}{\rule{0.421pt}{0.400pt}}
\multiput(1042.00,616.59)(0.950,0.485){11}{\rule{0.843pt}{0.117pt}}
\multiput(1042.00,615.17)(11.251,7.000){2}{\rule{0.421pt}{0.400pt}}
\multiput(1055.00,623.59)(0.950,0.485){11}{\rule{0.843pt}{0.117pt}}
\multiput(1055.00,622.17)(11.251,7.000){2}{\rule{0.421pt}{0.400pt}}
\multiput(1068.00,630.59)(0.874,0.485){11}{\rule{0.786pt}{0.117pt}}
\multiput(1068.00,629.17)(10.369,7.000){2}{\rule{0.393pt}{0.400pt}}
\multiput(1080.00,637.59)(0.950,0.485){11}{\rule{0.843pt}{0.117pt}}
\multiput(1080.00,636.17)(11.251,7.000){2}{\rule{0.421pt}{0.400pt}}
\multiput(1093.00,644.59)(0.950,0.485){11}{\rule{0.843pt}{0.117pt}}
\multiput(1093.00,643.17)(11.251,7.000){2}{\rule{0.421pt}{0.400pt}}
\multiput(1106.00,651.59)(0.950,0.485){11}{\rule{0.843pt}{0.117pt}}
\multiput(1106.00,650.17)(11.251,7.000){2}{\rule{0.421pt}{0.400pt}}
\multiput(1119.00,658.59)(0.824,0.488){13}{\rule{0.750pt}{0.117pt}}
\multiput(1119.00,657.17)(11.443,8.000){2}{\rule{0.375pt}{0.400pt}}
\multiput(1132.00,666.59)(0.874,0.485){11}{\rule{0.786pt}{0.117pt}}
\multiput(1132.00,665.17)(10.369,7.000){2}{\rule{0.393pt}{0.400pt}}
\multiput(1144.00,673.59)(0.950,0.485){11}{\rule{0.843pt}{0.117pt}}
\multiput(1144.00,672.17)(11.251,7.000){2}{\rule{0.421pt}{0.400pt}}
\multiput(1157.00,680.59)(0.950,0.485){11}{\rule{0.843pt}{0.117pt}}
\multiput(1157.00,679.17)(11.251,7.000){2}{\rule{0.421pt}{0.400pt}}
\multiput(1170.00,687.59)(0.950,0.485){11}{\rule{0.843pt}{0.117pt}}
\multiput(1170.00,686.17)(11.251,7.000){2}{\rule{0.421pt}{0.400pt}}
\multiput(1183.00,694.59)(0.950,0.485){11}{\rule{0.843pt}{0.117pt}}
\multiput(1183.00,693.17)(11.251,7.000){2}{\rule{0.421pt}{0.400pt}}
\multiput(1196.00,701.59)(0.874,0.485){11}{\rule{0.786pt}{0.117pt}}
\multiput(1196.00,700.17)(10.369,7.000){2}{\rule{0.393pt}{0.400pt}}
\multiput(1208.00,708.59)(0.950,0.485){11}{\rule{0.843pt}{0.117pt}}
\multiput(1208.00,707.17)(11.251,7.000){2}{\rule{0.421pt}{0.400pt}}
\multiput(1221.00,715.59)(0.950,0.485){11}{\rule{0.843pt}{0.117pt}}
\multiput(1221.00,714.17)(11.251,7.000){2}{\rule{0.421pt}{0.400pt}}
\multiput(1234.00,722.59)(0.950,0.485){11}{\rule{0.843pt}{0.117pt}}
\multiput(1234.00,721.17)(11.251,7.000){2}{\rule{0.421pt}{0.400pt}}
\multiput(1247.00,729.59)(0.950,0.485){11}{\rule{0.843pt}{0.117pt}}
\multiput(1247.00,728.17)(11.251,7.000){2}{\rule{0.421pt}{0.400pt}}
\multiput(1260.00,736.59)(0.874,0.485){11}{\rule{0.786pt}{0.117pt}}
\multiput(1260.00,735.17)(10.369,7.000){2}{\rule{0.393pt}{0.400pt}}
\multiput(1272.00,743.59)(0.950,0.485){11}{\rule{0.843pt}{0.117pt}}
\multiput(1272.00,742.17)(11.251,7.000){2}{\rule{0.421pt}{0.400pt}}
\multiput(1285.00,750.59)(0.824,0.488){13}{\rule{0.750pt}{0.117pt}}
\multiput(1285.00,749.17)(11.443,8.000){2}{\rule{0.375pt}{0.400pt}}
\multiput(1298.00,758.59)(0.950,0.485){11}{\rule{0.843pt}{0.117pt}}
\multiput(1298.00,757.17)(11.251,7.000){2}{\rule{0.421pt}{0.400pt}}
\multiput(1311.00,765.59)(0.950,0.485){11}{\rule{0.843pt}{0.117pt}}
\multiput(1311.00,764.17)(11.251,7.000){2}{\rule{0.421pt}{0.400pt}}
\multiput(1324.00,772.60)(0.920,0.468){5}{\rule{0.800pt}{0.113pt}}
\multiput(1324.00,771.17)(5.340,4.000){2}{\rule{0.400pt}{0.400pt}}
\put(171.0,131.0){\rule[-0.200pt]{0.400pt}{155.380pt}}
\put(171.0,131.0){\rule[-0.200pt]{305.461pt}{0.400pt}}
\put(1439.0,131.0){\rule[-0.200pt]{0.400pt}{155.380pt}}
\put(171.0,776.0){\rule[-0.200pt]{305.461pt}{0.400pt}}
\end{picture}
}
    \caption*{Absorbance = 19311C + 0.0094 \\ $R^2$ = 0.9954}
  \end{subfigure}
  %
  \hspace{0.25in}
  \begin{subfigure}[b]{0.5\textwidth}
  		\caption*{\textbf{Figure 4.} Blue 1 Calibration Curve.}
    	\resizebox{1\textwidth}{!}{% GNUPLOT: LaTeX picture
\setlength{\unitlength}{0.240900pt}
\ifx\plotpoint\undefined\newsavebox{\plotpoint}\fi
\begin{picture}(1500,900)(0,0)
\sbox{\plotpoint}{\rule[-0.200pt]{0.400pt}{0.400pt}}%
\put(171.0,131.0){\rule[-0.200pt]{4.818pt}{0.400pt}}
\put(151,131){\makebox(0,0)[r]{ 0}}
\put(1419.0,131.0){\rule[-0.200pt]{4.818pt}{0.400pt}}
\put(171.0,292.0){\rule[-0.200pt]{4.818pt}{0.400pt}}
\put(151,292){\makebox(0,0)[r]{ 0.5}}
\put(1419.0,292.0){\rule[-0.200pt]{4.818pt}{0.400pt}}
\put(171.0,454.0){\rule[-0.200pt]{4.818pt}{0.400pt}}
\put(151,454){\makebox(0,0)[r]{ 1}}
\put(1419.0,454.0){\rule[-0.200pt]{4.818pt}{0.400pt}}
\put(171.0,615.0){\rule[-0.200pt]{4.818pt}{0.400pt}}
\put(151,615){\makebox(0,0)[r]{ 1.5}}
\put(1419.0,615.0){\rule[-0.200pt]{4.818pt}{0.400pt}}
\put(171.0,776.0){\rule[-0.200pt]{4.818pt}{0.400pt}}
\put(151,776){\makebox(0,0)[r]{ 2}}
\put(1419.0,776.0){\rule[-0.200pt]{4.818pt}{0.400pt}}
\put(171.0,131.0){\rule[-0.200pt]{0.400pt}{4.818pt}}
\put(171,90){\makebox(0,0){0.00e+00}}
\put(171.0,756.0){\rule[-0.200pt]{0.400pt}{4.818pt}}
\put(488.0,131.0){\rule[-0.200pt]{0.400pt}{4.818pt}}
\put(488,90){\makebox(0,0){5.00e-06}}
\put(488.0,756.0){\rule[-0.200pt]{0.400pt}{4.818pt}}
\put(805.0,131.0){\rule[-0.200pt]{0.400pt}{4.818pt}}
\put(805,90){\makebox(0,0){1.00e-05}}
\put(805.0,756.0){\rule[-0.200pt]{0.400pt}{4.818pt}}
\put(1122.0,131.0){\rule[-0.200pt]{0.400pt}{4.818pt}}
\put(1122,90){\makebox(0,0){1.50e-05}}
\put(1122.0,756.0){\rule[-0.200pt]{0.400pt}{4.818pt}}
\put(1439.0,131.0){\rule[-0.200pt]{0.400pt}{4.818pt}}
\put(1439,90){\makebox(0,0){2.00e-05}}
\put(1439.0,756.0){\rule[-0.200pt]{0.400pt}{4.818pt}}
\put(171.0,131.0){\rule[-0.200pt]{0.400pt}{155.380pt}}
\put(171.0,131.0){\rule[-0.200pt]{305.461pt}{0.400pt}}
\put(1439.0,131.0){\rule[-0.200pt]{0.400pt}{155.380pt}}
\put(171.0,776.0){\rule[-0.200pt]{305.461pt}{0.400pt}}
\put(30,453){\makebox(0,0){\hspace{-0.5in}Absorbance}}
\put(805,29){\makebox(0,0){Concentration (M)}}
\put(805,838){\makebox(0,0){Blue 1 Calibration Curve}}
\put(425,246){\makebox(0,0){$+$}}
\put(678,417){\makebox(0,0){$+$}}
\put(932,542){\makebox(0,0){$+$}}
\put(1185,696){\makebox(0,0){$+$}}
\multiput(188.00,131.59)(0.933,0.477){7}{\rule{0.820pt}{0.115pt}}
\multiput(188.00,130.17)(7.298,5.000){2}{\rule{0.410pt}{0.400pt}}
\multiput(197.00,136.59)(0.874,0.485){11}{\rule{0.786pt}{0.117pt}}
\multiput(197.00,135.17)(10.369,7.000){2}{\rule{0.393pt}{0.400pt}}
\multiput(209.00,143.59)(0.950,0.485){11}{\rule{0.843pt}{0.117pt}}
\multiput(209.00,142.17)(11.251,7.000){2}{\rule{0.421pt}{0.400pt}}
\multiput(222.00,150.59)(0.950,0.485){11}{\rule{0.843pt}{0.117pt}}
\multiput(222.00,149.17)(11.251,7.000){2}{\rule{0.421pt}{0.400pt}}
\multiput(235.00,157.59)(0.824,0.488){13}{\rule{0.750pt}{0.117pt}}
\multiput(235.00,156.17)(11.443,8.000){2}{\rule{0.375pt}{0.400pt}}
\multiput(248.00,165.59)(0.950,0.485){11}{\rule{0.843pt}{0.117pt}}
\multiput(248.00,164.17)(11.251,7.000){2}{\rule{0.421pt}{0.400pt}}
\multiput(261.00,172.59)(0.874,0.485){11}{\rule{0.786pt}{0.117pt}}
\multiput(261.00,171.17)(10.369,7.000){2}{\rule{0.393pt}{0.400pt}}
\multiput(273.00,179.59)(0.950,0.485){11}{\rule{0.843pt}{0.117pt}}
\multiput(273.00,178.17)(11.251,7.000){2}{\rule{0.421pt}{0.400pt}}
\multiput(286.00,186.59)(0.950,0.485){11}{\rule{0.843pt}{0.117pt}}
\multiput(286.00,185.17)(11.251,7.000){2}{\rule{0.421pt}{0.400pt}}
\multiput(299.00,193.59)(0.824,0.488){13}{\rule{0.750pt}{0.117pt}}
\multiput(299.00,192.17)(11.443,8.000){2}{\rule{0.375pt}{0.400pt}}
\multiput(312.00,201.59)(0.950,0.485){11}{\rule{0.843pt}{0.117pt}}
\multiput(312.00,200.17)(11.251,7.000){2}{\rule{0.421pt}{0.400pt}}
\multiput(325.00,208.59)(0.950,0.485){11}{\rule{0.843pt}{0.117pt}}
\multiput(325.00,207.17)(11.251,7.000){2}{\rule{0.421pt}{0.400pt}}
\multiput(338.00,215.59)(0.874,0.485){11}{\rule{0.786pt}{0.117pt}}
\multiput(338.00,214.17)(10.369,7.000){2}{\rule{0.393pt}{0.400pt}}
\multiput(350.00,222.59)(0.950,0.485){11}{\rule{0.843pt}{0.117pt}}
\multiput(350.00,221.17)(11.251,7.000){2}{\rule{0.421pt}{0.400pt}}
\multiput(363.00,229.59)(0.824,0.488){13}{\rule{0.750pt}{0.117pt}}
\multiput(363.00,228.17)(11.443,8.000){2}{\rule{0.375pt}{0.400pt}}
\multiput(376.00,237.59)(0.950,0.485){11}{\rule{0.843pt}{0.117pt}}
\multiput(376.00,236.17)(11.251,7.000){2}{\rule{0.421pt}{0.400pt}}
\multiput(389.00,244.59)(0.950,0.485){11}{\rule{0.843pt}{0.117pt}}
\multiput(389.00,243.17)(11.251,7.000){2}{\rule{0.421pt}{0.400pt}}
\multiput(402.00,251.59)(0.874,0.485){11}{\rule{0.786pt}{0.117pt}}
\multiput(402.00,250.17)(10.369,7.000){2}{\rule{0.393pt}{0.400pt}}
\multiput(414.00,258.59)(0.950,0.485){11}{\rule{0.843pt}{0.117pt}}
\multiput(414.00,257.17)(11.251,7.000){2}{\rule{0.421pt}{0.400pt}}
\multiput(427.00,265.59)(0.824,0.488){13}{\rule{0.750pt}{0.117pt}}
\multiput(427.00,264.17)(11.443,8.000){2}{\rule{0.375pt}{0.400pt}}
\multiput(440.00,273.59)(0.950,0.485){11}{\rule{0.843pt}{0.117pt}}
\multiput(440.00,272.17)(11.251,7.000){2}{\rule{0.421pt}{0.400pt}}
\multiput(453.00,280.59)(0.950,0.485){11}{\rule{0.843pt}{0.117pt}}
\multiput(453.00,279.17)(11.251,7.000){2}{\rule{0.421pt}{0.400pt}}
\multiput(466.00,287.59)(0.874,0.485){11}{\rule{0.786pt}{0.117pt}}
\multiput(466.00,286.17)(10.369,7.000){2}{\rule{0.393pt}{0.400pt}}
\multiput(478.00,294.59)(0.950,0.485){11}{\rule{0.843pt}{0.117pt}}
\multiput(478.00,293.17)(11.251,7.000){2}{\rule{0.421pt}{0.400pt}}
\multiput(491.00,301.59)(0.824,0.488){13}{\rule{0.750pt}{0.117pt}}
\multiput(491.00,300.17)(11.443,8.000){2}{\rule{0.375pt}{0.400pt}}
\multiput(504.00,309.59)(0.950,0.485){11}{\rule{0.843pt}{0.117pt}}
\multiput(504.00,308.17)(11.251,7.000){2}{\rule{0.421pt}{0.400pt}}
\multiput(517.00,316.59)(0.950,0.485){11}{\rule{0.843pt}{0.117pt}}
\multiput(517.00,315.17)(11.251,7.000){2}{\rule{0.421pt}{0.400pt}}
\multiput(530.00,323.59)(0.874,0.485){11}{\rule{0.786pt}{0.117pt}}
\multiput(530.00,322.17)(10.369,7.000){2}{\rule{0.393pt}{0.400pt}}
\multiput(542.00,330.59)(0.950,0.485){11}{\rule{0.843pt}{0.117pt}}
\multiput(542.00,329.17)(11.251,7.000){2}{\rule{0.421pt}{0.400pt}}
\multiput(555.00,337.59)(0.824,0.488){13}{\rule{0.750pt}{0.117pt}}
\multiput(555.00,336.17)(11.443,8.000){2}{\rule{0.375pt}{0.400pt}}
\multiput(568.00,345.59)(0.950,0.485){11}{\rule{0.843pt}{0.117pt}}
\multiput(568.00,344.17)(11.251,7.000){2}{\rule{0.421pt}{0.400pt}}
\multiput(581.00,352.59)(0.950,0.485){11}{\rule{0.843pt}{0.117pt}}
\multiput(581.00,351.17)(11.251,7.000){2}{\rule{0.421pt}{0.400pt}}
\multiput(594.00,359.59)(0.874,0.485){11}{\rule{0.786pt}{0.117pt}}
\multiput(594.00,358.17)(10.369,7.000){2}{\rule{0.393pt}{0.400pt}}
\multiput(606.00,366.59)(0.950,0.485){11}{\rule{0.843pt}{0.117pt}}
\multiput(606.00,365.17)(11.251,7.000){2}{\rule{0.421pt}{0.400pt}}
\multiput(619.00,373.59)(0.824,0.488){13}{\rule{0.750pt}{0.117pt}}
\multiput(619.00,372.17)(11.443,8.000){2}{\rule{0.375pt}{0.400pt}}
\multiput(632.00,381.59)(0.950,0.485){11}{\rule{0.843pt}{0.117pt}}
\multiput(632.00,380.17)(11.251,7.000){2}{\rule{0.421pt}{0.400pt}}
\multiput(645.00,388.59)(0.950,0.485){11}{\rule{0.843pt}{0.117pt}}
\multiput(645.00,387.17)(11.251,7.000){2}{\rule{0.421pt}{0.400pt}}
\multiput(658.00,395.59)(0.950,0.485){11}{\rule{0.843pt}{0.117pt}}
\multiput(658.00,394.17)(11.251,7.000){2}{\rule{0.421pt}{0.400pt}}
\multiput(671.00,402.59)(0.874,0.485){11}{\rule{0.786pt}{0.117pt}}
\multiput(671.00,401.17)(10.369,7.000){2}{\rule{0.393pt}{0.400pt}}
\multiput(683.00,409.59)(0.824,0.488){13}{\rule{0.750pt}{0.117pt}}
\multiput(683.00,408.17)(11.443,8.000){2}{\rule{0.375pt}{0.400pt}}
\multiput(696.00,417.59)(0.950,0.485){11}{\rule{0.843pt}{0.117pt}}
\multiput(696.00,416.17)(11.251,7.000){2}{\rule{0.421pt}{0.400pt}}
\multiput(709.00,424.59)(0.950,0.485){11}{\rule{0.843pt}{0.117pt}}
\multiput(709.00,423.17)(11.251,7.000){2}{\rule{0.421pt}{0.400pt}}
\multiput(722.00,431.59)(0.950,0.485){11}{\rule{0.843pt}{0.117pt}}
\multiput(722.00,430.17)(11.251,7.000){2}{\rule{0.421pt}{0.400pt}}
\multiput(735.00,438.59)(0.874,0.485){11}{\rule{0.786pt}{0.117pt}}
\multiput(735.00,437.17)(10.369,7.000){2}{\rule{0.393pt}{0.400pt}}
\multiput(747.00,445.59)(0.824,0.488){13}{\rule{0.750pt}{0.117pt}}
\multiput(747.00,444.17)(11.443,8.000){2}{\rule{0.375pt}{0.400pt}}
\multiput(760.00,453.59)(0.950,0.485){11}{\rule{0.843pt}{0.117pt}}
\multiput(760.00,452.17)(11.251,7.000){2}{\rule{0.421pt}{0.400pt}}
\multiput(773.00,460.59)(0.950,0.485){11}{\rule{0.843pt}{0.117pt}}
\multiput(773.00,459.17)(11.251,7.000){2}{\rule{0.421pt}{0.400pt}}
\multiput(786.00,467.59)(0.950,0.485){11}{\rule{0.843pt}{0.117pt}}
\multiput(786.00,466.17)(11.251,7.000){2}{\rule{0.421pt}{0.400pt}}
\multiput(799.00,474.59)(0.874,0.485){11}{\rule{0.786pt}{0.117pt}}
\multiput(799.00,473.17)(10.369,7.000){2}{\rule{0.393pt}{0.400pt}}
\multiput(811.00,481.59)(0.824,0.488){13}{\rule{0.750pt}{0.117pt}}
\multiput(811.00,480.17)(11.443,8.000){2}{\rule{0.375pt}{0.400pt}}
\multiput(824.00,489.59)(0.950,0.485){11}{\rule{0.843pt}{0.117pt}}
\multiput(824.00,488.17)(11.251,7.000){2}{\rule{0.421pt}{0.400pt}}
\multiput(837.00,496.59)(0.950,0.485){11}{\rule{0.843pt}{0.117pt}}
\multiput(837.00,495.17)(11.251,7.000){2}{\rule{0.421pt}{0.400pt}}
\multiput(850.00,503.59)(0.950,0.485){11}{\rule{0.843pt}{0.117pt}}
\multiput(850.00,502.17)(11.251,7.000){2}{\rule{0.421pt}{0.400pt}}
\multiput(863.00,510.59)(0.874,0.485){11}{\rule{0.786pt}{0.117pt}}
\multiput(863.00,509.17)(10.369,7.000){2}{\rule{0.393pt}{0.400pt}}
\multiput(875.00,517.59)(0.824,0.488){13}{\rule{0.750pt}{0.117pt}}
\multiput(875.00,516.17)(11.443,8.000){2}{\rule{0.375pt}{0.400pt}}
\multiput(888.00,525.59)(0.950,0.485){11}{\rule{0.843pt}{0.117pt}}
\multiput(888.00,524.17)(11.251,7.000){2}{\rule{0.421pt}{0.400pt}}
\multiput(901.00,532.59)(0.950,0.485){11}{\rule{0.843pt}{0.117pt}}
\multiput(901.00,531.17)(11.251,7.000){2}{\rule{0.421pt}{0.400pt}}
\multiput(914.00,539.59)(0.950,0.485){11}{\rule{0.843pt}{0.117pt}}
\multiput(914.00,538.17)(11.251,7.000){2}{\rule{0.421pt}{0.400pt}}
\multiput(927.00,546.59)(0.874,0.485){11}{\rule{0.786pt}{0.117pt}}
\multiput(927.00,545.17)(10.369,7.000){2}{\rule{0.393pt}{0.400pt}}
\multiput(939.00,553.59)(0.824,0.488){13}{\rule{0.750pt}{0.117pt}}
\multiput(939.00,552.17)(11.443,8.000){2}{\rule{0.375pt}{0.400pt}}
\multiput(952.00,561.59)(0.950,0.485){11}{\rule{0.843pt}{0.117pt}}
\multiput(952.00,560.17)(11.251,7.000){2}{\rule{0.421pt}{0.400pt}}
\multiput(965.00,568.59)(0.950,0.485){11}{\rule{0.843pt}{0.117pt}}
\multiput(965.00,567.17)(11.251,7.000){2}{\rule{0.421pt}{0.400pt}}
\multiput(978.00,575.59)(0.950,0.485){11}{\rule{0.843pt}{0.117pt}}
\multiput(978.00,574.17)(11.251,7.000){2}{\rule{0.421pt}{0.400pt}}
\multiput(991.00,582.59)(0.950,0.485){11}{\rule{0.843pt}{0.117pt}}
\multiput(991.00,581.17)(11.251,7.000){2}{\rule{0.421pt}{0.400pt}}
\multiput(1004.00,589.59)(0.758,0.488){13}{\rule{0.700pt}{0.117pt}}
\multiput(1004.00,588.17)(10.547,8.000){2}{\rule{0.350pt}{0.400pt}}
\multiput(1016.00,597.59)(0.950,0.485){11}{\rule{0.843pt}{0.117pt}}
\multiput(1016.00,596.17)(11.251,7.000){2}{\rule{0.421pt}{0.400pt}}
\multiput(1029.00,604.59)(0.950,0.485){11}{\rule{0.843pt}{0.117pt}}
\multiput(1029.00,603.17)(11.251,7.000){2}{\rule{0.421pt}{0.400pt}}
\multiput(1042.00,611.59)(0.950,0.485){11}{\rule{0.843pt}{0.117pt}}
\multiput(1042.00,610.17)(11.251,7.000){2}{\rule{0.421pt}{0.400pt}}
\multiput(1055.00,618.59)(0.950,0.485){11}{\rule{0.843pt}{0.117pt}}
\multiput(1055.00,617.17)(11.251,7.000){2}{\rule{0.421pt}{0.400pt}}
\multiput(1068.00,625.59)(0.758,0.488){13}{\rule{0.700pt}{0.117pt}}
\multiput(1068.00,624.17)(10.547,8.000){2}{\rule{0.350pt}{0.400pt}}
\multiput(1080.00,633.59)(0.950,0.485){11}{\rule{0.843pt}{0.117pt}}
\multiput(1080.00,632.17)(11.251,7.000){2}{\rule{0.421pt}{0.400pt}}
\multiput(1093.00,640.59)(0.950,0.485){11}{\rule{0.843pt}{0.117pt}}
\multiput(1093.00,639.17)(11.251,7.000){2}{\rule{0.421pt}{0.400pt}}
\multiput(1106.00,647.59)(0.950,0.485){11}{\rule{0.843pt}{0.117pt}}
\multiput(1106.00,646.17)(11.251,7.000){2}{\rule{0.421pt}{0.400pt}}
\multiput(1119.00,654.59)(0.950,0.485){11}{\rule{0.843pt}{0.117pt}}
\multiput(1119.00,653.17)(11.251,7.000){2}{\rule{0.421pt}{0.400pt}}
\multiput(1132.00,661.59)(0.758,0.488){13}{\rule{0.700pt}{0.117pt}}
\multiput(1132.00,660.17)(10.547,8.000){2}{\rule{0.350pt}{0.400pt}}
\multiput(1144.00,669.59)(0.950,0.485){11}{\rule{0.843pt}{0.117pt}}
\multiput(1144.00,668.17)(11.251,7.000){2}{\rule{0.421pt}{0.400pt}}
\multiput(1157.00,676.59)(0.950,0.485){11}{\rule{0.843pt}{0.117pt}}
\multiput(1157.00,675.17)(11.251,7.000){2}{\rule{0.421pt}{0.400pt}}
\multiput(1170.00,683.59)(0.950,0.485){11}{\rule{0.843pt}{0.117pt}}
\multiput(1170.00,682.17)(11.251,7.000){2}{\rule{0.421pt}{0.400pt}}
\multiput(1183.00,690.59)(0.950,0.485){11}{\rule{0.843pt}{0.117pt}}
\multiput(1183.00,689.17)(11.251,7.000){2}{\rule{0.421pt}{0.400pt}}
\multiput(1196.00,697.59)(0.758,0.488){13}{\rule{0.700pt}{0.117pt}}
\multiput(1196.00,696.17)(10.547,8.000){2}{\rule{0.350pt}{0.400pt}}
\multiput(1208.00,705.59)(0.950,0.485){11}{\rule{0.843pt}{0.117pt}}
\multiput(1208.00,704.17)(11.251,7.000){2}{\rule{0.421pt}{0.400pt}}
\multiput(1221.00,712.59)(0.950,0.485){11}{\rule{0.843pt}{0.117pt}}
\multiput(1221.00,711.17)(11.251,7.000){2}{\rule{0.421pt}{0.400pt}}
\multiput(1234.00,719.59)(0.950,0.485){11}{\rule{0.843pt}{0.117pt}}
\multiput(1234.00,718.17)(11.251,7.000){2}{\rule{0.421pt}{0.400pt}}
\multiput(1247.00,726.59)(0.950,0.485){11}{\rule{0.843pt}{0.117pt}}
\multiput(1247.00,725.17)(11.251,7.000){2}{\rule{0.421pt}{0.400pt}}
\multiput(1260.00,733.59)(0.758,0.488){13}{\rule{0.700pt}{0.117pt}}
\multiput(1260.00,732.17)(10.547,8.000){2}{\rule{0.350pt}{0.400pt}}
\multiput(1272.00,741.59)(0.950,0.485){11}{\rule{0.843pt}{0.117pt}}
\multiput(1272.00,740.17)(11.251,7.000){2}{\rule{0.421pt}{0.400pt}}
\multiput(1285.00,748.59)(0.950,0.485){11}{\rule{0.843pt}{0.117pt}}
\multiput(1285.00,747.17)(11.251,7.000){2}{\rule{0.421pt}{0.400pt}}
\multiput(1298.00,755.59)(0.950,0.485){11}{\rule{0.843pt}{0.117pt}}
\multiput(1298.00,754.17)(11.251,7.000){2}{\rule{0.421pt}{0.400pt}}
\multiput(1311.00,762.59)(0.950,0.485){11}{\rule{0.843pt}{0.117pt}}
\multiput(1311.00,761.17)(11.251,7.000){2}{\rule{0.421pt}{0.400pt}}
\multiput(1324.00,769.59)(0.798,0.485){11}{\rule{0.729pt}{0.117pt}}
\multiput(1324.00,768.17)(9.488,7.000){2}{\rule{0.364pt}{0.400pt}}
\put(171.0,131.0){\rule[-0.200pt]{0.400pt}{155.380pt}}
\put(171.0,131.0){\rule[-0.200pt]{305.461pt}{0.400pt}}
\put(1439.0,131.0){\rule[-0.200pt]{0.400pt}{155.380pt}}
\put(171.0,776.0){\rule[-0.200pt]{305.461pt}{0.400pt}}
\end{picture}
}
    	\caption*{Absorbance = 110525C - 0.03\\$R^2$ = 0.9971}
  \end{subfigure}
\end{figure}\\
Using the calibration curve best fit lines, the absorption associated with each dye in the grape drink can be substituted into the respective equation and solved for concentration (C). Since the coefficient of determination ($R^2$) for both fitted lines is very close to 1, this indicates that the line fits the data well and there is little uncertainty associated with it.  For Red 40 of the drink, the associated absorption was 0.905 $\pm$ 0.00624, mathematically solving for C:
\begin{equation}
0.905 = 19311C + 0.0094 \hspace{12pt}\implies\hspace{12pt} C = \frac{0.905-0.0094}{19311} = 4.64 \times 10^{-5} M 
\end{equation}
This gives the Red 40 concentration to be $4.64 \times 10^{-5}$ M. The concentration for Blue 1 is solved for in a similar manner:
\begin{equation}
0.928 = 110525C - 0.03 \hspace{12pt}\implies\hspace{12pt} C = \frac{0.928+0.03}{110525} = 8.67 \times 10^{-6} M 
\end{equation} 
This gives the Blue 1 concentration to be $8.67 \times 10^{-6}$ M. With both concentrations known, a replica solution can be produced. It was arbitrarily chosen to make a 50 mL replica solution. Given the $9.0 \time 10^{-5}$ M Red 40 stock solution, the amount of full concentration Red 40 dye needed for a 50 mL is calculated as follows:
\begin{equation}
M_{red}V_{red} = M_{red,f}V_{solution} \hspace{12pt}\implies\hspace{12pt} V_{red} = \frac{M_{red,f}V_{solution}}{M_{red}}
\end{equation}
\begin{equation}
V_{red} = \frac{4.64 \times 10^{-5} \times 50.0 }{9.0 \times 10^{-5}} = \text{25.8 mL}
\end{equation}
Therefore 25.8 mL of Red 40 stock solution is needed. This is done the same way for the Blue 1 solution:
\begin{equation}
V_{blue} = \frac{M_{blue,f}V_{solution}}{M_{blue}} = \frac{8.67 \times 10^{-6} \times 50.0 }{2.0 \times 10^{-5}} = \text{21.8 mL}
\end{equation}
Therefore 21.8 mL of Blue 1 stock solution is needed. Finally, the excess volume required to bring the total volume of the solution to 50 mL will be calculated. This volume will be composed of deionized water as it has a negligible effect on light absorption in the visible spectrum. The water volume is found as follows:
\begin{equation}
V_{total} = V_{red} + V_{blue} + V_{water} \hspace{12pt}\implies\hspace{12pt} V_{water} = 50 - 21.8 - 25.8 = \text{2.4 mL}
\end{equation} 
The 50 mL replica solution was prepared by combining these volumes of stock dye solution and deionized water. The replica solution was then tested in the spectrometer to obtain its absorption spectrum. Below in \textit{Table 4} are the peak absorption values from the original drink, the replicated drink and the calculated error.\\
\renewcommand*{\arraystretch}{1.5}
\title{\textbf{Table 4.} Original and Replicated Drink Comparison}\vspace{-6pt}
\begin{center}
\begin{tabular}{|L|L|L|L|L|}
\hline 
• &Peak 1 Wavelength ($\lambda$) & Peak 1 Absorbance &Peak 2 Wavelength ($\lambda$) &Peak 2 Absorbance \\ 
\hline 
Grape Drink & 504.9 nm & 0.905 & 632.8  nm & 0.928 \\ 
\hline 
Replication & 502.4 nm & 0.984 & 629.4  nm& 0.948 \\ 
\hline 
\% Error & -0.50 \% & +8.7 \% & -0.54 \% & +2.2 \%\\ 
\hline 
\end{tabular} 
\end{center}
The wavelength of the peaks agree with each other given their uncertainties and the resolution of the spectrometer, asserting the dyes in both to be identical. However, there is a level of error between the peak absorption of the original and replicated drink. This is such that the replication's absorption is higher than the grape drink's for both dye peaks. This trend is visibly shown below in \textit{Figure 5}, where the absorption spectrum of both samples are plotted together. The upper curve is the replication and the lower is the original drink:\\
\title{\textbf{Figure 5.} Absorption spectrum of replicated drink.}
\begin {figure}[htb!]
  \begin{center}
    	\resizebox{0.6\textwidth}{!}{% GNUPLOT: LaTeX picture
\setlength{\unitlength}{0.240900pt}
\ifx\plotpoint\undefined\newsavebox{\plotpoint}\fi
\sbox{\plotpoint}{\rule[-0.200pt]{0.400pt}{0.400pt}}%
\begin{picture}(1500,900)(0,0)
\sbox{\plotpoint}{\rule[-0.200pt]{0.400pt}{0.400pt}}%
\put(171.0,131.0){\rule[-0.200pt]{4.818pt}{0.400pt}}
\put(151,131){\makebox(0,0)[r]{ 0}}
\put(1419.0,131.0){\rule[-0.200pt]{4.818pt}{0.400pt}}
\put(171.0,239.0){\rule[-0.200pt]{4.818pt}{0.400pt}}
\put(151,239){\makebox(0,0)[r]{ 0.2}}
\put(1419.0,239.0){\rule[-0.200pt]{4.818pt}{0.400pt}}
\put(171.0,346.0){\rule[-0.200pt]{4.818pt}{0.400pt}}
\put(151,346){\makebox(0,0)[r]{ 0.4}}
\put(1419.0,346.0){\rule[-0.200pt]{4.818pt}{0.400pt}}
\put(171.0,454.0){\rule[-0.200pt]{4.818pt}{0.400pt}}
\put(151,454){\makebox(0,0)[r]{ 0.6}}
\put(1419.0,454.0){\rule[-0.200pt]{4.818pt}{0.400pt}}
\put(171.0,561.0){\rule[-0.200pt]{4.818pt}{0.400pt}}
\put(151,561){\makebox(0,0)[r]{ 0.8}}
\put(1419.0,561.0){\rule[-0.200pt]{4.818pt}{0.400pt}}
\put(171.0,669.0){\rule[-0.200pt]{4.818pt}{0.400pt}}
\put(151,669){\makebox(0,0)[r]{ 1}}
\put(1419.0,669.0){\rule[-0.200pt]{4.818pt}{0.400pt}}
\put(171.0,776.0){\rule[-0.200pt]{4.818pt}{0.400pt}}
\put(151,776){\makebox(0,0)[r]{ 1.2}}
\put(1419.0,776.0){\rule[-0.200pt]{4.818pt}{0.400pt}}
\put(171.0,131.0){\rule[-0.200pt]{0.400pt}{4.818pt}}
\put(171,90){\makebox(0,0){ 400}}
\put(171.0,756.0){\rule[-0.200pt]{0.400pt}{4.818pt}}
\put(330.0,131.0){\rule[-0.200pt]{0.400pt}{4.818pt}}
\put(330,90){\makebox(0,0){ 450}}
\put(330.0,756.0){\rule[-0.200pt]{0.400pt}{4.818pt}}
\put(488.0,131.0){\rule[-0.200pt]{0.400pt}{4.818pt}}
\put(488,90){\makebox(0,0){ 500}}
\put(488.0,756.0){\rule[-0.200pt]{0.400pt}{4.818pt}}
\put(647.0,131.0){\rule[-0.200pt]{0.400pt}{4.818pt}}
\put(647,90){\makebox(0,0){ 550}}
\put(647.0,756.0){\rule[-0.200pt]{0.400pt}{4.818pt}}
\put(805.0,131.0){\rule[-0.200pt]{0.400pt}{4.818pt}}
\put(805,90){\makebox(0,0){ 600}}
\put(805.0,756.0){\rule[-0.200pt]{0.400pt}{4.818pt}}
\put(964.0,131.0){\rule[-0.200pt]{0.400pt}{4.818pt}}
\put(964,90){\makebox(0,0){ 650}}
\put(964.0,756.0){\rule[-0.200pt]{0.400pt}{4.818pt}}
\put(1122.0,131.0){\rule[-0.200pt]{0.400pt}{4.818pt}}
\put(1122,90){\makebox(0,0){ 700}}
\put(1122.0,756.0){\rule[-0.200pt]{0.400pt}{4.818pt}}
\put(1281.0,131.0){\rule[-0.200pt]{0.400pt}{4.818pt}}
\put(1281,90){\makebox(0,0){ 750}}
\put(1281.0,756.0){\rule[-0.200pt]{0.400pt}{4.818pt}}
\put(1439.0,131.0){\rule[-0.200pt]{0.400pt}{4.818pt}}
\put(1439,90){\makebox(0,0){ 800}}
\put(1439.0,756.0){\rule[-0.200pt]{0.400pt}{4.818pt}}
\put(171.0,131.0){\rule[-0.200pt]{0.400pt}{155.380pt}}
\put(171.0,131.0){\rule[-0.200pt]{305.461pt}{0.400pt}}
\put(1439.0,131.0){\rule[-0.200pt]{0.400pt}{155.380pt}}
\put(171.0,776.0){\rule[-0.200pt]{305.461pt}{0.400pt}}
\put(30,453){\makebox(0,0){Absorption}}
\put(805,29){\makebox(0,0){Wavelength (nm)}}
\put(805,838){\makebox(0,0){Original vs Replicated Drink Absorption}}
\put(171,284.67){\rule{0.241pt}{0.400pt}}
\multiput(171.00,284.17)(0.500,1.000){2}{\rule{0.120pt}{0.400pt}}
\put(172,286){\usebox{\plotpoint}}
\put(171.67,286){\rule{0.400pt}{0.482pt}}
\multiput(171.17,286.00)(1.000,1.000){2}{\rule{0.400pt}{0.241pt}}
\put(174,286.67){\rule{0.241pt}{0.400pt}}
\multiput(174.00,287.17)(0.500,-1.000){2}{\rule{0.120pt}{0.400pt}}
\put(174.67,287){\rule{0.400pt}{0.482pt}}
\multiput(174.17,287.00)(1.000,1.000){2}{\rule{0.400pt}{0.241pt}}
\put(173.0,288.0){\usebox{\plotpoint}}
\put(176,289){\usebox{\plotpoint}}
\put(175.67,289){\rule{0.400pt}{0.482pt}}
\multiput(175.17,289.00)(1.000,1.000){2}{\rule{0.400pt}{0.241pt}}
\put(177,290.67){\rule{0.241pt}{0.400pt}}
\multiput(177.00,290.17)(0.500,1.000){2}{\rule{0.120pt}{0.400pt}}
\put(177.67,289){\rule{0.400pt}{0.723pt}}
\multiput(177.17,289.00)(1.000,1.500){2}{\rule{0.400pt}{0.361pt}}
\put(178.67,288){\rule{0.400pt}{0.964pt}}
\multiput(178.17,290.00)(1.000,-2.000){2}{\rule{0.400pt}{0.482pt}}
\put(178.0,289.0){\rule[-0.200pt]{0.400pt}{0.723pt}}
\put(179.67,293){\rule{0.400pt}{0.482pt}}
\multiput(179.17,294.00)(1.000,-1.000){2}{\rule{0.400pt}{0.241pt}}
\put(180.67,293){\rule{0.400pt}{0.482pt}}
\multiput(180.17,293.00)(1.000,1.000){2}{\rule{0.400pt}{0.241pt}}
\put(180.0,288.0){\rule[-0.200pt]{0.400pt}{1.686pt}}
\put(182.0,293.0){\rule[-0.200pt]{0.400pt}{0.482pt}}
\put(183,291.67){\rule{0.241pt}{0.400pt}}
\multiput(183.00,292.17)(0.500,-1.000){2}{\rule{0.120pt}{0.400pt}}
\put(182.0,293.0){\usebox{\plotpoint}}
\put(184,294.67){\rule{0.241pt}{0.400pt}}
\multiput(184.00,295.17)(0.500,-1.000){2}{\rule{0.120pt}{0.400pt}}
\put(184.0,292.0){\rule[-0.200pt]{0.400pt}{0.964pt}}
\put(186,294.67){\rule{0.241pt}{0.400pt}}
\multiput(186.00,294.17)(0.500,1.000){2}{\rule{0.120pt}{0.400pt}}
\put(186.67,296){\rule{0.400pt}{0.482pt}}
\multiput(186.17,296.00)(1.000,1.000){2}{\rule{0.400pt}{0.241pt}}
\put(185.0,295.0){\usebox{\plotpoint}}
\put(187.67,297){\rule{0.400pt}{0.723pt}}
\multiput(187.17,298.50)(1.000,-1.500){2}{\rule{0.400pt}{0.361pt}}
\put(188.67,297){\rule{0.400pt}{0.723pt}}
\multiput(188.17,297.00)(1.000,1.500){2}{\rule{0.400pt}{0.361pt}}
\put(188.0,298.0){\rule[-0.200pt]{0.400pt}{0.482pt}}
\put(190.0,300.0){\usebox{\plotpoint}}
\put(191,297.67){\rule{0.241pt}{0.400pt}}
\multiput(191.00,298.17)(0.500,-1.000){2}{\rule{0.120pt}{0.400pt}}
\put(191.67,298){\rule{0.400pt}{0.482pt}}
\multiput(191.17,298.00)(1.000,1.000){2}{\rule{0.400pt}{0.241pt}}
\put(191.0,299.0){\usebox{\plotpoint}}
\put(193.0,300.0){\usebox{\plotpoint}}
\put(194,300.67){\rule{0.241pt}{0.400pt}}
\multiput(194.00,300.17)(0.500,1.000){2}{\rule{0.120pt}{0.400pt}}
\put(193.0,301.0){\usebox{\plotpoint}}
\put(195,302){\usebox{\plotpoint}}
\put(195,300.67){\rule{0.241pt}{0.400pt}}
\multiput(195.00,301.17)(0.500,-1.000){2}{\rule{0.120pt}{0.400pt}}
\put(196,300.67){\rule{0.241pt}{0.400pt}}
\multiput(196.00,300.17)(0.500,1.000){2}{\rule{0.120pt}{0.400pt}}
\put(197,302){\usebox{\plotpoint}}
\put(198,300.67){\rule{0.241pt}{0.400pt}}
\multiput(198.00,301.17)(0.500,-1.000){2}{\rule{0.120pt}{0.400pt}}
\put(197.0,302.0){\usebox{\plotpoint}}
\put(199,300.67){\rule{0.241pt}{0.400pt}}
\multiput(199.00,301.17)(0.500,-1.000){2}{\rule{0.120pt}{0.400pt}}
\put(200,300.67){\rule{0.241pt}{0.400pt}}
\multiput(200.00,300.17)(0.500,1.000){2}{\rule{0.120pt}{0.400pt}}
\put(199.0,301.0){\usebox{\plotpoint}}
\put(201,302){\usebox{\plotpoint}}
\put(201,301.67){\rule{0.241pt}{0.400pt}}
\multiput(201.00,301.17)(0.500,1.000){2}{\rule{0.120pt}{0.400pt}}
\put(202,301.67){\rule{0.241pt}{0.400pt}}
\multiput(202.00,302.17)(0.500,-1.000){2}{\rule{0.120pt}{0.400pt}}
\put(203,302){\usebox{\plotpoint}}
\put(205,301.67){\rule{0.241pt}{0.400pt}}
\multiput(205.00,301.17)(0.500,1.000){2}{\rule{0.120pt}{0.400pt}}
\put(206,301.67){\rule{0.241pt}{0.400pt}}
\multiput(206.00,302.17)(0.500,-1.000){2}{\rule{0.120pt}{0.400pt}}
\put(203.0,302.0){\rule[-0.200pt]{0.482pt}{0.400pt}}
\put(207,302){\usebox{\plotpoint}}
\put(207,301.67){\rule{0.241pt}{0.400pt}}
\multiput(207.00,301.17)(0.500,1.000){2}{\rule{0.120pt}{0.400pt}}
\put(208,302.67){\rule{0.241pt}{0.400pt}}
\multiput(208.00,302.17)(0.500,1.000){2}{\rule{0.120pt}{0.400pt}}
\put(209,301.67){\rule{0.241pt}{0.400pt}}
\multiput(209.00,302.17)(0.500,-1.000){2}{\rule{0.120pt}{0.400pt}}
\put(209.0,303.0){\usebox{\plotpoint}}
\put(210.67,300){\rule{0.400pt}{0.482pt}}
\multiput(210.17,301.00)(1.000,-1.000){2}{\rule{0.400pt}{0.241pt}}
\put(212,299.67){\rule{0.241pt}{0.400pt}}
\multiput(212.00,299.17)(0.500,1.000){2}{\rule{0.120pt}{0.400pt}}
\put(210.0,302.0){\usebox{\plotpoint}}
\put(213,299.67){\rule{0.241pt}{0.400pt}}
\multiput(213.00,299.17)(0.500,1.000){2}{\rule{0.120pt}{0.400pt}}
\put(213.67,299){\rule{0.400pt}{0.482pt}}
\multiput(213.17,300.00)(1.000,-1.000){2}{\rule{0.400pt}{0.241pt}}
\put(213.0,300.0){\usebox{\plotpoint}}
\put(215.0,299.0){\usebox{\plotpoint}}
\put(216,298.67){\rule{0.241pt}{0.400pt}}
\multiput(216.00,299.17)(0.500,-1.000){2}{\rule{0.120pt}{0.400pt}}
\put(215.0,300.0){\usebox{\plotpoint}}
\put(217,299){\usebox{\plotpoint}}
\put(218,297.67){\rule{0.241pt}{0.400pt}}
\multiput(218.00,298.17)(0.500,-1.000){2}{\rule{0.120pt}{0.400pt}}
\put(217.0,299.0){\usebox{\plotpoint}}
\put(219,297.67){\rule{0.241pt}{0.400pt}}
\multiput(219.00,298.17)(0.500,-1.000){2}{\rule{0.120pt}{0.400pt}}
\put(220,296.67){\rule{0.241pt}{0.400pt}}
\multiput(220.00,297.17)(0.500,-1.000){2}{\rule{0.120pt}{0.400pt}}
\put(219.0,298.0){\usebox{\plotpoint}}
\put(221,297){\usebox{\plotpoint}}
\put(221,295.67){\rule{0.241pt}{0.400pt}}
\multiput(221.00,296.17)(0.500,-1.000){2}{\rule{0.120pt}{0.400pt}}
\put(223.67,294){\rule{0.400pt}{0.482pt}}
\multiput(223.17,295.00)(1.000,-1.000){2}{\rule{0.400pt}{0.241pt}}
\put(222.0,296.0){\rule[-0.200pt]{0.482pt}{0.400pt}}
\put(225,293.67){\rule{0.241pt}{0.400pt}}
\multiput(225.00,294.17)(0.500,-1.000){2}{\rule{0.120pt}{0.400pt}}
\put(225.0,294.0){\usebox{\plotpoint}}
\put(227,292.67){\rule{0.241pt}{0.400pt}}
\multiput(227.00,293.17)(0.500,-1.000){2}{\rule{0.120pt}{0.400pt}}
\put(226.0,294.0){\usebox{\plotpoint}}
\put(228.0,293.0){\usebox{\plotpoint}}
\put(229,290.67){\rule{0.241pt}{0.400pt}}
\multiput(229.00,290.17)(0.500,1.000){2}{\rule{0.120pt}{0.400pt}}
\put(229.0,291.0){\rule[-0.200pt]{0.400pt}{0.482pt}}
\put(230.0,292.0){\usebox{\plotpoint}}
\put(231.0,290.0){\rule[-0.200pt]{0.400pt}{0.482pt}}
\put(231.0,290.0){\rule[-0.200pt]{0.482pt}{0.400pt}}
\put(233.0,289.0){\usebox{\plotpoint}}
\put(234,287.67){\rule{0.241pt}{0.400pt}}
\multiput(234.00,288.17)(0.500,-1.000){2}{\rule{0.120pt}{0.400pt}}
\put(233.0,289.0){\usebox{\plotpoint}}
\put(237,286.67){\rule{0.241pt}{0.400pt}}
\multiput(237.00,287.17)(0.500,-1.000){2}{\rule{0.120pt}{0.400pt}}
\put(235.0,288.0){\rule[-0.200pt]{0.482pt}{0.400pt}}
\put(238.0,286.0){\usebox{\plotpoint}}
\put(239,284.67){\rule{0.241pt}{0.400pt}}
\multiput(239.00,285.17)(0.500,-1.000){2}{\rule{0.120pt}{0.400pt}}
\put(238.0,286.0){\usebox{\plotpoint}}
\put(240,285){\usebox{\plotpoint}}
\put(241.67,283){\rule{0.400pt}{0.482pt}}
\multiput(241.17,284.00)(1.000,-1.000){2}{\rule{0.400pt}{0.241pt}}
\put(240.0,285.0){\rule[-0.200pt]{0.482pt}{0.400pt}}
\put(243.0,283.0){\usebox{\plotpoint}}
\put(244.0,283.0){\usebox{\plotpoint}}
\put(244.67,282){\rule{0.400pt}{0.482pt}}
\multiput(244.17,283.00)(1.000,-1.000){2}{\rule{0.400pt}{0.241pt}}
\put(244.0,284.0){\usebox{\plotpoint}}
\put(246.0,281.0){\usebox{\plotpoint}}
\put(246.0,281.0){\rule[-0.200pt]{0.482pt}{0.400pt}}
\put(248,280.67){\rule{0.241pt}{0.400pt}}
\multiput(248.00,281.17)(0.500,-1.000){2}{\rule{0.120pt}{0.400pt}}
\put(248.0,281.0){\usebox{\plotpoint}}
\put(249.0,281.0){\usebox{\plotpoint}}
\put(249.67,279){\rule{0.400pt}{0.482pt}}
\multiput(249.17,279.00)(1.000,1.000){2}{\rule{0.400pt}{0.241pt}}
\put(251,279.67){\rule{0.241pt}{0.400pt}}
\multiput(251.00,280.17)(0.500,-1.000){2}{\rule{0.120pt}{0.400pt}}
\put(250.0,279.0){\rule[-0.200pt]{0.400pt}{0.482pt}}
\put(252,277.67){\rule{0.241pt}{0.400pt}}
\multiput(252.00,278.17)(0.500,-1.000){2}{\rule{0.120pt}{0.400pt}}
\put(253,276.67){\rule{0.241pt}{0.400pt}}
\multiput(253.00,277.17)(0.500,-1.000){2}{\rule{0.120pt}{0.400pt}}
\put(252.0,279.0){\usebox{\plotpoint}}
\put(254,276.67){\rule{0.241pt}{0.400pt}}
\multiput(254.00,277.17)(0.500,-1.000){2}{\rule{0.120pt}{0.400pt}}
\put(255,276.67){\rule{0.241pt}{0.400pt}}
\multiput(255.00,276.17)(0.500,1.000){2}{\rule{0.120pt}{0.400pt}}
\put(254.0,277.0){\usebox{\plotpoint}}
\put(256.0,277.0){\usebox{\plotpoint}}
\put(258,275.67){\rule{0.241pt}{0.400pt}}
\multiput(258.00,276.17)(0.500,-1.000){2}{\rule{0.120pt}{0.400pt}}
\put(259,274.67){\rule{0.241pt}{0.400pt}}
\multiput(259.00,275.17)(0.500,-1.000){2}{\rule{0.120pt}{0.400pt}}
\put(256.0,277.0){\rule[-0.200pt]{0.482pt}{0.400pt}}
\put(260,275){\usebox{\plotpoint}}
\put(261,274.67){\rule{0.241pt}{0.400pt}}
\multiput(261.00,274.17)(0.500,1.000){2}{\rule{0.120pt}{0.400pt}}
\put(260.0,275.0){\usebox{\plotpoint}}
\put(262.0,275.0){\usebox{\plotpoint}}
\put(263,273.67){\rule{0.241pt}{0.400pt}}
\multiput(263.00,274.17)(0.500,-1.000){2}{\rule{0.120pt}{0.400pt}}
\put(262.0,275.0){\usebox{\plotpoint}}
\put(264,274){\usebox{\plotpoint}}
\put(264.0,274.0){\rule[-0.200pt]{0.482pt}{0.400pt}}
\put(265.67,273){\rule{0.400pt}{0.482pt}}
\multiput(265.17,274.00)(1.000,-1.000){2}{\rule{0.400pt}{0.241pt}}
\put(266.0,274.0){\usebox{\plotpoint}}
\put(267.0,273.0){\usebox{\plotpoint}}
\put(268,272.67){\rule{0.241pt}{0.400pt}}
\multiput(268.00,273.17)(0.500,-1.000){2}{\rule{0.120pt}{0.400pt}}
\put(268.0,273.0){\usebox{\plotpoint}}
\put(269.0,273.0){\rule[-0.200pt]{0.723pt}{0.400pt}}
\put(272.0,272.0){\usebox{\plotpoint}}
\put(273,271.67){\rule{0.241pt}{0.400pt}}
\multiput(273.00,271.17)(0.500,1.000){2}{\rule{0.120pt}{0.400pt}}
\put(272.0,272.0){\usebox{\plotpoint}}
\put(274,273){\usebox{\plotpoint}}
\put(274,271.67){\rule{0.241pt}{0.400pt}}
\multiput(274.00,272.17)(0.500,-1.000){2}{\rule{0.120pt}{0.400pt}}
\put(275.0,272.0){\usebox{\plotpoint}}
\put(275.67,271){\rule{0.400pt}{0.482pt}}
\multiput(275.17,271.00)(1.000,1.000){2}{\rule{0.400pt}{0.241pt}}
\put(276.0,271.0){\usebox{\plotpoint}}
\put(278,271.67){\rule{0.241pt}{0.400pt}}
\multiput(278.00,272.17)(0.500,-1.000){2}{\rule{0.120pt}{0.400pt}}
\put(277.0,273.0){\usebox{\plotpoint}}
\put(280,271.67){\rule{0.241pt}{0.400pt}}
\multiput(280.00,271.17)(0.500,1.000){2}{\rule{0.120pt}{0.400pt}}
\put(281,271.67){\rule{0.241pt}{0.400pt}}
\multiput(281.00,272.17)(0.500,-1.000){2}{\rule{0.120pt}{0.400pt}}
\put(279.0,272.0){\usebox{\plotpoint}}
\put(282,271.67){\rule{0.241pt}{0.400pt}}
\multiput(282.00,272.17)(0.500,-1.000){2}{\rule{0.120pt}{0.400pt}}
\put(283,271.67){\rule{0.241pt}{0.400pt}}
\multiput(283.00,271.17)(0.500,1.000){2}{\rule{0.120pt}{0.400pt}}
\put(282.0,272.0){\usebox{\plotpoint}}
\put(284,273){\usebox{\plotpoint}}
\put(284.67,271){\rule{0.400pt}{0.482pt}}
\multiput(284.17,272.00)(1.000,-1.000){2}{\rule{0.400pt}{0.241pt}}
\put(284.0,273.0){\usebox{\plotpoint}}
\put(286.0,271.0){\rule[-0.200pt]{0.400pt}{0.723pt}}
\put(288,272.67){\rule{0.241pt}{0.400pt}}
\multiput(288.00,273.17)(0.500,-1.000){2}{\rule{0.120pt}{0.400pt}}
\put(288.67,273){\rule{0.400pt}{0.482pt}}
\multiput(288.17,273.00)(1.000,1.000){2}{\rule{0.400pt}{0.241pt}}
\put(286.0,274.0){\rule[-0.200pt]{0.482pt}{0.400pt}}
\put(290,275){\usebox{\plotpoint}}
\put(291,274.67){\rule{0.241pt}{0.400pt}}
\multiput(291.00,274.17)(0.500,1.000){2}{\rule{0.120pt}{0.400pt}}
\put(290.0,275.0){\usebox{\plotpoint}}
\put(292,274.67){\rule{0.241pt}{0.400pt}}
\multiput(292.00,274.17)(0.500,1.000){2}{\rule{0.120pt}{0.400pt}}
\put(292.0,275.0){\usebox{\plotpoint}}
\put(294,275.67){\rule{0.241pt}{0.400pt}}
\multiput(294.00,275.17)(0.500,1.000){2}{\rule{0.120pt}{0.400pt}}
\put(295,276.67){\rule{0.241pt}{0.400pt}}
\multiput(295.00,276.17)(0.500,1.000){2}{\rule{0.120pt}{0.400pt}}
\put(293.0,276.0){\usebox{\plotpoint}}
\put(296,278){\usebox{\plotpoint}}
\put(296,277.67){\rule{0.241pt}{0.400pt}}
\multiput(296.00,277.17)(0.500,1.000){2}{\rule{0.120pt}{0.400pt}}
\put(297,277.67){\rule{0.241pt}{0.400pt}}
\multiput(297.00,278.17)(0.500,-1.000){2}{\rule{0.120pt}{0.400pt}}
\put(298,278){\usebox{\plotpoint}}
\put(299,277.67){\rule{0.241pt}{0.400pt}}
\multiput(299.00,277.17)(0.500,1.000){2}{\rule{0.120pt}{0.400pt}}
\put(298.0,278.0){\usebox{\plotpoint}}
\put(300,278.67){\rule{0.241pt}{0.400pt}}
\multiput(300.00,279.17)(0.500,-1.000){2}{\rule{0.120pt}{0.400pt}}
\put(300.67,279){\rule{0.400pt}{0.723pt}}
\multiput(300.17,279.00)(1.000,1.500){2}{\rule{0.400pt}{0.361pt}}
\put(300.0,279.0){\usebox{\plotpoint}}
\put(302,280.67){\rule{0.241pt}{0.400pt}}
\multiput(302.00,280.17)(0.500,1.000){2}{\rule{0.120pt}{0.400pt}}
\put(302.0,281.0){\usebox{\plotpoint}}
\put(303.0,282.0){\usebox{\plotpoint}}
\put(304,282.67){\rule{0.241pt}{0.400pt}}
\multiput(304.00,282.17)(0.500,1.000){2}{\rule{0.120pt}{0.400pt}}
\put(304.0,282.0){\usebox{\plotpoint}}
\put(305.0,284.0){\usebox{\plotpoint}}
\put(305.67,283){\rule{0.400pt}{0.482pt}}
\multiput(305.17,283.00)(1.000,1.000){2}{\rule{0.400pt}{0.241pt}}
\put(307,283.67){\rule{0.241pt}{0.400pt}}
\multiput(307.00,284.17)(0.500,-1.000){2}{\rule{0.120pt}{0.400pt}}
\put(306.0,283.0){\usebox{\plotpoint}}
\put(308.0,284.0){\rule[-0.200pt]{0.400pt}{0.723pt}}
\put(309,286.67){\rule{0.241pt}{0.400pt}}
\multiput(309.00,286.17)(0.500,1.000){2}{\rule{0.120pt}{0.400pt}}
\put(308.0,287.0){\usebox{\plotpoint}}
\put(310,288){\usebox{\plotpoint}}
\put(310,287.67){\rule{0.241pt}{0.400pt}}
\multiput(310.00,287.17)(0.500,1.000){2}{\rule{0.120pt}{0.400pt}}
\put(311.67,289){\rule{0.400pt}{0.482pt}}
\multiput(311.17,289.00)(1.000,1.000){2}{\rule{0.400pt}{0.241pt}}
\put(312.67,291){\rule{0.400pt}{0.482pt}}
\multiput(312.17,291.00)(1.000,1.000){2}{\rule{0.400pt}{0.241pt}}
\put(311.0,289.0){\usebox{\plotpoint}}
\put(314,293){\usebox{\plotpoint}}
\put(314.67,293){\rule{0.400pt}{0.723pt}}
\multiput(314.17,293.00)(1.000,1.500){2}{\rule{0.400pt}{0.361pt}}
\put(314.0,293.0){\usebox{\plotpoint}}
\put(316,294.67){\rule{0.241pt}{0.400pt}}
\multiput(316.00,294.17)(0.500,1.000){2}{\rule{0.120pt}{0.400pt}}
\put(317,295.67){\rule{0.241pt}{0.400pt}}
\multiput(317.00,295.17)(0.500,1.000){2}{\rule{0.120pt}{0.400pt}}
\put(316.0,295.0){\usebox{\plotpoint}}
\put(318,297.67){\rule{0.241pt}{0.400pt}}
\multiput(318.00,297.17)(0.500,1.000){2}{\rule{0.120pt}{0.400pt}}
\put(318.0,297.0){\usebox{\plotpoint}}
\put(319.0,299.0){\usebox{\plotpoint}}
\put(320,300.67){\rule{0.241pt}{0.400pt}}
\multiput(320.00,300.17)(0.500,1.000){2}{\rule{0.120pt}{0.400pt}}
\put(320.0,299.0){\rule[-0.200pt]{0.400pt}{0.482pt}}
\put(321.67,302){\rule{0.400pt}{0.723pt}}
\multiput(321.17,302.00)(1.000,1.500){2}{\rule{0.400pt}{0.361pt}}
\put(321.0,302.0){\usebox{\plotpoint}}
\put(323.0,305.0){\usebox{\plotpoint}}
\put(324,305.67){\rule{0.241pt}{0.400pt}}
\multiput(324.00,305.17)(0.500,1.000){2}{\rule{0.120pt}{0.400pt}}
\put(325,306.67){\rule{0.241pt}{0.400pt}}
\multiput(325.00,306.17)(0.500,1.000){2}{\rule{0.120pt}{0.400pt}}
\put(324.0,305.0){\usebox{\plotpoint}}
\put(326,308.67){\rule{0.241pt}{0.400pt}}
\multiput(326.00,308.17)(0.500,1.000){2}{\rule{0.120pt}{0.400pt}}
\put(326.0,308.0){\usebox{\plotpoint}}
\put(327.0,310.0){\usebox{\plotpoint}}
\put(327.67,312){\rule{0.400pt}{0.482pt}}
\multiput(327.17,312.00)(1.000,1.000){2}{\rule{0.400pt}{0.241pt}}
\put(328.0,310.0){\rule[-0.200pt]{0.400pt}{0.482pt}}
\put(329.0,314.0){\usebox{\plotpoint}}
\put(330,314.67){\rule{0.241pt}{0.400pt}}
\multiput(330.00,314.17)(0.500,1.000){2}{\rule{0.120pt}{0.400pt}}
\put(331,315.67){\rule{0.241pt}{0.400pt}}
\multiput(331.00,315.17)(0.500,1.000){2}{\rule{0.120pt}{0.400pt}}
\put(330.0,314.0){\usebox{\plotpoint}}
\put(332,317.67){\rule{0.241pt}{0.400pt}}
\multiput(332.00,318.17)(0.500,-1.000){2}{\rule{0.120pt}{0.400pt}}
\put(332.67,318){\rule{0.400pt}{0.723pt}}
\multiput(332.17,318.00)(1.000,1.500){2}{\rule{0.400pt}{0.361pt}}
\put(332.0,317.0){\rule[-0.200pt]{0.400pt}{0.482pt}}
\put(334,321){\usebox{\plotpoint}}
\put(334,320.67){\rule{0.241pt}{0.400pt}}
\multiput(334.00,320.17)(0.500,1.000){2}{\rule{0.120pt}{0.400pt}}
\put(334.67,322){\rule{0.400pt}{0.482pt}}
\multiput(334.17,322.00)(1.000,1.000){2}{\rule{0.400pt}{0.241pt}}
\put(336,324){\usebox{\plotpoint}}
\put(335.67,324){\rule{0.400pt}{0.482pt}}
\multiput(335.17,324.00)(1.000,1.000){2}{\rule{0.400pt}{0.241pt}}
\put(337.0,326.0){\usebox{\plotpoint}}
\put(338.0,326.0){\rule[-0.200pt]{0.400pt}{0.964pt}}
\put(339,329.67){\rule{0.241pt}{0.400pt}}
\multiput(339.00,329.17)(0.500,1.000){2}{\rule{0.120pt}{0.400pt}}
\put(338.0,330.0){\usebox{\plotpoint}}
\put(339.67,332){\rule{0.400pt}{0.482pt}}
\multiput(339.17,332.00)(1.000,1.000){2}{\rule{0.400pt}{0.241pt}}
\put(341,333.67){\rule{0.241pt}{0.400pt}}
\multiput(341.00,333.17)(0.500,1.000){2}{\rule{0.120pt}{0.400pt}}
\put(340.0,331.0){\usebox{\plotpoint}}
\put(342,335){\usebox{\plotpoint}}
\put(341.67,335){\rule{0.400pt}{0.482pt}}
\multiput(341.17,335.00)(1.000,1.000){2}{\rule{0.400pt}{0.241pt}}
\put(342.67,337){\rule{0.400pt}{0.482pt}}
\multiput(342.17,337.00)(1.000,1.000){2}{\rule{0.400pt}{0.241pt}}
\put(344.0,339.0){\usebox{\plotpoint}}
\put(344.67,340){\rule{0.400pt}{0.723pt}}
\multiput(344.17,340.00)(1.000,1.500){2}{\rule{0.400pt}{0.361pt}}
\put(344.0,340.0){\usebox{\plotpoint}}
\put(346,343.67){\rule{0.241pt}{0.400pt}}
\multiput(346.00,343.17)(0.500,1.000){2}{\rule{0.120pt}{0.400pt}}
\put(347,344.67){\rule{0.241pt}{0.400pt}}
\multiput(347.00,344.17)(0.500,1.000){2}{\rule{0.120pt}{0.400pt}}
\put(346.0,343.0){\usebox{\plotpoint}}
\put(347.67,347){\rule{0.400pt}{0.482pt}}
\multiput(347.17,347.00)(1.000,1.000){2}{\rule{0.400pt}{0.241pt}}
\put(348.67,349){\rule{0.400pt}{0.482pt}}
\multiput(348.17,349.00)(1.000,1.000){2}{\rule{0.400pt}{0.241pt}}
\put(348.0,346.0){\usebox{\plotpoint}}
\put(350.0,351.0){\usebox{\plotpoint}}
\put(350.67,352){\rule{0.400pt}{0.723pt}}
\multiput(350.17,352.00)(1.000,1.500){2}{\rule{0.400pt}{0.361pt}}
\put(350.0,352.0){\usebox{\plotpoint}}
\put(352,355){\usebox{\plotpoint}}
\put(352.67,355){\rule{0.400pt}{0.964pt}}
\multiput(352.17,355.00)(1.000,2.000){2}{\rule{0.400pt}{0.482pt}}
\put(352.0,355.0){\usebox{\plotpoint}}
\put(354.0,359.0){\usebox{\plotpoint}}
\put(355,359.67){\rule{0.241pt}{0.400pt}}
\multiput(355.00,359.17)(0.500,1.000){2}{\rule{0.120pt}{0.400pt}}
\put(354.0,360.0){\usebox{\plotpoint}}
\put(355.67,364){\rule{0.400pt}{0.482pt}}
\multiput(355.17,364.00)(1.000,1.000){2}{\rule{0.400pt}{0.241pt}}
\put(356.0,361.0){\rule[-0.200pt]{0.400pt}{0.723pt}}
\put(357.0,366.0){\usebox{\plotpoint}}
\put(358,367.67){\rule{0.241pt}{0.400pt}}
\multiput(358.00,368.17)(0.500,-1.000){2}{\rule{0.120pt}{0.400pt}}
\put(358.0,366.0){\rule[-0.200pt]{0.400pt}{0.723pt}}
\put(358.67,369){\rule{0.400pt}{1.204pt}}
\multiput(358.17,369.00)(1.000,2.500){2}{\rule{0.400pt}{0.602pt}}
\put(360,372.67){\rule{0.241pt}{0.400pt}}
\multiput(360.00,373.17)(0.500,-1.000){2}{\rule{0.120pt}{0.400pt}}
\put(359.0,368.0){\usebox{\plotpoint}}
\put(360.67,374){\rule{0.400pt}{0.723pt}}
\multiput(360.17,374.00)(1.000,1.500){2}{\rule{0.400pt}{0.361pt}}
\put(362,376.67){\rule{0.241pt}{0.400pt}}
\multiput(362.00,376.17)(0.500,1.000){2}{\rule{0.120pt}{0.400pt}}
\put(361.0,373.0){\usebox{\plotpoint}}
\put(363,378){\usebox{\plotpoint}}
\put(363,377.67){\rule{0.241pt}{0.400pt}}
\multiput(363.00,377.17)(0.500,1.000){2}{\rule{0.120pt}{0.400pt}}
\put(363.67,379){\rule{0.400pt}{0.482pt}}
\multiput(363.17,379.00)(1.000,1.000){2}{\rule{0.400pt}{0.241pt}}
\put(364.67,383){\rule{0.400pt}{0.482pt}}
\multiput(364.17,383.00)(1.000,1.000){2}{\rule{0.400pt}{0.241pt}}
\put(366,383.67){\rule{0.241pt}{0.400pt}}
\multiput(366.00,384.17)(0.500,-1.000){2}{\rule{0.120pt}{0.400pt}}
\put(365.0,381.0){\rule[-0.200pt]{0.400pt}{0.482pt}}
\put(366.67,386){\rule{0.400pt}{0.482pt}}
\multiput(366.17,386.00)(1.000,1.000){2}{\rule{0.400pt}{0.241pt}}
\put(367.67,388){\rule{0.400pt}{0.482pt}}
\multiput(367.17,388.00)(1.000,1.000){2}{\rule{0.400pt}{0.241pt}}
\put(367.0,384.0){\rule[-0.200pt]{0.400pt}{0.482pt}}
\put(369.0,390.0){\rule[-0.200pt]{0.400pt}{0.482pt}}
\put(369.67,392){\rule{0.400pt}{0.482pt}}
\multiput(369.17,392.00)(1.000,1.000){2}{\rule{0.400pt}{0.241pt}}
\put(369.0,392.0){\usebox{\plotpoint}}
\put(371.0,394.0){\rule[-0.200pt]{0.400pt}{0.482pt}}
\put(371.67,396){\rule{0.400pt}{0.482pt}}
\multiput(371.17,396.00)(1.000,1.000){2}{\rule{0.400pt}{0.241pt}}
\put(371.0,396.0){\usebox{\plotpoint}}
\put(372.67,401){\rule{0.400pt}{0.482pt}}
\multiput(372.17,401.00)(1.000,1.000){2}{\rule{0.400pt}{0.241pt}}
\put(373.0,398.0){\rule[-0.200pt]{0.400pt}{0.723pt}}
\put(374.67,403){\rule{0.400pt}{0.964pt}}
\multiput(374.17,403.00)(1.000,2.000){2}{\rule{0.400pt}{0.482pt}}
\put(374.0,403.0){\usebox{\plotpoint}}
\put(376.0,407.0){\usebox{\plotpoint}}
\put(376.67,408){\rule{0.400pt}{0.723pt}}
\multiput(376.17,408.00)(1.000,1.500){2}{\rule{0.400pt}{0.361pt}}
\put(378,409.67){\rule{0.241pt}{0.400pt}}
\multiput(378.00,410.17)(0.500,-1.000){2}{\rule{0.120pt}{0.400pt}}
\put(377.0,407.0){\usebox{\plotpoint}}
\put(379.0,410.0){\rule[-0.200pt]{0.400pt}{0.964pt}}
\put(379.67,414){\rule{0.400pt}{0.482pt}}
\multiput(379.17,414.00)(1.000,1.000){2}{\rule{0.400pt}{0.241pt}}
\put(379.0,414.0){\usebox{\plotpoint}}
\put(381.0,416.0){\rule[-0.200pt]{0.400pt}{0.723pt}}
\put(381.67,419){\rule{0.400pt}{0.482pt}}
\multiput(381.17,419.00)(1.000,1.000){2}{\rule{0.400pt}{0.241pt}}
\put(381.0,419.0){\usebox{\plotpoint}}
\put(382.67,422){\rule{0.400pt}{0.482pt}}
\multiput(382.17,422.00)(1.000,1.000){2}{\rule{0.400pt}{0.241pt}}
\put(384,423.67){\rule{0.241pt}{0.400pt}}
\multiput(384.00,423.17)(0.500,1.000){2}{\rule{0.120pt}{0.400pt}}
\put(383.0,421.0){\usebox{\plotpoint}}
\put(385,425.67){\rule{0.241pt}{0.400pt}}
\multiput(385.00,426.17)(0.500,-1.000){2}{\rule{0.120pt}{0.400pt}}
\put(385.67,426){\rule{0.400pt}{0.723pt}}
\multiput(385.17,426.00)(1.000,1.500){2}{\rule{0.400pt}{0.361pt}}
\put(385.0,425.0){\rule[-0.200pt]{0.400pt}{0.482pt}}
\put(387,429){\usebox{\plotpoint}}
\put(387.67,429){\rule{0.400pt}{1.204pt}}
\multiput(387.17,429.00)(1.000,2.500){2}{\rule{0.400pt}{0.602pt}}
\put(387.0,429.0){\usebox{\plotpoint}}
\put(389.0,434.0){\rule[-0.200pt]{0.400pt}{0.482pt}}
\put(389.67,436){\rule{0.400pt}{0.482pt}}
\multiput(389.17,436.00)(1.000,1.000){2}{\rule{0.400pt}{0.241pt}}
\put(389.0,436.0){\usebox{\plotpoint}}
\put(391.0,438.0){\rule[-0.200pt]{0.400pt}{0.723pt}}
\put(391.67,441){\rule{0.400pt}{0.723pt}}
\multiput(391.17,441.00)(1.000,1.500){2}{\rule{0.400pt}{0.361pt}}
\put(391.0,441.0){\usebox{\plotpoint}}
\put(393.0,444.0){\rule[-0.200pt]{0.400pt}{0.482pt}}
\put(394,445.67){\rule{0.241pt}{0.400pt}}
\multiput(394.00,445.17)(0.500,1.000){2}{\rule{0.120pt}{0.400pt}}
\put(393.0,446.0){\usebox{\plotpoint}}
\put(395,448.67){\rule{0.241pt}{0.400pt}}
\multiput(395.00,449.17)(0.500,-1.000){2}{\rule{0.120pt}{0.400pt}}
\put(395.67,449){\rule{0.400pt}{0.482pt}}
\multiput(395.17,449.00)(1.000,1.000){2}{\rule{0.400pt}{0.241pt}}
\put(395.0,447.0){\rule[-0.200pt]{0.400pt}{0.723pt}}
\put(396.67,453){\rule{0.400pt}{0.482pt}}
\multiput(396.17,453.00)(1.000,1.000){2}{\rule{0.400pt}{0.241pt}}
\put(397.0,451.0){\rule[-0.200pt]{0.400pt}{0.482pt}}
\put(398.0,455.0){\usebox{\plotpoint}}
\put(399,458.67){\rule{0.241pt}{0.400pt}}
\multiput(399.00,459.17)(0.500,-1.000){2}{\rule{0.120pt}{0.400pt}}
\put(399.67,459){\rule{0.400pt}{0.482pt}}
\multiput(399.17,459.00)(1.000,1.000){2}{\rule{0.400pt}{0.241pt}}
\put(399.0,455.0){\rule[-0.200pt]{0.400pt}{1.204pt}}
\put(401,461.67){\rule{0.241pt}{0.400pt}}
\multiput(401.00,461.17)(0.500,1.000){2}{\rule{0.120pt}{0.400pt}}
\put(401.67,463){\rule{0.400pt}{0.723pt}}
\multiput(401.17,463.00)(1.000,1.500){2}{\rule{0.400pt}{0.361pt}}
\put(401.0,461.0){\usebox{\plotpoint}}
\put(402.67,465){\rule{0.400pt}{0.964pt}}
\multiput(402.17,465.00)(1.000,2.000){2}{\rule{0.400pt}{0.482pt}}
\put(403.0,465.0){\usebox{\plotpoint}}
\put(404,466.67){\rule{0.241pt}{0.400pt}}
\multiput(404.00,467.17)(0.500,-1.000){2}{\rule{0.120pt}{0.400pt}}
\put(404.67,467){\rule{0.400pt}{1.445pt}}
\multiput(404.17,467.00)(1.000,3.000){2}{\rule{0.400pt}{0.723pt}}
\put(404.0,468.0){\usebox{\plotpoint}}
\put(406,473.67){\rule{0.241pt}{0.400pt}}
\multiput(406.00,473.17)(0.500,1.000){2}{\rule{0.120pt}{0.400pt}}
\put(406.0,473.0){\usebox{\plotpoint}}
\put(407.0,475.0){\usebox{\plotpoint}}
\put(408.0,475.0){\rule[-0.200pt]{0.400pt}{1.204pt}}
\put(408.67,480){\rule{0.400pt}{0.482pt}}
\multiput(408.17,480.00)(1.000,1.000){2}{\rule{0.400pt}{0.241pt}}
\put(408.0,480.0){\usebox{\plotpoint}}
\put(409.67,485){\rule{0.400pt}{0.482pt}}
\multiput(409.17,485.00)(1.000,1.000){2}{\rule{0.400pt}{0.241pt}}
\put(411,485.67){\rule{0.241pt}{0.400pt}}
\multiput(411.00,486.17)(0.500,-1.000){2}{\rule{0.120pt}{0.400pt}}
\put(410.0,482.0){\rule[-0.200pt]{0.400pt}{0.723pt}}
\put(411.67,487){\rule{0.400pt}{0.482pt}}
\multiput(411.17,487.00)(1.000,1.000){2}{\rule{0.400pt}{0.241pt}}
\put(412.67,489){\rule{0.400pt}{0.723pt}}
\multiput(412.17,489.00)(1.000,1.500){2}{\rule{0.400pt}{0.361pt}}
\put(412.0,486.0){\usebox{\plotpoint}}
\put(414,492){\usebox{\plotpoint}}
\put(413.67,492){\rule{0.400pt}{0.723pt}}
\multiput(413.17,492.00)(1.000,1.500){2}{\rule{0.400pt}{0.361pt}}
\put(415,494.67){\rule{0.241pt}{0.400pt}}
\multiput(415.00,494.17)(0.500,1.000){2}{\rule{0.120pt}{0.400pt}}
\put(416,498.67){\rule{0.241pt}{0.400pt}}
\multiput(416.00,498.17)(0.500,1.000){2}{\rule{0.120pt}{0.400pt}}
\put(416.67,497){\rule{0.400pt}{0.723pt}}
\multiput(416.17,498.50)(1.000,-1.500){2}{\rule{0.400pt}{0.361pt}}
\put(416.0,496.0){\rule[-0.200pt]{0.400pt}{0.723pt}}
\put(417.67,501){\rule{0.400pt}{0.482pt}}
\multiput(417.17,501.00)(1.000,1.000){2}{\rule{0.400pt}{0.241pt}}
\put(418.67,503){\rule{0.400pt}{0.482pt}}
\multiput(418.17,503.00)(1.000,1.000){2}{\rule{0.400pt}{0.241pt}}
\put(418.0,497.0){\rule[-0.200pt]{0.400pt}{0.964pt}}
\put(420.0,505.0){\rule[-0.200pt]{0.400pt}{0.482pt}}
\put(420.67,507){\rule{0.400pt}{0.723pt}}
\multiput(420.17,507.00)(1.000,1.500){2}{\rule{0.400pt}{0.361pt}}
\put(420.0,507.0){\usebox{\plotpoint}}
\put(421.67,511){\rule{0.400pt}{0.482pt}}
\multiput(421.17,511.00)(1.000,1.000){2}{\rule{0.400pt}{0.241pt}}
\put(422.67,513){\rule{0.400pt}{0.482pt}}
\multiput(422.17,513.00)(1.000,1.000){2}{\rule{0.400pt}{0.241pt}}
\put(422.0,510.0){\usebox{\plotpoint}}
\put(423.67,514){\rule{0.400pt}{0.482pt}}
\multiput(423.17,515.00)(1.000,-1.000){2}{\rule{0.400pt}{0.241pt}}
\put(424.67,514){\rule{0.400pt}{1.927pt}}
\multiput(424.17,514.00)(1.000,4.000){2}{\rule{0.400pt}{0.964pt}}
\put(424.0,515.0){\usebox{\plotpoint}}
\put(425.67,517){\rule{0.400pt}{1.204pt}}
\multiput(425.17,517.00)(1.000,2.500){2}{\rule{0.400pt}{0.602pt}}
\put(427,521.67){\rule{0.241pt}{0.400pt}}
\multiput(427.00,521.17)(0.500,1.000){2}{\rule{0.120pt}{0.400pt}}
\put(426.0,517.0){\rule[-0.200pt]{0.400pt}{1.204pt}}
\put(428,523.67){\rule{0.241pt}{0.400pt}}
\multiput(428.00,524.17)(0.500,-1.000){2}{\rule{0.120pt}{0.400pt}}
\put(428.67,524){\rule{0.400pt}{0.964pt}}
\multiput(428.17,524.00)(1.000,2.000){2}{\rule{0.400pt}{0.482pt}}
\put(428.0,523.0){\rule[-0.200pt]{0.400pt}{0.482pt}}
\put(430,529.67){\rule{0.241pt}{0.400pt}}
\multiput(430.00,529.17)(0.500,1.000){2}{\rule{0.120pt}{0.400pt}}
\put(430.67,528){\rule{0.400pt}{0.723pt}}
\multiput(430.17,529.50)(1.000,-1.500){2}{\rule{0.400pt}{0.361pt}}
\put(430.0,528.0){\rule[-0.200pt]{0.400pt}{0.482pt}}
\put(432.0,528.0){\rule[-0.200pt]{0.400pt}{1.445pt}}
\put(432.0,534.0){\usebox{\plotpoint}}
\put(432.67,537){\rule{0.400pt}{0.723pt}}
\multiput(432.17,537.00)(1.000,1.500){2}{\rule{0.400pt}{0.361pt}}
\put(433.0,534.0){\rule[-0.200pt]{0.400pt}{0.723pt}}
\put(434.0,540.0){\usebox{\plotpoint}}
\put(434.67,543){\rule{0.400pt}{0.482pt}}
\multiput(434.17,544.00)(1.000,-1.000){2}{\rule{0.400pt}{0.241pt}}
\put(436,542.67){\rule{0.241pt}{0.400pt}}
\multiput(436.00,542.17)(0.500,1.000){2}{\rule{0.120pt}{0.400pt}}
\put(435.0,540.0){\rule[-0.200pt]{0.400pt}{1.204pt}}
\put(436.67,542){\rule{0.400pt}{1.204pt}}
\multiput(436.17,542.00)(1.000,2.500){2}{\rule{0.400pt}{0.602pt}}
\put(437.67,547){\rule{0.400pt}{0.482pt}}
\multiput(437.17,547.00)(1.000,1.000){2}{\rule{0.400pt}{0.241pt}}
\put(437.0,542.0){\rule[-0.200pt]{0.400pt}{0.482pt}}
\put(439,549){\usebox{\plotpoint}}
\put(438.67,549){\rule{0.400pt}{0.482pt}}
\multiput(438.17,549.00)(1.000,1.000){2}{\rule{0.400pt}{0.241pt}}
\put(440,549.67){\rule{0.241pt}{0.400pt}}
\multiput(440.00,550.17)(0.500,-1.000){2}{\rule{0.120pt}{0.400pt}}
\put(440.67,553){\rule{0.400pt}{0.723pt}}
\multiput(440.17,553.00)(1.000,1.500){2}{\rule{0.400pt}{0.361pt}}
\put(441.67,556){\rule{0.400pt}{0.723pt}}
\multiput(441.17,556.00)(1.000,1.500){2}{\rule{0.400pt}{0.361pt}}
\put(441.0,550.0){\rule[-0.200pt]{0.400pt}{0.723pt}}
\put(442.67,557){\rule{0.400pt}{0.482pt}}
\multiput(442.17,557.00)(1.000,1.000){2}{\rule{0.400pt}{0.241pt}}
\put(443.67,556){\rule{0.400pt}{0.723pt}}
\multiput(443.17,557.50)(1.000,-1.500){2}{\rule{0.400pt}{0.361pt}}
\put(443.0,557.0){\rule[-0.200pt]{0.400pt}{0.482pt}}
\put(445.0,556.0){\rule[-0.200pt]{0.400pt}{1.927pt}}
\put(445.67,564){\rule{0.400pt}{0.482pt}}
\multiput(445.17,564.00)(1.000,1.000){2}{\rule{0.400pt}{0.241pt}}
\put(445.0,564.0){\usebox{\plotpoint}}
\put(446.67,564){\rule{0.400pt}{0.723pt}}
\multiput(446.17,565.50)(1.000,-1.500){2}{\rule{0.400pt}{0.361pt}}
\put(447.67,564){\rule{0.400pt}{0.723pt}}
\multiput(447.17,564.00)(1.000,1.500){2}{\rule{0.400pt}{0.361pt}}
\put(447.0,566.0){\usebox{\plotpoint}}
\put(449,569.67){\rule{0.241pt}{0.400pt}}
\multiput(449.00,569.17)(0.500,1.000){2}{\rule{0.120pt}{0.400pt}}
\put(449.67,571){\rule{0.400pt}{1.204pt}}
\multiput(449.17,571.00)(1.000,2.500){2}{\rule{0.400pt}{0.602pt}}
\put(449.0,567.0){\rule[-0.200pt]{0.400pt}{0.723pt}}
\put(451,572.67){\rule{0.241pt}{0.400pt}}
\multiput(451.00,573.17)(0.500,-1.000){2}{\rule{0.120pt}{0.400pt}}
\put(451.67,573){\rule{0.400pt}{0.723pt}}
\multiput(451.17,573.00)(1.000,1.500){2}{\rule{0.400pt}{0.361pt}}
\put(451.0,574.0){\rule[-0.200pt]{0.400pt}{0.482pt}}
\put(452.67,577){\rule{0.400pt}{0.482pt}}
\multiput(452.17,578.00)(1.000,-1.000){2}{\rule{0.400pt}{0.241pt}}
\put(453.67,577){\rule{0.400pt}{0.482pt}}
\multiput(453.17,577.00)(1.000,1.000){2}{\rule{0.400pt}{0.241pt}}
\put(453.0,576.0){\rule[-0.200pt]{0.400pt}{0.723pt}}
\put(454.67,580){\rule{0.400pt}{0.482pt}}
\multiput(454.17,580.00)(1.000,1.000){2}{\rule{0.400pt}{0.241pt}}
\put(456,580.67){\rule{0.241pt}{0.400pt}}
\multiput(456.00,581.17)(0.500,-1.000){2}{\rule{0.120pt}{0.400pt}}
\put(455.0,579.0){\usebox{\plotpoint}}
\put(457,583.67){\rule{0.241pt}{0.400pt}}
\multiput(457.00,584.17)(0.500,-1.000){2}{\rule{0.120pt}{0.400pt}}
\put(457.0,581.0){\rule[-0.200pt]{0.400pt}{0.964pt}}
\put(457.67,582){\rule{0.400pt}{0.723pt}}
\multiput(457.17,583.50)(1.000,-1.500){2}{\rule{0.400pt}{0.361pt}}
\put(458.67,582){\rule{0.400pt}{1.445pt}}
\multiput(458.17,582.00)(1.000,3.000){2}{\rule{0.400pt}{0.723pt}}
\put(458.0,584.0){\usebox{\plotpoint}}
\put(460,588){\usebox{\plotpoint}}
\put(460.67,588){\rule{0.400pt}{1.204pt}}
\multiput(460.17,588.00)(1.000,2.500){2}{\rule{0.400pt}{0.602pt}}
\put(460.0,588.0){\usebox{\plotpoint}}
\put(462,593){\usebox{\plotpoint}}
\put(463,592.67){\rule{0.241pt}{0.400pt}}
\multiput(463.00,592.17)(0.500,1.000){2}{\rule{0.120pt}{0.400pt}}
\put(462.0,593.0){\usebox{\plotpoint}}
\put(464.0,593.0){\usebox{\plotpoint}}
\put(464.67,593){\rule{0.400pt}{0.964pt}}
\multiput(464.17,593.00)(1.000,2.000){2}{\rule{0.400pt}{0.482pt}}
\put(464.0,593.0){\usebox{\plotpoint}}
\put(466,592.67){\rule{0.241pt}{0.400pt}}
\multiput(466.00,593.17)(0.500,-1.000){2}{\rule{0.120pt}{0.400pt}}
\put(466.67,593){\rule{0.400pt}{1.686pt}}
\multiput(466.17,593.00)(1.000,3.500){2}{\rule{0.400pt}{0.843pt}}
\put(466.0,594.0){\rule[-0.200pt]{0.400pt}{0.723pt}}
\put(468,595.67){\rule{0.241pt}{0.400pt}}
\multiput(468.00,596.17)(0.500,-1.000){2}{\rule{0.120pt}{0.400pt}}
\put(468.67,596){\rule{0.400pt}{0.723pt}}
\multiput(468.17,596.00)(1.000,1.500){2}{\rule{0.400pt}{0.361pt}}
\put(468.0,597.0){\rule[-0.200pt]{0.400pt}{0.723pt}}
\put(469.67,597){\rule{0.400pt}{0.723pt}}
\multiput(469.17,598.50)(1.000,-1.500){2}{\rule{0.400pt}{0.361pt}}
\put(470.67,597){\rule{0.400pt}{0.723pt}}
\multiput(470.17,597.00)(1.000,1.500){2}{\rule{0.400pt}{0.361pt}}
\put(470.0,599.0){\usebox{\plotpoint}}
\put(472,602.67){\rule{0.241pt}{0.400pt}}
\multiput(472.00,602.17)(0.500,1.000){2}{\rule{0.120pt}{0.400pt}}
\put(473,602.67){\rule{0.241pt}{0.400pt}}
\multiput(473.00,603.17)(0.500,-1.000){2}{\rule{0.120pt}{0.400pt}}
\put(472.0,600.0){\rule[-0.200pt]{0.400pt}{0.723pt}}
\put(473.67,600){\rule{0.400pt}{0.482pt}}
\multiput(473.17,601.00)(1.000,-1.000){2}{\rule{0.400pt}{0.241pt}}
\put(474.67,600){\rule{0.400pt}{0.964pt}}
\multiput(474.17,600.00)(1.000,2.000){2}{\rule{0.400pt}{0.482pt}}
\put(474.0,602.0){\usebox{\plotpoint}}
\put(476,604){\usebox{\plotpoint}}
\put(476,603.67){\rule{0.241pt}{0.400pt}}
\multiput(476.00,603.17)(0.500,1.000){2}{\rule{0.120pt}{0.400pt}}
\put(477,603.67){\rule{0.241pt}{0.400pt}}
\multiput(477.00,604.17)(0.500,-1.000){2}{\rule{0.120pt}{0.400pt}}
\put(478.0,604.0){\usebox{\plotpoint}}
\put(478.0,605.0){\usebox{\plotpoint}}
\put(479,606.67){\rule{0.241pt}{0.400pt}}
\multiput(479.00,606.17)(0.500,1.000){2}{\rule{0.120pt}{0.400pt}}
\put(479.67,605){\rule{0.400pt}{0.723pt}}
\multiput(479.17,606.50)(1.000,-1.500){2}{\rule{0.400pt}{0.361pt}}
\put(479.0,605.0){\rule[-0.200pt]{0.400pt}{0.482pt}}
\put(481.0,605.0){\rule[-0.200pt]{0.400pt}{0.482pt}}
\put(482,605.67){\rule{0.241pt}{0.400pt}}
\multiput(482.00,606.17)(0.500,-1.000){2}{\rule{0.120pt}{0.400pt}}
\put(481.0,607.0){\usebox{\plotpoint}}
\put(482.67,606){\rule{0.400pt}{0.723pt}}
\multiput(482.17,607.50)(1.000,-1.500){2}{\rule{0.400pt}{0.361pt}}
\put(484,605.67){\rule{0.241pt}{0.400pt}}
\multiput(484.00,605.17)(0.500,1.000){2}{\rule{0.120pt}{0.400pt}}
\put(483.0,606.0){\rule[-0.200pt]{0.400pt}{0.723pt}}
\put(485.0,606.0){\usebox{\plotpoint}}
\put(485.0,606.0){\rule[-0.200pt]{0.482pt}{0.400pt}}
\put(486.67,604){\rule{0.400pt}{0.964pt}}
\multiput(486.17,606.00)(1.000,-2.000){2}{\rule{0.400pt}{0.482pt}}
\put(487.67,600){\rule{0.400pt}{0.964pt}}
\multiput(487.17,602.00)(1.000,-2.000){2}{\rule{0.400pt}{0.482pt}}
\put(487.0,606.0){\rule[-0.200pt]{0.400pt}{0.482pt}}
\put(489.0,600.0){\rule[-0.200pt]{0.400pt}{2.409pt}}
\put(490,608.67){\rule{0.241pt}{0.400pt}}
\multiput(490.00,609.17)(0.500,-1.000){2}{\rule{0.120pt}{0.400pt}}
\put(489.0,610.0){\usebox{\plotpoint}}
\put(491,609){\usebox{\plotpoint}}
\put(490.67,600){\rule{0.400pt}{2.168pt}}
\multiput(490.17,604.50)(1.000,-4.500){2}{\rule{0.400pt}{1.084pt}}
\put(491.67,600){\rule{0.400pt}{0.723pt}}
\multiput(491.17,600.00)(1.000,1.500){2}{\rule{0.400pt}{0.361pt}}
\put(493,603){\usebox{\plotpoint}}
\put(492.67,603){\rule{0.400pt}{1.445pt}}
\multiput(492.17,603.00)(1.000,3.000){2}{\rule{0.400pt}{0.723pt}}
\put(494,608.67){\rule{0.241pt}{0.400pt}}
\multiput(494.00,608.17)(0.500,1.000){2}{\rule{0.120pt}{0.400pt}}
\put(494.67,605){\rule{0.400pt}{0.723pt}}
\multiput(494.17,605.00)(1.000,1.500){2}{\rule{0.400pt}{0.361pt}}
\put(496,606.67){\rule{0.241pt}{0.400pt}}
\multiput(496.00,607.17)(0.500,-1.000){2}{\rule{0.120pt}{0.400pt}}
\put(495.0,605.0){\rule[-0.200pt]{0.400pt}{1.204pt}}
\put(497,607){\usebox{\plotpoint}}
\put(496.67,607){\rule{0.400pt}{0.723pt}}
\multiput(496.17,607.00)(1.000,1.500){2}{\rule{0.400pt}{0.361pt}}
\put(497.67,600){\rule{0.400pt}{2.650pt}}
\multiput(497.17,605.50)(1.000,-5.500){2}{\rule{0.400pt}{1.325pt}}
\put(498.67,600){\rule{0.400pt}{0.482pt}}
\multiput(498.17,600.00)(1.000,1.000){2}{\rule{0.400pt}{0.241pt}}
\put(498.0,610.0){\usebox{\plotpoint}}
\put(500,602){\usebox{\plotpoint}}
\put(499.67,602){\rule{0.400pt}{0.723pt}}
\multiput(499.17,602.00)(1.000,1.500){2}{\rule{0.400pt}{0.361pt}}
\put(500.67,603){\rule{0.400pt}{0.482pt}}
\multiput(500.17,604.00)(1.000,-1.000){2}{\rule{0.400pt}{0.241pt}}
\put(501.67,600){\rule{0.400pt}{0.964pt}}
\multiput(501.17,600.00)(1.000,2.000){2}{\rule{0.400pt}{0.482pt}}
\put(502.67,600){\rule{0.400pt}{0.964pt}}
\multiput(502.17,602.00)(1.000,-2.000){2}{\rule{0.400pt}{0.482pt}}
\put(502.0,600.0){\rule[-0.200pt]{0.400pt}{0.723pt}}
\put(504,602.67){\rule{0.241pt}{0.400pt}}
\multiput(504.00,603.17)(0.500,-1.000){2}{\rule{0.120pt}{0.400pt}}
\put(504.67,603){\rule{0.400pt}{0.482pt}}
\multiput(504.17,603.00)(1.000,1.000){2}{\rule{0.400pt}{0.241pt}}
\put(504.0,600.0){\rule[-0.200pt]{0.400pt}{0.964pt}}
\put(505.67,601){\rule{0.400pt}{0.482pt}}
\multiput(505.17,602.00)(1.000,-1.000){2}{\rule{0.400pt}{0.241pt}}
\put(506.67,601){\rule{0.400pt}{0.482pt}}
\multiput(506.17,601.00)(1.000,1.000){2}{\rule{0.400pt}{0.241pt}}
\put(506.0,603.0){\rule[-0.200pt]{0.400pt}{0.482pt}}
\put(508,603){\usebox{\plotpoint}}
\put(507.67,600){\rule{0.400pt}{0.723pt}}
\multiput(507.17,601.50)(1.000,-1.500){2}{\rule{0.400pt}{0.361pt}}
\put(508.67,600){\rule{0.400pt}{0.482pt}}
\multiput(508.17,600.00)(1.000,1.000){2}{\rule{0.400pt}{0.241pt}}
\put(509.67,597){\rule{0.400pt}{0.723pt}}
\multiput(509.17,597.00)(1.000,1.500){2}{\rule{0.400pt}{0.361pt}}
\put(510.67,598){\rule{0.400pt}{0.482pt}}
\multiput(510.17,599.00)(1.000,-1.000){2}{\rule{0.400pt}{0.241pt}}
\put(510.0,597.0){\rule[-0.200pt]{0.400pt}{1.204pt}}
\put(512,598){\usebox{\plotpoint}}
\put(511.67,598){\rule{0.400pt}{0.723pt}}
\multiput(511.17,598.00)(1.000,1.500){2}{\rule{0.400pt}{0.361pt}}
\put(513,600.67){\rule{0.241pt}{0.400pt}}
\multiput(513.00,600.17)(0.500,1.000){2}{\rule{0.120pt}{0.400pt}}
\put(513.67,598){\rule{0.400pt}{0.482pt}}
\multiput(513.17,598.00)(1.000,1.000){2}{\rule{0.400pt}{0.241pt}}
\put(514.0,598.0){\rule[-0.200pt]{0.400pt}{0.964pt}}
\put(514.67,592){\rule{0.400pt}{2.409pt}}
\multiput(514.17,597.00)(1.000,-5.000){2}{\rule{0.400pt}{1.204pt}}
\put(515.67,592){\rule{0.400pt}{0.482pt}}
\multiput(515.17,592.00)(1.000,1.000){2}{\rule{0.400pt}{0.241pt}}
\put(515.0,600.0){\rule[-0.200pt]{0.400pt}{0.482pt}}
\put(516.67,590){\rule{0.400pt}{0.723pt}}
\multiput(516.17,591.50)(1.000,-1.500){2}{\rule{0.400pt}{0.361pt}}
\put(518,589.67){\rule{0.241pt}{0.400pt}}
\multiput(518.00,589.17)(0.500,1.000){2}{\rule{0.120pt}{0.400pt}}
\put(517.0,593.0){\usebox{\plotpoint}}
\put(518.67,587){\rule{0.400pt}{1.686pt}}
\multiput(518.17,590.50)(1.000,-3.500){2}{\rule{0.400pt}{0.843pt}}
\put(519.67,587){\rule{0.400pt}{0.723pt}}
\multiput(519.17,587.00)(1.000,1.500){2}{\rule{0.400pt}{0.361pt}}
\put(519.0,591.0){\rule[-0.200pt]{0.400pt}{0.723pt}}
\put(521,594.67){\rule{0.241pt}{0.400pt}}
\multiput(521.00,595.17)(0.500,-1.000){2}{\rule{0.120pt}{0.400pt}}
\put(521.67,588){\rule{0.400pt}{1.686pt}}
\multiput(521.17,591.50)(1.000,-3.500){2}{\rule{0.400pt}{0.843pt}}
\put(521.0,590.0){\rule[-0.200pt]{0.400pt}{1.445pt}}
\put(522.67,586){\rule{0.400pt}{0.964pt}}
\multiput(522.17,586.00)(1.000,2.000){2}{\rule{0.400pt}{0.482pt}}
\put(523.67,587){\rule{0.400pt}{0.723pt}}
\multiput(523.17,588.50)(1.000,-1.500){2}{\rule{0.400pt}{0.361pt}}
\put(523.0,586.0){\rule[-0.200pt]{0.400pt}{0.482pt}}
\put(525,583.67){\rule{0.241pt}{0.400pt}}
\multiput(525.00,584.17)(0.500,-1.000){2}{\rule{0.120pt}{0.400pt}}
\put(525.67,584){\rule{0.400pt}{0.482pt}}
\multiput(525.17,584.00)(1.000,1.000){2}{\rule{0.400pt}{0.241pt}}
\put(525.0,585.0){\rule[-0.200pt]{0.400pt}{0.482pt}}
\put(527.0,586.0){\usebox{\plotpoint}}
\put(527.67,585){\rule{0.400pt}{0.482pt}}
\multiput(527.17,586.00)(1.000,-1.000){2}{\rule{0.400pt}{0.241pt}}
\put(527.0,587.0){\usebox{\plotpoint}}
\put(529,580.67){\rule{0.241pt}{0.400pt}}
\multiput(529.00,581.17)(0.500,-1.000){2}{\rule{0.120pt}{0.400pt}}
\put(529.0,582.0){\rule[-0.200pt]{0.400pt}{0.723pt}}
\put(529.67,582){\rule{0.400pt}{0.723pt}}
\multiput(529.17,582.00)(1.000,1.500){2}{\rule{0.400pt}{0.361pt}}
\put(530.67,582){\rule{0.400pt}{0.723pt}}
\multiput(530.17,583.50)(1.000,-1.500){2}{\rule{0.400pt}{0.361pt}}
\put(530.0,581.0){\usebox{\plotpoint}}
\put(531.67,579){\rule{0.400pt}{0.482pt}}
\multiput(531.17,579.00)(1.000,1.000){2}{\rule{0.400pt}{0.241pt}}
\put(532.67,579){\rule{0.400pt}{0.482pt}}
\multiput(532.17,580.00)(1.000,-1.000){2}{\rule{0.400pt}{0.241pt}}
\put(532.0,579.0){\rule[-0.200pt]{0.400pt}{0.723pt}}
\put(534.0,578.0){\usebox{\plotpoint}}
\put(534.67,576){\rule{0.400pt}{0.482pt}}
\multiput(534.17,577.00)(1.000,-1.000){2}{\rule{0.400pt}{0.241pt}}
\put(534.0,578.0){\usebox{\plotpoint}}
\put(536,576){\usebox{\plotpoint}}
\put(535.67,576){\rule{0.400pt}{0.723pt}}
\multiput(535.17,576.00)(1.000,1.500){2}{\rule{0.400pt}{0.361pt}}
\put(537.0,579.0){\usebox{\plotpoint}}
\put(537.67,572){\rule{0.400pt}{0.482pt}}
\multiput(537.17,573.00)(1.000,-1.000){2}{\rule{0.400pt}{0.241pt}}
\put(538.67,572){\rule{0.400pt}{0.482pt}}
\multiput(538.17,572.00)(1.000,1.000){2}{\rule{0.400pt}{0.241pt}}
\put(538.0,574.0){\rule[-0.200pt]{0.400pt}{1.204pt}}
\put(540,570.67){\rule{0.241pt}{0.400pt}}
\multiput(540.00,570.17)(0.500,1.000){2}{\rule{0.120pt}{0.400pt}}
\put(540.67,568){\rule{0.400pt}{0.964pt}}
\multiput(540.17,570.00)(1.000,-2.000){2}{\rule{0.400pt}{0.482pt}}
\put(540.0,571.0){\rule[-0.200pt]{0.400pt}{0.723pt}}
\put(541.67,569){\rule{0.400pt}{1.204pt}}
\multiput(541.17,571.50)(1.000,-2.500){2}{\rule{0.400pt}{0.602pt}}
\put(542.67,569){\rule{0.400pt}{0.964pt}}
\multiput(542.17,569.00)(1.000,2.000){2}{\rule{0.400pt}{0.482pt}}
\put(542.0,568.0){\rule[-0.200pt]{0.400pt}{1.445pt}}
\put(544,568.67){\rule{0.241pt}{0.400pt}}
\multiput(544.00,568.17)(0.500,1.000){2}{\rule{0.120pt}{0.400pt}}
\put(544.0,569.0){\rule[-0.200pt]{0.400pt}{0.964pt}}
\put(544.67,567){\rule{0.400pt}{0.482pt}}
\multiput(544.17,568.00)(1.000,-1.000){2}{\rule{0.400pt}{0.241pt}}
\put(545.67,567){\rule{0.400pt}{0.964pt}}
\multiput(545.17,567.00)(1.000,2.000){2}{\rule{0.400pt}{0.482pt}}
\put(545.0,569.0){\usebox{\plotpoint}}
\put(546.67,567){\rule{0.400pt}{0.482pt}}
\multiput(546.17,568.00)(1.000,-1.000){2}{\rule{0.400pt}{0.241pt}}
\put(547.0,569.0){\rule[-0.200pt]{0.400pt}{0.482pt}}
\put(548.0,567.0){\usebox{\plotpoint}}
\put(548.67,566){\rule{0.400pt}{0.723pt}}
\multiput(548.17,567.50)(1.000,-1.500){2}{\rule{0.400pt}{0.361pt}}
\put(549.67,566){\rule{0.400pt}{0.482pt}}
\multiput(549.17,566.00)(1.000,1.000){2}{\rule{0.400pt}{0.241pt}}
\put(549.0,567.0){\rule[-0.200pt]{0.400pt}{0.482pt}}
\put(550.67,560){\rule{0.400pt}{0.723pt}}
\multiput(550.17,561.50)(1.000,-1.500){2}{\rule{0.400pt}{0.361pt}}
\put(551.67,560){\rule{0.400pt}{0.482pt}}
\multiput(551.17,560.00)(1.000,1.000){2}{\rule{0.400pt}{0.241pt}}
\put(551.0,563.0){\rule[-0.200pt]{0.400pt}{1.204pt}}
\put(552.67,559){\rule{0.400pt}{0.482pt}}
\multiput(552.17,560.00)(1.000,-1.000){2}{\rule{0.400pt}{0.241pt}}
\put(554,557.67){\rule{0.241pt}{0.400pt}}
\multiput(554.00,558.17)(0.500,-1.000){2}{\rule{0.120pt}{0.400pt}}
\put(553.0,561.0){\usebox{\plotpoint}}
\put(554.67,555){\rule{0.400pt}{1.686pt}}
\multiput(554.17,555.00)(1.000,3.500){2}{\rule{0.400pt}{0.843pt}}
\put(555.67,557){\rule{0.400pt}{1.204pt}}
\multiput(555.17,559.50)(1.000,-2.500){2}{\rule{0.400pt}{0.602pt}}
\put(555.0,555.0){\rule[-0.200pt]{0.400pt}{0.723pt}}
\put(556.67,559){\rule{0.400pt}{0.482pt}}
\multiput(556.17,559.00)(1.000,1.000){2}{\rule{0.400pt}{0.241pt}}
\put(557.67,559){\rule{0.400pt}{0.482pt}}
\multiput(557.17,560.00)(1.000,-1.000){2}{\rule{0.400pt}{0.241pt}}
\put(557.0,557.0){\rule[-0.200pt]{0.400pt}{0.482pt}}
\put(558.67,554){\rule{0.400pt}{0.482pt}}
\multiput(558.17,555.00)(1.000,-1.000){2}{\rule{0.400pt}{0.241pt}}
\put(559.0,556.0){\rule[-0.200pt]{0.400pt}{0.723pt}}
\put(560,554){\usebox{\plotpoint}}
\put(559.67,554){\rule{0.400pt}{0.482pt}}
\multiput(559.17,554.00)(1.000,1.000){2}{\rule{0.400pt}{0.241pt}}
\put(560.67,554){\rule{0.400pt}{0.482pt}}
\multiput(560.17,555.00)(1.000,-1.000){2}{\rule{0.400pt}{0.241pt}}
\put(562,554){\usebox{\plotpoint}}
\put(563,553.67){\rule{0.241pt}{0.400pt}}
\multiput(563.00,553.17)(0.500,1.000){2}{\rule{0.120pt}{0.400pt}}
\put(562.0,554.0){\usebox{\plotpoint}}
\put(563.67,545){\rule{0.400pt}{2.409pt}}
\multiput(563.17,545.00)(1.000,5.000){2}{\rule{0.400pt}{1.204pt}}
\put(564.67,551){\rule{0.400pt}{0.964pt}}
\multiput(564.17,553.00)(1.000,-2.000){2}{\rule{0.400pt}{0.482pt}}
\put(564.0,545.0){\rule[-0.200pt]{0.400pt}{2.409pt}}
\put(565.67,549){\rule{0.400pt}{0.723pt}}
\multiput(565.17,550.50)(1.000,-1.500){2}{\rule{0.400pt}{0.361pt}}
\put(566.0,551.0){\usebox{\plotpoint}}
\put(568,547.67){\rule{0.241pt}{0.400pt}}
\multiput(568.00,548.17)(0.500,-1.000){2}{\rule{0.120pt}{0.400pt}}
\put(568.67,546){\rule{0.400pt}{0.482pt}}
\multiput(568.17,547.00)(1.000,-1.000){2}{\rule{0.400pt}{0.241pt}}
\put(567.0,549.0){\usebox{\plotpoint}}
\put(570.0,546.0){\usebox{\plotpoint}}
\put(570.67,541){\rule{0.400pt}{1.445pt}}
\multiput(570.17,544.00)(1.000,-3.000){2}{\rule{0.400pt}{0.723pt}}
\put(570.0,547.0){\usebox{\plotpoint}}
\put(571.67,542){\rule{0.400pt}{0.723pt}}
\multiput(571.17,542.00)(1.000,1.500){2}{\rule{0.400pt}{0.361pt}}
\put(572.0,541.0){\usebox{\plotpoint}}
\put(573,545){\usebox{\plotpoint}}
\put(573.67,540){\rule{0.400pt}{1.204pt}}
\multiput(573.17,542.50)(1.000,-2.500){2}{\rule{0.400pt}{0.602pt}}
\put(573.0,545.0){\usebox{\plotpoint}}
\put(574.67,543){\rule{0.400pt}{0.723pt}}
\multiput(574.17,544.50)(1.000,-1.500){2}{\rule{0.400pt}{0.361pt}}
\put(575.0,540.0){\rule[-0.200pt]{0.400pt}{1.445pt}}
\put(576.0,543.0){\usebox{\plotpoint}}
\put(576.67,537){\rule{0.400pt}{0.723pt}}
\multiput(576.17,537.00)(1.000,1.500){2}{\rule{0.400pt}{0.361pt}}
\put(577.67,537){\rule{0.400pt}{0.723pt}}
\multiput(577.17,538.50)(1.000,-1.500){2}{\rule{0.400pt}{0.361pt}}
\put(577.0,537.0){\rule[-0.200pt]{0.400pt}{1.445pt}}
\put(579.0,537.0){\rule[-0.200pt]{0.400pt}{0.723pt}}
\put(579.67,533){\rule{0.400pt}{1.686pt}}
\multiput(579.17,536.50)(1.000,-3.500){2}{\rule{0.400pt}{0.843pt}}
\put(579.0,540.0){\usebox{\plotpoint}}
\put(580.67,529){\rule{0.400pt}{1.445pt}}
\multiput(580.17,529.00)(1.000,3.000){2}{\rule{0.400pt}{0.723pt}}
\put(581.67,535){\rule{0.400pt}{0.964pt}}
\multiput(581.17,535.00)(1.000,2.000){2}{\rule{0.400pt}{0.482pt}}
\put(581.0,529.0){\rule[-0.200pt]{0.400pt}{0.964pt}}
\put(583,533.67){\rule{0.241pt}{0.400pt}}
\multiput(583.00,533.17)(0.500,1.000){2}{\rule{0.120pt}{0.400pt}}
\put(583.0,534.0){\rule[-0.200pt]{0.400pt}{1.204pt}}
\put(584,532.67){\rule{0.241pt}{0.400pt}}
\multiput(584.00,532.17)(0.500,1.000){2}{\rule{0.120pt}{0.400pt}}
\put(585,532.67){\rule{0.241pt}{0.400pt}}
\multiput(585.00,533.17)(0.500,-1.000){2}{\rule{0.120pt}{0.400pt}}
\put(584.0,533.0){\rule[-0.200pt]{0.400pt}{0.482pt}}
\put(586.0,529.0){\rule[-0.200pt]{0.400pt}{0.964pt}}
\put(587,528.67){\rule{0.241pt}{0.400pt}}
\multiput(587.00,528.17)(0.500,1.000){2}{\rule{0.120pt}{0.400pt}}
\put(586.0,529.0){\usebox{\plotpoint}}
\put(587.67,527){\rule{0.400pt}{0.723pt}}
\multiput(587.17,527.00)(1.000,1.500){2}{\rule{0.400pt}{0.361pt}}
\put(588.67,528){\rule{0.400pt}{0.482pt}}
\multiput(588.17,529.00)(1.000,-1.000){2}{\rule{0.400pt}{0.241pt}}
\put(588.0,527.0){\rule[-0.200pt]{0.400pt}{0.723pt}}
\put(590.0,525.0){\rule[-0.200pt]{0.400pt}{0.723pt}}
\put(590.67,525){\rule{0.400pt}{0.723pt}}
\multiput(590.17,525.00)(1.000,1.500){2}{\rule{0.400pt}{0.361pt}}
\put(590.0,525.0){\usebox{\plotpoint}}
\put(592,528){\usebox{\plotpoint}}
\put(591.67,524){\rule{0.400pt}{0.964pt}}
\multiput(591.17,526.00)(1.000,-2.000){2}{\rule{0.400pt}{0.482pt}}
\put(592.67,521){\rule{0.400pt}{0.723pt}}
\multiput(592.17,522.50)(1.000,-1.500){2}{\rule{0.400pt}{0.361pt}}
\put(593.67,522){\rule{0.400pt}{0.482pt}}
\multiput(593.17,522.00)(1.000,1.000){2}{\rule{0.400pt}{0.241pt}}
\put(594.67,518){\rule{0.400pt}{1.445pt}}
\multiput(594.17,521.00)(1.000,-3.000){2}{\rule{0.400pt}{0.723pt}}
\put(594.0,521.0){\usebox{\plotpoint}}
\put(595.67,519){\rule{0.400pt}{0.482pt}}
\multiput(595.17,519.00)(1.000,1.000){2}{\rule{0.400pt}{0.241pt}}
\put(596.0,518.0){\usebox{\plotpoint}}
\put(596.67,519){\rule{0.400pt}{0.723pt}}
\multiput(596.17,520.50)(1.000,-1.500){2}{\rule{0.400pt}{0.361pt}}
\put(597.67,516){\rule{0.400pt}{0.723pt}}
\multiput(597.17,517.50)(1.000,-1.500){2}{\rule{0.400pt}{0.361pt}}
\put(597.0,521.0){\usebox{\plotpoint}}
\put(598.67,509){\rule{0.400pt}{0.482pt}}
\multiput(598.17,510.00)(1.000,-1.000){2}{\rule{0.400pt}{0.241pt}}
\put(599.67,509){\rule{0.400pt}{1.686pt}}
\multiput(599.17,509.00)(1.000,3.500){2}{\rule{0.400pt}{0.843pt}}
\put(599.0,511.0){\rule[-0.200pt]{0.400pt}{1.204pt}}
\put(601,509.67){\rule{0.241pt}{0.400pt}}
\multiput(601.00,510.17)(0.500,-1.000){2}{\rule{0.120pt}{0.400pt}}
\put(602,509.67){\rule{0.241pt}{0.400pt}}
\multiput(602.00,509.17)(0.500,1.000){2}{\rule{0.120pt}{0.400pt}}
\put(601.0,511.0){\rule[-0.200pt]{0.400pt}{1.204pt}}
\put(603,511){\usebox{\plotpoint}}
\put(603.67,504){\rule{0.400pt}{1.686pt}}
\multiput(603.17,507.50)(1.000,-3.500){2}{\rule{0.400pt}{0.843pt}}
\put(603.0,511.0){\usebox{\plotpoint}}
\put(604.67,503){\rule{0.400pt}{1.445pt}}
\multiput(604.17,506.00)(1.000,-3.000){2}{\rule{0.400pt}{0.723pt}}
\put(605.0,504.0){\rule[-0.200pt]{0.400pt}{1.204pt}}
\put(606.0,503.0){\usebox{\plotpoint}}
\put(606.67,500){\rule{0.400pt}{1.204pt}}
\multiput(606.17,502.50)(1.000,-2.500){2}{\rule{0.400pt}{0.602pt}}
\put(607.0,503.0){\rule[-0.200pt]{0.400pt}{0.482pt}}
\put(608,499.67){\rule{0.241pt}{0.400pt}}
\multiput(608.00,500.17)(0.500,-1.000){2}{\rule{0.120pt}{0.400pt}}
\put(609,499.67){\rule{0.241pt}{0.400pt}}
\multiput(609.00,499.17)(0.500,1.000){2}{\rule{0.120pt}{0.400pt}}
\put(608.0,500.0){\usebox{\plotpoint}}
\put(610,496.67){\rule{0.241pt}{0.400pt}}
\multiput(610.00,496.17)(0.500,1.000){2}{\rule{0.120pt}{0.400pt}}
\put(610.67,494){\rule{0.400pt}{0.964pt}}
\multiput(610.17,496.00)(1.000,-2.000){2}{\rule{0.400pt}{0.482pt}}
\put(610.0,497.0){\rule[-0.200pt]{0.400pt}{0.964pt}}
\put(611.67,489){\rule{0.400pt}{1.927pt}}
\multiput(611.17,493.00)(1.000,-4.000){2}{\rule{0.400pt}{0.964pt}}
\put(612.67,489){\rule{0.400pt}{1.445pt}}
\multiput(612.17,489.00)(1.000,3.000){2}{\rule{0.400pt}{0.723pt}}
\put(612.0,494.0){\rule[-0.200pt]{0.400pt}{0.723pt}}
\put(613.67,483){\rule{0.400pt}{2.168pt}}
\multiput(613.17,487.50)(1.000,-4.500){2}{\rule{0.400pt}{1.084pt}}
\put(614.67,483){\rule{0.400pt}{1.445pt}}
\multiput(614.17,483.00)(1.000,3.000){2}{\rule{0.400pt}{0.723pt}}
\put(614.0,492.0){\rule[-0.200pt]{0.400pt}{0.723pt}}
\put(616,486.67){\rule{0.241pt}{0.400pt}}
\multiput(616.00,486.17)(0.500,1.000){2}{\rule{0.120pt}{0.400pt}}
\put(616.67,484){\rule{0.400pt}{0.964pt}}
\multiput(616.17,486.00)(1.000,-2.000){2}{\rule{0.400pt}{0.482pt}}
\put(616.0,487.0){\rule[-0.200pt]{0.400pt}{0.482pt}}
\put(618,484){\usebox{\plotpoint}}
\put(617.67,481){\rule{0.400pt}{0.723pt}}
\multiput(617.17,482.50)(1.000,-1.500){2}{\rule{0.400pt}{0.361pt}}
\put(619,481){\usebox{\plotpoint}}
\put(618.67,478){\rule{0.400pt}{0.723pt}}
\multiput(618.17,479.50)(1.000,-1.500){2}{\rule{0.400pt}{0.361pt}}
\put(619.67,474){\rule{0.400pt}{0.964pt}}
\multiput(619.17,476.00)(1.000,-2.000){2}{\rule{0.400pt}{0.482pt}}
\put(621,475.67){\rule{0.241pt}{0.400pt}}
\multiput(621.00,476.17)(0.500,-1.000){2}{\rule{0.120pt}{0.400pt}}
\put(622,474.67){\rule{0.241pt}{0.400pt}}
\multiput(622.00,475.17)(0.500,-1.000){2}{\rule{0.120pt}{0.400pt}}
\put(621.0,474.0){\rule[-0.200pt]{0.400pt}{0.723pt}}
\put(623,471.67){\rule{0.241pt}{0.400pt}}
\multiput(623.00,472.17)(0.500,-1.000){2}{\rule{0.120pt}{0.400pt}}
\put(623.67,470){\rule{0.400pt}{0.482pt}}
\multiput(623.17,471.00)(1.000,-1.000){2}{\rule{0.400pt}{0.241pt}}
\put(623.0,473.0){\rule[-0.200pt]{0.400pt}{0.482pt}}
\put(624.67,465){\rule{0.400pt}{1.445pt}}
\multiput(624.17,468.00)(1.000,-3.000){2}{\rule{0.400pt}{0.723pt}}
\put(625.67,465){\rule{0.400pt}{1.204pt}}
\multiput(625.17,465.00)(1.000,2.500){2}{\rule{0.400pt}{0.602pt}}
\put(625.0,470.0){\usebox{\plotpoint}}
\put(626.67,464){\rule{0.400pt}{0.723pt}}
\multiput(626.17,465.50)(1.000,-1.500){2}{\rule{0.400pt}{0.361pt}}
\put(627.67,461){\rule{0.400pt}{0.723pt}}
\multiput(627.17,462.50)(1.000,-1.500){2}{\rule{0.400pt}{0.361pt}}
\put(627.0,467.0){\rule[-0.200pt]{0.400pt}{0.723pt}}
\put(629,457.67){\rule{0.241pt}{0.400pt}}
\multiput(629.00,457.17)(0.500,1.000){2}{\rule{0.120pt}{0.400pt}}
\put(629.0,458.0){\rule[-0.200pt]{0.400pt}{0.723pt}}
\put(629.67,453){\rule{0.400pt}{1.686pt}}
\multiput(629.17,456.50)(1.000,-3.500){2}{\rule{0.400pt}{0.843pt}}
\put(630.0,459.0){\usebox{\plotpoint}}
\put(631.0,453.0){\usebox{\plotpoint}}
\put(631.67,450){\rule{0.400pt}{0.482pt}}
\multiput(631.17,451.00)(1.000,-1.000){2}{\rule{0.400pt}{0.241pt}}
\put(632.0,452.0){\usebox{\plotpoint}}
\put(633.0,450.0){\usebox{\plotpoint}}
\put(634,446.67){\rule{0.241pt}{0.400pt}}
\multiput(634.00,447.17)(0.500,-1.000){2}{\rule{0.120pt}{0.400pt}}
\put(634.67,445){\rule{0.400pt}{0.482pt}}
\multiput(634.17,446.00)(1.000,-1.000){2}{\rule{0.400pt}{0.241pt}}
\put(634.0,448.0){\rule[-0.200pt]{0.400pt}{0.482pt}}
\put(635.67,440){\rule{0.400pt}{0.964pt}}
\multiput(635.17,442.00)(1.000,-2.000){2}{\rule{0.400pt}{0.482pt}}
\put(636.0,444.0){\usebox{\plotpoint}}
\put(637.0,440.0){\usebox{\plotpoint}}
\put(637.67,435){\rule{0.400pt}{0.964pt}}
\multiput(637.17,437.00)(1.000,-2.000){2}{\rule{0.400pt}{0.482pt}}
\put(638.0,439.0){\usebox{\plotpoint}}
\put(638.67,431){\rule{0.400pt}{0.723pt}}
\multiput(638.17,432.50)(1.000,-1.500){2}{\rule{0.400pt}{0.361pt}}
\put(639.67,431){\rule{0.400pt}{0.482pt}}
\multiput(639.17,431.00)(1.000,1.000){2}{\rule{0.400pt}{0.241pt}}
\put(639.0,434.0){\usebox{\plotpoint}}
\put(640.67,427){\rule{0.400pt}{0.964pt}}
\multiput(640.17,427.00)(1.000,2.000){2}{\rule{0.400pt}{0.482pt}}
\put(641.67,429){\rule{0.400pt}{0.482pt}}
\multiput(641.17,430.00)(1.000,-1.000){2}{\rule{0.400pt}{0.241pt}}
\put(641.0,427.0){\rule[-0.200pt]{0.400pt}{1.445pt}}
\put(642.67,423){\rule{0.400pt}{0.482pt}}
\multiput(642.17,424.00)(1.000,-1.000){2}{\rule{0.400pt}{0.241pt}}
\put(643.67,420){\rule{0.400pt}{0.723pt}}
\multiput(643.17,421.50)(1.000,-1.500){2}{\rule{0.400pt}{0.361pt}}
\put(643.0,425.0){\rule[-0.200pt]{0.400pt}{0.964pt}}
\put(644.67,417){\rule{0.400pt}{0.482pt}}
\multiput(644.17,418.00)(1.000,-1.000){2}{\rule{0.400pt}{0.241pt}}
\put(646,415.67){\rule{0.241pt}{0.400pt}}
\multiput(646.00,416.17)(0.500,-1.000){2}{\rule{0.120pt}{0.400pt}}
\put(645.0,419.0){\usebox{\plotpoint}}
\put(647,412.67){\rule{0.241pt}{0.400pt}}
\multiput(647.00,413.17)(0.500,-1.000){2}{\rule{0.120pt}{0.400pt}}
\put(647.0,414.0){\rule[-0.200pt]{0.400pt}{0.482pt}}
\put(648.0,413.0){\usebox{\plotpoint}}
\put(649,405.67){\rule{0.241pt}{0.400pt}}
\multiput(649.00,406.17)(0.500,-1.000){2}{\rule{0.120pt}{0.400pt}}
\put(649.0,407.0){\rule[-0.200pt]{0.400pt}{1.445pt}}
\put(650,406){\usebox{\plotpoint}}
\put(649.67,404){\rule{0.400pt}{0.482pt}}
\multiput(649.17,405.00)(1.000,-1.000){2}{\rule{0.400pt}{0.241pt}}
\put(651,402.67){\rule{0.241pt}{0.400pt}}
\multiput(651.00,403.17)(0.500,-1.000){2}{\rule{0.120pt}{0.400pt}}
\put(651.67,400){\rule{0.400pt}{0.964pt}}
\multiput(651.17,402.00)(1.000,-2.000){2}{\rule{0.400pt}{0.482pt}}
\put(653,398.67){\rule{0.241pt}{0.400pt}}
\multiput(653.00,399.17)(0.500,-1.000){2}{\rule{0.120pt}{0.400pt}}
\put(652.0,403.0){\usebox{\plotpoint}}
\put(654.0,395.0){\rule[-0.200pt]{0.400pt}{0.964pt}}
\put(654.67,392){\rule{0.400pt}{0.723pt}}
\multiput(654.17,393.50)(1.000,-1.500){2}{\rule{0.400pt}{0.361pt}}
\put(654.0,395.0){\usebox{\plotpoint}}
\put(656,392){\usebox{\plotpoint}}
\put(655.67,389){\rule{0.400pt}{0.723pt}}
\multiput(655.17,390.50)(1.000,-1.500){2}{\rule{0.400pt}{0.361pt}}
\put(656.67,386){\rule{0.400pt}{0.723pt}}
\multiput(656.17,387.50)(1.000,-1.500){2}{\rule{0.400pt}{0.361pt}}
\put(658.0,385.0){\usebox{\plotpoint}}
\put(658.0,385.0){\usebox{\plotpoint}}
\put(659.0,382.0){\rule[-0.200pt]{0.400pt}{0.723pt}}
\put(659.67,380){\rule{0.400pt}{0.482pt}}
\multiput(659.17,381.00)(1.000,-1.000){2}{\rule{0.400pt}{0.241pt}}
\put(659.0,382.0){\usebox{\plotpoint}}
\put(660.67,374){\rule{0.400pt}{1.204pt}}
\multiput(660.17,376.50)(1.000,-2.500){2}{\rule{0.400pt}{0.602pt}}
\put(661.67,374){\rule{0.400pt}{0.723pt}}
\multiput(661.17,374.00)(1.000,1.500){2}{\rule{0.400pt}{0.361pt}}
\put(661.0,379.0){\usebox{\plotpoint}}
\put(662.67,371){\rule{0.400pt}{0.723pt}}
\multiput(662.17,372.50)(1.000,-1.500){2}{\rule{0.400pt}{0.361pt}}
\put(663.67,368){\rule{0.400pt}{0.723pt}}
\multiput(663.17,369.50)(1.000,-1.500){2}{\rule{0.400pt}{0.361pt}}
\put(663.0,374.0){\rule[-0.200pt]{0.400pt}{0.723pt}}
\put(665,366.67){\rule{0.241pt}{0.400pt}}
\multiput(665.00,366.17)(0.500,1.000){2}{\rule{0.120pt}{0.400pt}}
\put(665.67,363){\rule{0.400pt}{1.204pt}}
\multiput(665.17,365.50)(1.000,-2.500){2}{\rule{0.400pt}{0.602pt}}
\put(665.0,367.0){\usebox{\plotpoint}}
\put(667,361.67){\rule{0.241pt}{0.400pt}}
\multiput(667.00,361.17)(0.500,1.000){2}{\rule{0.120pt}{0.400pt}}
\put(667.0,362.0){\usebox{\plotpoint}}
\put(668.0,360.0){\rule[-0.200pt]{0.400pt}{0.723pt}}
\put(668.67,354){\rule{0.400pt}{1.445pt}}
\multiput(668.17,357.00)(1.000,-3.000){2}{\rule{0.400pt}{0.723pt}}
\put(668.0,360.0){\usebox{\plotpoint}}
\put(670,354){\usebox{\plotpoint}}
\put(670,353.67){\rule{0.241pt}{0.400pt}}
\multiput(670.00,353.17)(0.500,1.000){2}{\rule{0.120pt}{0.400pt}}
\put(670.67,351){\rule{0.400pt}{0.964pt}}
\multiput(670.17,353.00)(1.000,-2.000){2}{\rule{0.400pt}{0.482pt}}
\put(672,348.67){\rule{0.241pt}{0.400pt}}
\multiput(672.00,349.17)(0.500,-1.000){2}{\rule{0.120pt}{0.400pt}}
\put(673,347.67){\rule{0.241pt}{0.400pt}}
\multiput(673.00,348.17)(0.500,-1.000){2}{\rule{0.120pt}{0.400pt}}
\put(672.0,350.0){\usebox{\plotpoint}}
\put(674,345.67){\rule{0.241pt}{0.400pt}}
\multiput(674.00,345.17)(0.500,1.000){2}{\rule{0.120pt}{0.400pt}}
\put(674.67,345){\rule{0.400pt}{0.482pt}}
\multiput(674.17,346.00)(1.000,-1.000){2}{\rule{0.400pt}{0.241pt}}
\put(674.0,346.0){\rule[-0.200pt]{0.400pt}{0.482pt}}
\put(675.67,341){\rule{0.400pt}{0.723pt}}
\multiput(675.17,342.50)(1.000,-1.500){2}{\rule{0.400pt}{0.361pt}}
\put(676.0,344.0){\usebox{\plotpoint}}
\put(676.67,337){\rule{0.400pt}{0.482pt}}
\multiput(676.17,338.00)(1.000,-1.000){2}{\rule{0.400pt}{0.241pt}}
\put(677.0,339.0){\rule[-0.200pt]{0.400pt}{0.482pt}}
\put(678.0,337.0){\usebox{\plotpoint}}
\put(679,334.67){\rule{0.241pt}{0.400pt}}
\multiput(679.00,335.17)(0.500,-1.000){2}{\rule{0.120pt}{0.400pt}}
\put(680,334.67){\rule{0.241pt}{0.400pt}}
\multiput(680.00,334.17)(0.500,1.000){2}{\rule{0.120pt}{0.400pt}}
\put(679.0,336.0){\usebox{\plotpoint}}
\put(680.67,331){\rule{0.400pt}{0.482pt}}
\multiput(680.17,332.00)(1.000,-1.000){2}{\rule{0.400pt}{0.241pt}}
\put(681.67,328){\rule{0.400pt}{0.723pt}}
\multiput(681.17,329.50)(1.000,-1.500){2}{\rule{0.400pt}{0.361pt}}
\put(681.0,333.0){\rule[-0.200pt]{0.400pt}{0.723pt}}
\put(683,328.67){\rule{0.241pt}{0.400pt}}
\multiput(683.00,329.17)(0.500,-1.000){2}{\rule{0.120pt}{0.400pt}}
\put(683.67,326){\rule{0.400pt}{0.723pt}}
\multiput(683.17,327.50)(1.000,-1.500){2}{\rule{0.400pt}{0.361pt}}
\put(683.0,328.0){\rule[-0.200pt]{0.400pt}{0.482pt}}
\put(685,323.67){\rule{0.241pt}{0.400pt}}
\multiput(685.00,324.17)(0.500,-1.000){2}{\rule{0.120pt}{0.400pt}}
\put(685.0,325.0){\usebox{\plotpoint}}
\put(686.0,323.0){\usebox{\plotpoint}}
\put(686.0,323.0){\rule[-0.200pt]{0.482pt}{0.400pt}}
\put(688,319.67){\rule{0.241pt}{0.400pt}}
\multiput(688.00,320.17)(0.500,-1.000){2}{\rule{0.120pt}{0.400pt}}
\put(689,318.67){\rule{0.241pt}{0.400pt}}
\multiput(689.00,319.17)(0.500,-1.000){2}{\rule{0.120pt}{0.400pt}}
\put(688.0,321.0){\rule[-0.200pt]{0.400pt}{0.482pt}}
\put(690,319){\usebox{\plotpoint}}
\put(689.67,317){\rule{0.400pt}{0.482pt}}
\multiput(689.17,318.00)(1.000,-1.000){2}{\rule{0.400pt}{0.241pt}}
\put(691,315.67){\rule{0.241pt}{0.400pt}}
\multiput(691.00,316.17)(0.500,-1.000){2}{\rule{0.120pt}{0.400pt}}
\put(692,313.67){\rule{0.241pt}{0.400pt}}
\multiput(692.00,314.17)(0.500,-1.000){2}{\rule{0.120pt}{0.400pt}}
\put(693,312.67){\rule{0.241pt}{0.400pt}}
\multiput(693.00,313.17)(0.500,-1.000){2}{\rule{0.120pt}{0.400pt}}
\put(692.0,315.0){\usebox{\plotpoint}}
\put(694,311.67){\rule{0.241pt}{0.400pt}}
\multiput(694.00,311.17)(0.500,1.000){2}{\rule{0.120pt}{0.400pt}}
\put(694.0,312.0){\usebox{\plotpoint}}
\put(695.0,310.0){\rule[-0.200pt]{0.400pt}{0.723pt}}
\put(696,308.67){\rule{0.241pt}{0.400pt}}
\multiput(696.00,309.17)(0.500,-1.000){2}{\rule{0.120pt}{0.400pt}}
\put(695.0,310.0){\usebox{\plotpoint}}
\put(697,307.67){\rule{0.241pt}{0.400pt}}
\multiput(697.00,307.17)(0.500,1.000){2}{\rule{0.120pt}{0.400pt}}
\put(697.67,305){\rule{0.400pt}{0.964pt}}
\multiput(697.17,307.00)(1.000,-2.000){2}{\rule{0.400pt}{0.482pt}}
\put(697.0,308.0){\usebox{\plotpoint}}
\put(698.67,304){\rule{0.400pt}{0.482pt}}
\multiput(698.17,305.00)(1.000,-1.000){2}{\rule{0.400pt}{0.241pt}}
\put(700,303.67){\rule{0.241pt}{0.400pt}}
\multiput(700.00,303.17)(0.500,1.000){2}{\rule{0.120pt}{0.400pt}}
\put(699.0,305.0){\usebox{\plotpoint}}
\put(701.0,303.0){\rule[-0.200pt]{0.400pt}{0.482pt}}
\put(701.67,301){\rule{0.400pt}{0.482pt}}
\multiput(701.17,302.00)(1.000,-1.000){2}{\rule{0.400pt}{0.241pt}}
\put(701.0,303.0){\usebox{\plotpoint}}
\put(703.0,301.0){\rule[-0.200pt]{0.400pt}{0.482pt}}
\put(703.0,303.0){\usebox{\plotpoint}}
\put(704,299.67){\rule{0.241pt}{0.400pt}}
\multiput(704.00,300.17)(0.500,-1.000){2}{\rule{0.120pt}{0.400pt}}
\put(704.0,301.0){\rule[-0.200pt]{0.400pt}{0.482pt}}
\put(706,298.67){\rule{0.241pt}{0.400pt}}
\multiput(706.00,299.17)(0.500,-1.000){2}{\rule{0.120pt}{0.400pt}}
\put(707,297.67){\rule{0.241pt}{0.400pt}}
\multiput(707.00,298.17)(0.500,-1.000){2}{\rule{0.120pt}{0.400pt}}
\put(705.0,300.0){\usebox{\plotpoint}}
\put(708,298.67){\rule{0.241pt}{0.400pt}}
\multiput(708.00,299.17)(0.500,-1.000){2}{\rule{0.120pt}{0.400pt}}
\put(708.67,297){\rule{0.400pt}{0.482pt}}
\multiput(708.17,298.00)(1.000,-1.000){2}{\rule{0.400pt}{0.241pt}}
\put(708.0,298.0){\rule[-0.200pt]{0.400pt}{0.482pt}}
\put(710.0,296.0){\usebox{\plotpoint}}
\put(710.0,296.0){\usebox{\plotpoint}}
\put(711.0,296.0){\usebox{\plotpoint}}
\put(711.0,297.0){\rule[-0.200pt]{0.482pt}{0.400pt}}
\put(713,293.67){\rule{0.241pt}{0.400pt}}
\multiput(713.00,293.17)(0.500,1.000){2}{\rule{0.120pt}{0.400pt}}
\put(714,293.67){\rule{0.241pt}{0.400pt}}
\multiput(714.00,294.17)(0.500,-1.000){2}{\rule{0.120pt}{0.400pt}}
\put(713.0,294.0){\rule[-0.200pt]{0.400pt}{0.723pt}}
\put(715,294.67){\rule{0.241pt}{0.400pt}}
\multiput(715.00,295.17)(0.500,-1.000){2}{\rule{0.120pt}{0.400pt}}
\put(716,294.67){\rule{0.241pt}{0.400pt}}
\multiput(716.00,294.17)(0.500,1.000){2}{\rule{0.120pt}{0.400pt}}
\put(715.0,294.0){\rule[-0.200pt]{0.400pt}{0.482pt}}
\put(717.0,295.0){\usebox{\plotpoint}}
\put(717.0,295.0){\rule[-0.200pt]{0.482pt}{0.400pt}}
\put(719.0,294.0){\usebox{\plotpoint}}
\put(719.0,294.0){\usebox{\plotpoint}}
\put(720,292.67){\rule{0.241pt}{0.400pt}}
\multiput(720.00,292.17)(0.500,1.000){2}{\rule{0.120pt}{0.400pt}}
\put(721,292.67){\rule{0.241pt}{0.400pt}}
\multiput(721.00,293.17)(0.500,-1.000){2}{\rule{0.120pt}{0.400pt}}
\put(720.0,293.0){\usebox{\plotpoint}}
\put(722,293){\usebox{\plotpoint}}
\put(722,291.67){\rule{0.241pt}{0.400pt}}
\multiput(722.00,292.17)(0.500,-1.000){2}{\rule{0.120pt}{0.400pt}}
\put(723.0,292.0){\usebox{\plotpoint}}
\put(723.67,291){\rule{0.400pt}{0.723pt}}
\multiput(723.17,291.00)(1.000,1.500){2}{\rule{0.400pt}{0.361pt}}
\put(725,292.67){\rule{0.241pt}{0.400pt}}
\multiput(725.00,293.17)(0.500,-1.000){2}{\rule{0.120pt}{0.400pt}}
\put(724.0,291.0){\usebox{\plotpoint}}
\put(726,291.67){\rule{0.241pt}{0.400pt}}
\multiput(726.00,291.17)(0.500,1.000){2}{\rule{0.120pt}{0.400pt}}
\put(726.0,292.0){\usebox{\plotpoint}}
\put(727,291.67){\rule{0.241pt}{0.400pt}}
\multiput(727.00,291.17)(0.500,1.000){2}{\rule{0.120pt}{0.400pt}}
\put(728,292.67){\rule{0.241pt}{0.400pt}}
\multiput(728.00,292.17)(0.500,1.000){2}{\rule{0.120pt}{0.400pt}}
\put(727.0,292.0){\usebox{\plotpoint}}
\put(729,291.67){\rule{0.241pt}{0.400pt}}
\multiput(729.00,292.17)(0.500,-1.000){2}{\rule{0.120pt}{0.400pt}}
\put(730,291.67){\rule{0.241pt}{0.400pt}}
\multiput(730.00,291.17)(0.500,1.000){2}{\rule{0.120pt}{0.400pt}}
\put(729.0,293.0){\usebox{\plotpoint}}
\put(731,290.67){\rule{0.241pt}{0.400pt}}
\multiput(731.00,290.17)(0.500,1.000){2}{\rule{0.120pt}{0.400pt}}
\put(732,290.67){\rule{0.241pt}{0.400pt}}
\multiput(732.00,291.17)(0.500,-1.000){2}{\rule{0.120pt}{0.400pt}}
\put(731.0,291.0){\rule[-0.200pt]{0.400pt}{0.482pt}}
\put(733,292.67){\rule{0.241pt}{0.400pt}}
\multiput(733.00,293.17)(0.500,-1.000){2}{\rule{0.120pt}{0.400pt}}
\put(733.0,291.0){\rule[-0.200pt]{0.400pt}{0.723pt}}
\put(734.0,293.0){\usebox{\plotpoint}}
\put(735.0,292.0){\usebox{\plotpoint}}
\put(735.0,292.0){\usebox{\plotpoint}}
\put(735.67,292){\rule{0.400pt}{0.482pt}}
\multiput(735.17,293.00)(1.000,-1.000){2}{\rule{0.400pt}{0.241pt}}
\put(737,291.67){\rule{0.241pt}{0.400pt}}
\multiput(737.00,291.17)(0.500,1.000){2}{\rule{0.120pt}{0.400pt}}
\put(736.0,292.0){\rule[-0.200pt]{0.400pt}{0.482pt}}
\put(738,293){\usebox{\plotpoint}}
\put(738,292.67){\rule{0.241pt}{0.400pt}}
\multiput(738.00,292.17)(0.500,1.000){2}{\rule{0.120pt}{0.400pt}}
\put(739.0,294.0){\usebox{\plotpoint}}
\put(740,292.67){\rule{0.241pt}{0.400pt}}
\multiput(740.00,292.17)(0.500,1.000){2}{\rule{0.120pt}{0.400pt}}
\put(741,293.67){\rule{0.241pt}{0.400pt}}
\multiput(741.00,293.17)(0.500,1.000){2}{\rule{0.120pt}{0.400pt}}
\put(740.0,293.0){\usebox{\plotpoint}}
\put(742,293.67){\rule{0.241pt}{0.400pt}}
\multiput(742.00,293.17)(0.500,1.000){2}{\rule{0.120pt}{0.400pt}}
\put(742.0,294.0){\usebox{\plotpoint}}
\put(743.0,294.0){\usebox{\plotpoint}}
\put(743.67,294){\rule{0.400pt}{0.482pt}}
\multiput(743.17,294.00)(1.000,1.000){2}{\rule{0.400pt}{0.241pt}}
\put(743.0,294.0){\usebox{\plotpoint}}
\put(745,294.67){\rule{0.241pt}{0.400pt}}
\multiput(745.00,294.17)(0.500,1.000){2}{\rule{0.120pt}{0.400pt}}
\put(746,294.67){\rule{0.241pt}{0.400pt}}
\multiput(746.00,295.17)(0.500,-1.000){2}{\rule{0.120pt}{0.400pt}}
\put(745.0,295.0){\usebox{\plotpoint}}
\put(747,293.67){\rule{0.241pt}{0.400pt}}
\multiput(747.00,293.17)(0.500,1.000){2}{\rule{0.120pt}{0.400pt}}
\put(748,294.67){\rule{0.241pt}{0.400pt}}
\multiput(748.00,294.17)(0.500,1.000){2}{\rule{0.120pt}{0.400pt}}
\put(747.0,294.0){\usebox{\plotpoint}}
\put(748.67,295){\rule{0.400pt}{0.482pt}}
\multiput(748.17,295.00)(1.000,1.000){2}{\rule{0.400pt}{0.241pt}}
\put(749.0,295.0){\usebox{\plotpoint}}
\put(750,297){\usebox{\plotpoint}}
\put(751,296.67){\rule{0.241pt}{0.400pt}}
\multiput(751.00,296.17)(0.500,1.000){2}{\rule{0.120pt}{0.400pt}}
\put(750.0,297.0){\usebox{\plotpoint}}
\put(752,296.67){\rule{0.241pt}{0.400pt}}
\multiput(752.00,296.17)(0.500,1.000){2}{\rule{0.120pt}{0.400pt}}
\put(753,296.67){\rule{0.241pt}{0.400pt}}
\multiput(753.00,297.17)(0.500,-1.000){2}{\rule{0.120pt}{0.400pt}}
\put(752.0,297.0){\usebox{\plotpoint}}
\put(754,297.67){\rule{0.241pt}{0.400pt}}
\multiput(754.00,297.17)(0.500,1.000){2}{\rule{0.120pt}{0.400pt}}
\put(755,297.67){\rule{0.241pt}{0.400pt}}
\multiput(755.00,298.17)(0.500,-1.000){2}{\rule{0.120pt}{0.400pt}}
\put(754.0,297.0){\usebox{\plotpoint}}
\put(756,298.67){\rule{0.241pt}{0.400pt}}
\multiput(756.00,299.17)(0.500,-1.000){2}{\rule{0.120pt}{0.400pt}}
\put(756.0,298.0){\rule[-0.200pt]{0.400pt}{0.482pt}}
\put(757,297.67){\rule{0.241pt}{0.400pt}}
\multiput(757.00,297.17)(0.500,1.000){2}{\rule{0.120pt}{0.400pt}}
\put(758,298.67){\rule{0.241pt}{0.400pt}}
\multiput(758.00,298.17)(0.500,1.000){2}{\rule{0.120pt}{0.400pt}}
\put(757.0,298.0){\usebox{\plotpoint}}
\put(759,300){\usebox{\plotpoint}}
\put(759,299.67){\rule{0.241pt}{0.400pt}}
\multiput(759.00,299.17)(0.500,1.000){2}{\rule{0.120pt}{0.400pt}}
\put(760,300.67){\rule{0.241pt}{0.400pt}}
\multiput(760.00,300.17)(0.500,1.000){2}{\rule{0.120pt}{0.400pt}}
\put(761,302){\usebox{\plotpoint}}
\put(761,300.67){\rule{0.241pt}{0.400pt}}
\multiput(761.00,301.17)(0.500,-1.000){2}{\rule{0.120pt}{0.400pt}}
\put(762,300.67){\rule{0.241pt}{0.400pt}}
\multiput(762.00,300.17)(0.500,1.000){2}{\rule{0.120pt}{0.400pt}}
\put(762.67,301){\rule{0.400pt}{0.482pt}}
\multiput(762.17,302.00)(1.000,-1.000){2}{\rule{0.400pt}{0.241pt}}
\put(764,300.67){\rule{0.241pt}{0.400pt}}
\multiput(764.00,300.17)(0.500,1.000){2}{\rule{0.120pt}{0.400pt}}
\put(763.0,302.0){\usebox{\plotpoint}}
\put(765.0,302.0){\usebox{\plotpoint}}
\put(765.0,303.0){\usebox{\plotpoint}}
\put(766.0,303.0){\usebox{\plotpoint}}
\put(766.67,304){\rule{0.400pt}{0.723pt}}
\multiput(766.17,304.00)(1.000,1.500){2}{\rule{0.400pt}{0.361pt}}
\put(766.0,304.0){\usebox{\plotpoint}}
\put(768,304.67){\rule{0.241pt}{0.400pt}}
\multiput(768.00,305.17)(0.500,-1.000){2}{\rule{0.120pt}{0.400pt}}
\put(768.0,306.0){\usebox{\plotpoint}}
\put(769.0,305.0){\usebox{\plotpoint}}
\put(770,304.67){\rule{0.241pt}{0.400pt}}
\multiput(770.00,305.17)(0.500,-1.000){2}{\rule{0.120pt}{0.400pt}}
\put(770.67,305){\rule{0.400pt}{0.482pt}}
\multiput(770.17,305.00)(1.000,1.000){2}{\rule{0.400pt}{0.241pt}}
\put(770.0,305.0){\usebox{\plotpoint}}
\put(771.67,306){\rule{0.400pt}{0.482pt}}
\multiput(771.17,306.00)(1.000,1.000){2}{\rule{0.400pt}{0.241pt}}
\put(772.0,306.0){\usebox{\plotpoint}}
\put(773,308){\usebox{\plotpoint}}
\put(773,306.67){\rule{0.241pt}{0.400pt}}
\multiput(773.00,307.17)(0.500,-1.000){2}{\rule{0.120pt}{0.400pt}}
\put(774,306.67){\rule{0.241pt}{0.400pt}}
\multiput(774.00,306.17)(0.500,1.000){2}{\rule{0.120pt}{0.400pt}}
\put(775,309.67){\rule{0.241pt}{0.400pt}}
\multiput(775.00,310.17)(0.500,-1.000){2}{\rule{0.120pt}{0.400pt}}
\put(775.0,308.0){\rule[-0.200pt]{0.400pt}{0.723pt}}
\put(776.0,310.0){\usebox{\plotpoint}}
\put(777.0,310.0){\usebox{\plotpoint}}
\put(777.0,311.0){\rule[-0.200pt]{0.482pt}{0.400pt}}
\put(779.0,311.0){\rule[-0.200pt]{0.400pt}{0.482pt}}
\put(779.0,313.0){\usebox{\plotpoint}}
\put(780.0,313.0){\usebox{\plotpoint}}
\put(780.67,314){\rule{0.400pt}{0.482pt}}
\multiput(780.17,314.00)(1.000,1.000){2}{\rule{0.400pt}{0.241pt}}
\put(780.0,314.0){\usebox{\plotpoint}}
\put(781.67,315){\rule{0.400pt}{0.482pt}}
\multiput(781.17,316.00)(1.000,-1.000){2}{\rule{0.400pt}{0.241pt}}
\put(783,314.67){\rule{0.241pt}{0.400pt}}
\multiput(783.00,314.17)(0.500,1.000){2}{\rule{0.120pt}{0.400pt}}
\put(782.0,316.0){\usebox{\plotpoint}}
\put(784,316){\usebox{\plotpoint}}
\put(783.67,316){\rule{0.400pt}{0.723pt}}
\multiput(783.17,316.00)(1.000,1.500){2}{\rule{0.400pt}{0.361pt}}
\put(785,316.67){\rule{0.241pt}{0.400pt}}
\multiput(785.00,316.17)(0.500,1.000){2}{\rule{0.120pt}{0.400pt}}
\put(785.67,318){\rule{0.400pt}{0.723pt}}
\multiput(785.17,318.00)(1.000,1.500){2}{\rule{0.400pt}{0.361pt}}
\put(785.0,317.0){\rule[-0.200pt]{0.400pt}{0.482pt}}
\put(787,319.67){\rule{0.241pt}{0.400pt}}
\multiput(787.00,319.17)(0.500,1.000){2}{\rule{0.120pt}{0.400pt}}
\put(788,320.67){\rule{0.241pt}{0.400pt}}
\multiput(788.00,320.17)(0.500,1.000){2}{\rule{0.120pt}{0.400pt}}
\put(787.0,320.0){\usebox{\plotpoint}}
\put(789.0,322.0){\usebox{\plotpoint}}
\put(790,322.67){\rule{0.241pt}{0.400pt}}
\multiput(790.00,322.17)(0.500,1.000){2}{\rule{0.120pt}{0.400pt}}
\put(789.0,323.0){\usebox{\plotpoint}}
\put(791,324){\usebox{\plotpoint}}
\put(790.67,324){\rule{0.400pt}{0.482pt}}
\multiput(790.17,324.00)(1.000,1.000){2}{\rule{0.400pt}{0.241pt}}
\put(792,326){\usebox{\plotpoint}}
\put(791.67,326){\rule{0.400pt}{0.482pt}}
\multiput(791.17,326.00)(1.000,1.000){2}{\rule{0.400pt}{0.241pt}}
\put(793.0,328.0){\usebox{\plotpoint}}
\put(794,328.67){\rule{0.241pt}{0.400pt}}
\multiput(794.00,329.17)(0.500,-1.000){2}{\rule{0.120pt}{0.400pt}}
\put(794.67,329){\rule{0.400pt}{0.482pt}}
\multiput(794.17,329.00)(1.000,1.000){2}{\rule{0.400pt}{0.241pt}}
\put(794.0,328.0){\rule[-0.200pt]{0.400pt}{0.482pt}}
\put(796,331){\usebox{\plotpoint}}
\put(795.67,331){\rule{0.400pt}{0.482pt}}
\multiput(795.17,331.00)(1.000,1.000){2}{\rule{0.400pt}{0.241pt}}
\put(797.0,333.0){\usebox{\plotpoint}}
\put(798,333.67){\rule{0.241pt}{0.400pt}}
\multiput(798.00,333.17)(0.500,1.000){2}{\rule{0.120pt}{0.400pt}}
\put(798.0,333.0){\usebox{\plotpoint}}
\put(798.67,334){\rule{0.400pt}{0.723pt}}
\multiput(798.17,334.00)(1.000,1.500){2}{\rule{0.400pt}{0.361pt}}
\put(799.67,337){\rule{0.400pt}{0.482pt}}
\multiput(799.17,337.00)(1.000,1.000){2}{\rule{0.400pt}{0.241pt}}
\put(799.0,334.0){\usebox{\plotpoint}}
\put(801,339.67){\rule{0.241pt}{0.400pt}}
\multiput(801.00,339.17)(0.500,1.000){2}{\rule{0.120pt}{0.400pt}}
\put(802,340.67){\rule{0.241pt}{0.400pt}}
\multiput(802.00,340.17)(0.500,1.000){2}{\rule{0.120pt}{0.400pt}}
\put(801.0,339.0){\usebox{\plotpoint}}
\put(803.0,342.0){\usebox{\plotpoint}}
\put(804,342.67){\rule{0.241pt}{0.400pt}}
\multiput(804.00,342.17)(0.500,1.000){2}{\rule{0.120pt}{0.400pt}}
\put(803.0,343.0){\usebox{\plotpoint}}
\put(805,345.67){\rule{0.241pt}{0.400pt}}
\multiput(805.00,345.17)(0.500,1.000){2}{\rule{0.120pt}{0.400pt}}
\put(805.0,344.0){\rule[-0.200pt]{0.400pt}{0.482pt}}
\put(806.0,347.0){\usebox{\plotpoint}}
\put(806.67,348){\rule{0.400pt}{0.482pt}}
\multiput(806.17,348.00)(1.000,1.000){2}{\rule{0.400pt}{0.241pt}}
\put(806.0,348.0){\usebox{\plotpoint}}
\put(808.0,350.0){\rule[-0.200pt]{0.400pt}{0.482pt}}
\put(808.67,352){\rule{0.400pt}{0.482pt}}
\multiput(808.17,352.00)(1.000,1.000){2}{\rule{0.400pt}{0.241pt}}
\put(808.0,352.0){\usebox{\plotpoint}}
\put(810,355.67){\rule{0.241pt}{0.400pt}}
\multiput(810.00,355.17)(0.500,1.000){2}{\rule{0.120pt}{0.400pt}}
\put(811,356.67){\rule{0.241pt}{0.400pt}}
\multiput(811.00,356.17)(0.500,1.000){2}{\rule{0.120pt}{0.400pt}}
\put(810.0,354.0){\rule[-0.200pt]{0.400pt}{0.482pt}}
\put(812,359.67){\rule{0.241pt}{0.400pt}}
\multiput(812.00,360.17)(0.500,-1.000){2}{\rule{0.120pt}{0.400pt}}
\put(812.0,358.0){\rule[-0.200pt]{0.400pt}{0.723pt}}
\put(812.67,362){\rule{0.400pt}{0.482pt}}
\multiput(812.17,362.00)(1.000,1.000){2}{\rule{0.400pt}{0.241pt}}
\put(813.67,364){\rule{0.400pt}{0.482pt}}
\multiput(813.17,364.00)(1.000,1.000){2}{\rule{0.400pt}{0.241pt}}
\put(813.0,360.0){\rule[-0.200pt]{0.400pt}{0.482pt}}
\put(814.67,367){\rule{0.400pt}{0.723pt}}
\multiput(814.17,367.00)(1.000,1.500){2}{\rule{0.400pt}{0.361pt}}
\put(816,368.67){\rule{0.241pt}{0.400pt}}
\multiput(816.00,369.17)(0.500,-1.000){2}{\rule{0.120pt}{0.400pt}}
\put(815.0,366.0){\usebox{\plotpoint}}
\put(817,370.67){\rule{0.241pt}{0.400pt}}
\multiput(817.00,370.17)(0.500,1.000){2}{\rule{0.120pt}{0.400pt}}
\put(817.0,369.0){\rule[-0.200pt]{0.400pt}{0.482pt}}
\put(818,374.67){\rule{0.241pt}{0.400pt}}
\multiput(818.00,374.17)(0.500,1.000){2}{\rule{0.120pt}{0.400pt}}
\put(819,375.67){\rule{0.241pt}{0.400pt}}
\multiput(819.00,375.17)(0.500,1.000){2}{\rule{0.120pt}{0.400pt}}
\put(818.0,372.0){\rule[-0.200pt]{0.400pt}{0.723pt}}
\put(819.67,379){\rule{0.400pt}{0.723pt}}
\multiput(819.17,379.00)(1.000,1.500){2}{\rule{0.400pt}{0.361pt}}
\put(821,380.67){\rule{0.241pt}{0.400pt}}
\multiput(821.00,381.17)(0.500,-1.000){2}{\rule{0.120pt}{0.400pt}}
\put(820.0,377.0){\rule[-0.200pt]{0.400pt}{0.482pt}}
\put(821.67,385){\rule{0.400pt}{0.482pt}}
\multiput(821.17,385.00)(1.000,1.000){2}{\rule{0.400pt}{0.241pt}}
\put(822.67,387){\rule{0.400pt}{0.723pt}}
\multiput(822.17,387.00)(1.000,1.500){2}{\rule{0.400pt}{0.361pt}}
\put(822.0,381.0){\rule[-0.200pt]{0.400pt}{0.964pt}}
\put(824,388.67){\rule{0.241pt}{0.400pt}}
\multiput(824.00,388.17)(0.500,1.000){2}{\rule{0.120pt}{0.400pt}}
\put(824.0,389.0){\usebox{\plotpoint}}
\put(824.67,393){\rule{0.400pt}{0.482pt}}
\multiput(824.17,393.00)(1.000,1.000){2}{\rule{0.400pt}{0.241pt}}
\put(825.67,395){\rule{0.400pt}{0.723pt}}
\multiput(825.17,395.00)(1.000,1.500){2}{\rule{0.400pt}{0.361pt}}
\put(825.0,390.0){\rule[-0.200pt]{0.400pt}{0.723pt}}
\put(827,398){\usebox{\plotpoint}}
\put(826.67,398){\rule{0.400pt}{0.482pt}}
\multiput(826.17,398.00)(1.000,1.000){2}{\rule{0.400pt}{0.241pt}}
\put(827.67,400){\rule{0.400pt}{1.204pt}}
\multiput(827.17,400.00)(1.000,2.500){2}{\rule{0.400pt}{0.602pt}}
\put(829,405){\usebox{\plotpoint}}
\put(828.67,405){\rule{0.400pt}{0.482pt}}
\multiput(828.17,405.00)(1.000,1.000){2}{\rule{0.400pt}{0.241pt}}
\put(829.67,408){\rule{0.400pt}{0.723pt}}
\multiput(829.17,408.00)(1.000,1.500){2}{\rule{0.400pt}{0.361pt}}
\put(830.67,411){\rule{0.400pt}{0.482pt}}
\multiput(830.17,411.00)(1.000,1.000){2}{\rule{0.400pt}{0.241pt}}
\put(830.0,407.0){\usebox{\plotpoint}}
\put(831.67,414){\rule{0.400pt}{0.482pt}}
\multiput(831.17,414.00)(1.000,1.000){2}{\rule{0.400pt}{0.241pt}}
\put(832.67,416){\rule{0.400pt}{0.964pt}}
\multiput(832.17,416.00)(1.000,2.000){2}{\rule{0.400pt}{0.482pt}}
\put(832.0,413.0){\usebox{\plotpoint}}
\put(834,421.67){\rule{0.241pt}{0.400pt}}
\multiput(834.00,421.17)(0.500,1.000){2}{\rule{0.120pt}{0.400pt}}
\put(834.67,423){\rule{0.400pt}{0.964pt}}
\multiput(834.17,423.00)(1.000,2.000){2}{\rule{0.400pt}{0.482pt}}
\put(834.0,420.0){\rule[-0.200pt]{0.400pt}{0.482pt}}
\put(836,428.67){\rule{0.241pt}{0.400pt}}
\multiput(836.00,428.17)(0.500,1.000){2}{\rule{0.120pt}{0.400pt}}
\put(836.0,427.0){\rule[-0.200pt]{0.400pt}{0.482pt}}
\put(837,432.67){\rule{0.241pt}{0.400pt}}
\multiput(837.00,432.17)(0.500,1.000){2}{\rule{0.120pt}{0.400pt}}
\put(837.67,434){\rule{0.400pt}{0.723pt}}
\multiput(837.17,434.00)(1.000,1.500){2}{\rule{0.400pt}{0.361pt}}
\put(837.0,430.0){\rule[-0.200pt]{0.400pt}{0.723pt}}
\put(838.67,438){\rule{0.400pt}{0.723pt}}
\multiput(838.17,438.00)(1.000,1.500){2}{\rule{0.400pt}{0.361pt}}
\put(839.67,441){\rule{0.400pt}{0.482pt}}
\multiput(839.17,441.00)(1.000,1.000){2}{\rule{0.400pt}{0.241pt}}
\put(839.0,437.0){\usebox{\plotpoint}}
\put(840.67,445){\rule{0.400pt}{0.482pt}}
\multiput(840.17,445.00)(1.000,1.000){2}{\rule{0.400pt}{0.241pt}}
\put(841.0,443.0){\rule[-0.200pt]{0.400pt}{0.482pt}}
\put(841.67,451){\rule{0.400pt}{0.964pt}}
\multiput(841.17,451.00)(1.000,2.000){2}{\rule{0.400pt}{0.482pt}}
\put(842.67,455){\rule{0.400pt}{0.482pt}}
\multiput(842.17,455.00)(1.000,1.000){2}{\rule{0.400pt}{0.241pt}}
\put(842.0,447.0){\rule[-0.200pt]{0.400pt}{0.964pt}}
\put(844.0,457.0){\rule[-0.200pt]{0.400pt}{0.723pt}}
\put(845,459.67){\rule{0.241pt}{0.400pt}}
\multiput(845.00,459.17)(0.500,1.000){2}{\rule{0.120pt}{0.400pt}}
\put(844.0,460.0){\usebox{\plotpoint}}
\put(846,465.67){\rule{0.241pt}{0.400pt}}
\multiput(846.00,465.17)(0.500,1.000){2}{\rule{0.120pt}{0.400pt}}
\put(846.67,467){\rule{0.400pt}{1.445pt}}
\multiput(846.17,467.00)(1.000,3.000){2}{\rule{0.400pt}{0.723pt}}
\put(846.0,461.0){\rule[-0.200pt]{0.400pt}{1.204pt}}
\put(847.67,472){\rule{0.400pt}{0.964pt}}
\multiput(847.17,472.00)(1.000,2.000){2}{\rule{0.400pt}{0.482pt}}
\put(848.0,472.0){\usebox{\plotpoint}}
\put(848.67,477){\rule{0.400pt}{0.964pt}}
\multiput(848.17,477.00)(1.000,2.000){2}{\rule{0.400pt}{0.482pt}}
\put(849.67,481){\rule{0.400pt}{0.723pt}}
\multiput(849.17,481.00)(1.000,1.500){2}{\rule{0.400pt}{0.361pt}}
\put(849.0,476.0){\usebox{\plotpoint}}
\put(851.0,484.0){\rule[-0.200pt]{0.400pt}{0.482pt}}
\put(851.67,486){\rule{0.400pt}{0.482pt}}
\multiput(851.17,486.00)(1.000,1.000){2}{\rule{0.400pt}{0.241pt}}
\put(851.0,486.0){\usebox{\plotpoint}}
\put(852.67,492){\rule{0.400pt}{0.723pt}}
\multiput(852.17,492.00)(1.000,1.500){2}{\rule{0.400pt}{0.361pt}}
\put(853.0,488.0){\rule[-0.200pt]{0.400pt}{0.964pt}}
\put(853.67,498){\rule{0.400pt}{0.482pt}}
\multiput(853.17,498.00)(1.000,1.000){2}{\rule{0.400pt}{0.241pt}}
\put(855,499.67){\rule{0.241pt}{0.400pt}}
\multiput(855.00,499.17)(0.500,1.000){2}{\rule{0.120pt}{0.400pt}}
\put(854.0,495.0){\rule[-0.200pt]{0.400pt}{0.723pt}}
\put(855.67,504){\rule{0.400pt}{1.204pt}}
\multiput(855.17,504.00)(1.000,2.500){2}{\rule{0.400pt}{0.602pt}}
\put(856.67,507){\rule{0.400pt}{0.482pt}}
\multiput(856.17,508.00)(1.000,-1.000){2}{\rule{0.400pt}{0.241pt}}
\put(856.0,501.0){\rule[-0.200pt]{0.400pt}{0.723pt}}
\put(858,513.67){\rule{0.241pt}{0.400pt}}
\multiput(858.00,513.17)(0.500,1.000){2}{\rule{0.120pt}{0.400pt}}
\put(859,514.67){\rule{0.241pt}{0.400pt}}
\multiput(859.00,514.17)(0.500,1.000){2}{\rule{0.120pt}{0.400pt}}
\put(858.0,507.0){\rule[-0.200pt]{0.400pt}{1.686pt}}
\put(860,520.67){\rule{0.241pt}{0.400pt}}
\multiput(860.00,521.17)(0.500,-1.000){2}{\rule{0.120pt}{0.400pt}}
\put(860.0,516.0){\rule[-0.200pt]{0.400pt}{1.445pt}}
\put(860.67,523){\rule{0.400pt}{0.964pt}}
\multiput(860.17,523.00)(1.000,2.000){2}{\rule{0.400pt}{0.482pt}}
\put(861.67,527){\rule{0.400pt}{0.964pt}}
\multiput(861.17,527.00)(1.000,2.000){2}{\rule{0.400pt}{0.482pt}}
\put(861.0,521.0){\rule[-0.200pt]{0.400pt}{0.482pt}}
\put(863.0,531.0){\rule[-0.200pt]{0.400pt}{0.482pt}}
\put(863.67,533){\rule{0.400pt}{0.964pt}}
\multiput(863.17,533.00)(1.000,2.000){2}{\rule{0.400pt}{0.482pt}}
\put(863.0,533.0){\usebox{\plotpoint}}
\put(864.67,540){\rule{0.400pt}{0.482pt}}
\multiput(864.17,540.00)(1.000,1.000){2}{\rule{0.400pt}{0.241pt}}
\put(865.0,537.0){\rule[-0.200pt]{0.400pt}{0.723pt}}
\put(865.67,544){\rule{0.400pt}{0.482pt}}
\multiput(865.17,544.00)(1.000,1.000){2}{\rule{0.400pt}{0.241pt}}
\put(866.67,546){\rule{0.400pt}{1.204pt}}
\multiput(866.17,546.00)(1.000,2.500){2}{\rule{0.400pt}{0.602pt}}
\put(866.0,542.0){\rule[-0.200pt]{0.400pt}{0.482pt}}
\put(868,551){\usebox{\plotpoint}}
\put(867.67,551){\rule{0.400pt}{0.964pt}}
\multiput(867.17,551.00)(1.000,2.000){2}{\rule{0.400pt}{0.482pt}}
\put(868.67,555){\rule{0.400pt}{0.482pt}}
\multiput(868.17,555.00)(1.000,1.000){2}{\rule{0.400pt}{0.241pt}}
\put(870,558.67){\rule{0.241pt}{0.400pt}}
\multiput(870.00,559.17)(0.500,-1.000){2}{\rule{0.120pt}{0.400pt}}
\put(870.0,557.0){\rule[-0.200pt]{0.400pt}{0.723pt}}
\put(870.67,562){\rule{0.400pt}{0.482pt}}
\multiput(870.17,563.00)(1.000,-1.000){2}{\rule{0.400pt}{0.241pt}}
\put(871.67,562){\rule{0.400pt}{1.204pt}}
\multiput(871.17,562.00)(1.000,2.500){2}{\rule{0.400pt}{0.602pt}}
\put(871.0,559.0){\rule[-0.200pt]{0.400pt}{1.204pt}}
\put(873.0,567.0){\rule[-0.200pt]{0.400pt}{0.482pt}}
\put(874,567.67){\rule{0.241pt}{0.400pt}}
\multiput(874.00,568.17)(0.500,-1.000){2}{\rule{0.120pt}{0.400pt}}
\put(873.0,569.0){\usebox{\plotpoint}}
\put(875,574.67){\rule{0.241pt}{0.400pt}}
\multiput(875.00,574.17)(0.500,1.000){2}{\rule{0.120pt}{0.400pt}}
\put(875.0,568.0){\rule[-0.200pt]{0.400pt}{1.686pt}}
\put(876.0,576.0){\usebox{\plotpoint}}
\put(876.67,582){\rule{0.400pt}{0.482pt}}
\multiput(876.17,582.00)(1.000,1.000){2}{\rule{0.400pt}{0.241pt}}
\put(877.0,576.0){\rule[-0.200pt]{0.400pt}{1.445pt}}
\put(878,584.67){\rule{0.241pt}{0.400pt}}
\multiput(878.00,585.17)(0.500,-1.000){2}{\rule{0.120pt}{0.400pt}}
\put(878.67,585){\rule{0.400pt}{1.204pt}}
\multiput(878.17,585.00)(1.000,2.500){2}{\rule{0.400pt}{0.602pt}}
\put(878.0,584.0){\rule[-0.200pt]{0.400pt}{0.482pt}}
\put(879.67,588){\rule{0.400pt}{1.445pt}}
\multiput(879.17,588.00)(1.000,3.000){2}{\rule{0.400pt}{0.723pt}}
\put(880.0,588.0){\rule[-0.200pt]{0.400pt}{0.482pt}}
\put(881.0,594.0){\rule[-0.200pt]{0.482pt}{0.400pt}}
\put(882.67,597){\rule{0.400pt}{0.723pt}}
\multiput(882.17,597.00)(1.000,1.500){2}{\rule{0.400pt}{0.361pt}}
\put(884,598.67){\rule{0.241pt}{0.400pt}}
\multiput(884.00,599.17)(0.500,-1.000){2}{\rule{0.120pt}{0.400pt}}
\put(883.0,594.0){\rule[-0.200pt]{0.400pt}{0.723pt}}
\put(884.67,603){\rule{0.400pt}{0.964pt}}
\multiput(884.17,605.00)(1.000,-2.000){2}{\rule{0.400pt}{0.482pt}}
\put(885.67,603){\rule{0.400pt}{1.204pt}}
\multiput(885.17,603.00)(1.000,2.500){2}{\rule{0.400pt}{0.602pt}}
\put(885.0,599.0){\rule[-0.200pt]{0.400pt}{1.927pt}}
\put(886.67,606){\rule{0.400pt}{0.964pt}}
\multiput(886.17,608.00)(1.000,-2.000){2}{\rule{0.400pt}{0.482pt}}
\put(887.0,608.0){\rule[-0.200pt]{0.400pt}{0.482pt}}
\put(888,606){\usebox{\plotpoint}}
\put(887.67,606){\rule{0.400pt}{1.204pt}}
\multiput(887.17,606.00)(1.000,2.500){2}{\rule{0.400pt}{0.602pt}}
\put(888.67,611){\rule{0.400pt}{0.482pt}}
\multiput(888.17,611.00)(1.000,1.000){2}{\rule{0.400pt}{0.241pt}}
\put(890.0,613.0){\usebox{\plotpoint}}
\put(891.67,614){\rule{0.400pt}{0.723pt}}
\multiput(891.17,614.00)(1.000,1.500){2}{\rule{0.400pt}{0.361pt}}
\put(890.0,614.0){\rule[-0.200pt]{0.482pt}{0.400pt}}
\put(892.67,614){\rule{0.400pt}{1.927pt}}
\multiput(892.17,614.00)(1.000,4.000){2}{\rule{0.400pt}{0.964pt}}
\put(893.67,616){\rule{0.400pt}{1.445pt}}
\multiput(893.17,619.00)(1.000,-3.000){2}{\rule{0.400pt}{0.723pt}}
\put(893.0,614.0){\rule[-0.200pt]{0.400pt}{0.723pt}}
\put(894.67,618){\rule{0.400pt}{0.482pt}}
\multiput(894.17,618.00)(1.000,1.000){2}{\rule{0.400pt}{0.241pt}}
\put(896,618.67){\rule{0.241pt}{0.400pt}}
\multiput(896.00,619.17)(0.500,-1.000){2}{\rule{0.120pt}{0.400pt}}
\put(895.0,616.0){\rule[-0.200pt]{0.400pt}{0.482pt}}
\put(896.67,622){\rule{0.400pt}{0.482pt}}
\multiput(896.17,622.00)(1.000,1.000){2}{\rule{0.400pt}{0.241pt}}
\put(897.67,622){\rule{0.400pt}{0.482pt}}
\multiput(897.17,623.00)(1.000,-1.000){2}{\rule{0.400pt}{0.241pt}}
\put(897.0,619.0){\rule[-0.200pt]{0.400pt}{0.723pt}}
\put(898.67,617){\rule{0.400pt}{0.723pt}}
\multiput(898.17,618.50)(1.000,-1.500){2}{\rule{0.400pt}{0.361pt}}
\put(899.0,620.0){\rule[-0.200pt]{0.400pt}{0.482pt}}
\put(899.67,619){\rule{0.400pt}{0.723pt}}
\multiput(899.17,620.50)(1.000,-1.500){2}{\rule{0.400pt}{0.361pt}}
\put(900.67,614){\rule{0.400pt}{1.204pt}}
\multiput(900.17,616.50)(1.000,-2.500){2}{\rule{0.400pt}{0.602pt}}
\put(900.0,617.0){\rule[-0.200pt]{0.400pt}{1.204pt}}
\put(901.67,618){\rule{0.400pt}{0.482pt}}
\multiput(901.17,619.00)(1.000,-1.000){2}{\rule{0.400pt}{0.241pt}}
\put(902.67,615){\rule{0.400pt}{0.723pt}}
\multiput(902.17,616.50)(1.000,-1.500){2}{\rule{0.400pt}{0.361pt}}
\put(902.0,614.0){\rule[-0.200pt]{0.400pt}{1.445pt}}
\put(904.0,615.0){\rule[-0.200pt]{0.400pt}{0.723pt}}
\put(904.67,615){\rule{0.400pt}{0.723pt}}
\multiput(904.17,616.50)(1.000,-1.500){2}{\rule{0.400pt}{0.361pt}}
\put(906,613.67){\rule{0.241pt}{0.400pt}}
\multiput(906.00,614.17)(0.500,-1.000){2}{\rule{0.120pt}{0.400pt}}
\put(904.0,618.0){\usebox{\plotpoint}}
\put(907.0,613.0){\usebox{\plotpoint}}
\put(907.67,613){\rule{0.400pt}{0.482pt}}
\multiput(907.17,613.00)(1.000,1.000){2}{\rule{0.400pt}{0.241pt}}
\put(907.0,613.0){\usebox{\plotpoint}}
\put(908.67,610){\rule{0.400pt}{0.482pt}}
\multiput(908.17,610.00)(1.000,1.000){2}{\rule{0.400pt}{0.241pt}}
\put(909.0,610.0){\rule[-0.200pt]{0.400pt}{1.204pt}}
\put(910,612){\usebox{\plotpoint}}
\put(909.67,609){\rule{0.400pt}{0.723pt}}
\multiput(909.17,610.50)(1.000,-1.500){2}{\rule{0.400pt}{0.361pt}}
\put(910.67,606){\rule{0.400pt}{0.723pt}}
\multiput(910.17,607.50)(1.000,-1.500){2}{\rule{0.400pt}{0.361pt}}
\put(911.67,603){\rule{0.400pt}{0.482pt}}
\multiput(911.17,603.00)(1.000,1.000){2}{\rule{0.400pt}{0.241pt}}
\put(912.67,602){\rule{0.400pt}{0.723pt}}
\multiput(912.17,603.50)(1.000,-1.500){2}{\rule{0.400pt}{0.361pt}}
\put(912.0,603.0){\rule[-0.200pt]{0.400pt}{0.723pt}}
\put(913.67,597){\rule{0.400pt}{1.445pt}}
\multiput(913.17,600.00)(1.000,-3.000){2}{\rule{0.400pt}{0.723pt}}
\put(914.0,602.0){\usebox{\plotpoint}}
\put(915,597){\usebox{\plotpoint}}
\put(916,596.67){\rule{0.241pt}{0.400pt}}
\multiput(916.00,596.17)(0.500,1.000){2}{\rule{0.120pt}{0.400pt}}
\put(915.0,597.0){\usebox{\plotpoint}}
\put(917.0,593.0){\rule[-0.200pt]{0.400pt}{1.204pt}}
\put(917.67,587){\rule{0.400pt}{1.445pt}}
\multiput(917.17,590.00)(1.000,-3.000){2}{\rule{0.400pt}{0.723pt}}
\put(917.0,593.0){\usebox{\plotpoint}}
\put(919,586.67){\rule{0.241pt}{0.400pt}}
\multiput(919.00,587.17)(0.500,-1.000){2}{\rule{0.120pt}{0.400pt}}
\put(919.0,587.0){\usebox{\plotpoint}}
\put(919.67,580){\rule{0.400pt}{0.964pt}}
\multiput(919.17,582.00)(1.000,-2.000){2}{\rule{0.400pt}{0.482pt}}
\put(920.0,584.0){\rule[-0.200pt]{0.400pt}{0.723pt}}
\put(921.0,580.0){\usebox{\plotpoint}}
\put(921.67,572){\rule{0.400pt}{1.204pt}}
\multiput(921.17,574.50)(1.000,-2.500){2}{\rule{0.400pt}{0.602pt}}
\put(922.67,572){\rule{0.400pt}{0.482pt}}
\multiput(922.17,572.00)(1.000,1.000){2}{\rule{0.400pt}{0.241pt}}
\put(922.0,577.0){\rule[-0.200pt]{0.400pt}{0.723pt}}
\put(924,567.67){\rule{0.241pt}{0.400pt}}
\multiput(924.00,568.17)(0.500,-1.000){2}{\rule{0.120pt}{0.400pt}}
\put(924.0,569.0){\rule[-0.200pt]{0.400pt}{1.204pt}}
\put(925,568){\usebox{\plotpoint}}
\put(924.67,561){\rule{0.400pt}{1.686pt}}
\multiput(924.17,564.50)(1.000,-3.500){2}{\rule{0.400pt}{0.843pt}}
\put(926,559.67){\rule{0.241pt}{0.400pt}}
\multiput(926.00,560.17)(0.500,-1.000){2}{\rule{0.120pt}{0.400pt}}
\put(926.67,555){\rule{0.400pt}{0.482pt}}
\multiput(926.17,556.00)(1.000,-1.000){2}{\rule{0.400pt}{0.241pt}}
\put(927.67,550){\rule{0.400pt}{1.204pt}}
\multiput(927.17,552.50)(1.000,-2.500){2}{\rule{0.400pt}{0.602pt}}
\put(927.0,557.0){\rule[-0.200pt]{0.400pt}{0.723pt}}
\put(928.67,544){\rule{0.400pt}{0.482pt}}
\multiput(928.17,544.00)(1.000,1.000){2}{\rule{0.400pt}{0.241pt}}
\put(929.0,544.0){\rule[-0.200pt]{0.400pt}{1.445pt}}
\put(929.67,539){\rule{0.400pt}{0.723pt}}
\multiput(929.17,540.50)(1.000,-1.500){2}{\rule{0.400pt}{0.361pt}}
\put(930.67,536){\rule{0.400pt}{0.723pt}}
\multiput(930.17,537.50)(1.000,-1.500){2}{\rule{0.400pt}{0.361pt}}
\put(930.0,542.0){\rule[-0.200pt]{0.400pt}{0.964pt}}
\put(932.0,532.0){\rule[-0.200pt]{0.400pt}{0.964pt}}
\put(932.67,527){\rule{0.400pt}{1.204pt}}
\multiput(932.17,529.50)(1.000,-2.500){2}{\rule{0.400pt}{0.602pt}}
\put(932.0,532.0){\usebox{\plotpoint}}
\put(934,520.67){\rule{0.241pt}{0.400pt}}
\multiput(934.00,520.17)(0.500,1.000){2}{\rule{0.120pt}{0.400pt}}
\put(934.0,521.0){\rule[-0.200pt]{0.400pt}{1.445pt}}
\put(935.0,515.0){\rule[-0.200pt]{0.400pt}{1.686pt}}
\put(935.67,510){\rule{0.400pt}{1.204pt}}
\multiput(935.17,512.50)(1.000,-2.500){2}{\rule{0.400pt}{0.602pt}}
\put(935.0,515.0){\usebox{\plotpoint}}
\put(936.67,505){\rule{0.400pt}{0.723pt}}
\multiput(936.17,506.50)(1.000,-1.500){2}{\rule{0.400pt}{0.361pt}}
\put(937.67,500){\rule{0.400pt}{1.204pt}}
\multiput(937.17,502.50)(1.000,-2.500){2}{\rule{0.400pt}{0.602pt}}
\put(937.0,508.0){\rule[-0.200pt]{0.400pt}{0.482pt}}
\put(938.67,493){\rule{0.400pt}{0.723pt}}
\multiput(938.17,494.50)(1.000,-1.500){2}{\rule{0.400pt}{0.361pt}}
\put(939.0,496.0){\rule[-0.200pt]{0.400pt}{0.964pt}}
\put(940,493){\usebox{\plotpoint}}
\put(939.67,490){\rule{0.400pt}{0.723pt}}
\multiput(939.17,491.50)(1.000,-1.500){2}{\rule{0.400pt}{0.361pt}}
\put(940.67,484){\rule{0.400pt}{1.445pt}}
\multiput(940.17,487.00)(1.000,-3.000){2}{\rule{0.400pt}{0.723pt}}
\put(941.67,476){\rule{0.400pt}{0.964pt}}
\multiput(941.17,478.00)(1.000,-2.000){2}{\rule{0.400pt}{0.482pt}}
\put(942.67,472){\rule{0.400pt}{0.964pt}}
\multiput(942.17,474.00)(1.000,-2.000){2}{\rule{0.400pt}{0.482pt}}
\put(942.0,480.0){\rule[-0.200pt]{0.400pt}{0.964pt}}
\put(943.67,467){\rule{0.400pt}{0.723pt}}
\multiput(943.17,468.50)(1.000,-1.500){2}{\rule{0.400pt}{0.361pt}}
\put(944.0,470.0){\rule[-0.200pt]{0.400pt}{0.482pt}}
\put(944.67,460){\rule{0.400pt}{0.964pt}}
\multiput(944.17,462.00)(1.000,-2.000){2}{\rule{0.400pt}{0.482pt}}
\put(945.67,457){\rule{0.400pt}{0.723pt}}
\multiput(945.17,458.50)(1.000,-1.500){2}{\rule{0.400pt}{0.361pt}}
\put(945.0,464.0){\rule[-0.200pt]{0.400pt}{0.723pt}}
\put(946.67,450){\rule{0.400pt}{0.482pt}}
\multiput(946.17,451.00)(1.000,-1.000){2}{\rule{0.400pt}{0.241pt}}
\put(947.67,445){\rule{0.400pt}{1.204pt}}
\multiput(947.17,447.50)(1.000,-2.500){2}{\rule{0.400pt}{0.602pt}}
\put(947.0,452.0){\rule[-0.200pt]{0.400pt}{1.204pt}}
\put(949.0,441.0){\rule[-0.200pt]{0.400pt}{0.964pt}}
\put(949.0,441.0){\usebox{\plotpoint}}
\put(949.67,433){\rule{0.400pt}{0.482pt}}
\multiput(949.17,434.00)(1.000,-1.000){2}{\rule{0.400pt}{0.241pt}}
\put(950.67,429){\rule{0.400pt}{0.964pt}}
\multiput(950.17,431.00)(1.000,-2.000){2}{\rule{0.400pt}{0.482pt}}
\put(950.0,435.0){\rule[-0.200pt]{0.400pt}{1.445pt}}
\put(951.67,422){\rule{0.400pt}{0.723pt}}
\multiput(951.17,423.50)(1.000,-1.500){2}{\rule{0.400pt}{0.361pt}}
\put(952.67,417){\rule{0.400pt}{1.204pt}}
\multiput(952.17,419.50)(1.000,-2.500){2}{\rule{0.400pt}{0.602pt}}
\put(952.0,425.0){\rule[-0.200pt]{0.400pt}{0.964pt}}
\put(953.67,411){\rule{0.400pt}{0.964pt}}
\multiput(953.17,413.00)(1.000,-2.000){2}{\rule{0.400pt}{0.482pt}}
\put(954.0,415.0){\rule[-0.200pt]{0.400pt}{0.482pt}}
\put(954.67,403){\rule{0.400pt}{1.445pt}}
\multiput(954.17,406.00)(1.000,-3.000){2}{\rule{0.400pt}{0.723pt}}
\put(955.67,400){\rule{0.400pt}{0.723pt}}
\multiput(955.17,401.50)(1.000,-1.500){2}{\rule{0.400pt}{0.361pt}}
\put(955.0,409.0){\rule[-0.200pt]{0.400pt}{0.482pt}}
\put(957,394.67){\rule{0.241pt}{0.400pt}}
\multiput(957.00,395.17)(0.500,-1.000){2}{\rule{0.120pt}{0.400pt}}
\put(957.67,388){\rule{0.400pt}{1.686pt}}
\multiput(957.17,391.50)(1.000,-3.500){2}{\rule{0.400pt}{0.843pt}}
\put(957.0,396.0){\rule[-0.200pt]{0.400pt}{0.964pt}}
\put(958.67,383){\rule{0.400pt}{0.964pt}}
\multiput(958.17,385.00)(1.000,-2.000){2}{\rule{0.400pt}{0.482pt}}
\put(959.0,387.0){\usebox{\plotpoint}}
\put(959.67,376){\rule{0.400pt}{0.723pt}}
\multiput(959.17,377.50)(1.000,-1.500){2}{\rule{0.400pt}{0.361pt}}
\put(960.67,373){\rule{0.400pt}{0.723pt}}
\multiput(960.17,374.50)(1.000,-1.500){2}{\rule{0.400pt}{0.361pt}}
\put(960.0,379.0){\rule[-0.200pt]{0.400pt}{0.964pt}}
\put(961.67,367){\rule{0.400pt}{0.482pt}}
\multiput(961.17,368.00)(1.000,-1.000){2}{\rule{0.400pt}{0.241pt}}
\put(962.67,363){\rule{0.400pt}{0.964pt}}
\multiput(962.17,365.00)(1.000,-2.000){2}{\rule{0.400pt}{0.482pt}}
\put(962.0,369.0){\rule[-0.200pt]{0.400pt}{0.964pt}}
\put(963.67,358){\rule{0.400pt}{0.723pt}}
\multiput(963.17,359.50)(1.000,-1.500){2}{\rule{0.400pt}{0.361pt}}
\put(964.0,361.0){\rule[-0.200pt]{0.400pt}{0.482pt}}
\put(964.67,350){\rule{0.400pt}{0.723pt}}
\multiput(964.17,351.50)(1.000,-1.500){2}{\rule{0.400pt}{0.361pt}}
\put(965.67,346){\rule{0.400pt}{0.964pt}}
\multiput(965.17,348.00)(1.000,-2.000){2}{\rule{0.400pt}{0.482pt}}
\put(965.0,353.0){\rule[-0.200pt]{0.400pt}{1.204pt}}
\put(967,340.67){\rule{0.241pt}{0.400pt}}
\multiput(967.00,341.17)(0.500,-1.000){2}{\rule{0.120pt}{0.400pt}}
\put(967.0,342.0){\rule[-0.200pt]{0.400pt}{0.964pt}}
\put(967.67,334){\rule{0.400pt}{0.964pt}}
\multiput(967.17,336.00)(1.000,-2.000){2}{\rule{0.400pt}{0.482pt}}
\put(968.67,331){\rule{0.400pt}{0.723pt}}
\multiput(968.17,332.50)(1.000,-1.500){2}{\rule{0.400pt}{0.361pt}}
\put(968.0,338.0){\rule[-0.200pt]{0.400pt}{0.723pt}}
\put(969.67,325){\rule{0.400pt}{0.964pt}}
\multiput(969.17,327.00)(1.000,-2.000){2}{\rule{0.400pt}{0.482pt}}
\put(970.67,322){\rule{0.400pt}{0.723pt}}
\multiput(970.17,323.50)(1.000,-1.500){2}{\rule{0.400pt}{0.361pt}}
\put(970.0,329.0){\rule[-0.200pt]{0.400pt}{0.482pt}}
\put(971.67,317){\rule{0.400pt}{0.723pt}}
\multiput(971.17,318.50)(1.000,-1.500){2}{\rule{0.400pt}{0.361pt}}
\put(972.0,320.0){\rule[-0.200pt]{0.400pt}{0.482pt}}
\put(972.67,311){\rule{0.400pt}{0.482pt}}
\multiput(972.17,312.00)(1.000,-1.000){2}{\rule{0.400pt}{0.241pt}}
\put(973.67,309){\rule{0.400pt}{0.482pt}}
\multiput(973.17,310.00)(1.000,-1.000){2}{\rule{0.400pt}{0.241pt}}
\put(973.0,313.0){\rule[-0.200pt]{0.400pt}{0.964pt}}
\put(974.67,303){\rule{0.400pt}{0.482pt}}
\multiput(974.17,304.00)(1.000,-1.000){2}{\rule{0.400pt}{0.241pt}}
\put(975.67,300){\rule{0.400pt}{0.723pt}}
\multiput(975.17,301.50)(1.000,-1.500){2}{\rule{0.400pt}{0.361pt}}
\put(975.0,305.0){\rule[-0.200pt]{0.400pt}{0.964pt}}
\put(976.67,295){\rule{0.400pt}{0.723pt}}
\multiput(976.17,296.50)(1.000,-1.500){2}{\rule{0.400pt}{0.361pt}}
\put(977.0,298.0){\rule[-0.200pt]{0.400pt}{0.482pt}}
\put(977.67,289){\rule{0.400pt}{0.482pt}}
\multiput(977.17,290.00)(1.000,-1.000){2}{\rule{0.400pt}{0.241pt}}
\put(978.67,286){\rule{0.400pt}{0.723pt}}
\multiput(978.17,287.50)(1.000,-1.500){2}{\rule{0.400pt}{0.361pt}}
\put(978.0,291.0){\rule[-0.200pt]{0.400pt}{0.964pt}}
\put(979.67,282){\rule{0.400pt}{0.482pt}}
\multiput(979.17,283.00)(1.000,-1.000){2}{\rule{0.400pt}{0.241pt}}
\put(980.67,279){\rule{0.400pt}{0.723pt}}
\multiput(980.17,280.50)(1.000,-1.500){2}{\rule{0.400pt}{0.361pt}}
\put(980.0,284.0){\rule[-0.200pt]{0.400pt}{0.482pt}}
\put(981.67,274){\rule{0.400pt}{0.723pt}}
\multiput(981.17,275.50)(1.000,-1.500){2}{\rule{0.400pt}{0.361pt}}
\put(982.0,277.0){\rule[-0.200pt]{0.400pt}{0.482pt}}
\put(982.67,269){\rule{0.400pt}{0.964pt}}
\multiput(982.17,271.00)(1.000,-2.000){2}{\rule{0.400pt}{0.482pt}}
\put(983.67,267){\rule{0.400pt}{0.482pt}}
\multiput(983.17,268.00)(1.000,-1.000){2}{\rule{0.400pt}{0.241pt}}
\put(983.0,273.0){\usebox{\plotpoint}}
\put(984.67,262){\rule{0.400pt}{0.482pt}}
\multiput(984.17,263.00)(1.000,-1.000){2}{\rule{0.400pt}{0.241pt}}
\put(985.67,260){\rule{0.400pt}{0.482pt}}
\multiput(985.17,261.00)(1.000,-1.000){2}{\rule{0.400pt}{0.241pt}}
\put(985.0,264.0){\rule[-0.200pt]{0.400pt}{0.723pt}}
\put(986.67,256){\rule{0.400pt}{0.723pt}}
\multiput(986.17,257.50)(1.000,-1.500){2}{\rule{0.400pt}{0.361pt}}
\put(987.0,259.0){\usebox{\plotpoint}}
\put(987.67,251){\rule{0.400pt}{0.482pt}}
\multiput(987.17,252.00)(1.000,-1.000){2}{\rule{0.400pt}{0.241pt}}
\put(989,249.67){\rule{0.241pt}{0.400pt}}
\multiput(989.00,250.17)(0.500,-1.000){2}{\rule{0.120pt}{0.400pt}}
\put(988.0,253.0){\rule[-0.200pt]{0.400pt}{0.723pt}}
\put(989.67,246){\rule{0.400pt}{0.482pt}}
\multiput(989.17,247.00)(1.000,-1.000){2}{\rule{0.400pt}{0.241pt}}
\put(990.0,248.0){\rule[-0.200pt]{0.400pt}{0.482pt}}
\put(990.67,241){\rule{0.400pt}{0.482pt}}
\multiput(990.17,242.00)(1.000,-1.000){2}{\rule{0.400pt}{0.241pt}}
\put(991.67,239){\rule{0.400pt}{0.482pt}}
\multiput(991.17,240.00)(1.000,-1.000){2}{\rule{0.400pt}{0.241pt}}
\put(991.0,243.0){\rule[-0.200pt]{0.400pt}{0.723pt}}
\put(992.67,236){\rule{0.400pt}{0.482pt}}
\multiput(992.17,237.00)(1.000,-1.000){2}{\rule{0.400pt}{0.241pt}}
\put(993.67,233){\rule{0.400pt}{0.723pt}}
\multiput(993.17,234.50)(1.000,-1.500){2}{\rule{0.400pt}{0.361pt}}
\put(993.0,238.0){\usebox{\plotpoint}}
\put(994.67,230){\rule{0.400pt}{0.482pt}}
\multiput(994.17,231.00)(1.000,-1.000){2}{\rule{0.400pt}{0.241pt}}
\put(995.0,232.0){\usebox{\plotpoint}}
\put(996,226.67){\rule{0.241pt}{0.400pt}}
\multiput(996.00,227.17)(0.500,-1.000){2}{\rule{0.120pt}{0.400pt}}
\put(996.67,225){\rule{0.400pt}{0.482pt}}
\multiput(996.17,226.00)(1.000,-1.000){2}{\rule{0.400pt}{0.241pt}}
\put(996.0,228.0){\rule[-0.200pt]{0.400pt}{0.482pt}}
\put(998,220.67){\rule{0.241pt}{0.400pt}}
\multiput(998.00,221.17)(0.500,-1.000){2}{\rule{0.120pt}{0.400pt}}
\put(998.67,219){\rule{0.400pt}{0.482pt}}
\multiput(998.17,220.00)(1.000,-1.000){2}{\rule{0.400pt}{0.241pt}}
\put(998.0,222.0){\rule[-0.200pt]{0.400pt}{0.723pt}}
\put(999.67,216){\rule{0.400pt}{0.482pt}}
\multiput(999.17,217.00)(1.000,-1.000){2}{\rule{0.400pt}{0.241pt}}
\put(1000.0,218.0){\usebox{\plotpoint}}
\put(1001,213.67){\rule{0.241pt}{0.400pt}}
\multiput(1001.00,214.17)(0.500,-1.000){2}{\rule{0.120pt}{0.400pt}}
\put(1002,212.67){\rule{0.241pt}{0.400pt}}
\multiput(1002.00,213.17)(0.500,-1.000){2}{\rule{0.120pt}{0.400pt}}
\put(1001.0,215.0){\usebox{\plotpoint}}
\put(1003,207.67){\rule{0.241pt}{0.400pt}}
\multiput(1003.00,208.17)(0.500,-1.000){2}{\rule{0.120pt}{0.400pt}}
\put(1003.0,209.0){\rule[-0.200pt]{0.400pt}{0.964pt}}
\put(1004,208){\usebox{\plotpoint}}
\put(1003.67,205){\rule{0.400pt}{0.723pt}}
\multiput(1003.17,206.50)(1.000,-1.500){2}{\rule{0.400pt}{0.361pt}}
\put(1005.0,205.0){\usebox{\plotpoint}}
\put(1005.67,201){\rule{0.400pt}{0.482pt}}
\multiput(1005.17,202.00)(1.000,-1.000){2}{\rule{0.400pt}{0.241pt}}
\put(1006.0,203.0){\rule[-0.200pt]{0.400pt}{0.482pt}}
\put(1007.0,201.0){\usebox{\plotpoint}}
\put(1008.0,198.0){\rule[-0.200pt]{0.400pt}{0.723pt}}
\put(1008.0,198.0){\usebox{\plotpoint}}
\put(1009,194.67){\rule{0.241pt}{0.400pt}}
\multiput(1009.00,195.17)(0.500,-1.000){2}{\rule{0.120pt}{0.400pt}}
\put(1010,193.67){\rule{0.241pt}{0.400pt}}
\multiput(1010.00,194.17)(0.500,-1.000){2}{\rule{0.120pt}{0.400pt}}
\put(1009.0,196.0){\rule[-0.200pt]{0.400pt}{0.482pt}}
\put(1010.67,191){\rule{0.400pt}{0.482pt}}
\multiput(1010.17,192.00)(1.000,-1.000){2}{\rule{0.400pt}{0.241pt}}
\put(1011.0,193.0){\usebox{\plotpoint}}
\put(1012.0,191.0){\usebox{\plotpoint}}
\put(1012.67,188){\rule{0.400pt}{0.482pt}}
\multiput(1012.17,189.00)(1.000,-1.000){2}{\rule{0.400pt}{0.241pt}}
\put(1013.0,190.0){\usebox{\plotpoint}}
\put(1013.67,185){\rule{0.400pt}{0.482pt}}
\multiput(1013.17,186.00)(1.000,-1.000){2}{\rule{0.400pt}{0.241pt}}
\put(1014.0,187.0){\usebox{\plotpoint}}
\put(1015.0,185.0){\usebox{\plotpoint}}
\put(1016,181.67){\rule{0.241pt}{0.400pt}}
\multiput(1016.00,182.17)(0.500,-1.000){2}{\rule{0.120pt}{0.400pt}}
\put(1016.0,183.0){\rule[-0.200pt]{0.400pt}{0.482pt}}
\put(1017,179.67){\rule{0.241pt}{0.400pt}}
\multiput(1017.00,180.17)(0.500,-1.000){2}{\rule{0.120pt}{0.400pt}}
\put(1018,178.67){\rule{0.241pt}{0.400pt}}
\multiput(1018.00,179.17)(0.500,-1.000){2}{\rule{0.120pt}{0.400pt}}
\put(1017.0,181.0){\usebox{\plotpoint}}
\put(1019,176.67){\rule{0.241pt}{0.400pt}}
\multiput(1019.00,177.17)(0.500,-1.000){2}{\rule{0.120pt}{0.400pt}}
\put(1019.0,178.0){\usebox{\plotpoint}}
\put(1020.0,177.0){\usebox{\plotpoint}}
\put(1021.0,175.0){\rule[-0.200pt]{0.400pt}{0.482pt}}
\put(1021.0,175.0){\usebox{\plotpoint}}
\put(1022,172.67){\rule{0.241pt}{0.400pt}}
\multiput(1022.00,173.17)(0.500,-1.000){2}{\rule{0.120pt}{0.400pt}}
\put(1023,171.67){\rule{0.241pt}{0.400pt}}
\multiput(1023.00,172.17)(0.500,-1.000){2}{\rule{0.120pt}{0.400pt}}
\put(1022.0,174.0){\usebox{\plotpoint}}
\put(1024,169.67){\rule{0.241pt}{0.400pt}}
\multiput(1024.00,170.17)(0.500,-1.000){2}{\rule{0.120pt}{0.400pt}}
\put(1024.0,171.0){\usebox{\plotpoint}}
\put(1025.0,170.0){\usebox{\plotpoint}}
\put(1025.67,167){\rule{0.400pt}{0.482pt}}
\multiput(1025.17,168.00)(1.000,-1.000){2}{\rule{0.400pt}{0.241pt}}
\put(1026.0,169.0){\usebox{\plotpoint}}
\put(1027,167){\usebox{\plotpoint}}
\put(1028,165.67){\rule{0.241pt}{0.400pt}}
\multiput(1028.00,166.17)(0.500,-1.000){2}{\rule{0.120pt}{0.400pt}}
\put(1027.0,167.0){\usebox{\plotpoint}}
\put(1029.0,164.0){\rule[-0.200pt]{0.400pt}{0.482pt}}
\put(1029.0,164.0){\usebox{\plotpoint}}
\put(1030.0,163.0){\usebox{\plotpoint}}
\put(1031,161.67){\rule{0.241pt}{0.400pt}}
\multiput(1031.00,162.17)(0.500,-1.000){2}{\rule{0.120pt}{0.400pt}}
\put(1030.0,163.0){\usebox{\plotpoint}}
\put(1032.0,161.0){\usebox{\plotpoint}}
\put(1033,159.67){\rule{0.241pt}{0.400pt}}
\multiput(1033.00,160.17)(0.500,-1.000){2}{\rule{0.120pt}{0.400pt}}
\put(1032.0,161.0){\usebox{\plotpoint}}
\put(1034.0,159.0){\usebox{\plotpoint}}
\put(1034.0,159.0){\usebox{\plotpoint}}
\put(1035,156.67){\rule{0.241pt}{0.400pt}}
\multiput(1035.00,157.17)(0.500,-1.000){2}{\rule{0.120pt}{0.400pt}}
\put(1035.0,158.0){\usebox{\plotpoint}}
\put(1037,155.67){\rule{0.241pt}{0.400pt}}
\multiput(1037.00,156.17)(0.500,-1.000){2}{\rule{0.120pt}{0.400pt}}
\put(1036.0,157.0){\usebox{\plotpoint}}
\put(1038.67,154){\rule{0.400pt}{0.482pt}}
\multiput(1038.17,155.00)(1.000,-1.000){2}{\rule{0.400pt}{0.241pt}}
\put(1038.0,156.0){\usebox{\plotpoint}}
\put(1040,154){\usebox{\plotpoint}}
\put(1040,152.67){\rule{0.241pt}{0.400pt}}
\multiput(1040.00,153.17)(0.500,-1.000){2}{\rule{0.120pt}{0.400pt}}
\put(1041.0,153.0){\usebox{\plotpoint}}
\put(1042.0,152.0){\usebox{\plotpoint}}
\put(1042.0,152.0){\usebox{\plotpoint}}
\put(1043.0,151.0){\usebox{\plotpoint}}
\put(1044,149.67){\rule{0.241pt}{0.400pt}}
\multiput(1044.00,150.17)(0.500,-1.000){2}{\rule{0.120pt}{0.400pt}}
\put(1043.0,151.0){\usebox{\plotpoint}}
\put(1045,150){\usebox{\plotpoint}}
\put(1045,148.67){\rule{0.241pt}{0.400pt}}
\multiput(1045.00,149.17)(0.500,-1.000){2}{\rule{0.120pt}{0.400pt}}
\put(1047,147.67){\rule{0.241pt}{0.400pt}}
\multiput(1047.00,148.17)(0.500,-1.000){2}{\rule{0.120pt}{0.400pt}}
\put(1046.0,149.0){\usebox{\plotpoint}}
\put(1048,148){\usebox{\plotpoint}}
\put(1049,146.67){\rule{0.241pt}{0.400pt}}
\multiput(1049.00,147.17)(0.500,-1.000){2}{\rule{0.120pt}{0.400pt}}
\put(1048.0,148.0){\usebox{\plotpoint}}
\put(1050,147){\usebox{\plotpoint}}
\put(1050.0,147.0){\usebox{\plotpoint}}
\put(1051,144.67){\rule{0.241pt}{0.400pt}}
\multiput(1051.00,144.17)(0.500,1.000){2}{\rule{0.120pt}{0.400pt}}
\put(1052,144.67){\rule{0.241pt}{0.400pt}}
\multiput(1052.00,145.17)(0.500,-1.000){2}{\rule{0.120pt}{0.400pt}}
\put(1051.0,145.0){\rule[-0.200pt]{0.400pt}{0.482pt}}
\put(1053,145){\usebox{\plotpoint}}
\put(1054,143.67){\rule{0.241pt}{0.400pt}}
\multiput(1054.00,144.17)(0.500,-1.000){2}{\rule{0.120pt}{0.400pt}}
\put(1053.0,145.0){\usebox{\plotpoint}}
\put(1055,144){\usebox{\plotpoint}}
\put(1055.0,144.0){\usebox{\plotpoint}}
\put(1056.0,143.0){\usebox{\plotpoint}}
\put(1058,141.67){\rule{0.241pt}{0.400pt}}
\multiput(1058.00,142.17)(0.500,-1.000){2}{\rule{0.120pt}{0.400pt}}
\put(1056.0,143.0){\rule[-0.200pt]{0.482pt}{0.400pt}}
\put(1059,142){\usebox{\plotpoint}}
\put(1059,140.67){\rule{0.241pt}{0.400pt}}
\multiput(1059.00,141.17)(0.500,-1.000){2}{\rule{0.120pt}{0.400pt}}
\put(1062,139.67){\rule{0.241pt}{0.400pt}}
\multiput(1062.00,140.17)(0.500,-1.000){2}{\rule{0.120pt}{0.400pt}}
\put(1060.0,141.0){\rule[-0.200pt]{0.482pt}{0.400pt}}
\put(1063,140){\usebox{\plotpoint}}
\put(1063,138.67){\rule{0.241pt}{0.400pt}}
\multiput(1063.00,139.17)(0.500,-1.000){2}{\rule{0.120pt}{0.400pt}}
\put(1064,139){\usebox{\plotpoint}}
\put(1064,138.67){\rule{0.241pt}{0.400pt}}
\multiput(1064.00,138.17)(0.500,1.000){2}{\rule{0.120pt}{0.400pt}}
\put(1065,138.67){\rule{0.241pt}{0.400pt}}
\multiput(1065.00,139.17)(0.500,-1.000){2}{\rule{0.120pt}{0.400pt}}
\put(1066,139){\usebox{\plotpoint}}
\put(1067,137.67){\rule{0.241pt}{0.400pt}}
\multiput(1067.00,138.17)(0.500,-1.000){2}{\rule{0.120pt}{0.400pt}}
\put(1068,137.67){\rule{0.241pt}{0.400pt}}
\multiput(1068.00,137.17)(0.500,1.000){2}{\rule{0.120pt}{0.400pt}}
\put(1066.0,139.0){\usebox{\plotpoint}}
\put(1069.0,138.0){\usebox{\plotpoint}}
\put(1071,136.67){\rule{0.241pt}{0.400pt}}
\multiput(1071.00,137.17)(0.500,-1.000){2}{\rule{0.120pt}{0.400pt}}
\put(1069.0,138.0){\rule[-0.200pt]{0.482pt}{0.400pt}}
\put(1072,137){\usebox{\plotpoint}}
\put(1073,135.67){\rule{0.241pt}{0.400pt}}
\multiput(1073.00,136.17)(0.500,-1.000){2}{\rule{0.120pt}{0.400pt}}
\put(1072.0,137.0){\usebox{\plotpoint}}
\put(1073.67,135){\rule{0.400pt}{0.482pt}}
\multiput(1073.17,135.00)(1.000,1.000){2}{\rule{0.400pt}{0.241pt}}
\put(1074.0,135.0){\usebox{\plotpoint}}
\put(1075.0,136.0){\usebox{\plotpoint}}
\put(1077,134.67){\rule{0.241pt}{0.400pt}}
\multiput(1077.00,135.17)(0.500,-1.000){2}{\rule{0.120pt}{0.400pt}}
\put(1078,134.67){\rule{0.241pt}{0.400pt}}
\multiput(1078.00,134.17)(0.500,1.000){2}{\rule{0.120pt}{0.400pt}}
\put(1075.0,136.0){\rule[-0.200pt]{0.482pt}{0.400pt}}
\put(1079.0,135.0){\usebox{\plotpoint}}
\put(1079.0,135.0){\usebox{\plotpoint}}
\put(1080,133.67){\rule{0.241pt}{0.400pt}}
\multiput(1080.00,133.17)(0.500,1.000){2}{\rule{0.120pt}{0.400pt}}
\put(1081,133.67){\rule{0.241pt}{0.400pt}}
\multiput(1081.00,134.17)(0.500,-1.000){2}{\rule{0.120pt}{0.400pt}}
\put(1080.0,134.0){\usebox{\plotpoint}}
\put(1082,134){\usebox{\plotpoint}}
\put(1082.0,134.0){\rule[-0.200pt]{0.723pt}{0.400pt}}
\put(1085.0,133.0){\usebox{\plotpoint}}
\put(1085.0,133.0){\usebox{\plotpoint}}
\put(1086,132.67){\rule{0.241pt}{0.400pt}}
\multiput(1086.00,133.17)(0.500,-1.000){2}{\rule{0.120pt}{0.400pt}}
\put(1087,132.67){\rule{0.241pt}{0.400pt}}
\multiput(1087.00,132.17)(0.500,1.000){2}{\rule{0.120pt}{0.400pt}}
\put(1086.0,133.0){\usebox{\plotpoint}}
\put(1088,131.67){\rule{0.241pt}{0.400pt}}
\multiput(1088.00,131.17)(0.500,1.000){2}{\rule{0.120pt}{0.400pt}}
\put(1089,132.67){\rule{0.241pt}{0.400pt}}
\multiput(1089.00,132.17)(0.500,1.000){2}{\rule{0.120pt}{0.400pt}}
\put(1088.0,132.0){\rule[-0.200pt]{0.400pt}{0.482pt}}
\put(1090.0,133.0){\usebox{\plotpoint}}
\put(1090.0,133.0){\usebox{\plotpoint}}
\put(1091,131.67){\rule{0.241pt}{0.400pt}}
\multiput(1091.00,131.17)(0.500,1.000){2}{\rule{0.120pt}{0.400pt}}
\put(1092,131.67){\rule{0.241pt}{0.400pt}}
\multiput(1092.00,132.17)(0.500,-1.000){2}{\rule{0.120pt}{0.400pt}}
\put(1091.0,132.0){\usebox{\plotpoint}}
\put(1093,131.67){\rule{0.241pt}{0.400pt}}
\multiput(1093.00,132.17)(0.500,-1.000){2}{\rule{0.120pt}{0.400pt}}
\put(1093.0,132.0){\usebox{\plotpoint}}
\put(1094,132){\usebox{\plotpoint}}
\put(1094.0,132.0){\rule[-0.200pt]{0.964pt}{0.400pt}}
\put(1098,130.67){\rule{0.241pt}{0.400pt}}
\multiput(1098.00,130.17)(0.500,1.000){2}{\rule{0.120pt}{0.400pt}}
\put(1098.0,131.0){\usebox{\plotpoint}}
\put(1099,132){\usebox{\plotpoint}}
\put(1099,130.67){\rule{0.241pt}{0.400pt}}
\multiput(1099.00,131.17)(0.500,-1.000){2}{\rule{0.120pt}{0.400pt}}
\put(1100.0,131.0){\rule[-0.200pt]{0.964pt}{0.400pt}}
\put(1105,131){\usebox{\plotpoint}}
\put(1105.0,131.0){\usebox{\plotpoint}}
\put(171.0,288.0){\usebox{\plotpoint}}
\put(171.67,289){\rule{0.400pt}{0.723pt}}
\multiput(171.17,289.00)(1.000,1.500){2}{\rule{0.400pt}{0.361pt}}
\put(172.67,290){\rule{0.400pt}{0.482pt}}
\multiput(172.17,291.00)(1.000,-1.000){2}{\rule{0.400pt}{0.241pt}}
\put(172.0,288.0){\usebox{\plotpoint}}
\put(173.67,290){\rule{0.400pt}{0.723pt}}
\multiput(173.17,291.50)(1.000,-1.500){2}{\rule{0.400pt}{0.361pt}}
\put(174.67,290){\rule{0.400pt}{0.482pt}}
\multiput(174.17,290.00)(1.000,1.000){2}{\rule{0.400pt}{0.241pt}}
\put(174.0,290.0){\rule[-0.200pt]{0.400pt}{0.723pt}}
\put(176.0,292.0){\usebox{\plotpoint}}
\put(176.67,293){\rule{0.400pt}{0.482pt}}
\multiput(176.17,293.00)(1.000,1.000){2}{\rule{0.400pt}{0.241pt}}
\put(176.0,293.0){\usebox{\plotpoint}}
\put(177.67,292){\rule{0.400pt}{0.723pt}}
\multiput(177.17,292.00)(1.000,1.500){2}{\rule{0.400pt}{0.361pt}}
\put(178.67,292){\rule{0.400pt}{0.723pt}}
\multiput(178.17,293.50)(1.000,-1.500){2}{\rule{0.400pt}{0.361pt}}
\put(178.0,292.0){\rule[-0.200pt]{0.400pt}{0.723pt}}
\put(179.67,295){\rule{0.400pt}{0.482pt}}
\multiput(179.17,296.00)(1.000,-1.000){2}{\rule{0.400pt}{0.241pt}}
\put(181,294.67){\rule{0.241pt}{0.400pt}}
\multiput(181.00,294.17)(0.500,1.000){2}{\rule{0.120pt}{0.400pt}}
\put(180.0,292.0){\rule[-0.200pt]{0.400pt}{1.204pt}}
\put(182,296){\usebox{\plotpoint}}
\put(182.0,296.0){\rule[-0.200pt]{0.482pt}{0.400pt}}
\put(184,296.67){\rule{0.241pt}{0.400pt}}
\multiput(184.00,296.17)(0.500,1.000){2}{\rule{0.120pt}{0.400pt}}
\put(185,297.67){\rule{0.241pt}{0.400pt}}
\multiput(185.00,297.17)(0.500,1.000){2}{\rule{0.120pt}{0.400pt}}
\put(184.0,296.0){\usebox{\plotpoint}}
\put(186,299){\usebox{\plotpoint}}
\put(186,298.67){\rule{0.241pt}{0.400pt}}
\multiput(186.00,298.17)(0.500,1.000){2}{\rule{0.120pt}{0.400pt}}
\put(186.67,300){\rule{0.400pt}{0.482pt}}
\multiput(186.17,300.00)(1.000,1.000){2}{\rule{0.400pt}{0.241pt}}
\put(187.67,301){\rule{0.400pt}{0.723pt}}
\multiput(187.17,302.50)(1.000,-1.500){2}{\rule{0.400pt}{0.361pt}}
\put(188.67,301){\rule{0.400pt}{0.964pt}}
\multiput(188.17,301.00)(1.000,2.000){2}{\rule{0.400pt}{0.482pt}}
\put(189.67,302){\rule{0.400pt}{0.723pt}}
\multiput(189.17,303.50)(1.000,-1.500){2}{\rule{0.400pt}{0.361pt}}
\put(188.0,302.0){\rule[-0.200pt]{0.400pt}{0.482pt}}
\put(191,302){\usebox{\plotpoint}}
\put(190.67,302){\rule{0.400pt}{0.482pt}}
\multiput(190.17,302.00)(1.000,1.000){2}{\rule{0.400pt}{0.241pt}}
\put(192,303.67){\rule{0.241pt}{0.400pt}}
\multiput(192.00,303.17)(0.500,1.000){2}{\rule{0.120pt}{0.400pt}}
\put(192.67,303){\rule{0.400pt}{0.482pt}}
\multiput(192.17,303.00)(1.000,1.000){2}{\rule{0.400pt}{0.241pt}}
\put(194,303.67){\rule{0.241pt}{0.400pt}}
\multiput(194.00,304.17)(0.500,-1.000){2}{\rule{0.120pt}{0.400pt}}
\put(193.0,303.0){\rule[-0.200pt]{0.400pt}{0.482pt}}
\put(195,304){\usebox{\plotpoint}}
\put(195,302.67){\rule{0.241pt}{0.400pt}}
\multiput(195.00,303.17)(0.500,-1.000){2}{\rule{0.120pt}{0.400pt}}
\put(195.67,303){\rule{0.400pt}{0.482pt}}
\multiput(195.17,303.00)(1.000,1.000){2}{\rule{0.400pt}{0.241pt}}
\put(197,305.67){\rule{0.241pt}{0.400pt}}
\multiput(197.00,305.17)(0.500,1.000){2}{\rule{0.120pt}{0.400pt}}
\put(197.67,304){\rule{0.400pt}{0.723pt}}
\multiput(197.17,305.50)(1.000,-1.500){2}{\rule{0.400pt}{0.361pt}}
\put(197.0,305.0){\usebox{\plotpoint}}
\put(199.0,304.0){\usebox{\plotpoint}}
\put(201,304.67){\rule{0.241pt}{0.400pt}}
\multiput(201.00,304.17)(0.500,1.000){2}{\rule{0.120pt}{0.400pt}}
\put(201.67,304){\rule{0.400pt}{0.482pt}}
\multiput(201.17,305.00)(1.000,-1.000){2}{\rule{0.400pt}{0.241pt}}
\put(199.0,305.0){\rule[-0.200pt]{0.482pt}{0.400pt}}
\put(203,304){\usebox{\plotpoint}}
\put(203,302.67){\rule{0.241pt}{0.400pt}}
\multiput(203.00,303.17)(0.500,-1.000){2}{\rule{0.120pt}{0.400pt}}
\put(203.67,303){\rule{0.400pt}{0.482pt}}
\multiput(203.17,303.00)(1.000,1.000){2}{\rule{0.400pt}{0.241pt}}
\put(205,304.67){\rule{0.241pt}{0.400pt}}
\multiput(205.00,305.17)(0.500,-1.000){2}{\rule{0.120pt}{0.400pt}}
\put(206,303.67){\rule{0.241pt}{0.400pt}}
\multiput(206.00,304.17)(0.500,-1.000){2}{\rule{0.120pt}{0.400pt}}
\put(205.0,305.0){\usebox{\plotpoint}}
\put(207,304.67){\rule{0.241pt}{0.400pt}}
\multiput(207.00,304.17)(0.500,1.000){2}{\rule{0.120pt}{0.400pt}}
\put(208,304.67){\rule{0.241pt}{0.400pt}}
\multiput(208.00,305.17)(0.500,-1.000){2}{\rule{0.120pt}{0.400pt}}
\put(207.0,304.0){\usebox{\plotpoint}}
\put(209,303.67){\rule{0.241pt}{0.400pt}}
\multiput(209.00,303.17)(0.500,1.000){2}{\rule{0.120pt}{0.400pt}}
\put(209.0,304.0){\usebox{\plotpoint}}
\put(210.67,302){\rule{0.400pt}{0.723pt}}
\multiput(210.17,303.50)(1.000,-1.500){2}{\rule{0.400pt}{0.361pt}}
\put(212,301.67){\rule{0.241pt}{0.400pt}}
\multiput(212.00,301.17)(0.500,1.000){2}{\rule{0.120pt}{0.400pt}}
\put(210.0,305.0){\usebox{\plotpoint}}
\put(213,303){\usebox{\plotpoint}}
\put(213,302.67){\rule{0.241pt}{0.400pt}}
\multiput(213.00,302.17)(0.500,1.000){2}{\rule{0.120pt}{0.400pt}}
\put(213.67,302){\rule{0.400pt}{0.482pt}}
\multiput(213.17,303.00)(1.000,-1.000){2}{\rule{0.400pt}{0.241pt}}
\put(215,301.67){\rule{0.241pt}{0.400pt}}
\multiput(215.00,302.17)(0.500,-1.000){2}{\rule{0.120pt}{0.400pt}}
\put(216,300.67){\rule{0.241pt}{0.400pt}}
\multiput(216.00,301.17)(0.500,-1.000){2}{\rule{0.120pt}{0.400pt}}
\put(215.0,302.0){\usebox{\plotpoint}}
\put(217,301){\usebox{\plotpoint}}
\put(217,300.67){\rule{0.241pt}{0.400pt}}
\multiput(217.00,300.17)(0.500,1.000){2}{\rule{0.120pt}{0.400pt}}
\put(217.67,300){\rule{0.400pt}{0.482pt}}
\multiput(217.17,301.00)(1.000,-1.000){2}{\rule{0.400pt}{0.241pt}}
\put(219,300){\usebox{\plotpoint}}
\put(219,299.67){\rule{0.241pt}{0.400pt}}
\multiput(219.00,299.17)(0.500,1.000){2}{\rule{0.120pt}{0.400pt}}
\put(220,299.67){\rule{0.241pt}{0.400pt}}
\multiput(220.00,300.17)(0.500,-1.000){2}{\rule{0.120pt}{0.400pt}}
\put(221.0,299.0){\usebox{\plotpoint}}
\put(221.0,299.0){\rule[-0.200pt]{0.482pt}{0.400pt}}
\put(223.0,298.0){\usebox{\plotpoint}}
\put(224,296.67){\rule{0.241pt}{0.400pt}}
\multiput(224.00,297.17)(0.500,-1.000){2}{\rule{0.120pt}{0.400pt}}
\put(223.0,298.0){\usebox{\plotpoint}}
\put(225,297){\usebox{\plotpoint}}
\put(226,295.67){\rule{0.241pt}{0.400pt}}
\multiput(226.00,296.17)(0.500,-1.000){2}{\rule{0.120pt}{0.400pt}}
\put(225.0,297.0){\usebox{\plotpoint}}
\put(227,296){\usebox{\plotpoint}}
\put(228,294.67){\rule{0.241pt}{0.400pt}}
\multiput(228.00,295.17)(0.500,-1.000){2}{\rule{0.120pt}{0.400pt}}
\put(227.0,296.0){\usebox{\plotpoint}}
\put(229.0,294.0){\usebox{\plotpoint}}
\put(231,292.67){\rule{0.241pt}{0.400pt}}
\multiput(231.00,293.17)(0.500,-1.000){2}{\rule{0.120pt}{0.400pt}}
\put(229.0,294.0){\rule[-0.200pt]{0.482pt}{0.400pt}}
\put(232.0,293.0){\usebox{\plotpoint}}
\put(233,290.67){\rule{0.241pt}{0.400pt}}
\multiput(233.00,291.17)(0.500,-1.000){2}{\rule{0.120pt}{0.400pt}}
\put(233.0,292.0){\usebox{\plotpoint}}
\put(234.0,291.0){\rule[-0.200pt]{0.482pt}{0.400pt}}
\put(236,289.67){\rule{0.241pt}{0.400pt}}
\multiput(236.00,289.17)(0.500,1.000){2}{\rule{0.120pt}{0.400pt}}
\put(236.67,289){\rule{0.400pt}{0.482pt}}
\multiput(236.17,290.00)(1.000,-1.000){2}{\rule{0.400pt}{0.241pt}}
\put(236.0,290.0){\usebox{\plotpoint}}
\put(238,289){\usebox{\plotpoint}}
\put(239,287.67){\rule{0.241pt}{0.400pt}}
\multiput(239.00,288.17)(0.500,-1.000){2}{\rule{0.120pt}{0.400pt}}
\put(238.0,289.0){\usebox{\plotpoint}}
\put(240,288){\usebox{\plotpoint}}
\put(240,287.67){\rule{0.241pt}{0.400pt}}
\multiput(240.00,287.17)(0.500,1.000){2}{\rule{0.120pt}{0.400pt}}
\put(241,287.67){\rule{0.241pt}{0.400pt}}
\multiput(241.00,288.17)(0.500,-1.000){2}{\rule{0.120pt}{0.400pt}}
\put(241.67,285){\rule{0.400pt}{0.482pt}}
\multiput(241.17,286.00)(1.000,-1.000){2}{\rule{0.400pt}{0.241pt}}
\put(242.67,285){\rule{0.400pt}{0.482pt}}
\multiput(242.17,285.00)(1.000,1.000){2}{\rule{0.400pt}{0.241pt}}
\put(242.0,287.0){\usebox{\plotpoint}}
\put(244,284.67){\rule{0.241pt}{0.400pt}}
\multiput(244.00,285.17)(0.500,-1.000){2}{\rule{0.120pt}{0.400pt}}
\put(245,283.67){\rule{0.241pt}{0.400pt}}
\multiput(245.00,284.17)(0.500,-1.000){2}{\rule{0.120pt}{0.400pt}}
\put(244.0,286.0){\usebox{\plotpoint}}
\put(246,283.67){\rule{0.241pt}{0.400pt}}
\multiput(246.00,284.17)(0.500,-1.000){2}{\rule{0.120pt}{0.400pt}}
\put(247,282.67){\rule{0.241pt}{0.400pt}}
\multiput(247.00,283.17)(0.500,-1.000){2}{\rule{0.120pt}{0.400pt}}
\put(246.0,284.0){\usebox{\plotpoint}}
\put(248,282.67){\rule{0.241pt}{0.400pt}}
\multiput(248.00,283.17)(0.500,-1.000){2}{\rule{0.120pt}{0.400pt}}
\put(248.0,283.0){\usebox{\plotpoint}}
\put(249.0,283.0){\usebox{\plotpoint}}
\put(249.67,281){\rule{0.400pt}{0.482pt}}
\multiput(249.17,281.00)(1.000,1.000){2}{\rule{0.400pt}{0.241pt}}
\put(251,281.67){\rule{0.241pt}{0.400pt}}
\multiput(251.00,282.17)(0.500,-1.000){2}{\rule{0.120pt}{0.400pt}}
\put(250.0,281.0){\rule[-0.200pt]{0.400pt}{0.482pt}}
\put(252.0,281.0){\usebox{\plotpoint}}
\put(252.0,281.0){\rule[-0.200pt]{0.964pt}{0.400pt}}
\put(256,278.67){\rule{0.241pt}{0.400pt}}
\multiput(256.00,279.17)(0.500,-1.000){2}{\rule{0.120pt}{0.400pt}}
\put(256.0,280.0){\usebox{\plotpoint}}
\put(257.0,279.0){\usebox{\plotpoint}}
\put(258,277.67){\rule{0.241pt}{0.400pt}}
\multiput(258.00,277.17)(0.500,1.000){2}{\rule{0.120pt}{0.400pt}}
\put(259,277.67){\rule{0.241pt}{0.400pt}}
\multiput(259.00,278.17)(0.500,-1.000){2}{\rule{0.120pt}{0.400pt}}
\put(258.0,278.0){\usebox{\plotpoint}}
\put(260,278){\usebox{\plotpoint}}
\put(260,276.67){\rule{0.241pt}{0.400pt}}
\multiput(260.00,277.17)(0.500,-1.000){2}{\rule{0.120pt}{0.400pt}}
\put(261,276.67){\rule{0.241pt}{0.400pt}}
\multiput(261.00,276.17)(0.500,1.000){2}{\rule{0.120pt}{0.400pt}}
\put(262,278){\usebox{\plotpoint}}
\put(262,276.67){\rule{0.241pt}{0.400pt}}
\multiput(262.00,277.17)(0.500,-1.000){2}{\rule{0.120pt}{0.400pt}}
\put(263,275.67){\rule{0.241pt}{0.400pt}}
\multiput(263.00,276.17)(0.500,-1.000){2}{\rule{0.120pt}{0.400pt}}
\put(264.0,276.0){\usebox{\plotpoint}}
\put(266,275.67){\rule{0.241pt}{0.400pt}}
\multiput(266.00,276.17)(0.500,-1.000){2}{\rule{0.120pt}{0.400pt}}
\put(264.0,277.0){\rule[-0.200pt]{0.482pt}{0.400pt}}
\put(268,274.67){\rule{0.241pt}{0.400pt}}
\multiput(268.00,275.17)(0.500,-1.000){2}{\rule{0.120pt}{0.400pt}}
\put(267.0,276.0){\usebox{\plotpoint}}
\put(269.0,275.0){\usebox{\plotpoint}}
\put(270.0,275.0){\usebox{\plotpoint}}
\put(271,274.67){\rule{0.241pt}{0.400pt}}
\multiput(271.00,275.17)(0.500,-1.000){2}{\rule{0.120pt}{0.400pt}}
\put(270.0,276.0){\usebox{\plotpoint}}
\put(272,275){\usebox{\plotpoint}}
\put(273,274.67){\rule{0.241pt}{0.400pt}}
\multiput(273.00,274.17)(0.500,1.000){2}{\rule{0.120pt}{0.400pt}}
\put(272.0,275.0){\usebox{\plotpoint}}
\put(274,273.67){\rule{0.241pt}{0.400pt}}
\multiput(274.00,274.17)(0.500,-1.000){2}{\rule{0.120pt}{0.400pt}}
\put(274.0,275.0){\usebox{\plotpoint}}
\put(275.67,274){\rule{0.400pt}{0.482pt}}
\multiput(275.17,274.00)(1.000,1.000){2}{\rule{0.400pt}{0.241pt}}
\put(277,274.67){\rule{0.241pt}{0.400pt}}
\multiput(277.00,275.17)(0.500,-1.000){2}{\rule{0.120pt}{0.400pt}}
\put(275.0,274.0){\usebox{\plotpoint}}
\put(278,274.67){\rule{0.241pt}{0.400pt}}
\multiput(278.00,275.17)(0.500,-1.000){2}{\rule{0.120pt}{0.400pt}}
\put(278.0,275.0){\usebox{\plotpoint}}
\put(281,273.67){\rule{0.241pt}{0.400pt}}
\multiput(281.00,274.17)(0.500,-1.000){2}{\rule{0.120pt}{0.400pt}}
\put(279.0,275.0){\rule[-0.200pt]{0.482pt}{0.400pt}}
\put(282,274.67){\rule{0.241pt}{0.400pt}}
\multiput(282.00,275.17)(0.500,-1.000){2}{\rule{0.120pt}{0.400pt}}
\put(283,274.67){\rule{0.241pt}{0.400pt}}
\multiput(283.00,274.17)(0.500,1.000){2}{\rule{0.120pt}{0.400pt}}
\put(282.0,274.0){\rule[-0.200pt]{0.400pt}{0.482pt}}
\put(284,276){\usebox{\plotpoint}}
\put(287,275.67){\rule{0.241pt}{0.400pt}}
\multiput(287.00,275.17)(0.500,1.000){2}{\rule{0.120pt}{0.400pt}}
\put(284.0,276.0){\rule[-0.200pt]{0.723pt}{0.400pt}}
\put(288,274.67){\rule{0.241pt}{0.400pt}}
\multiput(288.00,275.17)(0.500,-1.000){2}{\rule{0.120pt}{0.400pt}}
\put(288.67,275){\rule{0.400pt}{0.723pt}}
\multiput(288.17,275.00)(1.000,1.500){2}{\rule{0.400pt}{0.361pt}}
\put(288.0,276.0){\usebox{\plotpoint}}
\put(290.0,277.0){\usebox{\plotpoint}}
\put(291,276.67){\rule{0.241pt}{0.400pt}}
\multiput(291.00,276.17)(0.500,1.000){2}{\rule{0.120pt}{0.400pt}}
\put(290.0,277.0){\usebox{\plotpoint}}
\put(292,278){\usebox{\plotpoint}}
\put(292,277.67){\rule{0.241pt}{0.400pt}}
\multiput(292.00,277.17)(0.500,1.000){2}{\rule{0.120pt}{0.400pt}}
\put(294.67,279){\rule{0.400pt}{0.482pt}}
\multiput(294.17,279.00)(1.000,1.000){2}{\rule{0.400pt}{0.241pt}}
\put(293.0,279.0){\rule[-0.200pt]{0.482pt}{0.400pt}}
\put(296,279.67){\rule{0.241pt}{0.400pt}}
\multiput(296.00,279.17)(0.500,1.000){2}{\rule{0.120pt}{0.400pt}}
\put(297,279.67){\rule{0.241pt}{0.400pt}}
\multiput(297.00,280.17)(0.500,-1.000){2}{\rule{0.120pt}{0.400pt}}
\put(296.0,280.0){\usebox{\plotpoint}}
\put(298,280){\usebox{\plotpoint}}
\put(298,279.67){\rule{0.241pt}{0.400pt}}
\multiput(298.00,279.17)(0.500,1.000){2}{\rule{0.120pt}{0.400pt}}
\put(299.0,281.0){\usebox{\plotpoint}}
\put(300,280.67){\rule{0.241pt}{0.400pt}}
\multiput(300.00,281.17)(0.500,-1.000){2}{\rule{0.120pt}{0.400pt}}
\put(300.67,281){\rule{0.400pt}{0.723pt}}
\multiput(300.17,281.00)(1.000,1.500){2}{\rule{0.400pt}{0.361pt}}
\put(300.0,281.0){\usebox{\plotpoint}}
\put(302,284){\usebox{\plotpoint}}
\put(303,283.67){\rule{0.241pt}{0.400pt}}
\multiput(303.00,283.17)(0.500,1.000){2}{\rule{0.120pt}{0.400pt}}
\put(302.0,284.0){\usebox{\plotpoint}}
\put(304.0,285.0){\usebox{\plotpoint}}
\put(304.0,286.0){\rule[-0.200pt]{0.482pt}{0.400pt}}
\put(306,286.67){\rule{0.241pt}{0.400pt}}
\multiput(306.00,286.17)(0.500,1.000){2}{\rule{0.120pt}{0.400pt}}
\put(306.0,286.0){\usebox{\plotpoint}}
\put(307.0,288.0){\usebox{\plotpoint}}
\put(308.0,288.0){\usebox{\plotpoint}}
\put(308.67,289){\rule{0.400pt}{0.482pt}}
\multiput(308.17,289.00)(1.000,1.000){2}{\rule{0.400pt}{0.241pt}}
\put(308.0,289.0){\usebox{\plotpoint}}
\put(310,291){\usebox{\plotpoint}}
\put(309.67,291){\rule{0.400pt}{0.482pt}}
\multiput(309.17,291.00)(1.000,1.000){2}{\rule{0.400pt}{0.241pt}}
\put(313,292.67){\rule{0.241pt}{0.400pt}}
\multiput(313.00,292.17)(0.500,1.000){2}{\rule{0.120pt}{0.400pt}}
\put(311.0,293.0){\rule[-0.200pt]{0.482pt}{0.400pt}}
\put(314,295.67){\rule{0.241pt}{0.400pt}}
\multiput(314.00,296.17)(0.500,-1.000){2}{\rule{0.120pt}{0.400pt}}
\put(314.67,296){\rule{0.400pt}{0.482pt}}
\multiput(314.17,296.00)(1.000,1.000){2}{\rule{0.400pt}{0.241pt}}
\put(314.0,294.0){\rule[-0.200pt]{0.400pt}{0.723pt}}
\put(316,296.67){\rule{0.241pt}{0.400pt}}
\multiput(316.00,296.17)(0.500,1.000){2}{\rule{0.120pt}{0.400pt}}
\put(317,297.67){\rule{0.241pt}{0.400pt}}
\multiput(317.00,297.17)(0.500,1.000){2}{\rule{0.120pt}{0.400pt}}
\put(316.0,297.0){\usebox{\plotpoint}}
\put(318,299.67){\rule{0.241pt}{0.400pt}}
\multiput(318.00,299.17)(0.500,1.000){2}{\rule{0.120pt}{0.400pt}}
\put(319,300.67){\rule{0.241pt}{0.400pt}}
\multiput(319.00,300.17)(0.500,1.000){2}{\rule{0.120pt}{0.400pt}}
\put(318.0,299.0){\usebox{\plotpoint}}
\put(320,302.67){\rule{0.241pt}{0.400pt}}
\multiput(320.00,302.17)(0.500,1.000){2}{\rule{0.120pt}{0.400pt}}
\put(320.0,302.0){\usebox{\plotpoint}}
\put(321.0,304.0){\usebox{\plotpoint}}
\put(322,304.67){\rule{0.241pt}{0.400pt}}
\multiput(322.00,304.17)(0.500,1.000){2}{\rule{0.120pt}{0.400pt}}
\put(323,305.67){\rule{0.241pt}{0.400pt}}
\multiput(323.00,305.17)(0.500,1.000){2}{\rule{0.120pt}{0.400pt}}
\put(322.0,304.0){\usebox{\plotpoint}}
\put(324.0,307.0){\rule[-0.200pt]{0.400pt}{0.482pt}}
\put(324.67,309){\rule{0.400pt}{0.482pt}}
\multiput(324.17,309.00)(1.000,1.000){2}{\rule{0.400pt}{0.241pt}}
\put(324.0,309.0){\usebox{\plotpoint}}
\put(326,311.67){\rule{0.241pt}{0.400pt}}
\multiput(326.00,312.17)(0.500,-1.000){2}{\rule{0.120pt}{0.400pt}}
\put(327,311.67){\rule{0.241pt}{0.400pt}}
\multiput(327.00,311.17)(0.500,1.000){2}{\rule{0.120pt}{0.400pt}}
\put(326.0,311.0){\rule[-0.200pt]{0.400pt}{0.482pt}}
\put(327.67,314){\rule{0.400pt}{0.482pt}}
\multiput(327.17,314.00)(1.000,1.000){2}{\rule{0.400pt}{0.241pt}}
\put(329,315.67){\rule{0.241pt}{0.400pt}}
\multiput(329.00,315.17)(0.500,1.000){2}{\rule{0.120pt}{0.400pt}}
\put(328.0,313.0){\usebox{\plotpoint}}
\put(330,317){\usebox{\plotpoint}}
\put(329.67,317){\rule{0.400pt}{0.482pt}}
\multiput(329.17,317.00)(1.000,1.000){2}{\rule{0.400pt}{0.241pt}}
\put(330.67,319){\rule{0.400pt}{0.482pt}}
\multiput(330.17,319.00)(1.000,1.000){2}{\rule{0.400pt}{0.241pt}}
\put(332,321){\usebox{\plotpoint}}
\put(332,320.67){\rule{0.241pt}{0.400pt}}
\multiput(332.00,320.17)(0.500,1.000){2}{\rule{0.120pt}{0.400pt}}
\put(333,321.67){\rule{0.241pt}{0.400pt}}
\multiput(333.00,321.17)(0.500,1.000){2}{\rule{0.120pt}{0.400pt}}
\put(334,323){\usebox{\plotpoint}}
\put(333.67,323){\rule{0.400pt}{0.482pt}}
\multiput(333.17,323.00)(1.000,1.000){2}{\rule{0.400pt}{0.241pt}}
\put(334.67,325){\rule{0.400pt}{0.482pt}}
\multiput(334.17,325.00)(1.000,1.000){2}{\rule{0.400pt}{0.241pt}}
\put(336,327){\usebox{\plotpoint}}
\put(335.67,327){\rule{0.400pt}{0.723pt}}
\multiput(335.17,327.00)(1.000,1.500){2}{\rule{0.400pt}{0.361pt}}
\put(337.0,330.0){\usebox{\plotpoint}}
\put(337.67,332){\rule{0.400pt}{0.482pt}}
\multiput(337.17,332.00)(1.000,1.000){2}{\rule{0.400pt}{0.241pt}}
\put(339,332.67){\rule{0.241pt}{0.400pt}}
\multiput(339.00,333.17)(0.500,-1.000){2}{\rule{0.120pt}{0.400pt}}
\put(338.0,330.0){\rule[-0.200pt]{0.400pt}{0.482pt}}
\put(339.67,335){\rule{0.400pt}{0.482pt}}
\multiput(339.17,335.00)(1.000,1.000){2}{\rule{0.400pt}{0.241pt}}
\put(340.0,333.0){\rule[-0.200pt]{0.400pt}{0.482pt}}
\put(341.0,337.0){\usebox{\plotpoint}}
\put(342,338.67){\rule{0.241pt}{0.400pt}}
\multiput(342.00,338.17)(0.500,1.000){2}{\rule{0.120pt}{0.400pt}}
\put(343,339.67){\rule{0.241pt}{0.400pt}}
\multiput(343.00,339.17)(0.500,1.000){2}{\rule{0.120pt}{0.400pt}}
\put(342.0,337.0){\rule[-0.200pt]{0.400pt}{0.482pt}}
\put(343.67,342){\rule{0.400pt}{0.482pt}}
\multiput(343.17,342.00)(1.000,1.000){2}{\rule{0.400pt}{0.241pt}}
\put(345,343.67){\rule{0.241pt}{0.400pt}}
\multiput(345.00,343.17)(0.500,1.000){2}{\rule{0.120pt}{0.400pt}}
\put(344.0,341.0){\usebox{\plotpoint}}
\put(346.0,345.0){\usebox{\plotpoint}}
\put(346.67,346){\rule{0.400pt}{0.723pt}}
\multiput(346.17,346.00)(1.000,1.500){2}{\rule{0.400pt}{0.361pt}}
\put(346.0,346.0){\usebox{\plotpoint}}
\put(348,349.67){\rule{0.241pt}{0.400pt}}
\multiput(348.00,350.17)(0.500,-1.000){2}{\rule{0.120pt}{0.400pt}}
\put(348.67,350){\rule{0.400pt}{0.723pt}}
\multiput(348.17,350.00)(1.000,1.500){2}{\rule{0.400pt}{0.361pt}}
\put(348.0,349.0){\rule[-0.200pt]{0.400pt}{0.482pt}}
\put(349.67,352){\rule{0.400pt}{0.723pt}}
\multiput(349.17,352.00)(1.000,1.500){2}{\rule{0.400pt}{0.361pt}}
\put(350.67,355){\rule{0.400pt}{0.482pt}}
\multiput(350.17,355.00)(1.000,1.000){2}{\rule{0.400pt}{0.241pt}}
\put(350.0,352.0){\usebox{\plotpoint}}
\put(352.0,357.0){\usebox{\plotpoint}}
\put(352.67,358){\rule{0.400pt}{0.964pt}}
\multiput(352.17,358.00)(1.000,2.000){2}{\rule{0.400pt}{0.482pt}}
\put(352.0,358.0){\usebox{\plotpoint}}
\put(354,362.67){\rule{0.241pt}{0.400pt}}
\multiput(354.00,362.17)(0.500,1.000){2}{\rule{0.120pt}{0.400pt}}
\put(354.0,362.0){\usebox{\plotpoint}}
\put(355.0,364.0){\usebox{\plotpoint}}
\put(355.67,367){\rule{0.400pt}{0.723pt}}
\multiput(355.17,367.00)(1.000,1.500){2}{\rule{0.400pt}{0.361pt}}
\put(357,368.67){\rule{0.241pt}{0.400pt}}
\multiput(357.00,369.17)(0.500,-1.000){2}{\rule{0.120pt}{0.400pt}}
\put(356.0,364.0){\rule[-0.200pt]{0.400pt}{0.723pt}}
\put(357.67,371){\rule{0.400pt}{0.482pt}}
\multiput(357.17,371.00)(1.000,1.000){2}{\rule{0.400pt}{0.241pt}}
\put(358.0,369.0){\rule[-0.200pt]{0.400pt}{0.482pt}}
\put(358.67,372){\rule{0.400pt}{0.964pt}}
\multiput(358.17,372.00)(1.000,2.000){2}{\rule{0.400pt}{0.482pt}}
\put(359.0,372.0){\usebox{\plotpoint}}
\put(360.0,376.0){\usebox{\plotpoint}}
\put(360.67,377){\rule{0.400pt}{0.964pt}}
\multiput(360.17,377.00)(1.000,2.000){2}{\rule{0.400pt}{0.482pt}}
\put(362,379.67){\rule{0.241pt}{0.400pt}}
\multiput(362.00,380.17)(0.500,-1.000){2}{\rule{0.120pt}{0.400pt}}
\put(361.0,376.0){\usebox{\plotpoint}}
\put(363,380.67){\rule{0.241pt}{0.400pt}}
\multiput(363.00,380.17)(0.500,1.000){2}{\rule{0.120pt}{0.400pt}}
\put(363.67,382){\rule{0.400pt}{0.723pt}}
\multiput(363.17,382.00)(1.000,1.500){2}{\rule{0.400pt}{0.361pt}}
\put(363.0,380.0){\usebox{\plotpoint}}
\put(365,385){\usebox{\plotpoint}}
\put(364.67,385){\rule{0.400pt}{0.964pt}}
\multiput(364.17,385.00)(1.000,2.000){2}{\rule{0.400pt}{0.482pt}}
\put(366,387.67){\rule{0.241pt}{0.400pt}}
\multiput(366.00,388.17)(0.500,-1.000){2}{\rule{0.120pt}{0.400pt}}
\put(367,389.67){\rule{0.241pt}{0.400pt}}
\multiput(367.00,389.17)(0.500,1.000){2}{\rule{0.120pt}{0.400pt}}
\put(367.67,391){\rule{0.400pt}{0.482pt}}
\multiput(367.17,391.00)(1.000,1.000){2}{\rule{0.400pt}{0.241pt}}
\put(367.0,388.0){\rule[-0.200pt]{0.400pt}{0.482pt}}
\put(368.67,394){\rule{0.400pt}{0.482pt}}
\multiput(368.17,394.00)(1.000,1.000){2}{\rule{0.400pt}{0.241pt}}
\put(369.67,396){\rule{0.400pt}{0.723pt}}
\multiput(369.17,396.00)(1.000,1.500){2}{\rule{0.400pt}{0.361pt}}
\put(369.0,393.0){\usebox{\plotpoint}}
\put(371,399){\usebox{\plotpoint}}
\put(371,398.67){\rule{0.241pt}{0.400pt}}
\multiput(371.00,398.17)(0.500,1.000){2}{\rule{0.120pt}{0.400pt}}
\put(372,399.67){\rule{0.241pt}{0.400pt}}
\multiput(372.00,399.17)(0.500,1.000){2}{\rule{0.120pt}{0.400pt}}
\put(372.67,404){\rule{0.400pt}{0.482pt}}
\multiput(372.17,404.00)(1.000,1.000){2}{\rule{0.400pt}{0.241pt}}
\put(374,405.67){\rule{0.241pt}{0.400pt}}
\multiput(374.00,405.17)(0.500,1.000){2}{\rule{0.120pt}{0.400pt}}
\put(373.0,401.0){\rule[-0.200pt]{0.400pt}{0.723pt}}
\put(375,407){\usebox{\plotpoint}}
\put(374.67,407){\rule{0.400pt}{0.482pt}}
\multiput(374.17,407.00)(1.000,1.000){2}{\rule{0.400pt}{0.241pt}}
\put(376,408.67){\rule{0.241pt}{0.400pt}}
\multiput(376.00,408.17)(0.500,1.000){2}{\rule{0.120pt}{0.400pt}}
\put(376.67,411){\rule{0.400pt}{0.964pt}}
\multiput(376.17,411.00)(1.000,2.000){2}{\rule{0.400pt}{0.482pt}}
\put(378,413.67){\rule{0.241pt}{0.400pt}}
\multiput(378.00,414.17)(0.500,-1.000){2}{\rule{0.120pt}{0.400pt}}
\put(377.0,410.0){\usebox{\plotpoint}}
\put(379.0,414.0){\rule[-0.200pt]{0.400pt}{0.723pt}}
\put(379.67,417){\rule{0.400pt}{0.482pt}}
\multiput(379.17,417.00)(1.000,1.000){2}{\rule{0.400pt}{0.241pt}}
\put(379.0,417.0){\usebox{\plotpoint}}
\put(381,420.67){\rule{0.241pt}{0.400pt}}
\multiput(381.00,420.17)(0.500,1.000){2}{\rule{0.120pt}{0.400pt}}
\put(381.67,422){\rule{0.400pt}{0.723pt}}
\multiput(381.17,422.00)(1.000,1.500){2}{\rule{0.400pt}{0.361pt}}
\put(381.0,419.0){\rule[-0.200pt]{0.400pt}{0.482pt}}
\put(383,425){\usebox{\plotpoint}}
\put(383,424.67){\rule{0.241pt}{0.400pt}}
\multiput(383.00,424.17)(0.500,1.000){2}{\rule{0.120pt}{0.400pt}}
\put(383.67,426){\rule{0.400pt}{0.723pt}}
\multiput(383.17,426.00)(1.000,1.500){2}{\rule{0.400pt}{0.361pt}}
\put(385,429){\usebox{\plotpoint}}
\put(385,428.67){\rule{0.241pt}{0.400pt}}
\multiput(385.00,428.17)(0.500,1.000){2}{\rule{0.120pt}{0.400pt}}
\put(385.67,430){\rule{0.400pt}{0.964pt}}
\multiput(385.17,430.00)(1.000,2.000){2}{\rule{0.400pt}{0.482pt}}
\put(387,431.67){\rule{0.241pt}{0.400pt}}
\multiput(387.00,431.17)(0.500,1.000){2}{\rule{0.120pt}{0.400pt}}
\put(387.67,433){\rule{0.400pt}{1.204pt}}
\multiput(387.17,433.00)(1.000,2.500){2}{\rule{0.400pt}{0.602pt}}
\put(387.0,432.0){\rule[-0.200pt]{0.400pt}{0.482pt}}
\put(389.0,438.0){\usebox{\plotpoint}}
\put(389.67,439){\rule{0.400pt}{0.482pt}}
\multiput(389.17,439.00)(1.000,1.000){2}{\rule{0.400pt}{0.241pt}}
\put(389.0,439.0){\usebox{\plotpoint}}
\put(391.0,441.0){\rule[-0.200pt]{0.400pt}{0.723pt}}
\put(391.67,444){\rule{0.400pt}{0.482pt}}
\multiput(391.17,444.00)(1.000,1.000){2}{\rule{0.400pt}{0.241pt}}
\put(391.0,444.0){\usebox{\plotpoint}}
\put(392.67,449){\rule{0.400pt}{0.482pt}}
\multiput(392.17,449.00)(1.000,1.000){2}{\rule{0.400pt}{0.241pt}}
\put(394,449.67){\rule{0.241pt}{0.400pt}}
\multiput(394.00,450.17)(0.500,-1.000){2}{\rule{0.120pt}{0.400pt}}
\put(393.0,446.0){\rule[-0.200pt]{0.400pt}{0.723pt}}
\put(395.0,450.0){\rule[-0.200pt]{0.400pt}{0.482pt}}
\put(395.67,452){\rule{0.400pt}{0.964pt}}
\multiput(395.17,452.00)(1.000,2.000){2}{\rule{0.400pt}{0.482pt}}
\put(395.0,452.0){\usebox{\plotpoint}}
\put(397,457.67){\rule{0.241pt}{0.400pt}}
\multiput(397.00,457.17)(0.500,1.000){2}{\rule{0.120pt}{0.400pt}}
\put(397.0,456.0){\rule[-0.200pt]{0.400pt}{0.482pt}}
\put(398.0,459.0){\usebox{\plotpoint}}
\put(399.0,459.0){\rule[-0.200pt]{0.400pt}{0.723pt}}
\put(400,461.67){\rule{0.241pt}{0.400pt}}
\multiput(400.00,461.17)(0.500,1.000){2}{\rule{0.120pt}{0.400pt}}
\put(399.0,462.0){\usebox{\plotpoint}}
\put(400.67,465){\rule{0.400pt}{0.482pt}}
\multiput(400.17,465.00)(1.000,1.000){2}{\rule{0.400pt}{0.241pt}}
\put(402,466.67){\rule{0.241pt}{0.400pt}}
\multiput(402.00,466.17)(0.500,1.000){2}{\rule{0.120pt}{0.400pt}}
\put(401.0,463.0){\rule[-0.200pt]{0.400pt}{0.482pt}}
\put(402.67,470){\rule{0.400pt}{0.482pt}}
\multiput(402.17,470.00)(1.000,1.000){2}{\rule{0.400pt}{0.241pt}}
\put(403.0,468.0){\rule[-0.200pt]{0.400pt}{0.482pt}}
\put(403.67,470){\rule{0.400pt}{0.723pt}}
\multiput(403.17,471.50)(1.000,-1.500){2}{\rule{0.400pt}{0.361pt}}
\put(404.67,470){\rule{0.400pt}{1.927pt}}
\multiput(404.17,470.00)(1.000,4.000){2}{\rule{0.400pt}{0.964pt}}
\put(404.0,472.0){\usebox{\plotpoint}}
\put(405.67,477){\rule{0.400pt}{0.482pt}}
\multiput(405.17,477.00)(1.000,1.000){2}{\rule{0.400pt}{0.241pt}}
\put(406.0,477.0){\usebox{\plotpoint}}
\put(407.0,479.0){\usebox{\plotpoint}}
\put(408,481.67){\rule{0.241pt}{0.400pt}}
\multiput(408.00,481.17)(0.500,1.000){2}{\rule{0.120pt}{0.400pt}}
\put(408.67,483){\rule{0.400pt}{0.964pt}}
\multiput(408.17,483.00)(1.000,2.000){2}{\rule{0.400pt}{0.482pt}}
\put(408.0,479.0){\rule[-0.200pt]{0.400pt}{0.723pt}}
\put(409.67,487){\rule{0.400pt}{0.482pt}}
\multiput(409.17,488.00)(1.000,-1.000){2}{\rule{0.400pt}{0.241pt}}
\put(410.67,487){\rule{0.400pt}{0.482pt}}
\multiput(410.17,487.00)(1.000,1.000){2}{\rule{0.400pt}{0.241pt}}
\put(410.0,487.0){\rule[-0.200pt]{0.400pt}{0.482pt}}
\put(411.67,490){\rule{0.400pt}{0.723pt}}
\multiput(411.17,490.00)(1.000,1.500){2}{\rule{0.400pt}{0.361pt}}
\put(413,492.67){\rule{0.241pt}{0.400pt}}
\multiput(413.00,492.17)(0.500,1.000){2}{\rule{0.120pt}{0.400pt}}
\put(412.0,489.0){\usebox{\plotpoint}}
\put(413.67,496){\rule{0.400pt}{0.723pt}}
\multiput(413.17,496.00)(1.000,1.500){2}{\rule{0.400pt}{0.361pt}}
\put(415,498.67){\rule{0.241pt}{0.400pt}}
\multiput(415.00,498.17)(0.500,1.000){2}{\rule{0.120pt}{0.400pt}}
\put(414.0,494.0){\rule[-0.200pt]{0.400pt}{0.482pt}}
\put(415.67,501){\rule{0.400pt}{0.723pt}}
\multiput(415.17,501.00)(1.000,1.500){2}{\rule{0.400pt}{0.361pt}}
\put(416.67,502){\rule{0.400pt}{0.482pt}}
\multiput(416.17,503.00)(1.000,-1.000){2}{\rule{0.400pt}{0.241pt}}
\put(416.0,500.0){\usebox{\plotpoint}}
\put(417.67,504){\rule{0.400pt}{0.482pt}}
\multiput(417.17,504.00)(1.000,1.000){2}{\rule{0.400pt}{0.241pt}}
\put(419,505.67){\rule{0.241pt}{0.400pt}}
\multiput(419.00,505.17)(0.500,1.000){2}{\rule{0.120pt}{0.400pt}}
\put(418.0,502.0){\rule[-0.200pt]{0.400pt}{0.482pt}}
\put(420.0,507.0){\rule[-0.200pt]{0.400pt}{0.964pt}}
\put(420.67,511){\rule{0.400pt}{0.964pt}}
\multiput(420.17,511.00)(1.000,2.000){2}{\rule{0.400pt}{0.482pt}}
\put(420.0,511.0){\usebox{\plotpoint}}
\put(422,514.67){\rule{0.241pt}{0.400pt}}
\multiput(422.00,515.17)(0.500,-1.000){2}{\rule{0.120pt}{0.400pt}}
\put(422.67,515){\rule{0.400pt}{0.482pt}}
\multiput(422.17,515.00)(1.000,1.000){2}{\rule{0.400pt}{0.241pt}}
\put(422.0,515.0){\usebox{\plotpoint}}
\put(424,518.67){\rule{0.241pt}{0.400pt}}
\multiput(424.00,519.17)(0.500,-1.000){2}{\rule{0.120pt}{0.400pt}}
\put(424.67,519){\rule{0.400pt}{1.204pt}}
\multiput(424.17,519.00)(1.000,2.500){2}{\rule{0.400pt}{0.602pt}}
\put(424.0,517.0){\rule[-0.200pt]{0.400pt}{0.723pt}}
\put(425.67,521){\rule{0.400pt}{0.964pt}}
\multiput(425.17,521.00)(1.000,2.000){2}{\rule{0.400pt}{0.482pt}}
\put(426.67,525){\rule{0.400pt}{0.482pt}}
\multiput(426.17,525.00)(1.000,1.000){2}{\rule{0.400pt}{0.241pt}}
\put(426.0,521.0){\rule[-0.200pt]{0.400pt}{0.723pt}}
\put(428.0,527.0){\rule[-0.200pt]{0.400pt}{0.482pt}}
\put(428.67,529){\rule{0.400pt}{0.964pt}}
\multiput(428.17,529.00)(1.000,2.000){2}{\rule{0.400pt}{0.482pt}}
\put(428.0,529.0){\usebox{\plotpoint}}
\put(430,534.67){\rule{0.241pt}{0.400pt}}
\multiput(430.00,534.17)(0.500,1.000){2}{\rule{0.120pt}{0.400pt}}
\put(430.67,532){\rule{0.400pt}{0.964pt}}
\multiput(430.17,534.00)(1.000,-2.000){2}{\rule{0.400pt}{0.482pt}}
\put(430.0,533.0){\rule[-0.200pt]{0.400pt}{0.482pt}}
\put(431.67,537){\rule{0.400pt}{0.482pt}}
\multiput(431.17,537.00)(1.000,1.000){2}{\rule{0.400pt}{0.241pt}}
\put(432.0,532.0){\rule[-0.200pt]{0.400pt}{1.204pt}}
\put(432.67,541){\rule{0.400pt}{0.723pt}}
\multiput(432.17,541.00)(1.000,1.500){2}{\rule{0.400pt}{0.361pt}}
\put(433.0,539.0){\rule[-0.200pt]{0.400pt}{0.482pt}}
\put(434.0,544.0){\usebox{\plotpoint}}
\put(435.0,544.0){\rule[-0.200pt]{0.400pt}{0.482pt}}
\put(435.67,546){\rule{0.400pt}{0.482pt}}
\multiput(435.17,546.00)(1.000,1.000){2}{\rule{0.400pt}{0.241pt}}
\put(435.0,546.0){\usebox{\plotpoint}}
\put(437,548){\usebox{\plotpoint}}
\put(436.67,548){\rule{0.400pt}{0.964pt}}
\multiput(436.17,548.00)(1.000,2.000){2}{\rule{0.400pt}{0.482pt}}
\put(438.0,552.0){\usebox{\plotpoint}}
\put(438.67,551){\rule{0.400pt}{0.964pt}}
\multiput(438.17,551.00)(1.000,2.000){2}{\rule{0.400pt}{0.482pt}}
\put(439.0,551.0){\usebox{\plotpoint}}
\put(440.0,555.0){\usebox{\plotpoint}}
\put(440.67,557){\rule{0.400pt}{0.723pt}}
\multiput(440.17,557.00)(1.000,1.500){2}{\rule{0.400pt}{0.361pt}}
\put(441.67,560){\rule{0.400pt}{0.482pt}}
\multiput(441.17,560.00)(1.000,1.000){2}{\rule{0.400pt}{0.241pt}}
\put(441.0,555.0){\rule[-0.200pt]{0.400pt}{0.482pt}}
\put(442.67,560){\rule{0.400pt}{0.482pt}}
\multiput(442.17,560.00)(1.000,1.000){2}{\rule{0.400pt}{0.241pt}}
\put(444,560.67){\rule{0.241pt}{0.400pt}}
\multiput(444.00,561.17)(0.500,-1.000){2}{\rule{0.120pt}{0.400pt}}
\put(443.0,560.0){\rule[-0.200pt]{0.400pt}{0.482pt}}
\put(445,565.67){\rule{0.241pt}{0.400pt}}
\multiput(445.00,566.17)(0.500,-1.000){2}{\rule{0.120pt}{0.400pt}}
\put(445.67,566){\rule{0.400pt}{0.482pt}}
\multiput(445.17,566.00)(1.000,1.000){2}{\rule{0.400pt}{0.241pt}}
\put(445.0,561.0){\rule[-0.200pt]{0.400pt}{1.445pt}}
\put(447.0,568.0){\rule[-0.200pt]{0.400pt}{0.482pt}}
\put(447.0,570.0){\rule[-0.200pt]{0.482pt}{0.400pt}}
\put(448.67,573){\rule{0.400pt}{0.482pt}}
\multiput(448.17,573.00)(1.000,1.000){2}{\rule{0.400pt}{0.241pt}}
\put(450,574.67){\rule{0.241pt}{0.400pt}}
\multiput(450.00,574.17)(0.500,1.000){2}{\rule{0.120pt}{0.400pt}}
\put(449.0,570.0){\rule[-0.200pt]{0.400pt}{0.723pt}}
\put(450.67,576){\rule{0.400pt}{0.482pt}}
\multiput(450.17,577.00)(1.000,-1.000){2}{\rule{0.400pt}{0.241pt}}
\put(451.67,576){\rule{0.400pt}{0.723pt}}
\multiput(451.17,576.00)(1.000,1.500){2}{\rule{0.400pt}{0.361pt}}
\put(451.0,576.0){\rule[-0.200pt]{0.400pt}{0.482pt}}
\put(452.67,579){\rule{0.400pt}{0.482pt}}
\multiput(452.17,580.00)(1.000,-1.000){2}{\rule{0.400pt}{0.241pt}}
\put(453.0,579.0){\rule[-0.200pt]{0.400pt}{0.482pt}}
\put(454.0,579.0){\usebox{\plotpoint}}
\put(454.67,581){\rule{0.400pt}{0.723pt}}
\multiput(454.17,581.00)(1.000,1.500){2}{\rule{0.400pt}{0.361pt}}
\put(456,582.67){\rule{0.241pt}{0.400pt}}
\multiput(456.00,583.17)(0.500,-1.000){2}{\rule{0.120pt}{0.400pt}}
\put(455.0,579.0){\rule[-0.200pt]{0.400pt}{0.482pt}}
\put(457.0,583.0){\rule[-0.200pt]{0.400pt}{0.723pt}}
\put(457.0,586.0){\usebox{\plotpoint}}
\put(458,585.67){\rule{0.241pt}{0.400pt}}
\multiput(458.00,586.17)(0.500,-1.000){2}{\rule{0.120pt}{0.400pt}}
\put(458.67,586){\rule{0.400pt}{1.445pt}}
\multiput(458.17,586.00)(1.000,3.000){2}{\rule{0.400pt}{0.723pt}}
\put(458.0,586.0){\usebox{\plotpoint}}
\put(460,592){\usebox{\plotpoint}}
\put(459.67,590){\rule{0.400pt}{0.482pt}}
\multiput(459.17,591.00)(1.000,-1.000){2}{\rule{0.400pt}{0.241pt}}
\put(460.67,590){\rule{0.400pt}{1.445pt}}
\multiput(460.17,590.00)(1.000,3.000){2}{\rule{0.400pt}{0.723pt}}
\put(462,596){\usebox{\plotpoint}}
\put(461.67,592){\rule{0.400pt}{0.964pt}}
\multiput(461.17,594.00)(1.000,-2.000){2}{\rule{0.400pt}{0.482pt}}
\put(462.67,592){\rule{0.400pt}{1.686pt}}
\multiput(462.17,592.00)(1.000,3.500){2}{\rule{0.400pt}{0.843pt}}
\put(463.67,594){\rule{0.400pt}{0.723pt}}
\multiput(463.17,595.50)(1.000,-1.500){2}{\rule{0.400pt}{0.361pt}}
\put(464.67,594){\rule{0.400pt}{1.204pt}}
\multiput(464.17,594.00)(1.000,2.500){2}{\rule{0.400pt}{0.602pt}}
\put(464.0,597.0){\rule[-0.200pt]{0.400pt}{0.482pt}}
\put(465.67,598){\rule{0.400pt}{0.482pt}}
\multiput(465.17,599.00)(1.000,-1.000){2}{\rule{0.400pt}{0.241pt}}
\put(466.67,598){\rule{0.400pt}{1.204pt}}
\multiput(466.17,598.00)(1.000,2.500){2}{\rule{0.400pt}{0.602pt}}
\put(466.0,599.0){\usebox{\plotpoint}}
\put(468,603){\usebox{\plotpoint}}
\put(467.67,598){\rule{0.400pt}{1.204pt}}
\multiput(467.17,600.50)(1.000,-2.500){2}{\rule{0.400pt}{0.602pt}}
\put(468.67,598){\rule{0.400pt}{1.445pt}}
\multiput(468.17,598.00)(1.000,3.000){2}{\rule{0.400pt}{0.723pt}}
\put(470,599.67){\rule{0.241pt}{0.400pt}}
\multiput(470.00,600.17)(0.500,-1.000){2}{\rule{0.120pt}{0.400pt}}
\put(470.67,600){\rule{0.400pt}{0.723pt}}
\multiput(470.17,600.00)(1.000,1.500){2}{\rule{0.400pt}{0.361pt}}
\put(470.0,601.0){\rule[-0.200pt]{0.400pt}{0.723pt}}
\put(471.67,605){\rule{0.400pt}{0.482pt}}
\multiput(471.17,605.00)(1.000,1.000){2}{\rule{0.400pt}{0.241pt}}
\put(472.0,603.0){\rule[-0.200pt]{0.400pt}{0.482pt}}
\put(473.0,607.0){\usebox{\plotpoint}}
\put(473.67,605){\rule{0.400pt}{0.723pt}}
\multiput(473.17,605.00)(1.000,1.500){2}{\rule{0.400pt}{0.361pt}}
\put(474.67,606){\rule{0.400pt}{0.482pt}}
\multiput(474.17,607.00)(1.000,-1.000){2}{\rule{0.400pt}{0.241pt}}
\put(474.0,605.0){\rule[-0.200pt]{0.400pt}{0.482pt}}
\put(475.67,606){\rule{0.400pt}{0.723pt}}
\multiput(475.17,607.50)(1.000,-1.500){2}{\rule{0.400pt}{0.361pt}}
\put(476.0,606.0){\rule[-0.200pt]{0.400pt}{0.723pt}}
\put(477.0,606.0){\usebox{\plotpoint}}
\put(478,610.67){\rule{0.241pt}{0.400pt}}
\multiput(478.00,611.17)(0.500,-1.000){2}{\rule{0.120pt}{0.400pt}}
\put(478.0,606.0){\rule[-0.200pt]{0.400pt}{1.445pt}}
\put(479,610.67){\rule{0.241pt}{0.400pt}}
\multiput(479.00,611.17)(0.500,-1.000){2}{\rule{0.120pt}{0.400pt}}
\put(479.67,609){\rule{0.400pt}{0.482pt}}
\multiput(479.17,610.00)(1.000,-1.000){2}{\rule{0.400pt}{0.241pt}}
\put(479.0,611.0){\usebox{\plotpoint}}
\put(481.0,609.0){\rule[-0.200pt]{0.400pt}{0.482pt}}
\put(481.67,607){\rule{0.400pt}{0.964pt}}
\multiput(481.17,609.00)(1.000,-2.000){2}{\rule{0.400pt}{0.482pt}}
\put(481.0,611.0){\usebox{\plotpoint}}
\put(483,610.67){\rule{0.241pt}{0.400pt}}
\multiput(483.00,611.17)(0.500,-1.000){2}{\rule{0.120pt}{0.400pt}}
\put(483.0,607.0){\rule[-0.200pt]{0.400pt}{1.204pt}}
\put(484.0,611.0){\usebox{\plotpoint}}
\put(484.67,609){\rule{0.400pt}{0.723pt}}
\multiput(484.17,609.00)(1.000,1.500){2}{\rule{0.400pt}{0.361pt}}
\put(485.67,610){\rule{0.400pt}{0.482pt}}
\multiput(485.17,611.00)(1.000,-1.000){2}{\rule{0.400pt}{0.241pt}}
\put(485.0,609.0){\rule[-0.200pt]{0.400pt}{0.482pt}}
\put(486.67,608){\rule{0.400pt}{0.964pt}}
\multiput(486.17,610.00)(1.000,-2.000){2}{\rule{0.400pt}{0.482pt}}
\put(487.67,605){\rule{0.400pt}{0.723pt}}
\multiput(487.17,606.50)(1.000,-1.500){2}{\rule{0.400pt}{0.361pt}}
\put(487.0,610.0){\rule[-0.200pt]{0.400pt}{0.482pt}}
\put(488.67,610){\rule{0.400pt}{0.964pt}}
\multiput(488.17,612.00)(1.000,-2.000){2}{\rule{0.400pt}{0.482pt}}
\put(489.67,610){\rule{0.400pt}{0.723pt}}
\multiput(489.17,610.00)(1.000,1.500){2}{\rule{0.400pt}{0.361pt}}
\put(489.0,605.0){\rule[-0.200pt]{0.400pt}{2.168pt}}
\put(490.67,606){\rule{0.400pt}{0.964pt}}
\multiput(490.17,608.00)(1.000,-2.000){2}{\rule{0.400pt}{0.482pt}}
\put(491.67,604){\rule{0.400pt}{0.482pt}}
\multiput(491.17,605.00)(1.000,-1.000){2}{\rule{0.400pt}{0.241pt}}
\put(491.0,610.0){\rule[-0.200pt]{0.400pt}{0.723pt}}
\put(492.67,608){\rule{0.400pt}{0.964pt}}
\multiput(492.17,608.00)(1.000,2.000){2}{\rule{0.400pt}{0.482pt}}
\put(494,610.67){\rule{0.241pt}{0.400pt}}
\multiput(494.00,611.17)(0.500,-1.000){2}{\rule{0.120pt}{0.400pt}}
\put(493.0,604.0){\rule[-0.200pt]{0.400pt}{0.964pt}}
\put(495,609.67){\rule{0.241pt}{0.400pt}}
\multiput(495.00,609.17)(0.500,1.000){2}{\rule{0.120pt}{0.400pt}}
\put(496,609.67){\rule{0.241pt}{0.400pt}}
\multiput(496.00,610.17)(0.500,-1.000){2}{\rule{0.120pt}{0.400pt}}
\put(495.0,610.0){\usebox{\plotpoint}}
\put(496.67,609){\rule{0.400pt}{0.723pt}}
\multiput(496.17,609.00)(1.000,1.500){2}{\rule{0.400pt}{0.361pt}}
\put(497.0,609.0){\usebox{\plotpoint}}
\put(498,612){\usebox{\plotpoint}}
\put(497.67,604){\rule{0.400pt}{1.927pt}}
\multiput(497.17,608.00)(1.000,-4.000){2}{\rule{0.400pt}{0.964pt}}
\put(499,603.67){\rule{0.241pt}{0.400pt}}
\multiput(499.00,603.17)(0.500,1.000){2}{\rule{0.120pt}{0.400pt}}
\put(499.67,606){\rule{0.400pt}{1.204pt}}
\multiput(499.17,606.00)(1.000,2.500){2}{\rule{0.400pt}{0.602pt}}
\put(500.67,606){\rule{0.400pt}{1.204pt}}
\multiput(500.17,608.50)(1.000,-2.500){2}{\rule{0.400pt}{0.602pt}}
\put(500.0,605.0){\usebox{\plotpoint}}
\put(501.67,602){\rule{0.400pt}{0.964pt}}
\multiput(501.17,602.00)(1.000,2.000){2}{\rule{0.400pt}{0.482pt}}
\put(502.0,602.0){\rule[-0.200pt]{0.400pt}{0.964pt}}
\put(503.0,606.0){\usebox{\plotpoint}}
\put(503.67,605){\rule{0.400pt}{0.964pt}}
\multiput(503.17,607.00)(1.000,-2.000){2}{\rule{0.400pt}{0.482pt}}
\put(504.67,605){\rule{0.400pt}{0.723pt}}
\multiput(504.17,605.00)(1.000,1.500){2}{\rule{0.400pt}{0.361pt}}
\put(504.0,606.0){\rule[-0.200pt]{0.400pt}{0.723pt}}
\put(506,605.67){\rule{0.241pt}{0.400pt}}
\multiput(506.00,605.17)(0.500,1.000){2}{\rule{0.120pt}{0.400pt}}
\put(506.67,605){\rule{0.400pt}{0.482pt}}
\multiput(506.17,606.00)(1.000,-1.000){2}{\rule{0.400pt}{0.241pt}}
\put(506.0,606.0){\rule[-0.200pt]{0.400pt}{0.482pt}}
\put(507.67,605){\rule{0.400pt}{0.482pt}}
\multiput(507.17,606.00)(1.000,-1.000){2}{\rule{0.400pt}{0.241pt}}
\put(509,604.67){\rule{0.241pt}{0.400pt}}
\multiput(509.00,604.17)(0.500,1.000){2}{\rule{0.120pt}{0.400pt}}
\put(508.0,605.0){\rule[-0.200pt]{0.400pt}{0.482pt}}
\put(509.67,601){\rule{0.400pt}{0.723pt}}
\multiput(509.17,601.00)(1.000,1.500){2}{\rule{0.400pt}{0.361pt}}
\put(510.67,602){\rule{0.400pt}{0.482pt}}
\multiput(510.17,603.00)(1.000,-1.000){2}{\rule{0.400pt}{0.241pt}}
\put(510.0,601.0){\rule[-0.200pt]{0.400pt}{1.204pt}}
\put(511.67,601){\rule{0.400pt}{0.482pt}}
\multiput(511.17,601.00)(1.000,1.000){2}{\rule{0.400pt}{0.241pt}}
\put(512.67,603){\rule{0.400pt}{0.482pt}}
\multiput(512.17,603.00)(1.000,1.000){2}{\rule{0.400pt}{0.241pt}}
\put(512.0,601.0){\usebox{\plotpoint}}
\put(514,601.67){\rule{0.241pt}{0.400pt}}
\multiput(514.00,602.17)(0.500,-1.000){2}{\rule{0.120pt}{0.400pt}}
\put(514.0,603.0){\rule[-0.200pt]{0.400pt}{0.482pt}}
\put(514.67,598){\rule{0.400pt}{1.204pt}}
\multiput(514.17,600.50)(1.000,-2.500){2}{\rule{0.400pt}{0.602pt}}
\put(515.67,595){\rule{0.400pt}{0.723pt}}
\multiput(515.17,596.50)(1.000,-1.500){2}{\rule{0.400pt}{0.361pt}}
\put(515.0,602.0){\usebox{\plotpoint}}
\put(516.67,595){\rule{0.400pt}{0.723pt}}
\multiput(516.17,596.50)(1.000,-1.500){2}{\rule{0.400pt}{0.361pt}}
\put(518,594.67){\rule{0.241pt}{0.400pt}}
\multiput(518.00,594.17)(0.500,1.000){2}{\rule{0.120pt}{0.400pt}}
\put(517.0,595.0){\rule[-0.200pt]{0.400pt}{0.723pt}}
\put(518.67,592){\rule{0.400pt}{1.445pt}}
\multiput(518.17,595.00)(1.000,-3.000){2}{\rule{0.400pt}{0.723pt}}
\put(520,591.67){\rule{0.241pt}{0.400pt}}
\multiput(520.00,591.17)(0.500,1.000){2}{\rule{0.120pt}{0.400pt}}
\put(519.0,596.0){\rule[-0.200pt]{0.400pt}{0.482pt}}
\put(521.0,593.0){\rule[-0.200pt]{0.400pt}{0.482pt}}
\put(521.67,592){\rule{0.400pt}{0.723pt}}
\multiput(521.17,593.50)(1.000,-1.500){2}{\rule{0.400pt}{0.361pt}}
\put(521.0,595.0){\usebox{\plotpoint}}
\put(522.67,591){\rule{0.400pt}{0.482pt}}
\multiput(522.17,592.00)(1.000,-1.000){2}{\rule{0.400pt}{0.241pt}}
\put(524,590.67){\rule{0.241pt}{0.400pt}}
\multiput(524.00,590.17)(0.500,1.000){2}{\rule{0.120pt}{0.400pt}}
\put(523.0,592.0){\usebox{\plotpoint}}
\put(524.67,587){\rule{0.400pt}{0.723pt}}
\multiput(524.17,587.00)(1.000,1.500){2}{\rule{0.400pt}{0.361pt}}
\put(525.67,588){\rule{0.400pt}{0.482pt}}
\multiput(525.17,589.00)(1.000,-1.000){2}{\rule{0.400pt}{0.241pt}}
\put(525.0,587.0){\rule[-0.200pt]{0.400pt}{1.204pt}}
\put(527.0,588.0){\usebox{\plotpoint}}
\put(528,587.67){\rule{0.241pt}{0.400pt}}
\multiput(528.00,588.17)(0.500,-1.000){2}{\rule{0.120pt}{0.400pt}}
\put(527.0,589.0){\usebox{\plotpoint}}
\put(529,583.67){\rule{0.241pt}{0.400pt}}
\multiput(529.00,583.17)(0.500,1.000){2}{\rule{0.120pt}{0.400pt}}
\put(529.0,584.0){\rule[-0.200pt]{0.400pt}{0.964pt}}
\put(529.67,583){\rule{0.400pt}{1.686pt}}
\multiput(529.17,583.00)(1.000,3.500){2}{\rule{0.400pt}{0.843pt}}
\put(530.67,585){\rule{0.400pt}{1.204pt}}
\multiput(530.17,587.50)(1.000,-2.500){2}{\rule{0.400pt}{0.602pt}}
\put(530.0,583.0){\rule[-0.200pt]{0.400pt}{0.482pt}}
\put(531.67,579){\rule{0.400pt}{1.686pt}}
\multiput(531.17,579.00)(1.000,3.500){2}{\rule{0.400pt}{0.843pt}}
\put(532.67,582){\rule{0.400pt}{0.964pt}}
\multiput(532.17,584.00)(1.000,-2.000){2}{\rule{0.400pt}{0.482pt}}
\put(532.0,579.0){\rule[-0.200pt]{0.400pt}{1.445pt}}
\put(533.67,579){\rule{0.400pt}{0.964pt}}
\multiput(533.17,579.00)(1.000,2.000){2}{\rule{0.400pt}{0.482pt}}
\put(534.67,579){\rule{0.400pt}{0.964pt}}
\multiput(534.17,581.00)(1.000,-2.000){2}{\rule{0.400pt}{0.482pt}}
\put(534.0,579.0){\rule[-0.200pt]{0.400pt}{0.723pt}}
\put(535.67,577){\rule{0.400pt}{1.445pt}}
\multiput(535.17,577.00)(1.000,3.000){2}{\rule{0.400pt}{0.723pt}}
\put(536.0,577.0){\rule[-0.200pt]{0.400pt}{0.482pt}}
\put(537.0,583.0){\usebox{\plotpoint}}
\put(537.67,573){\rule{0.400pt}{1.204pt}}
\multiput(537.17,575.50)(1.000,-2.500){2}{\rule{0.400pt}{0.602pt}}
\put(538.67,573){\rule{0.400pt}{0.482pt}}
\multiput(538.17,573.00)(1.000,1.000){2}{\rule{0.400pt}{0.241pt}}
\put(538.0,578.0){\rule[-0.200pt]{0.400pt}{1.204pt}}
\put(540,570.67){\rule{0.241pt}{0.400pt}}
\multiput(540.00,570.17)(0.500,1.000){2}{\rule{0.120pt}{0.400pt}}
\put(541,570.67){\rule{0.241pt}{0.400pt}}
\multiput(541.00,571.17)(0.500,-1.000){2}{\rule{0.120pt}{0.400pt}}
\put(540.0,571.0){\rule[-0.200pt]{0.400pt}{0.964pt}}
\put(541.67,573){\rule{0.400pt}{1.204pt}}
\multiput(541.17,575.50)(1.000,-2.500){2}{\rule{0.400pt}{0.602pt}}
\put(542.67,573){\rule{0.400pt}{0.723pt}}
\multiput(542.17,573.00)(1.000,1.500){2}{\rule{0.400pt}{0.361pt}}
\put(542.0,571.0){\rule[-0.200pt]{0.400pt}{1.686pt}}
\put(543.67,572){\rule{0.400pt}{0.723pt}}
\multiput(543.17,572.00)(1.000,1.500){2}{\rule{0.400pt}{0.361pt}}
\put(544.0,572.0){\rule[-0.200pt]{0.400pt}{0.964pt}}
\put(545,575){\usebox{\plotpoint}}
\put(544.67,572){\rule{0.400pt}{0.723pt}}
\multiput(544.17,573.50)(1.000,-1.500){2}{\rule{0.400pt}{0.361pt}}
\put(545.67,572){\rule{0.400pt}{0.482pt}}
\multiput(545.17,572.00)(1.000,1.000){2}{\rule{0.400pt}{0.241pt}}
\put(546.67,571){\rule{0.400pt}{0.482pt}}
\multiput(546.17,572.00)(1.000,-1.000){2}{\rule{0.400pt}{0.241pt}}
\put(547.0,573.0){\usebox{\plotpoint}}
\put(548.0,571.0){\usebox{\plotpoint}}
\put(548.67,569){\rule{0.400pt}{0.964pt}}
\multiput(548.17,571.00)(1.000,-2.000){2}{\rule{0.400pt}{0.482pt}}
\put(550,567.67){\rule{0.241pt}{0.400pt}}
\multiput(550.00,568.17)(0.500,-1.000){2}{\rule{0.120pt}{0.400pt}}
\put(549.0,571.0){\rule[-0.200pt]{0.400pt}{0.482pt}}
\put(551,565.67){\rule{0.241pt}{0.400pt}}
\multiput(551.00,566.17)(0.500,-1.000){2}{\rule{0.120pt}{0.400pt}}
\put(552,565.67){\rule{0.241pt}{0.400pt}}
\multiput(552.00,565.17)(0.500,1.000){2}{\rule{0.120pt}{0.400pt}}
\put(551.0,567.0){\usebox{\plotpoint}}
\put(552.67,563){\rule{0.400pt}{0.482pt}}
\multiput(552.17,564.00)(1.000,-1.000){2}{\rule{0.400pt}{0.241pt}}
\put(553.67,559){\rule{0.400pt}{0.964pt}}
\multiput(553.17,561.00)(1.000,-2.000){2}{\rule{0.400pt}{0.482pt}}
\put(553.0,565.0){\rule[-0.200pt]{0.400pt}{0.482pt}}
\put(554.67,560){\rule{0.400pt}{0.723pt}}
\multiput(554.17,561.50)(1.000,-1.500){2}{\rule{0.400pt}{0.361pt}}
\put(555.67,560){\rule{0.400pt}{0.482pt}}
\multiput(555.17,560.00)(1.000,1.000){2}{\rule{0.400pt}{0.241pt}}
\put(555.0,559.0){\rule[-0.200pt]{0.400pt}{0.964pt}}
\put(556.67,563){\rule{0.400pt}{0.482pt}}
\multiput(556.17,564.00)(1.000,-1.000){2}{\rule{0.400pt}{0.241pt}}
\put(557.0,562.0){\rule[-0.200pt]{0.400pt}{0.723pt}}
\put(558.0,563.0){\usebox{\plotpoint}}
\put(558.67,556){\rule{0.400pt}{0.964pt}}
\multiput(558.17,556.00)(1.000,2.000){2}{\rule{0.400pt}{0.482pt}}
\put(559.0,556.0){\rule[-0.200pt]{0.400pt}{1.686pt}}
\put(560.0,559.0){\usebox{\plotpoint}}
\put(560.0,559.0){\rule[-0.200pt]{0.482pt}{0.400pt}}
\put(561.67,554){\rule{0.400pt}{0.964pt}}
\multiput(561.17,554.00)(1.000,2.000){2}{\rule{0.400pt}{0.482pt}}
\put(563,556.67){\rule{0.241pt}{0.400pt}}
\multiput(563.00,557.17)(0.500,-1.000){2}{\rule{0.120pt}{0.400pt}}
\put(562.0,554.0){\rule[-0.200pt]{0.400pt}{1.204pt}}
\put(563.67,547){\rule{0.400pt}{2.168pt}}
\multiput(563.17,547.00)(1.000,4.500){2}{\rule{0.400pt}{1.084pt}}
\put(565,554.67){\rule{0.241pt}{0.400pt}}
\multiput(565.00,555.17)(0.500,-1.000){2}{\rule{0.120pt}{0.400pt}}
\put(564.0,547.0){\rule[-0.200pt]{0.400pt}{2.409pt}}
\put(566,552.67){\rule{0.241pt}{0.400pt}}
\multiput(566.00,553.17)(0.500,-1.000){2}{\rule{0.120pt}{0.400pt}}
\put(567,551.67){\rule{0.241pt}{0.400pt}}
\multiput(567.00,552.17)(0.500,-1.000){2}{\rule{0.120pt}{0.400pt}}
\put(566.0,554.0){\usebox{\plotpoint}}
\put(568,549.67){\rule{0.241pt}{0.400pt}}
\multiput(568.00,550.17)(0.500,-1.000){2}{\rule{0.120pt}{0.400pt}}
\put(569,549.67){\rule{0.241pt}{0.400pt}}
\multiput(569.00,549.17)(0.500,1.000){2}{\rule{0.120pt}{0.400pt}}
\put(568.0,551.0){\usebox{\plotpoint}}
\put(569.67,550){\rule{0.400pt}{0.964pt}}
\multiput(569.17,550.00)(1.000,2.000){2}{\rule{0.400pt}{0.482pt}}
\put(570.67,546){\rule{0.400pt}{1.927pt}}
\multiput(570.17,550.00)(1.000,-4.000){2}{\rule{0.400pt}{0.964pt}}
\put(570.0,550.0){\usebox{\plotpoint}}
\put(572.0,546.0){\rule[-0.200pt]{0.400pt}{1.445pt}}
\put(572.0,552.0){\usebox{\plotpoint}}
\put(572.67,545){\rule{0.400pt}{0.964pt}}
\multiput(572.17,545.00)(1.000,2.000){2}{\rule{0.400pt}{0.482pt}}
\put(573.67,541){\rule{0.400pt}{1.927pt}}
\multiput(573.17,545.00)(1.000,-4.000){2}{\rule{0.400pt}{0.964pt}}
\put(573.0,545.0){\rule[-0.200pt]{0.400pt}{1.686pt}}
\put(574.67,545){\rule{0.400pt}{0.964pt}}
\multiput(574.17,547.00)(1.000,-2.000){2}{\rule{0.400pt}{0.482pt}}
\put(575.67,545){\rule{0.400pt}{1.204pt}}
\multiput(575.17,545.00)(1.000,2.500){2}{\rule{0.400pt}{0.602pt}}
\put(575.0,541.0){\rule[-0.200pt]{0.400pt}{1.927pt}}
\put(576.67,539){\rule{0.400pt}{1.445pt}}
\multiput(576.17,539.00)(1.000,3.000){2}{\rule{0.400pt}{0.723pt}}
\put(577.67,538){\rule{0.400pt}{1.686pt}}
\multiput(577.17,541.50)(1.000,-3.500){2}{\rule{0.400pt}{0.843pt}}
\put(577.0,539.0){\rule[-0.200pt]{0.400pt}{2.650pt}}
\put(579,539.67){\rule{0.241pt}{0.400pt}}
\multiput(579.00,540.17)(0.500,-1.000){2}{\rule{0.120pt}{0.400pt}}
\put(579.67,536){\rule{0.400pt}{0.964pt}}
\multiput(579.17,538.00)(1.000,-2.000){2}{\rule{0.400pt}{0.482pt}}
\put(579.0,538.0){\rule[-0.200pt]{0.400pt}{0.723pt}}
\put(580.67,533){\rule{0.400pt}{0.964pt}}
\multiput(580.17,533.00)(1.000,2.000){2}{\rule{0.400pt}{0.482pt}}
\put(581.67,537){\rule{0.400pt}{1.686pt}}
\multiput(581.17,537.00)(1.000,3.500){2}{\rule{0.400pt}{0.843pt}}
\put(581.0,533.0){\rule[-0.200pt]{0.400pt}{0.723pt}}
\put(583.0,538.0){\rule[-0.200pt]{0.400pt}{1.445pt}}
\put(583.0,538.0){\usebox{\plotpoint}}
\put(584,534.67){\rule{0.241pt}{0.400pt}}
\multiput(584.00,535.17)(0.500,-1.000){2}{\rule{0.120pt}{0.400pt}}
\put(585,534.67){\rule{0.241pt}{0.400pt}}
\multiput(585.00,534.17)(0.500,1.000){2}{\rule{0.120pt}{0.400pt}}
\put(584.0,536.0){\rule[-0.200pt]{0.400pt}{0.482pt}}
\put(586,529.67){\rule{0.241pt}{0.400pt}}
\multiput(586.00,530.17)(0.500,-1.000){2}{\rule{0.120pt}{0.400pt}}
\put(586.67,530){\rule{0.400pt}{0.482pt}}
\multiput(586.17,530.00)(1.000,1.000){2}{\rule{0.400pt}{0.241pt}}
\put(586.0,531.0){\rule[-0.200pt]{0.400pt}{1.204pt}}
\put(587.67,530){\rule{0.400pt}{0.723pt}}
\multiput(587.17,530.00)(1.000,1.500){2}{\rule{0.400pt}{0.361pt}}
\put(588.67,530){\rule{0.400pt}{0.723pt}}
\multiput(588.17,531.50)(1.000,-1.500){2}{\rule{0.400pt}{0.361pt}}
\put(588.0,530.0){\rule[-0.200pt]{0.400pt}{0.482pt}}
\put(590,530){\usebox{\plotpoint}}
\put(589.67,526){\rule{0.400pt}{0.964pt}}
\multiput(589.17,528.00)(1.000,-2.000){2}{\rule{0.400pt}{0.482pt}}
\put(590.67,526){\rule{0.400pt}{0.723pt}}
\multiput(590.17,526.00)(1.000,1.500){2}{\rule{0.400pt}{0.361pt}}
\put(591.67,527){\rule{0.400pt}{0.964pt}}
\multiput(591.17,529.00)(1.000,-2.000){2}{\rule{0.400pt}{0.482pt}}
\put(592.67,525){\rule{0.400pt}{0.482pt}}
\multiput(592.17,526.00)(1.000,-1.000){2}{\rule{0.400pt}{0.241pt}}
\put(592.0,529.0){\rule[-0.200pt]{0.400pt}{0.482pt}}
\put(593.67,521){\rule{0.400pt}{0.482pt}}
\multiput(593.17,521.00)(1.000,1.000){2}{\rule{0.400pt}{0.241pt}}
\put(594.0,521.0){\rule[-0.200pt]{0.400pt}{0.964pt}}
\put(596,521.67){\rule{0.241pt}{0.400pt}}
\multiput(596.00,522.17)(0.500,-1.000){2}{\rule{0.120pt}{0.400pt}}
\put(595.0,523.0){\usebox{\plotpoint}}
\put(596.67,521){\rule{0.400pt}{0.482pt}}
\multiput(596.17,522.00)(1.000,-1.000){2}{\rule{0.400pt}{0.241pt}}
\put(597.67,519){\rule{0.400pt}{0.482pt}}
\multiput(597.17,520.00)(1.000,-1.000){2}{\rule{0.400pt}{0.241pt}}
\put(597.0,522.0){\usebox{\plotpoint}}
\put(598.67,510){\rule{0.400pt}{1.686pt}}
\multiput(598.17,513.50)(1.000,-3.500){2}{\rule{0.400pt}{0.843pt}}
\put(599.67,510){\rule{0.400pt}{2.168pt}}
\multiput(599.17,510.00)(1.000,4.500){2}{\rule{0.400pt}{1.084pt}}
\put(599.0,517.0){\rule[-0.200pt]{0.400pt}{0.482pt}}
\put(600.67,511){\rule{0.400pt}{0.964pt}}
\multiput(600.17,513.00)(1.000,-2.000){2}{\rule{0.400pt}{0.482pt}}
\put(602,509.67){\rule{0.241pt}{0.400pt}}
\multiput(602.00,510.17)(0.500,-1.000){2}{\rule{0.120pt}{0.400pt}}
\put(601.0,515.0){\rule[-0.200pt]{0.400pt}{0.964pt}}
\put(602.67,512){\rule{0.400pt}{0.482pt}}
\multiput(602.17,513.00)(1.000,-1.000){2}{\rule{0.400pt}{0.241pt}}
\put(603.67,506){\rule{0.400pt}{1.445pt}}
\multiput(603.17,509.00)(1.000,-3.000){2}{\rule{0.400pt}{0.723pt}}
\put(603.0,510.0){\rule[-0.200pt]{0.400pt}{0.964pt}}
\put(604.67,504){\rule{0.400pt}{2.168pt}}
\multiput(604.17,508.50)(1.000,-4.500){2}{\rule{0.400pt}{1.084pt}}
\put(606,502.67){\rule{0.241pt}{0.400pt}}
\multiput(606.00,503.17)(0.500,-1.000){2}{\rule{0.120pt}{0.400pt}}
\put(605.0,506.0){\rule[-0.200pt]{0.400pt}{1.686pt}}
\put(606.67,504){\rule{0.400pt}{0.482pt}}
\multiput(606.17,505.00)(1.000,-1.000){2}{\rule{0.400pt}{0.241pt}}
\put(607.0,503.0){\rule[-0.200pt]{0.400pt}{0.723pt}}
\put(608,504){\usebox{\plotpoint}}
\put(608.0,504.0){\rule[-0.200pt]{0.482pt}{0.400pt}}
\put(610.0,499.0){\rule[-0.200pt]{0.400pt}{1.204pt}}
\put(611,497.67){\rule{0.241pt}{0.400pt}}
\multiput(611.00,498.17)(0.500,-1.000){2}{\rule{0.120pt}{0.400pt}}
\put(610.0,499.0){\usebox{\plotpoint}}
\put(612,498){\usebox{\plotpoint}}
\put(611.67,494){\rule{0.400pt}{0.964pt}}
\multiput(611.17,496.00)(1.000,-2.000){2}{\rule{0.400pt}{0.482pt}}
\put(612.67,494){\rule{0.400pt}{0.482pt}}
\multiput(612.17,494.00)(1.000,1.000){2}{\rule{0.400pt}{0.241pt}}
\put(613.67,489){\rule{0.400pt}{1.204pt}}
\multiput(613.17,491.50)(1.000,-2.500){2}{\rule{0.400pt}{0.602pt}}
\put(614.67,489){\rule{0.400pt}{0.482pt}}
\multiput(614.17,489.00)(1.000,1.000){2}{\rule{0.400pt}{0.241pt}}
\put(614.0,494.0){\rule[-0.200pt]{0.400pt}{0.482pt}}
\put(616,486.67){\rule{0.241pt}{0.400pt}}
\multiput(616.00,486.17)(0.500,1.000){2}{\rule{0.120pt}{0.400pt}}
\put(617,486.67){\rule{0.241pt}{0.400pt}}
\multiput(617.00,487.17)(0.500,-1.000){2}{\rule{0.120pt}{0.400pt}}
\put(616.0,487.0){\rule[-0.200pt]{0.400pt}{0.964pt}}
\put(618,487){\usebox{\plotpoint}}
\put(617.67,484){\rule{0.400pt}{0.723pt}}
\multiput(617.17,485.50)(1.000,-1.500){2}{\rule{0.400pt}{0.361pt}}
\put(618.67,481){\rule{0.400pt}{0.482pt}}
\multiput(618.17,482.00)(1.000,-1.000){2}{\rule{0.400pt}{0.241pt}}
\put(620,479.67){\rule{0.241pt}{0.400pt}}
\multiput(620.00,480.17)(0.500,-1.000){2}{\rule{0.120pt}{0.400pt}}
\put(619.0,483.0){\usebox{\plotpoint}}
\put(620.67,477){\rule{0.400pt}{0.964pt}}
\multiput(620.17,479.00)(1.000,-2.000){2}{\rule{0.400pt}{0.482pt}}
\put(621.67,477){\rule{0.400pt}{0.723pt}}
\multiput(621.17,477.00)(1.000,1.500){2}{\rule{0.400pt}{0.361pt}}
\put(621.0,480.0){\usebox{\plotpoint}}
\put(622.67,473){\rule{0.400pt}{0.964pt}}
\multiput(622.17,475.00)(1.000,-2.000){2}{\rule{0.400pt}{0.482pt}}
\put(623.0,477.0){\rule[-0.200pt]{0.400pt}{0.723pt}}
\put(624.67,469){\rule{0.400pt}{0.964pt}}
\multiput(624.17,471.00)(1.000,-2.000){2}{\rule{0.400pt}{0.482pt}}
\put(625.67,469){\rule{0.400pt}{0.964pt}}
\multiput(625.17,469.00)(1.000,2.000){2}{\rule{0.400pt}{0.482pt}}
\put(624.0,473.0){\usebox{\plotpoint}}
\put(627,465.67){\rule{0.241pt}{0.400pt}}
\multiput(627.00,465.17)(0.500,1.000){2}{\rule{0.120pt}{0.400pt}}
\put(627.67,463){\rule{0.400pt}{0.964pt}}
\multiput(627.17,465.00)(1.000,-2.000){2}{\rule{0.400pt}{0.482pt}}
\put(627.0,466.0){\rule[-0.200pt]{0.400pt}{1.686pt}}
\put(629.0,460.0){\rule[-0.200pt]{0.400pt}{0.723pt}}
\put(629.0,460.0){\usebox{\plotpoint}}
\put(629.67,457){\rule{0.400pt}{0.964pt}}
\multiput(629.17,459.00)(1.000,-2.000){2}{\rule{0.400pt}{0.482pt}}
\put(631,455.67){\rule{0.241pt}{0.400pt}}
\multiput(631.00,456.17)(0.500,-1.000){2}{\rule{0.120pt}{0.400pt}}
\put(630.0,460.0){\usebox{\plotpoint}}
\put(632,453.67){\rule{0.241pt}{0.400pt}}
\multiput(632.00,454.17)(0.500,-1.000){2}{\rule{0.120pt}{0.400pt}}
\put(633,452.67){\rule{0.241pt}{0.400pt}}
\multiput(633.00,453.17)(0.500,-1.000){2}{\rule{0.120pt}{0.400pt}}
\put(632.0,455.0){\usebox{\plotpoint}}
\put(634.0,450.0){\rule[-0.200pt]{0.400pt}{0.723pt}}
\put(634.67,446){\rule{0.400pt}{0.964pt}}
\multiput(634.17,448.00)(1.000,-2.000){2}{\rule{0.400pt}{0.482pt}}
\put(634.0,450.0){\usebox{\plotpoint}}
\put(636.0,444.0){\rule[-0.200pt]{0.400pt}{0.482pt}}
\put(636.0,444.0){\rule[-0.200pt]{0.482pt}{0.400pt}}
\put(638,439.67){\rule{0.241pt}{0.400pt}}
\multiput(638.00,440.17)(0.500,-1.000){2}{\rule{0.120pt}{0.400pt}}
\put(638.0,441.0){\rule[-0.200pt]{0.400pt}{0.723pt}}
\put(638.67,435){\rule{0.400pt}{0.723pt}}
\multiput(638.17,436.50)(1.000,-1.500){2}{\rule{0.400pt}{0.361pt}}
\put(639.67,435){\rule{0.400pt}{0.482pt}}
\multiput(639.17,435.00)(1.000,1.000){2}{\rule{0.400pt}{0.241pt}}
\put(639.0,438.0){\rule[-0.200pt]{0.400pt}{0.482pt}}
\put(641,431.67){\rule{0.241pt}{0.400pt}}
\multiput(641.00,431.17)(0.500,1.000){2}{\rule{0.120pt}{0.400pt}}
\put(641.67,430){\rule{0.400pt}{0.723pt}}
\multiput(641.17,431.50)(1.000,-1.500){2}{\rule{0.400pt}{0.361pt}}
\put(641.0,432.0){\rule[-0.200pt]{0.400pt}{1.204pt}}
\put(642.67,425){\rule{0.400pt}{0.482pt}}
\multiput(642.17,426.00)(1.000,-1.000){2}{\rule{0.400pt}{0.241pt}}
\put(643.67,422){\rule{0.400pt}{0.723pt}}
\multiput(643.17,423.50)(1.000,-1.500){2}{\rule{0.400pt}{0.361pt}}
\put(643.0,427.0){\rule[-0.200pt]{0.400pt}{0.723pt}}
\put(644.67,420){\rule{0.400pt}{0.964pt}}
\multiput(644.17,422.00)(1.000,-2.000){2}{\rule{0.400pt}{0.482pt}}
\put(645.67,417){\rule{0.400pt}{0.723pt}}
\multiput(645.17,418.50)(1.000,-1.500){2}{\rule{0.400pt}{0.361pt}}
\put(645.0,422.0){\rule[-0.200pt]{0.400pt}{0.482pt}}
\put(646.67,416){\rule{0.400pt}{0.482pt}}
\multiput(646.17,417.00)(1.000,-1.000){2}{\rule{0.400pt}{0.241pt}}
\put(648,414.67){\rule{0.241pt}{0.400pt}}
\multiput(648.00,415.17)(0.500,-1.000){2}{\rule{0.120pt}{0.400pt}}
\put(647.0,417.0){\usebox{\plotpoint}}
\put(649,408.67){\rule{0.241pt}{0.400pt}}
\multiput(649.00,409.17)(0.500,-1.000){2}{\rule{0.120pt}{0.400pt}}
\put(649.0,410.0){\rule[-0.200pt]{0.400pt}{1.204pt}}
\put(649.67,406){\rule{0.400pt}{0.482pt}}
\multiput(649.17,407.00)(1.000,-1.000){2}{\rule{0.400pt}{0.241pt}}
\put(651,405.67){\rule{0.241pt}{0.400pt}}
\multiput(651.00,405.17)(0.500,1.000){2}{\rule{0.120pt}{0.400pt}}
\put(650.0,408.0){\usebox{\plotpoint}}
\put(651.67,403){\rule{0.400pt}{0.482pt}}
\multiput(651.17,404.00)(1.000,-1.000){2}{\rule{0.400pt}{0.241pt}}
\put(652.67,401){\rule{0.400pt}{0.482pt}}
\multiput(652.17,402.00)(1.000,-1.000){2}{\rule{0.400pt}{0.241pt}}
\put(652.0,405.0){\rule[-0.200pt]{0.400pt}{0.482pt}}
\put(654.0,398.0){\rule[-0.200pt]{0.400pt}{0.723pt}}
\put(654.67,395){\rule{0.400pt}{0.723pt}}
\multiput(654.17,396.50)(1.000,-1.500){2}{\rule{0.400pt}{0.361pt}}
\put(654.0,398.0){\usebox{\plotpoint}}
\put(656,390.67){\rule{0.241pt}{0.400pt}}
\multiput(656.00,391.17)(0.500,-1.000){2}{\rule{0.120pt}{0.400pt}}
\put(657,389.67){\rule{0.241pt}{0.400pt}}
\multiput(657.00,390.17)(0.500,-1.000){2}{\rule{0.120pt}{0.400pt}}
\put(656.0,392.0){\rule[-0.200pt]{0.400pt}{0.723pt}}
\put(658,386.67){\rule{0.241pt}{0.400pt}}
\multiput(658.00,387.17)(0.500,-1.000){2}{\rule{0.120pt}{0.400pt}}
\put(658.0,388.0){\rule[-0.200pt]{0.400pt}{0.482pt}}
\put(659,384.67){\rule{0.241pt}{0.400pt}}
\multiput(659.00,385.17)(0.500,-1.000){2}{\rule{0.120pt}{0.400pt}}
\put(659.67,381){\rule{0.400pt}{0.964pt}}
\multiput(659.17,383.00)(1.000,-2.000){2}{\rule{0.400pt}{0.482pt}}
\put(659.0,386.0){\usebox{\plotpoint}}
\put(660.67,377){\rule{0.400pt}{0.723pt}}
\multiput(660.17,378.50)(1.000,-1.500){2}{\rule{0.400pt}{0.361pt}}
\put(661.67,377){\rule{0.400pt}{0.482pt}}
\multiput(661.17,377.00)(1.000,1.000){2}{\rule{0.400pt}{0.241pt}}
\put(661.0,380.0){\usebox{\plotpoint}}
\put(663,372.67){\rule{0.241pt}{0.400pt}}
\multiput(663.00,373.17)(0.500,-1.000){2}{\rule{0.120pt}{0.400pt}}
\put(663.67,371){\rule{0.400pt}{0.482pt}}
\multiput(663.17,372.00)(1.000,-1.000){2}{\rule{0.400pt}{0.241pt}}
\put(663.0,374.0){\rule[-0.200pt]{0.400pt}{1.204pt}}
\put(665.0,369.0){\rule[-0.200pt]{0.400pt}{0.482pt}}
\put(665.67,367){\rule{0.400pt}{0.482pt}}
\multiput(665.17,368.00)(1.000,-1.000){2}{\rule{0.400pt}{0.241pt}}
\put(665.0,369.0){\usebox{\plotpoint}}
\put(667.0,365.0){\rule[-0.200pt]{0.400pt}{0.482pt}}
\put(667.0,365.0){\usebox{\plotpoint}}
\put(667.67,361){\rule{0.400pt}{0.482pt}}
\multiput(667.17,362.00)(1.000,-1.000){2}{\rule{0.400pt}{0.241pt}}
\put(668.67,358){\rule{0.400pt}{0.723pt}}
\multiput(668.17,359.50)(1.000,-1.500){2}{\rule{0.400pt}{0.361pt}}
\put(668.0,363.0){\rule[-0.200pt]{0.400pt}{0.482pt}}
\put(670,358){\usebox{\plotpoint}}
\put(670,356.67){\rule{0.241pt}{0.400pt}}
\multiput(670.00,357.17)(0.500,-1.000){2}{\rule{0.120pt}{0.400pt}}
\put(670.67,355){\rule{0.400pt}{0.482pt}}
\multiput(670.17,356.00)(1.000,-1.000){2}{\rule{0.400pt}{0.241pt}}
\put(672,351.67){\rule{0.241pt}{0.400pt}}
\multiput(672.00,352.17)(0.500,-1.000){2}{\rule{0.120pt}{0.400pt}}
\put(673,350.67){\rule{0.241pt}{0.400pt}}
\multiput(673.00,351.17)(0.500,-1.000){2}{\rule{0.120pt}{0.400pt}}
\put(672.0,353.0){\rule[-0.200pt]{0.400pt}{0.482pt}}
\put(674,348.67){\rule{0.241pt}{0.400pt}}
\multiput(674.00,349.17)(0.500,-1.000){2}{\rule{0.120pt}{0.400pt}}
\put(675,347.67){\rule{0.241pt}{0.400pt}}
\multiput(675.00,348.17)(0.500,-1.000){2}{\rule{0.120pt}{0.400pt}}
\put(674.0,350.0){\usebox{\plotpoint}}
\put(675.67,344){\rule{0.400pt}{0.482pt}}
\multiput(675.17,345.00)(1.000,-1.000){2}{\rule{0.400pt}{0.241pt}}
\put(676.0,346.0){\rule[-0.200pt]{0.400pt}{0.482pt}}
\put(676.67,340){\rule{0.400pt}{0.723pt}}
\multiput(676.17,341.50)(1.000,-1.500){2}{\rule{0.400pt}{0.361pt}}
\put(678,338.67){\rule{0.241pt}{0.400pt}}
\multiput(678.00,339.17)(0.500,-1.000){2}{\rule{0.120pt}{0.400pt}}
\put(677.0,343.0){\usebox{\plotpoint}}
\put(679,339){\usebox{\plotpoint}}
\put(678.67,337){\rule{0.400pt}{0.482pt}}
\multiput(678.17,338.00)(1.000,-1.000){2}{\rule{0.400pt}{0.241pt}}
\put(680,335.67){\rule{0.241pt}{0.400pt}}
\multiput(680.00,336.17)(0.500,-1.000){2}{\rule{0.120pt}{0.400pt}}
\put(681,336){\usebox{\plotpoint}}
\put(680.67,334){\rule{0.400pt}{0.482pt}}
\multiput(680.17,335.00)(1.000,-1.000){2}{\rule{0.400pt}{0.241pt}}
\put(681.67,332){\rule{0.400pt}{0.482pt}}
\multiput(681.17,333.00)(1.000,-1.000){2}{\rule{0.400pt}{0.241pt}}
\put(683.0,331.0){\usebox{\plotpoint}}
\put(683.67,329){\rule{0.400pt}{0.482pt}}
\multiput(683.17,330.00)(1.000,-1.000){2}{\rule{0.400pt}{0.241pt}}
\put(683.0,331.0){\usebox{\plotpoint}}
\put(685,329){\usebox{\plotpoint}}
\put(685,327.67){\rule{0.241pt}{0.400pt}}
\multiput(685.00,328.17)(0.500,-1.000){2}{\rule{0.120pt}{0.400pt}}
\put(686.0,325.0){\rule[-0.200pt]{0.400pt}{0.723pt}}
\put(687,323.67){\rule{0.241pt}{0.400pt}}
\multiput(687.00,324.17)(0.500,-1.000){2}{\rule{0.120pt}{0.400pt}}
\put(686.0,325.0){\usebox{\plotpoint}}
\put(688,324){\usebox{\plotpoint}}
\put(687.67,321){\rule{0.400pt}{0.723pt}}
\multiput(687.17,322.50)(1.000,-1.500){2}{\rule{0.400pt}{0.361pt}}
\put(689.67,317){\rule{0.400pt}{0.964pt}}
\multiput(689.17,319.00)(1.000,-2.000){2}{\rule{0.400pt}{0.482pt}}
\put(690.67,317){\rule{0.400pt}{0.723pt}}
\multiput(690.17,317.00)(1.000,1.500){2}{\rule{0.400pt}{0.361pt}}
\put(689.0,321.0){\usebox{\plotpoint}}
\put(692.0,317.0){\rule[-0.200pt]{0.400pt}{0.723pt}}
\put(693,315.67){\rule{0.241pt}{0.400pt}}
\multiput(693.00,316.17)(0.500,-1.000){2}{\rule{0.120pt}{0.400pt}}
\put(692.0,317.0){\usebox{\plotpoint}}
\put(694.0,315.0){\usebox{\plotpoint}}
\put(694.0,315.0){\usebox{\plotpoint}}
\put(695,311.67){\rule{0.241pt}{0.400pt}}
\multiput(695.00,312.17)(0.500,-1.000){2}{\rule{0.120pt}{0.400pt}}
\put(695.0,313.0){\rule[-0.200pt]{0.400pt}{0.482pt}}
\put(696.0,312.0){\usebox{\plotpoint}}
\put(697,308.67){\rule{0.241pt}{0.400pt}}
\multiput(697.00,309.17)(0.500,-1.000){2}{\rule{0.120pt}{0.400pt}}
\put(698,307.67){\rule{0.241pt}{0.400pt}}
\multiput(698.00,308.17)(0.500,-1.000){2}{\rule{0.120pt}{0.400pt}}
\put(697.0,310.0){\rule[-0.200pt]{0.400pt}{0.482pt}}
\put(699,305.67){\rule{0.241pt}{0.400pt}}
\multiput(699.00,306.17)(0.500,-1.000){2}{\rule{0.120pt}{0.400pt}}
\put(700,305.67){\rule{0.241pt}{0.400pt}}
\multiput(700.00,305.17)(0.500,1.000){2}{\rule{0.120pt}{0.400pt}}
\put(699.0,307.0){\usebox{\plotpoint}}
\put(701,305.67){\rule{0.241pt}{0.400pt}}
\multiput(701.00,305.17)(0.500,1.000){2}{\rule{0.120pt}{0.400pt}}
\put(701.67,304){\rule{0.400pt}{0.723pt}}
\multiput(701.17,305.50)(1.000,-1.500){2}{\rule{0.400pt}{0.361pt}}
\put(701.0,306.0){\usebox{\plotpoint}}
\put(703,304){\usebox{\plotpoint}}
\put(703.0,304.0){\usebox{\plotpoint}}
\put(704.0,303.0){\usebox{\plotpoint}}
\put(704.0,303.0){\rule[-0.200pt]{0.482pt}{0.400pt}}
\put(706,300.67){\rule{0.241pt}{0.400pt}}
\multiput(706.00,301.17)(0.500,-1.000){2}{\rule{0.120pt}{0.400pt}}
\put(707,300.67){\rule{0.241pt}{0.400pt}}
\multiput(707.00,300.17)(0.500,1.000){2}{\rule{0.120pt}{0.400pt}}
\put(706.0,302.0){\usebox{\plotpoint}}
\put(708,300.67){\rule{0.241pt}{0.400pt}}
\multiput(708.00,300.17)(0.500,1.000){2}{\rule{0.120pt}{0.400pt}}
\put(708.67,299){\rule{0.400pt}{0.723pt}}
\multiput(708.17,300.50)(1.000,-1.500){2}{\rule{0.400pt}{0.361pt}}
\put(708.0,301.0){\usebox{\plotpoint}}
\put(709.67,298){\rule{0.400pt}{0.482pt}}
\multiput(709.17,298.00)(1.000,1.000){2}{\rule{0.400pt}{0.241pt}}
\put(710.0,298.0){\usebox{\plotpoint}}
\put(711.0,299.0){\usebox{\plotpoint}}
\put(711.0,299.0){\rule[-0.200pt]{0.482pt}{0.400pt}}
\put(713,296.67){\rule{0.241pt}{0.400pt}}
\multiput(713.00,296.17)(0.500,1.000){2}{\rule{0.120pt}{0.400pt}}
\put(714,296.67){\rule{0.241pt}{0.400pt}}
\multiput(714.00,297.17)(0.500,-1.000){2}{\rule{0.120pt}{0.400pt}}
\put(713.0,297.0){\rule[-0.200pt]{0.400pt}{0.482pt}}
\put(715,296.67){\rule{0.241pt}{0.400pt}}
\multiput(715.00,297.17)(0.500,-1.000){2}{\rule{0.120pt}{0.400pt}}
\put(716,296.67){\rule{0.241pt}{0.400pt}}
\multiput(716.00,296.17)(0.500,1.000){2}{\rule{0.120pt}{0.400pt}}
\put(715.0,297.0){\usebox{\plotpoint}}
\put(716.67,296){\rule{0.400pt}{0.482pt}}
\multiput(716.17,296.00)(1.000,1.000){2}{\rule{0.400pt}{0.241pt}}
\put(717.67,296){\rule{0.400pt}{0.482pt}}
\multiput(717.17,297.00)(1.000,-1.000){2}{\rule{0.400pt}{0.241pt}}
\put(717.0,296.0){\rule[-0.200pt]{0.400pt}{0.482pt}}
\put(719,296){\usebox{\plotpoint}}
\put(719,295.67){\rule{0.241pt}{0.400pt}}
\multiput(719.00,295.17)(0.500,1.000){2}{\rule{0.120pt}{0.400pt}}
\put(720,297){\usebox{\plotpoint}}
\put(719.67,295){\rule{0.400pt}{0.482pt}}
\multiput(719.17,296.00)(1.000,-1.000){2}{\rule{0.400pt}{0.241pt}}
\put(721.67,293){\rule{0.400pt}{0.482pt}}
\multiput(721.17,294.00)(1.000,-1.000){2}{\rule{0.400pt}{0.241pt}}
\put(723,292.67){\rule{0.241pt}{0.400pt}}
\multiput(723.00,292.17)(0.500,1.000){2}{\rule{0.120pt}{0.400pt}}
\put(721.0,295.0){\usebox{\plotpoint}}
\put(724,294){\usebox{\plotpoint}}
\put(723.67,294){\rule{0.400pt}{0.482pt}}
\multiput(723.17,294.00)(1.000,1.000){2}{\rule{0.400pt}{0.241pt}}
\put(725,294.67){\rule{0.241pt}{0.400pt}}
\multiput(725.00,295.17)(0.500,-1.000){2}{\rule{0.120pt}{0.400pt}}
\put(725.67,294){\rule{0.400pt}{0.723pt}}
\multiput(725.17,294.00)(1.000,1.500){2}{\rule{0.400pt}{0.361pt}}
\put(726.0,294.0){\usebox{\plotpoint}}
\put(727.0,295.0){\rule[-0.200pt]{0.400pt}{0.482pt}}
\put(728,294.67){\rule{0.241pt}{0.400pt}}
\multiput(728.00,294.17)(0.500,1.000){2}{\rule{0.120pt}{0.400pt}}
\put(727.0,295.0){\usebox{\plotpoint}}
\put(729.0,295.0){\usebox{\plotpoint}}
\put(729.0,295.0){\rule[-0.200pt]{0.482pt}{0.400pt}}
\put(730.67,293){\rule{0.400pt}{0.482pt}}
\multiput(730.17,293.00)(1.000,1.000){2}{\rule{0.400pt}{0.241pt}}
\put(731.0,293.0){\rule[-0.200pt]{0.400pt}{0.482pt}}
\put(735,294.67){\rule{0.241pt}{0.400pt}}
\multiput(735.00,294.17)(0.500,1.000){2}{\rule{0.120pt}{0.400pt}}
\put(732.0,295.0){\rule[-0.200pt]{0.723pt}{0.400pt}}
\put(736,296){\usebox{\plotpoint}}
\put(737,294.67){\rule{0.241pt}{0.400pt}}
\multiput(737.00,295.17)(0.500,-1.000){2}{\rule{0.120pt}{0.400pt}}
\put(736.0,296.0){\usebox{\plotpoint}}
\put(738.0,295.0){\usebox{\plotpoint}}
\put(739,295.67){\rule{0.241pt}{0.400pt}}
\multiput(739.00,295.17)(0.500,1.000){2}{\rule{0.120pt}{0.400pt}}
\put(738.0,296.0){\usebox{\plotpoint}}
\put(740.0,296.0){\usebox{\plotpoint}}
\put(741,295.67){\rule{0.241pt}{0.400pt}}
\multiput(741.00,295.17)(0.500,1.000){2}{\rule{0.120pt}{0.400pt}}
\put(740.0,296.0){\usebox{\plotpoint}}
\put(742,297){\usebox{\plotpoint}}
\put(742.0,297.0){\usebox{\plotpoint}}
\put(743,295.67){\rule{0.241pt}{0.400pt}}
\multiput(743.00,295.17)(0.500,1.000){2}{\rule{0.120pt}{0.400pt}}
\put(743.67,297){\rule{0.400pt}{0.482pt}}
\multiput(743.17,297.00)(1.000,1.000){2}{\rule{0.400pt}{0.241pt}}
\put(743.0,296.0){\usebox{\plotpoint}}
\put(745.0,298.0){\usebox{\plotpoint}}
\put(745.67,296){\rule{0.400pt}{0.482pt}}
\multiput(745.17,297.00)(1.000,-1.000){2}{\rule{0.400pt}{0.241pt}}
\put(745.0,298.0){\usebox{\plotpoint}}
\put(747,296){\usebox{\plotpoint}}
\put(746.67,296){\rule{0.400pt}{0.482pt}}
\multiput(746.17,296.00)(1.000,1.000){2}{\rule{0.400pt}{0.241pt}}
\put(749,296.67){\rule{0.241pt}{0.400pt}}
\multiput(749.00,297.17)(0.500,-1.000){2}{\rule{0.120pt}{0.400pt}}
\put(748.0,298.0){\usebox{\plotpoint}}
\put(750,298.67){\rule{0.241pt}{0.400pt}}
\multiput(750.00,298.17)(0.500,1.000){2}{\rule{0.120pt}{0.400pt}}
\put(751,298.67){\rule{0.241pt}{0.400pt}}
\multiput(751.00,299.17)(0.500,-1.000){2}{\rule{0.120pt}{0.400pt}}
\put(750.0,297.0){\rule[-0.200pt]{0.400pt}{0.482pt}}
\put(752,299){\usebox{\plotpoint}}
\put(752,298.67){\rule{0.241pt}{0.400pt}}
\multiput(752.00,298.17)(0.500,1.000){2}{\rule{0.120pt}{0.400pt}}
\put(754,299.67){\rule{0.241pt}{0.400pt}}
\multiput(754.00,299.17)(0.500,1.000){2}{\rule{0.120pt}{0.400pt}}
\put(755,299.67){\rule{0.241pt}{0.400pt}}
\multiput(755.00,300.17)(0.500,-1.000){2}{\rule{0.120pt}{0.400pt}}
\put(753.0,300.0){\usebox{\plotpoint}}
\put(756,300.67){\rule{0.241pt}{0.400pt}}
\multiput(756.00,301.17)(0.500,-1.000){2}{\rule{0.120pt}{0.400pt}}
\put(756.0,300.0){\rule[-0.200pt]{0.400pt}{0.482pt}}
\put(756.67,300){\rule{0.400pt}{0.482pt}}
\multiput(756.17,300.00)(1.000,1.000){2}{\rule{0.400pt}{0.241pt}}
\put(758,301.67){\rule{0.241pt}{0.400pt}}
\multiput(758.00,301.17)(0.500,1.000){2}{\rule{0.120pt}{0.400pt}}
\put(757.0,300.0){\usebox{\plotpoint}}
\put(759.0,302.0){\usebox{\plotpoint}}
\put(759.0,302.0){\rule[-0.200pt]{0.482pt}{0.400pt}}
\put(761.0,302.0){\usebox{\plotpoint}}
\put(761.0,303.0){\rule[-0.200pt]{0.482pt}{0.400pt}}
\put(763,303.67){\rule{0.241pt}{0.400pt}}
\multiput(763.00,303.17)(0.500,1.000){2}{\rule{0.120pt}{0.400pt}}
\put(764,303.67){\rule{0.241pt}{0.400pt}}
\multiput(764.00,304.17)(0.500,-1.000){2}{\rule{0.120pt}{0.400pt}}
\put(763.0,303.0){\usebox{\plotpoint}}
\put(765,304.67){\rule{0.241pt}{0.400pt}}
\multiput(765.00,305.17)(0.500,-1.000){2}{\rule{0.120pt}{0.400pt}}
\put(765.0,304.0){\rule[-0.200pt]{0.400pt}{0.482pt}}
\put(766.0,305.0){\usebox{\plotpoint}}
\put(767,305.67){\rule{0.241pt}{0.400pt}}
\multiput(767.00,305.17)(0.500,1.000){2}{\rule{0.120pt}{0.400pt}}
\put(766.0,306.0){\usebox{\plotpoint}}
\put(768,307){\usebox{\plotpoint}}
\put(768,306.67){\rule{0.241pt}{0.400pt}}
\multiput(768.00,306.17)(0.500,1.000){2}{\rule{0.120pt}{0.400pt}}
\put(770,307.67){\rule{0.241pt}{0.400pt}}
\multiput(770.00,307.17)(0.500,1.000){2}{\rule{0.120pt}{0.400pt}}
\put(771,308.67){\rule{0.241pt}{0.400pt}}
\multiput(771.00,308.17)(0.500,1.000){2}{\rule{0.120pt}{0.400pt}}
\put(769.0,308.0){\usebox{\plotpoint}}
\put(772,308.67){\rule{0.241pt}{0.400pt}}
\multiput(772.00,308.17)(0.500,1.000){2}{\rule{0.120pt}{0.400pt}}
\put(772.0,309.0){\usebox{\plotpoint}}
\put(773.0,310.0){\usebox{\plotpoint}}
\put(774,309.67){\rule{0.241pt}{0.400pt}}
\multiput(774.00,310.17)(0.500,-1.000){2}{\rule{0.120pt}{0.400pt}}
\put(773.0,311.0){\usebox{\plotpoint}}
\put(775,310.67){\rule{0.241pt}{0.400pt}}
\multiput(775.00,310.17)(0.500,1.000){2}{\rule{0.120pt}{0.400pt}}
\put(775.0,310.0){\usebox{\plotpoint}}
\put(776.0,312.0){\usebox{\plotpoint}}
\put(777,311.67){\rule{0.241pt}{0.400pt}}
\multiput(777.00,312.17)(0.500,-1.000){2}{\rule{0.120pt}{0.400pt}}
\put(777.67,312){\rule{0.400pt}{0.482pt}}
\multiput(777.17,312.00)(1.000,1.000){2}{\rule{0.400pt}{0.241pt}}
\put(777.0,312.0){\usebox{\plotpoint}}
\put(779,314){\usebox{\plotpoint}}
\put(779,313.67){\rule{0.241pt}{0.400pt}}
\multiput(779.00,313.17)(0.500,1.000){2}{\rule{0.120pt}{0.400pt}}
\put(780,314.67){\rule{0.241pt}{0.400pt}}
\multiput(780.00,315.17)(0.500,-1.000){2}{\rule{0.120pt}{0.400pt}}
\put(780.67,315){\rule{0.400pt}{0.482pt}}
\multiput(780.17,315.00)(1.000,1.000){2}{\rule{0.400pt}{0.241pt}}
\put(780.0,315.0){\usebox{\plotpoint}}
\put(782,316.67){\rule{0.241pt}{0.400pt}}
\multiput(782.00,317.17)(0.500,-1.000){2}{\rule{0.120pt}{0.400pt}}
\put(783,316.67){\rule{0.241pt}{0.400pt}}
\multiput(783.00,316.17)(0.500,1.000){2}{\rule{0.120pt}{0.400pt}}
\put(782.0,317.0){\usebox{\plotpoint}}
\put(783.67,319){\rule{0.400pt}{0.482pt}}
\multiput(783.17,319.00)(1.000,1.000){2}{\rule{0.400pt}{0.241pt}}
\put(784.0,318.0){\usebox{\plotpoint}}
\put(785,318.67){\rule{0.241pt}{0.400pt}}
\multiput(785.00,319.17)(0.500,-1.000){2}{\rule{0.120pt}{0.400pt}}
\put(785.67,319){\rule{0.400pt}{0.964pt}}
\multiput(785.17,319.00)(1.000,2.000){2}{\rule{0.400pt}{0.482pt}}
\put(785.0,320.0){\usebox{\plotpoint}}
\put(787,321.67){\rule{0.241pt}{0.400pt}}
\multiput(787.00,321.17)(0.500,1.000){2}{\rule{0.120pt}{0.400pt}}
\put(787.67,323){\rule{0.400pt}{0.482pt}}
\multiput(787.17,323.00)(1.000,1.000){2}{\rule{0.400pt}{0.241pt}}
\put(787.0,322.0){\usebox{\plotpoint}}
\put(789,325){\usebox{\plotpoint}}
\put(789,324.67){\rule{0.241pt}{0.400pt}}
\multiput(789.00,324.17)(0.500,1.000){2}{\rule{0.120pt}{0.400pt}}
\put(790,324.67){\rule{0.241pt}{0.400pt}}
\multiput(790.00,325.17)(0.500,-1.000){2}{\rule{0.120pt}{0.400pt}}
\put(791,326.67){\rule{0.241pt}{0.400pt}}
\multiput(791.00,326.17)(0.500,1.000){2}{\rule{0.120pt}{0.400pt}}
\put(791.0,325.0){\rule[-0.200pt]{0.400pt}{0.482pt}}
\put(792,328){\usebox{\plotpoint}}
\put(791.67,328){\rule{0.400pt}{0.723pt}}
\multiput(791.17,328.00)(1.000,1.500){2}{\rule{0.400pt}{0.361pt}}
\put(793.0,331.0){\usebox{\plotpoint}}
\put(794,331.67){\rule{0.241pt}{0.400pt}}
\multiput(794.00,332.17)(0.500,-1.000){2}{\rule{0.120pt}{0.400pt}}
\put(794.67,332){\rule{0.400pt}{0.482pt}}
\multiput(794.17,332.00)(1.000,1.000){2}{\rule{0.400pt}{0.241pt}}
\put(794.0,331.0){\rule[-0.200pt]{0.400pt}{0.482pt}}
\put(796,334.67){\rule{0.241pt}{0.400pt}}
\multiput(796.00,334.17)(0.500,1.000){2}{\rule{0.120pt}{0.400pt}}
\put(796.0,334.0){\usebox{\plotpoint}}
\put(798,335.67){\rule{0.241pt}{0.400pt}}
\multiput(798.00,335.17)(0.500,1.000){2}{\rule{0.120pt}{0.400pt}}
\put(797.0,336.0){\usebox{\plotpoint}}
\put(799,337){\usebox{\plotpoint}}
\put(798.67,337){\rule{0.400pt}{0.482pt}}
\multiput(798.17,337.00)(1.000,1.000){2}{\rule{0.400pt}{0.241pt}}
\put(799.67,339){\rule{0.400pt}{0.723pt}}
\multiput(799.17,339.00)(1.000,1.500){2}{\rule{0.400pt}{0.361pt}}
\put(800.67,341){\rule{0.400pt}{0.964pt}}
\multiput(800.17,341.00)(1.000,2.000){2}{\rule{0.400pt}{0.482pt}}
\put(801.67,343){\rule{0.400pt}{0.482pt}}
\multiput(801.17,344.00)(1.000,-1.000){2}{\rule{0.400pt}{0.241pt}}
\put(801.0,341.0){\usebox{\plotpoint}}
\put(803.0,343.0){\rule[-0.200pt]{0.400pt}{0.482pt}}
\put(803.67,345){\rule{0.400pt}{0.723pt}}
\multiput(803.17,345.00)(1.000,1.500){2}{\rule{0.400pt}{0.361pt}}
\put(803.0,345.0){\usebox{\plotpoint}}
\put(805,346.67){\rule{0.241pt}{0.400pt}}
\multiput(805.00,346.17)(0.500,1.000){2}{\rule{0.120pt}{0.400pt}}
\put(805.0,347.0){\usebox{\plotpoint}}
\put(806,350.67){\rule{0.241pt}{0.400pt}}
\multiput(806.00,350.17)(0.500,1.000){2}{\rule{0.120pt}{0.400pt}}
\put(807,351.67){\rule{0.241pt}{0.400pt}}
\multiput(807.00,351.17)(0.500,1.000){2}{\rule{0.120pt}{0.400pt}}
\put(806.0,348.0){\rule[-0.200pt]{0.400pt}{0.723pt}}
\put(808,353){\usebox{\plotpoint}}
\put(807.67,353){\rule{0.400pt}{0.482pt}}
\multiput(807.17,353.00)(1.000,1.000){2}{\rule{0.400pt}{0.241pt}}
\put(809.0,355.0){\usebox{\plotpoint}}
\put(809.67,357){\rule{0.400pt}{0.723pt}}
\multiput(809.17,357.00)(1.000,1.500){2}{\rule{0.400pt}{0.361pt}}
\put(811,359.67){\rule{0.241pt}{0.400pt}}
\multiput(811.00,359.17)(0.500,1.000){2}{\rule{0.120pt}{0.400pt}}
\put(810.0,355.0){\rule[-0.200pt]{0.400pt}{0.482pt}}
\put(812,361){\usebox{\plotpoint}}
\put(811.67,361){\rule{0.400pt}{0.723pt}}
\multiput(811.17,361.00)(1.000,1.500){2}{\rule{0.400pt}{0.361pt}}
\put(812.67,362){\rule{0.400pt}{0.964pt}}
\multiput(812.17,362.00)(1.000,2.000){2}{\rule{0.400pt}{0.482pt}}
\put(813.67,366){\rule{0.400pt}{0.482pt}}
\multiput(813.17,366.00)(1.000,1.000){2}{\rule{0.400pt}{0.241pt}}
\put(813.0,362.0){\rule[-0.200pt]{0.400pt}{0.482pt}}
\put(815,369.67){\rule{0.241pt}{0.400pt}}
\multiput(815.00,369.17)(0.500,1.000){2}{\rule{0.120pt}{0.400pt}}
\put(816,370.67){\rule{0.241pt}{0.400pt}}
\multiput(816.00,370.17)(0.500,1.000){2}{\rule{0.120pt}{0.400pt}}
\put(815.0,368.0){\rule[-0.200pt]{0.400pt}{0.482pt}}
\put(816.67,373){\rule{0.400pt}{0.723pt}}
\multiput(816.17,373.00)(1.000,1.500){2}{\rule{0.400pt}{0.361pt}}
\put(817.0,372.0){\usebox{\plotpoint}}
\put(818,376.67){\rule{0.241pt}{0.400pt}}
\multiput(818.00,376.17)(0.500,1.000){2}{\rule{0.120pt}{0.400pt}}
\put(818.67,378){\rule{0.400pt}{0.723pt}}
\multiput(818.17,378.00)(1.000,1.500){2}{\rule{0.400pt}{0.361pt}}
\put(818.0,376.0){\usebox{\plotpoint}}
\put(819.67,380){\rule{0.400pt}{0.482pt}}
\multiput(819.17,380.00)(1.000,1.000){2}{\rule{0.400pt}{0.241pt}}
\put(820.67,382){\rule{0.400pt}{0.723pt}}
\multiput(820.17,382.00)(1.000,1.500){2}{\rule{0.400pt}{0.361pt}}
\put(820.0,380.0){\usebox{\plotpoint}}
\put(822,387.67){\rule{0.241pt}{0.400pt}}
\multiput(822.00,387.17)(0.500,1.000){2}{\rule{0.120pt}{0.400pt}}
\put(822.67,389){\rule{0.400pt}{0.482pt}}
\multiput(822.17,389.00)(1.000,1.000){2}{\rule{0.400pt}{0.241pt}}
\put(822.0,385.0){\rule[-0.200pt]{0.400pt}{0.723pt}}
\put(824,391.67){\rule{0.241pt}{0.400pt}}
\multiput(824.00,391.17)(0.500,1.000){2}{\rule{0.120pt}{0.400pt}}
\put(824.0,391.0){\usebox{\plotpoint}}
\put(824.67,395){\rule{0.400pt}{0.723pt}}
\multiput(824.17,395.00)(1.000,1.500){2}{\rule{0.400pt}{0.361pt}}
\put(825.67,398){\rule{0.400pt}{0.482pt}}
\multiput(825.17,398.00)(1.000,1.000){2}{\rule{0.400pt}{0.241pt}}
\put(825.0,393.0){\rule[-0.200pt]{0.400pt}{0.482pt}}
\put(827,400){\usebox{\plotpoint}}
\put(826.67,400){\rule{0.400pt}{0.723pt}}
\multiput(826.17,400.00)(1.000,1.500){2}{\rule{0.400pt}{0.361pt}}
\put(827.67,403){\rule{0.400pt}{0.482pt}}
\multiput(827.17,403.00)(1.000,1.000){2}{\rule{0.400pt}{0.241pt}}
\put(829,407.67){\rule{0.241pt}{0.400pt}}
\multiput(829.00,408.17)(0.500,-1.000){2}{\rule{0.120pt}{0.400pt}}
\put(829.0,405.0){\rule[-0.200pt]{0.400pt}{0.964pt}}
\put(829.67,411){\rule{0.400pt}{0.723pt}}
\multiput(829.17,411.00)(1.000,1.500){2}{\rule{0.400pt}{0.361pt}}
\put(831,413.67){\rule{0.241pt}{0.400pt}}
\multiput(831.00,413.17)(0.500,1.000){2}{\rule{0.120pt}{0.400pt}}
\put(830.0,408.0){\rule[-0.200pt]{0.400pt}{0.723pt}}
\put(831.67,416){\rule{0.400pt}{0.482pt}}
\multiput(831.17,416.00)(1.000,1.000){2}{\rule{0.400pt}{0.241pt}}
\put(832.67,418){\rule{0.400pt}{1.445pt}}
\multiput(832.17,418.00)(1.000,3.000){2}{\rule{0.400pt}{0.723pt}}
\put(832.0,415.0){\usebox{\plotpoint}}
\put(833.67,423){\rule{0.400pt}{0.964pt}}
\multiput(833.17,423.00)(1.000,2.000){2}{\rule{0.400pt}{0.482pt}}
\put(834.67,427){\rule{0.400pt}{0.482pt}}
\multiput(834.17,427.00)(1.000,1.000){2}{\rule{0.400pt}{0.241pt}}
\put(834.0,423.0){\usebox{\plotpoint}}
\put(835.67,431){\rule{0.400pt}{0.482pt}}
\multiput(835.17,431.00)(1.000,1.000){2}{\rule{0.400pt}{0.241pt}}
\put(836.0,429.0){\rule[-0.200pt]{0.400pt}{0.482pt}}
\put(836.67,434){\rule{0.400pt}{0.482pt}}
\multiput(836.17,434.00)(1.000,1.000){2}{\rule{0.400pt}{0.241pt}}
\put(837.67,436){\rule{0.400pt}{0.482pt}}
\multiput(837.17,436.00)(1.000,1.000){2}{\rule{0.400pt}{0.241pt}}
\put(837.0,433.0){\usebox{\plotpoint}}
\put(838.67,440){\rule{0.400pt}{0.964pt}}
\multiput(838.17,440.00)(1.000,2.000){2}{\rule{0.400pt}{0.482pt}}
\put(839.67,444){\rule{0.400pt}{0.482pt}}
\multiput(839.17,444.00)(1.000,1.000){2}{\rule{0.400pt}{0.241pt}}
\put(839.0,438.0){\rule[-0.200pt]{0.400pt}{0.482pt}}
\put(840.67,447){\rule{0.400pt}{0.482pt}}
\multiput(840.17,447.00)(1.000,1.000){2}{\rule{0.400pt}{0.241pt}}
\put(841.0,446.0){\usebox{\plotpoint}}
\put(841.67,453){\rule{0.400pt}{0.723pt}}
\multiput(841.17,453.00)(1.000,1.500){2}{\rule{0.400pt}{0.361pt}}
\put(843,455.67){\rule{0.241pt}{0.400pt}}
\multiput(843.00,455.17)(0.500,1.000){2}{\rule{0.120pt}{0.400pt}}
\put(842.0,449.0){\rule[-0.200pt]{0.400pt}{0.964pt}}
\put(844,461.67){\rule{0.241pt}{0.400pt}}
\multiput(844.00,461.17)(0.500,1.000){2}{\rule{0.120pt}{0.400pt}}
\put(844.67,463){\rule{0.400pt}{0.482pt}}
\multiput(844.17,463.00)(1.000,1.000){2}{\rule{0.400pt}{0.241pt}}
\put(844.0,457.0){\rule[-0.200pt]{0.400pt}{1.204pt}}
\put(845.67,467){\rule{0.400pt}{0.723pt}}
\multiput(845.17,467.00)(1.000,1.500){2}{\rule{0.400pt}{0.361pt}}
\put(846.67,470){\rule{0.400pt}{1.204pt}}
\multiput(846.17,470.00)(1.000,2.500){2}{\rule{0.400pt}{0.602pt}}
\put(846.0,465.0){\rule[-0.200pt]{0.400pt}{0.482pt}}
\put(848,475){\usebox{\plotpoint}}
\put(847.67,475){\rule{0.400pt}{0.723pt}}
\multiput(847.17,475.00)(1.000,1.500){2}{\rule{0.400pt}{0.361pt}}
\put(849,481.67){\rule{0.241pt}{0.400pt}}
\multiput(849.00,481.17)(0.500,1.000){2}{\rule{0.120pt}{0.400pt}}
\put(850,482.67){\rule{0.241pt}{0.400pt}}
\multiput(850.00,482.17)(0.500,1.000){2}{\rule{0.120pt}{0.400pt}}
\put(849.0,478.0){\rule[-0.200pt]{0.400pt}{0.964pt}}
\put(851,487.67){\rule{0.241pt}{0.400pt}}
\multiput(851.00,487.17)(0.500,1.000){2}{\rule{0.120pt}{0.400pt}}
\put(851.67,489){\rule{0.400pt}{0.482pt}}
\multiput(851.17,489.00)(1.000,1.000){2}{\rule{0.400pt}{0.241pt}}
\put(851.0,484.0){\rule[-0.200pt]{0.400pt}{0.964pt}}
\put(852.67,494){\rule{0.400pt}{0.964pt}}
\multiput(852.17,494.00)(1.000,2.000){2}{\rule{0.400pt}{0.482pt}}
\put(853.0,491.0){\rule[-0.200pt]{0.400pt}{0.723pt}}
\put(854,498){\usebox{\plotpoint}}
\put(853.67,498){\rule{0.400pt}{1.204pt}}
\multiput(853.17,498.00)(1.000,2.500){2}{\rule{0.400pt}{0.602pt}}
\put(854.67,503){\rule{0.400pt}{0.482pt}}
\multiput(854.17,503.00)(1.000,1.000){2}{\rule{0.400pt}{0.241pt}}
\put(855.67,506){\rule{0.400pt}{1.204pt}}
\multiput(855.17,506.00)(1.000,2.500){2}{\rule{0.400pt}{0.602pt}}
\put(856.67,508){\rule{0.400pt}{0.723pt}}
\multiput(856.17,509.50)(1.000,-1.500){2}{\rule{0.400pt}{0.361pt}}
\put(856.0,505.0){\usebox{\plotpoint}}
\put(858,515.67){\rule{0.241pt}{0.400pt}}
\multiput(858.00,515.17)(0.500,1.000){2}{\rule{0.120pt}{0.400pt}}
\put(858.67,517){\rule{0.400pt}{0.482pt}}
\multiput(858.17,517.00)(1.000,1.000){2}{\rule{0.400pt}{0.241pt}}
\put(858.0,508.0){\rule[-0.200pt]{0.400pt}{1.927pt}}
\put(860.0,519.0){\rule[-0.200pt]{0.400pt}{0.723pt}}
\put(860.0,522.0){\usebox{\plotpoint}}
\put(860.67,526){\rule{0.400pt}{0.964pt}}
\multiput(860.17,526.00)(1.000,2.000){2}{\rule{0.400pt}{0.482pt}}
\put(861.67,530){\rule{0.400pt}{0.723pt}}
\multiput(861.17,530.00)(1.000,1.500){2}{\rule{0.400pt}{0.361pt}}
\put(861.0,522.0){\rule[-0.200pt]{0.400pt}{0.964pt}}
\put(862.67,536){\rule{0.400pt}{0.482pt}}
\multiput(862.17,537.00)(1.000,-1.000){2}{\rule{0.400pt}{0.241pt}}
\put(863.67,536){\rule{0.400pt}{0.482pt}}
\multiput(863.17,536.00)(1.000,1.000){2}{\rule{0.400pt}{0.241pt}}
\put(863.0,533.0){\rule[-0.200pt]{0.400pt}{1.204pt}}
\put(864.67,542){\rule{0.400pt}{0.723pt}}
\multiput(864.17,542.00)(1.000,1.500){2}{\rule{0.400pt}{0.361pt}}
\put(865.0,538.0){\rule[-0.200pt]{0.400pt}{0.964pt}}
\put(866,545){\usebox{\plotpoint}}
\put(865.67,545){\rule{0.400pt}{0.723pt}}
\multiput(865.17,545.00)(1.000,1.500){2}{\rule{0.400pt}{0.361pt}}
\put(866.67,548){\rule{0.400pt}{0.723pt}}
\multiput(866.17,548.00)(1.000,1.500){2}{\rule{0.400pt}{0.361pt}}
\put(867.67,555){\rule{0.400pt}{0.723pt}}
\multiput(867.17,555.00)(1.000,1.500){2}{\rule{0.400pt}{0.361pt}}
\put(868.67,558){\rule{0.400pt}{0.482pt}}
\multiput(868.17,558.00)(1.000,1.000){2}{\rule{0.400pt}{0.241pt}}
\put(868.0,551.0){\rule[-0.200pt]{0.400pt}{0.964pt}}
\put(870,559.67){\rule{0.241pt}{0.400pt}}
\multiput(870.00,560.17)(0.500,-1.000){2}{\rule{0.120pt}{0.400pt}}
\put(870.0,560.0){\usebox{\plotpoint}}
\put(870.67,563){\rule{0.400pt}{0.964pt}}
\multiput(870.17,563.00)(1.000,2.000){2}{\rule{0.400pt}{0.482pt}}
\put(871.67,567){\rule{0.400pt}{0.723pt}}
\multiput(871.17,567.00)(1.000,1.500){2}{\rule{0.400pt}{0.361pt}}
\put(871.0,560.0){\rule[-0.200pt]{0.400pt}{0.723pt}}
\put(872.67,571){\rule{0.400pt}{0.723pt}}
\multiput(872.17,571.00)(1.000,1.500){2}{\rule{0.400pt}{0.361pt}}
\put(874,572.67){\rule{0.241pt}{0.400pt}}
\multiput(874.00,573.17)(0.500,-1.000){2}{\rule{0.120pt}{0.400pt}}
\put(873.0,570.0){\usebox{\plotpoint}}
\put(874.67,576){\rule{0.400pt}{0.964pt}}
\multiput(874.17,576.00)(1.000,2.000){2}{\rule{0.400pt}{0.482pt}}
\put(875.0,573.0){\rule[-0.200pt]{0.400pt}{0.723pt}}
\put(876.0,580.0){\usebox{\plotpoint}}
\put(876.67,584){\rule{0.400pt}{0.723pt}}
\multiput(876.17,584.00)(1.000,1.500){2}{\rule{0.400pt}{0.361pt}}
\put(877.0,580.0){\rule[-0.200pt]{0.400pt}{0.964pt}}
\put(878,587.67){\rule{0.241pt}{0.400pt}}
\multiput(878.00,587.17)(0.500,1.000){2}{\rule{0.120pt}{0.400pt}}
\put(878.67,589){\rule{0.400pt}{0.723pt}}
\multiput(878.17,589.00)(1.000,1.500){2}{\rule{0.400pt}{0.361pt}}
\put(878.0,587.0){\usebox{\plotpoint}}
\put(880,592){\usebox{\plotpoint}}
\put(879.67,592){\rule{0.400pt}{1.204pt}}
\multiput(879.17,592.00)(1.000,2.500){2}{\rule{0.400pt}{0.602pt}}
\put(881,595.67){\rule{0.241pt}{0.400pt}}
\multiput(881.00,596.17)(0.500,-1.000){2}{\rule{0.120pt}{0.400pt}}
\put(882.0,596.0){\usebox{\plotpoint}}
\put(882.0,597.0){\usebox{\plotpoint}}
\put(882.67,600){\rule{0.400pt}{0.482pt}}
\multiput(882.17,600.00)(1.000,1.000){2}{\rule{0.400pt}{0.241pt}}
\put(884,601.67){\rule{0.241pt}{0.400pt}}
\multiput(884.00,601.17)(0.500,1.000){2}{\rule{0.120pt}{0.400pt}}
\put(883.0,597.0){\rule[-0.200pt]{0.400pt}{0.723pt}}
\put(884.67,604){\rule{0.400pt}{1.204pt}}
\multiput(884.17,606.50)(1.000,-2.500){2}{\rule{0.400pt}{0.602pt}}
\put(885.67,604){\rule{0.400pt}{0.723pt}}
\multiput(885.17,604.00)(1.000,1.500){2}{\rule{0.400pt}{0.361pt}}
\put(885.0,603.0){\rule[-0.200pt]{0.400pt}{1.445pt}}
\put(887.0,607.0){\rule[-0.200pt]{0.400pt}{0.964pt}}
\put(887.0,611.0){\usebox{\plotpoint}}
\put(887.67,609){\rule{0.400pt}{0.723pt}}
\multiput(887.17,609.00)(1.000,1.500){2}{\rule{0.400pt}{0.361pt}}
\put(888.67,612){\rule{0.400pt}{0.482pt}}
\multiput(888.17,612.00)(1.000,1.000){2}{\rule{0.400pt}{0.241pt}}
\put(888.0,609.0){\rule[-0.200pt]{0.400pt}{0.482pt}}
\put(889.67,616){\rule{0.400pt}{0.482pt}}
\multiput(889.17,616.00)(1.000,1.000){2}{\rule{0.400pt}{0.241pt}}
\put(890.67,614){\rule{0.400pt}{0.964pt}}
\multiput(890.17,616.00)(1.000,-2.000){2}{\rule{0.400pt}{0.482pt}}
\put(890.0,614.0){\rule[-0.200pt]{0.400pt}{0.482pt}}
\put(891.67,615){\rule{0.400pt}{0.964pt}}
\multiput(891.17,615.00)(1.000,2.000){2}{\rule{0.400pt}{0.482pt}}
\put(892.0,614.0){\usebox{\plotpoint}}
\put(892.67,617){\rule{0.400pt}{1.445pt}}
\multiput(892.17,617.00)(1.000,3.000){2}{\rule{0.400pt}{0.723pt}}
\put(893.67,617){\rule{0.400pt}{1.445pt}}
\multiput(893.17,620.00)(1.000,-3.000){2}{\rule{0.400pt}{0.723pt}}
\put(893.0,617.0){\rule[-0.200pt]{0.400pt}{0.482pt}}
\put(894.67,618){\rule{0.400pt}{1.204pt}}
\multiput(894.17,618.00)(1.000,2.500){2}{\rule{0.400pt}{0.602pt}}
\put(896,621.67){\rule{0.241pt}{0.400pt}}
\multiput(896.00,622.17)(0.500,-1.000){2}{\rule{0.120pt}{0.400pt}}
\put(895.0,617.0){\usebox{\plotpoint}}
\put(897,622.67){\rule{0.241pt}{0.400pt}}
\multiput(897.00,622.17)(0.500,1.000){2}{\rule{0.120pt}{0.400pt}}
\put(897.0,622.0){\usebox{\plotpoint}}
\put(898.0,624.0){\usebox{\plotpoint}}
\put(899.0,621.0){\rule[-0.200pt]{0.400pt}{0.723pt}}
\put(899.0,621.0){\usebox{\plotpoint}}
\put(899.67,620){\rule{0.400pt}{0.964pt}}
\multiput(899.17,622.00)(1.000,-2.000){2}{\rule{0.400pt}{0.482pt}}
\put(900.67,618){\rule{0.400pt}{0.482pt}}
\multiput(900.17,619.00)(1.000,-1.000){2}{\rule{0.400pt}{0.241pt}}
\put(900.0,621.0){\rule[-0.200pt]{0.400pt}{0.723pt}}
\put(902,621.67){\rule{0.241pt}{0.400pt}}
\multiput(902.00,622.17)(0.500,-1.000){2}{\rule{0.120pt}{0.400pt}}
\put(902.67,619){\rule{0.400pt}{0.723pt}}
\multiput(902.17,620.50)(1.000,-1.500){2}{\rule{0.400pt}{0.361pt}}
\put(902.0,618.0){\rule[-0.200pt]{0.400pt}{1.204pt}}
\put(904,619){\usebox{\plotpoint}}
\put(904,618.67){\rule{0.241pt}{0.400pt}}
\multiput(904.00,618.17)(0.500,1.000){2}{\rule{0.120pt}{0.400pt}}
\put(904.67,614){\rule{0.400pt}{1.927pt}}
\multiput(904.17,618.00)(1.000,-4.000){2}{\rule{0.400pt}{0.964pt}}
\put(906,613.67){\rule{0.241pt}{0.400pt}}
\multiput(906.00,613.17)(0.500,1.000){2}{\rule{0.120pt}{0.400pt}}
\put(905.0,620.0){\rule[-0.200pt]{0.400pt}{0.482pt}}
\put(907,615.67){\rule{0.241pt}{0.400pt}}
\multiput(907.00,615.17)(0.500,1.000){2}{\rule{0.120pt}{0.400pt}}
\put(908,616.67){\rule{0.241pt}{0.400pt}}
\multiput(908.00,616.17)(0.500,1.000){2}{\rule{0.120pt}{0.400pt}}
\put(907.0,615.0){\usebox{\plotpoint}}
\put(909,612.67){\rule{0.241pt}{0.400pt}}
\multiput(909.00,613.17)(0.500,-1.000){2}{\rule{0.120pt}{0.400pt}}
\put(909.0,614.0){\rule[-0.200pt]{0.400pt}{0.964pt}}
\put(909.67,611){\rule{0.400pt}{1.204pt}}
\multiput(909.17,613.50)(1.000,-2.500){2}{\rule{0.400pt}{0.602pt}}
\put(911,609.67){\rule{0.241pt}{0.400pt}}
\multiput(911.00,610.17)(0.500,-1.000){2}{\rule{0.120pt}{0.400pt}}
\put(910.0,613.0){\rule[-0.200pt]{0.400pt}{0.723pt}}
\put(911.67,605){\rule{0.400pt}{0.964pt}}
\multiput(911.17,605.00)(1.000,2.000){2}{\rule{0.400pt}{0.482pt}}
\put(912.67,604){\rule{0.400pt}{1.204pt}}
\multiput(912.17,606.50)(1.000,-2.500){2}{\rule{0.400pt}{0.602pt}}
\put(912.0,605.0){\rule[-0.200pt]{0.400pt}{1.204pt}}
\put(914,604){\usebox{\plotpoint}}
\put(913.67,599){\rule{0.400pt}{1.204pt}}
\multiput(913.17,601.50)(1.000,-2.500){2}{\rule{0.400pt}{0.602pt}}
\put(915,599){\usebox{\plotpoint}}
\put(914.67,597){\rule{0.400pt}{0.482pt}}
\multiput(914.17,598.00)(1.000,-1.000){2}{\rule{0.400pt}{0.241pt}}
\put(915.67,597){\rule{0.400pt}{1.204pt}}
\multiput(915.17,597.00)(1.000,2.500){2}{\rule{0.400pt}{0.602pt}}
\put(917,594.67){\rule{0.241pt}{0.400pt}}
\multiput(917.00,595.17)(0.500,-1.000){2}{\rule{0.120pt}{0.400pt}}
\put(917.67,587){\rule{0.400pt}{1.927pt}}
\multiput(917.17,591.00)(1.000,-4.000){2}{\rule{0.400pt}{0.964pt}}
\put(917.0,596.0){\rule[-0.200pt]{0.400pt}{1.445pt}}
\put(918.67,590){\rule{0.400pt}{0.723pt}}
\multiput(918.17,591.50)(1.000,-1.500){2}{\rule{0.400pt}{0.361pt}}
\put(919.0,587.0){\rule[-0.200pt]{0.400pt}{1.445pt}}
\put(919.67,583){\rule{0.400pt}{0.723pt}}
\multiput(919.17,584.50)(1.000,-1.500){2}{\rule{0.400pt}{0.361pt}}
\put(921,581.67){\rule{0.241pt}{0.400pt}}
\multiput(921.00,582.17)(0.500,-1.000){2}{\rule{0.120pt}{0.400pt}}
\put(920.0,586.0){\rule[-0.200pt]{0.400pt}{0.964pt}}
\put(921.67,574){\rule{0.400pt}{1.445pt}}
\multiput(921.17,577.00)(1.000,-3.000){2}{\rule{0.400pt}{0.723pt}}
\put(923,573.67){\rule{0.241pt}{0.400pt}}
\multiput(923.00,573.17)(0.500,1.000){2}{\rule{0.120pt}{0.400pt}}
\put(922.0,580.0){\rule[-0.200pt]{0.400pt}{0.482pt}}
\put(923.67,569){\rule{0.400pt}{0.964pt}}
\multiput(923.17,571.00)(1.000,-2.000){2}{\rule{0.400pt}{0.482pt}}
\put(924.0,573.0){\rule[-0.200pt]{0.400pt}{0.482pt}}
\put(924.67,563){\rule{0.400pt}{1.686pt}}
\multiput(924.17,566.50)(1.000,-3.500){2}{\rule{0.400pt}{0.843pt}}
\put(925.67,561){\rule{0.400pt}{0.482pt}}
\multiput(925.17,562.00)(1.000,-1.000){2}{\rule{0.400pt}{0.241pt}}
\put(925.0,569.0){\usebox{\plotpoint}}
\put(926.67,557){\rule{0.400pt}{0.482pt}}
\multiput(926.17,558.00)(1.000,-1.000){2}{\rule{0.400pt}{0.241pt}}
\put(927.67,555){\rule{0.400pt}{0.482pt}}
\multiput(927.17,556.00)(1.000,-1.000){2}{\rule{0.400pt}{0.241pt}}
\put(927.0,559.0){\rule[-0.200pt]{0.400pt}{0.482pt}}
\put(928.67,547){\rule{0.400pt}{0.482pt}}
\multiput(928.17,548.00)(1.000,-1.000){2}{\rule{0.400pt}{0.241pt}}
\put(929.0,549.0){\rule[-0.200pt]{0.400pt}{1.445pt}}
\put(929.67,541){\rule{0.400pt}{0.723pt}}
\multiput(929.17,542.50)(1.000,-1.500){2}{\rule{0.400pt}{0.361pt}}
\put(930.67,539){\rule{0.400pt}{0.482pt}}
\multiput(930.17,540.00)(1.000,-1.000){2}{\rule{0.400pt}{0.241pt}}
\put(930.0,544.0){\rule[-0.200pt]{0.400pt}{0.723pt}}
\put(931.67,533){\rule{0.400pt}{0.723pt}}
\multiput(931.17,534.50)(1.000,-1.500){2}{\rule{0.400pt}{0.361pt}}
\put(932.67,528){\rule{0.400pt}{1.204pt}}
\multiput(932.17,530.50)(1.000,-2.500){2}{\rule{0.400pt}{0.602pt}}
\put(932.0,536.0){\rule[-0.200pt]{0.400pt}{0.723pt}}
\put(934,523.67){\rule{0.241pt}{0.400pt}}
\multiput(934.00,524.17)(0.500,-1.000){2}{\rule{0.120pt}{0.400pt}}
\put(934.0,525.0){\rule[-0.200pt]{0.400pt}{0.723pt}}
\put(934.67,517){\rule{0.400pt}{0.723pt}}
\multiput(934.17,518.50)(1.000,-1.500){2}{\rule{0.400pt}{0.361pt}}
\put(935.67,514){\rule{0.400pt}{0.723pt}}
\multiput(935.17,515.50)(1.000,-1.500){2}{\rule{0.400pt}{0.361pt}}
\put(935.0,520.0){\rule[-0.200pt]{0.400pt}{0.964pt}}
\put(937.0,509.0){\rule[-0.200pt]{0.400pt}{1.204pt}}
\put(937.67,503){\rule{0.400pt}{1.445pt}}
\multiput(937.17,506.00)(1.000,-3.000){2}{\rule{0.400pt}{0.723pt}}
\put(937.0,509.0){\usebox{\plotpoint}}
\put(938.67,495){\rule{0.400pt}{0.964pt}}
\multiput(938.17,497.00)(1.000,-2.000){2}{\rule{0.400pt}{0.482pt}}
\put(939.0,499.0){\rule[-0.200pt]{0.400pt}{0.964pt}}
\put(939.67,491){\rule{0.400pt}{1.204pt}}
\multiput(939.17,493.50)(1.000,-2.500){2}{\rule{0.400pt}{0.602pt}}
\put(940.67,487){\rule{0.400pt}{0.964pt}}
\multiput(940.17,489.00)(1.000,-2.000){2}{\rule{0.400pt}{0.482pt}}
\put(940.0,495.0){\usebox{\plotpoint}}
\put(941.67,478){\rule{0.400pt}{1.204pt}}
\multiput(941.17,480.50)(1.000,-2.500){2}{\rule{0.400pt}{0.602pt}}
\put(943,476.67){\rule{0.241pt}{0.400pt}}
\multiput(943.00,477.17)(0.500,-1.000){2}{\rule{0.120pt}{0.400pt}}
\put(942.0,483.0){\rule[-0.200pt]{0.400pt}{0.964pt}}
\put(943.67,469){\rule{0.400pt}{0.482pt}}
\multiput(943.17,470.00)(1.000,-1.000){2}{\rule{0.400pt}{0.241pt}}
\put(944.0,471.0){\rule[-0.200pt]{0.400pt}{1.445pt}}
\put(944.67,462){\rule{0.400pt}{0.964pt}}
\multiput(944.17,464.00)(1.000,-2.000){2}{\rule{0.400pt}{0.482pt}}
\put(945.67,458){\rule{0.400pt}{0.964pt}}
\multiput(945.17,460.00)(1.000,-2.000){2}{\rule{0.400pt}{0.482pt}}
\put(945.0,466.0){\rule[-0.200pt]{0.400pt}{0.723pt}}
\put(946.67,452){\rule{0.400pt}{0.482pt}}
\multiput(946.17,453.00)(1.000,-1.000){2}{\rule{0.400pt}{0.241pt}}
\put(947.67,447){\rule{0.400pt}{1.204pt}}
\multiput(947.17,449.50)(1.000,-2.500){2}{\rule{0.400pt}{0.602pt}}
\put(947.0,454.0){\rule[-0.200pt]{0.400pt}{0.964pt}}
\put(948.67,442){\rule{0.400pt}{0.723pt}}
\multiput(948.17,443.50)(1.000,-1.500){2}{\rule{0.400pt}{0.361pt}}
\put(949.0,445.0){\rule[-0.200pt]{0.400pt}{0.482pt}}
\put(949.67,435){\rule{0.400pt}{0.723pt}}
\multiput(949.17,436.50)(1.000,-1.500){2}{\rule{0.400pt}{0.361pt}}
\put(950.67,431){\rule{0.400pt}{0.964pt}}
\multiput(950.17,433.00)(1.000,-2.000){2}{\rule{0.400pt}{0.482pt}}
\put(950.0,438.0){\rule[-0.200pt]{0.400pt}{0.964pt}}
\put(951.67,423){\rule{0.400pt}{0.964pt}}
\multiput(951.17,425.00)(1.000,-2.000){2}{\rule{0.400pt}{0.482pt}}
\put(952.67,419){\rule{0.400pt}{0.964pt}}
\multiput(952.17,421.00)(1.000,-2.000){2}{\rule{0.400pt}{0.482pt}}
\put(952.0,427.0){\rule[-0.200pt]{0.400pt}{0.964pt}}
\put(953.67,413){\rule{0.400pt}{0.964pt}}
\multiput(953.17,415.00)(1.000,-2.000){2}{\rule{0.400pt}{0.482pt}}
\put(954.0,417.0){\rule[-0.200pt]{0.400pt}{0.482pt}}
\put(954.67,404){\rule{0.400pt}{1.445pt}}
\multiput(954.17,407.00)(1.000,-3.000){2}{\rule{0.400pt}{0.723pt}}
\put(956,402.67){\rule{0.241pt}{0.400pt}}
\multiput(956.00,403.17)(0.500,-1.000){2}{\rule{0.120pt}{0.400pt}}
\put(955.0,410.0){\rule[-0.200pt]{0.400pt}{0.723pt}}
\put(956.67,396){\rule{0.400pt}{0.482pt}}
\multiput(956.17,397.00)(1.000,-1.000){2}{\rule{0.400pt}{0.241pt}}
\put(957.67,391){\rule{0.400pt}{1.204pt}}
\multiput(957.17,393.50)(1.000,-2.500){2}{\rule{0.400pt}{0.602pt}}
\put(957.0,398.0){\rule[-0.200pt]{0.400pt}{1.204pt}}
\put(958.67,385){\rule{0.400pt}{0.723pt}}
\multiput(958.17,386.50)(1.000,-1.500){2}{\rule{0.400pt}{0.361pt}}
\put(959.0,388.0){\rule[-0.200pt]{0.400pt}{0.723pt}}
\put(959.67,379){\rule{0.400pt}{0.723pt}}
\multiput(959.17,380.50)(1.000,-1.500){2}{\rule{0.400pt}{0.361pt}}
\put(960.67,375){\rule{0.400pt}{0.964pt}}
\multiput(960.17,377.00)(1.000,-2.000){2}{\rule{0.400pt}{0.482pt}}
\put(960.0,382.0){\rule[-0.200pt]{0.400pt}{0.723pt}}
\put(961.67,368){\rule{0.400pt}{0.964pt}}
\multiput(961.17,370.00)(1.000,-2.000){2}{\rule{0.400pt}{0.482pt}}
\put(962.67,366){\rule{0.400pt}{0.482pt}}
\multiput(962.17,367.00)(1.000,-1.000){2}{\rule{0.400pt}{0.241pt}}
\put(962.0,372.0){\rule[-0.200pt]{0.400pt}{0.723pt}}
\put(963.67,359){\rule{0.400pt}{0.723pt}}
\multiput(963.17,360.50)(1.000,-1.500){2}{\rule{0.400pt}{0.361pt}}
\put(964.0,362.0){\rule[-0.200pt]{0.400pt}{0.964pt}}
\put(964.67,352){\rule{0.400pt}{0.723pt}}
\multiput(964.17,353.50)(1.000,-1.500){2}{\rule{0.400pt}{0.361pt}}
\put(965.67,349){\rule{0.400pt}{0.723pt}}
\multiput(965.17,350.50)(1.000,-1.500){2}{\rule{0.400pt}{0.361pt}}
\put(965.0,355.0){\rule[-0.200pt]{0.400pt}{0.964pt}}
\put(967,342.67){\rule{0.241pt}{0.400pt}}
\multiput(967.00,343.17)(0.500,-1.000){2}{\rule{0.120pt}{0.400pt}}
\put(967.0,344.0){\rule[-0.200pt]{0.400pt}{1.204pt}}
\put(967.67,336){\rule{0.400pt}{0.482pt}}
\multiput(967.17,337.00)(1.000,-1.000){2}{\rule{0.400pt}{0.241pt}}
\put(968.67,333){\rule{0.400pt}{0.723pt}}
\multiput(968.17,334.50)(1.000,-1.500){2}{\rule{0.400pt}{0.361pt}}
\put(968.0,338.0){\rule[-0.200pt]{0.400pt}{1.204pt}}
\put(969.67,328){\rule{0.400pt}{0.482pt}}
\multiput(969.17,329.00)(1.000,-1.000){2}{\rule{0.400pt}{0.241pt}}
\put(970.67,325){\rule{0.400pt}{0.723pt}}
\multiput(970.17,326.50)(1.000,-1.500){2}{\rule{0.400pt}{0.361pt}}
\put(970.0,330.0){\rule[-0.200pt]{0.400pt}{0.723pt}}
\put(971.67,318){\rule{0.400pt}{0.723pt}}
\multiput(971.17,319.50)(1.000,-1.500){2}{\rule{0.400pt}{0.361pt}}
\put(972.0,321.0){\rule[-0.200pt]{0.400pt}{0.964pt}}
\put(972.67,314){\rule{0.400pt}{0.482pt}}
\multiput(972.17,315.00)(1.000,-1.000){2}{\rule{0.400pt}{0.241pt}}
\put(973.67,310){\rule{0.400pt}{0.964pt}}
\multiput(973.17,312.00)(1.000,-2.000){2}{\rule{0.400pt}{0.482pt}}
\put(973.0,316.0){\rule[-0.200pt]{0.400pt}{0.482pt}}
\put(974.67,304){\rule{0.400pt}{0.964pt}}
\multiput(974.17,306.00)(1.000,-2.000){2}{\rule{0.400pt}{0.482pt}}
\put(975.67,302){\rule{0.400pt}{0.482pt}}
\multiput(975.17,303.00)(1.000,-1.000){2}{\rule{0.400pt}{0.241pt}}
\put(975.0,308.0){\rule[-0.200pt]{0.400pt}{0.482pt}}
\put(976.67,296){\rule{0.400pt}{0.723pt}}
\multiput(976.17,297.50)(1.000,-1.500){2}{\rule{0.400pt}{0.361pt}}
\put(977.0,299.0){\rule[-0.200pt]{0.400pt}{0.723pt}}
\put(978.0,292.0){\rule[-0.200pt]{0.400pt}{0.964pt}}
\put(978.67,288){\rule{0.400pt}{0.964pt}}
\multiput(978.17,290.00)(1.000,-2.000){2}{\rule{0.400pt}{0.482pt}}
\put(978.0,292.0){\usebox{\plotpoint}}
\put(979.67,284){\rule{0.400pt}{0.723pt}}
\multiput(979.17,285.50)(1.000,-1.500){2}{\rule{0.400pt}{0.361pt}}
\put(980.67,281){\rule{0.400pt}{0.723pt}}
\multiput(980.17,282.50)(1.000,-1.500){2}{\rule{0.400pt}{0.361pt}}
\put(980.0,287.0){\usebox{\plotpoint}}
\put(981.67,276){\rule{0.400pt}{0.723pt}}
\multiput(981.17,277.50)(1.000,-1.500){2}{\rule{0.400pt}{0.361pt}}
\put(982.0,279.0){\rule[-0.200pt]{0.400pt}{0.482pt}}
\put(982.67,271){\rule{0.400pt}{0.482pt}}
\multiput(982.17,272.00)(1.000,-1.000){2}{\rule{0.400pt}{0.241pt}}
\put(983.67,269){\rule{0.400pt}{0.482pt}}
\multiput(983.17,270.00)(1.000,-1.000){2}{\rule{0.400pt}{0.241pt}}
\put(983.0,273.0){\rule[-0.200pt]{0.400pt}{0.723pt}}
\put(984.67,265){\rule{0.400pt}{0.482pt}}
\multiput(984.17,266.00)(1.000,-1.000){2}{\rule{0.400pt}{0.241pt}}
\put(985.67,262){\rule{0.400pt}{0.723pt}}
\multiput(985.17,263.50)(1.000,-1.500){2}{\rule{0.400pt}{0.361pt}}
\put(985.0,267.0){\rule[-0.200pt]{0.400pt}{0.482pt}}
\put(986.67,257){\rule{0.400pt}{0.964pt}}
\multiput(986.17,259.00)(1.000,-2.000){2}{\rule{0.400pt}{0.482pt}}
\put(987.0,261.0){\usebox{\plotpoint}}
\put(987.67,253){\rule{0.400pt}{0.482pt}}
\multiput(987.17,254.00)(1.000,-1.000){2}{\rule{0.400pt}{0.241pt}}
\put(988.67,251){\rule{0.400pt}{0.482pt}}
\multiput(988.17,252.00)(1.000,-1.000){2}{\rule{0.400pt}{0.241pt}}
\put(988.0,255.0){\rule[-0.200pt]{0.400pt}{0.482pt}}
\put(990,247.67){\rule{0.241pt}{0.400pt}}
\multiput(990.00,248.17)(0.500,-1.000){2}{\rule{0.120pt}{0.400pt}}
\put(990.0,249.0){\rule[-0.200pt]{0.400pt}{0.482pt}}
\put(991,243.67){\rule{0.241pt}{0.400pt}}
\multiput(991.00,244.17)(0.500,-1.000){2}{\rule{0.120pt}{0.400pt}}
\put(991.67,241){\rule{0.400pt}{0.723pt}}
\multiput(991.17,242.50)(1.000,-1.500){2}{\rule{0.400pt}{0.361pt}}
\put(991.0,245.0){\rule[-0.200pt]{0.400pt}{0.723pt}}
\put(993,237.67){\rule{0.241pt}{0.400pt}}
\multiput(993.00,238.17)(0.500,-1.000){2}{\rule{0.120pt}{0.400pt}}
\put(993.67,235){\rule{0.400pt}{0.723pt}}
\multiput(993.17,236.50)(1.000,-1.500){2}{\rule{0.400pt}{0.361pt}}
\put(993.0,239.0){\rule[-0.200pt]{0.400pt}{0.482pt}}
\put(995,231.67){\rule{0.241pt}{0.400pt}}
\multiput(995.00,232.17)(0.500,-1.000){2}{\rule{0.120pt}{0.400pt}}
\put(995.0,233.0){\rule[-0.200pt]{0.400pt}{0.482pt}}
\put(996.0,229.0){\rule[-0.200pt]{0.400pt}{0.723pt}}
\put(996.67,226){\rule{0.400pt}{0.723pt}}
\multiput(996.17,227.50)(1.000,-1.500){2}{\rule{0.400pt}{0.361pt}}
\put(996.0,229.0){\usebox{\plotpoint}}
\put(997.67,223){\rule{0.400pt}{0.482pt}}
\multiput(997.17,224.00)(1.000,-1.000){2}{\rule{0.400pt}{0.241pt}}
\put(998.67,221){\rule{0.400pt}{0.482pt}}
\multiput(998.17,222.00)(1.000,-1.000){2}{\rule{0.400pt}{0.241pt}}
\put(998.0,225.0){\usebox{\plotpoint}}
\put(1000,218.67){\rule{0.241pt}{0.400pt}}
\multiput(1000.00,219.17)(0.500,-1.000){2}{\rule{0.120pt}{0.400pt}}
\put(1000.0,220.0){\usebox{\plotpoint}}
\put(1000.67,215){\rule{0.400pt}{0.482pt}}
\multiput(1000.17,216.00)(1.000,-1.000){2}{\rule{0.400pt}{0.241pt}}
\put(1001.67,213){\rule{0.400pt}{0.482pt}}
\multiput(1001.17,214.00)(1.000,-1.000){2}{\rule{0.400pt}{0.241pt}}
\put(1001.0,217.0){\rule[-0.200pt]{0.400pt}{0.482pt}}
\put(1003,209.67){\rule{0.241pt}{0.400pt}}
\multiput(1003.00,210.17)(0.500,-1.000){2}{\rule{0.120pt}{0.400pt}}
\put(1003.0,211.0){\rule[-0.200pt]{0.400pt}{0.482pt}}
\put(1003.67,207){\rule{0.400pt}{0.482pt}}
\multiput(1003.17,208.00)(1.000,-1.000){2}{\rule{0.400pt}{0.241pt}}
\put(1005,205.67){\rule{0.241pt}{0.400pt}}
\multiput(1005.00,206.17)(0.500,-1.000){2}{\rule{0.120pt}{0.400pt}}
\put(1004.0,209.0){\usebox{\plotpoint}}
\put(1005.67,203){\rule{0.400pt}{0.482pt}}
\multiput(1005.17,204.00)(1.000,-1.000){2}{\rule{0.400pt}{0.241pt}}
\put(1006.0,205.0){\usebox{\plotpoint}}
\put(1007.0,203.0){\usebox{\plotpoint}}
\put(1008.0,200.0){\rule[-0.200pt]{0.400pt}{0.723pt}}
\put(1008.0,200.0){\usebox{\plotpoint}}
\put(1009,196.67){\rule{0.241pt}{0.400pt}}
\multiput(1009.00,197.17)(0.500,-1.000){2}{\rule{0.120pt}{0.400pt}}
\put(1009.67,195){\rule{0.400pt}{0.482pt}}
\multiput(1009.17,196.00)(1.000,-1.000){2}{\rule{0.400pt}{0.241pt}}
\put(1009.0,198.0){\rule[-0.200pt]{0.400pt}{0.482pt}}
\put(1011,192.67){\rule{0.241pt}{0.400pt}}
\multiput(1011.00,193.17)(0.500,-1.000){2}{\rule{0.120pt}{0.400pt}}
\put(1012,191.67){\rule{0.241pt}{0.400pt}}
\multiput(1012.00,192.17)(0.500,-1.000){2}{\rule{0.120pt}{0.400pt}}
\put(1011.0,194.0){\usebox{\plotpoint}}
\put(1013,189.67){\rule{0.241pt}{0.400pt}}
\multiput(1013.00,190.17)(0.500,-1.000){2}{\rule{0.120pt}{0.400pt}}
\put(1013.0,191.0){\usebox{\plotpoint}}
\put(1014.0,188.0){\rule[-0.200pt]{0.400pt}{0.482pt}}
\put(1015,186.67){\rule{0.241pt}{0.400pt}}
\multiput(1015.00,187.17)(0.500,-1.000){2}{\rule{0.120pt}{0.400pt}}
\put(1014.0,188.0){\usebox{\plotpoint}}
\put(1016,183.67){\rule{0.241pt}{0.400pt}}
\multiput(1016.00,184.17)(0.500,-1.000){2}{\rule{0.120pt}{0.400pt}}
\put(1016.0,185.0){\rule[-0.200pt]{0.400pt}{0.482pt}}
\put(1017,181.67){\rule{0.241pt}{0.400pt}}
\multiput(1017.00,182.17)(0.500,-1.000){2}{\rule{0.120pt}{0.400pt}}
\put(1018,180.67){\rule{0.241pt}{0.400pt}}
\multiput(1018.00,181.17)(0.500,-1.000){2}{\rule{0.120pt}{0.400pt}}
\put(1017.0,183.0){\usebox{\plotpoint}}
\put(1019,178.67){\rule{0.241pt}{0.400pt}}
\multiput(1019.00,178.17)(0.500,1.000){2}{\rule{0.120pt}{0.400pt}}
\put(1020,178.67){\rule{0.241pt}{0.400pt}}
\multiput(1020.00,179.17)(0.500,-1.000){2}{\rule{0.120pt}{0.400pt}}
\put(1019.0,179.0){\rule[-0.200pt]{0.400pt}{0.482pt}}
\put(1021,176.67){\rule{0.241pt}{0.400pt}}
\multiput(1021.00,177.17)(0.500,-1.000){2}{\rule{0.120pt}{0.400pt}}
\put(1021.0,178.0){\usebox{\plotpoint}}
\put(1022,174.67){\rule{0.241pt}{0.400pt}}
\multiput(1022.00,175.17)(0.500,-1.000){2}{\rule{0.120pt}{0.400pt}}
\put(1023,173.67){\rule{0.241pt}{0.400pt}}
\multiput(1023.00,174.17)(0.500,-1.000){2}{\rule{0.120pt}{0.400pt}}
\put(1022.0,176.0){\usebox{\plotpoint}}
\put(1024,171.67){\rule{0.241pt}{0.400pt}}
\multiput(1024.00,172.17)(0.500,-1.000){2}{\rule{0.120pt}{0.400pt}}
\put(1025,170.67){\rule{0.241pt}{0.400pt}}
\multiput(1025.00,171.17)(0.500,-1.000){2}{\rule{0.120pt}{0.400pt}}
\put(1024.0,173.0){\usebox{\plotpoint}}
\put(1026,168.67){\rule{0.241pt}{0.400pt}}
\multiput(1026.00,169.17)(0.500,-1.000){2}{\rule{0.120pt}{0.400pt}}
\put(1026.0,170.0){\usebox{\plotpoint}}
\put(1027,169){\usebox{\plotpoint}}
\put(1027.67,167){\rule{0.400pt}{0.482pt}}
\multiput(1027.17,168.00)(1.000,-1.000){2}{\rule{0.400pt}{0.241pt}}
\put(1027.0,169.0){\usebox{\plotpoint}}
\put(1029.0,166.0){\usebox{\plotpoint}}
\put(1029.0,166.0){\usebox{\plotpoint}}
\put(1030,163.67){\rule{0.241pt}{0.400pt}}
\multiput(1030.00,164.17)(0.500,-1.000){2}{\rule{0.120pt}{0.400pt}}
\put(1030.0,165.0){\usebox{\plotpoint}}
\put(1031.0,164.0){\usebox{\plotpoint}}
\put(1032,161.67){\rule{0.241pt}{0.400pt}}
\multiput(1032.00,162.17)(0.500,-1.000){2}{\rule{0.120pt}{0.400pt}}
\put(1032.0,163.0){\usebox{\plotpoint}}
\put(1033.0,162.0){\usebox{\plotpoint}}
\put(1034.0,161.0){\usebox{\plotpoint}}
\put(1034.0,161.0){\usebox{\plotpoint}}
\put(1035,158.67){\rule{0.241pt}{0.400pt}}
\multiput(1035.00,159.17)(0.500,-1.000){2}{\rule{0.120pt}{0.400pt}}
\put(1036,157.67){\rule{0.241pt}{0.400pt}}
\multiput(1036.00,158.17)(0.500,-1.000){2}{\rule{0.120pt}{0.400pt}}
\put(1035.0,160.0){\usebox{\plotpoint}}
\put(1037,157.67){\rule{0.241pt}{0.400pt}}
\multiput(1037.00,158.17)(0.500,-1.000){2}{\rule{0.120pt}{0.400pt}}
\put(1038,156.67){\rule{0.241pt}{0.400pt}}
\multiput(1038.00,157.17)(0.500,-1.000){2}{\rule{0.120pt}{0.400pt}}
\put(1037.0,158.0){\usebox{\plotpoint}}
\put(1039,157){\usebox{\plotpoint}}
\put(1039,155.67){\rule{0.241pt}{0.400pt}}
\multiput(1039.00,156.17)(0.500,-1.000){2}{\rule{0.120pt}{0.400pt}}
\put(1040,156){\usebox{\plotpoint}}
\put(1041,154.67){\rule{0.241pt}{0.400pt}}
\multiput(1041.00,155.17)(0.500,-1.000){2}{\rule{0.120pt}{0.400pt}}
\put(1040.0,156.0){\usebox{\plotpoint}}
\put(1042.0,154.0){\usebox{\plotpoint}}
\put(1042.0,154.0){\usebox{\plotpoint}}
\put(1043,151.67){\rule{0.241pt}{0.400pt}}
\multiput(1043.00,152.17)(0.500,-1.000){2}{\rule{0.120pt}{0.400pt}}
\put(1043.0,153.0){\usebox{\plotpoint}}
\put(1045,150.67){\rule{0.241pt}{0.400pt}}
\multiput(1045.00,151.17)(0.500,-1.000){2}{\rule{0.120pt}{0.400pt}}
\put(1044.0,152.0){\usebox{\plotpoint}}
\put(1047.67,149){\rule{0.400pt}{0.482pt}}
\multiput(1047.17,150.00)(1.000,-1.000){2}{\rule{0.400pt}{0.241pt}}
\put(1046.0,151.0){\rule[-0.200pt]{0.482pt}{0.400pt}}
\put(1049.0,149.0){\usebox{\plotpoint}}
\put(1050.0,148.0){\usebox{\plotpoint}}
\put(1050.0,148.0){\usebox{\plotpoint}}
\put(1051.0,147.0){\usebox{\plotpoint}}
\put(1054,145.67){\rule{0.241pt}{0.400pt}}
\multiput(1054.00,146.17)(0.500,-1.000){2}{\rule{0.120pt}{0.400pt}}
\put(1051.0,147.0){\rule[-0.200pt]{0.723pt}{0.400pt}}
\put(1055,146){\usebox{\plotpoint}}
\put(1055,144.67){\rule{0.241pt}{0.400pt}}
\multiput(1055.00,145.17)(0.500,-1.000){2}{\rule{0.120pt}{0.400pt}}
\put(1056,145){\usebox{\plotpoint}}
\put(1056.0,145.0){\rule[-0.200pt]{0.482pt}{0.400pt}}
\put(1058.0,144.0){\usebox{\plotpoint}}
\put(1060,142.67){\rule{0.241pt}{0.400pt}}
\multiput(1060.00,143.17)(0.500,-1.000){2}{\rule{0.120pt}{0.400pt}}
\put(1058.0,144.0){\rule[-0.200pt]{0.482pt}{0.400pt}}
\put(1061,143){\usebox{\plotpoint}}
\put(1062,141.67){\rule{0.241pt}{0.400pt}}
\multiput(1062.00,142.17)(0.500,-1.000){2}{\rule{0.120pt}{0.400pt}}
\put(1061.0,143.0){\usebox{\plotpoint}}
\put(1063,142){\usebox{\plotpoint}}
\put(1063.0,142.0){\usebox{\plotpoint}}
\put(1064.0,141.0){\usebox{\plotpoint}}
\put(1064.0,141.0){\rule[-0.200pt]{0.723pt}{0.400pt}}
\put(1067,138.67){\rule{0.241pt}{0.400pt}}
\multiput(1067.00,139.17)(0.500,-1.000){2}{\rule{0.120pt}{0.400pt}}
\put(1068,138.67){\rule{0.241pt}{0.400pt}}
\multiput(1068.00,138.17)(0.500,1.000){2}{\rule{0.120pt}{0.400pt}}
\put(1067.0,140.0){\usebox{\plotpoint}}
\put(1069,140){\usebox{\plotpoint}}
\put(1069,138.67){\rule{0.241pt}{0.400pt}}
\multiput(1069.00,139.17)(0.500,-1.000){2}{\rule{0.120pt}{0.400pt}}
\put(1070.0,139.0){\usebox{\plotpoint}}
\put(1071,138.67){\rule{0.241pt}{0.400pt}}
\multiput(1071.00,139.17)(0.500,-1.000){2}{\rule{0.120pt}{0.400pt}}
\put(1071.0,139.0){\usebox{\plotpoint}}
\put(1072,139){\usebox{\plotpoint}}
\put(1073,137.67){\rule{0.241pt}{0.400pt}}
\multiput(1073.00,138.17)(0.500,-1.000){2}{\rule{0.120pt}{0.400pt}}
\put(1072.0,139.0){\usebox{\plotpoint}}
\put(1074,136.67){\rule{0.241pt}{0.400pt}}
\multiput(1074.00,136.17)(0.500,1.000){2}{\rule{0.120pt}{0.400pt}}
\put(1074.0,137.0){\usebox{\plotpoint}}
\put(1075,136.67){\rule{0.241pt}{0.400pt}}
\multiput(1075.00,136.17)(0.500,1.000){2}{\rule{0.120pt}{0.400pt}}
\put(1076,136.67){\rule{0.241pt}{0.400pt}}
\multiput(1076.00,137.17)(0.500,-1.000){2}{\rule{0.120pt}{0.400pt}}
\put(1075.0,137.0){\usebox{\plotpoint}}
\put(1077,137){\usebox{\plotpoint}}
\put(1077,136.67){\rule{0.241pt}{0.400pt}}
\multiput(1077.00,136.17)(0.500,1.000){2}{\rule{0.120pt}{0.400pt}}
\put(1078.0,138.0){\usebox{\plotpoint}}
\put(1079,135.67){\rule{0.241pt}{0.400pt}}
\multiput(1079.00,136.17)(0.500,-1.000){2}{\rule{0.120pt}{0.400pt}}
\put(1079.0,137.0){\usebox{\plotpoint}}
\put(1080.0,136.0){\usebox{\plotpoint}}
\put(1081,135.67){\rule{0.241pt}{0.400pt}}
\multiput(1081.00,136.17)(0.500,-1.000){2}{\rule{0.120pt}{0.400pt}}
\put(1080.0,137.0){\usebox{\plotpoint}}
\put(1082,136){\usebox{\plotpoint}}
\put(1082,134.67){\rule{0.241pt}{0.400pt}}
\multiput(1082.00,135.17)(0.500,-1.000){2}{\rule{0.120pt}{0.400pt}}
\put(1083.0,135.0){\usebox{\plotpoint}}
\put(1083.0,136.0){\rule[-0.200pt]{0.482pt}{0.400pt}}
\put(1085.0,135.0){\usebox{\plotpoint}}
\put(1085.0,135.0){\usebox{\plotpoint}}
\put(1086.0,135.0){\usebox{\plotpoint}}
\put(1086.0,136.0){\rule[-0.200pt]{0.482pt}{0.400pt}}
\put(1088,133.67){\rule{0.241pt}{0.400pt}}
\multiput(1088.00,134.17)(0.500,-1.000){2}{\rule{0.120pt}{0.400pt}}
\put(1088.67,134){\rule{0.400pt}{0.482pt}}
\multiput(1088.17,134.00)(1.000,1.000){2}{\rule{0.400pt}{0.241pt}}
\put(1088.0,135.0){\usebox{\plotpoint}}
\put(1090,133.67){\rule{0.241pt}{0.400pt}}
\multiput(1090.00,133.17)(0.500,1.000){2}{\rule{0.120pt}{0.400pt}}
\put(1090.0,134.0){\rule[-0.200pt]{0.400pt}{0.482pt}}
\put(1091,133.67){\rule{0.241pt}{0.400pt}}
\multiput(1091.00,133.17)(0.500,1.000){2}{\rule{0.120pt}{0.400pt}}
\put(1092,133.67){\rule{0.241pt}{0.400pt}}
\multiput(1092.00,134.17)(0.500,-1.000){2}{\rule{0.120pt}{0.400pt}}
\put(1091.0,134.0){\usebox{\plotpoint}}
\put(1093,133.67){\rule{0.241pt}{0.400pt}}
\multiput(1093.00,134.17)(0.500,-1.000){2}{\rule{0.120pt}{0.400pt}}
\put(1093.0,134.0){\usebox{\plotpoint}}
\put(1094,134){\usebox{\plotpoint}}
\put(1094.0,134.0){\rule[-0.200pt]{0.964pt}{0.400pt}}
\put(1098,132.67){\rule{0.241pt}{0.400pt}}
\multiput(1098.00,132.17)(0.500,1.000){2}{\rule{0.120pt}{0.400pt}}
\put(1098.0,133.0){\usebox{\plotpoint}}
\put(1099.0,133.0){\usebox{\plotpoint}}
\put(1099.0,133.0){\rule[-0.200pt]{0.482pt}{0.400pt}}
\put(1101,132.67){\rule{0.241pt}{0.400pt}}
\multiput(1101.00,133.17)(0.500,-1.000){2}{\rule{0.120pt}{0.400pt}}
\put(1101.0,133.0){\usebox{\plotpoint}}
\put(1102,133){\usebox{\plotpoint}}
\put(1104,131.67){\rule{0.241pt}{0.400pt}}
\multiput(1104.00,132.17)(0.500,-1.000){2}{\rule{0.120pt}{0.400pt}}
\put(1102.0,133.0){\rule[-0.200pt]{0.482pt}{0.400pt}}
\put(1105,131.67){\rule{0.241pt}{0.400pt}}
\multiput(1105.00,132.17)(0.500,-1.000){2}{\rule{0.120pt}{0.400pt}}
\put(1106,131.67){\rule{0.241pt}{0.400pt}}
\multiput(1106.00,131.17)(0.500,1.000){2}{\rule{0.120pt}{0.400pt}}
\put(1105.0,132.0){\usebox{\plotpoint}}
\put(1107.0,132.0){\usebox{\plotpoint}}
\put(1108,131.67){\rule{0.241pt}{0.400pt}}
\multiput(1108.00,131.17)(0.500,1.000){2}{\rule{0.120pt}{0.400pt}}
\put(1107.0,132.0){\usebox{\plotpoint}}
\put(1109.0,132.0){\usebox{\plotpoint}}
\put(1109.0,132.0){\rule[-0.200pt]{0.723pt}{0.400pt}}
\put(1112,130.67){\rule{0.241pt}{0.400pt}}
\multiput(1112.00,130.17)(0.500,1.000){2}{\rule{0.120pt}{0.400pt}}
\put(1112.0,131.0){\usebox{\plotpoint}}
\put(1113,132){\usebox{\plotpoint}}
\put(1113.0,132.0){\rule[-0.200pt]{0.964pt}{0.400pt}}
\put(1117.0,131.0){\usebox{\plotpoint}}
\put(1118.0,131.0){\usebox{\plotpoint}}
\put(1119,130.67){\rule{0.241pt}{0.400pt}}
\multiput(1119.00,131.17)(0.500,-1.000){2}{\rule{0.120pt}{0.400pt}}
\put(1118.0,132.0){\usebox{\plotpoint}}
\put(1120,131){\usebox{\plotpoint}}
\put(1120,130.67){\rule{0.241pt}{0.400pt}}
\multiput(1120.00,130.17)(0.500,1.000){2}{\rule{0.120pt}{0.400pt}}
\put(1121.0,131.0){\usebox{\plotpoint}}
\put(1121.0,131.0){\rule[-0.200pt]{0.482pt}{0.400pt}}
\put(1124,131){\usebox{\plotpoint}}
\put(1124,131){\usebox{\plotpoint}}
\put(1124.0,131.0){\rule[-0.200pt]{0.482pt}{0.400pt}}
\put(1127,131){\usebox{\plotpoint}}
\put(1127.0,131.0){\rule[-0.200pt]{0.482pt}{0.400pt}}
\put(1130,131){\usebox{\plotpoint}}
\put(1130,131){\usebox{\plotpoint}}
\put(1132,131){\usebox{\plotpoint}}
\put(1132,131){\usebox{\plotpoint}}
\put(1392,131){\usebox{\plotpoint}}
\put(1392.0,131.0){\usebox{\plotpoint}}
\put(1409,131){\usebox{\plotpoint}}
\put(1409,131){\usebox{\plotpoint}}
\put(1420.0,131.0){\usebox{\plotpoint}}
\put(1434,131){\usebox{\plotpoint}}
\put(1434,131){\usebox{\plotpoint}}
\put(170.67,283){\rule{0.400pt}{0.482pt}}
\multiput(170.17,283.00)(1.000,1.000){2}{\rule{0.400pt}{0.241pt}}
\put(1437.0,131.0){\usebox{\plotpoint}}
\put(172,287.67){\rule{0.241pt}{0.400pt}}
\multiput(172.00,287.17)(0.500,1.000){2}{\rule{0.120pt}{0.400pt}}
\put(172.67,289){\rule{0.400pt}{0.482pt}}
\multiput(172.17,289.00)(1.000,1.000){2}{\rule{0.400pt}{0.241pt}}
\put(172.0,285.0){\rule[-0.200pt]{0.400pt}{0.723pt}}
\put(174,291){\usebox{\plotpoint}}
\put(173.67,288){\rule{0.400pt}{0.723pt}}
\multiput(173.17,289.50)(1.000,-1.500){2}{\rule{0.400pt}{0.361pt}}
\put(175,287.67){\rule{0.241pt}{0.400pt}}
\multiput(175.00,287.17)(0.500,1.000){2}{\rule{0.120pt}{0.400pt}}
\put(175.67,290){\rule{0.400pt}{0.482pt}}
\multiput(175.17,291.00)(1.000,-1.000){2}{\rule{0.400pt}{0.241pt}}
\put(176.67,290){\rule{0.400pt}{0.723pt}}
\multiput(176.17,290.00)(1.000,1.500){2}{\rule{0.400pt}{0.361pt}}
\put(176.0,289.0){\rule[-0.200pt]{0.400pt}{0.723pt}}
\put(177.67,290){\rule{0.400pt}{0.964pt}}
\multiput(177.17,290.00)(1.000,2.000){2}{\rule{0.400pt}{0.482pt}}
\put(178.67,292){\rule{0.400pt}{0.482pt}}
\multiput(178.17,293.00)(1.000,-1.000){2}{\rule{0.400pt}{0.241pt}}
\put(178.0,290.0){\rule[-0.200pt]{0.400pt}{0.723pt}}
\put(179.67,292){\rule{0.400pt}{0.964pt}}
\multiput(179.17,294.00)(1.000,-2.000){2}{\rule{0.400pt}{0.482pt}}
\put(180.67,292){\rule{0.400pt}{0.723pt}}
\multiput(180.17,292.00)(1.000,1.500){2}{\rule{0.400pt}{0.361pt}}
\put(180.0,292.0){\rule[-0.200pt]{0.400pt}{0.964pt}}
\put(182,295){\usebox{\plotpoint}}
\put(183,294.67){\rule{0.241pt}{0.400pt}}
\multiput(183.00,294.17)(0.500,1.000){2}{\rule{0.120pt}{0.400pt}}
\put(182.0,295.0){\usebox{\plotpoint}}
\put(184,295.67){\rule{0.241pt}{0.400pt}}
\multiput(184.00,296.17)(0.500,-1.000){2}{\rule{0.120pt}{0.400pt}}
\put(184.67,296){\rule{0.400pt}{0.482pt}}
\multiput(184.17,296.00)(1.000,1.000){2}{\rule{0.400pt}{0.241pt}}
\put(184.0,296.0){\usebox{\plotpoint}}
\put(186,295.67){\rule{0.241pt}{0.400pt}}
\multiput(186.00,295.17)(0.500,1.000){2}{\rule{0.120pt}{0.400pt}}
\put(186.67,297){\rule{0.400pt}{0.482pt}}
\multiput(186.17,297.00)(1.000,1.000){2}{\rule{0.400pt}{0.241pt}}
\put(186.0,296.0){\rule[-0.200pt]{0.400pt}{0.482pt}}
\put(187.67,298){\rule{0.400pt}{0.723pt}}
\multiput(187.17,299.50)(1.000,-1.500){2}{\rule{0.400pt}{0.361pt}}
\put(188.67,298){\rule{0.400pt}{1.204pt}}
\multiput(188.17,298.00)(1.000,2.500){2}{\rule{0.400pt}{0.602pt}}
\put(189.67,301){\rule{0.400pt}{0.482pt}}
\multiput(189.17,302.00)(1.000,-1.000){2}{\rule{0.400pt}{0.241pt}}
\put(188.0,299.0){\rule[-0.200pt]{0.400pt}{0.482pt}}
\put(191,301){\usebox{\plotpoint}}
\put(192,300.67){\rule{0.241pt}{0.400pt}}
\multiput(192.00,300.17)(0.500,1.000){2}{\rule{0.120pt}{0.400pt}}
\put(191.0,301.0){\usebox{\plotpoint}}
\put(193,302){\usebox{\plotpoint}}
\put(193,301.67){\rule{0.241pt}{0.400pt}}
\multiput(193.00,301.17)(0.500,1.000){2}{\rule{0.120pt}{0.400pt}}
\put(194,301.67){\rule{0.241pt}{0.400pt}}
\multiput(194.00,302.17)(0.500,-1.000){2}{\rule{0.120pt}{0.400pt}}
\put(195,302.67){\rule{0.241pt}{0.400pt}}
\multiput(195.00,303.17)(0.500,-1.000){2}{\rule{0.120pt}{0.400pt}}
\put(196,302.67){\rule{0.241pt}{0.400pt}}
\multiput(196.00,302.17)(0.500,1.000){2}{\rule{0.120pt}{0.400pt}}
\put(195.0,302.0){\rule[-0.200pt]{0.400pt}{0.482pt}}
\put(196.67,302){\rule{0.400pt}{0.723pt}}
\multiput(196.17,303.50)(1.000,-1.500){2}{\rule{0.400pt}{0.361pt}}
\put(197.0,304.0){\usebox{\plotpoint}}
\put(198.67,302){\rule{0.400pt}{0.482pt}}
\multiput(198.17,302.00)(1.000,1.000){2}{\rule{0.400pt}{0.241pt}}
\put(199.67,302){\rule{0.400pt}{0.482pt}}
\multiput(199.17,303.00)(1.000,-1.000){2}{\rule{0.400pt}{0.241pt}}
\put(198.0,302.0){\usebox{\plotpoint}}
\put(201,302.67){\rule{0.241pt}{0.400pt}}
\multiput(201.00,302.17)(0.500,1.000){2}{\rule{0.120pt}{0.400pt}}
\put(201.67,302){\rule{0.400pt}{0.482pt}}
\multiput(201.17,303.00)(1.000,-1.000){2}{\rule{0.400pt}{0.241pt}}
\put(201.0,302.0){\usebox{\plotpoint}}
\put(203,301.67){\rule{0.241pt}{0.400pt}}
\multiput(203.00,302.17)(0.500,-1.000){2}{\rule{0.120pt}{0.400pt}}
\put(203.67,302){\rule{0.400pt}{0.482pt}}
\multiput(203.17,302.00)(1.000,1.000){2}{\rule{0.400pt}{0.241pt}}
\put(203.0,302.0){\usebox{\plotpoint}}
\put(205,304){\usebox{\plotpoint}}
\put(205,302.67){\rule{0.241pt}{0.400pt}}
\multiput(205.00,303.17)(0.500,-1.000){2}{\rule{0.120pt}{0.400pt}}
\put(206,301.67){\rule{0.241pt}{0.400pt}}
\multiput(206.00,302.17)(0.500,-1.000){2}{\rule{0.120pt}{0.400pt}}
\put(207.0,302.0){\usebox{\plotpoint}}
\put(208,302.67){\rule{0.241pt}{0.400pt}}
\multiput(208.00,302.17)(0.500,1.000){2}{\rule{0.120pt}{0.400pt}}
\put(207.0,303.0){\usebox{\plotpoint}}
\put(209,301.67){\rule{0.241pt}{0.400pt}}
\multiput(209.00,302.17)(0.500,-1.000){2}{\rule{0.120pt}{0.400pt}}
\put(209.0,303.0){\usebox{\plotpoint}}
\put(210.0,302.0){\usebox{\plotpoint}}
\put(211,301.67){\rule{0.241pt}{0.400pt}}
\multiput(211.00,302.17)(0.500,-1.000){2}{\rule{0.120pt}{0.400pt}}
\put(211.0,302.0){\usebox{\plotpoint}}
\put(212.0,302.0){\usebox{\plotpoint}}
\put(213.0,301.0){\usebox{\plotpoint}}
\put(214,299.67){\rule{0.241pt}{0.400pt}}
\multiput(214.00,300.17)(0.500,-1.000){2}{\rule{0.120pt}{0.400pt}}
\put(213.0,301.0){\usebox{\plotpoint}}
\put(215,299.67){\rule{0.241pt}{0.400pt}}
\multiput(215.00,300.17)(0.500,-1.000){2}{\rule{0.120pt}{0.400pt}}
\put(216,299.67){\rule{0.241pt}{0.400pt}}
\multiput(216.00,299.17)(0.500,1.000){2}{\rule{0.120pt}{0.400pt}}
\put(215.0,300.0){\usebox{\plotpoint}}
\put(217.0,300.0){\usebox{\plotpoint}}
\put(217.0,300.0){\rule[-0.200pt]{0.482pt}{0.400pt}}
\put(218.67,298){\rule{0.400pt}{0.482pt}}
\multiput(218.17,298.00)(1.000,1.000){2}{\rule{0.400pt}{0.241pt}}
\put(219.67,298){\rule{0.400pt}{0.482pt}}
\multiput(219.17,299.00)(1.000,-1.000){2}{\rule{0.400pt}{0.241pt}}
\put(219.0,298.0){\rule[-0.200pt]{0.400pt}{0.482pt}}
\put(221,298){\usebox{\plotpoint}}
\put(223,296.67){\rule{0.241pt}{0.400pt}}
\multiput(223.00,297.17)(0.500,-1.000){2}{\rule{0.120pt}{0.400pt}}
\put(224,295.67){\rule{0.241pt}{0.400pt}}
\multiput(224.00,296.17)(0.500,-1.000){2}{\rule{0.120pt}{0.400pt}}
\put(221.0,298.0){\rule[-0.200pt]{0.482pt}{0.400pt}}
\put(225.0,295.0){\usebox{\plotpoint}}
\put(226,293.67){\rule{0.241pt}{0.400pt}}
\multiput(226.00,294.17)(0.500,-1.000){2}{\rule{0.120pt}{0.400pt}}
\put(225.0,295.0){\usebox{\plotpoint}}
\put(227,292.67){\rule{0.241pt}{0.400pt}}
\multiput(227.00,292.17)(0.500,1.000){2}{\rule{0.120pt}{0.400pt}}
\put(228,292.67){\rule{0.241pt}{0.400pt}}
\multiput(228.00,293.17)(0.500,-1.000){2}{\rule{0.120pt}{0.400pt}}
\put(227.0,293.0){\usebox{\plotpoint}}
\put(229,293){\usebox{\plotpoint}}
\put(229,291.67){\rule{0.241pt}{0.400pt}}
\multiput(229.00,292.17)(0.500,-1.000){2}{\rule{0.120pt}{0.400pt}}
\put(230,291.67){\rule{0.241pt}{0.400pt}}
\multiput(230.00,291.17)(0.500,1.000){2}{\rule{0.120pt}{0.400pt}}
\put(231.0,292.0){\usebox{\plotpoint}}
\put(232,290.67){\rule{0.241pt}{0.400pt}}
\multiput(232.00,291.17)(0.500,-1.000){2}{\rule{0.120pt}{0.400pt}}
\put(231.0,292.0){\usebox{\plotpoint}}
\put(233.0,290.0){\usebox{\plotpoint}}
\put(234,288.67){\rule{0.241pt}{0.400pt}}
\multiput(234.00,289.17)(0.500,-1.000){2}{\rule{0.120pt}{0.400pt}}
\put(233.0,290.0){\usebox{\plotpoint}}
\put(235.0,289.0){\usebox{\plotpoint}}
\put(236,287.67){\rule{0.241pt}{0.400pt}}
\multiput(236.00,287.17)(0.500,1.000){2}{\rule{0.120pt}{0.400pt}}
\put(236.67,287){\rule{0.400pt}{0.482pt}}
\multiput(236.17,288.00)(1.000,-1.000){2}{\rule{0.400pt}{0.241pt}}
\put(236.0,288.0){\usebox{\plotpoint}}
\put(238,287){\usebox{\plotpoint}}
\put(241,285.67){\rule{0.241pt}{0.400pt}}
\multiput(241.00,286.17)(0.500,-1.000){2}{\rule{0.120pt}{0.400pt}}
\put(238.0,287.0){\rule[-0.200pt]{0.723pt}{0.400pt}}
\put(242,286){\usebox{\plotpoint}}
\put(241.67,284){\rule{0.400pt}{0.482pt}}
\multiput(241.17,285.00)(1.000,-1.000){2}{\rule{0.400pt}{0.241pt}}
\put(243,282.67){\rule{0.241pt}{0.400pt}}
\multiput(243.00,283.17)(0.500,-1.000){2}{\rule{0.120pt}{0.400pt}}
\put(244.0,283.0){\usebox{\plotpoint}}
\put(245,282.67){\rule{0.241pt}{0.400pt}}
\multiput(245.00,283.17)(0.500,-1.000){2}{\rule{0.120pt}{0.400pt}}
\put(244.0,284.0){\usebox{\plotpoint}}
\put(246,281.67){\rule{0.241pt}{0.400pt}}
\multiput(246.00,281.17)(0.500,1.000){2}{\rule{0.120pt}{0.400pt}}
\put(246.67,281){\rule{0.400pt}{0.482pt}}
\multiput(246.17,282.00)(1.000,-1.000){2}{\rule{0.400pt}{0.241pt}}
\put(246.0,282.0){\usebox{\plotpoint}}
\put(248,280.67){\rule{0.241pt}{0.400pt}}
\multiput(248.00,281.17)(0.500,-1.000){2}{\rule{0.120pt}{0.400pt}}
\put(248.0,281.0){\usebox{\plotpoint}}
\put(249.0,281.0){\rule[-0.200pt]{0.723pt}{0.400pt}}
\put(252,278.67){\rule{0.241pt}{0.400pt}}
\multiput(252.00,279.17)(0.500,-1.000){2}{\rule{0.120pt}{0.400pt}}
\put(252.0,280.0){\usebox{\plotpoint}}
\put(253.0,279.0){\usebox{\plotpoint}}
\put(254.0,278.0){\usebox{\plotpoint}}
\put(255,277.67){\rule{0.241pt}{0.400pt}}
\multiput(255.00,277.17)(0.500,1.000){2}{\rule{0.120pt}{0.400pt}}
\put(254.0,278.0){\usebox{\plotpoint}}
\put(256,279){\usebox{\plotpoint}}
\put(256,277.67){\rule{0.241pt}{0.400pt}}
\multiput(256.00,278.17)(0.500,-1.000){2}{\rule{0.120pt}{0.400pt}}
\put(257,276.67){\rule{0.241pt}{0.400pt}}
\multiput(257.00,277.17)(0.500,-1.000){2}{\rule{0.120pt}{0.400pt}}
\put(258,277){\usebox{\plotpoint}}
\put(258.0,277.0){\rule[-0.200pt]{0.482pt}{0.400pt}}
\put(260.0,276.0){\usebox{\plotpoint}}
\put(263,274.67){\rule{0.241pt}{0.400pt}}
\multiput(263.00,275.17)(0.500,-1.000){2}{\rule{0.120pt}{0.400pt}}
\put(260.0,276.0){\rule[-0.200pt]{0.723pt}{0.400pt}}
\put(263.67,274){\rule{0.400pt}{0.482pt}}
\multiput(263.17,275.00)(1.000,-1.000){2}{\rule{0.400pt}{0.241pt}}
\put(265,273.67){\rule{0.241pt}{0.400pt}}
\multiput(265.00,273.17)(0.500,1.000){2}{\rule{0.120pt}{0.400pt}}
\put(264.0,275.0){\usebox{\plotpoint}}
\put(265.67,274){\rule{0.400pt}{0.482pt}}
\multiput(265.17,275.00)(1.000,-1.000){2}{\rule{0.400pt}{0.241pt}}
\put(266.0,275.0){\usebox{\plotpoint}}
\put(269,273.67){\rule{0.241pt}{0.400pt}}
\multiput(269.00,273.17)(0.500,1.000){2}{\rule{0.120pt}{0.400pt}}
\put(267.0,274.0){\rule[-0.200pt]{0.482pt}{0.400pt}}
\put(270,275){\usebox{\plotpoint}}
\put(270,273.67){\rule{0.241pt}{0.400pt}}
\multiput(270.00,274.17)(0.500,-1.000){2}{\rule{0.120pt}{0.400pt}}
\put(271,272.67){\rule{0.241pt}{0.400pt}}
\multiput(271.00,273.17)(0.500,-1.000){2}{\rule{0.120pt}{0.400pt}}
\put(272,272.67){\rule{0.241pt}{0.400pt}}
\multiput(272.00,273.17)(0.500,-1.000){2}{\rule{0.120pt}{0.400pt}}
\put(272.67,273){\rule{0.400pt}{0.482pt}}
\multiput(272.17,273.00)(1.000,1.000){2}{\rule{0.400pt}{0.241pt}}
\put(272.0,273.0){\usebox{\plotpoint}}
\put(274,272.67){\rule{0.241pt}{0.400pt}}
\multiput(274.00,273.17)(0.500,-1.000){2}{\rule{0.120pt}{0.400pt}}
\put(274.0,274.0){\usebox{\plotpoint}}
\put(275.0,273.0){\usebox{\plotpoint}}
\put(275.67,272){\rule{0.400pt}{0.482pt}}
\multiput(275.17,272.00)(1.000,1.000){2}{\rule{0.400pt}{0.241pt}}
\put(276.0,272.0){\usebox{\plotpoint}}
\put(278,272.67){\rule{0.241pt}{0.400pt}}
\multiput(278.00,273.17)(0.500,-1.000){2}{\rule{0.120pt}{0.400pt}}
\put(279,272.67){\rule{0.241pt}{0.400pt}}
\multiput(279.00,272.17)(0.500,1.000){2}{\rule{0.120pt}{0.400pt}}
\put(277.0,274.0){\usebox{\plotpoint}}
\put(280,272.67){\rule{0.241pt}{0.400pt}}
\multiput(280.00,272.17)(0.500,1.000){2}{\rule{0.120pt}{0.400pt}}
\put(281,272.67){\rule{0.241pt}{0.400pt}}
\multiput(281.00,273.17)(0.500,-1.000){2}{\rule{0.120pt}{0.400pt}}
\put(280.0,273.0){\usebox{\plotpoint}}
\put(282,272.67){\rule{0.241pt}{0.400pt}}
\multiput(282.00,273.17)(0.500,-1.000){2}{\rule{0.120pt}{0.400pt}}
\put(283,272.67){\rule{0.241pt}{0.400pt}}
\multiput(283.00,272.17)(0.500,1.000){2}{\rule{0.120pt}{0.400pt}}
\put(282.0,273.0){\usebox{\plotpoint}}
\put(283.67,273){\rule{0.400pt}{0.482pt}}
\multiput(283.17,273.00)(1.000,1.000){2}{\rule{0.400pt}{0.241pt}}
\put(284.67,273){\rule{0.400pt}{0.482pt}}
\multiput(284.17,274.00)(1.000,-1.000){2}{\rule{0.400pt}{0.241pt}}
\put(284.0,273.0){\usebox{\plotpoint}}
\put(286,273.67){\rule{0.241pt}{0.400pt}}
\multiput(286.00,274.17)(0.500,-1.000){2}{\rule{0.120pt}{0.400pt}}
\put(287,273.67){\rule{0.241pt}{0.400pt}}
\multiput(287.00,273.17)(0.500,1.000){2}{\rule{0.120pt}{0.400pt}}
\put(286.0,273.0){\rule[-0.200pt]{0.400pt}{0.482pt}}
\put(288,275){\usebox{\plotpoint}}
\put(288.0,275.0){\rule[-0.200pt]{0.482pt}{0.400pt}}
\put(290,274.67){\rule{0.241pt}{0.400pt}}
\multiput(290.00,275.17)(0.500,-1.000){2}{\rule{0.120pt}{0.400pt}}
\put(291,274.67){\rule{0.241pt}{0.400pt}}
\multiput(291.00,274.17)(0.500,1.000){2}{\rule{0.120pt}{0.400pt}}
\put(290.0,275.0){\usebox{\plotpoint}}
\put(292,276){\usebox{\plotpoint}}
\put(292,275.67){\rule{0.241pt}{0.400pt}}
\multiput(292.00,275.17)(0.500,1.000){2}{\rule{0.120pt}{0.400pt}}
\put(294,276.67){\rule{0.241pt}{0.400pt}}
\multiput(294.00,276.17)(0.500,1.000){2}{\rule{0.120pt}{0.400pt}}
\put(293.0,277.0){\usebox{\plotpoint}}
\put(295.0,278.0){\usebox{\plotpoint}}
\put(296.0,278.0){\usebox{\plotpoint}}
\put(297,277.67){\rule{0.241pt}{0.400pt}}
\multiput(297.00,278.17)(0.500,-1.000){2}{\rule{0.120pt}{0.400pt}}
\put(296.0,279.0){\usebox{\plotpoint}}
\put(298,278.67){\rule{0.241pt}{0.400pt}}
\multiput(298.00,278.17)(0.500,1.000){2}{\rule{0.120pt}{0.400pt}}
\put(299,279.67){\rule{0.241pt}{0.400pt}}
\multiput(299.00,279.17)(0.500,1.000){2}{\rule{0.120pt}{0.400pt}}
\put(298.0,278.0){\usebox{\plotpoint}}
\put(300,279.67){\rule{0.241pt}{0.400pt}}
\multiput(300.00,279.17)(0.500,1.000){2}{\rule{0.120pt}{0.400pt}}
\put(301,280.67){\rule{0.241pt}{0.400pt}}
\multiput(301.00,280.17)(0.500,1.000){2}{\rule{0.120pt}{0.400pt}}
\put(300.0,280.0){\usebox{\plotpoint}}
\put(302,282){\usebox{\plotpoint}}
\put(302.67,282){\rule{0.400pt}{0.482pt}}
\multiput(302.17,282.00)(1.000,1.000){2}{\rule{0.400pt}{0.241pt}}
\put(302.0,282.0){\usebox{\plotpoint}}
\put(304.0,284.0){\usebox{\plotpoint}}
\put(306,284.67){\rule{0.241pt}{0.400pt}}
\multiput(306.00,284.17)(0.500,1.000){2}{\rule{0.120pt}{0.400pt}}
\put(304.0,285.0){\rule[-0.200pt]{0.482pt}{0.400pt}}
\put(307.0,286.0){\usebox{\plotpoint}}
\put(308.0,286.0){\rule[-0.200pt]{0.400pt}{0.482pt}}
\put(309,287.67){\rule{0.241pt}{0.400pt}}
\multiput(309.00,287.17)(0.500,1.000){2}{\rule{0.120pt}{0.400pt}}
\put(308.0,288.0){\usebox{\plotpoint}}
\put(310,289){\usebox{\plotpoint}}
\put(310,288.67){\rule{0.241pt}{0.400pt}}
\multiput(310.00,288.17)(0.500,1.000){2}{\rule{0.120pt}{0.400pt}}
\put(311,289.67){\rule{0.241pt}{0.400pt}}
\multiput(311.00,289.17)(0.500,1.000){2}{\rule{0.120pt}{0.400pt}}
\put(312,291){\usebox{\plotpoint}}
\put(312,290.67){\rule{0.241pt}{0.400pt}}
\multiput(312.00,290.17)(0.500,1.000){2}{\rule{0.120pt}{0.400pt}}
\put(313,291.67){\rule{0.241pt}{0.400pt}}
\multiput(313.00,291.17)(0.500,1.000){2}{\rule{0.120pt}{0.400pt}}
\put(314,293.67){\rule{0.241pt}{0.400pt}}
\multiput(314.00,293.17)(0.500,1.000){2}{\rule{0.120pt}{0.400pt}}
\put(314.67,295){\rule{0.400pt}{0.482pt}}
\multiput(314.17,295.00)(1.000,1.000){2}{\rule{0.400pt}{0.241pt}}
\put(314.0,293.0){\usebox{\plotpoint}}
\put(316,297){\usebox{\plotpoint}}
\put(317,296.67){\rule{0.241pt}{0.400pt}}
\multiput(317.00,296.17)(0.500,1.000){2}{\rule{0.120pt}{0.400pt}}
\put(316.0,297.0){\usebox{\plotpoint}}
\put(318,298){\usebox{\plotpoint}}
\put(317.67,298){\rule{0.400pt}{0.482pt}}
\multiput(317.17,298.00)(1.000,1.000){2}{\rule{0.400pt}{0.241pt}}
\put(319.0,300.0){\usebox{\plotpoint}}
\put(320.0,300.0){\rule[-0.200pt]{0.400pt}{0.482pt}}
\put(321,301.67){\rule{0.241pt}{0.400pt}}
\multiput(321.00,301.17)(0.500,1.000){2}{\rule{0.120pt}{0.400pt}}
\put(320.0,302.0){\usebox{\plotpoint}}
\put(322,303){\usebox{\plotpoint}}
\put(321.67,303){\rule{0.400pt}{0.482pt}}
\multiput(321.17,303.00)(1.000,1.000){2}{\rule{0.400pt}{0.241pt}}
\put(323,304.67){\rule{0.241pt}{0.400pt}}
\multiput(323.00,304.17)(0.500,1.000){2}{\rule{0.120pt}{0.400pt}}
\put(324,306.67){\rule{0.241pt}{0.400pt}}
\multiput(324.00,306.17)(0.500,1.000){2}{\rule{0.120pt}{0.400pt}}
\put(325,307.67){\rule{0.241pt}{0.400pt}}
\multiput(325.00,307.17)(0.500,1.000){2}{\rule{0.120pt}{0.400pt}}
\put(324.0,306.0){\usebox{\plotpoint}}
\put(325.67,310){\rule{0.400pt}{0.482pt}}
\multiput(325.17,310.00)(1.000,1.000){2}{\rule{0.400pt}{0.241pt}}
\put(327,310.67){\rule{0.241pt}{0.400pt}}
\multiput(327.00,311.17)(0.500,-1.000){2}{\rule{0.120pt}{0.400pt}}
\put(326.0,309.0){\usebox{\plotpoint}}
\put(327.67,313){\rule{0.400pt}{0.482pt}}
\multiput(327.17,313.00)(1.000,1.000){2}{\rule{0.400pt}{0.241pt}}
\put(328.0,311.0){\rule[-0.200pt]{0.400pt}{0.482pt}}
\put(329.67,315){\rule{0.400pt}{0.723pt}}
\multiput(329.17,315.00)(1.000,1.500){2}{\rule{0.400pt}{0.361pt}}
\put(331,317.67){\rule{0.241pt}{0.400pt}}
\multiput(331.00,317.17)(0.500,1.000){2}{\rule{0.120pt}{0.400pt}}
\put(329.0,315.0){\usebox{\plotpoint}}
\put(332,319){\usebox{\plotpoint}}
\put(331.67,319){\rule{0.400pt}{0.482pt}}
\multiput(331.17,319.00)(1.000,1.000){2}{\rule{0.400pt}{0.241pt}}
\put(333,320.67){\rule{0.241pt}{0.400pt}}
\multiput(333.00,320.17)(0.500,1.000){2}{\rule{0.120pt}{0.400pt}}
\put(334,322){\usebox{\plotpoint}}
\put(333.67,322){\rule{0.400pt}{0.482pt}}
\multiput(333.17,322.00)(1.000,1.000){2}{\rule{0.400pt}{0.241pt}}
\put(335,323.67){\rule{0.241pt}{0.400pt}}
\multiput(335.00,323.17)(0.500,1.000){2}{\rule{0.120pt}{0.400pt}}
\put(336,325){\usebox{\plotpoint}}
\put(335.67,325){\rule{0.400pt}{0.723pt}}
\multiput(335.17,325.00)(1.000,1.500){2}{\rule{0.400pt}{0.361pt}}
\put(337,327.67){\rule{0.241pt}{0.400pt}}
\multiput(337.00,327.17)(0.500,1.000){2}{\rule{0.120pt}{0.400pt}}
\put(338,329.67){\rule{0.241pt}{0.400pt}}
\multiput(338.00,329.17)(0.500,1.000){2}{\rule{0.120pt}{0.400pt}}
\put(339,330.67){\rule{0.241pt}{0.400pt}}
\multiput(339.00,330.17)(0.500,1.000){2}{\rule{0.120pt}{0.400pt}}
\put(338.0,329.0){\usebox{\plotpoint}}
\put(340,332.67){\rule{0.241pt}{0.400pt}}
\multiput(340.00,332.17)(0.500,1.000){2}{\rule{0.120pt}{0.400pt}}
\put(341,333.67){\rule{0.241pt}{0.400pt}}
\multiput(341.00,333.17)(0.500,1.000){2}{\rule{0.120pt}{0.400pt}}
\put(340.0,332.0){\usebox{\plotpoint}}
\put(342,336.67){\rule{0.241pt}{0.400pt}}
\multiput(342.00,336.17)(0.500,1.000){2}{\rule{0.120pt}{0.400pt}}
\put(342.67,338){\rule{0.400pt}{0.482pt}}
\multiput(342.17,338.00)(1.000,1.000){2}{\rule{0.400pt}{0.241pt}}
\put(342.0,335.0){\rule[-0.200pt]{0.400pt}{0.482pt}}
\put(344,340){\usebox{\plotpoint}}
\put(343.67,340){\rule{0.400pt}{0.482pt}}
\multiput(343.17,340.00)(1.000,1.000){2}{\rule{0.400pt}{0.241pt}}
\put(345,341.67){\rule{0.241pt}{0.400pt}}
\multiput(345.00,341.17)(0.500,1.000){2}{\rule{0.120pt}{0.400pt}}
\put(346,344.67){\rule{0.241pt}{0.400pt}}
\multiput(346.00,344.17)(0.500,1.000){2}{\rule{0.120pt}{0.400pt}}
\put(346.67,346){\rule{0.400pt}{0.482pt}}
\multiput(346.17,346.00)(1.000,1.000){2}{\rule{0.400pt}{0.241pt}}
\put(346.0,343.0){\rule[-0.200pt]{0.400pt}{0.482pt}}
\put(347.67,346){\rule{0.400pt}{0.723pt}}
\multiput(347.17,346.00)(1.000,1.500){2}{\rule{0.400pt}{0.361pt}}
\put(348.67,349){\rule{0.400pt}{0.723pt}}
\multiput(348.17,349.00)(1.000,1.500){2}{\rule{0.400pt}{0.361pt}}
\put(348.0,346.0){\rule[-0.200pt]{0.400pt}{0.482pt}}
\put(350,352){\usebox{\plotpoint}}
\put(350,351.67){\rule{0.241pt}{0.400pt}}
\multiput(350.00,351.17)(0.500,1.000){2}{\rule{0.120pt}{0.400pt}}
\put(350.67,353){\rule{0.400pt}{0.482pt}}
\multiput(350.17,353.00)(1.000,1.000){2}{\rule{0.400pt}{0.241pt}}
\put(351.67,356){\rule{0.400pt}{0.482pt}}
\multiput(351.17,356.00)(1.000,1.000){2}{\rule{0.400pt}{0.241pt}}
\put(353,357.67){\rule{0.241pt}{0.400pt}}
\multiput(353.00,357.17)(0.500,1.000){2}{\rule{0.120pt}{0.400pt}}
\put(352.0,355.0){\usebox{\plotpoint}}
\put(354,360.67){\rule{0.241pt}{0.400pt}}
\multiput(354.00,360.17)(0.500,1.000){2}{\rule{0.120pt}{0.400pt}}
\put(355,361.67){\rule{0.241pt}{0.400pt}}
\multiput(355.00,361.17)(0.500,1.000){2}{\rule{0.120pt}{0.400pt}}
\put(354.0,359.0){\rule[-0.200pt]{0.400pt}{0.482pt}}
\put(355.67,365){\rule{0.400pt}{0.482pt}}
\multiput(355.17,365.00)(1.000,1.000){2}{\rule{0.400pt}{0.241pt}}
\put(356.0,363.0){\rule[-0.200pt]{0.400pt}{0.482pt}}
\put(357.0,367.0){\usebox{\plotpoint}}
\put(358,368.67){\rule{0.241pt}{0.400pt}}
\multiput(358.00,368.17)(0.500,1.000){2}{\rule{0.120pt}{0.400pt}}
\put(358.0,367.0){\rule[-0.200pt]{0.400pt}{0.482pt}}
\put(359,370){\usebox{\plotpoint}}
\put(358.67,370){\rule{0.400pt}{0.723pt}}
\multiput(358.17,370.00)(1.000,1.500){2}{\rule{0.400pt}{0.361pt}}
\put(359.67,373){\rule{0.400pt}{0.482pt}}
\multiput(359.17,373.00)(1.000,1.000){2}{\rule{0.400pt}{0.241pt}}
\put(361,375){\usebox{\plotpoint}}
\put(360.67,375){\rule{0.400pt}{0.964pt}}
\multiput(360.17,375.00)(1.000,2.000){2}{\rule{0.400pt}{0.482pt}}
\put(362,377.67){\rule{0.241pt}{0.400pt}}
\multiput(362.00,378.17)(0.500,-1.000){2}{\rule{0.120pt}{0.400pt}}
\put(362.67,379){\rule{0.400pt}{0.482pt}}
\multiput(362.17,379.00)(1.000,1.000){2}{\rule{0.400pt}{0.241pt}}
\put(363.67,381){\rule{0.400pt}{0.482pt}}
\multiput(363.17,381.00)(1.000,1.000){2}{\rule{0.400pt}{0.241pt}}
\put(363.0,378.0){\usebox{\plotpoint}}
\put(364.67,384){\rule{0.400pt}{0.482pt}}
\multiput(364.17,384.00)(1.000,1.000){2}{\rule{0.400pt}{0.241pt}}
\put(366,384.67){\rule{0.241pt}{0.400pt}}
\multiput(366.00,385.17)(0.500,-1.000){2}{\rule{0.120pt}{0.400pt}}
\put(365.0,383.0){\usebox{\plotpoint}}
\put(366.67,386){\rule{0.400pt}{0.723pt}}
\multiput(366.17,386.00)(1.000,1.500){2}{\rule{0.400pt}{0.361pt}}
\put(367.67,389){\rule{0.400pt}{0.723pt}}
\multiput(367.17,389.00)(1.000,1.500){2}{\rule{0.400pt}{0.361pt}}
\put(367.0,385.0){\usebox{\plotpoint}}
\put(369,392){\usebox{\plotpoint}}
\put(368.67,392){\rule{0.400pt}{0.723pt}}
\multiput(368.17,392.00)(1.000,1.500){2}{\rule{0.400pt}{0.361pt}}
\put(370.0,395.0){\usebox{\plotpoint}}
\put(371,396.67){\rule{0.241pt}{0.400pt}}
\multiput(371.00,396.17)(0.500,1.000){2}{\rule{0.120pt}{0.400pt}}
\put(372,397.67){\rule{0.241pt}{0.400pt}}
\multiput(372.00,397.17)(0.500,1.000){2}{\rule{0.120pt}{0.400pt}}
\put(371.0,395.0){\rule[-0.200pt]{0.400pt}{0.482pt}}
\put(373,400.67){\rule{0.241pt}{0.400pt}}
\multiput(373.00,400.17)(0.500,1.000){2}{\rule{0.120pt}{0.400pt}}
\put(373.67,402){\rule{0.400pt}{0.723pt}}
\multiput(373.17,402.00)(1.000,1.500){2}{\rule{0.400pt}{0.361pt}}
\put(373.0,399.0){\rule[-0.200pt]{0.400pt}{0.482pt}}
\put(374.67,404){\rule{0.400pt}{1.204pt}}
\multiput(374.17,404.00)(1.000,2.500){2}{\rule{0.400pt}{0.602pt}}
\put(376,408.67){\rule{0.241pt}{0.400pt}}
\multiput(376.00,408.17)(0.500,1.000){2}{\rule{0.120pt}{0.400pt}}
\put(375.0,404.0){\usebox{\plotpoint}}
\put(376.67,408){\rule{0.400pt}{1.204pt}}
\multiput(376.17,408.00)(1.000,2.500){2}{\rule{0.400pt}{0.602pt}}
\put(377.0,408.0){\rule[-0.200pt]{0.400pt}{0.482pt}}
\put(378.0,413.0){\usebox{\plotpoint}}
\put(378.67,414){\rule{0.400pt}{0.482pt}}
\multiput(378.17,414.00)(1.000,1.000){2}{\rule{0.400pt}{0.241pt}}
\put(379.0,413.0){\usebox{\plotpoint}}
\put(380.0,416.0){\usebox{\plotpoint}}
\put(381.0,416.0){\rule[-0.200pt]{0.400pt}{1.204pt}}
\put(382,420.67){\rule{0.241pt}{0.400pt}}
\multiput(382.00,420.17)(0.500,1.000){2}{\rule{0.120pt}{0.400pt}}
\put(381.0,421.0){\usebox{\plotpoint}}
\put(383,422){\usebox{\plotpoint}}
\put(383,421.67){\rule{0.241pt}{0.400pt}}
\multiput(383.00,421.17)(0.500,1.000){2}{\rule{0.120pt}{0.400pt}}
\put(383.67,423){\rule{0.400pt}{0.964pt}}
\multiput(383.17,423.00)(1.000,2.000){2}{\rule{0.400pt}{0.482pt}}
\put(385,427.67){\rule{0.241pt}{0.400pt}}
\multiput(385.00,427.17)(0.500,1.000){2}{\rule{0.120pt}{0.400pt}}
\put(386,428.67){\rule{0.241pt}{0.400pt}}
\multiput(386.00,428.17)(0.500,1.000){2}{\rule{0.120pt}{0.400pt}}
\put(385.0,427.0){\usebox{\plotpoint}}
\put(387,430){\usebox{\plotpoint}}
\put(386.67,430){\rule{0.400pt}{0.482pt}}
\multiput(386.17,430.00)(1.000,1.000){2}{\rule{0.400pt}{0.241pt}}
\put(387.67,432){\rule{0.400pt}{0.723pt}}
\multiput(387.17,432.00)(1.000,1.500){2}{\rule{0.400pt}{0.361pt}}
\put(389,436.67){\rule{0.241pt}{0.400pt}}
\multiput(389.00,436.17)(0.500,1.000){2}{\rule{0.120pt}{0.400pt}}
\put(389.0,435.0){\rule[-0.200pt]{0.400pt}{0.482pt}}
\put(390.0,438.0){\usebox{\plotpoint}}
\put(390.67,441){\rule{0.400pt}{0.482pt}}
\multiput(390.17,441.00)(1.000,1.000){2}{\rule{0.400pt}{0.241pt}}
\put(392,442.67){\rule{0.241pt}{0.400pt}}
\multiput(392.00,442.17)(0.500,1.000){2}{\rule{0.120pt}{0.400pt}}
\put(391.0,438.0){\rule[-0.200pt]{0.400pt}{0.723pt}}
\put(392.67,447){\rule{0.400pt}{0.723pt}}
\multiput(392.17,447.00)(1.000,1.500){2}{\rule{0.400pt}{0.361pt}}
\put(394,448.67){\rule{0.241pt}{0.400pt}}
\multiput(394.00,449.17)(0.500,-1.000){2}{\rule{0.120pt}{0.400pt}}
\put(393.0,444.0){\rule[-0.200pt]{0.400pt}{0.723pt}}
\put(395.0,449.0){\usebox{\plotpoint}}
\put(395.67,450){\rule{0.400pt}{0.964pt}}
\multiput(395.17,450.00)(1.000,2.000){2}{\rule{0.400pt}{0.482pt}}
\put(395.0,450.0){\usebox{\plotpoint}}
\put(396.67,455){\rule{0.400pt}{0.482pt}}
\multiput(396.17,455.00)(1.000,1.000){2}{\rule{0.400pt}{0.241pt}}
\put(398,456.67){\rule{0.241pt}{0.400pt}}
\multiput(398.00,456.17)(0.500,1.000){2}{\rule{0.120pt}{0.400pt}}
\put(397.0,454.0){\usebox{\plotpoint}}
\put(399.0,458.0){\rule[-0.200pt]{0.400pt}{0.482pt}}
\put(399.67,460){\rule{0.400pt}{0.723pt}}
\multiput(399.17,460.00)(1.000,1.500){2}{\rule{0.400pt}{0.361pt}}
\put(399.0,460.0){\usebox{\plotpoint}}
\put(401,462.67){\rule{0.241pt}{0.400pt}}
\multiput(401.00,463.17)(0.500,-1.000){2}{\rule{0.120pt}{0.400pt}}
\put(401.67,463){\rule{0.400pt}{0.723pt}}
\multiput(401.17,463.00)(1.000,1.500){2}{\rule{0.400pt}{0.361pt}}
\put(401.0,463.0){\usebox{\plotpoint}}
\put(402.67,467){\rule{0.400pt}{0.964pt}}
\multiput(402.17,467.00)(1.000,2.000){2}{\rule{0.400pt}{0.482pt}}
\put(403.0,466.0){\usebox{\plotpoint}}
\put(403.67,468){\rule{0.400pt}{0.482pt}}
\multiput(403.17,469.00)(1.000,-1.000){2}{\rule{0.400pt}{0.241pt}}
\put(404.67,468){\rule{0.400pt}{1.686pt}}
\multiput(404.17,468.00)(1.000,3.500){2}{\rule{0.400pt}{0.843pt}}
\put(404.0,470.0){\usebox{\plotpoint}}
\put(406,475){\usebox{\plotpoint}}
\put(406.67,475){\rule{0.400pt}{0.723pt}}
\multiput(406.17,475.00)(1.000,1.500){2}{\rule{0.400pt}{0.361pt}}
\put(406.0,475.0){\usebox{\plotpoint}}
\put(408,480.67){\rule{0.241pt}{0.400pt}}
\multiput(408.00,480.17)(0.500,1.000){2}{\rule{0.120pt}{0.400pt}}
\put(408.67,482){\rule{0.400pt}{0.482pt}}
\multiput(408.17,482.00)(1.000,1.000){2}{\rule{0.400pt}{0.241pt}}
\put(408.0,478.0){\rule[-0.200pt]{0.400pt}{0.723pt}}
\put(410,485.67){\rule{0.241pt}{0.400pt}}
\multiput(410.00,485.17)(0.500,1.000){2}{\rule{0.120pt}{0.400pt}}
\put(411,486.67){\rule{0.241pt}{0.400pt}}
\multiput(411.00,486.17)(0.500,1.000){2}{\rule{0.120pt}{0.400pt}}
\put(410.0,484.0){\rule[-0.200pt]{0.400pt}{0.482pt}}
\put(411.67,489){\rule{0.400pt}{0.482pt}}
\multiput(411.17,489.00)(1.000,1.000){2}{\rule{0.400pt}{0.241pt}}
\put(412.67,491){\rule{0.400pt}{0.482pt}}
\multiput(412.17,491.00)(1.000,1.000){2}{\rule{0.400pt}{0.241pt}}
\put(412.0,488.0){\usebox{\plotpoint}}
\put(414,494.67){\rule{0.241pt}{0.400pt}}
\multiput(414.00,494.17)(0.500,1.000){2}{\rule{0.120pt}{0.400pt}}
\put(415,495.67){\rule{0.241pt}{0.400pt}}
\multiput(415.00,495.17)(0.500,1.000){2}{\rule{0.120pt}{0.400pt}}
\put(414.0,493.0){\rule[-0.200pt]{0.400pt}{0.482pt}}
\put(415.67,498){\rule{0.400pt}{0.723pt}}
\multiput(415.17,498.00)(1.000,1.500){2}{\rule{0.400pt}{0.361pt}}
\put(416.67,498){\rule{0.400pt}{0.723pt}}
\multiput(416.17,499.50)(1.000,-1.500){2}{\rule{0.400pt}{0.361pt}}
\put(416.0,497.0){\usebox{\plotpoint}}
\put(417.67,502){\rule{0.400pt}{0.482pt}}
\multiput(417.17,502.00)(1.000,1.000){2}{\rule{0.400pt}{0.241pt}}
\put(418.67,504){\rule{0.400pt}{0.482pt}}
\multiput(418.17,504.00)(1.000,1.000){2}{\rule{0.400pt}{0.241pt}}
\put(418.0,498.0){\rule[-0.200pt]{0.400pt}{0.964pt}}
\put(420.0,506.0){\rule[-0.200pt]{0.400pt}{0.723pt}}
\put(420.67,509){\rule{0.400pt}{0.723pt}}
\multiput(420.17,509.00)(1.000,1.500){2}{\rule{0.400pt}{0.361pt}}
\put(420.0,509.0){\usebox{\plotpoint}}
\put(422,512.67){\rule{0.241pt}{0.400pt}}
\multiput(422.00,513.17)(0.500,-1.000){2}{\rule{0.120pt}{0.400pt}}
\put(422.67,513){\rule{0.400pt}{0.482pt}}
\multiput(422.17,513.00)(1.000,1.000){2}{\rule{0.400pt}{0.241pt}}
\put(422.0,512.0){\rule[-0.200pt]{0.400pt}{0.482pt}}
\put(424,515){\usebox{\plotpoint}}
\put(424,514.67){\rule{0.241pt}{0.400pt}}
\multiput(424.00,514.17)(0.500,1.000){2}{\rule{0.120pt}{0.400pt}}
\put(424.67,516){\rule{0.400pt}{1.445pt}}
\multiput(424.17,516.00)(1.000,3.000){2}{\rule{0.400pt}{0.723pt}}
\put(425.67,520){\rule{0.400pt}{0.723pt}}
\multiput(425.17,520.00)(1.000,1.500){2}{\rule{0.400pt}{0.361pt}}
\put(427,522.67){\rule{0.241pt}{0.400pt}}
\multiput(427.00,522.17)(0.500,1.000){2}{\rule{0.120pt}{0.400pt}}
\put(426.0,520.0){\rule[-0.200pt]{0.400pt}{0.482pt}}
\put(428.0,524.0){\rule[-0.200pt]{0.400pt}{0.482pt}}
\put(428.67,526){\rule{0.400pt}{0.482pt}}
\multiput(428.17,526.00)(1.000,1.000){2}{\rule{0.400pt}{0.241pt}}
\put(428.0,526.0){\usebox{\plotpoint}}
\put(430,530.67){\rule{0.241pt}{0.400pt}}
\multiput(430.00,530.17)(0.500,1.000){2}{\rule{0.120pt}{0.400pt}}
\put(430.67,529){\rule{0.400pt}{0.723pt}}
\multiput(430.17,530.50)(1.000,-1.500){2}{\rule{0.400pt}{0.361pt}}
\put(430.0,528.0){\rule[-0.200pt]{0.400pt}{0.723pt}}
\put(432,534.67){\rule{0.241pt}{0.400pt}}
\multiput(432.00,534.17)(0.500,1.000){2}{\rule{0.120pt}{0.400pt}}
\put(432.0,529.0){\rule[-0.200pt]{0.400pt}{1.445pt}}
\put(432.67,538){\rule{0.400pt}{0.482pt}}
\multiput(432.17,538.00)(1.000,1.000){2}{\rule{0.400pt}{0.241pt}}
\put(433.67,540){\rule{0.400pt}{0.723pt}}
\multiput(433.17,540.00)(1.000,1.500){2}{\rule{0.400pt}{0.361pt}}
\put(433.0,536.0){\rule[-0.200pt]{0.400pt}{0.482pt}}
\put(434.67,544){\rule{0.400pt}{0.482pt}}
\multiput(434.17,545.00)(1.000,-1.000){2}{\rule{0.400pt}{0.241pt}}
\put(435.67,544){\rule{0.400pt}{0.723pt}}
\multiput(435.17,544.00)(1.000,1.500){2}{\rule{0.400pt}{0.361pt}}
\put(435.0,543.0){\rule[-0.200pt]{0.400pt}{0.723pt}}
\put(436.67,546){\rule{0.400pt}{0.482pt}}
\multiput(436.17,546.00)(1.000,1.000){2}{\rule{0.400pt}{0.241pt}}
\put(437.67,548){\rule{0.400pt}{0.723pt}}
\multiput(437.17,548.00)(1.000,1.500){2}{\rule{0.400pt}{0.361pt}}
\put(437.0,546.0){\usebox{\plotpoint}}
\put(438.67,548){\rule{0.400pt}{1.204pt}}
\multiput(438.17,548.00)(1.000,2.500){2}{\rule{0.400pt}{0.602pt}}
\put(439.67,551){\rule{0.400pt}{0.482pt}}
\multiput(439.17,552.00)(1.000,-1.000){2}{\rule{0.400pt}{0.241pt}}
\put(439.0,548.0){\rule[-0.200pt]{0.400pt}{0.723pt}}
\put(440.67,553){\rule{0.400pt}{1.445pt}}
\multiput(440.17,553.00)(1.000,3.000){2}{\rule{0.400pt}{0.723pt}}
\put(442,558.67){\rule{0.241pt}{0.400pt}}
\multiput(442.00,558.17)(0.500,1.000){2}{\rule{0.120pt}{0.400pt}}
\put(441.0,551.0){\rule[-0.200pt]{0.400pt}{0.482pt}}
\put(442.67,557){\rule{0.400pt}{0.964pt}}
\multiput(442.17,557.00)(1.000,2.000){2}{\rule{0.400pt}{0.482pt}}
\put(443.67,559){\rule{0.400pt}{0.482pt}}
\multiput(443.17,560.00)(1.000,-1.000){2}{\rule{0.400pt}{0.241pt}}
\put(443.0,557.0){\rule[-0.200pt]{0.400pt}{0.723pt}}
\put(445.0,559.0){\rule[-0.200pt]{0.400pt}{1.445pt}}
\put(445.67,565){\rule{0.400pt}{0.723pt}}
\multiput(445.17,565.00)(1.000,1.500){2}{\rule{0.400pt}{0.361pt}}
\put(445.0,565.0){\usebox{\plotpoint}}
\put(447,568){\usebox{\plotpoint}}
\put(446.67,566){\rule{0.400pt}{0.482pt}}
\multiput(446.17,567.00)(1.000,-1.000){2}{\rule{0.400pt}{0.241pt}}
\put(447.67,566){\rule{0.400pt}{0.482pt}}
\multiput(447.17,566.00)(1.000,1.000){2}{\rule{0.400pt}{0.241pt}}
\put(449,570.67){\rule{0.241pt}{0.400pt}}
\multiput(449.00,570.17)(0.500,1.000){2}{\rule{0.120pt}{0.400pt}}
\put(449.67,572){\rule{0.400pt}{0.723pt}}
\multiput(449.17,572.00)(1.000,1.500){2}{\rule{0.400pt}{0.361pt}}
\put(449.0,568.0){\rule[-0.200pt]{0.400pt}{0.723pt}}
\put(451,573.67){\rule{0.241pt}{0.400pt}}
\multiput(451.00,573.17)(0.500,1.000){2}{\rule{0.120pt}{0.400pt}}
\put(451.67,575){\rule{0.400pt}{0.482pt}}
\multiput(451.17,575.00)(1.000,1.000){2}{\rule{0.400pt}{0.241pt}}
\put(451.0,574.0){\usebox{\plotpoint}}
\put(453,576.67){\rule{0.241pt}{0.400pt}}
\multiput(453.00,577.17)(0.500,-1.000){2}{\rule{0.120pt}{0.400pt}}
\put(453.67,577){\rule{0.400pt}{0.964pt}}
\multiput(453.17,577.00)(1.000,2.000){2}{\rule{0.400pt}{0.482pt}}
\put(453.0,577.0){\usebox{\plotpoint}}
\put(455,581){\usebox{\plotpoint}}
\put(454.67,581){\rule{0.400pt}{0.482pt}}
\multiput(454.17,581.00)(1.000,1.000){2}{\rule{0.400pt}{0.241pt}}
\put(455.67,580){\rule{0.400pt}{0.723pt}}
\multiput(455.17,581.50)(1.000,-1.500){2}{\rule{0.400pt}{0.361pt}}
\put(457.0,580.0){\rule[-0.200pt]{0.400pt}{0.723pt}}
\put(457.0,583.0){\usebox{\plotpoint}}
\put(458.0,583.0){\rule[-0.200pt]{0.400pt}{0.723pt}}
\put(458.67,586){\rule{0.400pt}{0.482pt}}
\multiput(458.17,586.00)(1.000,1.000){2}{\rule{0.400pt}{0.241pt}}
\put(458.0,586.0){\usebox{\plotpoint}}
\put(460,588){\usebox{\plotpoint}}
\put(460,587.67){\rule{0.241pt}{0.400pt}}
\multiput(460.00,587.17)(0.500,1.000){2}{\rule{0.120pt}{0.400pt}}
\put(460.67,589){\rule{0.400pt}{0.964pt}}
\multiput(460.17,589.00)(1.000,2.000){2}{\rule{0.400pt}{0.482pt}}
\put(462,593){\usebox{\plotpoint}}
\put(461.67,590){\rule{0.400pt}{0.723pt}}
\multiput(461.17,591.50)(1.000,-1.500){2}{\rule{0.400pt}{0.361pt}}
\put(462.67,590){\rule{0.400pt}{1.204pt}}
\multiput(462.17,590.00)(1.000,2.500){2}{\rule{0.400pt}{0.602pt}}
\put(463.67,592){\rule{0.400pt}{0.964pt}}
\multiput(463.17,594.00)(1.000,-2.000){2}{\rule{0.400pt}{0.482pt}}
\put(464.67,592){\rule{0.400pt}{1.445pt}}
\multiput(464.17,592.00)(1.000,3.000){2}{\rule{0.400pt}{0.723pt}}
\put(464.0,595.0){\usebox{\plotpoint}}
\put(466.0,595.0){\rule[-0.200pt]{0.400pt}{0.723pt}}
\put(466.67,595){\rule{0.400pt}{1.445pt}}
\multiput(466.17,595.00)(1.000,3.000){2}{\rule{0.400pt}{0.723pt}}
\put(466.0,595.0){\usebox{\plotpoint}}
\put(468,597.67){\rule{0.241pt}{0.400pt}}
\multiput(468.00,598.17)(0.500,-1.000){2}{\rule{0.120pt}{0.400pt}}
\put(468.67,598){\rule{0.400pt}{0.723pt}}
\multiput(468.17,598.00)(1.000,1.500){2}{\rule{0.400pt}{0.361pt}}
\put(468.0,599.0){\rule[-0.200pt]{0.400pt}{0.482pt}}
\put(470.0,599.0){\rule[-0.200pt]{0.400pt}{0.482pt}}
\put(470.67,599){\rule{0.400pt}{0.482pt}}
\multiput(470.17,599.00)(1.000,1.000){2}{\rule{0.400pt}{0.241pt}}
\put(470.0,599.0){\usebox{\plotpoint}}
\put(472,604.67){\rule{0.241pt}{0.400pt}}
\multiput(472.00,605.17)(0.500,-1.000){2}{\rule{0.120pt}{0.400pt}}
\put(473,603.67){\rule{0.241pt}{0.400pt}}
\multiput(473.00,604.17)(0.500,-1.000){2}{\rule{0.120pt}{0.400pt}}
\put(472.0,601.0){\rule[-0.200pt]{0.400pt}{1.204pt}}
\put(473.67,603){\rule{0.400pt}{0.723pt}}
\multiput(473.17,603.00)(1.000,1.500){2}{\rule{0.400pt}{0.361pt}}
\put(474.67,604){\rule{0.400pt}{0.482pt}}
\multiput(474.17,605.00)(1.000,-1.000){2}{\rule{0.400pt}{0.241pt}}
\put(474.0,603.0){\usebox{\plotpoint}}
\put(476,605.67){\rule{0.241pt}{0.400pt}}
\multiput(476.00,605.17)(0.500,1.000){2}{\rule{0.120pt}{0.400pt}}
\put(476.67,603){\rule{0.400pt}{0.964pt}}
\multiput(476.17,605.00)(1.000,-2.000){2}{\rule{0.400pt}{0.482pt}}
\put(476.0,604.0){\rule[-0.200pt]{0.400pt}{0.482pt}}
\put(478,605.67){\rule{0.241pt}{0.400pt}}
\multiput(478.00,606.17)(0.500,-1.000){2}{\rule{0.120pt}{0.400pt}}
\put(478.0,603.0){\rule[-0.200pt]{0.400pt}{0.964pt}}
\put(479.0,606.0){\rule[-0.200pt]{0.400pt}{0.723pt}}
\put(479.67,604){\rule{0.400pt}{1.204pt}}
\multiput(479.17,606.50)(1.000,-2.500){2}{\rule{0.400pt}{0.602pt}}
\put(479.0,609.0){\usebox{\plotpoint}}
\put(481.0,604.0){\rule[-0.200pt]{0.400pt}{1.204pt}}
\put(481.67,605){\rule{0.400pt}{0.964pt}}
\multiput(481.17,607.00)(1.000,-2.000){2}{\rule{0.400pt}{0.482pt}}
\put(481.0,609.0){\usebox{\plotpoint}}
\put(482.67,608){\rule{0.400pt}{0.964pt}}
\multiput(482.17,610.00)(1.000,-2.000){2}{\rule{0.400pt}{0.482pt}}
\put(483.67,608){\rule{0.400pt}{0.482pt}}
\multiput(483.17,608.00)(1.000,1.000){2}{\rule{0.400pt}{0.241pt}}
\put(483.0,605.0){\rule[-0.200pt]{0.400pt}{1.686pt}}
\put(485,606.67){\rule{0.241pt}{0.400pt}}
\multiput(485.00,606.17)(0.500,1.000){2}{\rule{0.120pt}{0.400pt}}
\put(485.0,607.0){\rule[-0.200pt]{0.400pt}{0.723pt}}
\put(486.0,608.0){\usebox{\plotpoint}}
\put(486.67,603){\rule{0.400pt}{1.686pt}}
\multiput(486.17,606.50)(1.000,-3.500){2}{\rule{0.400pt}{0.843pt}}
\put(488,601.67){\rule{0.241pt}{0.400pt}}
\multiput(488.00,602.17)(0.500,-1.000){2}{\rule{0.120pt}{0.400pt}}
\put(487.0,608.0){\rule[-0.200pt]{0.400pt}{0.482pt}}
\put(488.67,608){\rule{0.400pt}{1.445pt}}
\multiput(488.17,611.00)(1.000,-3.000){2}{\rule{0.400pt}{0.723pt}}
\put(489.67,608){\rule{0.400pt}{0.964pt}}
\multiput(489.17,608.00)(1.000,2.000){2}{\rule{0.400pt}{0.482pt}}
\put(489.0,602.0){\rule[-0.200pt]{0.400pt}{2.891pt}}
\put(490.67,605){\rule{0.400pt}{1.445pt}}
\multiput(490.17,608.00)(1.000,-3.000){2}{\rule{0.400pt}{0.723pt}}
\put(491.67,602){\rule{0.400pt}{0.723pt}}
\multiput(491.17,603.50)(1.000,-1.500){2}{\rule{0.400pt}{0.361pt}}
\put(491.0,611.0){\usebox{\plotpoint}}
\put(492.67,606){\rule{0.400pt}{1.204pt}}
\multiput(492.17,606.00)(1.000,2.500){2}{\rule{0.400pt}{0.602pt}}
\put(494,610.67){\rule{0.241pt}{0.400pt}}
\multiput(494.00,610.17)(0.500,1.000){2}{\rule{0.120pt}{0.400pt}}
\put(493.0,602.0){\rule[-0.200pt]{0.400pt}{0.964pt}}
\put(495,606.67){\rule{0.241pt}{0.400pt}}
\multiput(495.00,606.17)(0.500,1.000){2}{\rule{0.120pt}{0.400pt}}
\put(495.67,608){\rule{0.400pt}{0.723pt}}
\multiput(495.17,608.00)(1.000,1.500){2}{\rule{0.400pt}{0.361pt}}
\put(495.0,607.0){\rule[-0.200pt]{0.400pt}{1.204pt}}
\put(497.0,608.0){\rule[-0.200pt]{0.400pt}{0.723pt}}
\put(497.0,608.0){\usebox{\plotpoint}}
\put(497.67,603){\rule{0.400pt}{1.927pt}}
\multiput(497.17,607.00)(1.000,-4.000){2}{\rule{0.400pt}{0.964pt}}
\put(499,602.67){\rule{0.241pt}{0.400pt}}
\multiput(499.00,602.17)(0.500,1.000){2}{\rule{0.120pt}{0.400pt}}
\put(498.0,608.0){\rule[-0.200pt]{0.400pt}{0.723pt}}
\put(499.67,603){\rule{0.400pt}{1.445pt}}
\multiput(499.17,603.00)(1.000,3.000){2}{\rule{0.400pt}{0.723pt}}
\put(500.67,605){\rule{0.400pt}{0.964pt}}
\multiput(500.17,607.00)(1.000,-2.000){2}{\rule{0.400pt}{0.482pt}}
\put(500.0,603.0){\usebox{\plotpoint}}
\put(501.67,601){\rule{0.400pt}{1.204pt}}
\multiput(501.17,601.00)(1.000,2.500){2}{\rule{0.400pt}{0.602pt}}
\put(502.67,602){\rule{0.400pt}{0.964pt}}
\multiput(502.17,604.00)(1.000,-2.000){2}{\rule{0.400pt}{0.482pt}}
\put(502.0,601.0){\rule[-0.200pt]{0.400pt}{0.964pt}}
\put(504,603.67){\rule{0.241pt}{0.400pt}}
\multiput(504.00,604.17)(0.500,-1.000){2}{\rule{0.120pt}{0.400pt}}
\put(504.0,602.0){\rule[-0.200pt]{0.400pt}{0.723pt}}
\put(505.0,604.0){\usebox{\plotpoint}}
\put(506.0,604.0){\usebox{\plotpoint}}
\put(506.67,602){\rule{0.400pt}{0.723pt}}
\multiput(506.17,603.50)(1.000,-1.500){2}{\rule{0.400pt}{0.361pt}}
\put(506.0,605.0){\usebox{\plotpoint}}
\put(507.67,602){\rule{0.400pt}{0.964pt}}
\multiput(507.17,604.00)(1.000,-2.000){2}{\rule{0.400pt}{0.482pt}}
\put(509,600.67){\rule{0.241pt}{0.400pt}}
\multiput(509.00,601.17)(0.500,-1.000){2}{\rule{0.120pt}{0.400pt}}
\put(508.0,602.0){\rule[-0.200pt]{0.400pt}{0.964pt}}
\put(510,601){\usebox{\plotpoint}}
\put(509.67,601){\rule{0.400pt}{0.723pt}}
\multiput(509.17,601.00)(1.000,1.500){2}{\rule{0.400pt}{0.361pt}}
\put(510.67,596){\rule{0.400pt}{1.927pt}}
\multiput(510.17,600.00)(1.000,-4.000){2}{\rule{0.400pt}{0.964pt}}
\put(512,598.67){\rule{0.241pt}{0.400pt}}
\multiput(512.00,598.17)(0.500,1.000){2}{\rule{0.120pt}{0.400pt}}
\put(512.67,600){\rule{0.400pt}{0.723pt}}
\multiput(512.17,600.00)(1.000,1.500){2}{\rule{0.400pt}{0.361pt}}
\put(512.0,596.0){\rule[-0.200pt]{0.400pt}{0.723pt}}
\put(514,598.67){\rule{0.241pt}{0.400pt}}
\multiput(514.00,599.17)(0.500,-1.000){2}{\rule{0.120pt}{0.400pt}}
\put(514.0,600.0){\rule[-0.200pt]{0.400pt}{0.723pt}}
\put(514.67,595){\rule{0.400pt}{1.204pt}}
\multiput(514.17,597.50)(1.000,-2.500){2}{\rule{0.400pt}{0.602pt}}
\put(516,593.67){\rule{0.241pt}{0.400pt}}
\multiput(516.00,594.17)(0.500,-1.000){2}{\rule{0.120pt}{0.400pt}}
\put(515.0,599.0){\usebox{\plotpoint}}
\put(516.67,593){\rule{0.400pt}{0.723pt}}
\multiput(516.17,594.50)(1.000,-1.500){2}{\rule{0.400pt}{0.361pt}}
\put(518,591.67){\rule{0.241pt}{0.400pt}}
\multiput(518.00,592.17)(0.500,-1.000){2}{\rule{0.120pt}{0.400pt}}
\put(517.0,594.0){\rule[-0.200pt]{0.400pt}{0.482pt}}
\put(518.67,587){\rule{0.400pt}{1.927pt}}
\multiput(518.17,591.00)(1.000,-4.000){2}{\rule{0.400pt}{0.964pt}}
\put(519.67,587){\rule{0.400pt}{0.482pt}}
\multiput(519.17,587.00)(1.000,1.000){2}{\rule{0.400pt}{0.241pt}}
\put(519.0,592.0){\rule[-0.200pt]{0.400pt}{0.723pt}}
\put(520.67,595){\rule{0.400pt}{0.482pt}}
\multiput(520.17,595.00)(1.000,1.000){2}{\rule{0.400pt}{0.241pt}}
\put(521.67,591){\rule{0.400pt}{1.445pt}}
\multiput(521.17,594.00)(1.000,-3.000){2}{\rule{0.400pt}{0.723pt}}
\put(521.0,589.0){\rule[-0.200pt]{0.400pt}{1.445pt}}
\put(523.0,590.0){\usebox{\plotpoint}}
\put(524,589.67){\rule{0.241pt}{0.400pt}}
\multiput(524.00,589.17)(0.500,1.000){2}{\rule{0.120pt}{0.400pt}}
\put(523.0,590.0){\usebox{\plotpoint}}
\put(524.67,586){\rule{0.400pt}{0.964pt}}
\multiput(524.17,586.00)(1.000,2.000){2}{\rule{0.400pt}{0.482pt}}
\put(525.67,588){\rule{0.400pt}{0.482pt}}
\multiput(525.17,589.00)(1.000,-1.000){2}{\rule{0.400pt}{0.241pt}}
\put(525.0,586.0){\rule[-0.200pt]{0.400pt}{1.204pt}}
\put(526.67,587){\rule{0.400pt}{0.482pt}}
\multiput(526.17,587.00)(1.000,1.000){2}{\rule{0.400pt}{0.241pt}}
\put(527.67,584){\rule{0.400pt}{1.204pt}}
\multiput(527.17,586.50)(1.000,-2.500){2}{\rule{0.400pt}{0.602pt}}
\put(527.0,587.0){\usebox{\plotpoint}}
\put(529,580.67){\rule{0.241pt}{0.400pt}}
\multiput(529.00,581.17)(0.500,-1.000){2}{\rule{0.120pt}{0.400pt}}
\put(529.0,582.0){\rule[-0.200pt]{0.400pt}{0.482pt}}
\put(529.67,585){\rule{0.400pt}{0.482pt}}
\multiput(529.17,585.00)(1.000,1.000){2}{\rule{0.400pt}{0.241pt}}
\put(530.67,583){\rule{0.400pt}{0.964pt}}
\multiput(530.17,585.00)(1.000,-2.000){2}{\rule{0.400pt}{0.482pt}}
\put(530.0,581.0){\rule[-0.200pt]{0.400pt}{0.964pt}}
\put(531.67,580){\rule{0.400pt}{1.204pt}}
\multiput(531.17,580.00)(1.000,2.500){2}{\rule{0.400pt}{0.602pt}}
\put(532.67,579){\rule{0.400pt}{1.445pt}}
\multiput(532.17,582.00)(1.000,-3.000){2}{\rule{0.400pt}{0.723pt}}
\put(532.0,580.0){\rule[-0.200pt]{0.400pt}{0.723pt}}
\put(534.0,579.0){\usebox{\plotpoint}}
\put(534.67,578){\rule{0.400pt}{0.482pt}}
\multiput(534.17,579.00)(1.000,-1.000){2}{\rule{0.400pt}{0.241pt}}
\put(534.0,580.0){\usebox{\plotpoint}}
\put(535.67,574){\rule{0.400pt}{1.686pt}}
\multiput(535.17,574.00)(1.000,3.500){2}{\rule{0.400pt}{0.843pt}}
\put(536.67,578){\rule{0.400pt}{0.723pt}}
\multiput(536.17,579.50)(1.000,-1.500){2}{\rule{0.400pt}{0.361pt}}
\put(536.0,574.0){\rule[-0.200pt]{0.400pt}{0.964pt}}
\put(537.67,574){\rule{0.400pt}{0.482pt}}
\multiput(537.17,575.00)(1.000,-1.000){2}{\rule{0.400pt}{0.241pt}}
\put(538.0,576.0){\rule[-0.200pt]{0.400pt}{0.482pt}}
\put(539.0,574.0){\usebox{\plotpoint}}
\put(539.67,572){\rule{0.400pt}{0.482pt}}
\multiput(539.17,572.00)(1.000,1.000){2}{\rule{0.400pt}{0.241pt}}
\put(540.67,570){\rule{0.400pt}{0.964pt}}
\multiput(540.17,572.00)(1.000,-2.000){2}{\rule{0.400pt}{0.482pt}}
\put(540.0,572.0){\rule[-0.200pt]{0.400pt}{0.482pt}}
\put(541.67,569){\rule{0.400pt}{1.927pt}}
\multiput(541.17,573.00)(1.000,-4.000){2}{\rule{0.400pt}{0.964pt}}
\put(542.67,569){\rule{0.400pt}{1.445pt}}
\multiput(542.17,569.00)(1.000,3.000){2}{\rule{0.400pt}{0.723pt}}
\put(542.0,570.0){\rule[-0.200pt]{0.400pt}{1.686pt}}
\put(544.0,570.0){\rule[-0.200pt]{0.400pt}{1.204pt}}
\put(544.0,570.0){\usebox{\plotpoint}}
\put(544.67,567){\rule{0.400pt}{1.445pt}}
\multiput(544.17,570.00)(1.000,-3.000){2}{\rule{0.400pt}{0.723pt}}
\put(545.67,567){\rule{0.400pt}{0.964pt}}
\multiput(545.17,567.00)(1.000,2.000){2}{\rule{0.400pt}{0.482pt}}
\put(545.0,570.0){\rule[-0.200pt]{0.400pt}{0.723pt}}
\put(546.67,570){\rule{0.400pt}{0.723pt}}
\multiput(546.17,571.50)(1.000,-1.500){2}{\rule{0.400pt}{0.361pt}}
\put(547.67,566){\rule{0.400pt}{0.964pt}}
\multiput(547.17,568.00)(1.000,-2.000){2}{\rule{0.400pt}{0.482pt}}
\put(547.0,571.0){\rule[-0.200pt]{0.400pt}{0.482pt}}
\put(548.67,565){\rule{0.400pt}{1.686pt}}
\multiput(548.17,568.50)(1.000,-3.500){2}{\rule{0.400pt}{0.843pt}}
\put(549.67,565){\rule{0.400pt}{0.723pt}}
\multiput(549.17,565.00)(1.000,1.500){2}{\rule{0.400pt}{0.361pt}}
\put(549.0,566.0){\rule[-0.200pt]{0.400pt}{1.445pt}}
\put(551,562.67){\rule{0.241pt}{0.400pt}}
\multiput(551.00,562.17)(0.500,1.000){2}{\rule{0.120pt}{0.400pt}}
\put(551.0,563.0){\rule[-0.200pt]{0.400pt}{1.204pt}}
\put(552.0,564.0){\usebox{\plotpoint}}
\put(552.67,559){\rule{0.400pt}{0.964pt}}
\multiput(552.17,561.00)(1.000,-2.000){2}{\rule{0.400pt}{0.482pt}}
\put(553.67,557){\rule{0.400pt}{0.482pt}}
\multiput(553.17,558.00)(1.000,-1.000){2}{\rule{0.400pt}{0.241pt}}
\put(553.0,563.0){\usebox{\plotpoint}}
\put(555,560.67){\rule{0.241pt}{0.400pt}}
\multiput(555.00,560.17)(0.500,1.000){2}{\rule{0.120pt}{0.400pt}}
\put(555.67,559){\rule{0.400pt}{0.723pt}}
\multiput(555.17,560.50)(1.000,-1.500){2}{\rule{0.400pt}{0.361pt}}
\put(555.0,557.0){\rule[-0.200pt]{0.400pt}{0.964pt}}
\put(557.0,559.0){\rule[-0.200pt]{0.400pt}{0.964pt}}
\put(557.67,560){\rule{0.400pt}{0.723pt}}
\multiput(557.17,561.50)(1.000,-1.500){2}{\rule{0.400pt}{0.361pt}}
\put(557.0,563.0){\usebox{\plotpoint}}
\put(558.67,556){\rule{0.400pt}{0.482pt}}
\multiput(558.17,557.00)(1.000,-1.000){2}{\rule{0.400pt}{0.241pt}}
\put(559.0,558.0){\rule[-0.200pt]{0.400pt}{0.482pt}}
\put(559.67,555){\rule{0.400pt}{0.482pt}}
\multiput(559.17,555.00)(1.000,1.000){2}{\rule{0.400pt}{0.241pt}}
\put(560.67,555){\rule{0.400pt}{0.482pt}}
\multiput(560.17,556.00)(1.000,-1.000){2}{\rule{0.400pt}{0.241pt}}
\put(560.0,555.0){\usebox{\plotpoint}}
\put(561.67,554){\rule{0.400pt}{1.204pt}}
\multiput(561.17,554.00)(1.000,2.500){2}{\rule{0.400pt}{0.602pt}}
\put(562.0,554.0){\usebox{\plotpoint}}
\put(563.0,559.0){\usebox{\plotpoint}}
\put(563.67,548){\rule{0.400pt}{1.686pt}}
\multiput(563.17,548.00)(1.000,3.500){2}{\rule{0.400pt}{0.843pt}}
\put(565,553.67){\rule{0.241pt}{0.400pt}}
\multiput(565.00,554.17)(0.500,-1.000){2}{\rule{0.120pt}{0.400pt}}
\put(564.0,548.0){\rule[-0.200pt]{0.400pt}{2.650pt}}
\put(566.0,551.0){\rule[-0.200pt]{0.400pt}{0.723pt}}
\put(566.67,549){\rule{0.400pt}{0.482pt}}
\multiput(566.17,550.00)(1.000,-1.000){2}{\rule{0.400pt}{0.241pt}}
\put(566.0,551.0){\usebox{\plotpoint}}
\put(568,549.67){\rule{0.241pt}{0.400pt}}
\multiput(568.00,549.17)(0.500,1.000){2}{\rule{0.120pt}{0.400pt}}
\put(568.67,547){\rule{0.400pt}{0.964pt}}
\multiput(568.17,549.00)(1.000,-2.000){2}{\rule{0.400pt}{0.482pt}}
\put(568.0,549.0){\usebox{\plotpoint}}
\put(570,549.67){\rule{0.241pt}{0.400pt}}
\multiput(570.00,549.17)(0.500,1.000){2}{\rule{0.120pt}{0.400pt}}
\put(570.67,543){\rule{0.400pt}{1.927pt}}
\multiput(570.17,547.00)(1.000,-4.000){2}{\rule{0.400pt}{0.964pt}}
\put(570.0,547.0){\rule[-0.200pt]{0.400pt}{0.723pt}}
\put(571.67,546){\rule{0.400pt}{0.482pt}}
\multiput(571.17,546.00)(1.000,1.000){2}{\rule{0.400pt}{0.241pt}}
\put(572.0,543.0){\rule[-0.200pt]{0.400pt}{0.723pt}}
\put(572.67,545){\rule{0.400pt}{0.964pt}}
\multiput(572.17,545.00)(1.000,2.000){2}{\rule{0.400pt}{0.482pt}}
\put(573.67,541){\rule{0.400pt}{1.927pt}}
\multiput(573.17,545.00)(1.000,-4.000){2}{\rule{0.400pt}{0.964pt}}
\put(573.0,545.0){\rule[-0.200pt]{0.400pt}{0.723pt}}
\put(574.67,543){\rule{0.400pt}{0.723pt}}
\multiput(574.17,544.50)(1.000,-1.500){2}{\rule{0.400pt}{0.361pt}}
\put(575.67,543){\rule{0.400pt}{0.723pt}}
\multiput(575.17,543.00)(1.000,1.500){2}{\rule{0.400pt}{0.361pt}}
\put(575.0,541.0){\rule[-0.200pt]{0.400pt}{1.204pt}}
\put(577,539.67){\rule{0.241pt}{0.400pt}}
\multiput(577.00,539.17)(0.500,1.000){2}{\rule{0.120pt}{0.400pt}}
\put(577.67,536){\rule{0.400pt}{1.204pt}}
\multiput(577.17,538.50)(1.000,-2.500){2}{\rule{0.400pt}{0.602pt}}
\put(577.0,540.0){\rule[-0.200pt]{0.400pt}{1.445pt}}
\put(578.67,539){\rule{0.400pt}{0.482pt}}
\multiput(578.17,539.00)(1.000,1.000){2}{\rule{0.400pt}{0.241pt}}
\put(579.67,535){\rule{0.400pt}{1.445pt}}
\multiput(579.17,538.00)(1.000,-3.000){2}{\rule{0.400pt}{0.723pt}}
\put(579.0,536.0){\rule[-0.200pt]{0.400pt}{0.723pt}}
\put(580.67,531){\rule{0.400pt}{1.445pt}}
\multiput(580.17,531.00)(1.000,3.000){2}{\rule{0.400pt}{0.723pt}}
\put(581.67,537){\rule{0.400pt}{1.204pt}}
\multiput(581.17,537.00)(1.000,2.500){2}{\rule{0.400pt}{0.602pt}}
\put(581.0,531.0){\rule[-0.200pt]{0.400pt}{0.964pt}}
\put(583,535.67){\rule{0.241pt}{0.400pt}}
\multiput(583.00,535.17)(0.500,1.000){2}{\rule{0.120pt}{0.400pt}}
\put(583.0,536.0){\rule[-0.200pt]{0.400pt}{1.445pt}}
\put(584,533.67){\rule{0.241pt}{0.400pt}}
\multiput(584.00,533.17)(0.500,1.000){2}{\rule{0.120pt}{0.400pt}}
\put(584.67,532){\rule{0.400pt}{0.723pt}}
\multiput(584.17,533.50)(1.000,-1.500){2}{\rule{0.400pt}{0.361pt}}
\put(584.0,534.0){\rule[-0.200pt]{0.400pt}{0.723pt}}
\put(585.67,527){\rule{0.400pt}{1.204pt}}
\multiput(585.17,527.00)(1.000,2.500){2}{\rule{0.400pt}{0.602pt}}
\put(586.0,527.0){\rule[-0.200pt]{0.400pt}{1.204pt}}
\put(587.0,532.0){\usebox{\plotpoint}}
\put(587.67,528){\rule{0.400pt}{0.723pt}}
\multiput(587.17,528.00)(1.000,1.500){2}{\rule{0.400pt}{0.361pt}}
\put(588.67,525){\rule{0.400pt}{1.445pt}}
\multiput(588.17,528.00)(1.000,-3.000){2}{\rule{0.400pt}{0.723pt}}
\put(588.0,528.0){\rule[-0.200pt]{0.400pt}{0.964pt}}
\put(589.67,524){\rule{0.400pt}{0.482pt}}
\multiput(589.17,525.00)(1.000,-1.000){2}{\rule{0.400pt}{0.241pt}}
\put(590.67,524){\rule{0.400pt}{0.723pt}}
\multiput(590.17,524.00)(1.000,1.500){2}{\rule{0.400pt}{0.361pt}}
\put(590.0,525.0){\usebox{\plotpoint}}
\put(591.67,525){\rule{0.400pt}{0.964pt}}
\multiput(591.17,527.00)(1.000,-2.000){2}{\rule{0.400pt}{0.482pt}}
\put(592.67,522){\rule{0.400pt}{0.723pt}}
\multiput(592.17,523.50)(1.000,-1.500){2}{\rule{0.400pt}{0.361pt}}
\put(592.0,527.0){\rule[-0.200pt]{0.400pt}{0.482pt}}
\put(593.67,520){\rule{0.400pt}{0.723pt}}
\multiput(593.17,520.00)(1.000,1.500){2}{\rule{0.400pt}{0.361pt}}
\put(594.67,520){\rule{0.400pt}{0.723pt}}
\multiput(594.17,521.50)(1.000,-1.500){2}{\rule{0.400pt}{0.361pt}}
\put(594.0,520.0){\rule[-0.200pt]{0.400pt}{0.482pt}}
\put(596,520){\usebox{\plotpoint}}
\put(596,518.67){\rule{0.241pt}{0.400pt}}
\multiput(596.00,519.17)(0.500,-1.000){2}{\rule{0.120pt}{0.400pt}}
\put(597,519.67){\rule{0.241pt}{0.400pt}}
\multiput(597.00,520.17)(0.500,-1.000){2}{\rule{0.120pt}{0.400pt}}
\put(597.67,518){\rule{0.400pt}{0.482pt}}
\multiput(597.17,519.00)(1.000,-1.000){2}{\rule{0.400pt}{0.241pt}}
\put(597.0,519.0){\rule[-0.200pt]{0.400pt}{0.482pt}}
\put(598.67,509){\rule{0.400pt}{1.927pt}}
\multiput(598.17,513.00)(1.000,-4.000){2}{\rule{0.400pt}{0.964pt}}
\put(599.67,509){\rule{0.400pt}{2.650pt}}
\multiput(599.17,509.00)(1.000,5.500){2}{\rule{0.400pt}{1.325pt}}
\put(599.0,517.0){\usebox{\plotpoint}}
\put(601.0,513.0){\rule[-0.200pt]{0.400pt}{1.686pt}}
\put(601.67,508){\rule{0.400pt}{1.204pt}}
\multiput(601.17,510.50)(1.000,-2.500){2}{\rule{0.400pt}{0.602pt}}
\put(601.0,513.0){\usebox{\plotpoint}}
\put(603.0,508.0){\rule[-0.200pt]{0.400pt}{1.445pt}}
\put(603.67,505){\rule{0.400pt}{2.168pt}}
\multiput(603.17,509.50)(1.000,-4.500){2}{\rule{0.400pt}{1.084pt}}
\put(603.0,514.0){\usebox{\plotpoint}}
\put(604.67,506){\rule{0.400pt}{1.204pt}}
\multiput(604.17,508.50)(1.000,-2.500){2}{\rule{0.400pt}{0.602pt}}
\put(605.67,503){\rule{0.400pt}{0.723pt}}
\multiput(605.17,504.50)(1.000,-1.500){2}{\rule{0.400pt}{0.361pt}}
\put(605.0,505.0){\rule[-0.200pt]{0.400pt}{1.445pt}}
\put(606.67,503){\rule{0.400pt}{0.964pt}}
\multiput(606.17,505.00)(1.000,-2.000){2}{\rule{0.400pt}{0.482pt}}
\put(607.0,503.0){\rule[-0.200pt]{0.400pt}{0.964pt}}
\put(607.67,499){\rule{0.400pt}{1.204pt}}
\multiput(607.17,501.50)(1.000,-2.500){2}{\rule{0.400pt}{0.602pt}}
\put(608.67,499){\rule{0.400pt}{0.482pt}}
\multiput(608.17,499.00)(1.000,1.000){2}{\rule{0.400pt}{0.241pt}}
\put(608.0,503.0){\usebox{\plotpoint}}
\put(609.67,499){\rule{0.400pt}{0.482pt}}
\multiput(609.17,499.00)(1.000,1.000){2}{\rule{0.400pt}{0.241pt}}
\put(610.67,497){\rule{0.400pt}{0.964pt}}
\multiput(610.17,499.00)(1.000,-2.000){2}{\rule{0.400pt}{0.482pt}}
\put(610.0,499.0){\rule[-0.200pt]{0.400pt}{0.482pt}}
\put(611.67,489){\rule{0.400pt}{2.168pt}}
\multiput(611.17,493.50)(1.000,-4.500){2}{\rule{0.400pt}{1.084pt}}
\put(612.67,489){\rule{0.400pt}{1.686pt}}
\multiput(612.17,489.00)(1.000,3.500){2}{\rule{0.400pt}{0.843pt}}
\put(612.0,497.0){\usebox{\plotpoint}}
\put(613.67,490){\rule{0.400pt}{0.723pt}}
\multiput(613.17,491.50)(1.000,-1.500){2}{\rule{0.400pt}{0.361pt}}
\put(614.67,488){\rule{0.400pt}{0.482pt}}
\multiput(614.17,489.00)(1.000,-1.000){2}{\rule{0.400pt}{0.241pt}}
\put(614.0,493.0){\rule[-0.200pt]{0.400pt}{0.723pt}}
\put(616.0,487.0){\usebox{\plotpoint}}
\put(616.67,484){\rule{0.400pt}{0.723pt}}
\multiput(616.17,485.50)(1.000,-1.500){2}{\rule{0.400pt}{0.361pt}}
\put(616.0,487.0){\usebox{\plotpoint}}
\put(618,484){\usebox{\plotpoint}}
\put(617.67,481){\rule{0.400pt}{0.723pt}}
\multiput(617.17,482.50)(1.000,-1.500){2}{\rule{0.400pt}{0.361pt}}
\put(619,481){\usebox{\plotpoint}}
\put(618.67,479){\rule{0.400pt}{0.482pt}}
\multiput(618.17,480.00)(1.000,-1.000){2}{\rule{0.400pt}{0.241pt}}
\put(619.67,477){\rule{0.400pt}{0.482pt}}
\multiput(619.17,478.00)(1.000,-1.000){2}{\rule{0.400pt}{0.241pt}}
\put(620.67,475){\rule{0.400pt}{1.204pt}}
\multiput(620.17,477.50)(1.000,-2.500){2}{\rule{0.400pt}{0.602pt}}
\put(621.0,477.0){\rule[-0.200pt]{0.400pt}{0.723pt}}
\put(622.67,473){\rule{0.400pt}{0.482pt}}
\multiput(622.17,474.00)(1.000,-1.000){2}{\rule{0.400pt}{0.241pt}}
\put(623.67,471){\rule{0.400pt}{0.482pt}}
\multiput(623.17,472.00)(1.000,-1.000){2}{\rule{0.400pt}{0.241pt}}
\put(622.0,475.0){\usebox{\plotpoint}}
\put(625,471){\usebox{\plotpoint}}
\put(624.67,465){\rule{0.400pt}{1.445pt}}
\multiput(624.17,468.00)(1.000,-3.000){2}{\rule{0.400pt}{0.723pt}}
\put(625.67,465){\rule{0.400pt}{0.723pt}}
\multiput(625.17,465.00)(1.000,1.500){2}{\rule{0.400pt}{0.361pt}}
\put(627.0,467.0){\usebox{\plotpoint}}
\put(627.67,464){\rule{0.400pt}{0.723pt}}
\multiput(627.17,465.50)(1.000,-1.500){2}{\rule{0.400pt}{0.361pt}}
\put(627.0,467.0){\usebox{\plotpoint}}
\put(628.67,458){\rule{0.400pt}{0.482pt}}
\multiput(628.17,458.00)(1.000,1.000){2}{\rule{0.400pt}{0.241pt}}
\put(629.0,458.0){\rule[-0.200pt]{0.400pt}{1.445pt}}
\put(630,460){\usebox{\plotpoint}}
\put(629.67,455){\rule{0.400pt}{1.204pt}}
\multiput(629.17,457.50)(1.000,-2.500){2}{\rule{0.400pt}{0.602pt}}
\put(631,454.67){\rule{0.241pt}{0.400pt}}
\multiput(631.00,454.17)(0.500,1.000){2}{\rule{0.120pt}{0.400pt}}
\put(632.0,452.0){\rule[-0.200pt]{0.400pt}{0.964pt}}
\put(632.0,452.0){\rule[-0.200pt]{0.482pt}{0.400pt}}
\put(634.0,447.0){\rule[-0.200pt]{0.400pt}{1.204pt}}
\put(635,445.67){\rule{0.241pt}{0.400pt}}
\multiput(635.00,446.17)(0.500,-1.000){2}{\rule{0.120pt}{0.400pt}}
\put(634.0,447.0){\usebox{\plotpoint}}
\put(635.67,441){\rule{0.400pt}{0.482pt}}
\multiput(635.17,442.00)(1.000,-1.000){2}{\rule{0.400pt}{0.241pt}}
\put(636.0,443.0){\rule[-0.200pt]{0.400pt}{0.723pt}}
\put(637.0,441.0){\usebox{\plotpoint}}
\put(638,437.67){\rule{0.241pt}{0.400pt}}
\multiput(638.00,437.17)(0.500,1.000){2}{\rule{0.120pt}{0.400pt}}
\put(638.0,438.0){\rule[-0.200pt]{0.400pt}{0.723pt}}
\put(639.0,436.0){\rule[-0.200pt]{0.400pt}{0.723pt}}
\put(639.67,434){\rule{0.400pt}{0.482pt}}
\multiput(639.17,435.00)(1.000,-1.000){2}{\rule{0.400pt}{0.241pt}}
\put(639.0,436.0){\usebox{\plotpoint}}
\put(640.67,430){\rule{0.400pt}{0.482pt}}
\multiput(640.17,430.00)(1.000,1.000){2}{\rule{0.400pt}{0.241pt}}
\put(641.67,427){\rule{0.400pt}{1.204pt}}
\multiput(641.17,429.50)(1.000,-2.500){2}{\rule{0.400pt}{0.602pt}}
\put(641.0,430.0){\rule[-0.200pt]{0.400pt}{0.964pt}}
\put(642.67,424){\rule{0.400pt}{0.482pt}}
\multiput(642.17,425.00)(1.000,-1.000){2}{\rule{0.400pt}{0.241pt}}
\put(643.67,420){\rule{0.400pt}{0.964pt}}
\multiput(643.17,422.00)(1.000,-2.000){2}{\rule{0.400pt}{0.482pt}}
\put(643.0,426.0){\usebox{\plotpoint}}
\put(645,420){\usebox{\plotpoint}}
\put(645.67,417){\rule{0.400pt}{0.723pt}}
\multiput(645.17,418.50)(1.000,-1.500){2}{\rule{0.400pt}{0.361pt}}
\put(645.0,420.0){\usebox{\plotpoint}}
\put(646.67,413){\rule{0.400pt}{0.723pt}}
\multiput(646.17,414.50)(1.000,-1.500){2}{\rule{0.400pt}{0.361pt}}
\put(647.0,416.0){\usebox{\plotpoint}}
\put(648.0,413.0){\usebox{\plotpoint}}
\put(648.67,408){\rule{0.400pt}{0.482pt}}
\multiput(648.17,408.00)(1.000,1.000){2}{\rule{0.400pt}{0.241pt}}
\put(649.0,408.0){\rule[-0.200pt]{0.400pt}{1.204pt}}
\put(649.67,405){\rule{0.400pt}{0.723pt}}
\multiput(649.17,406.50)(1.000,-1.500){2}{\rule{0.400pt}{0.361pt}}
\put(651,404.67){\rule{0.241pt}{0.400pt}}
\multiput(651.00,404.17)(0.500,1.000){2}{\rule{0.120pt}{0.400pt}}
\put(650.0,408.0){\rule[-0.200pt]{0.400pt}{0.482pt}}
\put(651.67,402){\rule{0.400pt}{0.723pt}}
\multiput(651.17,403.50)(1.000,-1.500){2}{\rule{0.400pt}{0.361pt}}
\put(653,400.67){\rule{0.241pt}{0.400pt}}
\multiput(653.00,401.17)(0.500,-1.000){2}{\rule{0.120pt}{0.400pt}}
\put(652.0,405.0){\usebox{\plotpoint}}
\put(653.67,395){\rule{0.400pt}{0.482pt}}
\multiput(653.17,396.00)(1.000,-1.000){2}{\rule{0.400pt}{0.241pt}}
\put(654.67,392){\rule{0.400pt}{0.723pt}}
\multiput(654.17,393.50)(1.000,-1.500){2}{\rule{0.400pt}{0.361pt}}
\put(654.0,397.0){\rule[-0.200pt]{0.400pt}{0.964pt}}
\put(656,392){\usebox{\plotpoint}}
\put(655.67,387){\rule{0.400pt}{1.204pt}}
\multiput(655.17,389.50)(1.000,-2.500){2}{\rule{0.400pt}{0.602pt}}
\put(657,386.67){\rule{0.241pt}{0.400pt}}
\multiput(657.00,386.17)(0.500,1.000){2}{\rule{0.120pt}{0.400pt}}
\put(657.67,384){\rule{0.400pt}{0.482pt}}
\multiput(657.17,385.00)(1.000,-1.000){2}{\rule{0.400pt}{0.241pt}}
\put(658.0,386.0){\rule[-0.200pt]{0.400pt}{0.482pt}}
\put(658.67,381){\rule{0.400pt}{1.204pt}}
\multiput(658.17,383.50)(1.000,-2.500){2}{\rule{0.400pt}{0.602pt}}
\put(659.67,378){\rule{0.400pt}{0.723pt}}
\multiput(659.17,379.50)(1.000,-1.500){2}{\rule{0.400pt}{0.361pt}}
\put(659.0,384.0){\rule[-0.200pt]{0.400pt}{0.482pt}}
\put(660.67,375){\rule{0.400pt}{0.964pt}}
\multiput(660.17,377.00)(1.000,-2.000){2}{\rule{0.400pt}{0.482pt}}
\put(661.67,375){\rule{0.400pt}{0.482pt}}
\multiput(661.17,375.00)(1.000,1.000){2}{\rule{0.400pt}{0.241pt}}
\put(661.0,378.0){\usebox{\plotpoint}}
\put(662.67,371){\rule{0.400pt}{0.723pt}}
\multiput(662.17,372.50)(1.000,-1.500){2}{\rule{0.400pt}{0.361pt}}
\put(663.0,374.0){\rule[-0.200pt]{0.400pt}{0.723pt}}
\put(664.0,371.0){\usebox{\plotpoint}}
\put(665,367.67){\rule{0.241pt}{0.400pt}}
\multiput(665.00,367.17)(0.500,1.000){2}{\rule{0.120pt}{0.400pt}}
\put(665.67,366){\rule{0.400pt}{0.723pt}}
\multiput(665.17,367.50)(1.000,-1.500){2}{\rule{0.400pt}{0.361pt}}
\put(665.0,368.0){\rule[-0.200pt]{0.400pt}{0.723pt}}
\put(666.67,362){\rule{0.400pt}{0.482pt}}
\multiput(666.17,362.00)(1.000,1.000){2}{\rule{0.400pt}{0.241pt}}
\put(667.0,362.0){\rule[-0.200pt]{0.400pt}{0.964pt}}
\put(668.0,360.0){\rule[-0.200pt]{0.400pt}{0.964pt}}
\put(668.67,356){\rule{0.400pt}{0.964pt}}
\multiput(668.17,358.00)(1.000,-2.000){2}{\rule{0.400pt}{0.482pt}}
\put(668.0,360.0){\usebox{\plotpoint}}
\put(670,356){\usebox{\plotpoint}}
\put(670,354.67){\rule{0.241pt}{0.400pt}}
\multiput(670.00,355.17)(0.500,-1.000){2}{\rule{0.120pt}{0.400pt}}
\put(671,353.67){\rule{0.241pt}{0.400pt}}
\multiput(671.00,354.17)(0.500,-1.000){2}{\rule{0.120pt}{0.400pt}}
\put(672,350.67){\rule{0.241pt}{0.400pt}}
\multiput(672.00,351.17)(0.500,-1.000){2}{\rule{0.120pt}{0.400pt}}
\put(672.67,349){\rule{0.400pt}{0.482pt}}
\multiput(672.17,350.00)(1.000,-1.000){2}{\rule{0.400pt}{0.241pt}}
\put(672.0,352.0){\rule[-0.200pt]{0.400pt}{0.482pt}}
\put(674,346.67){\rule{0.241pt}{0.400pt}}
\multiput(674.00,346.17)(0.500,1.000){2}{\rule{0.120pt}{0.400pt}}
\put(674.67,346){\rule{0.400pt}{0.482pt}}
\multiput(674.17,347.00)(1.000,-1.000){2}{\rule{0.400pt}{0.241pt}}
\put(674.0,347.0){\rule[-0.200pt]{0.400pt}{0.482pt}}
\put(676.0,344.0){\rule[-0.200pt]{0.400pt}{0.482pt}}
\put(676.0,344.0){\usebox{\plotpoint}}
\put(676.67,340){\rule{0.400pt}{0.482pt}}
\multiput(676.17,341.00)(1.000,-1.000){2}{\rule{0.400pt}{0.241pt}}
\put(677.67,338){\rule{0.400pt}{0.482pt}}
\multiput(677.17,339.00)(1.000,-1.000){2}{\rule{0.400pt}{0.241pt}}
\put(677.0,342.0){\rule[-0.200pt]{0.400pt}{0.482pt}}
\put(679,338){\usebox{\plotpoint}}
\put(678.67,336){\rule{0.400pt}{0.482pt}}
\multiput(678.17,337.00)(1.000,-1.000){2}{\rule{0.400pt}{0.241pt}}
\put(680.0,336.0){\usebox{\plotpoint}}
\put(681.0,334.0){\rule[-0.200pt]{0.400pt}{0.482pt}}
\put(681.67,330){\rule{0.400pt}{0.964pt}}
\multiput(681.17,332.00)(1.000,-2.000){2}{\rule{0.400pt}{0.482pt}}
\put(681.0,334.0){\usebox{\plotpoint}}
\put(683,330){\usebox{\plotpoint}}
\put(683,329.67){\rule{0.241pt}{0.400pt}}
\multiput(683.00,329.17)(0.500,1.000){2}{\rule{0.120pt}{0.400pt}}
\put(683.67,327){\rule{0.400pt}{0.964pt}}
\multiput(683.17,329.00)(1.000,-2.000){2}{\rule{0.400pt}{0.482pt}}
\put(685,327){\usebox{\plotpoint}}
\put(685,325.67){\rule{0.241pt}{0.400pt}}
\multiput(685.00,326.17)(0.500,-1.000){2}{\rule{0.120pt}{0.400pt}}
\put(686,321.67){\rule{0.241pt}{0.400pt}}
\multiput(686.00,322.17)(0.500,-1.000){2}{\rule{0.120pt}{0.400pt}}
\put(686.67,322){\rule{0.400pt}{0.482pt}}
\multiput(686.17,322.00)(1.000,1.000){2}{\rule{0.400pt}{0.241pt}}
\put(686.0,323.0){\rule[-0.200pt]{0.400pt}{0.723pt}}
\put(688.0,322.0){\rule[-0.200pt]{0.400pt}{0.482pt}}
\put(688.67,319){\rule{0.400pt}{0.723pt}}
\multiput(688.17,320.50)(1.000,-1.500){2}{\rule{0.400pt}{0.361pt}}
\put(688.0,322.0){\usebox{\plotpoint}}
\put(690,315.67){\rule{0.241pt}{0.400pt}}
\multiput(690.00,316.17)(0.500,-1.000){2}{\rule{0.120pt}{0.400pt}}
\put(691,315.67){\rule{0.241pt}{0.400pt}}
\multiput(691.00,315.17)(0.500,1.000){2}{\rule{0.120pt}{0.400pt}}
\put(690.0,317.0){\rule[-0.200pt]{0.400pt}{0.482pt}}
\put(692.0,315.0){\rule[-0.200pt]{0.400pt}{0.482pt}}
\put(693,313.67){\rule{0.241pt}{0.400pt}}
\multiput(693.00,314.17)(0.500,-1.000){2}{\rule{0.120pt}{0.400pt}}
\put(692.0,315.0){\usebox{\plotpoint}}
\put(694,314){\usebox{\plotpoint}}
\put(694.0,314.0){\usebox{\plotpoint}}
\put(694.67,310){\rule{0.400pt}{0.482pt}}
\multiput(694.17,311.00)(1.000,-1.000){2}{\rule{0.400pt}{0.241pt}}
\put(696,309.67){\rule{0.241pt}{0.400pt}}
\multiput(696.00,309.17)(0.500,1.000){2}{\rule{0.120pt}{0.400pt}}
\put(695.0,312.0){\rule[-0.200pt]{0.400pt}{0.482pt}}
\put(697,307.67){\rule{0.241pt}{0.400pt}}
\multiput(697.00,308.17)(0.500,-1.000){2}{\rule{0.120pt}{0.400pt}}
\put(697.0,309.0){\rule[-0.200pt]{0.400pt}{0.482pt}}
\put(698.67,305){\rule{0.400pt}{0.723pt}}
\multiput(698.17,306.50)(1.000,-1.500){2}{\rule{0.400pt}{0.361pt}}
\put(698.0,308.0){\usebox{\plotpoint}}
\put(700.0,305.0){\usebox{\plotpoint}}
\put(700.67,305){\rule{0.400pt}{0.482pt}}
\multiput(700.17,306.00)(1.000,-1.000){2}{\rule{0.400pt}{0.241pt}}
\put(701.67,303){\rule{0.400pt}{0.482pt}}
\multiput(701.17,304.00)(1.000,-1.000){2}{\rule{0.400pt}{0.241pt}}
\put(701.0,305.0){\rule[-0.200pt]{0.400pt}{0.482pt}}
\put(703,303){\usebox{\plotpoint}}
\put(702.67,303){\rule{0.400pt}{0.482pt}}
\multiput(702.17,303.00)(1.000,1.000){2}{\rule{0.400pt}{0.241pt}}
\put(703.67,302){\rule{0.400pt}{0.482pt}}
\multiput(703.17,303.00)(1.000,-1.000){2}{\rule{0.400pt}{0.241pt}}
\put(704.0,304.0){\usebox{\plotpoint}}
\put(705.0,302.0){\usebox{\plotpoint}}
\put(706,299.67){\rule{0.241pt}{0.400pt}}
\multiput(706.00,299.17)(0.500,1.000){2}{\rule{0.120pt}{0.400pt}}
\put(707,299.67){\rule{0.241pt}{0.400pt}}
\multiput(707.00,300.17)(0.500,-1.000){2}{\rule{0.120pt}{0.400pt}}
\put(706.0,300.0){\rule[-0.200pt]{0.400pt}{0.482pt}}
\put(708,297.67){\rule{0.241pt}{0.400pt}}
\multiput(708.00,297.17)(0.500,1.000){2}{\rule{0.120pt}{0.400pt}}
\put(708.0,298.0){\rule[-0.200pt]{0.400pt}{0.482pt}}
\put(709.0,299.0){\usebox{\plotpoint}}
\put(710.0,298.0){\usebox{\plotpoint}}
\put(710.0,298.0){\usebox{\plotpoint}}
\put(711,296.67){\rule{0.241pt}{0.400pt}}
\multiput(711.00,296.17)(0.500,1.000){2}{\rule{0.120pt}{0.400pt}}
\put(711.0,297.0){\usebox{\plotpoint}}
\put(712.0,298.0){\usebox{\plotpoint}}
\put(713,294.67){\rule{0.241pt}{0.400pt}}
\multiput(713.00,294.17)(0.500,1.000){2}{\rule{0.120pt}{0.400pt}}
\put(713.0,295.0){\rule[-0.200pt]{0.400pt}{0.723pt}}
\put(715,295.67){\rule{0.241pt}{0.400pt}}
\multiput(715.00,295.17)(0.500,1.000){2}{\rule{0.120pt}{0.400pt}}
\put(715.67,294){\rule{0.400pt}{0.723pt}}
\multiput(715.17,295.50)(1.000,-1.500){2}{\rule{0.400pt}{0.361pt}}
\put(714.0,296.0){\usebox{\plotpoint}}
\put(717.0,294.0){\usebox{\plotpoint}}
\put(718,293.67){\rule{0.241pt}{0.400pt}}
\multiput(718.00,294.17)(0.500,-1.000){2}{\rule{0.120pt}{0.400pt}}
\put(717.0,295.0){\usebox{\plotpoint}}
\put(719.0,294.0){\usebox{\plotpoint}}
\put(720,293.67){\rule{0.241pt}{0.400pt}}
\multiput(720.00,294.17)(0.500,-1.000){2}{\rule{0.120pt}{0.400pt}}
\put(721,293.67){\rule{0.241pt}{0.400pt}}
\multiput(721.00,293.17)(0.500,1.000){2}{\rule{0.120pt}{0.400pt}}
\put(719.0,295.0){\usebox{\plotpoint}}
\put(722,295){\usebox{\plotpoint}}
\put(722,293.67){\rule{0.241pt}{0.400pt}}
\multiput(722.00,294.17)(0.500,-1.000){2}{\rule{0.120pt}{0.400pt}}
\put(722.67,292){\rule{0.400pt}{0.482pt}}
\multiput(722.17,293.00)(1.000,-1.000){2}{\rule{0.400pt}{0.241pt}}
\put(724,292){\usebox{\plotpoint}}
\put(723.67,292){\rule{0.400pt}{0.482pt}}
\multiput(723.17,292.00)(1.000,1.000){2}{\rule{0.400pt}{0.241pt}}
\put(726,293.67){\rule{0.241pt}{0.400pt}}
\multiput(726.00,293.17)(0.500,1.000){2}{\rule{0.120pt}{0.400pt}}
\put(725.0,294.0){\usebox{\plotpoint}}
\put(727,292.67){\rule{0.241pt}{0.400pt}}
\multiput(727.00,292.17)(0.500,1.000){2}{\rule{0.120pt}{0.400pt}}
\put(727.0,293.0){\rule[-0.200pt]{0.400pt}{0.482pt}}
\put(728.0,294.0){\usebox{\plotpoint}}
\put(728.67,292){\rule{0.400pt}{0.482pt}}
\multiput(728.17,292.00)(1.000,1.000){2}{\rule{0.400pt}{0.241pt}}
\put(730,293.67){\rule{0.241pt}{0.400pt}}
\multiput(730.00,293.17)(0.500,1.000){2}{\rule{0.120pt}{0.400pt}}
\put(729.0,292.0){\rule[-0.200pt]{0.400pt}{0.482pt}}
\put(731.0,293.0){\rule[-0.200pt]{0.400pt}{0.482pt}}
\put(731.0,293.0){\rule[-0.200pt]{0.482pt}{0.400pt}}
\put(733.0,293.0){\usebox{\plotpoint}}
\put(733.0,294.0){\rule[-0.200pt]{0.482pt}{0.400pt}}
\put(734.67,293){\rule{0.400pt}{0.482pt}}
\multiput(734.17,294.00)(1.000,-1.000){2}{\rule{0.400pt}{0.241pt}}
\put(735.0,294.0){\usebox{\plotpoint}}
\put(735.67,293){\rule{0.400pt}{0.482pt}}
\multiput(735.17,294.00)(1.000,-1.000){2}{\rule{0.400pt}{0.241pt}}
\put(737,292.67){\rule{0.241pt}{0.400pt}}
\multiput(737.00,292.17)(0.500,1.000){2}{\rule{0.120pt}{0.400pt}}
\put(736.0,293.0){\rule[-0.200pt]{0.400pt}{0.482pt}}
\put(738,294){\usebox{\plotpoint}}
\put(737.67,294){\rule{0.400pt}{0.482pt}}
\multiput(737.17,294.00)(1.000,1.000){2}{\rule{0.400pt}{0.241pt}}
\put(738.67,294){\rule{0.400pt}{0.482pt}}
\multiput(738.17,295.00)(1.000,-1.000){2}{\rule{0.400pt}{0.241pt}}
\put(740,293.67){\rule{0.241pt}{0.400pt}}
\multiput(740.00,294.17)(0.500,-1.000){2}{\rule{0.120pt}{0.400pt}}
\put(741,293.67){\rule{0.241pt}{0.400pt}}
\multiput(741.00,293.17)(0.500,1.000){2}{\rule{0.120pt}{0.400pt}}
\put(740.0,294.0){\usebox{\plotpoint}}
\put(742.0,295.0){\usebox{\plotpoint}}
\put(743,294.67){\rule{0.241pt}{0.400pt}}
\multiput(743.00,295.17)(0.500,-1.000){2}{\rule{0.120pt}{0.400pt}}
\put(744,294.67){\rule{0.241pt}{0.400pt}}
\multiput(744.00,294.17)(0.500,1.000){2}{\rule{0.120pt}{0.400pt}}
\put(742.0,296.0){\usebox{\plotpoint}}
\put(745,296){\usebox{\plotpoint}}
\put(745.0,296.0){\rule[-0.200pt]{0.482pt}{0.400pt}}
\put(747,295.67){\rule{0.241pt}{0.400pt}}
\multiput(747.00,296.17)(0.500,-1.000){2}{\rule{0.120pt}{0.400pt}}
\put(747.67,296){\rule{0.400pt}{0.482pt}}
\multiput(747.17,296.00)(1.000,1.000){2}{\rule{0.400pt}{0.241pt}}
\put(747.0,296.0){\usebox{\plotpoint}}
\put(749,298){\usebox{\plotpoint}}
\put(748.67,296){\rule{0.400pt}{0.482pt}}
\multiput(748.17,297.00)(1.000,-1.000){2}{\rule{0.400pt}{0.241pt}}
\put(750,297.67){\rule{0.241pt}{0.400pt}}
\multiput(750.00,297.17)(0.500,1.000){2}{\rule{0.120pt}{0.400pt}}
\put(751,297.67){\rule{0.241pt}{0.400pt}}
\multiput(751.00,298.17)(0.500,-1.000){2}{\rule{0.120pt}{0.400pt}}
\put(750.0,296.0){\rule[-0.200pt]{0.400pt}{0.482pt}}
\put(751.67,296){\rule{0.400pt}{0.723pt}}
\multiput(751.17,296.00)(1.000,1.500){2}{\rule{0.400pt}{0.361pt}}
\put(752.0,296.0){\rule[-0.200pt]{0.400pt}{0.482pt}}
\put(753.67,299){\rule{0.400pt}{0.482pt}}
\multiput(753.17,299.00)(1.000,1.000){2}{\rule{0.400pt}{0.241pt}}
\put(754.67,298){\rule{0.400pt}{0.723pt}}
\multiput(754.17,299.50)(1.000,-1.500){2}{\rule{0.400pt}{0.361pt}}
\put(753.0,299.0){\usebox{\plotpoint}}
\put(756,300.67){\rule{0.241pt}{0.400pt}}
\multiput(756.00,301.17)(0.500,-1.000){2}{\rule{0.120pt}{0.400pt}}
\put(756.0,298.0){\rule[-0.200pt]{0.400pt}{0.964pt}}
\put(757,298.67){\rule{0.241pt}{0.400pt}}
\multiput(757.00,299.17)(0.500,-1.000){2}{\rule{0.120pt}{0.400pt}}
\put(758,298.67){\rule{0.241pt}{0.400pt}}
\multiput(758.00,298.17)(0.500,1.000){2}{\rule{0.120pt}{0.400pt}}
\put(757.0,300.0){\usebox{\plotpoint}}
\put(759.0,300.0){\usebox{\plotpoint}}
\put(760,300.67){\rule{0.241pt}{0.400pt}}
\multiput(760.00,300.17)(0.500,1.000){2}{\rule{0.120pt}{0.400pt}}
\put(759.0,301.0){\usebox{\plotpoint}}
\put(761,302){\usebox{\plotpoint}}
\put(761,300.67){\rule{0.241pt}{0.400pt}}
\multiput(761.00,301.17)(0.500,-1.000){2}{\rule{0.120pt}{0.400pt}}
\put(761.67,301){\rule{0.400pt}{0.482pt}}
\multiput(761.17,301.00)(1.000,1.000){2}{\rule{0.400pt}{0.241pt}}
\put(763,302.67){\rule{0.241pt}{0.400pt}}
\multiput(763.00,303.17)(0.500,-1.000){2}{\rule{0.120pt}{0.400pt}}
\put(763.0,303.0){\usebox{\plotpoint}}
\put(765,302.67){\rule{0.241pt}{0.400pt}}
\multiput(765.00,302.17)(0.500,1.000){2}{\rule{0.120pt}{0.400pt}}
\put(764.0,303.0){\usebox{\plotpoint}}
\put(766,304){\usebox{\plotpoint}}
\put(765.67,304){\rule{0.400pt}{0.482pt}}
\multiput(765.17,304.00)(1.000,1.000){2}{\rule{0.400pt}{0.241pt}}
\put(767.0,306.0){\usebox{\plotpoint}}
\put(768,304.67){\rule{0.241pt}{0.400pt}}
\multiput(768.00,304.17)(0.500,1.000){2}{\rule{0.120pt}{0.400pt}}
\put(769,305.67){\rule{0.241pt}{0.400pt}}
\multiput(769.00,305.17)(0.500,1.000){2}{\rule{0.120pt}{0.400pt}}
\put(768.0,305.0){\usebox{\plotpoint}}
\put(770,307){\usebox{\plotpoint}}
\put(771,305.67){\rule{0.241pt}{0.400pt}}
\multiput(771.00,306.17)(0.500,-1.000){2}{\rule{0.120pt}{0.400pt}}
\put(770.0,307.0){\usebox{\plotpoint}}
\put(772,307.67){\rule{0.241pt}{0.400pt}}
\multiput(772.00,307.17)(0.500,1.000){2}{\rule{0.120pt}{0.400pt}}
\put(772.0,306.0){\rule[-0.200pt]{0.400pt}{0.482pt}}
\put(773,307.67){\rule{0.241pt}{0.400pt}}
\multiput(773.00,307.17)(0.500,1.000){2}{\rule{0.120pt}{0.400pt}}
\put(773.0,308.0){\usebox{\plotpoint}}
\put(774.0,309.0){\usebox{\plotpoint}}
\put(775.0,309.0){\rule[-0.200pt]{0.400pt}{0.482pt}}
\put(776,309.67){\rule{0.241pt}{0.400pt}}
\multiput(776.00,310.17)(0.500,-1.000){2}{\rule{0.120pt}{0.400pt}}
\put(775.0,311.0){\usebox{\plotpoint}}
\put(777.0,310.0){\rule[-0.200pt]{0.400pt}{0.482pt}}
\put(778,311.67){\rule{0.241pt}{0.400pt}}
\multiput(778.00,311.17)(0.500,1.000){2}{\rule{0.120pt}{0.400pt}}
\put(777.0,312.0){\usebox{\plotpoint}}
\put(778.67,312){\rule{0.400pt}{0.723pt}}
\multiput(778.17,312.00)(1.000,1.500){2}{\rule{0.400pt}{0.361pt}}
\put(779.0,312.0){\usebox{\plotpoint}}
\put(780,314.67){\rule{0.241pt}{0.400pt}}
\multiput(780.00,315.17)(0.500,-1.000){2}{\rule{0.120pt}{0.400pt}}
\put(781,314.67){\rule{0.241pt}{0.400pt}}
\multiput(781.00,314.17)(0.500,1.000){2}{\rule{0.120pt}{0.400pt}}
\put(780.0,315.0){\usebox{\plotpoint}}
\put(782.0,316.0){\usebox{\plotpoint}}
\put(783,315.67){\rule{0.241pt}{0.400pt}}
\multiput(783.00,316.17)(0.500,-1.000){2}{\rule{0.120pt}{0.400pt}}
\put(782.0,317.0){\usebox{\plotpoint}}
\put(783.67,317){\rule{0.400pt}{0.482pt}}
\multiput(783.17,317.00)(1.000,1.000){2}{\rule{0.400pt}{0.241pt}}
\put(784.0,316.0){\usebox{\plotpoint}}
\put(785,319){\usebox{\plotpoint}}
\put(785.67,319){\rule{0.400pt}{0.482pt}}
\multiput(785.17,319.00)(1.000,1.000){2}{\rule{0.400pt}{0.241pt}}
\put(785.0,319.0){\usebox{\plotpoint}}
\put(786.67,320){\rule{0.400pt}{0.482pt}}
\multiput(786.17,321.00)(1.000,-1.000){2}{\rule{0.400pt}{0.241pt}}
\put(787.67,320){\rule{0.400pt}{0.964pt}}
\multiput(787.17,320.00)(1.000,2.000){2}{\rule{0.400pt}{0.482pt}}
\put(787.0,321.0){\usebox{\plotpoint}}
\put(789,324){\usebox{\plotpoint}}
\put(789.0,324.0){\rule[-0.200pt]{0.482pt}{0.400pt}}
\put(791,324.67){\rule{0.241pt}{0.400pt}}
\multiput(791.00,324.17)(0.500,1.000){2}{\rule{0.120pt}{0.400pt}}
\put(791.0,324.0){\usebox{\plotpoint}}
\put(791.67,325){\rule{0.400pt}{0.723pt}}
\multiput(791.17,325.00)(1.000,1.500){2}{\rule{0.400pt}{0.361pt}}
\put(792.0,325.0){\usebox{\plotpoint}}
\put(793.0,328.0){\usebox{\plotpoint}}
\put(794,329.67){\rule{0.241pt}{0.400pt}}
\multiput(794.00,329.17)(0.500,1.000){2}{\rule{0.120pt}{0.400pt}}
\put(794.67,331){\rule{0.400pt}{0.482pt}}
\multiput(794.17,331.00)(1.000,1.000){2}{\rule{0.400pt}{0.241pt}}
\put(794.0,328.0){\rule[-0.200pt]{0.400pt}{0.482pt}}
\put(795.67,332){\rule{0.400pt}{0.723pt}}
\multiput(795.17,332.00)(1.000,1.500){2}{\rule{0.400pt}{0.361pt}}
\put(796.0,332.0){\usebox{\plotpoint}}
\put(798,334.67){\rule{0.241pt}{0.400pt}}
\multiput(798.00,334.17)(0.500,1.000){2}{\rule{0.120pt}{0.400pt}}
\put(797.0,335.0){\usebox{\plotpoint}}
\put(798.67,335){\rule{0.400pt}{0.482pt}}
\multiput(798.17,335.00)(1.000,1.000){2}{\rule{0.400pt}{0.241pt}}
\put(799.67,337){\rule{0.400pt}{0.723pt}}
\multiput(799.17,337.00)(1.000,1.500){2}{\rule{0.400pt}{0.361pt}}
\put(799.0,335.0){\usebox{\plotpoint}}
\put(801,340){\usebox{\plotpoint}}
\put(801,339.67){\rule{0.241pt}{0.400pt}}
\multiput(801.00,339.17)(0.500,1.000){2}{\rule{0.120pt}{0.400pt}}
\put(802,340.67){\rule{0.241pt}{0.400pt}}
\multiput(802.00,340.17)(0.500,1.000){2}{\rule{0.120pt}{0.400pt}}
\put(803,342.67){\rule{0.241pt}{0.400pt}}
\multiput(803.00,343.17)(0.500,-1.000){2}{\rule{0.120pt}{0.400pt}}
\put(803.67,343){\rule{0.400pt}{0.964pt}}
\multiput(803.17,343.00)(1.000,2.000){2}{\rule{0.400pt}{0.482pt}}
\put(803.0,342.0){\rule[-0.200pt]{0.400pt}{0.482pt}}
\put(805,347){\usebox{\plotpoint}}
\put(805.0,347.0){\usebox{\plotpoint}}
\put(806,349.67){\rule{0.241pt}{0.400pt}}
\multiput(806.00,349.17)(0.500,1.000){2}{\rule{0.120pt}{0.400pt}}
\put(807,349.67){\rule{0.241pt}{0.400pt}}
\multiput(807.00,350.17)(0.500,-1.000){2}{\rule{0.120pt}{0.400pt}}
\put(806.0,347.0){\rule[-0.200pt]{0.400pt}{0.723pt}}
\put(808,352.67){\rule{0.241pt}{0.400pt}}
\multiput(808.00,353.17)(0.500,-1.000){2}{\rule{0.120pt}{0.400pt}}
\put(808.67,353){\rule{0.400pt}{0.482pt}}
\multiput(808.17,353.00)(1.000,1.000){2}{\rule{0.400pt}{0.241pt}}
\put(808.0,350.0){\rule[-0.200pt]{0.400pt}{0.964pt}}
\put(810,356.67){\rule{0.241pt}{0.400pt}}
\multiput(810.00,356.17)(0.500,1.000){2}{\rule{0.120pt}{0.400pt}}
\put(810.0,355.0){\rule[-0.200pt]{0.400pt}{0.482pt}}
\put(811.0,358.0){\usebox{\plotpoint}}
\put(812,360.67){\rule{0.241pt}{0.400pt}}
\multiput(812.00,360.17)(0.500,1.000){2}{\rule{0.120pt}{0.400pt}}
\put(812.0,358.0){\rule[-0.200pt]{0.400pt}{0.723pt}}
\put(812.67,363){\rule{0.400pt}{0.482pt}}
\multiput(812.17,363.00)(1.000,1.000){2}{\rule{0.400pt}{0.241pt}}
\put(814,364.67){\rule{0.241pt}{0.400pt}}
\multiput(814.00,364.17)(0.500,1.000){2}{\rule{0.120pt}{0.400pt}}
\put(813.0,362.0){\usebox{\plotpoint}}
\put(814.67,368){\rule{0.400pt}{0.723pt}}
\multiput(814.17,368.00)(1.000,1.500){2}{\rule{0.400pt}{0.361pt}}
\put(815.0,366.0){\rule[-0.200pt]{0.400pt}{0.482pt}}
\put(816.0,371.0){\usebox{\plotpoint}}
\put(817,371.67){\rule{0.241pt}{0.400pt}}
\multiput(817.00,371.17)(0.500,1.000){2}{\rule{0.120pt}{0.400pt}}
\put(817.0,371.0){\usebox{\plotpoint}}
\put(818,375.67){\rule{0.241pt}{0.400pt}}
\multiput(818.00,375.17)(0.500,1.000){2}{\rule{0.120pt}{0.400pt}}
\put(819,376.67){\rule{0.241pt}{0.400pt}}
\multiput(819.00,376.17)(0.500,1.000){2}{\rule{0.120pt}{0.400pt}}
\put(818.0,373.0){\rule[-0.200pt]{0.400pt}{0.723pt}}
\put(820,378){\usebox{\plotpoint}}
\put(819.67,378){\rule{0.400pt}{0.723pt}}
\multiput(819.17,378.00)(1.000,1.500){2}{\rule{0.400pt}{0.361pt}}
\put(820.67,381){\rule{0.400pt}{0.723pt}}
\multiput(820.17,381.00)(1.000,1.500){2}{\rule{0.400pt}{0.361pt}}
\put(822,385.67){\rule{0.241pt}{0.400pt}}
\multiput(822.00,385.17)(0.500,1.000){2}{\rule{0.120pt}{0.400pt}}
\put(822.67,387){\rule{0.400pt}{0.964pt}}
\multiput(822.17,387.00)(1.000,2.000){2}{\rule{0.400pt}{0.482pt}}
\put(822.0,384.0){\rule[-0.200pt]{0.400pt}{0.482pt}}
\put(824.0,391.0){\usebox{\plotpoint}}
\put(824.0,392.0){\usebox{\plotpoint}}
\put(824.67,394){\rule{0.400pt}{0.723pt}}
\multiput(824.17,394.00)(1.000,1.500){2}{\rule{0.400pt}{0.361pt}}
\put(826,396.67){\rule{0.241pt}{0.400pt}}
\multiput(826.00,396.17)(0.500,1.000){2}{\rule{0.120pt}{0.400pt}}
\put(825.0,392.0){\rule[-0.200pt]{0.400pt}{0.482pt}}
\put(826.67,400){\rule{0.400pt}{0.723pt}}
\multiput(826.17,400.00)(1.000,1.500){2}{\rule{0.400pt}{0.361pt}}
\put(828,402.67){\rule{0.241pt}{0.400pt}}
\multiput(828.00,402.17)(0.500,1.000){2}{\rule{0.120pt}{0.400pt}}
\put(827.0,398.0){\rule[-0.200pt]{0.400pt}{0.482pt}}
\put(829,406.67){\rule{0.241pt}{0.400pt}}
\multiput(829.00,407.17)(0.500,-1.000){2}{\rule{0.120pt}{0.400pt}}
\put(829.0,404.0){\rule[-0.200pt]{0.400pt}{0.964pt}}
\put(829.67,411){\rule{0.400pt}{0.482pt}}
\multiput(829.17,411.00)(1.000,1.000){2}{\rule{0.400pt}{0.241pt}}
\put(831,412.67){\rule{0.241pt}{0.400pt}}
\multiput(831.00,412.17)(0.500,1.000){2}{\rule{0.120pt}{0.400pt}}
\put(830.0,407.0){\rule[-0.200pt]{0.400pt}{0.964pt}}
\put(832.0,414.0){\rule[-0.200pt]{0.400pt}{0.482pt}}
\put(832.67,416){\rule{0.400pt}{0.964pt}}
\multiput(832.17,416.00)(1.000,2.000){2}{\rule{0.400pt}{0.482pt}}
\put(832.0,416.0){\usebox{\plotpoint}}
\put(834,422.67){\rule{0.241pt}{0.400pt}}
\multiput(834.00,422.17)(0.500,1.000){2}{\rule{0.120pt}{0.400pt}}
\put(834.67,424){\rule{0.400pt}{0.723pt}}
\multiput(834.17,424.00)(1.000,1.500){2}{\rule{0.400pt}{0.361pt}}
\put(834.0,420.0){\rule[-0.200pt]{0.400pt}{0.723pt}}
\put(835.67,428){\rule{0.400pt}{0.964pt}}
\multiput(835.17,428.00)(1.000,2.000){2}{\rule{0.400pt}{0.482pt}}
\put(836.0,427.0){\usebox{\plotpoint}}
\put(836.67,433){\rule{0.400pt}{0.723pt}}
\multiput(836.17,433.00)(1.000,1.500){2}{\rule{0.400pt}{0.361pt}}
\put(838,435.67){\rule{0.241pt}{0.400pt}}
\multiput(838.00,435.17)(0.500,1.000){2}{\rule{0.120pt}{0.400pt}}
\put(837.0,432.0){\usebox{\plotpoint}}
\put(838.67,439){\rule{0.400pt}{0.723pt}}
\multiput(838.17,439.00)(1.000,1.500){2}{\rule{0.400pt}{0.361pt}}
\put(839.67,442){\rule{0.400pt}{0.482pt}}
\multiput(839.17,442.00)(1.000,1.000){2}{\rule{0.400pt}{0.241pt}}
\put(839.0,437.0){\rule[-0.200pt]{0.400pt}{0.482pt}}
\put(840.67,446){\rule{0.400pt}{0.482pt}}
\multiput(840.17,446.00)(1.000,1.000){2}{\rule{0.400pt}{0.241pt}}
\put(841.0,444.0){\rule[-0.200pt]{0.400pt}{0.482pt}}
\put(841.67,453){\rule{0.400pt}{0.482pt}}
\multiput(841.17,453.00)(1.000,1.000){2}{\rule{0.400pt}{0.241pt}}
\put(843,454.67){\rule{0.241pt}{0.400pt}}
\multiput(843.00,454.17)(0.500,1.000){2}{\rule{0.120pt}{0.400pt}}
\put(842.0,448.0){\rule[-0.200pt]{0.400pt}{1.204pt}}
\put(844,459.67){\rule{0.241pt}{0.400pt}}
\multiput(844.00,459.17)(0.500,1.000){2}{\rule{0.120pt}{0.400pt}}
\put(845,460.67){\rule{0.241pt}{0.400pt}}
\multiput(845.00,460.17)(0.500,1.000){2}{\rule{0.120pt}{0.400pt}}
\put(844.0,456.0){\rule[-0.200pt]{0.400pt}{0.964pt}}
\put(845.67,466){\rule{0.400pt}{0.482pt}}
\multiput(845.17,466.00)(1.000,1.000){2}{\rule{0.400pt}{0.241pt}}
\put(846.67,468){\rule{0.400pt}{0.482pt}}
\multiput(846.17,468.00)(1.000,1.000){2}{\rule{0.400pt}{0.241pt}}
\put(846.0,462.0){\rule[-0.200pt]{0.400pt}{0.964pt}}
\put(847.67,475){\rule{0.400pt}{0.482pt}}
\multiput(847.17,475.00)(1.000,1.000){2}{\rule{0.400pt}{0.241pt}}
\put(848.0,470.0){\rule[-0.200pt]{0.400pt}{1.204pt}}
\put(849,479.67){\rule{0.241pt}{0.400pt}}
\multiput(849.00,479.17)(0.500,1.000){2}{\rule{0.120pt}{0.400pt}}
\put(849.67,481){\rule{0.400pt}{0.723pt}}
\multiput(849.17,481.00)(1.000,1.500){2}{\rule{0.400pt}{0.361pt}}
\put(849.0,477.0){\rule[-0.200pt]{0.400pt}{0.723pt}}
\put(850.67,486){\rule{0.400pt}{0.482pt}}
\multiput(850.17,487.00)(1.000,-1.000){2}{\rule{0.400pt}{0.241pt}}
\put(852,485.67){\rule{0.241pt}{0.400pt}}
\multiput(852.00,485.17)(0.500,1.000){2}{\rule{0.120pt}{0.400pt}}
\put(851.0,484.0){\rule[-0.200pt]{0.400pt}{0.964pt}}
\put(852.67,491){\rule{0.400pt}{1.204pt}}
\multiput(852.17,491.00)(1.000,2.500){2}{\rule{0.400pt}{0.602pt}}
\put(853.0,487.0){\rule[-0.200pt]{0.400pt}{0.964pt}}
\put(853.67,497){\rule{0.400pt}{0.964pt}}
\multiput(853.17,497.00)(1.000,2.000){2}{\rule{0.400pt}{0.482pt}}
\put(855,500.67){\rule{0.241pt}{0.400pt}}
\multiput(855.00,500.17)(0.500,1.000){2}{\rule{0.120pt}{0.400pt}}
\put(854.0,496.0){\usebox{\plotpoint}}
\put(855.67,504){\rule{0.400pt}{1.204pt}}
\multiput(855.17,504.00)(1.000,2.500){2}{\rule{0.400pt}{0.602pt}}
\put(857,507.67){\rule{0.241pt}{0.400pt}}
\multiput(857.00,508.17)(0.500,-1.000){2}{\rule{0.120pt}{0.400pt}}
\put(856.0,502.0){\rule[-0.200pt]{0.400pt}{0.482pt}}
\put(858,514.67){\rule{0.241pt}{0.400pt}}
\multiput(858.00,514.17)(0.500,1.000){2}{\rule{0.120pt}{0.400pt}}
\put(858.0,508.0){\rule[-0.200pt]{0.400pt}{1.686pt}}
\put(859.0,516.0){\usebox{\plotpoint}}
\put(860,519.67){\rule{0.241pt}{0.400pt}}
\multiput(860.00,520.17)(0.500,-1.000){2}{\rule{0.120pt}{0.400pt}}
\put(860.0,516.0){\rule[-0.200pt]{0.400pt}{1.204pt}}
\put(861,525.67){\rule{0.241pt}{0.400pt}}
\multiput(861.00,525.17)(0.500,1.000){2}{\rule{0.120pt}{0.400pt}}
\put(861.67,527){\rule{0.400pt}{0.723pt}}
\multiput(861.17,527.00)(1.000,1.500){2}{\rule{0.400pt}{0.361pt}}
\put(861.0,520.0){\rule[-0.200pt]{0.400pt}{1.445pt}}
\put(863,534.67){\rule{0.241pt}{0.400pt}}
\multiput(863.00,535.17)(0.500,-1.000){2}{\rule{0.120pt}{0.400pt}}
\put(863.67,535){\rule{0.400pt}{0.723pt}}
\multiput(863.17,535.00)(1.000,1.500){2}{\rule{0.400pt}{0.361pt}}
\put(863.0,530.0){\rule[-0.200pt]{0.400pt}{1.445pt}}
\put(865.0,538.0){\rule[-0.200pt]{0.400pt}{0.964pt}}
\put(865.0,542.0){\usebox{\plotpoint}}
\put(865.67,545){\rule{0.400pt}{0.482pt}}
\multiput(865.17,545.00)(1.000,1.000){2}{\rule{0.400pt}{0.241pt}}
\put(866.67,547){\rule{0.400pt}{0.964pt}}
\multiput(866.17,547.00)(1.000,2.000){2}{\rule{0.400pt}{0.482pt}}
\put(866.0,542.0){\rule[-0.200pt]{0.400pt}{0.723pt}}
\put(867.67,552){\rule{0.400pt}{0.482pt}}
\multiput(867.17,552.00)(1.000,1.000){2}{\rule{0.400pt}{0.241pt}}
\put(868.67,554){\rule{0.400pt}{0.723pt}}
\multiput(868.17,554.00)(1.000,1.500){2}{\rule{0.400pt}{0.361pt}}
\put(868.0,551.0){\usebox{\plotpoint}}
\put(870.0,557.0){\rule[-0.200pt]{0.400pt}{0.482pt}}
\put(870.0,559.0){\usebox{\plotpoint}}
\put(870.67,562){\rule{0.400pt}{0.723pt}}
\multiput(870.17,562.00)(1.000,1.500){2}{\rule{0.400pt}{0.361pt}}
\put(871.67,565){\rule{0.400pt}{0.482pt}}
\multiput(871.17,565.00)(1.000,1.000){2}{\rule{0.400pt}{0.241pt}}
\put(871.0,559.0){\rule[-0.200pt]{0.400pt}{0.723pt}}
\put(873.0,567.0){\rule[-0.200pt]{0.400pt}{0.723pt}}
\put(873.0,570.0){\rule[-0.200pt]{0.482pt}{0.400pt}}
\put(874.67,576){\rule{0.400pt}{0.723pt}}
\multiput(874.17,576.00)(1.000,1.500){2}{\rule{0.400pt}{0.361pt}}
\put(875.67,576){\rule{0.400pt}{0.723pt}}
\multiput(875.17,577.50)(1.000,-1.500){2}{\rule{0.400pt}{0.361pt}}
\put(875.0,570.0){\rule[-0.200pt]{0.400pt}{1.445pt}}
\put(876.67,582){\rule{0.400pt}{0.723pt}}
\multiput(876.17,582.00)(1.000,1.500){2}{\rule{0.400pt}{0.361pt}}
\put(877.0,576.0){\rule[-0.200pt]{0.400pt}{1.445pt}}
\put(878,584.67){\rule{0.241pt}{0.400pt}}
\multiput(878.00,585.17)(0.500,-1.000){2}{\rule{0.120pt}{0.400pt}}
\put(878.67,585){\rule{0.400pt}{1.204pt}}
\multiput(878.17,585.00)(1.000,2.500){2}{\rule{0.400pt}{0.602pt}}
\put(878.0,585.0){\usebox{\plotpoint}}
\put(879.67,589){\rule{0.400pt}{1.686pt}}
\multiput(879.17,589.00)(1.000,3.500){2}{\rule{0.400pt}{0.843pt}}
\put(881,595.67){\rule{0.241pt}{0.400pt}}
\multiput(881.00,595.17)(0.500,1.000){2}{\rule{0.120pt}{0.400pt}}
\put(880.0,589.0){\usebox{\plotpoint}}
\put(882.0,596.0){\usebox{\plotpoint}}
\put(882.0,596.0){\usebox{\plotpoint}}
\put(882.67,598){\rule{0.400pt}{0.482pt}}
\multiput(882.17,598.00)(1.000,1.000){2}{\rule{0.400pt}{0.241pt}}
\put(883.0,596.0){\rule[-0.200pt]{0.400pt}{0.482pt}}
\put(884.0,600.0){\usebox{\plotpoint}}
\put(884.67,602){\rule{0.400pt}{1.445pt}}
\multiput(884.17,605.00)(1.000,-3.000){2}{\rule{0.400pt}{0.723pt}}
\put(885.67,602){\rule{0.400pt}{1.445pt}}
\multiput(885.17,602.00)(1.000,3.000){2}{\rule{0.400pt}{0.723pt}}
\put(885.0,600.0){\rule[-0.200pt]{0.400pt}{1.927pt}}
\put(886.67,607){\rule{0.400pt}{0.964pt}}
\multiput(886.17,609.00)(1.000,-2.000){2}{\rule{0.400pt}{0.482pt}}
\put(887.0,608.0){\rule[-0.200pt]{0.400pt}{0.723pt}}
\put(888,607){\usebox{\plotpoint}}
\put(887.67,607){\rule{0.400pt}{1.204pt}}
\multiput(887.17,607.00)(1.000,2.500){2}{\rule{0.400pt}{0.602pt}}
\put(889,611.67){\rule{0.241pt}{0.400pt}}
\multiput(889.00,611.17)(0.500,1.000){2}{\rule{0.120pt}{0.400pt}}
\put(890.0,613.0){\rule[-0.200pt]{0.400pt}{0.482pt}}
\put(891,613.67){\rule{0.241pt}{0.400pt}}
\multiput(891.00,614.17)(0.500,-1.000){2}{\rule{0.120pt}{0.400pt}}
\put(890.0,615.0){\usebox{\plotpoint}}
\put(891.67,616){\rule{0.400pt}{0.723pt}}
\multiput(891.17,616.00)(1.000,1.500){2}{\rule{0.400pt}{0.361pt}}
\put(892.0,614.0){\rule[-0.200pt]{0.400pt}{0.482pt}}
\put(892.67,615){\rule{0.400pt}{1.927pt}}
\multiput(892.17,615.00)(1.000,4.000){2}{\rule{0.400pt}{0.964pt}}
\put(893.67,613){\rule{0.400pt}{2.409pt}}
\multiput(893.17,618.00)(1.000,-5.000){2}{\rule{0.400pt}{1.204pt}}
\put(893.0,615.0){\rule[-0.200pt]{0.400pt}{0.964pt}}
\put(894.67,617){\rule{0.400pt}{1.204pt}}
\multiput(894.17,617.00)(1.000,2.500){2}{\rule{0.400pt}{0.602pt}}
\put(895.67,619){\rule{0.400pt}{0.723pt}}
\multiput(895.17,620.50)(1.000,-1.500){2}{\rule{0.400pt}{0.361pt}}
\put(895.0,613.0){\rule[-0.200pt]{0.400pt}{0.964pt}}
\put(897,620.67){\rule{0.241pt}{0.400pt}}
\multiput(897.00,620.17)(0.500,1.000){2}{\rule{0.120pt}{0.400pt}}
\put(897.67,620){\rule{0.400pt}{0.482pt}}
\multiput(897.17,621.00)(1.000,-1.000){2}{\rule{0.400pt}{0.241pt}}
\put(897.0,619.0){\rule[-0.200pt]{0.400pt}{0.482pt}}
\put(899.0,619.0){\usebox{\plotpoint}}
\put(899.0,619.0){\usebox{\plotpoint}}
\put(900,619.67){\rule{0.241pt}{0.400pt}}
\multiput(900.00,620.17)(0.500,-1.000){2}{\rule{0.120pt}{0.400pt}}
\put(900.67,614){\rule{0.400pt}{1.445pt}}
\multiput(900.17,617.00)(1.000,-3.000){2}{\rule{0.400pt}{0.723pt}}
\put(900.0,619.0){\rule[-0.200pt]{0.400pt}{0.482pt}}
\put(902,619.67){\rule{0.241pt}{0.400pt}}
\multiput(902.00,620.17)(0.500,-1.000){2}{\rule{0.120pt}{0.400pt}}
\put(902.67,615){\rule{0.400pt}{1.204pt}}
\multiput(902.17,617.50)(1.000,-2.500){2}{\rule{0.400pt}{0.602pt}}
\put(902.0,614.0){\rule[-0.200pt]{0.400pt}{1.686pt}}
\put(904,617.67){\rule{0.241pt}{0.400pt}}
\multiput(904.00,617.17)(0.500,1.000){2}{\rule{0.120pt}{0.400pt}}
\put(904.0,615.0){\rule[-0.200pt]{0.400pt}{0.723pt}}
\put(904.67,614){\rule{0.400pt}{1.445pt}}
\multiput(904.17,617.00)(1.000,-3.000){2}{\rule{0.400pt}{0.723pt}}
\put(905.0,619.0){\usebox{\plotpoint}}
\put(906.0,614.0){\usebox{\plotpoint}}
\put(906.67,613){\rule{0.400pt}{0.723pt}}
\multiput(906.17,613.00)(1.000,1.500){2}{\rule{0.400pt}{0.361pt}}
\put(908,615.67){\rule{0.241pt}{0.400pt}}
\multiput(908.00,615.17)(0.500,1.000){2}{\rule{0.120pt}{0.400pt}}
\put(907.0,613.0){\usebox{\plotpoint}}
\put(909,611.67){\rule{0.241pt}{0.400pt}}
\multiput(909.00,611.17)(0.500,1.000){2}{\rule{0.120pt}{0.400pt}}
\put(909.0,612.0){\rule[-0.200pt]{0.400pt}{1.204pt}}
\put(909.67,611){\rule{0.400pt}{0.723pt}}
\multiput(909.17,612.50)(1.000,-1.500){2}{\rule{0.400pt}{0.361pt}}
\put(910.67,606){\rule{0.400pt}{1.204pt}}
\multiput(910.17,608.50)(1.000,-2.500){2}{\rule{0.400pt}{0.602pt}}
\put(910.0,613.0){\usebox{\plotpoint}}
\put(911.67,603){\rule{0.400pt}{0.964pt}}
\multiput(911.17,603.00)(1.000,2.000){2}{\rule{0.400pt}{0.482pt}}
\put(912.67,603){\rule{0.400pt}{0.964pt}}
\multiput(912.17,605.00)(1.000,-2.000){2}{\rule{0.400pt}{0.482pt}}
\put(912.0,603.0){\rule[-0.200pt]{0.400pt}{0.723pt}}
\put(914,603){\usebox{\plotpoint}}
\put(913.67,598){\rule{0.400pt}{1.204pt}}
\multiput(913.17,600.50)(1.000,-2.500){2}{\rule{0.400pt}{0.602pt}}
\put(915,598){\usebox{\plotpoint}}
\put(916,596.67){\rule{0.241pt}{0.400pt}}
\multiput(916.00,597.17)(0.500,-1.000){2}{\rule{0.120pt}{0.400pt}}
\put(915.0,598.0){\usebox{\plotpoint}}
\put(916.67,594){\rule{0.400pt}{0.482pt}}
\multiput(916.17,595.00)(1.000,-1.000){2}{\rule{0.400pt}{0.241pt}}
\put(917.67,587){\rule{0.400pt}{1.686pt}}
\multiput(917.17,590.50)(1.000,-3.500){2}{\rule{0.400pt}{0.843pt}}
\put(917.0,596.0){\usebox{\plotpoint}}
\put(918.67,587){\rule{0.400pt}{0.723pt}}
\multiput(918.17,588.50)(1.000,-1.500){2}{\rule{0.400pt}{0.361pt}}
\put(919.0,587.0){\rule[-0.200pt]{0.400pt}{0.723pt}}
\put(919.67,580){\rule{0.400pt}{0.723pt}}
\multiput(919.17,581.50)(1.000,-1.500){2}{\rule{0.400pt}{0.361pt}}
\put(920.0,583.0){\rule[-0.200pt]{0.400pt}{0.964pt}}
\put(921.0,580.0){\usebox{\plotpoint}}
\put(921.67,573){\rule{0.400pt}{2.168pt}}
\multiput(921.17,577.50)(1.000,-4.500){2}{\rule{0.400pt}{1.084pt}}
\put(922.0,580.0){\rule[-0.200pt]{0.400pt}{0.482pt}}
\put(923.0,573.0){\usebox{\plotpoint}}
\put(923.67,566){\rule{0.400pt}{1.204pt}}
\multiput(923.17,568.50)(1.000,-2.500){2}{\rule{0.400pt}{0.602pt}}
\put(924.0,571.0){\rule[-0.200pt]{0.400pt}{0.482pt}}
\put(924.67,560){\rule{0.400pt}{2.409pt}}
\multiput(924.17,565.00)(1.000,-5.000){2}{\rule{0.400pt}{1.204pt}}
\put(925.0,566.0){\rule[-0.200pt]{0.400pt}{0.964pt}}
\put(926.0,560.0){\usebox{\plotpoint}}
\put(926.67,555){\rule{0.400pt}{0.482pt}}
\multiput(926.17,556.00)(1.000,-1.000){2}{\rule{0.400pt}{0.241pt}}
\put(927.67,552){\rule{0.400pt}{0.723pt}}
\multiput(927.17,553.50)(1.000,-1.500){2}{\rule{0.400pt}{0.361pt}}
\put(927.0,557.0){\rule[-0.200pt]{0.400pt}{0.723pt}}
\put(929,545.67){\rule{0.241pt}{0.400pt}}
\multiput(929.00,545.17)(0.500,1.000){2}{\rule{0.120pt}{0.400pt}}
\put(929.0,546.0){\rule[-0.200pt]{0.400pt}{1.445pt}}
\put(929.67,541){\rule{0.400pt}{0.723pt}}
\multiput(929.17,542.50)(1.000,-1.500){2}{\rule{0.400pt}{0.361pt}}
\put(930.67,539){\rule{0.400pt}{0.482pt}}
\multiput(930.17,540.00)(1.000,-1.000){2}{\rule{0.400pt}{0.241pt}}
\put(930.0,544.0){\rule[-0.200pt]{0.400pt}{0.723pt}}
\put(931.67,530){\rule{0.400pt}{0.482pt}}
\multiput(931.17,531.00)(1.000,-1.000){2}{\rule{0.400pt}{0.241pt}}
\put(932.67,528){\rule{0.400pt}{0.482pt}}
\multiput(932.17,529.00)(1.000,-1.000){2}{\rule{0.400pt}{0.241pt}}
\put(932.0,532.0){\rule[-0.200pt]{0.400pt}{1.686pt}}
\put(933.67,522){\rule{0.400pt}{0.482pt}}
\multiput(933.17,522.00)(1.000,1.000){2}{\rule{0.400pt}{0.241pt}}
\put(934.0,522.0){\rule[-0.200pt]{0.400pt}{1.445pt}}
\put(934.67,514){\rule{0.400pt}{0.723pt}}
\multiput(934.17,515.50)(1.000,-1.500){2}{\rule{0.400pt}{0.361pt}}
\put(935.67,511){\rule{0.400pt}{0.723pt}}
\multiput(935.17,512.50)(1.000,-1.500){2}{\rule{0.400pt}{0.361pt}}
\put(935.0,517.0){\rule[-0.200pt]{0.400pt}{1.686pt}}
\put(937,506.67){\rule{0.241pt}{0.400pt}}
\multiput(937.00,507.17)(0.500,-1.000){2}{\rule{0.120pt}{0.400pt}}
\put(937.67,501){\rule{0.400pt}{1.445pt}}
\multiput(937.17,504.00)(1.000,-3.000){2}{\rule{0.400pt}{0.723pt}}
\put(937.0,508.0){\rule[-0.200pt]{0.400pt}{0.723pt}}
\put(938.67,494){\rule{0.400pt}{1.204pt}}
\multiput(938.17,496.50)(1.000,-2.500){2}{\rule{0.400pt}{0.602pt}}
\put(939.0,499.0){\rule[-0.200pt]{0.400pt}{0.482pt}}
\put(939.67,488){\rule{0.400pt}{1.204pt}}
\multiput(939.17,490.50)(1.000,-2.500){2}{\rule{0.400pt}{0.602pt}}
\put(940.67,486){\rule{0.400pt}{0.482pt}}
\multiput(940.17,487.00)(1.000,-1.000){2}{\rule{0.400pt}{0.241pt}}
\put(940.0,493.0){\usebox{\plotpoint}}
\put(941.67,477){\rule{0.400pt}{0.723pt}}
\multiput(941.17,478.50)(1.000,-1.500){2}{\rule{0.400pt}{0.361pt}}
\put(942.67,473){\rule{0.400pt}{0.964pt}}
\multiput(942.17,475.00)(1.000,-2.000){2}{\rule{0.400pt}{0.482pt}}
\put(942.0,480.0){\rule[-0.200pt]{0.400pt}{1.445pt}}
\put(943.67,467){\rule{0.400pt}{0.482pt}}
\multiput(943.17,468.00)(1.000,-1.000){2}{\rule{0.400pt}{0.241pt}}
\put(944.0,469.0){\rule[-0.200pt]{0.400pt}{0.964pt}}
\put(944.67,461){\rule{0.400pt}{0.723pt}}
\multiput(944.17,462.50)(1.000,-1.500){2}{\rule{0.400pt}{0.361pt}}
\put(945.67,457){\rule{0.400pt}{0.964pt}}
\multiput(945.17,459.00)(1.000,-2.000){2}{\rule{0.400pt}{0.482pt}}
\put(945.0,464.0){\rule[-0.200pt]{0.400pt}{0.723pt}}
\put(946.67,451){\rule{0.400pt}{0.482pt}}
\multiput(946.17,452.00)(1.000,-1.000){2}{\rule{0.400pt}{0.241pt}}
\put(947.67,446){\rule{0.400pt}{1.204pt}}
\multiput(947.17,448.50)(1.000,-2.500){2}{\rule{0.400pt}{0.602pt}}
\put(947.0,453.0){\rule[-0.200pt]{0.400pt}{0.964pt}}
\put(948.67,440){\rule{0.400pt}{0.482pt}}
\multiput(948.17,441.00)(1.000,-1.000){2}{\rule{0.400pt}{0.241pt}}
\put(949.0,442.0){\rule[-0.200pt]{0.400pt}{0.964pt}}
\put(949.67,434){\rule{0.400pt}{0.723pt}}
\multiput(949.17,435.50)(1.000,-1.500){2}{\rule{0.400pt}{0.361pt}}
\put(950.67,431){\rule{0.400pt}{0.723pt}}
\multiput(950.17,432.50)(1.000,-1.500){2}{\rule{0.400pt}{0.361pt}}
\put(950.0,437.0){\rule[-0.200pt]{0.400pt}{0.723pt}}
\put(951.67,423){\rule{0.400pt}{0.723pt}}
\multiput(951.17,424.50)(1.000,-1.500){2}{\rule{0.400pt}{0.361pt}}
\put(952.67,418){\rule{0.400pt}{1.204pt}}
\multiput(952.17,420.50)(1.000,-2.500){2}{\rule{0.400pt}{0.602pt}}
\put(952.0,426.0){\rule[-0.200pt]{0.400pt}{1.204pt}}
\put(953.67,411){\rule{0.400pt}{1.204pt}}
\multiput(953.17,413.50)(1.000,-2.500){2}{\rule{0.400pt}{0.602pt}}
\put(954.0,416.0){\rule[-0.200pt]{0.400pt}{0.482pt}}
\put(954.67,403){\rule{0.400pt}{1.204pt}}
\multiput(954.17,405.50)(1.000,-2.500){2}{\rule{0.400pt}{0.602pt}}
\put(955.67,401){\rule{0.400pt}{0.482pt}}
\multiput(955.17,402.00)(1.000,-1.000){2}{\rule{0.400pt}{0.241pt}}
\put(955.0,408.0){\rule[-0.200pt]{0.400pt}{0.723pt}}
\put(956.67,395){\rule{0.400pt}{0.723pt}}
\multiput(956.17,396.50)(1.000,-1.500){2}{\rule{0.400pt}{0.361pt}}
\put(957.67,390){\rule{0.400pt}{1.204pt}}
\multiput(957.17,392.50)(1.000,-2.500){2}{\rule{0.400pt}{0.602pt}}
\put(957.0,398.0){\rule[-0.200pt]{0.400pt}{0.723pt}}
\put(958.67,384){\rule{0.400pt}{0.723pt}}
\multiput(958.17,385.50)(1.000,-1.500){2}{\rule{0.400pt}{0.361pt}}
\put(959.0,387.0){\rule[-0.200pt]{0.400pt}{0.723pt}}
\put(959.67,377){\rule{0.400pt}{0.964pt}}
\multiput(959.17,379.00)(1.000,-2.000){2}{\rule{0.400pt}{0.482pt}}
\put(960.67,374){\rule{0.400pt}{0.723pt}}
\multiput(960.17,375.50)(1.000,-1.500){2}{\rule{0.400pt}{0.361pt}}
\put(960.0,381.0){\rule[-0.200pt]{0.400pt}{0.723pt}}
\put(961.67,367){\rule{0.400pt}{0.723pt}}
\multiput(961.17,368.50)(1.000,-1.500){2}{\rule{0.400pt}{0.361pt}}
\put(962.67,364){\rule{0.400pt}{0.723pt}}
\multiput(962.17,365.50)(1.000,-1.500){2}{\rule{0.400pt}{0.361pt}}
\put(962.0,370.0){\rule[-0.200pt]{0.400pt}{0.964pt}}
\put(963.67,358){\rule{0.400pt}{0.723pt}}
\multiput(963.17,359.50)(1.000,-1.500){2}{\rule{0.400pt}{0.361pt}}
\put(964.0,361.0){\rule[-0.200pt]{0.400pt}{0.723pt}}
\put(964.67,350){\rule{0.400pt}{0.723pt}}
\multiput(964.17,351.50)(1.000,-1.500){2}{\rule{0.400pt}{0.361pt}}
\put(965.67,348){\rule{0.400pt}{0.482pt}}
\multiput(965.17,349.00)(1.000,-1.000){2}{\rule{0.400pt}{0.241pt}}
\put(965.0,353.0){\rule[-0.200pt]{0.400pt}{1.204pt}}
\put(966.67,342){\rule{0.400pt}{0.482pt}}
\multiput(966.17,343.00)(1.000,-1.000){2}{\rule{0.400pt}{0.241pt}}
\put(967.0,344.0){\rule[-0.200pt]{0.400pt}{0.964pt}}
\put(967.67,334){\rule{0.400pt}{0.964pt}}
\multiput(967.17,336.00)(1.000,-2.000){2}{\rule{0.400pt}{0.482pt}}
\put(968.67,332){\rule{0.400pt}{0.482pt}}
\multiput(968.17,333.00)(1.000,-1.000){2}{\rule{0.400pt}{0.241pt}}
\put(968.0,338.0){\rule[-0.200pt]{0.400pt}{0.964pt}}
\put(969.67,326){\rule{0.400pt}{0.723pt}}
\multiput(969.17,327.50)(1.000,-1.500){2}{\rule{0.400pt}{0.361pt}}
\put(970.67,323){\rule{0.400pt}{0.723pt}}
\multiput(970.17,324.50)(1.000,-1.500){2}{\rule{0.400pt}{0.361pt}}
\put(970.0,329.0){\rule[-0.200pt]{0.400pt}{0.723pt}}
\put(971.67,317){\rule{0.400pt}{0.723pt}}
\multiput(971.17,318.50)(1.000,-1.500){2}{\rule{0.400pt}{0.361pt}}
\put(972.0,320.0){\rule[-0.200pt]{0.400pt}{0.723pt}}
\put(973,312.67){\rule{0.241pt}{0.400pt}}
\multiput(973.00,313.17)(0.500,-1.000){2}{\rule{0.120pt}{0.400pt}}
\put(973.67,309){\rule{0.400pt}{0.964pt}}
\multiput(973.17,311.00)(1.000,-2.000){2}{\rule{0.400pt}{0.482pt}}
\put(973.0,314.0){\rule[-0.200pt]{0.400pt}{0.723pt}}
\put(974.67,304){\rule{0.400pt}{0.723pt}}
\multiput(974.17,305.50)(1.000,-1.500){2}{\rule{0.400pt}{0.361pt}}
\put(975.67,302){\rule{0.400pt}{0.482pt}}
\multiput(975.17,303.00)(1.000,-1.000){2}{\rule{0.400pt}{0.241pt}}
\put(975.0,307.0){\rule[-0.200pt]{0.400pt}{0.482pt}}
\put(976.67,295){\rule{0.400pt}{0.964pt}}
\multiput(976.17,297.00)(1.000,-2.000){2}{\rule{0.400pt}{0.482pt}}
\put(977.0,299.0){\rule[-0.200pt]{0.400pt}{0.723pt}}
\put(977.67,290){\rule{0.400pt}{0.482pt}}
\multiput(977.17,291.00)(1.000,-1.000){2}{\rule{0.400pt}{0.241pt}}
\put(978.67,287){\rule{0.400pt}{0.723pt}}
\multiput(978.17,288.50)(1.000,-1.500){2}{\rule{0.400pt}{0.361pt}}
\put(978.0,292.0){\rule[-0.200pt]{0.400pt}{0.723pt}}
\put(980,282.67){\rule{0.241pt}{0.400pt}}
\multiput(980.00,283.17)(0.500,-1.000){2}{\rule{0.120pt}{0.400pt}}
\put(980.67,280){\rule{0.400pt}{0.723pt}}
\multiput(980.17,281.50)(1.000,-1.500){2}{\rule{0.400pt}{0.361pt}}
\put(980.0,284.0){\rule[-0.200pt]{0.400pt}{0.723pt}}
\put(981.67,275){\rule{0.400pt}{0.482pt}}
\multiput(981.17,276.00)(1.000,-1.000){2}{\rule{0.400pt}{0.241pt}}
\put(982.0,277.0){\rule[-0.200pt]{0.400pt}{0.723pt}}
\put(983,270.67){\rule{0.241pt}{0.400pt}}
\multiput(983.00,271.17)(0.500,-1.000){2}{\rule{0.120pt}{0.400pt}}
\put(983.67,267){\rule{0.400pt}{0.964pt}}
\multiput(983.17,269.00)(1.000,-2.000){2}{\rule{0.400pt}{0.482pt}}
\put(983.0,272.0){\rule[-0.200pt]{0.400pt}{0.723pt}}
\put(984.67,263){\rule{0.400pt}{0.482pt}}
\multiput(984.17,264.00)(1.000,-1.000){2}{\rule{0.400pt}{0.241pt}}
\put(985.67,261){\rule{0.400pt}{0.482pt}}
\multiput(985.17,262.00)(1.000,-1.000){2}{\rule{0.400pt}{0.241pt}}
\put(985.0,265.0){\rule[-0.200pt]{0.400pt}{0.482pt}}
\put(986.67,256){\rule{0.400pt}{0.964pt}}
\multiput(986.17,258.00)(1.000,-2.000){2}{\rule{0.400pt}{0.482pt}}
\put(987.0,260.0){\usebox{\plotpoint}}
\put(987.67,252){\rule{0.400pt}{0.482pt}}
\multiput(987.17,253.00)(1.000,-1.000){2}{\rule{0.400pt}{0.241pt}}
\put(988.67,249){\rule{0.400pt}{0.723pt}}
\multiput(988.17,250.50)(1.000,-1.500){2}{\rule{0.400pt}{0.361pt}}
\put(988.0,254.0){\rule[-0.200pt]{0.400pt}{0.482pt}}
\put(989.67,246){\rule{0.400pt}{0.482pt}}
\multiput(989.17,247.00)(1.000,-1.000){2}{\rule{0.400pt}{0.241pt}}
\put(990.0,248.0){\usebox{\plotpoint}}
\put(991,242.67){\rule{0.241pt}{0.400pt}}
\multiput(991.00,243.17)(0.500,-1.000){2}{\rule{0.120pt}{0.400pt}}
\put(991.67,240){\rule{0.400pt}{0.723pt}}
\multiput(991.17,241.50)(1.000,-1.500){2}{\rule{0.400pt}{0.361pt}}
\put(991.0,244.0){\rule[-0.200pt]{0.400pt}{0.482pt}}
\put(992.67,236){\rule{0.400pt}{0.482pt}}
\multiput(992.17,237.00)(1.000,-1.000){2}{\rule{0.400pt}{0.241pt}}
\put(993.67,233){\rule{0.400pt}{0.723pt}}
\multiput(993.17,234.50)(1.000,-1.500){2}{\rule{0.400pt}{0.361pt}}
\put(993.0,238.0){\rule[-0.200pt]{0.400pt}{0.482pt}}
\put(995,233){\usebox{\plotpoint}}
\put(994.67,231){\rule{0.400pt}{0.482pt}}
\multiput(994.17,232.00)(1.000,-1.000){2}{\rule{0.400pt}{0.241pt}}
\put(996.0,228.0){\rule[-0.200pt]{0.400pt}{0.723pt}}
\put(996.67,225){\rule{0.400pt}{0.723pt}}
\multiput(996.17,226.50)(1.000,-1.500){2}{\rule{0.400pt}{0.361pt}}
\put(996.0,228.0){\usebox{\plotpoint}}
\put(998,221.67){\rule{0.241pt}{0.400pt}}
\multiput(998.00,222.17)(0.500,-1.000){2}{\rule{0.120pt}{0.400pt}}
\put(999,220.67){\rule{0.241pt}{0.400pt}}
\multiput(999.00,221.17)(0.500,-1.000){2}{\rule{0.120pt}{0.400pt}}
\put(998.0,223.0){\rule[-0.200pt]{0.400pt}{0.482pt}}
\put(1000,216.67){\rule{0.241pt}{0.400pt}}
\multiput(1000.00,217.17)(0.500,-1.000){2}{\rule{0.120pt}{0.400pt}}
\put(1000.0,218.0){\rule[-0.200pt]{0.400pt}{0.723pt}}
\put(1001,213.67){\rule{0.241pt}{0.400pt}}
\multiput(1001.00,214.17)(0.500,-1.000){2}{\rule{0.120pt}{0.400pt}}
\put(1001.67,212){\rule{0.400pt}{0.482pt}}
\multiput(1001.17,213.00)(1.000,-1.000){2}{\rule{0.400pt}{0.241pt}}
\put(1001.0,215.0){\rule[-0.200pt]{0.400pt}{0.482pt}}
\put(1002.67,209){\rule{0.400pt}{0.482pt}}
\multiput(1002.17,210.00)(1.000,-1.000){2}{\rule{0.400pt}{0.241pt}}
\put(1003.0,211.0){\usebox{\plotpoint}}
\put(1003.67,206){\rule{0.400pt}{0.482pt}}
\multiput(1003.17,207.00)(1.000,-1.000){2}{\rule{0.400pt}{0.241pt}}
\put(1005,204.67){\rule{0.241pt}{0.400pt}}
\multiput(1005.00,205.17)(0.500,-1.000){2}{\rule{0.120pt}{0.400pt}}
\put(1004.0,208.0){\usebox{\plotpoint}}
\put(1005.67,202){\rule{0.400pt}{0.482pt}}
\multiput(1005.17,203.00)(1.000,-1.000){2}{\rule{0.400pt}{0.241pt}}
\put(1007,200.67){\rule{0.241pt}{0.400pt}}
\multiput(1007.00,201.17)(0.500,-1.000){2}{\rule{0.120pt}{0.400pt}}
\put(1006.0,204.0){\usebox{\plotpoint}}
\put(1008.0,199.0){\rule[-0.200pt]{0.400pt}{0.482pt}}
\put(1008.0,199.0){\usebox{\plotpoint}}
\put(1009,195.67){\rule{0.241pt}{0.400pt}}
\multiput(1009.00,196.17)(0.500,-1.000){2}{\rule{0.120pt}{0.400pt}}
\put(1010,194.67){\rule{0.241pt}{0.400pt}}
\multiput(1010.00,195.17)(0.500,-1.000){2}{\rule{0.120pt}{0.400pt}}
\put(1009.0,197.0){\rule[-0.200pt]{0.400pt}{0.482pt}}
\put(1010.67,192){\rule{0.400pt}{0.482pt}}
\multiput(1010.17,193.00)(1.000,-1.000){2}{\rule{0.400pt}{0.241pt}}
\put(1011.0,194.0){\usebox{\plotpoint}}
\put(1012.0,192.0){\usebox{\plotpoint}}
\put(1012.67,188){\rule{0.400pt}{0.482pt}}
\multiput(1012.17,189.00)(1.000,-1.000){2}{\rule{0.400pt}{0.241pt}}
\put(1013.0,190.0){\rule[-0.200pt]{0.400pt}{0.482pt}}
\put(1014,185.67){\rule{0.241pt}{0.400pt}}
\multiput(1014.00,186.17)(0.500,-1.000){2}{\rule{0.120pt}{0.400pt}}
\put(1015,184.67){\rule{0.241pt}{0.400pt}}
\multiput(1015.00,185.17)(0.500,-1.000){2}{\rule{0.120pt}{0.400pt}}
\put(1014.0,187.0){\usebox{\plotpoint}}
\put(1016,182.67){\rule{0.241pt}{0.400pt}}
\multiput(1016.00,183.17)(0.500,-1.000){2}{\rule{0.120pt}{0.400pt}}
\put(1016.0,184.0){\usebox{\plotpoint}}
\put(1016.67,180){\rule{0.400pt}{0.482pt}}
\multiput(1016.17,181.00)(1.000,-1.000){2}{\rule{0.400pt}{0.241pt}}
\put(1017.0,182.0){\usebox{\plotpoint}}
\put(1018.0,180.0){\usebox{\plotpoint}}
\put(1019,177.67){\rule{0.241pt}{0.400pt}}
\multiput(1019.00,178.17)(0.500,-1.000){2}{\rule{0.120pt}{0.400pt}}
\put(1020,176.67){\rule{0.241pt}{0.400pt}}
\multiput(1020.00,177.17)(0.500,-1.000){2}{\rule{0.120pt}{0.400pt}}
\put(1019.0,179.0){\usebox{\plotpoint}}
\put(1021,177){\usebox{\plotpoint}}
\put(1021,175.67){\rule{0.241pt}{0.400pt}}
\multiput(1021.00,176.17)(0.500,-1.000){2}{\rule{0.120pt}{0.400pt}}
\put(1021.67,173){\rule{0.400pt}{0.482pt}}
\multiput(1021.17,174.00)(1.000,-1.000){2}{\rule{0.400pt}{0.241pt}}
\put(1022.0,175.0){\usebox{\plotpoint}}
\put(1023.67,170){\rule{0.400pt}{0.723pt}}
\multiput(1023.17,171.50)(1.000,-1.500){2}{\rule{0.400pt}{0.361pt}}
\put(1023.0,173.0){\usebox{\plotpoint}}
\put(1025.0,170.0){\usebox{\plotpoint}}
\put(1026.0,169.0){\usebox{\plotpoint}}
\put(1026.0,169.0){\usebox{\plotpoint}}
\put(1027.0,168.0){\usebox{\plotpoint}}
\put(1027.67,166){\rule{0.400pt}{0.482pt}}
\multiput(1027.17,167.00)(1.000,-1.000){2}{\rule{0.400pt}{0.241pt}}
\put(1027.0,168.0){\usebox{\plotpoint}}
\put(1029,166){\usebox{\plotpoint}}
\put(1029,164.67){\rule{0.241pt}{0.400pt}}
\multiput(1029.00,165.17)(0.500,-1.000){2}{\rule{0.120pt}{0.400pt}}
\put(1030.0,164.0){\usebox{\plotpoint}}
\put(1031,162.67){\rule{0.241pt}{0.400pt}}
\multiput(1031.00,163.17)(0.500,-1.000){2}{\rule{0.120pt}{0.400pt}}
\put(1030.0,164.0){\usebox{\plotpoint}}
\put(1032.0,162.0){\usebox{\plotpoint}}
\put(1033,160.67){\rule{0.241pt}{0.400pt}}
\multiput(1033.00,161.17)(0.500,-1.000){2}{\rule{0.120pt}{0.400pt}}
\put(1032.0,162.0){\usebox{\plotpoint}}
\put(1034.0,160.0){\usebox{\plotpoint}}
\put(1034.0,160.0){\usebox{\plotpoint}}
\put(1035,157.67){\rule{0.241pt}{0.400pt}}
\multiput(1035.00,158.17)(0.500,-1.000){2}{\rule{0.120pt}{0.400pt}}
\put(1035.0,159.0){\usebox{\plotpoint}}
\put(1037,156.67){\rule{0.241pt}{0.400pt}}
\multiput(1037.00,157.17)(0.500,-1.000){2}{\rule{0.120pt}{0.400pt}}
\put(1036.0,158.0){\usebox{\plotpoint}}
\put(1038.0,157.0){\usebox{\plotpoint}}
\put(1039,154.67){\rule{0.241pt}{0.400pt}}
\multiput(1039.00,155.17)(0.500,-1.000){2}{\rule{0.120pt}{0.400pt}}
\put(1039.0,156.0){\usebox{\plotpoint}}
\put(1040,155){\usebox{\plotpoint}}
\put(1040,153.67){\rule{0.241pt}{0.400pt}}
\multiput(1040.00,154.17)(0.500,-1.000){2}{\rule{0.120pt}{0.400pt}}
\put(1041.0,154.0){\usebox{\plotpoint}}
\put(1042.0,153.0){\usebox{\plotpoint}}
\put(1042.0,153.0){\usebox{\plotpoint}}
\put(1043,151.67){\rule{0.241pt}{0.400pt}}
\multiput(1043.00,151.17)(0.500,1.000){2}{\rule{0.120pt}{0.400pt}}
\put(1043.67,151){\rule{0.400pt}{0.482pt}}
\multiput(1043.17,152.00)(1.000,-1.000){2}{\rule{0.400pt}{0.241pt}}
\put(1043.0,152.0){\usebox{\plotpoint}}
\put(1045,151){\usebox{\plotpoint}}
\put(1046,149.67){\rule{0.241pt}{0.400pt}}
\multiput(1046.00,150.17)(0.500,-1.000){2}{\rule{0.120pt}{0.400pt}}
\put(1045.0,151.0){\usebox{\plotpoint}}
\put(1047,150){\usebox{\plotpoint}}
\put(1047,148.67){\rule{0.241pt}{0.400pt}}
\multiput(1047.00,149.17)(0.500,-1.000){2}{\rule{0.120pt}{0.400pt}}
\put(1048,149){\usebox{\plotpoint}}
\put(1048,147.67){\rule{0.241pt}{0.400pt}}
\multiput(1048.00,148.17)(0.500,-1.000){2}{\rule{0.120pt}{0.400pt}}
\put(1049.0,148.0){\usebox{\plotpoint}}
\put(1050.0,147.0){\usebox{\plotpoint}}
\put(1050.0,147.0){\usebox{\plotpoint}}
\put(1051,145.67){\rule{0.241pt}{0.400pt}}
\multiput(1051.00,145.17)(0.500,1.000){2}{\rule{0.120pt}{0.400pt}}
\put(1052,145.67){\rule{0.241pt}{0.400pt}}
\multiput(1052.00,146.17)(0.500,-1.000){2}{\rule{0.120pt}{0.400pt}}
\put(1051.0,146.0){\usebox{\plotpoint}}
\put(1053,146){\usebox{\plotpoint}}
\put(1053,144.67){\rule{0.241pt}{0.400pt}}
\multiput(1053.00,145.17)(0.500,-1.000){2}{\rule{0.120pt}{0.400pt}}
\put(1054,143.67){\rule{0.241pt}{0.400pt}}
\multiput(1054.00,144.17)(0.500,-1.000){2}{\rule{0.120pt}{0.400pt}}
\put(1055.0,144.0){\usebox{\plotpoint}}
\put(1055.0,145.0){\usebox{\plotpoint}}
\put(1056,142.67){\rule{0.241pt}{0.400pt}}
\multiput(1056.00,143.17)(0.500,-1.000){2}{\rule{0.120pt}{0.400pt}}
\put(1057,142.67){\rule{0.241pt}{0.400pt}}
\multiput(1057.00,142.17)(0.500,1.000){2}{\rule{0.120pt}{0.400pt}}
\put(1056.0,144.0){\usebox{\plotpoint}}
\put(1058.0,143.0){\usebox{\plotpoint}}
\put(1058.0,143.0){\usebox{\plotpoint}}
\put(1059.0,142.0){\usebox{\plotpoint}}
\put(1062,140.67){\rule{0.241pt}{0.400pt}}
\multiput(1062.00,141.17)(0.500,-1.000){2}{\rule{0.120pt}{0.400pt}}
\put(1059.0,142.0){\rule[-0.200pt]{0.723pt}{0.400pt}}
\put(1063,141){\usebox{\plotpoint}}
\put(1063.0,141.0){\usebox{\plotpoint}}
\put(1064.0,140.0){\usebox{\plotpoint}}
\put(1067,138.67){\rule{0.241pt}{0.400pt}}
\multiput(1067.00,139.17)(0.500,-1.000){2}{\rule{0.120pt}{0.400pt}}
\put(1068,137.67){\rule{0.241pt}{0.400pt}}
\multiput(1068.00,138.17)(0.500,-1.000){2}{\rule{0.120pt}{0.400pt}}
\put(1064.0,140.0){\rule[-0.200pt]{0.723pt}{0.400pt}}
\put(1069,138){\usebox{\plotpoint}}
\put(1070,137.67){\rule{0.241pt}{0.400pt}}
\multiput(1070.00,137.17)(0.500,1.000){2}{\rule{0.120pt}{0.400pt}}
\put(1069.0,138.0){\usebox{\plotpoint}}
\put(1071,139){\usebox{\plotpoint}}
\put(1071,137.67){\rule{0.241pt}{0.400pt}}
\multiput(1071.00,138.17)(0.500,-1.000){2}{\rule{0.120pt}{0.400pt}}
\put(1072,138){\usebox{\plotpoint}}
\put(1072.0,138.0){\rule[-0.200pt]{0.482pt}{0.400pt}}
\put(1074,135.67){\rule{0.241pt}{0.400pt}}
\multiput(1074.00,135.17)(0.500,1.000){2}{\rule{0.120pt}{0.400pt}}
\put(1074.0,136.0){\rule[-0.200pt]{0.400pt}{0.482pt}}
\put(1075,135.67){\rule{0.241pt}{0.400pt}}
\multiput(1075.00,135.17)(0.500,1.000){2}{\rule{0.120pt}{0.400pt}}
\put(1076,135.67){\rule{0.241pt}{0.400pt}}
\multiput(1076.00,136.17)(0.500,-1.000){2}{\rule{0.120pt}{0.400pt}}
\put(1075.0,136.0){\usebox{\plotpoint}}
\put(1077,136){\usebox{\plotpoint}}
\put(1077.0,136.0){\rule[-0.200pt]{0.723pt}{0.400pt}}
\put(1080,134.67){\rule{0.241pt}{0.400pt}}
\multiput(1080.00,134.17)(0.500,1.000){2}{\rule{0.120pt}{0.400pt}}
\put(1081,134.67){\rule{0.241pt}{0.400pt}}
\multiput(1081.00,135.17)(0.500,-1.000){2}{\rule{0.120pt}{0.400pt}}
\put(1080.0,135.0){\usebox{\plotpoint}}
\put(1082,135){\usebox{\plotpoint}}
\put(1082,133.67){\rule{0.241pt}{0.400pt}}
\multiput(1082.00,134.17)(0.500,-1.000){2}{\rule{0.120pt}{0.400pt}}
\put(1083,133.67){\rule{0.241pt}{0.400pt}}
\multiput(1083.00,134.17)(0.500,-1.000){2}{\rule{0.120pt}{0.400pt}}
\put(1084,133.67){\rule{0.241pt}{0.400pt}}
\multiput(1084.00,133.17)(0.500,1.000){2}{\rule{0.120pt}{0.400pt}}
\put(1083.0,134.0){\usebox{\plotpoint}}
\put(1085,135){\usebox{\plotpoint}}
\put(1085,133.67){\rule{0.241pt}{0.400pt}}
\multiput(1085.00,134.17)(0.500,-1.000){2}{\rule{0.120pt}{0.400pt}}
\put(1086,133.67){\rule{0.241pt}{0.400pt}}
\multiput(1086.00,134.17)(0.500,-1.000){2}{\rule{0.120pt}{0.400pt}}
\put(1086.0,134.0){\usebox{\plotpoint}}
\put(1087.0,134.0){\usebox{\plotpoint}}
\put(1088,132.67){\rule{0.241pt}{0.400pt}}
\multiput(1088.00,132.17)(0.500,1.000){2}{\rule{0.120pt}{0.400pt}}
\put(1088.0,133.0){\usebox{\plotpoint}}
\put(1089.0,134.0){\usebox{\plotpoint}}
\put(1090.0,133.0){\usebox{\plotpoint}}
\put(1090.0,133.0){\rule[-0.200pt]{0.723pt}{0.400pt}}
\put(1093,132.67){\rule{0.241pt}{0.400pt}}
\multiput(1093.00,133.17)(0.500,-1.000){2}{\rule{0.120pt}{0.400pt}}
\put(1093.0,133.0){\usebox{\plotpoint}}
\put(1094,133){\usebox{\plotpoint}}
\put(1094.0,133.0){\rule[-0.200pt]{0.964pt}{0.400pt}}
\put(1098,131.67){\rule{0.241pt}{0.400pt}}
\multiput(1098.00,131.17)(0.500,1.000){2}{\rule{0.120pt}{0.400pt}}
\put(1098.0,132.0){\usebox{\plotpoint}}
\put(1099.0,132.0){\usebox{\plotpoint}}
\put(1104,130.67){\rule{0.241pt}{0.400pt}}
\multiput(1104.00,131.17)(0.500,-1.000){2}{\rule{0.120pt}{0.400pt}}
\put(1099.0,132.0){\rule[-0.200pt]{1.204pt}{0.400pt}}
\put(1105,130.67){\rule{0.241pt}{0.400pt}}
\multiput(1105.00,131.17)(0.500,-1.000){2}{\rule{0.120pt}{0.400pt}}
\put(1105.0,131.0){\usebox{\plotpoint}}
\put(1106.0,131.0){\usebox{\plotpoint}}
\put(1108.0,131.0){\rule[-0.200pt]{0.482pt}{0.400pt}}
\put(1119,131){\usebox{\plotpoint}}
\put(1119,131){\usebox{\plotpoint}}
\put(171.0,314.0){\usebox{\plotpoint}}
\put(171.67,318){\rule{0.400pt}{0.482pt}}
\multiput(171.17,318.00)(1.000,1.000){2}{\rule{0.400pt}{0.241pt}}
\put(173,318.67){\rule{0.241pt}{0.400pt}}
\multiput(173.00,319.17)(0.500,-1.000){2}{\rule{0.120pt}{0.400pt}}
\put(172.0,314.0){\rule[-0.200pt]{0.400pt}{0.964pt}}
\put(173.67,319){\rule{0.400pt}{0.482pt}}
\multiput(173.17,320.00)(1.000,-1.000){2}{\rule{0.400pt}{0.241pt}}
\put(174.67,319){\rule{0.400pt}{0.482pt}}
\multiput(174.17,319.00)(1.000,1.000){2}{\rule{0.400pt}{0.241pt}}
\put(174.0,319.0){\rule[-0.200pt]{0.400pt}{0.482pt}}
\put(176,321){\usebox{\plotpoint}}
\put(176.67,321){\rule{0.400pt}{0.964pt}}
\multiput(176.17,321.00)(1.000,2.000){2}{\rule{0.400pt}{0.482pt}}
\put(176.0,321.0){\usebox{\plotpoint}}
\put(177.67,322){\rule{0.400pt}{0.482pt}}
\multiput(177.17,322.00)(1.000,1.000){2}{\rule{0.400pt}{0.241pt}}
\put(178.67,321){\rule{0.400pt}{0.723pt}}
\multiput(178.17,322.50)(1.000,-1.500){2}{\rule{0.400pt}{0.361pt}}
\put(178.0,322.0){\rule[-0.200pt]{0.400pt}{0.723pt}}
\put(180.0,321.0){\rule[-0.200pt]{0.400pt}{0.723pt}}
\put(180.67,324){\rule{0.400pt}{0.723pt}}
\multiput(180.17,324.00)(1.000,1.500){2}{\rule{0.400pt}{0.361pt}}
\put(180.0,324.0){\usebox{\plotpoint}}
\put(182,325.67){\rule{0.241pt}{0.400pt}}
\multiput(182.00,325.17)(0.500,1.000){2}{\rule{0.120pt}{0.400pt}}
\put(182.67,324){\rule{0.400pt}{0.723pt}}
\multiput(182.17,325.50)(1.000,-1.500){2}{\rule{0.400pt}{0.361pt}}
\put(182.0,326.0){\usebox{\plotpoint}}
\put(183.67,325){\rule{0.400pt}{0.723pt}}
\multiput(183.17,326.50)(1.000,-1.500){2}{\rule{0.400pt}{0.361pt}}
\put(184.67,325){\rule{0.400pt}{0.964pt}}
\multiput(184.17,325.00)(1.000,2.000){2}{\rule{0.400pt}{0.482pt}}
\put(184.0,324.0){\rule[-0.200pt]{0.400pt}{0.964pt}}
\put(186,326.67){\rule{0.241pt}{0.400pt}}
\multiput(186.00,326.17)(0.500,1.000){2}{\rule{0.120pt}{0.400pt}}
\put(186.67,328){\rule{0.400pt}{1.204pt}}
\multiput(186.17,328.00)(1.000,2.500){2}{\rule{0.400pt}{0.602pt}}
\put(186.0,327.0){\rule[-0.200pt]{0.400pt}{0.482pt}}
\put(187.67,329){\rule{0.400pt}{0.723pt}}
\multiput(187.17,330.50)(1.000,-1.500){2}{\rule{0.400pt}{0.361pt}}
\put(188.67,329){\rule{0.400pt}{1.204pt}}
\multiput(188.17,329.00)(1.000,2.500){2}{\rule{0.400pt}{0.602pt}}
\put(189.67,330){\rule{0.400pt}{0.964pt}}
\multiput(189.17,332.00)(1.000,-2.000){2}{\rule{0.400pt}{0.482pt}}
\put(188.0,332.0){\usebox{\plotpoint}}
\put(191,330.67){\rule{0.241pt}{0.400pt}}
\multiput(191.00,331.17)(0.500,-1.000){2}{\rule{0.120pt}{0.400pt}}
\put(191.67,331){\rule{0.400pt}{0.482pt}}
\multiput(191.17,331.00)(1.000,1.000){2}{\rule{0.400pt}{0.241pt}}
\put(191.0,330.0){\rule[-0.200pt]{0.400pt}{0.482pt}}
\put(193,333){\usebox{\plotpoint}}
\put(193,332.67){\rule{0.241pt}{0.400pt}}
\multiput(193.00,332.17)(0.500,1.000){2}{\rule{0.120pt}{0.400pt}}
\put(193.67,332){\rule{0.400pt}{0.482pt}}
\multiput(193.17,333.00)(1.000,-1.000){2}{\rule{0.400pt}{0.241pt}}
\put(194.67,333){\rule{0.400pt}{0.482pt}}
\multiput(194.17,334.00)(1.000,-1.000){2}{\rule{0.400pt}{0.241pt}}
\put(196,332.67){\rule{0.241pt}{0.400pt}}
\multiput(196.00,332.17)(0.500,1.000){2}{\rule{0.120pt}{0.400pt}}
\put(195.0,332.0){\rule[-0.200pt]{0.400pt}{0.723pt}}
\put(196.67,334){\rule{0.400pt}{0.723pt}}
\multiput(196.17,335.50)(1.000,-1.500){2}{\rule{0.400pt}{0.361pt}}
\put(197.0,334.0){\rule[-0.200pt]{0.400pt}{0.723pt}}
\put(200,332.67){\rule{0.241pt}{0.400pt}}
\multiput(200.00,333.17)(0.500,-1.000){2}{\rule{0.120pt}{0.400pt}}
\put(198.0,334.0){\rule[-0.200pt]{0.482pt}{0.400pt}}
\put(201,333.67){\rule{0.241pt}{0.400pt}}
\multiput(201.00,333.17)(0.500,1.000){2}{\rule{0.120pt}{0.400pt}}
\put(201.0,333.0){\usebox{\plotpoint}}
\put(202.0,335.0){\usebox{\plotpoint}}
\put(203.0,333.0){\rule[-0.200pt]{0.400pt}{0.482pt}}
\put(204,332.67){\rule{0.241pt}{0.400pt}}
\multiput(204.00,332.17)(0.500,1.000){2}{\rule{0.120pt}{0.400pt}}
\put(203.0,333.0){\usebox{\plotpoint}}
\put(204.67,333){\rule{0.400pt}{0.482pt}}
\multiput(204.17,333.00)(1.000,1.000){2}{\rule{0.400pt}{0.241pt}}
\put(206,333.67){\rule{0.241pt}{0.400pt}}
\multiput(206.00,334.17)(0.500,-1.000){2}{\rule{0.120pt}{0.400pt}}
\put(205.0,333.0){\usebox{\plotpoint}}
\put(207.0,334.0){\usebox{\plotpoint}}
\put(208,334.67){\rule{0.241pt}{0.400pt}}
\multiput(208.00,334.17)(0.500,1.000){2}{\rule{0.120pt}{0.400pt}}
\put(207.0,335.0){\usebox{\plotpoint}}
\put(209.0,333.0){\rule[-0.200pt]{0.400pt}{0.723pt}}
\put(210,331.67){\rule{0.241pt}{0.400pt}}
\multiput(210.00,332.17)(0.500,-1.000){2}{\rule{0.120pt}{0.400pt}}
\put(209.0,333.0){\usebox{\plotpoint}}
\put(210.67,333){\rule{0.400pt}{0.482pt}}
\multiput(210.17,334.00)(1.000,-1.000){2}{\rule{0.400pt}{0.241pt}}
\put(212,331.67){\rule{0.241pt}{0.400pt}}
\multiput(212.00,332.17)(0.500,-1.000){2}{\rule{0.120pt}{0.400pt}}
\put(211.0,332.0){\rule[-0.200pt]{0.400pt}{0.723pt}}
\put(213,332){\usebox{\plotpoint}}
\put(213.67,330){\rule{0.400pt}{0.482pt}}
\multiput(213.17,331.00)(1.000,-1.000){2}{\rule{0.400pt}{0.241pt}}
\put(213.0,332.0){\usebox{\plotpoint}}
\put(215,329.67){\rule{0.241pt}{0.400pt}}
\multiput(215.00,330.17)(0.500,-1.000){2}{\rule{0.120pt}{0.400pt}}
\put(215.67,330){\rule{0.400pt}{0.482pt}}
\multiput(215.17,330.00)(1.000,1.000){2}{\rule{0.400pt}{0.241pt}}
\put(215.0,330.0){\usebox{\plotpoint}}
\put(216.67,329){\rule{0.400pt}{0.482pt}}
\multiput(216.17,330.00)(1.000,-1.000){2}{\rule{0.400pt}{0.241pt}}
\put(217.0,331.0){\usebox{\plotpoint}}
\put(218.0,329.0){\usebox{\plotpoint}}
\put(219.0,329.0){\usebox{\plotpoint}}
\put(219.67,328){\rule{0.400pt}{0.482pt}}
\multiput(219.17,329.00)(1.000,-1.000){2}{\rule{0.400pt}{0.241pt}}
\put(219.0,330.0){\usebox{\plotpoint}}
\put(221,328){\usebox{\plotpoint}}
\put(221.0,328.0){\rule[-0.200pt]{0.482pt}{0.400pt}}
\put(223,325.67){\rule{0.241pt}{0.400pt}}
\multiput(223.00,326.17)(0.500,-1.000){2}{\rule{0.120pt}{0.400pt}}
\put(224,324.67){\rule{0.241pt}{0.400pt}}
\multiput(224.00,325.17)(0.500,-1.000){2}{\rule{0.120pt}{0.400pt}}
\put(223.0,327.0){\usebox{\plotpoint}}
\put(225.0,325.0){\usebox{\plotpoint}}
\put(225.67,324){\rule{0.400pt}{0.482pt}}
\multiput(225.17,325.00)(1.000,-1.000){2}{\rule{0.400pt}{0.241pt}}
\put(225.0,326.0){\usebox{\plotpoint}}
\put(227,323.67){\rule{0.241pt}{0.400pt}}
\multiput(227.00,324.17)(0.500,-1.000){2}{\rule{0.120pt}{0.400pt}}
\put(228,322.67){\rule{0.241pt}{0.400pt}}
\multiput(228.00,323.17)(0.500,-1.000){2}{\rule{0.120pt}{0.400pt}}
\put(227.0,324.0){\usebox{\plotpoint}}
\put(229,321.67){\rule{0.241pt}{0.400pt}}
\multiput(229.00,321.17)(0.500,1.000){2}{\rule{0.120pt}{0.400pt}}
\put(230,322.67){\rule{0.241pt}{0.400pt}}
\multiput(230.00,322.17)(0.500,1.000){2}{\rule{0.120pt}{0.400pt}}
\put(229.0,322.0){\usebox{\plotpoint}}
\put(231,320.67){\rule{0.241pt}{0.400pt}}
\multiput(231.00,321.17)(0.500,-1.000){2}{\rule{0.120pt}{0.400pt}}
\put(232,320.67){\rule{0.241pt}{0.400pt}}
\multiput(232.00,320.17)(0.500,1.000){2}{\rule{0.120pt}{0.400pt}}
\put(231.0,322.0){\rule[-0.200pt]{0.400pt}{0.482pt}}
\put(233.0,320.0){\rule[-0.200pt]{0.400pt}{0.482pt}}
\put(234,318.67){\rule{0.241pt}{0.400pt}}
\multiput(234.00,319.17)(0.500,-1.000){2}{\rule{0.120pt}{0.400pt}}
\put(233.0,320.0){\usebox{\plotpoint}}
\put(235.0,319.0){\usebox{\plotpoint}}
\put(236,317.67){\rule{0.241pt}{0.400pt}}
\multiput(236.00,317.17)(0.500,1.000){2}{\rule{0.120pt}{0.400pt}}
\put(237,317.67){\rule{0.241pt}{0.400pt}}
\multiput(237.00,318.17)(0.500,-1.000){2}{\rule{0.120pt}{0.400pt}}
\put(236.0,318.0){\usebox{\plotpoint}}
\put(238,315.67){\rule{0.241pt}{0.400pt}}
\multiput(238.00,316.17)(0.500,-1.000){2}{\rule{0.120pt}{0.400pt}}
\put(239,314.67){\rule{0.241pt}{0.400pt}}
\multiput(239.00,315.17)(0.500,-1.000){2}{\rule{0.120pt}{0.400pt}}
\put(238.0,317.0){\usebox{\plotpoint}}
\put(240.0,315.0){\usebox{\plotpoint}}
\put(240.67,314){\rule{0.400pt}{0.482pt}}
\multiput(240.17,315.00)(1.000,-1.000){2}{\rule{0.400pt}{0.241pt}}
\put(240.0,316.0){\usebox{\plotpoint}}
\put(241.67,313){\rule{0.400pt}{0.482pt}}
\multiput(241.17,314.00)(1.000,-1.000){2}{\rule{0.400pt}{0.241pt}}
\put(242.0,314.0){\usebox{\plotpoint}}
\put(244,312.67){\rule{0.241pt}{0.400pt}}
\multiput(244.00,312.17)(0.500,1.000){2}{\rule{0.120pt}{0.400pt}}
\put(244.67,311){\rule{0.400pt}{0.723pt}}
\multiput(244.17,312.50)(1.000,-1.500){2}{\rule{0.400pt}{0.361pt}}
\put(243.0,313.0){\usebox{\plotpoint}}
\put(245.67,310){\rule{0.400pt}{0.482pt}}
\multiput(245.17,311.00)(1.000,-1.000){2}{\rule{0.400pt}{0.241pt}}
\put(246.0,311.0){\usebox{\plotpoint}}
\put(247.0,310.0){\usebox{\plotpoint}}
\put(248,310.67){\rule{0.241pt}{0.400pt}}
\multiput(248.00,310.17)(0.500,1.000){2}{\rule{0.120pt}{0.400pt}}
\put(248.67,309){\rule{0.400pt}{0.723pt}}
\multiput(248.17,310.50)(1.000,-1.500){2}{\rule{0.400pt}{0.361pt}}
\put(248.0,310.0){\usebox{\plotpoint}}
\put(250.0,309.0){\usebox{\plotpoint}}
\put(251.67,308){\rule{0.400pt}{0.482pt}}
\multiput(251.17,309.00)(1.000,-1.000){2}{\rule{0.400pt}{0.241pt}}
\put(253,306.67){\rule{0.241pt}{0.400pt}}
\multiput(253.00,307.17)(0.500,-1.000){2}{\rule{0.120pt}{0.400pt}}
\put(250.0,310.0){\rule[-0.200pt]{0.482pt}{0.400pt}}
\put(254,307){\usebox{\plotpoint}}
\put(255,306.67){\rule{0.241pt}{0.400pt}}
\multiput(255.00,306.17)(0.500,1.000){2}{\rule{0.120pt}{0.400pt}}
\put(254.0,307.0){\usebox{\plotpoint}}
\put(256.0,306.0){\rule[-0.200pt]{0.400pt}{0.482pt}}
\put(258,304.67){\rule{0.241pt}{0.400pt}}
\multiput(258.00,305.17)(0.500,-1.000){2}{\rule{0.120pt}{0.400pt}}
\put(259,303.67){\rule{0.241pt}{0.400pt}}
\multiput(259.00,304.17)(0.500,-1.000){2}{\rule{0.120pt}{0.400pt}}
\put(256.0,306.0){\rule[-0.200pt]{0.482pt}{0.400pt}}
\put(259.67,303){\rule{0.400pt}{0.482pt}}
\multiput(259.17,304.00)(1.000,-1.000){2}{\rule{0.400pt}{0.241pt}}
\put(261,302.67){\rule{0.241pt}{0.400pt}}
\multiput(261.00,302.17)(0.500,1.000){2}{\rule{0.120pt}{0.400pt}}
\put(260.0,304.0){\usebox{\plotpoint}}
\put(262,303.67){\rule{0.241pt}{0.400pt}}
\multiput(262.00,304.17)(0.500,-1.000){2}{\rule{0.120pt}{0.400pt}}
\put(262.0,304.0){\usebox{\plotpoint}}
\put(263.0,304.0){\usebox{\plotpoint}}
\put(264.0,303.0){\usebox{\plotpoint}}
\put(265,301.67){\rule{0.241pt}{0.400pt}}
\multiput(265.00,302.17)(0.500,-1.000){2}{\rule{0.120pt}{0.400pt}}
\put(264.0,303.0){\usebox{\plotpoint}}
\put(266,301.67){\rule{0.241pt}{0.400pt}}
\multiput(266.00,302.17)(0.500,-1.000){2}{\rule{0.120pt}{0.400pt}}
\put(267,301.67){\rule{0.241pt}{0.400pt}}
\multiput(267.00,301.17)(0.500,1.000){2}{\rule{0.120pt}{0.400pt}}
\put(266.0,302.0){\usebox{\plotpoint}}
\put(268,300.67){\rule{0.241pt}{0.400pt}}
\multiput(268.00,301.17)(0.500,-1.000){2}{\rule{0.120pt}{0.400pt}}
\put(269,300.67){\rule{0.241pt}{0.400pt}}
\multiput(269.00,300.17)(0.500,1.000){2}{\rule{0.120pt}{0.400pt}}
\put(268.0,302.0){\usebox{\plotpoint}}
\put(270,302){\usebox{\plotpoint}}
\put(270,300.67){\rule{0.241pt}{0.400pt}}
\multiput(270.00,301.17)(0.500,-1.000){2}{\rule{0.120pt}{0.400pt}}
\put(271,299.67){\rule{0.241pt}{0.400pt}}
\multiput(271.00,300.17)(0.500,-1.000){2}{\rule{0.120pt}{0.400pt}}
\put(272,300){\usebox{\plotpoint}}
\put(271.67,300){\rule{0.400pt}{0.482pt}}
\multiput(271.17,300.00)(1.000,1.000){2}{\rule{0.400pt}{0.241pt}}
\put(273,301.67){\rule{0.241pt}{0.400pt}}
\multiput(273.00,301.17)(0.500,1.000){2}{\rule{0.120pt}{0.400pt}}
\put(274.0,301.0){\rule[-0.200pt]{0.400pt}{0.482pt}}
\put(274.0,301.0){\rule[-0.200pt]{0.482pt}{0.400pt}}
\put(276,299.67){\rule{0.241pt}{0.400pt}}
\multiput(276.00,299.17)(0.500,1.000){2}{\rule{0.120pt}{0.400pt}}
\put(277,299.67){\rule{0.241pt}{0.400pt}}
\multiput(277.00,300.17)(0.500,-1.000){2}{\rule{0.120pt}{0.400pt}}
\put(276.0,300.0){\usebox{\plotpoint}}
\put(278.0,300.0){\usebox{\plotpoint}}
\put(278.0,301.0){\rule[-0.200pt]{0.482pt}{0.400pt}}
\put(280,299.67){\rule{0.241pt}{0.400pt}}
\multiput(280.00,299.17)(0.500,1.000){2}{\rule{0.120pt}{0.400pt}}
\put(281,299.67){\rule{0.241pt}{0.400pt}}
\multiput(281.00,300.17)(0.500,-1.000){2}{\rule{0.120pt}{0.400pt}}
\put(280.0,300.0){\usebox{\plotpoint}}
\put(282.0,300.0){\usebox{\plotpoint}}
\put(284,300.67){\rule{0.241pt}{0.400pt}}
\multiput(284.00,300.17)(0.500,1.000){2}{\rule{0.120pt}{0.400pt}}
\put(284.67,300){\rule{0.400pt}{0.482pt}}
\multiput(284.17,301.00)(1.000,-1.000){2}{\rule{0.400pt}{0.241pt}}
\put(282.0,301.0){\rule[-0.200pt]{0.482pt}{0.400pt}}
\put(286.0,300.0){\rule[-0.200pt]{0.400pt}{0.482pt}}
\put(287,301.67){\rule{0.241pt}{0.400pt}}
\multiput(287.00,301.17)(0.500,1.000){2}{\rule{0.120pt}{0.400pt}}
\put(286.0,302.0){\usebox{\plotpoint}}
\put(288,303){\usebox{\plotpoint}}
\put(288,301.67){\rule{0.241pt}{0.400pt}}
\multiput(288.00,302.17)(0.500,-1.000){2}{\rule{0.120pt}{0.400pt}}
\put(289,301.67){\rule{0.241pt}{0.400pt}}
\multiput(289.00,301.17)(0.500,1.000){2}{\rule{0.120pt}{0.400pt}}
\put(290,302.67){\rule{0.241pt}{0.400pt}}
\multiput(290.00,303.17)(0.500,-1.000){2}{\rule{0.120pt}{0.400pt}}
\put(291,302.67){\rule{0.241pt}{0.400pt}}
\multiput(291.00,302.17)(0.500,1.000){2}{\rule{0.120pt}{0.400pt}}
\put(290.0,303.0){\usebox{\plotpoint}}
\put(291.67,303){\rule{0.400pt}{0.482pt}}
\multiput(291.17,303.00)(1.000,1.000){2}{\rule{0.400pt}{0.241pt}}
\put(292.0,303.0){\usebox{\plotpoint}}
\put(295,304.67){\rule{0.241pt}{0.400pt}}
\multiput(295.00,304.17)(0.500,1.000){2}{\rule{0.120pt}{0.400pt}}
\put(293.0,305.0){\rule[-0.200pt]{0.482pt}{0.400pt}}
\put(296,306){\usebox{\plotpoint}}
\put(296,305.67){\rule{0.241pt}{0.400pt}}
\multiput(296.00,305.17)(0.500,1.000){2}{\rule{0.120pt}{0.400pt}}
\put(299,306.67){\rule{0.241pt}{0.400pt}}
\multiput(299.00,306.17)(0.500,1.000){2}{\rule{0.120pt}{0.400pt}}
\put(297.0,307.0){\rule[-0.200pt]{0.482pt}{0.400pt}}
\put(300,307.67){\rule{0.241pt}{0.400pt}}
\multiput(300.00,308.17)(0.500,-1.000){2}{\rule{0.120pt}{0.400pt}}
\put(300.67,308){\rule{0.400pt}{0.482pt}}
\multiput(300.17,308.00)(1.000,1.000){2}{\rule{0.400pt}{0.241pt}}
\put(300.0,308.0){\usebox{\plotpoint}}
\put(302.0,310.0){\usebox{\plotpoint}}
\put(302.0,311.0){\rule[-0.200pt]{0.482pt}{0.400pt}}
\put(304,311.67){\rule{0.241pt}{0.400pt}}
\multiput(304.00,311.17)(0.500,1.000){2}{\rule{0.120pt}{0.400pt}}
\put(305,312.67){\rule{0.241pt}{0.400pt}}
\multiput(305.00,312.17)(0.500,1.000){2}{\rule{0.120pt}{0.400pt}}
\put(304.0,311.0){\usebox{\plotpoint}}
\put(306,312.67){\rule{0.241pt}{0.400pt}}
\multiput(306.00,312.17)(0.500,1.000){2}{\rule{0.120pt}{0.400pt}}
\put(306.0,313.0){\usebox{\plotpoint}}
\put(307.0,314.0){\usebox{\plotpoint}}
\put(308.0,314.0){\rule[-0.200pt]{0.400pt}{0.482pt}}
\put(308.0,316.0){\rule[-0.200pt]{0.482pt}{0.400pt}}
\put(310,317.67){\rule{0.241pt}{0.400pt}}
\multiput(310.00,318.17)(0.500,-1.000){2}{\rule{0.120pt}{0.400pt}}
\put(310.67,318){\rule{0.400pt}{0.482pt}}
\multiput(310.17,318.00)(1.000,1.000){2}{\rule{0.400pt}{0.241pt}}
\put(310.0,316.0){\rule[-0.200pt]{0.400pt}{0.723pt}}
\put(311.67,319){\rule{0.400pt}{0.482pt}}
\multiput(311.17,319.00)(1.000,1.000){2}{\rule{0.400pt}{0.241pt}}
\put(312.0,319.0){\usebox{\plotpoint}}
\put(313.0,321.0){\usebox{\plotpoint}}
\put(313.67,323){\rule{0.400pt}{0.482pt}}
\multiput(313.17,323.00)(1.000,1.000){2}{\rule{0.400pt}{0.241pt}}
\put(315,324.67){\rule{0.241pt}{0.400pt}}
\multiput(315.00,324.17)(0.500,1.000){2}{\rule{0.120pt}{0.400pt}}
\put(314.0,321.0){\rule[-0.200pt]{0.400pt}{0.482pt}}
\put(316,325.67){\rule{0.241pt}{0.400pt}}
\multiput(316.00,326.17)(0.500,-1.000){2}{\rule{0.120pt}{0.400pt}}
\put(316.67,326){\rule{0.400pt}{0.723pt}}
\multiput(316.17,326.00)(1.000,1.500){2}{\rule{0.400pt}{0.361pt}}
\put(316.0,326.0){\usebox{\plotpoint}}
\put(318,329){\usebox{\plotpoint}}
\put(318,328.67){\rule{0.241pt}{0.400pt}}
\multiput(318.00,328.17)(0.500,1.000){2}{\rule{0.120pt}{0.400pt}}
\put(319,328.67){\rule{0.241pt}{0.400pt}}
\multiput(319.00,329.17)(0.500,-1.000){2}{\rule{0.120pt}{0.400pt}}
\put(320.0,329.0){\rule[-0.200pt]{0.400pt}{0.723pt}}
\put(320.67,332){\rule{0.400pt}{0.482pt}}
\multiput(320.17,332.00)(1.000,1.000){2}{\rule{0.400pt}{0.241pt}}
\put(320.0,332.0){\usebox{\plotpoint}}
\put(321.67,333){\rule{0.400pt}{0.482pt}}
\multiput(321.17,333.00)(1.000,1.000){2}{\rule{0.400pt}{0.241pt}}
\put(323,334.67){\rule{0.241pt}{0.400pt}}
\multiput(323.00,334.17)(0.500,1.000){2}{\rule{0.120pt}{0.400pt}}
\put(322.0,333.0){\usebox{\plotpoint}}
\put(324,336.67){\rule{0.241pt}{0.400pt}}
\multiput(324.00,337.17)(0.500,-1.000){2}{\rule{0.120pt}{0.400pt}}
\put(324.67,337){\rule{0.400pt}{0.482pt}}
\multiput(324.17,337.00)(1.000,1.000){2}{\rule{0.400pt}{0.241pt}}
\put(324.0,336.0){\rule[-0.200pt]{0.400pt}{0.482pt}}
\put(325.67,340){\rule{0.400pt}{0.482pt}}
\multiput(325.17,340.00)(1.000,1.000){2}{\rule{0.400pt}{0.241pt}}
\put(327,340.67){\rule{0.241pt}{0.400pt}}
\multiput(327.00,341.17)(0.500,-1.000){2}{\rule{0.120pt}{0.400pt}}
\put(326.0,339.0){\usebox{\plotpoint}}
\put(327.67,344){\rule{0.400pt}{0.482pt}}
\multiput(327.17,344.00)(1.000,1.000){2}{\rule{0.400pt}{0.241pt}}
\put(328.0,341.0){\rule[-0.200pt]{0.400pt}{0.723pt}}
\put(329.0,346.0){\usebox{\plotpoint}}
\put(330.0,346.0){\usebox{\plotpoint}}
\put(330.67,347){\rule{0.400pt}{0.723pt}}
\multiput(330.17,347.00)(1.000,1.500){2}{\rule{0.400pt}{0.361pt}}
\put(330.0,347.0){\usebox{\plotpoint}}
\put(332.0,350.0){\usebox{\plotpoint}}
\put(333,350.67){\rule{0.241pt}{0.400pt}}
\multiput(333.00,350.17)(0.500,1.000){2}{\rule{0.120pt}{0.400pt}}
\put(332.0,351.0){\usebox{\plotpoint}}
\put(334,352){\usebox{\plotpoint}}
\put(333.67,352){\rule{0.400pt}{0.964pt}}
\multiput(333.17,352.00)(1.000,2.000){2}{\rule{0.400pt}{0.482pt}}
\put(335,355.67){\rule{0.241pt}{0.400pt}}
\multiput(335.00,355.17)(0.500,1.000){2}{\rule{0.120pt}{0.400pt}}
\put(336,357){\usebox{\plotpoint}}
\put(335.67,357){\rule{0.400pt}{0.482pt}}
\multiput(335.17,357.00)(1.000,1.000){2}{\rule{0.400pt}{0.241pt}}
\put(337,358.67){\rule{0.241pt}{0.400pt}}
\multiput(337.00,358.17)(0.500,1.000){2}{\rule{0.120pt}{0.400pt}}
\put(338,361.67){\rule{0.241pt}{0.400pt}}
\multiput(338.00,361.17)(0.500,1.000){2}{\rule{0.120pt}{0.400pt}}
\put(338.0,360.0){\rule[-0.200pt]{0.400pt}{0.482pt}}
\put(339.0,363.0){\usebox{\plotpoint}}
\put(340,363.67){\rule{0.241pt}{0.400pt}}
\multiput(340.00,363.17)(0.500,1.000){2}{\rule{0.120pt}{0.400pt}}
\put(340.67,365){\rule{0.400pt}{0.723pt}}
\multiput(340.17,365.00)(1.000,1.500){2}{\rule{0.400pt}{0.361pt}}
\put(340.0,363.0){\usebox{\plotpoint}}
\put(341.67,369){\rule{0.400pt}{0.482pt}}
\multiput(341.17,369.00)(1.000,1.000){2}{\rule{0.400pt}{0.241pt}}
\put(342.0,368.0){\usebox{\plotpoint}}
\put(343.0,371.0){\usebox{\plotpoint}}
\put(343.67,372){\rule{0.400pt}{0.723pt}}
\multiput(343.17,372.00)(1.000,1.500){2}{\rule{0.400pt}{0.361pt}}
\put(345,374.67){\rule{0.241pt}{0.400pt}}
\multiput(345.00,374.17)(0.500,1.000){2}{\rule{0.120pt}{0.400pt}}
\put(344.0,371.0){\usebox{\plotpoint}}
\put(346,376){\usebox{\plotpoint}}
\put(345.67,376){\rule{0.400pt}{0.482pt}}
\multiput(345.17,376.00)(1.000,1.000){2}{\rule{0.400pt}{0.241pt}}
\put(347,377.67){\rule{0.241pt}{0.400pt}}
\multiput(347.00,377.17)(0.500,1.000){2}{\rule{0.120pt}{0.400pt}}
\put(348,379){\usebox{\plotpoint}}
\put(347.67,379){\rule{0.400pt}{0.723pt}}
\multiput(347.17,379.00)(1.000,1.500){2}{\rule{0.400pt}{0.361pt}}
\put(348.67,382){\rule{0.400pt}{0.723pt}}
\multiput(348.17,382.00)(1.000,1.500){2}{\rule{0.400pt}{0.361pt}}
\put(349.67,384){\rule{0.400pt}{0.482pt}}
\multiput(349.17,384.00)(1.000,1.000){2}{\rule{0.400pt}{0.241pt}}
\put(350.67,386){\rule{0.400pt}{0.723pt}}
\multiput(350.17,386.00)(1.000,1.500){2}{\rule{0.400pt}{0.361pt}}
\put(350.0,384.0){\usebox{\plotpoint}}
\put(351.67,388){\rule{0.400pt}{0.482pt}}
\multiput(351.17,388.00)(1.000,1.000){2}{\rule{0.400pt}{0.241pt}}
\put(352.67,390){\rule{0.400pt}{0.482pt}}
\multiput(352.17,390.00)(1.000,1.000){2}{\rule{0.400pt}{0.241pt}}
\put(352.0,388.0){\usebox{\plotpoint}}
\put(354,393.67){\rule{0.241pt}{0.400pt}}
\multiput(354.00,393.17)(0.500,1.000){2}{\rule{0.120pt}{0.400pt}}
\put(355,393.67){\rule{0.241pt}{0.400pt}}
\multiput(355.00,394.17)(0.500,-1.000){2}{\rule{0.120pt}{0.400pt}}
\put(354.0,392.0){\rule[-0.200pt]{0.400pt}{0.482pt}}
\put(356,398.67){\rule{0.241pt}{0.400pt}}
\multiput(356.00,398.17)(0.500,1.000){2}{\rule{0.120pt}{0.400pt}}
\put(356.0,394.0){\rule[-0.200pt]{0.400pt}{1.204pt}}
\put(357.0,400.0){\usebox{\plotpoint}}
\put(358,402.67){\rule{0.241pt}{0.400pt}}
\multiput(358.00,402.17)(0.500,1.000){2}{\rule{0.120pt}{0.400pt}}
\put(358.0,400.0){\rule[-0.200pt]{0.400pt}{0.723pt}}
\put(359,404){\usebox{\plotpoint}}
\put(358.67,404){\rule{0.400pt}{0.964pt}}
\multiput(358.17,404.00)(1.000,2.000){2}{\rule{0.400pt}{0.482pt}}
\put(360.0,408.0){\usebox{\plotpoint}}
\put(360.67,410){\rule{0.400pt}{0.723pt}}
\multiput(360.17,410.00)(1.000,1.500){2}{\rule{0.400pt}{0.361pt}}
\put(362,411.67){\rule{0.241pt}{0.400pt}}
\multiput(362.00,412.17)(0.500,-1.000){2}{\rule{0.120pt}{0.400pt}}
\put(361.0,408.0){\rule[-0.200pt]{0.400pt}{0.482pt}}
\put(363,413.67){\rule{0.241pt}{0.400pt}}
\multiput(363.00,413.17)(0.500,1.000){2}{\rule{0.120pt}{0.400pt}}
\put(363.67,415){\rule{0.400pt}{0.482pt}}
\multiput(363.17,415.00)(1.000,1.000){2}{\rule{0.400pt}{0.241pt}}
\put(363.0,412.0){\rule[-0.200pt]{0.400pt}{0.482pt}}
\put(365,418.67){\rule{0.241pt}{0.400pt}}
\multiput(365.00,418.17)(0.500,1.000){2}{\rule{0.120pt}{0.400pt}}
\put(366,418.67){\rule{0.241pt}{0.400pt}}
\multiput(366.00,419.17)(0.500,-1.000){2}{\rule{0.120pt}{0.400pt}}
\put(365.0,417.0){\rule[-0.200pt]{0.400pt}{0.482pt}}
\put(366.67,422){\rule{0.400pt}{0.723pt}}
\multiput(366.17,422.00)(1.000,1.500){2}{\rule{0.400pt}{0.361pt}}
\put(367.0,419.0){\rule[-0.200pt]{0.400pt}{0.723pt}}
\put(368.0,425.0){\usebox{\plotpoint}}
\put(368.67,427){\rule{0.400pt}{0.482pt}}
\multiput(368.17,427.00)(1.000,1.000){2}{\rule{0.400pt}{0.241pt}}
\put(369.67,429){\rule{0.400pt}{0.482pt}}
\multiput(369.17,429.00)(1.000,1.000){2}{\rule{0.400pt}{0.241pt}}
\put(369.0,425.0){\rule[-0.200pt]{0.400pt}{0.482pt}}
\put(371,431.67){\rule{0.241pt}{0.400pt}}
\multiput(371.00,432.17)(0.500,-1.000){2}{\rule{0.120pt}{0.400pt}}
\put(371.67,432){\rule{0.400pt}{0.723pt}}
\multiput(371.17,432.00)(1.000,1.500){2}{\rule{0.400pt}{0.361pt}}
\put(371.0,431.0){\rule[-0.200pt]{0.400pt}{0.482pt}}
\put(372.67,437){\rule{0.400pt}{0.723pt}}
\multiput(372.17,437.00)(1.000,1.500){2}{\rule{0.400pt}{0.361pt}}
\put(374,439.67){\rule{0.241pt}{0.400pt}}
\multiput(374.00,439.17)(0.500,1.000){2}{\rule{0.120pt}{0.400pt}}
\put(373.0,435.0){\rule[-0.200pt]{0.400pt}{0.482pt}}
\put(374.67,440){\rule{0.400pt}{0.723pt}}
\multiput(374.17,440.00)(1.000,1.500){2}{\rule{0.400pt}{0.361pt}}
\put(375.67,443){\rule{0.400pt}{0.482pt}}
\multiput(375.17,443.00)(1.000,1.000){2}{\rule{0.400pt}{0.241pt}}
\put(375.0,440.0){\usebox{\plotpoint}}
\put(376.67,446){\rule{0.400pt}{0.482pt}}
\multiput(376.17,446.00)(1.000,1.000){2}{\rule{0.400pt}{0.241pt}}
\put(377.0,445.0){\usebox{\plotpoint}}
\put(378.0,448.0){\usebox{\plotpoint}}
\put(379.0,448.0){\rule[-0.200pt]{0.400pt}{0.723pt}}
\put(379.67,451){\rule{0.400pt}{0.723pt}}
\multiput(379.17,451.00)(1.000,1.500){2}{\rule{0.400pt}{0.361pt}}
\put(379.0,451.0){\usebox{\plotpoint}}
\put(381,456.67){\rule{0.241pt}{0.400pt}}
\multiput(381.00,456.17)(0.500,1.000){2}{\rule{0.120pt}{0.400pt}}
\put(382,457.67){\rule{0.241pt}{0.400pt}}
\multiput(382.00,457.17)(0.500,1.000){2}{\rule{0.120pt}{0.400pt}}
\put(381.0,454.0){\rule[-0.200pt]{0.400pt}{0.723pt}}
\put(383,459.67){\rule{0.241pt}{0.400pt}}
\multiput(383.00,459.17)(0.500,1.000){2}{\rule{0.120pt}{0.400pt}}
\put(383.67,461){\rule{0.400pt}{0.482pt}}
\multiput(383.17,461.00)(1.000,1.000){2}{\rule{0.400pt}{0.241pt}}
\put(383.0,459.0){\usebox{\plotpoint}}
\put(385.0,463.0){\rule[-0.200pt]{0.400pt}{0.482pt}}
\put(385.67,465){\rule{0.400pt}{0.723pt}}
\multiput(385.17,465.00)(1.000,1.500){2}{\rule{0.400pt}{0.361pt}}
\put(385.0,465.0){\usebox{\plotpoint}}
\put(386.67,467){\rule{0.400pt}{0.482pt}}
\multiput(386.17,468.00)(1.000,-1.000){2}{\rule{0.400pt}{0.241pt}}
\put(387.67,467){\rule{0.400pt}{1.445pt}}
\multiput(387.17,467.00)(1.000,3.000){2}{\rule{0.400pt}{0.723pt}}
\put(387.0,468.0){\usebox{\plotpoint}}
\put(389.0,473.0){\usebox{\plotpoint}}
\put(389.67,474){\rule{0.400pt}{0.482pt}}
\multiput(389.17,474.00)(1.000,1.000){2}{\rule{0.400pt}{0.241pt}}
\put(389.0,474.0){\usebox{\plotpoint}}
\put(391,478.67){\rule{0.241pt}{0.400pt}}
\multiput(391.00,478.17)(0.500,1.000){2}{\rule{0.120pt}{0.400pt}}
\put(391.67,480){\rule{0.400pt}{0.723pt}}
\multiput(391.17,480.00)(1.000,1.500){2}{\rule{0.400pt}{0.361pt}}
\put(391.0,476.0){\rule[-0.200pt]{0.400pt}{0.723pt}}
\put(393.0,483.0){\rule[-0.200pt]{0.400pt}{0.723pt}}
\put(393.0,486.0){\rule[-0.200pt]{0.482pt}{0.400pt}}
\put(395,486.67){\rule{0.241pt}{0.400pt}}
\multiput(395.00,487.17)(0.500,-1.000){2}{\rule{0.120pt}{0.400pt}}
\put(395.67,487){\rule{0.400pt}{1.204pt}}
\multiput(395.17,487.00)(1.000,2.500){2}{\rule{0.400pt}{0.602pt}}
\put(395.0,486.0){\rule[-0.200pt]{0.400pt}{0.482pt}}
\put(397,492.67){\rule{0.241pt}{0.400pt}}
\multiput(397.00,492.17)(0.500,1.000){2}{\rule{0.120pt}{0.400pt}}
\put(398,493.67){\rule{0.241pt}{0.400pt}}
\multiput(398.00,493.17)(0.500,1.000){2}{\rule{0.120pt}{0.400pt}}
\put(397.0,492.0){\usebox{\plotpoint}}
\put(399.0,495.0){\rule[-0.200pt]{0.400pt}{0.964pt}}
\put(399.67,499){\rule{0.400pt}{0.723pt}}
\multiput(399.17,499.00)(1.000,1.500){2}{\rule{0.400pt}{0.361pt}}
\put(399.0,499.0){\usebox{\plotpoint}}
\put(401.0,502.0){\rule[-0.200pt]{0.400pt}{0.482pt}}
\put(402,503.67){\rule{0.241pt}{0.400pt}}
\multiput(402.00,503.17)(0.500,1.000){2}{\rule{0.120pt}{0.400pt}}
\put(401.0,504.0){\usebox{\plotpoint}}
\put(402.67,506){\rule{0.400pt}{0.964pt}}
\multiput(402.17,506.00)(1.000,2.000){2}{\rule{0.400pt}{0.482pt}}
\put(403.0,505.0){\usebox{\plotpoint}}
\put(404,510){\usebox{\plotpoint}}
\put(403.67,508){\rule{0.400pt}{0.482pt}}
\multiput(403.17,509.00)(1.000,-1.000){2}{\rule{0.400pt}{0.241pt}}
\put(404.67,508){\rule{0.400pt}{1.686pt}}
\multiput(404.17,508.00)(1.000,3.500){2}{\rule{0.400pt}{0.843pt}}
\put(406,515){\usebox{\plotpoint}}
\put(406,514.67){\rule{0.241pt}{0.400pt}}
\multiput(406.00,514.17)(0.500,1.000){2}{\rule{0.120pt}{0.400pt}}
\put(406.67,516){\rule{0.400pt}{0.482pt}}
\multiput(406.17,516.00)(1.000,1.000){2}{\rule{0.400pt}{0.241pt}}
\put(408,520.67){\rule{0.241pt}{0.400pt}}
\multiput(408.00,520.17)(0.500,1.000){2}{\rule{0.120pt}{0.400pt}}
\put(408.67,522){\rule{0.400pt}{0.723pt}}
\multiput(408.17,522.00)(1.000,1.500){2}{\rule{0.400pt}{0.361pt}}
\put(408.0,518.0){\rule[-0.200pt]{0.400pt}{0.723pt}}
\put(410,526.67){\rule{0.241pt}{0.400pt}}
\multiput(410.00,526.17)(0.500,1.000){2}{\rule{0.120pt}{0.400pt}}
\put(411,527.67){\rule{0.241pt}{0.400pt}}
\multiput(411.00,527.17)(0.500,1.000){2}{\rule{0.120pt}{0.400pt}}
\put(410.0,525.0){\rule[-0.200pt]{0.400pt}{0.482pt}}
\put(412,529){\usebox{\plotpoint}}
\put(411.67,529){\rule{0.400pt}{0.482pt}}
\multiput(411.17,529.00)(1.000,1.000){2}{\rule{0.400pt}{0.241pt}}
\put(412.67,531){\rule{0.400pt}{0.723pt}}
\multiput(412.17,531.00)(1.000,1.500){2}{\rule{0.400pt}{0.361pt}}
\put(413.67,535){\rule{0.400pt}{0.723pt}}
\multiput(413.17,535.00)(1.000,1.500){2}{\rule{0.400pt}{0.361pt}}
\put(414.0,534.0){\usebox{\plotpoint}}
\put(415.0,538.0){\usebox{\plotpoint}}
\put(415.67,540){\rule{0.400pt}{0.482pt}}
\multiput(415.17,540.00)(1.000,1.000){2}{\rule{0.400pt}{0.241pt}}
\put(416.67,540){\rule{0.400pt}{0.482pt}}
\multiput(416.17,541.00)(1.000,-1.000){2}{\rule{0.400pt}{0.241pt}}
\put(416.0,538.0){\rule[-0.200pt]{0.400pt}{0.482pt}}
\put(417.67,542){\rule{0.400pt}{0.964pt}}
\multiput(417.17,542.00)(1.000,2.000){2}{\rule{0.400pt}{0.482pt}}
\put(419,545.67){\rule{0.241pt}{0.400pt}}
\multiput(419.00,545.17)(0.500,1.000){2}{\rule{0.120pt}{0.400pt}}
\put(418.0,540.0){\rule[-0.200pt]{0.400pt}{0.482pt}}
\put(419.67,548){\rule{0.400pt}{0.482pt}}
\multiput(419.17,548.00)(1.000,1.000){2}{\rule{0.400pt}{0.241pt}}
\put(420.67,550){\rule{0.400pt}{0.964pt}}
\multiput(420.17,550.00)(1.000,2.000){2}{\rule{0.400pt}{0.482pt}}
\put(420.0,547.0){\usebox{\plotpoint}}
\put(422.0,554.0){\rule[-0.200pt]{0.400pt}{0.482pt}}
\put(423,555.67){\rule{0.241pt}{0.400pt}}
\multiput(423.00,555.17)(0.500,1.000){2}{\rule{0.120pt}{0.400pt}}
\put(422.0,556.0){\usebox{\plotpoint}}
\put(424,556.67){\rule{0.241pt}{0.400pt}}
\multiput(424.00,557.17)(0.500,-1.000){2}{\rule{0.120pt}{0.400pt}}
\put(424.67,557){\rule{0.400pt}{1.686pt}}
\multiput(424.17,557.00)(1.000,3.500){2}{\rule{0.400pt}{0.843pt}}
\put(424.0,557.0){\usebox{\plotpoint}}
\put(425.67,561){\rule{0.400pt}{0.964pt}}
\multiput(425.17,561.00)(1.000,2.000){2}{\rule{0.400pt}{0.482pt}}
\put(427,564.67){\rule{0.241pt}{0.400pt}}
\multiput(427.00,564.17)(0.500,1.000){2}{\rule{0.120pt}{0.400pt}}
\put(426.0,561.0){\rule[-0.200pt]{0.400pt}{0.723pt}}
\put(427.67,567){\rule{0.400pt}{0.964pt}}
\multiput(427.17,569.00)(1.000,-2.000){2}{\rule{0.400pt}{0.482pt}}
\put(428.67,567){\rule{0.400pt}{1.204pt}}
\multiput(428.17,567.00)(1.000,2.500){2}{\rule{0.400pt}{0.602pt}}
\put(428.0,566.0){\rule[-0.200pt]{0.400pt}{1.204pt}}
\put(429.67,573){\rule{0.400pt}{0.482pt}}
\multiput(429.17,573.00)(1.000,1.000){2}{\rule{0.400pt}{0.241pt}}
\put(430.67,568){\rule{0.400pt}{1.686pt}}
\multiput(430.17,571.50)(1.000,-3.500){2}{\rule{0.400pt}{0.843pt}}
\put(430.0,572.0){\usebox{\plotpoint}}
\put(431.67,577){\rule{0.400pt}{0.723pt}}
\multiput(431.17,577.00)(1.000,1.500){2}{\rule{0.400pt}{0.361pt}}
\put(432.0,568.0){\rule[-0.200pt]{0.400pt}{2.168pt}}
\put(432.67,581){\rule{0.400pt}{0.723pt}}
\multiput(432.17,581.00)(1.000,1.500){2}{\rule{0.400pt}{0.361pt}}
\put(433.67,584){\rule{0.400pt}{0.482pt}}
\multiput(433.17,584.00)(1.000,1.000){2}{\rule{0.400pt}{0.241pt}}
\put(433.0,580.0){\usebox{\plotpoint}}
\put(434.67,587){\rule{0.400pt}{0.723pt}}
\multiput(434.17,588.50)(1.000,-1.500){2}{\rule{0.400pt}{0.361pt}}
\put(435.67,587){\rule{0.400pt}{0.964pt}}
\multiput(435.17,587.00)(1.000,2.000){2}{\rule{0.400pt}{0.482pt}}
\put(435.0,586.0){\rule[-0.200pt]{0.400pt}{0.964pt}}
\put(436.67,588){\rule{0.400pt}{0.964pt}}
\multiput(436.17,588.00)(1.000,2.000){2}{\rule{0.400pt}{0.482pt}}
\put(438,591.67){\rule{0.241pt}{0.400pt}}
\multiput(438.00,591.17)(0.500,1.000){2}{\rule{0.120pt}{0.400pt}}
\put(437.0,588.0){\rule[-0.200pt]{0.400pt}{0.723pt}}
\put(438.67,594){\rule{0.400pt}{0.723pt}}
\multiput(438.17,594.00)(1.000,1.500){2}{\rule{0.400pt}{0.361pt}}
\put(440,595.67){\rule{0.241pt}{0.400pt}}
\multiput(440.00,596.17)(0.500,-1.000){2}{\rule{0.120pt}{0.400pt}}
\put(439.0,593.0){\usebox{\plotpoint}}
\put(440.67,600){\rule{0.400pt}{0.482pt}}
\multiput(440.17,600.00)(1.000,1.000){2}{\rule{0.400pt}{0.241pt}}
\put(442,601.67){\rule{0.241pt}{0.400pt}}
\multiput(442.00,601.17)(0.500,1.000){2}{\rule{0.120pt}{0.400pt}}
\put(441.0,596.0){\rule[-0.200pt]{0.400pt}{0.964pt}}
\put(442.67,602){\rule{0.400pt}{0.723pt}}
\multiput(442.17,602.00)(1.000,1.500){2}{\rule{0.400pt}{0.361pt}}
\put(443.0,602.0){\usebox{\plotpoint}}
\put(444.0,605.0){\usebox{\plotpoint}}
\put(445.0,605.0){\rule[-0.200pt]{0.400pt}{0.964pt}}
\put(445.67,609){\rule{0.400pt}{0.482pt}}
\multiput(445.17,609.00)(1.000,1.000){2}{\rule{0.400pt}{0.241pt}}
\put(445.0,609.0){\usebox{\plotpoint}}
\put(447,611){\usebox{\plotpoint}}
\put(447.67,611){\rule{0.400pt}{0.482pt}}
\multiput(447.17,611.00)(1.000,1.000){2}{\rule{0.400pt}{0.241pt}}
\put(447.0,611.0){\usebox{\plotpoint}}
\put(449,614.67){\rule{0.241pt}{0.400pt}}
\multiput(449.00,614.17)(0.500,1.000){2}{\rule{0.120pt}{0.400pt}}
\put(449.67,616){\rule{0.400pt}{1.204pt}}
\multiput(449.17,616.00)(1.000,2.500){2}{\rule{0.400pt}{0.602pt}}
\put(449.0,613.0){\rule[-0.200pt]{0.400pt}{0.482pt}}
\put(450.67,618){\rule{0.400pt}{0.964pt}}
\multiput(450.17,620.00)(1.000,-2.000){2}{\rule{0.400pt}{0.482pt}}
\put(451.67,618){\rule{0.400pt}{1.686pt}}
\multiput(451.17,618.00)(1.000,3.500){2}{\rule{0.400pt}{0.843pt}}
\put(451.0,621.0){\usebox{\plotpoint}}
\put(453,625){\usebox{\plotpoint}}
\put(453,623.67){\rule{0.241pt}{0.400pt}}
\multiput(453.00,624.17)(0.500,-1.000){2}{\rule{0.120pt}{0.400pt}}
\put(454.0,624.0){\usebox{\plotpoint}}
\put(454.67,626){\rule{0.400pt}{0.723pt}}
\multiput(454.17,626.00)(1.000,1.500){2}{\rule{0.400pt}{0.361pt}}
\put(455.67,627){\rule{0.400pt}{0.482pt}}
\multiput(455.17,628.00)(1.000,-1.000){2}{\rule{0.400pt}{0.241pt}}
\put(455.0,624.0){\rule[-0.200pt]{0.400pt}{0.482pt}}
\put(457,631.67){\rule{0.241pt}{0.400pt}}
\multiput(457.00,632.17)(0.500,-1.000){2}{\rule{0.120pt}{0.400pt}}
\put(457.0,627.0){\rule[-0.200pt]{0.400pt}{1.445pt}}
\put(458,632){\usebox{\plotpoint}}
\put(457.67,630){\rule{0.400pt}{0.482pt}}
\multiput(457.17,631.00)(1.000,-1.000){2}{\rule{0.400pt}{0.241pt}}
\put(458.67,630){\rule{0.400pt}{1.445pt}}
\multiput(458.17,630.00)(1.000,3.000){2}{\rule{0.400pt}{0.723pt}}
\put(460,636){\usebox{\plotpoint}}
\put(460,634.67){\rule{0.241pt}{0.400pt}}
\multiput(460.00,635.17)(0.500,-1.000){2}{\rule{0.120pt}{0.400pt}}
\put(460.67,635){\rule{0.400pt}{1.445pt}}
\multiput(460.17,635.00)(1.000,3.000){2}{\rule{0.400pt}{0.723pt}}
\put(462,641){\usebox{\plotpoint}}
\put(461.67,638){\rule{0.400pt}{0.723pt}}
\multiput(461.17,639.50)(1.000,-1.500){2}{\rule{0.400pt}{0.361pt}}
\put(462.67,638){\rule{0.400pt}{1.204pt}}
\multiput(462.17,638.00)(1.000,2.500){2}{\rule{0.400pt}{0.602pt}}
\put(463.67,638){\rule{0.400pt}{1.927pt}}
\multiput(463.17,642.00)(1.000,-4.000){2}{\rule{0.400pt}{0.964pt}}
\put(464.67,638){\rule{0.400pt}{1.686pt}}
\multiput(464.17,638.00)(1.000,3.500){2}{\rule{0.400pt}{0.843pt}}
\put(464.0,643.0){\rule[-0.200pt]{0.400pt}{0.723pt}}
\put(466,642.67){\rule{0.241pt}{0.400pt}}
\multiput(466.00,643.17)(0.500,-1.000){2}{\rule{0.120pt}{0.400pt}}
\put(466.67,643){\rule{0.400pt}{1.927pt}}
\multiput(466.17,643.00)(1.000,4.000){2}{\rule{0.400pt}{0.964pt}}
\put(466.0,644.0){\usebox{\plotpoint}}
\put(467.67,645){\rule{0.400pt}{0.964pt}}
\multiput(467.17,647.00)(1.000,-2.000){2}{\rule{0.400pt}{0.482pt}}
\put(468.67,645){\rule{0.400pt}{0.482pt}}
\multiput(468.17,645.00)(1.000,1.000){2}{\rule{0.400pt}{0.241pt}}
\put(468.0,649.0){\rule[-0.200pt]{0.400pt}{0.482pt}}
\put(469.67,645){\rule{0.400pt}{0.964pt}}
\multiput(469.17,647.00)(1.000,-2.000){2}{\rule{0.400pt}{0.482pt}}
\put(470.67,645){\rule{0.400pt}{1.686pt}}
\multiput(470.17,645.00)(1.000,3.500){2}{\rule{0.400pt}{0.843pt}}
\put(470.0,647.0){\rule[-0.200pt]{0.400pt}{0.482pt}}
\put(471.67,655){\rule{0.400pt}{0.482pt}}
\multiput(471.17,655.00)(1.000,1.000){2}{\rule{0.400pt}{0.241pt}}
\put(472.67,654){\rule{0.400pt}{0.723pt}}
\multiput(472.17,655.50)(1.000,-1.500){2}{\rule{0.400pt}{0.361pt}}
\put(472.0,652.0){\rule[-0.200pt]{0.400pt}{0.723pt}}
\put(473.67,653){\rule{0.400pt}{1.204pt}}
\multiput(473.17,653.00)(1.000,2.500){2}{\rule{0.400pt}{0.602pt}}
\put(474.67,653){\rule{0.400pt}{1.204pt}}
\multiput(474.17,655.50)(1.000,-2.500){2}{\rule{0.400pt}{0.602pt}}
\put(474.0,653.0){\usebox{\plotpoint}}
\put(476.0,653.0){\rule[-0.200pt]{0.400pt}{0.723pt}}
\put(477,655.67){\rule{0.241pt}{0.400pt}}
\multiput(477.00,655.17)(0.500,1.000){2}{\rule{0.120pt}{0.400pt}}
\put(476.0,656.0){\usebox{\plotpoint}}
\put(477.67,656){\rule{0.400pt}{0.723pt}}
\multiput(477.17,657.50)(1.000,-1.500){2}{\rule{0.400pt}{0.361pt}}
\put(478.0,657.0){\rule[-0.200pt]{0.400pt}{0.482pt}}
\put(478.67,659){\rule{0.400pt}{1.686pt}}
\multiput(478.17,662.50)(1.000,-3.500){2}{\rule{0.400pt}{0.843pt}}
\put(479.67,655){\rule{0.400pt}{0.964pt}}
\multiput(479.17,657.00)(1.000,-2.000){2}{\rule{0.400pt}{0.482pt}}
\put(479.0,656.0){\rule[-0.200pt]{0.400pt}{2.409pt}}
\put(480.67,661){\rule{0.400pt}{0.482pt}}
\multiput(480.17,661.00)(1.000,1.000){2}{\rule{0.400pt}{0.241pt}}
\put(482,661.67){\rule{0.241pt}{0.400pt}}
\multiput(482.00,662.17)(0.500,-1.000){2}{\rule{0.120pt}{0.400pt}}
\put(481.0,655.0){\rule[-0.200pt]{0.400pt}{1.445pt}}
\put(483,659.67){\rule{0.241pt}{0.400pt}}
\multiput(483.00,660.17)(0.500,-1.000){2}{\rule{0.120pt}{0.400pt}}
\put(484,659.67){\rule{0.241pt}{0.400pt}}
\multiput(484.00,659.17)(0.500,1.000){2}{\rule{0.120pt}{0.400pt}}
\put(483.0,661.0){\usebox{\plotpoint}}
\put(484.67,659){\rule{0.400pt}{0.482pt}}
\multiput(484.17,659.00)(1.000,1.000){2}{\rule{0.400pt}{0.241pt}}
\put(486,660.67){\rule{0.241pt}{0.400pt}}
\multiput(486.00,660.17)(0.500,1.000){2}{\rule{0.120pt}{0.400pt}}
\put(485.0,659.0){\rule[-0.200pt]{0.400pt}{0.482pt}}
\put(486.67,657){\rule{0.400pt}{1.927pt}}
\multiput(486.17,661.00)(1.000,-4.000){2}{\rule{0.400pt}{0.964pt}}
\put(487.67,650){\rule{0.400pt}{1.686pt}}
\multiput(487.17,653.50)(1.000,-3.500){2}{\rule{0.400pt}{0.843pt}}
\put(487.0,662.0){\rule[-0.200pt]{0.400pt}{0.723pt}}
\put(488.67,663){\rule{0.400pt}{0.482pt}}
\multiput(488.17,664.00)(1.000,-1.000){2}{\rule{0.400pt}{0.241pt}}
\put(490,662.67){\rule{0.241pt}{0.400pt}}
\multiput(490.00,662.17)(0.500,1.000){2}{\rule{0.120pt}{0.400pt}}
\put(489.0,650.0){\rule[-0.200pt]{0.400pt}{3.613pt}}
\put(490.67,658){\rule{0.400pt}{1.927pt}}
\multiput(490.17,662.00)(1.000,-4.000){2}{\rule{0.400pt}{0.964pt}}
\put(491.67,658){\rule{0.400pt}{0.482pt}}
\multiput(491.17,658.00)(1.000,1.000){2}{\rule{0.400pt}{0.241pt}}
\put(491.0,664.0){\rule[-0.200pt]{0.400pt}{0.482pt}}
\put(492.67,659){\rule{0.400pt}{2.650pt}}
\multiput(492.17,659.00)(1.000,5.500){2}{\rule{0.400pt}{1.325pt}}
\put(493.67,665){\rule{0.400pt}{1.204pt}}
\multiput(493.17,667.50)(1.000,-2.500){2}{\rule{0.400pt}{0.602pt}}
\put(493.0,659.0){\usebox{\plotpoint}}
\put(494.67,661){\rule{0.400pt}{0.482pt}}
\multiput(494.17,661.00)(1.000,1.000){2}{\rule{0.400pt}{0.241pt}}
\put(496,661.67){\rule{0.241pt}{0.400pt}}
\multiput(496.00,662.17)(0.500,-1.000){2}{\rule{0.120pt}{0.400pt}}
\put(495.0,661.0){\rule[-0.200pt]{0.400pt}{0.964pt}}
\put(497,662){\usebox{\plotpoint}}
\put(496.67,662){\rule{0.400pt}{0.482pt}}
\multiput(496.17,662.00)(1.000,1.000){2}{\rule{0.400pt}{0.241pt}}
\put(497.67,660){\rule{0.400pt}{1.927pt}}
\multiput(497.17,664.00)(1.000,-4.000){2}{\rule{0.400pt}{0.964pt}}
\put(498.67,657){\rule{0.400pt}{0.723pt}}
\multiput(498.17,658.50)(1.000,-1.500){2}{\rule{0.400pt}{0.361pt}}
\put(498.0,664.0){\rule[-0.200pt]{0.400pt}{0.964pt}}
\put(499.67,661){\rule{0.400pt}{0.964pt}}
\multiput(499.17,661.00)(1.000,2.000){2}{\rule{0.400pt}{0.482pt}}
\put(500.67,661){\rule{0.400pt}{0.964pt}}
\multiput(500.17,663.00)(1.000,-2.000){2}{\rule{0.400pt}{0.482pt}}
\put(500.0,657.0){\rule[-0.200pt]{0.400pt}{0.964pt}}
\put(501.67,656){\rule{0.400pt}{1.204pt}}
\multiput(501.17,656.00)(1.000,2.500){2}{\rule{0.400pt}{0.602pt}}
\put(502.0,656.0){\rule[-0.200pt]{0.400pt}{1.204pt}}
\put(503.0,661.0){\usebox{\plotpoint}}
\put(503.67,660){\rule{0.400pt}{1.927pt}}
\multiput(503.17,664.00)(1.000,-4.000){2}{\rule{0.400pt}{0.964pt}}
\put(504.67,660){\rule{0.400pt}{1.445pt}}
\multiput(504.17,660.00)(1.000,3.000){2}{\rule{0.400pt}{0.723pt}}
\put(504.0,661.0){\rule[-0.200pt]{0.400pt}{1.686pt}}
\put(505.67,661){\rule{0.400pt}{0.964pt}}
\multiput(505.17,661.00)(1.000,2.000){2}{\rule{0.400pt}{0.482pt}}
\put(506.67,662){\rule{0.400pt}{0.723pt}}
\multiput(506.17,663.50)(1.000,-1.500){2}{\rule{0.400pt}{0.361pt}}
\put(506.0,661.0){\rule[-0.200pt]{0.400pt}{1.204pt}}
\put(507.67,663){\rule{0.400pt}{1.445pt}}
\multiput(507.17,666.00)(1.000,-3.000){2}{\rule{0.400pt}{0.723pt}}
\put(509,662.67){\rule{0.241pt}{0.400pt}}
\multiput(509.00,662.17)(0.500,1.000){2}{\rule{0.120pt}{0.400pt}}
\put(508.0,662.0){\rule[-0.200pt]{0.400pt}{1.686pt}}
\put(509.67,659){\rule{0.400pt}{1.445pt}}
\multiput(509.17,659.00)(1.000,3.000){2}{\rule{0.400pt}{0.723pt}}
\put(510.67,655){\rule{0.400pt}{2.409pt}}
\multiput(510.17,660.00)(1.000,-5.000){2}{\rule{0.400pt}{1.204pt}}
\put(510.0,659.0){\rule[-0.200pt]{0.400pt}{1.204pt}}
\put(511.67,660){\rule{0.400pt}{0.482pt}}
\multiput(511.17,660.00)(1.000,1.000){2}{\rule{0.400pt}{0.241pt}}
\put(512.67,662){\rule{0.400pt}{0.723pt}}
\multiput(512.17,662.00)(1.000,1.500){2}{\rule{0.400pt}{0.361pt}}
\put(512.0,655.0){\rule[-0.200pt]{0.400pt}{1.204pt}}
\put(513.67,657){\rule{0.400pt}{1.204pt}}
\multiput(513.17,657.00)(1.000,2.500){2}{\rule{0.400pt}{0.602pt}}
\put(514.0,657.0){\rule[-0.200pt]{0.400pt}{1.927pt}}
\put(514.67,655){\rule{0.400pt}{1.927pt}}
\multiput(514.17,659.00)(1.000,-4.000){2}{\rule{0.400pt}{0.964pt}}
\put(515.67,653){\rule{0.400pt}{0.482pt}}
\multiput(515.17,654.00)(1.000,-1.000){2}{\rule{0.400pt}{0.241pt}}
\put(515.0,662.0){\usebox{\plotpoint}}
\put(516.67,653){\rule{0.400pt}{0.482pt}}
\multiput(516.17,654.00)(1.000,-1.000){2}{\rule{0.400pt}{0.241pt}}
\put(517.67,651){\rule{0.400pt}{0.482pt}}
\multiput(517.17,652.00)(1.000,-1.000){2}{\rule{0.400pt}{0.241pt}}
\put(517.0,653.0){\rule[-0.200pt]{0.400pt}{0.482pt}}
\put(518.67,649){\rule{0.400pt}{2.168pt}}
\multiput(518.17,653.50)(1.000,-4.500){2}{\rule{0.400pt}{1.084pt}}
\put(519.67,649){\rule{0.400pt}{1.204pt}}
\multiput(519.17,649.00)(1.000,2.500){2}{\rule{0.400pt}{0.602pt}}
\put(519.0,651.0){\rule[-0.200pt]{0.400pt}{1.686pt}}
\put(521.0,654.0){\rule[-0.200pt]{0.400pt}{1.204pt}}
\put(521.67,650){\rule{0.400pt}{2.168pt}}
\multiput(521.17,654.50)(1.000,-4.500){2}{\rule{0.400pt}{1.084pt}}
\put(521.0,659.0){\usebox{\plotpoint}}
\put(523,650){\usebox{\plotpoint}}
\put(522.67,650){\rule{0.400pt}{0.723pt}}
\multiput(522.17,650.00)(1.000,1.500){2}{\rule{0.400pt}{0.361pt}}
\put(523.67,651){\rule{0.400pt}{0.482pt}}
\multiput(523.17,652.00)(1.000,-1.000){2}{\rule{0.400pt}{0.241pt}}
\put(525.0,649.0){\rule[-0.200pt]{0.400pt}{0.482pt}}
\put(526,648.67){\rule{0.241pt}{0.400pt}}
\multiput(526.00,648.17)(0.500,1.000){2}{\rule{0.120pt}{0.400pt}}
\put(525.0,649.0){\usebox{\plotpoint}}
\put(526.67,648){\rule{0.400pt}{1.204pt}}
\multiput(526.17,648.00)(1.000,2.500){2}{\rule{0.400pt}{0.602pt}}
\put(527.67,651){\rule{0.400pt}{0.482pt}}
\multiput(527.17,652.00)(1.000,-1.000){2}{\rule{0.400pt}{0.241pt}}
\put(527.0,648.0){\rule[-0.200pt]{0.400pt}{0.482pt}}
\put(528.67,643){\rule{0.400pt}{0.482pt}}
\multiput(528.17,643.00)(1.000,1.000){2}{\rule{0.400pt}{0.241pt}}
\put(529.0,643.0){\rule[-0.200pt]{0.400pt}{1.927pt}}
\put(529.67,646){\rule{0.400pt}{1.927pt}}
\multiput(529.17,646.00)(1.000,4.000){2}{\rule{0.400pt}{0.964pt}}
\put(530.67,644){\rule{0.400pt}{2.409pt}}
\multiput(530.17,649.00)(1.000,-5.000){2}{\rule{0.400pt}{1.204pt}}
\put(530.0,645.0){\usebox{\plotpoint}}
\put(531.67,642){\rule{0.400pt}{1.204pt}}
\multiput(531.17,642.00)(1.000,2.500){2}{\rule{0.400pt}{0.602pt}}
\put(532.67,640){\rule{0.400pt}{1.686pt}}
\multiput(532.17,643.50)(1.000,-3.500){2}{\rule{0.400pt}{0.843pt}}
\put(532.0,642.0){\rule[-0.200pt]{0.400pt}{0.482pt}}
\put(534,641.67){\rule{0.241pt}{0.400pt}}
\multiput(534.00,642.17)(0.500,-1.000){2}{\rule{0.120pt}{0.400pt}}
\put(535,640.67){\rule{0.241pt}{0.400pt}}
\multiput(535.00,641.17)(0.500,-1.000){2}{\rule{0.120pt}{0.400pt}}
\put(534.0,640.0){\rule[-0.200pt]{0.400pt}{0.723pt}}
\put(535.67,640){\rule{0.400pt}{2.409pt}}
\multiput(535.17,640.00)(1.000,5.000){2}{\rule{0.400pt}{1.204pt}}
\put(536.67,642){\rule{0.400pt}{1.927pt}}
\multiput(536.17,646.00)(1.000,-4.000){2}{\rule{0.400pt}{0.964pt}}
\put(536.0,640.0){\usebox{\plotpoint}}
\put(537.67,635){\rule{0.400pt}{1.445pt}}
\multiput(537.17,638.00)(1.000,-3.000){2}{\rule{0.400pt}{0.723pt}}
\put(538.67,635){\rule{0.400pt}{0.964pt}}
\multiput(538.17,635.00)(1.000,2.000){2}{\rule{0.400pt}{0.482pt}}
\put(538.0,641.0){\usebox{\plotpoint}}
\put(540,635.67){\rule{0.241pt}{0.400pt}}
\multiput(540.00,636.17)(0.500,-1.000){2}{\rule{0.120pt}{0.400pt}}
\put(540.67,631){\rule{0.400pt}{1.204pt}}
\multiput(540.17,633.50)(1.000,-2.500){2}{\rule{0.400pt}{0.602pt}}
\put(540.0,637.0){\rule[-0.200pt]{0.400pt}{0.482pt}}
\put(541.67,636){\rule{0.400pt}{1.445pt}}
\multiput(541.17,639.00)(1.000,-3.000){2}{\rule{0.400pt}{0.723pt}}
\put(542.0,631.0){\rule[-0.200pt]{0.400pt}{2.650pt}}
\put(543.0,636.0){\usebox{\plotpoint}}
\put(543.67,635){\rule{0.400pt}{1.927pt}}
\multiput(543.17,635.00)(1.000,4.000){2}{\rule{0.400pt}{0.964pt}}
\put(544.0,635.0){\usebox{\plotpoint}}
\put(544.67,630){\rule{0.400pt}{1.204pt}}
\multiput(544.17,632.50)(1.000,-2.500){2}{\rule{0.400pt}{0.602pt}}
\put(545.67,630){\rule{0.400pt}{0.964pt}}
\multiput(545.17,630.00)(1.000,2.000){2}{\rule{0.400pt}{0.482pt}}
\put(545.0,635.0){\rule[-0.200pt]{0.400pt}{1.927pt}}
\put(547,634.67){\rule{0.241pt}{0.400pt}}
\multiput(547.00,635.17)(0.500,-1.000){2}{\rule{0.120pt}{0.400pt}}
\put(548,633.67){\rule{0.241pt}{0.400pt}}
\multiput(548.00,634.17)(0.500,-1.000){2}{\rule{0.120pt}{0.400pt}}
\put(547.0,634.0){\rule[-0.200pt]{0.400pt}{0.482pt}}
\put(548.67,630){\rule{0.400pt}{2.168pt}}
\multiput(548.17,634.50)(1.000,-4.500){2}{\rule{0.400pt}{1.084pt}}
\put(549.67,630){\rule{0.400pt}{0.723pt}}
\multiput(549.17,630.00)(1.000,1.500){2}{\rule{0.400pt}{0.361pt}}
\put(549.0,634.0){\rule[-0.200pt]{0.400pt}{1.204pt}}
\put(550.67,629){\rule{0.400pt}{0.482pt}}
\multiput(550.17,629.00)(1.000,1.000){2}{\rule{0.400pt}{0.241pt}}
\put(551.0,629.0){\rule[-0.200pt]{0.400pt}{0.964pt}}
\put(552.0,631.0){\usebox{\plotpoint}}
\put(553,626.67){\rule{0.241pt}{0.400pt}}
\multiput(553.00,627.17)(0.500,-1.000){2}{\rule{0.120pt}{0.400pt}}
\put(553.67,621){\rule{0.400pt}{1.445pt}}
\multiput(553.17,624.00)(1.000,-3.000){2}{\rule{0.400pt}{0.723pt}}
\put(553.0,628.0){\rule[-0.200pt]{0.400pt}{0.723pt}}
\put(554.67,624){\rule{0.400pt}{0.482pt}}
\multiput(554.17,624.00)(1.000,1.000){2}{\rule{0.400pt}{0.241pt}}
\put(555.67,623){\rule{0.400pt}{0.723pt}}
\multiput(555.17,624.50)(1.000,-1.500){2}{\rule{0.400pt}{0.361pt}}
\put(555.0,621.0){\rule[-0.200pt]{0.400pt}{0.723pt}}
\put(557,624.67){\rule{0.241pt}{0.400pt}}
\multiput(557.00,625.17)(0.500,-1.000){2}{\rule{0.120pt}{0.400pt}}
\put(558,623.67){\rule{0.241pt}{0.400pt}}
\multiput(558.00,624.17)(0.500,-1.000){2}{\rule{0.120pt}{0.400pt}}
\put(557.0,623.0){\rule[-0.200pt]{0.400pt}{0.723pt}}
\put(559,619.67){\rule{0.241pt}{0.400pt}}
\multiput(559.00,619.17)(0.500,1.000){2}{\rule{0.120pt}{0.400pt}}
\put(559.0,620.0){\rule[-0.200pt]{0.400pt}{0.964pt}}
\put(560.0,620.0){\usebox{\plotpoint}}
\put(561,619.67){\rule{0.241pt}{0.400pt}}
\multiput(561.00,619.17)(0.500,1.000){2}{\rule{0.120pt}{0.400pt}}
\put(560.0,620.0){\usebox{\plotpoint}}
\put(562,617.67){\rule{0.241pt}{0.400pt}}
\multiput(562.00,618.17)(0.500,-1.000){2}{\rule{0.120pt}{0.400pt}}
\put(562.67,618){\rule{0.400pt}{1.204pt}}
\multiput(562.17,618.00)(1.000,2.500){2}{\rule{0.400pt}{0.602pt}}
\put(562.0,619.0){\rule[-0.200pt]{0.400pt}{0.482pt}}
\put(563.67,610){\rule{0.400pt}{2.168pt}}
\multiput(563.17,610.00)(1.000,4.500){2}{\rule{0.400pt}{1.084pt}}
\put(564.67,616){\rule{0.400pt}{0.723pt}}
\multiput(564.17,617.50)(1.000,-1.500){2}{\rule{0.400pt}{0.361pt}}
\put(564.0,610.0){\rule[-0.200pt]{0.400pt}{3.132pt}}
\put(566,616){\usebox{\plotpoint}}
\put(566.67,614){\rule{0.400pt}{0.482pt}}
\multiput(566.17,615.00)(1.000,-1.000){2}{\rule{0.400pt}{0.241pt}}
\put(566.0,616.0){\usebox{\plotpoint}}
\put(567.67,612){\rule{0.400pt}{0.723pt}}
\multiput(567.17,612.00)(1.000,1.500){2}{\rule{0.400pt}{0.361pt}}
\put(568.67,608){\rule{0.400pt}{1.686pt}}
\multiput(568.17,611.50)(1.000,-3.500){2}{\rule{0.400pt}{0.843pt}}
\put(568.0,612.0){\rule[-0.200pt]{0.400pt}{0.482pt}}
\put(570.0,608.0){\rule[-0.200pt]{0.400pt}{0.964pt}}
\put(570.67,603){\rule{0.400pt}{2.168pt}}
\multiput(570.17,607.50)(1.000,-4.500){2}{\rule{0.400pt}{1.084pt}}
\put(570.0,612.0){\usebox{\plotpoint}}
\put(572,608.67){\rule{0.241pt}{0.400pt}}
\multiput(572.00,609.17)(0.500,-1.000){2}{\rule{0.120pt}{0.400pt}}
\put(572.0,603.0){\rule[-0.200pt]{0.400pt}{1.686pt}}
\put(573,610.67){\rule{0.241pt}{0.400pt}}
\multiput(573.00,610.17)(0.500,1.000){2}{\rule{0.120pt}{0.400pt}}
\put(573.67,603){\rule{0.400pt}{2.168pt}}
\multiput(573.17,607.50)(1.000,-4.500){2}{\rule{0.400pt}{1.084pt}}
\put(573.0,609.0){\rule[-0.200pt]{0.400pt}{0.482pt}}
\put(574.67,605){\rule{0.400pt}{0.964pt}}
\multiput(574.17,607.00)(1.000,-2.000){2}{\rule{0.400pt}{0.482pt}}
\put(575.67,605){\rule{0.400pt}{1.204pt}}
\multiput(575.17,605.00)(1.000,2.500){2}{\rule{0.400pt}{0.602pt}}
\put(575.0,603.0){\rule[-0.200pt]{0.400pt}{1.445pt}}
\put(576.67,601){\rule{0.400pt}{0.482pt}}
\multiput(576.17,602.00)(1.000,-1.000){2}{\rule{0.400pt}{0.241pt}}
\put(577.67,596){\rule{0.400pt}{1.204pt}}
\multiput(577.17,598.50)(1.000,-2.500){2}{\rule{0.400pt}{0.602pt}}
\put(577.0,603.0){\rule[-0.200pt]{0.400pt}{1.686pt}}
\put(578.67,598){\rule{0.400pt}{1.204pt}}
\multiput(578.17,598.00)(1.000,2.500){2}{\rule{0.400pt}{0.602pt}}
\put(579.67,596){\rule{0.400pt}{1.686pt}}
\multiput(579.17,599.50)(1.000,-3.500){2}{\rule{0.400pt}{0.843pt}}
\put(579.0,596.0){\rule[-0.200pt]{0.400pt}{0.482pt}}
\put(580.67,592){\rule{0.400pt}{1.686pt}}
\multiput(580.17,592.00)(1.000,3.500){2}{\rule{0.400pt}{0.843pt}}
\put(581.67,599){\rule{0.400pt}{0.723pt}}
\multiput(581.17,599.00)(1.000,1.500){2}{\rule{0.400pt}{0.361pt}}
\put(581.0,592.0){\rule[-0.200pt]{0.400pt}{0.964pt}}
\put(582.67,595){\rule{0.400pt}{0.964pt}}
\multiput(582.17,595.00)(1.000,2.000){2}{\rule{0.400pt}{0.482pt}}
\put(583.0,595.0){\rule[-0.200pt]{0.400pt}{1.686pt}}
\put(583.67,592){\rule{0.400pt}{0.482pt}}
\multiput(583.17,592.00)(1.000,1.000){2}{\rule{0.400pt}{0.241pt}}
\put(585,592.67){\rule{0.241pt}{0.400pt}}
\multiput(585.00,593.17)(0.500,-1.000){2}{\rule{0.120pt}{0.400pt}}
\put(584.0,592.0){\rule[-0.200pt]{0.400pt}{1.686pt}}
\put(585.67,588){\rule{0.400pt}{0.723pt}}
\multiput(585.17,588.00)(1.000,1.500){2}{\rule{0.400pt}{0.361pt}}
\put(586.0,588.0){\rule[-0.200pt]{0.400pt}{1.204pt}}
\put(587.0,591.0){\usebox{\plotpoint}}
\put(587.67,585){\rule{0.400pt}{1.686pt}}
\multiput(587.17,585.00)(1.000,3.500){2}{\rule{0.400pt}{0.843pt}}
\put(588.67,585){\rule{0.400pt}{1.686pt}}
\multiput(588.17,588.50)(1.000,-3.500){2}{\rule{0.400pt}{0.843pt}}
\put(588.0,585.0){\rule[-0.200pt]{0.400pt}{1.445pt}}
\put(589.67,580){\rule{0.400pt}{0.723pt}}
\multiput(589.17,581.50)(1.000,-1.500){2}{\rule{0.400pt}{0.361pt}}
\put(590.67,580){\rule{0.400pt}{0.964pt}}
\multiput(590.17,580.00)(1.000,2.000){2}{\rule{0.400pt}{0.482pt}}
\put(590.0,583.0){\rule[-0.200pt]{0.400pt}{0.482pt}}
\put(591.67,580){\rule{0.400pt}{1.445pt}}
\multiput(591.17,583.00)(1.000,-3.000){2}{\rule{0.400pt}{0.723pt}}
\put(592.67,578){\rule{0.400pt}{0.482pt}}
\multiput(592.17,579.00)(1.000,-1.000){2}{\rule{0.400pt}{0.241pt}}
\put(592.0,584.0){\rule[-0.200pt]{0.400pt}{0.482pt}}
\put(593.67,577){\rule{0.400pt}{0.723pt}}
\multiput(593.17,577.00)(1.000,1.500){2}{\rule{0.400pt}{0.361pt}}
\put(594.67,573){\rule{0.400pt}{1.686pt}}
\multiput(594.17,576.50)(1.000,-3.500){2}{\rule{0.400pt}{0.843pt}}
\put(594.0,577.0){\usebox{\plotpoint}}
\put(596,573){\usebox{\plotpoint}}
\put(595.67,573){\rule{0.400pt}{0.482pt}}
\multiput(595.17,573.00)(1.000,1.000){2}{\rule{0.400pt}{0.241pt}}
\put(596.67,574){\rule{0.400pt}{0.482pt}}
\multiput(596.17,575.00)(1.000,-1.000){2}{\rule{0.400pt}{0.241pt}}
\put(597.67,572){\rule{0.400pt}{0.482pt}}
\multiput(597.17,573.00)(1.000,-1.000){2}{\rule{0.400pt}{0.241pt}}
\put(597.0,575.0){\usebox{\plotpoint}}
\put(598.67,557){\rule{0.400pt}{2.409pt}}
\multiput(598.17,562.00)(1.000,-5.000){2}{\rule{0.400pt}{1.204pt}}
\put(599.67,557){\rule{0.400pt}{3.614pt}}
\multiput(599.17,557.00)(1.000,7.500){2}{\rule{0.400pt}{1.807pt}}
\put(599.0,567.0){\rule[-0.200pt]{0.400pt}{1.204pt}}
\put(600.67,560){\rule{0.400pt}{0.964pt}}
\multiput(600.17,562.00)(1.000,-2.000){2}{\rule{0.400pt}{0.482pt}}
\put(602,558.67){\rule{0.241pt}{0.400pt}}
\multiput(602.00,559.17)(0.500,-1.000){2}{\rule{0.120pt}{0.400pt}}
\put(601.0,564.0){\rule[-0.200pt]{0.400pt}{1.927pt}}
\put(602.67,563){\rule{0.400pt}{0.723pt}}
\multiput(602.17,564.50)(1.000,-1.500){2}{\rule{0.400pt}{0.361pt}}
\put(603.67,555){\rule{0.400pt}{1.927pt}}
\multiput(603.17,559.00)(1.000,-4.000){2}{\rule{0.400pt}{0.964pt}}
\put(603.0,559.0){\rule[-0.200pt]{0.400pt}{1.686pt}}
\put(604.67,554){\rule{0.400pt}{0.964pt}}
\multiput(604.17,556.00)(1.000,-2.000){2}{\rule{0.400pt}{0.482pt}}
\put(605.67,550){\rule{0.400pt}{0.964pt}}
\multiput(605.17,552.00)(1.000,-2.000){2}{\rule{0.400pt}{0.482pt}}
\put(605.0,555.0){\rule[-0.200pt]{0.400pt}{0.723pt}}
\put(606.67,549){\rule{0.400pt}{1.445pt}}
\multiput(606.17,552.00)(1.000,-3.000){2}{\rule{0.400pt}{0.723pt}}
\put(607.0,550.0){\rule[-0.200pt]{0.400pt}{1.204pt}}
\put(607.67,547){\rule{0.400pt}{1.204pt}}
\multiput(607.17,549.50)(1.000,-2.500){2}{\rule{0.400pt}{0.602pt}}
\put(609,546.67){\rule{0.241pt}{0.400pt}}
\multiput(609.00,546.17)(0.500,1.000){2}{\rule{0.120pt}{0.400pt}}
\put(608.0,549.0){\rule[-0.200pt]{0.400pt}{0.723pt}}
\put(609.67,543){\rule{0.400pt}{1.204pt}}
\multiput(609.17,543.00)(1.000,2.500){2}{\rule{0.400pt}{0.602pt}}
\put(610.67,544){\rule{0.400pt}{0.964pt}}
\multiput(610.17,546.00)(1.000,-2.000){2}{\rule{0.400pt}{0.482pt}}
\put(610.0,543.0){\rule[-0.200pt]{0.400pt}{1.204pt}}
\put(611.67,538){\rule{0.400pt}{0.964pt}}
\multiput(611.17,540.00)(1.000,-2.000){2}{\rule{0.400pt}{0.482pt}}
\put(612.0,542.0){\rule[-0.200pt]{0.400pt}{0.482pt}}
\put(613.67,529){\rule{0.400pt}{2.168pt}}
\multiput(613.17,533.50)(1.000,-4.500){2}{\rule{0.400pt}{1.084pt}}
\put(614.67,529){\rule{0.400pt}{0.964pt}}
\multiput(614.17,529.00)(1.000,2.000){2}{\rule{0.400pt}{0.482pt}}
\put(613.0,538.0){\usebox{\plotpoint}}
\put(616,533){\usebox{\plotpoint}}
\put(616,532.67){\rule{0.241pt}{0.400pt}}
\multiput(616.00,532.17)(0.500,1.000){2}{\rule{0.120pt}{0.400pt}}
\put(616.67,528){\rule{0.400pt}{1.445pt}}
\multiput(616.17,531.00)(1.000,-3.000){2}{\rule{0.400pt}{0.723pt}}
\put(617.67,523){\rule{0.400pt}{0.964pt}}
\multiput(617.17,525.00)(1.000,-2.000){2}{\rule{0.400pt}{0.482pt}}
\put(618.0,527.0){\usebox{\plotpoint}}
\put(619,520.67){\rule{0.241pt}{0.400pt}}
\multiput(619.00,521.17)(0.500,-1.000){2}{\rule{0.120pt}{0.400pt}}
\put(619.67,518){\rule{0.400pt}{0.723pt}}
\multiput(619.17,519.50)(1.000,-1.500){2}{\rule{0.400pt}{0.361pt}}
\put(619.0,522.0){\usebox{\plotpoint}}
\put(620.67,515){\rule{0.400pt}{0.964pt}}
\multiput(620.17,517.00)(1.000,-2.000){2}{\rule{0.400pt}{0.482pt}}
\put(621.0,518.0){\usebox{\plotpoint}}
\put(622.0,515.0){\usebox{\plotpoint}}
\put(623.0,512.0){\rule[-0.200pt]{0.400pt}{0.723pt}}
\put(623.67,509){\rule{0.400pt}{0.723pt}}
\multiput(623.17,510.50)(1.000,-1.500){2}{\rule{0.400pt}{0.361pt}}
\put(623.0,512.0){\usebox{\plotpoint}}
\put(625,509){\usebox{\plotpoint}}
\put(624.67,499){\rule{0.400pt}{2.409pt}}
\multiput(624.17,504.00)(1.000,-5.000){2}{\rule{0.400pt}{1.204pt}}
\put(625.67,499){\rule{0.400pt}{1.927pt}}
\multiput(625.17,499.00)(1.000,4.000){2}{\rule{0.400pt}{0.964pt}}
\put(627.0,502.0){\rule[-0.200pt]{0.400pt}{1.204pt}}
\put(627.67,497){\rule{0.400pt}{1.204pt}}
\multiput(627.17,499.50)(1.000,-2.500){2}{\rule{0.400pt}{0.602pt}}
\put(627.0,502.0){\usebox{\plotpoint}}
\put(628.67,493){\rule{0.400pt}{0.482pt}}
\multiput(628.17,493.00)(1.000,1.000){2}{\rule{0.400pt}{0.241pt}}
\put(629.0,493.0){\rule[-0.200pt]{0.400pt}{0.964pt}}
\put(630,495){\usebox{\plotpoint}}
\put(629.67,489){\rule{0.400pt}{1.445pt}}
\multiput(629.17,492.00)(1.000,-3.000){2}{\rule{0.400pt}{0.723pt}}
\put(631,487.67){\rule{0.241pt}{0.400pt}}
\multiput(631.00,488.17)(0.500,-1.000){2}{\rule{0.120pt}{0.400pt}}
\put(632,484.67){\rule{0.241pt}{0.400pt}}
\multiput(632.00,484.17)(0.500,1.000){2}{\rule{0.120pt}{0.400pt}}
\put(632.67,484){\rule{0.400pt}{0.482pt}}
\multiput(632.17,485.00)(1.000,-1.000){2}{\rule{0.400pt}{0.241pt}}
\put(632.0,485.0){\rule[-0.200pt]{0.400pt}{0.723pt}}
\put(634,478.67){\rule{0.241pt}{0.400pt}}
\multiput(634.00,479.17)(0.500,-1.000){2}{\rule{0.120pt}{0.400pt}}
\put(635,478.67){\rule{0.241pt}{0.400pt}}
\multiput(635.00,478.17)(0.500,1.000){2}{\rule{0.120pt}{0.400pt}}
\put(634.0,480.0){\rule[-0.200pt]{0.400pt}{0.964pt}}
\put(636.0,472.0){\rule[-0.200pt]{0.400pt}{1.927pt}}
\put(637,470.67){\rule{0.241pt}{0.400pt}}
\multiput(637.00,471.17)(0.500,-1.000){2}{\rule{0.120pt}{0.400pt}}
\put(636.0,472.0){\usebox{\plotpoint}}
\put(637.67,466){\rule{0.400pt}{0.723pt}}
\multiput(637.17,467.50)(1.000,-1.500){2}{\rule{0.400pt}{0.361pt}}
\put(638.0,469.0){\rule[-0.200pt]{0.400pt}{0.482pt}}
\put(638.67,461){\rule{0.400pt}{0.964pt}}
\multiput(638.17,463.00)(1.000,-2.000){2}{\rule{0.400pt}{0.482pt}}
\put(640,459.67){\rule{0.241pt}{0.400pt}}
\multiput(640.00,460.17)(0.500,-1.000){2}{\rule{0.120pt}{0.400pt}}
\put(639.0,465.0){\usebox{\plotpoint}}
\put(641,455.67){\rule{0.241pt}{0.400pt}}
\multiput(641.00,455.17)(0.500,1.000){2}{\rule{0.120pt}{0.400pt}}
\put(642,455.67){\rule{0.241pt}{0.400pt}}
\multiput(642.00,456.17)(0.500,-1.000){2}{\rule{0.120pt}{0.400pt}}
\put(641.0,456.0){\rule[-0.200pt]{0.400pt}{0.964pt}}
\put(642.67,447){\rule{0.400pt}{0.964pt}}
\multiput(642.17,449.00)(1.000,-2.000){2}{\rule{0.400pt}{0.482pt}}
\put(644,445.67){\rule{0.241pt}{0.400pt}}
\multiput(644.00,446.17)(0.500,-1.000){2}{\rule{0.120pt}{0.400pt}}
\put(643.0,451.0){\rule[-0.200pt]{0.400pt}{1.204pt}}
\put(644.67,443){\rule{0.400pt}{0.482pt}}
\multiput(644.17,444.00)(1.000,-1.000){2}{\rule{0.400pt}{0.241pt}}
\put(645.67,440){\rule{0.400pt}{0.723pt}}
\multiput(645.17,441.50)(1.000,-1.500){2}{\rule{0.400pt}{0.361pt}}
\put(645.0,445.0){\usebox{\plotpoint}}
\put(646.67,435){\rule{0.400pt}{0.964pt}}
\multiput(646.17,437.00)(1.000,-2.000){2}{\rule{0.400pt}{0.482pt}}
\put(647.0,439.0){\usebox{\plotpoint}}
\put(648.0,435.0){\usebox{\plotpoint}}
\put(649,427.67){\rule{0.241pt}{0.400pt}}
\multiput(649.00,427.17)(0.500,1.000){2}{\rule{0.120pt}{0.400pt}}
\put(649.0,428.0){\rule[-0.200pt]{0.400pt}{1.686pt}}
\put(650,429){\usebox{\plotpoint}}
\put(649.67,427){\rule{0.400pt}{0.482pt}}
\multiput(649.17,428.00)(1.000,-1.000){2}{\rule{0.400pt}{0.241pt}}
\put(650.67,425){\rule{0.400pt}{0.482pt}}
\multiput(650.17,426.00)(1.000,-1.000){2}{\rule{0.400pt}{0.241pt}}
\put(651.67,418){\rule{0.400pt}{1.204pt}}
\multiput(651.17,420.50)(1.000,-2.500){2}{\rule{0.400pt}{0.602pt}}
\put(653,416.67){\rule{0.241pt}{0.400pt}}
\multiput(653.00,417.17)(0.500,-1.000){2}{\rule{0.120pt}{0.400pt}}
\put(652.0,423.0){\rule[-0.200pt]{0.400pt}{0.482pt}}
\put(654,417){\usebox{\plotpoint}}
\put(653.67,415){\rule{0.400pt}{0.482pt}}
\multiput(653.17,416.00)(1.000,-1.000){2}{\rule{0.400pt}{0.241pt}}
\put(654.67,409){\rule{0.400pt}{1.445pt}}
\multiput(654.17,412.00)(1.000,-3.000){2}{\rule{0.400pt}{0.723pt}}
\put(655.67,405){\rule{0.400pt}{1.445pt}}
\multiput(655.17,408.00)(1.000,-3.000){2}{\rule{0.400pt}{0.723pt}}
\put(656.0,409.0){\rule[-0.200pt]{0.400pt}{0.482pt}}
\put(657.0,405.0){\usebox{\plotpoint}}
\put(658,400.67){\rule{0.241pt}{0.400pt}}
\multiput(658.00,400.17)(0.500,1.000){2}{\rule{0.120pt}{0.400pt}}
\put(658.0,401.0){\rule[-0.200pt]{0.400pt}{0.964pt}}
\put(659,402){\usebox{\plotpoint}}
\put(658.67,398){\rule{0.400pt}{0.964pt}}
\multiput(658.17,400.00)(1.000,-2.000){2}{\rule{0.400pt}{0.482pt}}
\put(659.67,393){\rule{0.400pt}{1.204pt}}
\multiput(659.17,395.50)(1.000,-2.500){2}{\rule{0.400pt}{0.602pt}}
\put(660.67,392){\rule{0.400pt}{0.482pt}}
\multiput(660.17,393.00)(1.000,-1.000){2}{\rule{0.400pt}{0.241pt}}
\put(661.67,389){\rule{0.400pt}{0.723pt}}
\multiput(661.17,390.50)(1.000,-1.500){2}{\rule{0.400pt}{0.361pt}}
\put(661.0,393.0){\usebox{\plotpoint}}
\put(662.67,384){\rule{0.400pt}{0.964pt}}
\multiput(662.17,386.00)(1.000,-2.000){2}{\rule{0.400pt}{0.482pt}}
\put(663.67,382){\rule{0.400pt}{0.482pt}}
\multiput(663.17,383.00)(1.000,-1.000){2}{\rule{0.400pt}{0.241pt}}
\put(663.0,388.0){\usebox{\plotpoint}}
\put(665.0,381.0){\usebox{\plotpoint}}
\put(665.67,378){\rule{0.400pt}{0.723pt}}
\multiput(665.17,379.50)(1.000,-1.500){2}{\rule{0.400pt}{0.361pt}}
\put(665.0,381.0){\usebox{\plotpoint}}
\put(666.67,375){\rule{0.400pt}{0.482pt}}
\multiput(666.17,375.00)(1.000,1.000){2}{\rule{0.400pt}{0.241pt}}
\put(667.0,375.0){\rule[-0.200pt]{0.400pt}{0.723pt}}
\put(668,369.67){\rule{0.241pt}{0.400pt}}
\multiput(668.00,370.17)(0.500,-1.000){2}{\rule{0.120pt}{0.400pt}}
\put(668.67,367){\rule{0.400pt}{0.723pt}}
\multiput(668.17,368.50)(1.000,-1.500){2}{\rule{0.400pt}{0.361pt}}
\put(668.0,371.0){\rule[-0.200pt]{0.400pt}{1.445pt}}
\put(670.0,366.0){\usebox{\plotpoint}}
\put(670.67,364){\rule{0.400pt}{0.482pt}}
\multiput(670.17,365.00)(1.000,-1.000){2}{\rule{0.400pt}{0.241pt}}
\put(670.0,366.0){\usebox{\plotpoint}}
\put(671.67,359){\rule{0.400pt}{0.723pt}}
\multiput(671.17,360.50)(1.000,-1.500){2}{\rule{0.400pt}{0.361pt}}
\put(672.0,362.0){\rule[-0.200pt]{0.400pt}{0.482pt}}
\put(673.0,359.0){\usebox{\plotpoint}}
\put(674.0,356.0){\rule[-0.200pt]{0.400pt}{0.723pt}}
\put(675,354.67){\rule{0.241pt}{0.400pt}}
\multiput(675.00,355.17)(0.500,-1.000){2}{\rule{0.120pt}{0.400pt}}
\put(674.0,356.0){\usebox{\plotpoint}}
\put(675.67,350){\rule{0.400pt}{0.482pt}}
\multiput(675.17,351.00)(1.000,-1.000){2}{\rule{0.400pt}{0.241pt}}
\put(676.0,352.0){\rule[-0.200pt]{0.400pt}{0.723pt}}
\put(677,346.67){\rule{0.241pt}{0.400pt}}
\multiput(677.00,347.17)(0.500,-1.000){2}{\rule{0.120pt}{0.400pt}}
\put(677.67,345){\rule{0.400pt}{0.482pt}}
\multiput(677.17,346.00)(1.000,-1.000){2}{\rule{0.400pt}{0.241pt}}
\put(677.0,348.0){\rule[-0.200pt]{0.400pt}{0.482pt}}
\put(678.67,342){\rule{0.400pt}{0.964pt}}
\multiput(678.17,344.00)(1.000,-2.000){2}{\rule{0.400pt}{0.482pt}}
\put(680,341.67){\rule{0.241pt}{0.400pt}}
\multiput(680.00,341.17)(0.500,1.000){2}{\rule{0.120pt}{0.400pt}}
\put(679.0,345.0){\usebox{\plotpoint}}
\put(681,339.67){\rule{0.241pt}{0.400pt}}
\multiput(681.00,340.17)(0.500,-1.000){2}{\rule{0.120pt}{0.400pt}}
\put(681.67,336){\rule{0.400pt}{0.964pt}}
\multiput(681.17,338.00)(1.000,-2.000){2}{\rule{0.400pt}{0.482pt}}
\put(681.0,341.0){\rule[-0.200pt]{0.400pt}{0.482pt}}
\put(683,336){\usebox{\plotpoint}}
\put(683,334.67){\rule{0.241pt}{0.400pt}}
\multiput(683.00,335.17)(0.500,-1.000){2}{\rule{0.120pt}{0.400pt}}
\put(683.67,332){\rule{0.400pt}{0.723pt}}
\multiput(683.17,333.50)(1.000,-1.500){2}{\rule{0.400pt}{0.361pt}}
\put(685.0,330.0){\rule[-0.200pt]{0.400pt}{0.482pt}}
\put(685.0,330.0){\usebox{\plotpoint}}
\put(686.0,329.0){\usebox{\plotpoint}}
\put(687,327.67){\rule{0.241pt}{0.400pt}}
\multiput(687.00,328.17)(0.500,-1.000){2}{\rule{0.120pt}{0.400pt}}
\put(686.0,329.0){\usebox{\plotpoint}}
\put(687.67,325){\rule{0.400pt}{0.482pt}}
\multiput(687.17,326.00)(1.000,-1.000){2}{\rule{0.400pt}{0.241pt}}
\put(689,323.67){\rule{0.241pt}{0.400pt}}
\multiput(689.00,324.17)(0.500,-1.000){2}{\rule{0.120pt}{0.400pt}}
\put(688.0,327.0){\usebox{\plotpoint}}
\put(690,321.67){\rule{0.241pt}{0.400pt}}
\multiput(690.00,322.17)(0.500,-1.000){2}{\rule{0.120pt}{0.400pt}}
\put(690.67,320){\rule{0.400pt}{0.482pt}}
\multiput(690.17,321.00)(1.000,-1.000){2}{\rule{0.400pt}{0.241pt}}
\put(690.0,323.0){\usebox{\plotpoint}}
\put(692,317.67){\rule{0.241pt}{0.400pt}}
\multiput(692.00,318.17)(0.500,-1.000){2}{\rule{0.120pt}{0.400pt}}
\put(693,316.67){\rule{0.241pt}{0.400pt}}
\multiput(693.00,317.17)(0.500,-1.000){2}{\rule{0.120pt}{0.400pt}}
\put(692.0,319.0){\usebox{\plotpoint}}
\put(694.0,317.0){\usebox{\plotpoint}}
\put(694.0,318.0){\usebox{\plotpoint}}
\put(695,313.67){\rule{0.241pt}{0.400pt}}
\multiput(695.00,314.17)(0.500,-1.000){2}{\rule{0.120pt}{0.400pt}}
\put(695.0,315.0){\rule[-0.200pt]{0.400pt}{0.723pt}}
\put(696.0,314.0){\usebox{\plotpoint}}
\put(697,310.67){\rule{0.241pt}{0.400pt}}
\multiput(697.00,311.17)(0.500,-1.000){2}{\rule{0.120pt}{0.400pt}}
\put(697.67,309){\rule{0.400pt}{0.482pt}}
\multiput(697.17,310.00)(1.000,-1.000){2}{\rule{0.400pt}{0.241pt}}
\put(697.0,312.0){\rule[-0.200pt]{0.400pt}{0.482pt}}
\put(699,309){\usebox{\plotpoint}}
\put(698.67,307){\rule{0.400pt}{0.482pt}}
\multiput(698.17,308.00)(1.000,-1.000){2}{\rule{0.400pt}{0.241pt}}
\put(700,306.67){\rule{0.241pt}{0.400pt}}
\multiput(700.00,306.17)(0.500,1.000){2}{\rule{0.120pt}{0.400pt}}
\put(701,308){\usebox{\plotpoint}}
\put(700.67,306){\rule{0.400pt}{0.482pt}}
\multiput(700.17,307.00)(1.000,-1.000){2}{\rule{0.400pt}{0.241pt}}
\put(702,305.67){\rule{0.241pt}{0.400pt}}
\multiput(702.00,305.17)(0.500,1.000){2}{\rule{0.120pt}{0.400pt}}
\put(703,304.67){\rule{0.241pt}{0.400pt}}
\multiput(703.00,305.17)(0.500,-1.000){2}{\rule{0.120pt}{0.400pt}}
\put(703.0,306.0){\usebox{\plotpoint}}
\put(704,302.67){\rule{0.241pt}{0.400pt}}
\multiput(704.00,303.17)(0.500,-1.000){2}{\rule{0.120pt}{0.400pt}}
\put(704.0,304.0){\usebox{\plotpoint}}
\put(705.0,303.0){\usebox{\plotpoint}}
\put(706,301.67){\rule{0.241pt}{0.400pt}}
\multiput(706.00,301.17)(0.500,1.000){2}{\rule{0.120pt}{0.400pt}}
\put(707,301.67){\rule{0.241pt}{0.400pt}}
\multiput(707.00,302.17)(0.500,-1.000){2}{\rule{0.120pt}{0.400pt}}
\put(706.0,302.0){\usebox{\plotpoint}}
\put(708,299.67){\rule{0.241pt}{0.400pt}}
\multiput(708.00,299.17)(0.500,1.000){2}{\rule{0.120pt}{0.400pt}}
\put(709,299.67){\rule{0.241pt}{0.400pt}}
\multiput(709.00,300.17)(0.500,-1.000){2}{\rule{0.120pt}{0.400pt}}
\put(708.0,300.0){\rule[-0.200pt]{0.400pt}{0.482pt}}
\put(710,298.67){\rule{0.241pt}{0.400pt}}
\multiput(710.00,298.17)(0.500,1.000){2}{\rule{0.120pt}{0.400pt}}
\put(710.0,299.0){\usebox{\plotpoint}}
\put(711,297.67){\rule{0.241pt}{0.400pt}}
\multiput(711.00,298.17)(0.500,-1.000){2}{\rule{0.120pt}{0.400pt}}
\put(711.0,299.0){\usebox{\plotpoint}}
\put(714.67,296){\rule{0.400pt}{0.482pt}}
\multiput(714.17,297.00)(1.000,-1.000){2}{\rule{0.400pt}{0.241pt}}
\put(715.67,296){\rule{0.400pt}{0.723pt}}
\multiput(715.17,296.00)(1.000,1.500){2}{\rule{0.400pt}{0.361pt}}
\put(712.0,298.0){\rule[-0.200pt]{0.723pt}{0.400pt}}
\put(716.67,296){\rule{0.400pt}{0.482pt}}
\multiput(716.17,296.00)(1.000,1.000){2}{\rule{0.400pt}{0.241pt}}
\put(718,296.67){\rule{0.241pt}{0.400pt}}
\multiput(718.00,297.17)(0.500,-1.000){2}{\rule{0.120pt}{0.400pt}}
\put(717.0,296.0){\rule[-0.200pt]{0.400pt}{0.723pt}}
\put(719.0,296.0){\usebox{\plotpoint}}
\put(719.0,296.0){\usebox{\plotpoint}}
\put(720,294.67){\rule{0.241pt}{0.400pt}}
\multiput(720.00,294.17)(0.500,1.000){2}{\rule{0.120pt}{0.400pt}}
\put(720.0,295.0){\usebox{\plotpoint}}
\put(722,294.67){\rule{0.241pt}{0.400pt}}
\multiput(722.00,295.17)(0.500,-1.000){2}{\rule{0.120pt}{0.400pt}}
\put(721.0,296.0){\usebox{\plotpoint}}
\put(723.0,295.0){\usebox{\plotpoint}}
\put(723.67,294){\rule{0.400pt}{0.482pt}}
\multiput(723.17,294.00)(1.000,1.000){2}{\rule{0.400pt}{0.241pt}}
\put(724.0,294.0){\usebox{\plotpoint}}
\put(725.0,296.0){\usebox{\plotpoint}}
\put(726,294.67){\rule{0.241pt}{0.400pt}}
\multiput(726.00,294.17)(0.500,1.000){2}{\rule{0.120pt}{0.400pt}}
\put(726.0,295.0){\usebox{\plotpoint}}
\put(726.67,295){\rule{0.400pt}{0.482pt}}
\multiput(726.17,295.00)(1.000,1.000){2}{\rule{0.400pt}{0.241pt}}
\put(727.0,295.0){\usebox{\plotpoint}}
\put(728.0,297.0){\usebox{\plotpoint}}
\put(729,294.67){\rule{0.241pt}{0.400pt}}
\multiput(729.00,294.17)(0.500,1.000){2}{\rule{0.120pt}{0.400pt}}
\put(729.0,295.0){\rule[-0.200pt]{0.400pt}{0.482pt}}
\put(730.0,296.0){\usebox{\plotpoint}}
\put(731.0,295.0){\usebox{\plotpoint}}
\put(732,294.67){\rule{0.241pt}{0.400pt}}
\multiput(732.00,294.17)(0.500,1.000){2}{\rule{0.120pt}{0.400pt}}
\put(731.0,295.0){\usebox{\plotpoint}}
\put(733,296){\usebox{\plotpoint}}
\put(734,295.67){\rule{0.241pt}{0.400pt}}
\multiput(734.00,295.17)(0.500,1.000){2}{\rule{0.120pt}{0.400pt}}
\put(733.0,296.0){\usebox{\plotpoint}}
\put(735.0,296.0){\usebox{\plotpoint}}
\put(735.0,296.0){\usebox{\plotpoint}}
\put(736,295.67){\rule{0.241pt}{0.400pt}}
\multiput(736.00,296.17)(0.500,-1.000){2}{\rule{0.120pt}{0.400pt}}
\put(737,295.67){\rule{0.241pt}{0.400pt}}
\multiput(737.00,295.17)(0.500,1.000){2}{\rule{0.120pt}{0.400pt}}
\put(736.0,296.0){\usebox{\plotpoint}}
\put(737.67,296){\rule{0.400pt}{0.482pt}}
\multiput(737.17,296.00)(1.000,1.000){2}{\rule{0.400pt}{0.241pt}}
\put(738.0,296.0){\usebox{\plotpoint}}
\put(740,296.67){\rule{0.241pt}{0.400pt}}
\multiput(740.00,297.17)(0.500,-1.000){2}{\rule{0.120pt}{0.400pt}}
\put(741,296.67){\rule{0.241pt}{0.400pt}}
\multiput(741.00,296.17)(0.500,1.000){2}{\rule{0.120pt}{0.400pt}}
\put(739.0,298.0){\usebox{\plotpoint}}
\put(742,298){\usebox{\plotpoint}}
\put(741.67,298){\rule{0.400pt}{0.482pt}}
\multiput(741.17,298.00)(1.000,1.000){2}{\rule{0.400pt}{0.241pt}}
\put(743.0,298.0){\rule[-0.200pt]{0.400pt}{0.482pt}}
\put(744,297.67){\rule{0.241pt}{0.400pt}}
\multiput(744.00,297.17)(0.500,1.000){2}{\rule{0.120pt}{0.400pt}}
\put(743.0,298.0){\usebox{\plotpoint}}
\put(745,298.67){\rule{0.241pt}{0.400pt}}
\multiput(745.00,299.17)(0.500,-1.000){2}{\rule{0.120pt}{0.400pt}}
\put(745.0,299.0){\usebox{\plotpoint}}
\put(746.0,299.0){\usebox{\plotpoint}}
\put(746.67,298){\rule{0.400pt}{0.482pt}}
\multiput(746.17,298.00)(1.000,1.000){2}{\rule{0.400pt}{0.241pt}}
\put(748,299.67){\rule{0.241pt}{0.400pt}}
\multiput(748.00,299.17)(0.500,1.000){2}{\rule{0.120pt}{0.400pt}}
\put(747.0,298.0){\usebox{\plotpoint}}
\put(748.67,299){\rule{0.400pt}{0.723pt}}
\multiput(748.17,300.50)(1.000,-1.500){2}{\rule{0.400pt}{0.361pt}}
\put(749.0,301.0){\usebox{\plotpoint}}
\put(750.0,299.0){\rule[-0.200pt]{0.400pt}{0.723pt}}
\put(751,301.67){\rule{0.241pt}{0.400pt}}
\multiput(751.00,301.17)(0.500,1.000){2}{\rule{0.120pt}{0.400pt}}
\put(750.0,302.0){\usebox{\plotpoint}}
\put(751.67,299){\rule{0.400pt}{0.482pt}}
\multiput(751.17,299.00)(1.000,1.000){2}{\rule{0.400pt}{0.241pt}}
\put(752.67,301){\rule{0.400pt}{0.482pt}}
\multiput(752.17,301.00)(1.000,1.000){2}{\rule{0.400pt}{0.241pt}}
\put(752.0,299.0){\rule[-0.200pt]{0.400pt}{0.964pt}}
\put(754,301.67){\rule{0.241pt}{0.400pt}}
\multiput(754.00,301.17)(0.500,1.000){2}{\rule{0.120pt}{0.400pt}}
\put(754.0,302.0){\usebox{\plotpoint}}
\put(755.0,303.0){\usebox{\plotpoint}}
\put(756,302.67){\rule{0.241pt}{0.400pt}}
\multiput(756.00,303.17)(0.500,-1.000){2}{\rule{0.120pt}{0.400pt}}
\put(756.0,303.0){\usebox{\plotpoint}}
\put(757.0,303.0){\usebox{\plotpoint}}
\put(758,303.67){\rule{0.241pt}{0.400pt}}
\multiput(758.00,303.17)(0.500,1.000){2}{\rule{0.120pt}{0.400pt}}
\put(757.0,304.0){\usebox{\plotpoint}}
\put(759,305){\usebox{\plotpoint}}
\put(761,304.67){\rule{0.241pt}{0.400pt}}
\multiput(761.00,304.17)(0.500,1.000){2}{\rule{0.120pt}{0.400pt}}
\put(762,305.67){\rule{0.241pt}{0.400pt}}
\multiput(762.00,305.17)(0.500,1.000){2}{\rule{0.120pt}{0.400pt}}
\put(759.0,305.0){\rule[-0.200pt]{0.482pt}{0.400pt}}
\put(763.0,307.0){\usebox{\plotpoint}}
\put(765,307.67){\rule{0.241pt}{0.400pt}}
\multiput(765.00,307.17)(0.500,1.000){2}{\rule{0.120pt}{0.400pt}}
\put(763.0,308.0){\rule[-0.200pt]{0.482pt}{0.400pt}}
\put(766.0,309.0){\usebox{\plotpoint}}
\put(767,309.67){\rule{0.241pt}{0.400pt}}
\multiput(767.00,309.17)(0.500,1.000){2}{\rule{0.120pt}{0.400pt}}
\put(766.0,310.0){\usebox{\plotpoint}}
\put(768,311){\usebox{\plotpoint}}
\put(769,310.67){\rule{0.241pt}{0.400pt}}
\multiput(769.00,310.17)(0.500,1.000){2}{\rule{0.120pt}{0.400pt}}
\put(768.0,311.0){\usebox{\plotpoint}}
\put(770,312){\usebox{\plotpoint}}
\put(771,311.67){\rule{0.241pt}{0.400pt}}
\multiput(771.00,311.17)(0.500,1.000){2}{\rule{0.120pt}{0.400pt}}
\put(770.0,312.0){\usebox{\plotpoint}}
\put(772,313){\usebox{\plotpoint}}
\put(772,312.67){\rule{0.241pt}{0.400pt}}
\multiput(772.00,312.17)(0.500,1.000){2}{\rule{0.120pt}{0.400pt}}
\put(773,312.67){\rule{0.241pt}{0.400pt}}
\multiput(773.00,312.17)(0.500,1.000){2}{\rule{0.120pt}{0.400pt}}
\put(774,313.67){\rule{0.241pt}{0.400pt}}
\multiput(774.00,313.17)(0.500,1.000){2}{\rule{0.120pt}{0.400pt}}
\put(773.0,313.0){\usebox{\plotpoint}}
\put(775.0,315.0){\usebox{\plotpoint}}
\put(775.0,316.0){\rule[-0.200pt]{0.482pt}{0.400pt}}
\put(777,317.67){\rule{0.241pt}{0.400pt}}
\multiput(777.00,318.17)(0.500,-1.000){2}{\rule{0.120pt}{0.400pt}}
\put(777.0,316.0){\rule[-0.200pt]{0.400pt}{0.723pt}}
\put(778.0,318.0){\usebox{\plotpoint}}
\put(779,318.67){\rule{0.241pt}{0.400pt}}
\multiput(779.00,318.17)(0.500,1.000){2}{\rule{0.120pt}{0.400pt}}
\put(779.0,318.0){\usebox{\plotpoint}}
\put(780,320){\usebox{\plotpoint}}
\put(780,319.67){\rule{0.241pt}{0.400pt}}
\multiput(780.00,319.17)(0.500,1.000){2}{\rule{0.120pt}{0.400pt}}
\put(781,320.67){\rule{0.241pt}{0.400pt}}
\multiput(781.00,320.17)(0.500,1.000){2}{\rule{0.120pt}{0.400pt}}
\put(782,322.67){\rule{0.241pt}{0.400pt}}
\multiput(782.00,322.17)(0.500,1.000){2}{\rule{0.120pt}{0.400pt}}
\put(783,322.67){\rule{0.241pt}{0.400pt}}
\multiput(783.00,323.17)(0.500,-1.000){2}{\rule{0.120pt}{0.400pt}}
\put(782.0,322.0){\usebox{\plotpoint}}
\put(784,323){\usebox{\plotpoint}}
\put(783.67,323){\rule{0.400pt}{0.482pt}}
\multiput(783.17,323.00)(1.000,1.000){2}{\rule{0.400pt}{0.241pt}}
\put(785.0,325.0){\usebox{\plotpoint}}
\put(785.67,326){\rule{0.400pt}{0.723pt}}
\multiput(785.17,326.00)(1.000,1.500){2}{\rule{0.400pt}{0.361pt}}
\put(785.0,326.0){\usebox{\plotpoint}}
\put(787,327.67){\rule{0.241pt}{0.400pt}}
\multiput(787.00,327.17)(0.500,1.000){2}{\rule{0.120pt}{0.400pt}}
\put(788,328.67){\rule{0.241pt}{0.400pt}}
\multiput(788.00,328.17)(0.500,1.000){2}{\rule{0.120pt}{0.400pt}}
\put(787.0,328.0){\usebox{\plotpoint}}
\put(789.0,330.0){\usebox{\plotpoint}}
\put(789.0,331.0){\rule[-0.200pt]{0.482pt}{0.400pt}}
\put(790.67,332){\rule{0.400pt}{0.482pt}}
\multiput(790.17,332.00)(1.000,1.000){2}{\rule{0.400pt}{0.241pt}}
\put(791.0,331.0){\usebox{\plotpoint}}
\put(792,334){\usebox{\plotpoint}}
\put(791.67,334){\rule{0.400pt}{0.723pt}}
\multiput(791.17,334.00)(1.000,1.500){2}{\rule{0.400pt}{0.361pt}}
\put(793,335.67){\rule{0.241pt}{0.400pt}}
\multiput(793.00,336.17)(0.500,-1.000){2}{\rule{0.120pt}{0.400pt}}
\put(794.0,336.0){\rule[-0.200pt]{0.400pt}{0.723pt}}
\put(795,338.67){\rule{0.241pt}{0.400pt}}
\multiput(795.00,338.17)(0.500,1.000){2}{\rule{0.120pt}{0.400pt}}
\put(794.0,339.0){\usebox{\plotpoint}}
\put(796,340){\usebox{\plotpoint}}
\put(795.67,340){\rule{0.400pt}{0.482pt}}
\multiput(795.17,340.00)(1.000,1.000){2}{\rule{0.400pt}{0.241pt}}
\put(797,341.67){\rule{0.241pt}{0.400pt}}
\multiput(797.00,341.17)(0.500,1.000){2}{\rule{0.120pt}{0.400pt}}
\put(798,341.67){\rule{0.241pt}{0.400pt}}
\multiput(798.00,341.17)(0.500,1.000){2}{\rule{0.120pt}{0.400pt}}
\put(798.0,342.0){\usebox{\plotpoint}}
\put(798.67,346){\rule{0.400pt}{0.482pt}}
\multiput(798.17,346.00)(1.000,1.000){2}{\rule{0.400pt}{0.241pt}}
\put(799.0,343.0){\rule[-0.200pt]{0.400pt}{0.723pt}}
\put(800.0,348.0){\usebox{\plotpoint}}
\put(800.67,349){\rule{0.400pt}{0.482pt}}
\multiput(800.17,349.00)(1.000,1.000){2}{\rule{0.400pt}{0.241pt}}
\put(801.0,348.0){\usebox{\plotpoint}}
\put(802.0,351.0){\usebox{\plotpoint}}
\put(803,351.67){\rule{0.241pt}{0.400pt}}
\multiput(803.00,351.17)(0.500,1.000){2}{\rule{0.120pt}{0.400pt}}
\put(803.67,353){\rule{0.400pt}{0.723pt}}
\multiput(803.17,353.00)(1.000,1.500){2}{\rule{0.400pt}{0.361pt}}
\put(803.0,351.0){\usebox{\plotpoint}}
\put(804.67,355){\rule{0.400pt}{0.482pt}}
\multiput(804.17,355.00)(1.000,1.000){2}{\rule{0.400pt}{0.241pt}}
\put(805.0,355.0){\usebox{\plotpoint}}
\put(806,359.67){\rule{0.241pt}{0.400pt}}
\multiput(806.00,359.17)(0.500,1.000){2}{\rule{0.120pt}{0.400pt}}
\put(806.0,357.0){\rule[-0.200pt]{0.400pt}{0.723pt}}
\put(807.0,361.0){\usebox{\plotpoint}}
\put(808.0,361.0){\rule[-0.200pt]{0.400pt}{0.723pt}}
\put(809,362.67){\rule{0.241pt}{0.400pt}}
\multiput(809.00,363.17)(0.500,-1.000){2}{\rule{0.120pt}{0.400pt}}
\put(808.0,364.0){\usebox{\plotpoint}}
\put(810,366.67){\rule{0.241pt}{0.400pt}}
\multiput(810.00,366.17)(0.500,1.000){2}{\rule{0.120pt}{0.400pt}}
\put(811,367.67){\rule{0.241pt}{0.400pt}}
\multiput(811.00,367.17)(0.500,1.000){2}{\rule{0.120pt}{0.400pt}}
\put(810.0,363.0){\rule[-0.200pt]{0.400pt}{0.964pt}}
\put(812,370.67){\rule{0.241pt}{0.400pt}}
\multiput(812.00,370.17)(0.500,1.000){2}{\rule{0.120pt}{0.400pt}}
\put(812.0,369.0){\rule[-0.200pt]{0.400pt}{0.482pt}}
\put(813,373.67){\rule{0.241pt}{0.400pt}}
\multiput(813.00,373.17)(0.500,1.000){2}{\rule{0.120pt}{0.400pt}}
\put(813.67,375){\rule{0.400pt}{0.723pt}}
\multiput(813.17,375.00)(1.000,1.500){2}{\rule{0.400pt}{0.361pt}}
\put(813.0,372.0){\rule[-0.200pt]{0.400pt}{0.482pt}}
\put(814.67,379){\rule{0.400pt}{0.723pt}}
\multiput(814.17,379.00)(1.000,1.500){2}{\rule{0.400pt}{0.361pt}}
\put(816,380.67){\rule{0.241pt}{0.400pt}}
\multiput(816.00,381.17)(0.500,-1.000){2}{\rule{0.120pt}{0.400pt}}
\put(815.0,378.0){\usebox{\plotpoint}}
\put(816.67,382){\rule{0.400pt}{0.482pt}}
\multiput(816.17,382.00)(1.000,1.000){2}{\rule{0.400pt}{0.241pt}}
\put(817.0,381.0){\usebox{\plotpoint}}
\put(817.67,387){\rule{0.400pt}{0.482pt}}
\multiput(817.17,387.00)(1.000,1.000){2}{\rule{0.400pt}{0.241pt}}
\put(819,388.67){\rule{0.241pt}{0.400pt}}
\multiput(819.00,388.17)(0.500,1.000){2}{\rule{0.120pt}{0.400pt}}
\put(818.0,384.0){\rule[-0.200pt]{0.400pt}{0.723pt}}
\put(820,390){\usebox{\plotpoint}}
\put(819.67,390){\rule{0.400pt}{0.723pt}}
\multiput(819.17,390.00)(1.000,1.500){2}{\rule{0.400pt}{0.361pt}}
\put(820.67,393){\rule{0.400pt}{0.723pt}}
\multiput(820.17,393.00)(1.000,1.500){2}{\rule{0.400pt}{0.361pt}}
\put(821.67,399){\rule{0.400pt}{0.482pt}}
\multiput(821.17,399.00)(1.000,1.000){2}{\rule{0.400pt}{0.241pt}}
\put(823,400.67){\rule{0.241pt}{0.400pt}}
\multiput(823.00,400.17)(0.500,1.000){2}{\rule{0.120pt}{0.400pt}}
\put(822.0,396.0){\rule[-0.200pt]{0.400pt}{0.723pt}}
\put(824,402.67){\rule{0.241pt}{0.400pt}}
\multiput(824.00,402.17)(0.500,1.000){2}{\rule{0.120pt}{0.400pt}}
\put(824.0,402.0){\usebox{\plotpoint}}
\put(824.67,408){\rule{0.400pt}{0.482pt}}
\multiput(824.17,408.00)(1.000,1.000){2}{\rule{0.400pt}{0.241pt}}
\put(825.67,410){\rule{0.400pt}{0.482pt}}
\multiput(825.17,410.00)(1.000,1.000){2}{\rule{0.400pt}{0.241pt}}
\put(825.0,404.0){\rule[-0.200pt]{0.400pt}{0.964pt}}
\put(826.67,411){\rule{0.400pt}{0.964pt}}
\multiput(826.17,411.00)(1.000,2.000){2}{\rule{0.400pt}{0.482pt}}
\put(827.67,415){\rule{0.400pt}{0.482pt}}
\multiput(827.17,415.00)(1.000,1.000){2}{\rule{0.400pt}{0.241pt}}
\put(827.0,411.0){\usebox{\plotpoint}}
\put(829,420.67){\rule{0.241pt}{0.400pt}}
\multiput(829.00,420.17)(0.500,1.000){2}{\rule{0.120pt}{0.400pt}}
\put(829.0,417.0){\rule[-0.200pt]{0.400pt}{0.964pt}}
\put(829.67,423){\rule{0.400pt}{0.723pt}}
\multiput(829.17,423.00)(1.000,1.500){2}{\rule{0.400pt}{0.361pt}}
\put(830.67,426){\rule{0.400pt}{0.482pt}}
\multiput(830.17,426.00)(1.000,1.000){2}{\rule{0.400pt}{0.241pt}}
\put(830.0,422.0){\usebox{\plotpoint}}
\put(832,429.67){\rule{0.241pt}{0.400pt}}
\multiput(832.00,430.17)(0.500,-1.000){2}{\rule{0.120pt}{0.400pt}}
\put(832.67,430){\rule{0.400pt}{1.445pt}}
\multiput(832.17,430.00)(1.000,3.000){2}{\rule{0.400pt}{0.723pt}}
\put(832.0,428.0){\rule[-0.200pt]{0.400pt}{0.723pt}}
\put(833.67,437){\rule{0.400pt}{0.482pt}}
\multiput(833.17,437.00)(1.000,1.000){2}{\rule{0.400pt}{0.241pt}}
\put(834.67,439){\rule{0.400pt}{0.482pt}}
\multiput(834.17,439.00)(1.000,1.000){2}{\rule{0.400pt}{0.241pt}}
\put(834.0,436.0){\usebox{\plotpoint}}
\put(836.0,441.0){\rule[-0.200pt]{0.400pt}{1.204pt}}
\put(836.0,446.0){\usebox{\plotpoint}}
\put(836.67,449){\rule{0.400pt}{0.482pt}}
\multiput(836.17,449.00)(1.000,1.000){2}{\rule{0.400pt}{0.241pt}}
\put(837.67,451){\rule{0.400pt}{0.482pt}}
\multiput(837.17,451.00)(1.000,1.000){2}{\rule{0.400pt}{0.241pt}}
\put(837.0,446.0){\rule[-0.200pt]{0.400pt}{0.723pt}}
\put(839,453){\usebox{\plotpoint}}
\put(838.67,453){\rule{0.400pt}{1.445pt}}
\multiput(838.17,453.00)(1.000,3.000){2}{\rule{0.400pt}{0.723pt}}
\put(840,458.67){\rule{0.241pt}{0.400pt}}
\multiput(840.00,458.17)(0.500,1.000){2}{\rule{0.120pt}{0.400pt}}
\put(840.67,461){\rule{0.400pt}{0.723pt}}
\multiput(840.17,461.00)(1.000,1.500){2}{\rule{0.400pt}{0.361pt}}
\put(841.0,460.0){\usebox{\plotpoint}}
\put(841.67,467){\rule{0.400pt}{0.964pt}}
\multiput(841.17,467.00)(1.000,2.000){2}{\rule{0.400pt}{0.482pt}}
\put(842.67,471){\rule{0.400pt}{0.723pt}}
\multiput(842.17,471.00)(1.000,1.500){2}{\rule{0.400pt}{0.361pt}}
\put(842.0,464.0){\rule[-0.200pt]{0.400pt}{0.723pt}}
\put(843.67,475){\rule{0.400pt}{0.482pt}}
\multiput(843.17,475.00)(1.000,1.000){2}{\rule{0.400pt}{0.241pt}}
\put(844.67,477){\rule{0.400pt}{0.482pt}}
\multiput(844.17,477.00)(1.000,1.000){2}{\rule{0.400pt}{0.241pt}}
\put(844.0,474.0){\usebox{\plotpoint}}
\put(846,482.67){\rule{0.241pt}{0.400pt}}
\multiput(846.00,482.17)(0.500,1.000){2}{\rule{0.120pt}{0.400pt}}
\put(846.67,484){\rule{0.400pt}{1.204pt}}
\multiput(846.17,484.00)(1.000,2.500){2}{\rule{0.400pt}{0.602pt}}
\put(846.0,479.0){\rule[-0.200pt]{0.400pt}{0.964pt}}
\put(847.67,491){\rule{0.400pt}{0.964pt}}
\multiput(847.17,491.00)(1.000,2.000){2}{\rule{0.400pt}{0.482pt}}
\put(848.0,489.0){\rule[-0.200pt]{0.400pt}{0.482pt}}
\put(849,495.67){\rule{0.241pt}{0.400pt}}
\multiput(849.00,495.17)(0.500,1.000){2}{\rule{0.120pt}{0.400pt}}
\put(849.67,497){\rule{0.400pt}{1.204pt}}
\multiput(849.17,497.00)(1.000,2.500){2}{\rule{0.400pt}{0.602pt}}
\put(849.0,495.0){\usebox{\plotpoint}}
\put(851,503.67){\rule{0.241pt}{0.400pt}}
\multiput(851.00,503.17)(0.500,1.000){2}{\rule{0.120pt}{0.400pt}}
\put(851.0,502.0){\rule[-0.200pt]{0.400pt}{0.482pt}}
\put(852.0,505.0){\usebox{\plotpoint}}
\put(852.67,511){\rule{0.400pt}{0.964pt}}
\multiput(852.17,511.00)(1.000,2.000){2}{\rule{0.400pt}{0.482pt}}
\put(853.0,505.0){\rule[-0.200pt]{0.400pt}{1.445pt}}
\put(854,515){\usebox{\plotpoint}}
\put(853.67,515){\rule{0.400pt}{0.723pt}}
\multiput(853.17,515.00)(1.000,1.500){2}{\rule{0.400pt}{0.361pt}}
\put(854.67,518){\rule{0.400pt}{0.723pt}}
\multiput(854.17,518.00)(1.000,1.500){2}{\rule{0.400pt}{0.361pt}}
\put(855.67,523){\rule{0.400pt}{0.723pt}}
\multiput(855.17,523.00)(1.000,1.500){2}{\rule{0.400pt}{0.361pt}}
\put(857,524.67){\rule{0.241pt}{0.400pt}}
\multiput(857.00,525.17)(0.500,-1.000){2}{\rule{0.120pt}{0.400pt}}
\put(856.0,521.0){\rule[-0.200pt]{0.400pt}{0.482pt}}
\put(857.67,530){\rule{0.400pt}{1.445pt}}
\multiput(857.17,530.00)(1.000,3.000){2}{\rule{0.400pt}{0.723pt}}
\put(859,534.67){\rule{0.241pt}{0.400pt}}
\multiput(859.00,535.17)(0.500,-1.000){2}{\rule{0.120pt}{0.400pt}}
\put(858.0,525.0){\rule[-0.200pt]{0.400pt}{1.204pt}}
\put(860,538.67){\rule{0.241pt}{0.400pt}}
\multiput(860.00,539.17)(0.500,-1.000){2}{\rule{0.120pt}{0.400pt}}
\put(860.0,535.0){\rule[-0.200pt]{0.400pt}{1.204pt}}
\put(860.67,542){\rule{0.400pt}{0.964pt}}
\multiput(860.17,542.00)(1.000,2.000){2}{\rule{0.400pt}{0.482pt}}
\put(861.67,546){\rule{0.400pt}{1.204pt}}
\multiput(861.17,546.00)(1.000,2.500){2}{\rule{0.400pt}{0.602pt}}
\put(861.0,539.0){\rule[-0.200pt]{0.400pt}{0.723pt}}
\put(863.0,551.0){\rule[-0.200pt]{0.400pt}{0.723pt}}
\put(863.67,554){\rule{0.400pt}{0.482pt}}
\multiput(863.17,554.00)(1.000,1.000){2}{\rule{0.400pt}{0.241pt}}
\put(863.0,554.0){\usebox{\plotpoint}}
\put(864.67,559){\rule{0.400pt}{0.723pt}}
\multiput(864.17,559.00)(1.000,1.500){2}{\rule{0.400pt}{0.361pt}}
\put(865.0,556.0){\rule[-0.200pt]{0.400pt}{0.723pt}}
\put(866,562){\usebox{\plotpoint}}
\put(865.67,562){\rule{0.400pt}{1.204pt}}
\multiput(865.17,562.00)(1.000,2.500){2}{\rule{0.400pt}{0.602pt}}
\put(866.67,567){\rule{0.400pt}{0.482pt}}
\multiput(866.17,567.00)(1.000,1.000){2}{\rule{0.400pt}{0.241pt}}
\put(867.67,571){\rule{0.400pt}{1.204pt}}
\multiput(867.17,571.00)(1.000,2.500){2}{\rule{0.400pt}{0.602pt}}
\put(868.67,576){\rule{0.400pt}{0.723pt}}
\multiput(868.17,576.00)(1.000,1.500){2}{\rule{0.400pt}{0.361pt}}
\put(868.0,569.0){\rule[-0.200pt]{0.400pt}{0.482pt}}
\put(870,578.67){\rule{0.241pt}{0.400pt}}
\multiput(870.00,579.17)(0.500,-1.000){2}{\rule{0.120pt}{0.400pt}}
\put(870.0,579.0){\usebox{\plotpoint}}
\put(870.67,581){\rule{0.400pt}{0.723pt}}
\multiput(870.17,582.50)(1.000,-1.500){2}{\rule{0.400pt}{0.361pt}}
\put(871.67,581){\rule{0.400pt}{1.927pt}}
\multiput(871.17,581.00)(1.000,4.000){2}{\rule{0.400pt}{0.964pt}}
\put(871.0,579.0){\rule[-0.200pt]{0.400pt}{1.204pt}}
\put(873.0,589.0){\usebox{\plotpoint}}
\put(873.0,590.0){\rule[-0.200pt]{0.482pt}{0.400pt}}
\put(875,595.67){\rule{0.241pt}{0.400pt}}
\multiput(875.00,595.17)(0.500,1.000){2}{\rule{0.120pt}{0.400pt}}
\put(875.67,595){\rule{0.400pt}{0.482pt}}
\multiput(875.17,596.00)(1.000,-1.000){2}{\rule{0.400pt}{0.241pt}}
\put(875.0,590.0){\rule[-0.200pt]{0.400pt}{1.445pt}}
\put(877,601.67){\rule{0.241pt}{0.400pt}}
\multiput(877.00,601.17)(0.500,1.000){2}{\rule{0.120pt}{0.400pt}}
\put(877.0,595.0){\rule[-0.200pt]{0.400pt}{1.686pt}}
\put(877.67,604){\rule{0.400pt}{0.482pt}}
\multiput(877.17,605.00)(1.000,-1.000){2}{\rule{0.400pt}{0.241pt}}
\put(878.67,604){\rule{0.400pt}{1.445pt}}
\multiput(878.17,604.00)(1.000,3.000){2}{\rule{0.400pt}{0.723pt}}
\put(878.0,603.0){\rule[-0.200pt]{0.400pt}{0.723pt}}
\put(879.67,609){\rule{0.400pt}{1.445pt}}
\multiput(879.17,609.00)(1.000,3.000){2}{\rule{0.400pt}{0.723pt}}
\put(880.67,612){\rule{0.400pt}{0.723pt}}
\multiput(880.17,613.50)(1.000,-1.500){2}{\rule{0.400pt}{0.361pt}}
\put(880.0,609.0){\usebox{\plotpoint}}
\put(882.0,612.0){\rule[-0.200pt]{0.400pt}{0.964pt}}
\put(882.0,616.0){\usebox{\plotpoint}}
\put(882.67,618){\rule{0.400pt}{0.482pt}}
\multiput(882.17,619.00)(1.000,-1.000){2}{\rule{0.400pt}{0.241pt}}
\put(883.0,616.0){\rule[-0.200pt]{0.400pt}{0.964pt}}
\put(884.0,618.0){\usebox{\plotpoint}}
\put(884.67,622){\rule{0.400pt}{1.445pt}}
\multiput(884.17,625.00)(1.000,-3.000){2}{\rule{0.400pt}{0.723pt}}
\put(885.67,622){\rule{0.400pt}{1.204pt}}
\multiput(885.17,622.00)(1.000,2.500){2}{\rule{0.400pt}{0.602pt}}
\put(885.0,618.0){\rule[-0.200pt]{0.400pt}{2.409pt}}
\put(886.67,627){\rule{0.400pt}{1.204pt}}
\multiput(886.17,629.50)(1.000,-2.500){2}{\rule{0.400pt}{0.602pt}}
\put(887.0,627.0){\rule[-0.200pt]{0.400pt}{1.204pt}}
\put(887.67,625){\rule{0.400pt}{1.686pt}}
\multiput(887.17,625.00)(1.000,3.500){2}{\rule{0.400pt}{0.843pt}}
\put(889,630.67){\rule{0.241pt}{0.400pt}}
\multiput(889.00,631.17)(0.500,-1.000){2}{\rule{0.120pt}{0.400pt}}
\put(888.0,625.0){\rule[-0.200pt]{0.400pt}{0.482pt}}
\put(890.0,631.0){\rule[-0.200pt]{0.400pt}{0.723pt}}
\put(890.67,630){\rule{0.400pt}{0.964pt}}
\multiput(890.17,632.00)(1.000,-2.000){2}{\rule{0.400pt}{0.482pt}}
\put(890.0,634.0){\usebox{\plotpoint}}
\put(891.67,632){\rule{0.400pt}{0.964pt}}
\multiput(891.17,632.00)(1.000,2.000){2}{\rule{0.400pt}{0.482pt}}
\put(892.0,630.0){\rule[-0.200pt]{0.400pt}{0.482pt}}
\put(892.67,634){\rule{0.400pt}{1.686pt}}
\multiput(892.17,634.00)(1.000,3.500){2}{\rule{0.400pt}{0.843pt}}
\put(893.67,634){\rule{0.400pt}{1.686pt}}
\multiput(893.17,637.50)(1.000,-3.500){2}{\rule{0.400pt}{0.843pt}}
\put(893.0,634.0){\rule[-0.200pt]{0.400pt}{0.482pt}}
\put(894.67,637){\rule{0.400pt}{0.482pt}}
\multiput(894.17,637.00)(1.000,1.000){2}{\rule{0.400pt}{0.241pt}}
\put(895.67,637){\rule{0.400pt}{0.482pt}}
\multiput(895.17,638.00)(1.000,-1.000){2}{\rule{0.400pt}{0.241pt}}
\put(895.0,634.0){\rule[-0.200pt]{0.400pt}{0.723pt}}
\put(896.67,638){\rule{0.400pt}{0.482pt}}
\multiput(896.17,638.00)(1.000,1.000){2}{\rule{0.400pt}{0.241pt}}
\put(898,639.67){\rule{0.241pt}{0.400pt}}
\multiput(898.00,639.17)(0.500,1.000){2}{\rule{0.120pt}{0.400pt}}
\put(897.0,637.0){\usebox{\plotpoint}}
\put(899,636.67){\rule{0.241pt}{0.400pt}}
\multiput(899.00,636.17)(0.500,1.000){2}{\rule{0.120pt}{0.400pt}}
\put(899.0,637.0){\rule[-0.200pt]{0.400pt}{0.964pt}}
\put(899.67,638){\rule{0.400pt}{0.482pt}}
\multiput(899.17,639.00)(1.000,-1.000){2}{\rule{0.400pt}{0.241pt}}
\put(900.67,633){\rule{0.400pt}{1.204pt}}
\multiput(900.17,635.50)(1.000,-2.500){2}{\rule{0.400pt}{0.602pt}}
\put(900.0,638.0){\rule[-0.200pt]{0.400pt}{0.482pt}}
\put(902,639.67){\rule{0.241pt}{0.400pt}}
\multiput(902.00,639.17)(0.500,1.000){2}{\rule{0.120pt}{0.400pt}}
\put(902.67,634){\rule{0.400pt}{1.686pt}}
\multiput(902.17,637.50)(1.000,-3.500){2}{\rule{0.400pt}{0.843pt}}
\put(902.0,633.0){\rule[-0.200pt]{0.400pt}{1.686pt}}
\put(904,634.67){\rule{0.241pt}{0.400pt}}
\multiput(904.00,634.17)(0.500,1.000){2}{\rule{0.120pt}{0.400pt}}
\put(904.0,634.0){\usebox{\plotpoint}}
\put(904.67,630){\rule{0.400pt}{1.686pt}}
\multiput(904.17,633.50)(1.000,-3.500){2}{\rule{0.400pt}{0.843pt}}
\put(905.67,630){\rule{0.400pt}{0.723pt}}
\multiput(905.17,630.00)(1.000,1.500){2}{\rule{0.400pt}{0.361pt}}
\put(905.0,636.0){\usebox{\plotpoint}}
\put(906.67,629){\rule{0.400pt}{1.204pt}}
\multiput(906.17,629.00)(1.000,2.500){2}{\rule{0.400pt}{0.602pt}}
\put(907.67,630){\rule{0.400pt}{0.964pt}}
\multiput(907.17,632.00)(1.000,-2.000){2}{\rule{0.400pt}{0.482pt}}
\put(907.0,629.0){\rule[-0.200pt]{0.400pt}{0.964pt}}
\put(909.0,629.0){\usebox{\plotpoint}}
\put(909.0,629.0){\usebox{\plotpoint}}
\put(910.0,628.0){\usebox{\plotpoint}}
\put(910.67,624){\rule{0.400pt}{0.964pt}}
\multiput(910.17,626.00)(1.000,-2.000){2}{\rule{0.400pt}{0.482pt}}
\put(910.0,628.0){\usebox{\plotpoint}}
\put(911.67,619){\rule{0.400pt}{0.964pt}}
\multiput(911.17,619.00)(1.000,2.000){2}{\rule{0.400pt}{0.482pt}}
\put(912.67,618){\rule{0.400pt}{1.204pt}}
\multiput(912.17,620.50)(1.000,-2.500){2}{\rule{0.400pt}{0.602pt}}
\put(912.0,619.0){\rule[-0.200pt]{0.400pt}{1.204pt}}
\put(914,618){\usebox{\plotpoint}}
\put(913.67,613){\rule{0.400pt}{1.204pt}}
\multiput(913.17,615.50)(1.000,-2.500){2}{\rule{0.400pt}{0.602pt}}
\put(915.0,612.0){\usebox{\plotpoint}}
\put(916,611.67){\rule{0.241pt}{0.400pt}}
\multiput(916.00,611.17)(0.500,1.000){2}{\rule{0.120pt}{0.400pt}}
\put(915.0,612.0){\usebox{\plotpoint}}
\put(916.67,607){\rule{0.400pt}{0.723pt}}
\multiput(916.17,608.50)(1.000,-1.500){2}{\rule{0.400pt}{0.361pt}}
\put(917.67,599){\rule{0.400pt}{1.927pt}}
\multiput(917.17,603.00)(1.000,-4.000){2}{\rule{0.400pt}{0.964pt}}
\put(917.0,610.0){\rule[-0.200pt]{0.400pt}{0.723pt}}
\put(918.67,600){\rule{0.400pt}{0.964pt}}
\multiput(918.17,602.00)(1.000,-2.000){2}{\rule{0.400pt}{0.482pt}}
\put(919.0,599.0){\rule[-0.200pt]{0.400pt}{1.204pt}}
\put(919.67,593){\rule{0.400pt}{0.964pt}}
\multiput(919.17,595.00)(1.000,-2.000){2}{\rule{0.400pt}{0.482pt}}
\put(920.0,597.0){\rule[-0.200pt]{0.400pt}{0.723pt}}
\put(921.67,586){\rule{0.400pt}{1.686pt}}
\multiput(921.17,589.50)(1.000,-3.500){2}{\rule{0.400pt}{0.843pt}}
\put(921.0,593.0){\usebox{\plotpoint}}
\put(923.0,586.0){\usebox{\plotpoint}}
\put(923.67,579){\rule{0.400pt}{0.964pt}}
\multiput(923.17,581.00)(1.000,-2.000){2}{\rule{0.400pt}{0.482pt}}
\put(924.0,583.0){\rule[-0.200pt]{0.400pt}{0.723pt}}
\put(924.67,575){\rule{0.400pt}{1.445pt}}
\multiput(924.17,578.00)(1.000,-3.000){2}{\rule{0.400pt}{0.723pt}}
\put(926,573.67){\rule{0.241pt}{0.400pt}}
\multiput(926.00,574.17)(0.500,-1.000){2}{\rule{0.120pt}{0.400pt}}
\put(925.0,579.0){\rule[-0.200pt]{0.400pt}{0.482pt}}
\put(927.0,568.0){\rule[-0.200pt]{0.400pt}{1.445pt}}
\put(927.67,563){\rule{0.400pt}{1.204pt}}
\multiput(927.17,565.50)(1.000,-2.500){2}{\rule{0.400pt}{0.602pt}}
\put(927.0,568.0){\usebox{\plotpoint}}
\put(928.67,557){\rule{0.400pt}{0.482pt}}
\multiput(928.17,557.00)(1.000,1.000){2}{\rule{0.400pt}{0.241pt}}
\put(929.0,557.0){\rule[-0.200pt]{0.400pt}{1.445pt}}
\put(929.67,551){\rule{0.400pt}{0.482pt}}
\multiput(929.17,552.00)(1.000,-1.000){2}{\rule{0.400pt}{0.241pt}}
\put(930.67,549){\rule{0.400pt}{0.482pt}}
\multiput(930.17,550.00)(1.000,-1.000){2}{\rule{0.400pt}{0.241pt}}
\put(930.0,553.0){\rule[-0.200pt]{0.400pt}{1.445pt}}
\put(932,541.67){\rule{0.241pt}{0.400pt}}
\multiput(932.00,542.17)(0.500,-1.000){2}{\rule{0.120pt}{0.400pt}}
\put(932.67,538){\rule{0.400pt}{0.964pt}}
\multiput(932.17,540.00)(1.000,-2.000){2}{\rule{0.400pt}{0.482pt}}
\put(932.0,543.0){\rule[-0.200pt]{0.400pt}{1.445pt}}
\put(934.0,533.0){\rule[-0.200pt]{0.400pt}{1.204pt}}
\put(934.0,533.0){\usebox{\plotpoint}}
\put(934.67,526){\rule{0.400pt}{0.482pt}}
\multiput(934.17,527.00)(1.000,-1.000){2}{\rule{0.400pt}{0.241pt}}
\put(935.67,521){\rule{0.400pt}{1.204pt}}
\multiput(935.17,523.50)(1.000,-2.500){2}{\rule{0.400pt}{0.602pt}}
\put(935.0,528.0){\rule[-0.200pt]{0.400pt}{1.204pt}}
\put(937,515.67){\rule{0.241pt}{0.400pt}}
\multiput(937.00,516.17)(0.500,-1.000){2}{\rule{0.120pt}{0.400pt}}
\put(937.67,509){\rule{0.400pt}{1.686pt}}
\multiput(937.17,512.50)(1.000,-3.500){2}{\rule{0.400pt}{0.843pt}}
\put(937.0,517.0){\rule[-0.200pt]{0.400pt}{0.964pt}}
\put(938.67,504){\rule{0.400pt}{0.482pt}}
\multiput(938.17,505.00)(1.000,-1.000){2}{\rule{0.400pt}{0.241pt}}
\put(939.0,506.0){\rule[-0.200pt]{0.400pt}{0.723pt}}
\put(939.67,498){\rule{0.400pt}{0.964pt}}
\multiput(939.17,500.00)(1.000,-2.000){2}{\rule{0.400pt}{0.482pt}}
\put(940.67,494){\rule{0.400pt}{0.964pt}}
\multiput(940.17,496.00)(1.000,-2.000){2}{\rule{0.400pt}{0.482pt}}
\put(940.0,502.0){\rule[-0.200pt]{0.400pt}{0.482pt}}
\put(941.67,485){\rule{0.400pt}{0.964pt}}
\multiput(941.17,487.00)(1.000,-2.000){2}{\rule{0.400pt}{0.482pt}}
\put(942.67,481){\rule{0.400pt}{0.964pt}}
\multiput(942.17,483.00)(1.000,-2.000){2}{\rule{0.400pt}{0.482pt}}
\put(942.0,489.0){\rule[-0.200pt]{0.400pt}{1.204pt}}
\put(943.67,475){\rule{0.400pt}{0.482pt}}
\multiput(943.17,476.00)(1.000,-1.000){2}{\rule{0.400pt}{0.241pt}}
\put(944.0,477.0){\rule[-0.200pt]{0.400pt}{0.964pt}}
\put(944.67,470){\rule{0.400pt}{0.482pt}}
\multiput(944.17,471.00)(1.000,-1.000){2}{\rule{0.400pt}{0.241pt}}
\put(945.67,466){\rule{0.400pt}{0.964pt}}
\multiput(945.17,468.00)(1.000,-2.000){2}{\rule{0.400pt}{0.482pt}}
\put(945.0,472.0){\rule[-0.200pt]{0.400pt}{0.723pt}}
\put(946.67,458){\rule{0.400pt}{0.482pt}}
\multiput(946.17,459.00)(1.000,-1.000){2}{\rule{0.400pt}{0.241pt}}
\put(947.67,452){\rule{0.400pt}{1.445pt}}
\multiput(947.17,455.00)(1.000,-3.000){2}{\rule{0.400pt}{0.723pt}}
\put(947.0,460.0){\rule[-0.200pt]{0.400pt}{1.445pt}}
\put(948.67,447){\rule{0.400pt}{0.482pt}}
\multiput(948.17,448.00)(1.000,-1.000){2}{\rule{0.400pt}{0.241pt}}
\put(949.0,449.0){\rule[-0.200pt]{0.400pt}{0.723pt}}
\put(949.67,441){\rule{0.400pt}{0.482pt}}
\multiput(949.17,442.00)(1.000,-1.000){2}{\rule{0.400pt}{0.241pt}}
\put(950.67,436){\rule{0.400pt}{1.204pt}}
\multiput(950.17,438.50)(1.000,-2.500){2}{\rule{0.400pt}{0.602pt}}
\put(950.0,443.0){\rule[-0.200pt]{0.400pt}{0.964pt}}
\put(951.67,430){\rule{0.400pt}{0.964pt}}
\multiput(951.17,432.00)(1.000,-2.000){2}{\rule{0.400pt}{0.482pt}}
\put(952.67,424){\rule{0.400pt}{1.445pt}}
\multiput(952.17,427.00)(1.000,-3.000){2}{\rule{0.400pt}{0.723pt}}
\put(952.0,434.0){\rule[-0.200pt]{0.400pt}{0.482pt}}
\put(953.67,417){\rule{0.400pt}{0.964pt}}
\multiput(953.17,419.00)(1.000,-2.000){2}{\rule{0.400pt}{0.482pt}}
\put(954.0,421.0){\rule[-0.200pt]{0.400pt}{0.723pt}}
\put(954.67,410){\rule{0.400pt}{0.723pt}}
\multiput(954.17,411.50)(1.000,-1.500){2}{\rule{0.400pt}{0.361pt}}
\put(955.67,407){\rule{0.400pt}{0.723pt}}
\multiput(955.17,408.50)(1.000,-1.500){2}{\rule{0.400pt}{0.361pt}}
\put(955.0,413.0){\rule[-0.200pt]{0.400pt}{0.964pt}}
\put(956.67,399){\rule{0.400pt}{0.964pt}}
\multiput(956.17,401.00)(1.000,-2.000){2}{\rule{0.400pt}{0.482pt}}
\put(957.67,394){\rule{0.400pt}{1.204pt}}
\multiput(957.17,396.50)(1.000,-2.500){2}{\rule{0.400pt}{0.602pt}}
\put(957.0,403.0){\rule[-0.200pt]{0.400pt}{0.964pt}}
\put(958.67,389){\rule{0.400pt}{0.482pt}}
\multiput(958.17,390.00)(1.000,-1.000){2}{\rule{0.400pt}{0.241pt}}
\put(959.0,391.0){\rule[-0.200pt]{0.400pt}{0.723pt}}
\put(959.67,382){\rule{0.400pt}{0.964pt}}
\multiput(959.17,384.00)(1.000,-2.000){2}{\rule{0.400pt}{0.482pt}}
\put(960.67,379){\rule{0.400pt}{0.723pt}}
\multiput(960.17,380.50)(1.000,-1.500){2}{\rule{0.400pt}{0.361pt}}
\put(960.0,386.0){\rule[-0.200pt]{0.400pt}{0.723pt}}
\put(961.67,371){\rule{0.400pt}{0.964pt}}
\multiput(961.17,373.00)(1.000,-2.000){2}{\rule{0.400pt}{0.482pt}}
\put(962.67,368){\rule{0.400pt}{0.723pt}}
\multiput(962.17,369.50)(1.000,-1.500){2}{\rule{0.400pt}{0.361pt}}
\put(962.0,375.0){\rule[-0.200pt]{0.400pt}{0.964pt}}
\put(963.67,362){\rule{0.400pt}{0.723pt}}
\multiput(963.17,363.50)(1.000,-1.500){2}{\rule{0.400pt}{0.361pt}}
\put(964.0,365.0){\rule[-0.200pt]{0.400pt}{0.723pt}}
\put(965,355.67){\rule{0.241pt}{0.400pt}}
\multiput(965.00,356.17)(0.500,-1.000){2}{\rule{0.120pt}{0.400pt}}
\put(965.67,351){\rule{0.400pt}{1.204pt}}
\multiput(965.17,353.50)(1.000,-2.500){2}{\rule{0.400pt}{0.602pt}}
\put(965.0,357.0){\rule[-0.200pt]{0.400pt}{1.204pt}}
\put(966.67,345){\rule{0.400pt}{0.723pt}}
\multiput(966.17,346.50)(1.000,-1.500){2}{\rule{0.400pt}{0.361pt}}
\put(967.0,348.0){\rule[-0.200pt]{0.400pt}{0.723pt}}
\put(967.67,338){\rule{0.400pt}{0.964pt}}
\multiput(967.17,340.00)(1.000,-2.000){2}{\rule{0.400pt}{0.482pt}}
\put(968.67,336){\rule{0.400pt}{0.482pt}}
\multiput(968.17,337.00)(1.000,-1.000){2}{\rule{0.400pt}{0.241pt}}
\put(968.0,342.0){\rule[-0.200pt]{0.400pt}{0.723pt}}
\put(969.67,329){\rule{0.400pt}{0.723pt}}
\multiput(969.17,330.50)(1.000,-1.500){2}{\rule{0.400pt}{0.361pt}}
\put(970.67,327){\rule{0.400pt}{0.482pt}}
\multiput(970.17,328.00)(1.000,-1.000){2}{\rule{0.400pt}{0.241pt}}
\put(970.0,332.0){\rule[-0.200pt]{0.400pt}{0.964pt}}
\put(971.67,321){\rule{0.400pt}{0.482pt}}
\multiput(971.17,322.00)(1.000,-1.000){2}{\rule{0.400pt}{0.241pt}}
\put(972.0,323.0){\rule[-0.200pt]{0.400pt}{0.964pt}}
\put(973,315.67){\rule{0.241pt}{0.400pt}}
\multiput(973.00,316.17)(0.500,-1.000){2}{\rule{0.120pt}{0.400pt}}
\put(973.67,311){\rule{0.400pt}{1.204pt}}
\multiput(973.17,313.50)(1.000,-2.500){2}{\rule{0.400pt}{0.602pt}}
\put(973.0,317.0){\rule[-0.200pt]{0.400pt}{0.964pt}}
\put(974.67,307){\rule{0.400pt}{0.482pt}}
\multiput(974.17,308.00)(1.000,-1.000){2}{\rule{0.400pt}{0.241pt}}
\put(975.67,303){\rule{0.400pt}{0.964pt}}
\multiput(975.17,305.00)(1.000,-2.000){2}{\rule{0.400pt}{0.482pt}}
\put(975.0,309.0){\rule[-0.200pt]{0.400pt}{0.482pt}}
\put(976.67,298){\rule{0.400pt}{0.723pt}}
\multiput(976.17,299.50)(1.000,-1.500){2}{\rule{0.400pt}{0.361pt}}
\put(977.0,301.0){\rule[-0.200pt]{0.400pt}{0.482pt}}
\put(977.67,292){\rule{0.400pt}{0.482pt}}
\multiput(977.17,293.00)(1.000,-1.000){2}{\rule{0.400pt}{0.241pt}}
\put(978.67,289){\rule{0.400pt}{0.723pt}}
\multiput(978.17,290.50)(1.000,-1.500){2}{\rule{0.400pt}{0.361pt}}
\put(978.0,294.0){\rule[-0.200pt]{0.400pt}{0.964pt}}
\put(980,285.67){\rule{0.241pt}{0.400pt}}
\multiput(980.00,286.17)(0.500,-1.000){2}{\rule{0.120pt}{0.400pt}}
\put(980.67,282){\rule{0.400pt}{0.964pt}}
\multiput(980.17,284.00)(1.000,-2.000){2}{\rule{0.400pt}{0.482pt}}
\put(980.0,287.0){\rule[-0.200pt]{0.400pt}{0.482pt}}
\put(981.67,277){\rule{0.400pt}{0.723pt}}
\multiput(981.17,278.50)(1.000,-1.500){2}{\rule{0.400pt}{0.361pt}}
\put(982.0,280.0){\rule[-0.200pt]{0.400pt}{0.482pt}}
\put(982.67,272){\rule{0.400pt}{0.482pt}}
\multiput(982.17,273.00)(1.000,-1.000){2}{\rule{0.400pt}{0.241pt}}
\put(983.67,269){\rule{0.400pt}{0.723pt}}
\multiput(983.17,270.50)(1.000,-1.500){2}{\rule{0.400pt}{0.361pt}}
\put(983.0,274.0){\rule[-0.200pt]{0.400pt}{0.723pt}}
\put(984.67,265){\rule{0.400pt}{0.482pt}}
\multiput(984.17,266.00)(1.000,-1.000){2}{\rule{0.400pt}{0.241pt}}
\put(985.67,262){\rule{0.400pt}{0.723pt}}
\multiput(985.17,263.50)(1.000,-1.500){2}{\rule{0.400pt}{0.361pt}}
\put(985.0,267.0){\rule[-0.200pt]{0.400pt}{0.482pt}}
\put(986.67,258){\rule{0.400pt}{0.723pt}}
\multiput(986.17,259.50)(1.000,-1.500){2}{\rule{0.400pt}{0.361pt}}
\put(987.0,261.0){\usebox{\plotpoint}}
\put(987.67,253){\rule{0.400pt}{0.482pt}}
\multiput(987.17,254.00)(1.000,-1.000){2}{\rule{0.400pt}{0.241pt}}
\put(988.67,251){\rule{0.400pt}{0.482pt}}
\multiput(988.17,252.00)(1.000,-1.000){2}{\rule{0.400pt}{0.241pt}}
\put(988.0,255.0){\rule[-0.200pt]{0.400pt}{0.723pt}}
\put(990,246.67){\rule{0.241pt}{0.400pt}}
\multiput(990.00,247.17)(0.500,-1.000){2}{\rule{0.120pt}{0.400pt}}
\put(990.0,248.0){\rule[-0.200pt]{0.400pt}{0.723pt}}
\put(990.67,244){\rule{0.400pt}{0.482pt}}
\multiput(990.17,245.00)(1.000,-1.000){2}{\rule{0.400pt}{0.241pt}}
\put(991.67,241){\rule{0.400pt}{0.723pt}}
\multiput(991.17,242.50)(1.000,-1.500){2}{\rule{0.400pt}{0.361pt}}
\put(991.0,246.0){\usebox{\plotpoint}}
\put(992.67,237){\rule{0.400pt}{0.482pt}}
\multiput(992.17,238.00)(1.000,-1.000){2}{\rule{0.400pt}{0.241pt}}
\put(993.67,235){\rule{0.400pt}{0.482pt}}
\multiput(993.17,236.00)(1.000,-1.000){2}{\rule{0.400pt}{0.241pt}}
\put(993.0,239.0){\rule[-0.200pt]{0.400pt}{0.482pt}}
\put(994.67,231){\rule{0.400pt}{0.723pt}}
\multiput(994.17,232.50)(1.000,-1.500){2}{\rule{0.400pt}{0.361pt}}
\put(995.0,234.0){\usebox{\plotpoint}}
\put(996.0,228.0){\rule[-0.200pt]{0.400pt}{0.723pt}}
\put(996.67,226){\rule{0.400pt}{0.482pt}}
\multiput(996.17,227.00)(1.000,-1.000){2}{\rule{0.400pt}{0.241pt}}
\put(996.0,228.0){\usebox{\plotpoint}}
\put(998,222.67){\rule{0.241pt}{0.400pt}}
\multiput(998.00,223.17)(0.500,-1.000){2}{\rule{0.120pt}{0.400pt}}
\put(998.67,221){\rule{0.400pt}{0.482pt}}
\multiput(998.17,222.00)(1.000,-1.000){2}{\rule{0.400pt}{0.241pt}}
\put(998.0,224.0){\rule[-0.200pt]{0.400pt}{0.482pt}}
\put(1000,217.67){\rule{0.241pt}{0.400pt}}
\multiput(1000.00,218.17)(0.500,-1.000){2}{\rule{0.120pt}{0.400pt}}
\put(1000.0,219.0){\rule[-0.200pt]{0.400pt}{0.482pt}}
\put(1001,213.67){\rule{0.241pt}{0.400pt}}
\multiput(1001.00,214.17)(0.500,-1.000){2}{\rule{0.120pt}{0.400pt}}
\put(1001.0,215.0){\rule[-0.200pt]{0.400pt}{0.723pt}}
\put(1002.0,214.0){\usebox{\plotpoint}}
\put(1002.67,209){\rule{0.400pt}{0.482pt}}
\multiput(1002.17,210.00)(1.000,-1.000){2}{\rule{0.400pt}{0.241pt}}
\put(1003.0,211.0){\rule[-0.200pt]{0.400pt}{0.723pt}}
\put(1003.67,206){\rule{0.400pt}{0.482pt}}
\multiput(1003.17,207.00)(1.000,-1.000){2}{\rule{0.400pt}{0.241pt}}
\put(1004.0,208.0){\usebox{\plotpoint}}
\put(1005.0,206.0){\usebox{\plotpoint}}
\put(1005.67,202){\rule{0.400pt}{0.482pt}}
\multiput(1005.17,203.00)(1.000,-1.000){2}{\rule{0.400pt}{0.241pt}}
\put(1006.0,204.0){\rule[-0.200pt]{0.400pt}{0.482pt}}
\put(1007.0,202.0){\usebox{\plotpoint}}
\put(1008.0,199.0){\rule[-0.200pt]{0.400pt}{0.723pt}}
\put(1008.0,199.0){\usebox{\plotpoint}}
\put(1009.0,196.0){\rule[-0.200pt]{0.400pt}{0.723pt}}
\put(1010,194.67){\rule{0.241pt}{0.400pt}}
\multiput(1010.00,195.17)(0.500,-1.000){2}{\rule{0.120pt}{0.400pt}}
\put(1009.0,196.0){\usebox{\plotpoint}}
\put(1011.0,193.0){\rule[-0.200pt]{0.400pt}{0.482pt}}
\put(1011.67,191){\rule{0.400pt}{0.482pt}}
\multiput(1011.17,192.00)(1.000,-1.000){2}{\rule{0.400pt}{0.241pt}}
\put(1011.0,193.0){\usebox{\plotpoint}}
\put(1013,187.67){\rule{0.241pt}{0.400pt}}
\multiput(1013.00,188.17)(0.500,-1.000){2}{\rule{0.120pt}{0.400pt}}
\put(1013.0,189.0){\rule[-0.200pt]{0.400pt}{0.482pt}}
\put(1014,185.67){\rule{0.241pt}{0.400pt}}
\multiput(1014.00,186.17)(0.500,-1.000){2}{\rule{0.120pt}{0.400pt}}
\put(1015,184.67){\rule{0.241pt}{0.400pt}}
\multiput(1015.00,185.17)(0.500,-1.000){2}{\rule{0.120pt}{0.400pt}}
\put(1014.0,187.0){\usebox{\plotpoint}}
\put(1016.0,183.0){\rule[-0.200pt]{0.400pt}{0.482pt}}
\put(1016.0,183.0){\usebox{\plotpoint}}
\put(1017,179.67){\rule{0.241pt}{0.400pt}}
\multiput(1017.00,180.17)(0.500,-1.000){2}{\rule{0.120pt}{0.400pt}}
\put(1018,178.67){\rule{0.241pt}{0.400pt}}
\multiput(1018.00,179.17)(0.500,-1.000){2}{\rule{0.120pt}{0.400pt}}
\put(1017.0,181.0){\rule[-0.200pt]{0.400pt}{0.482pt}}
\put(1019,176.67){\rule{0.241pt}{0.400pt}}
\multiput(1019.00,177.17)(0.500,-1.000){2}{\rule{0.120pt}{0.400pt}}
\put(1019.0,178.0){\usebox{\plotpoint}}
\put(1020.0,177.0){\usebox{\plotpoint}}
\put(1021.0,176.0){\usebox{\plotpoint}}
\put(1021.0,176.0){\usebox{\plotpoint}}
\put(1022.0,173.0){\rule[-0.200pt]{0.400pt}{0.723pt}}
\put(1023,171.67){\rule{0.241pt}{0.400pt}}
\multiput(1023.00,172.17)(0.500,-1.000){2}{\rule{0.120pt}{0.400pt}}
\put(1022.0,173.0){\usebox{\plotpoint}}
\put(1024,172){\usebox{\plotpoint}}
\put(1023.67,170){\rule{0.400pt}{0.482pt}}
\multiput(1023.17,171.00)(1.000,-1.000){2}{\rule{0.400pt}{0.241pt}}
\put(1025.0,170.0){\usebox{\plotpoint}}
\put(1026,167.67){\rule{0.241pt}{0.400pt}}
\multiput(1026.00,168.17)(0.500,-1.000){2}{\rule{0.120pt}{0.400pt}}
\put(1026.0,169.0){\usebox{\plotpoint}}
\put(1027.0,167.0){\usebox{\plotpoint}}
\put(1028,165.67){\rule{0.241pt}{0.400pt}}
\multiput(1028.00,166.17)(0.500,-1.000){2}{\rule{0.120pt}{0.400pt}}
\put(1027.0,167.0){\usebox{\plotpoint}}
\put(1029.0,164.0){\rule[-0.200pt]{0.400pt}{0.482pt}}
\put(1029.0,164.0){\usebox{\plotpoint}}
\put(1030.0,163.0){\usebox{\plotpoint}}
\put(1031,161.67){\rule{0.241pt}{0.400pt}}
\multiput(1031.00,162.17)(0.500,-1.000){2}{\rule{0.120pt}{0.400pt}}
\put(1030.0,163.0){\usebox{\plotpoint}}
\put(1032,162){\usebox{\plotpoint}}
\put(1032,160.67){\rule{0.241pt}{0.400pt}}
\multiput(1032.00,161.17)(0.500,-1.000){2}{\rule{0.120pt}{0.400pt}}
\put(1033,159.67){\rule{0.241pt}{0.400pt}}
\multiput(1033.00,160.17)(0.500,-1.000){2}{\rule{0.120pt}{0.400pt}}
\put(1034,158.67){\rule{0.241pt}{0.400pt}}
\multiput(1034.00,158.17)(0.500,1.000){2}{\rule{0.120pt}{0.400pt}}
\put(1034.0,159.0){\usebox{\plotpoint}}
\put(1035,156.67){\rule{0.241pt}{0.400pt}}
\multiput(1035.00,157.17)(0.500,-1.000){2}{\rule{0.120pt}{0.400pt}}
\put(1036,155.67){\rule{0.241pt}{0.400pt}}
\multiput(1036.00,156.17)(0.500,-1.000){2}{\rule{0.120pt}{0.400pt}}
\put(1035.0,158.0){\rule[-0.200pt]{0.400pt}{0.482pt}}
\put(1037,155.67){\rule{0.241pt}{0.400pt}}
\multiput(1037.00,156.17)(0.500,-1.000){2}{\rule{0.120pt}{0.400pt}}
\put(1038,154.67){\rule{0.241pt}{0.400pt}}
\multiput(1038.00,155.17)(0.500,-1.000){2}{\rule{0.120pt}{0.400pt}}
\put(1037.0,156.0){\usebox{\plotpoint}}
\put(1039,154.67){\rule{0.241pt}{0.400pt}}
\multiput(1039.00,155.17)(0.500,-1.000){2}{\rule{0.120pt}{0.400pt}}
\put(1039.0,155.0){\usebox{\plotpoint}}
\put(1040,152.67){\rule{0.241pt}{0.400pt}}
\multiput(1040.00,153.17)(0.500,-1.000){2}{\rule{0.120pt}{0.400pt}}
\put(1040.0,154.0){\usebox{\plotpoint}}
\put(1041.0,153.0){\usebox{\plotpoint}}
\put(1042.0,152.0){\usebox{\plotpoint}}
\put(1042.0,152.0){\usebox{\plotpoint}}
\put(1043.0,151.0){\usebox{\plotpoint}}
\put(1044,149.67){\rule{0.241pt}{0.400pt}}
\multiput(1044.00,150.17)(0.500,-1.000){2}{\rule{0.120pt}{0.400pt}}
\put(1043.0,151.0){\usebox{\plotpoint}}
\put(1045,150){\usebox{\plotpoint}}
\put(1046,148.67){\rule{0.241pt}{0.400pt}}
\multiput(1046.00,149.17)(0.500,-1.000){2}{\rule{0.120pt}{0.400pt}}
\put(1045.0,150.0){\usebox{\plotpoint}}
\put(1047,149){\usebox{\plotpoint}}
\put(1047,147.67){\rule{0.241pt}{0.400pt}}
\multiput(1047.00,148.17)(0.500,-1.000){2}{\rule{0.120pt}{0.400pt}}
\put(1048,148){\usebox{\plotpoint}}
\put(1049,146.67){\rule{0.241pt}{0.400pt}}
\multiput(1049.00,147.17)(0.500,-1.000){2}{\rule{0.120pt}{0.400pt}}
\put(1048.0,148.0){\usebox{\plotpoint}}
\put(1050.0,146.0){\usebox{\plotpoint}}
\put(1050.0,146.0){\usebox{\plotpoint}}
\put(1051,144.67){\rule{0.241pt}{0.400pt}}
\multiput(1051.00,144.17)(0.500,1.000){2}{\rule{0.120pt}{0.400pt}}
\put(1051.67,144){\rule{0.400pt}{0.482pt}}
\multiput(1051.17,145.00)(1.000,-1.000){2}{\rule{0.400pt}{0.241pt}}
\put(1051.0,145.0){\usebox{\plotpoint}}
\put(1052.67,144){\rule{0.400pt}{0.482pt}}
\multiput(1052.17,145.00)(1.000,-1.000){2}{\rule{0.400pt}{0.241pt}}
\put(1054,142.67){\rule{0.241pt}{0.400pt}}
\multiput(1054.00,143.17)(0.500,-1.000){2}{\rule{0.120pt}{0.400pt}}
\put(1053.0,144.0){\rule[-0.200pt]{0.400pt}{0.482pt}}
\put(1055,142.67){\rule{0.241pt}{0.400pt}}
\multiput(1055.00,143.17)(0.500,-1.000){2}{\rule{0.120pt}{0.400pt}}
\put(1055.0,143.0){\usebox{\plotpoint}}
\put(1056,143){\usebox{\plotpoint}}
\put(1056,141.67){\rule{0.241pt}{0.400pt}}
\multiput(1056.00,142.17)(0.500,-1.000){2}{\rule{0.120pt}{0.400pt}}
\put(1057,141.67){\rule{0.241pt}{0.400pt}}
\multiput(1057.00,141.17)(0.500,1.000){2}{\rule{0.120pt}{0.400pt}}
\put(1058,140.67){\rule{0.241pt}{0.400pt}}
\multiput(1058.00,141.17)(0.500,-1.000){2}{\rule{0.120pt}{0.400pt}}
\put(1058.0,142.0){\usebox{\plotpoint}}
\put(1059,141){\usebox{\plotpoint}}
\put(1059.0,141.0){\rule[-0.200pt]{0.482pt}{0.400pt}}
\put(1061.0,140.0){\usebox{\plotpoint}}
\put(1062,138.67){\rule{0.241pt}{0.400pt}}
\multiput(1062.00,139.17)(0.500,-1.000){2}{\rule{0.120pt}{0.400pt}}
\put(1061.0,140.0){\usebox{\plotpoint}}
\put(1063,139){\usebox{\plotpoint}}
\put(1064,137.67){\rule{0.241pt}{0.400pt}}
\multiput(1064.00,138.17)(0.500,-1.000){2}{\rule{0.120pt}{0.400pt}}
\put(1063.0,139.0){\usebox{\plotpoint}}
\put(1067,136.67){\rule{0.241pt}{0.400pt}}
\multiput(1067.00,137.17)(0.500,-1.000){2}{\rule{0.120pt}{0.400pt}}
\put(1068,136.67){\rule{0.241pt}{0.400pt}}
\multiput(1068.00,136.17)(0.500,1.000){2}{\rule{0.120pt}{0.400pt}}
\put(1065.0,138.0){\rule[-0.200pt]{0.482pt}{0.400pt}}
\put(1069,135.67){\rule{0.241pt}{0.400pt}}
\multiput(1069.00,136.17)(0.500,-1.000){2}{\rule{0.120pt}{0.400pt}}
\put(1070,135.67){\rule{0.241pt}{0.400pt}}
\multiput(1070.00,135.17)(0.500,1.000){2}{\rule{0.120pt}{0.400pt}}
\put(1069.0,137.0){\usebox{\plotpoint}}
\put(1071,136.67){\rule{0.241pt}{0.400pt}}
\multiput(1071.00,137.17)(0.500,-1.000){2}{\rule{0.120pt}{0.400pt}}
\put(1071.0,137.0){\usebox{\plotpoint}}
\put(1072.0,136.0){\usebox{\plotpoint}}
\put(1072.0,136.0){\rule[-0.200pt]{0.482pt}{0.400pt}}
\put(1074,134.67){\rule{0.241pt}{0.400pt}}
\multiput(1074.00,134.17)(0.500,1.000){2}{\rule{0.120pt}{0.400pt}}
\put(1074.0,135.0){\usebox{\plotpoint}}
\put(1075.0,135.0){\usebox{\plotpoint}}
\put(1078,133.67){\rule{0.241pt}{0.400pt}}
\multiput(1078.00,134.17)(0.500,-1.000){2}{\rule{0.120pt}{0.400pt}}
\put(1075.0,135.0){\rule[-0.200pt]{0.723pt}{0.400pt}}
\put(1079,133.67){\rule{0.241pt}{0.400pt}}
\multiput(1079.00,134.17)(0.500,-1.000){2}{\rule{0.120pt}{0.400pt}}
\put(1079.0,134.0){\usebox{\plotpoint}}
\put(1080,134){\usebox{\plotpoint}}
\put(1080,133.67){\rule{0.241pt}{0.400pt}}
\multiput(1080.00,133.17)(0.500,1.000){2}{\rule{0.120pt}{0.400pt}}
\put(1080.67,133){\rule{0.400pt}{0.482pt}}
\multiput(1080.17,134.00)(1.000,-1.000){2}{\rule{0.400pt}{0.241pt}}
\put(1082,133){\usebox{\plotpoint}}
\put(1083,131.67){\rule{0.241pt}{0.400pt}}
\multiput(1083.00,132.17)(0.500,-1.000){2}{\rule{0.120pt}{0.400pt}}
\put(1084,131.67){\rule{0.241pt}{0.400pt}}
\multiput(1084.00,131.17)(0.500,1.000){2}{\rule{0.120pt}{0.400pt}}
\put(1082.0,133.0){\usebox{\plotpoint}}
\put(1085,133){\usebox{\plotpoint}}
\put(1087,131.67){\rule{0.241pt}{0.400pt}}
\multiput(1087.00,132.17)(0.500,-1.000){2}{\rule{0.120pt}{0.400pt}}
\put(1085.0,133.0){\rule[-0.200pt]{0.482pt}{0.400pt}}
\put(1088,132){\usebox{\plotpoint}}
\put(1088,131.67){\rule{0.241pt}{0.400pt}}
\multiput(1088.00,131.17)(0.500,1.000){2}{\rule{0.120pt}{0.400pt}}
\put(1089,131.67){\rule{0.241pt}{0.400pt}}
\multiput(1089.00,132.17)(0.500,-1.000){2}{\rule{0.120pt}{0.400pt}}
\put(1090,132){\usebox{\plotpoint}}
\put(1090.0,132.0){\usebox{\plotpoint}}
\put(1091.0,131.0){\usebox{\plotpoint}}
\put(1091.0,131.0){\rule[-0.200pt]{0.482pt}{0.400pt}}
\put(1093,130.67){\rule{0.241pt}{0.400pt}}
\multiput(1093.00,131.17)(0.500,-1.000){2}{\rule{0.120pt}{0.400pt}}
\put(1093.0,131.0){\usebox{\plotpoint}}
\put(1094.0,131.0){\usebox{\plotpoint}}
\put(1094.0,132.0){\rule[-0.200pt]{0.482pt}{0.400pt}}
\put(1096,130.67){\rule{0.241pt}{0.400pt}}
\multiput(1096.00,130.17)(0.500,1.000){2}{\rule{0.120pt}{0.400pt}}
\put(1097,130.67){\rule{0.241pt}{0.400pt}}
\multiput(1097.00,131.17)(0.500,-1.000){2}{\rule{0.120pt}{0.400pt}}
\put(1096.0,131.0){\usebox{\plotpoint}}
\put(1098,131){\usebox{\plotpoint}}
\put(1098,131){\usebox{\plotpoint}}
\put(1101,131){\usebox{\plotpoint}}
\put(1101,131){\usebox{\plotpoint}}
\put(171,308.67){\rule{0.241pt}{0.400pt}}
\multiput(171.00,308.17)(0.500,1.000){2}{\rule{0.120pt}{0.400pt}}
\put(172.0,310.0){\rule[-0.200pt]{0.400pt}{0.723pt}}
\put(173,311.67){\rule{0.241pt}{0.400pt}}
\multiput(173.00,312.17)(0.500,-1.000){2}{\rule{0.120pt}{0.400pt}}
\put(172.0,313.0){\usebox{\plotpoint}}
\put(173.67,312){\rule{0.400pt}{0.482pt}}
\multiput(173.17,313.00)(1.000,-1.000){2}{\rule{0.400pt}{0.241pt}}
\put(174.67,312){\rule{0.400pt}{0.723pt}}
\multiput(174.17,312.00)(1.000,1.500){2}{\rule{0.400pt}{0.361pt}}
\put(174.0,312.0){\rule[-0.200pt]{0.400pt}{0.482pt}}
\put(176,314.67){\rule{0.241pt}{0.400pt}}
\multiput(176.00,315.17)(0.500,-1.000){2}{\rule{0.120pt}{0.400pt}}
\put(176.67,315){\rule{0.400pt}{0.482pt}}
\multiput(176.17,315.00)(1.000,1.000){2}{\rule{0.400pt}{0.241pt}}
\put(176.0,315.0){\usebox{\plotpoint}}
\put(178,317){\usebox{\plotpoint}}
\put(178,316.67){\rule{0.241pt}{0.400pt}}
\multiput(178.00,316.17)(0.500,1.000){2}{\rule{0.120pt}{0.400pt}}
\put(178.67,316){\rule{0.400pt}{0.482pt}}
\multiput(178.17,317.00)(1.000,-1.000){2}{\rule{0.400pt}{0.241pt}}
\put(179.67,317){\rule{0.400pt}{0.723pt}}
\multiput(179.17,318.50)(1.000,-1.500){2}{\rule{0.400pt}{0.361pt}}
\put(180.67,317){\rule{0.400pt}{0.964pt}}
\multiput(180.17,317.00)(1.000,2.000){2}{\rule{0.400pt}{0.482pt}}
\put(180.0,316.0){\rule[-0.200pt]{0.400pt}{0.964pt}}
\put(182,321){\usebox{\plotpoint}}
\put(182,319.67){\rule{0.241pt}{0.400pt}}
\multiput(182.00,320.17)(0.500,-1.000){2}{\rule{0.120pt}{0.400pt}}
\put(183.0,320.0){\usebox{\plotpoint}}
\put(183.67,320){\rule{0.400pt}{0.723pt}}
\multiput(183.17,321.50)(1.000,-1.500){2}{\rule{0.400pt}{0.361pt}}
\put(184.67,320){\rule{0.400pt}{0.482pt}}
\multiput(184.17,320.00)(1.000,1.000){2}{\rule{0.400pt}{0.241pt}}
\put(184.0,320.0){\rule[-0.200pt]{0.400pt}{0.723pt}}
\put(186,320.67){\rule{0.241pt}{0.400pt}}
\multiput(186.00,320.17)(0.500,1.000){2}{\rule{0.120pt}{0.400pt}}
\put(186.67,322){\rule{0.400pt}{0.482pt}}
\multiput(186.17,322.00)(1.000,1.000){2}{\rule{0.400pt}{0.241pt}}
\put(186.0,321.0){\usebox{\plotpoint}}
\put(187.67,324){\rule{0.400pt}{0.482pt}}
\multiput(187.17,325.00)(1.000,-1.000){2}{\rule{0.400pt}{0.241pt}}
\put(188.67,324){\rule{0.400pt}{0.964pt}}
\multiput(188.17,324.00)(1.000,2.000){2}{\rule{0.400pt}{0.482pt}}
\put(189.67,325){\rule{0.400pt}{0.723pt}}
\multiput(189.17,326.50)(1.000,-1.500){2}{\rule{0.400pt}{0.361pt}}
\put(188.0,324.0){\rule[-0.200pt]{0.400pt}{0.482pt}}
\put(190.67,324){\rule{0.400pt}{0.482pt}}
\multiput(190.17,325.00)(1.000,-1.000){2}{\rule{0.400pt}{0.241pt}}
\put(191.67,324){\rule{0.400pt}{0.964pt}}
\multiput(191.17,324.00)(1.000,2.000){2}{\rule{0.400pt}{0.482pt}}
\put(191.0,325.0){\usebox{\plotpoint}}
\put(193,326.67){\rule{0.241pt}{0.400pt}}
\multiput(193.00,326.17)(0.500,1.000){2}{\rule{0.120pt}{0.400pt}}
\put(194,326.67){\rule{0.241pt}{0.400pt}}
\multiput(194.00,327.17)(0.500,-1.000){2}{\rule{0.120pt}{0.400pt}}
\put(193.0,327.0){\usebox{\plotpoint}}
\put(194.67,327){\rule{0.400pt}{0.482pt}}
\multiput(194.17,328.00)(1.000,-1.000){2}{\rule{0.400pt}{0.241pt}}
\put(196,326.67){\rule{0.241pt}{0.400pt}}
\multiput(196.00,326.17)(0.500,1.000){2}{\rule{0.120pt}{0.400pt}}
\put(195.0,327.0){\rule[-0.200pt]{0.400pt}{0.482pt}}
\put(196.67,327){\rule{0.400pt}{0.482pt}}
\multiput(196.17,328.00)(1.000,-1.000){2}{\rule{0.400pt}{0.241pt}}
\put(197.0,328.0){\usebox{\plotpoint}}
\put(198.0,327.0){\usebox{\plotpoint}}
\put(199,327.67){\rule{0.241pt}{0.400pt}}
\multiput(199.00,328.17)(0.500,-1.000){2}{\rule{0.120pt}{0.400pt}}
\put(199.0,327.0){\rule[-0.200pt]{0.400pt}{0.482pt}}
\put(200.0,328.0){\usebox{\plotpoint}}
\put(201.0,328.0){\usebox{\plotpoint}}
\put(202,327.67){\rule{0.241pt}{0.400pt}}
\multiput(202.00,328.17)(0.500,-1.000){2}{\rule{0.120pt}{0.400pt}}
\put(201.0,329.0){\usebox{\plotpoint}}
\put(203,328){\usebox{\plotpoint}}
\put(202.67,326){\rule{0.400pt}{0.482pt}}
\multiput(202.17,327.00)(1.000,-1.000){2}{\rule{0.400pt}{0.241pt}}
\put(203.67,326){\rule{0.400pt}{0.482pt}}
\multiput(203.17,326.00)(1.000,1.000){2}{\rule{0.400pt}{0.241pt}}
\put(204.67,327){\rule{0.400pt}{0.482pt}}
\multiput(204.17,327.00)(1.000,1.000){2}{\rule{0.400pt}{0.241pt}}
\put(205.67,327){\rule{0.400pt}{0.482pt}}
\multiput(205.17,328.00)(1.000,-1.000){2}{\rule{0.400pt}{0.241pt}}
\put(205.0,327.0){\usebox{\plotpoint}}
\put(207.0,327.0){\usebox{\plotpoint}}
\put(208,327.67){\rule{0.241pt}{0.400pt}}
\multiput(208.00,327.17)(0.500,1.000){2}{\rule{0.120pt}{0.400pt}}
\put(207.0,328.0){\usebox{\plotpoint}}
\put(209,326.67){\rule{0.241pt}{0.400pt}}
\multiput(209.00,326.17)(0.500,1.000){2}{\rule{0.120pt}{0.400pt}}
\put(209.67,326){\rule{0.400pt}{0.482pt}}
\multiput(209.17,327.00)(1.000,-1.000){2}{\rule{0.400pt}{0.241pt}}
\put(209.0,327.0){\rule[-0.200pt]{0.400pt}{0.482pt}}
\put(210.67,326){\rule{0.400pt}{0.723pt}}
\multiput(210.17,327.50)(1.000,-1.500){2}{\rule{0.400pt}{0.361pt}}
\put(212,325.67){\rule{0.241pt}{0.400pt}}
\multiput(212.00,325.17)(0.500,1.000){2}{\rule{0.120pt}{0.400pt}}
\put(211.0,326.0){\rule[-0.200pt]{0.400pt}{0.723pt}}
\put(213.0,326.0){\usebox{\plotpoint}}
\put(213.0,326.0){\rule[-0.200pt]{0.482pt}{0.400pt}}
\put(215.0,325.0){\usebox{\plotpoint}}
\put(215.0,325.0){\rule[-0.200pt]{0.482pt}{0.400pt}}
\put(217,323.67){\rule{0.241pt}{0.400pt}}
\multiput(217.00,323.17)(0.500,1.000){2}{\rule{0.120pt}{0.400pt}}
\put(217.0,324.0){\usebox{\plotpoint}}
\put(218.0,325.0){\usebox{\plotpoint}}
\put(219,323.67){\rule{0.241pt}{0.400pt}}
\multiput(219.00,323.17)(0.500,1.000){2}{\rule{0.120pt}{0.400pt}}
\put(219.67,323){\rule{0.400pt}{0.482pt}}
\multiput(219.17,324.00)(1.000,-1.000){2}{\rule{0.400pt}{0.241pt}}
\put(219.0,324.0){\usebox{\plotpoint}}
\put(220.67,321){\rule{0.400pt}{0.482pt}}
\multiput(220.17,321.00)(1.000,1.000){2}{\rule{0.400pt}{0.241pt}}
\put(221.0,321.0){\rule[-0.200pt]{0.400pt}{0.482pt}}
\put(222.0,323.0){\usebox{\plotpoint}}
\put(223.0,321.0){\rule[-0.200pt]{0.400pt}{0.482pt}}
\put(224.67,319){\rule{0.400pt}{0.482pt}}
\multiput(224.17,320.00)(1.000,-1.000){2}{\rule{0.400pt}{0.241pt}}
\put(226,317.67){\rule{0.241pt}{0.400pt}}
\multiput(226.00,318.17)(0.500,-1.000){2}{\rule{0.120pt}{0.400pt}}
\put(223.0,321.0){\rule[-0.200pt]{0.482pt}{0.400pt}}
\put(226.67,317){\rule{0.400pt}{0.482pt}}
\multiput(226.17,318.00)(1.000,-1.000){2}{\rule{0.400pt}{0.241pt}}
\put(228,316.67){\rule{0.241pt}{0.400pt}}
\multiput(228.00,316.17)(0.500,1.000){2}{\rule{0.120pt}{0.400pt}}
\put(227.0,318.0){\usebox{\plotpoint}}
\put(229,315.67){\rule{0.241pt}{0.400pt}}
\multiput(229.00,315.17)(0.500,1.000){2}{\rule{0.120pt}{0.400pt}}
\put(230,315.67){\rule{0.241pt}{0.400pt}}
\multiput(230.00,316.17)(0.500,-1.000){2}{\rule{0.120pt}{0.400pt}}
\put(229.0,316.0){\rule[-0.200pt]{0.400pt}{0.482pt}}
\put(231,316){\usebox{\plotpoint}}
\put(231,314.67){\rule{0.241pt}{0.400pt}}
\multiput(231.00,315.17)(0.500,-1.000){2}{\rule{0.120pt}{0.400pt}}
\put(232.0,315.0){\usebox{\plotpoint}}
\put(232.67,313){\rule{0.400pt}{0.482pt}}
\multiput(232.17,313.00)(1.000,1.000){2}{\rule{0.400pt}{0.241pt}}
\put(234,313.67){\rule{0.241pt}{0.400pt}}
\multiput(234.00,314.17)(0.500,-1.000){2}{\rule{0.120pt}{0.400pt}}
\put(235,312.67){\rule{0.241pt}{0.400pt}}
\multiput(235.00,313.17)(0.500,-1.000){2}{\rule{0.120pt}{0.400pt}}
\put(233.0,313.0){\rule[-0.200pt]{0.400pt}{0.482pt}}
\put(236.0,312.0){\usebox{\plotpoint}}
\put(236.0,312.0){\rule[-0.200pt]{0.482pt}{0.400pt}}
\put(238,309.67){\rule{0.241pt}{0.400pt}}
\multiput(238.00,310.17)(0.500,-1.000){2}{\rule{0.120pt}{0.400pt}}
\put(238.0,311.0){\usebox{\plotpoint}}
\put(240,308.67){\rule{0.241pt}{0.400pt}}
\multiput(240.00,309.17)(0.500,-1.000){2}{\rule{0.120pt}{0.400pt}}
\put(241,308.67){\rule{0.241pt}{0.400pt}}
\multiput(241.00,308.17)(0.500,1.000){2}{\rule{0.120pt}{0.400pt}}
\put(239.0,310.0){\usebox{\plotpoint}}
\put(242.0,308.0){\rule[-0.200pt]{0.400pt}{0.482pt}}
\put(244,306.67){\rule{0.241pt}{0.400pt}}
\multiput(244.00,307.17)(0.500,-1.000){2}{\rule{0.120pt}{0.400pt}}
\put(244.67,305){\rule{0.400pt}{0.482pt}}
\multiput(244.17,306.00)(1.000,-1.000){2}{\rule{0.400pt}{0.241pt}}
\put(242.0,308.0){\rule[-0.200pt]{0.482pt}{0.400pt}}
\put(246.0,305.0){\usebox{\plotpoint}}
\put(247,304.67){\rule{0.241pt}{0.400pt}}
\multiput(247.00,305.17)(0.500,-1.000){2}{\rule{0.120pt}{0.400pt}}
\put(246.0,306.0){\usebox{\plotpoint}}
\put(248,304.67){\rule{0.241pt}{0.400pt}}
\multiput(248.00,305.17)(0.500,-1.000){2}{\rule{0.120pt}{0.400pt}}
\put(249,303.67){\rule{0.241pt}{0.400pt}}
\multiput(249.00,304.17)(0.500,-1.000){2}{\rule{0.120pt}{0.400pt}}
\put(248.0,305.0){\usebox{\plotpoint}}
\put(249.67,303){\rule{0.400pt}{0.482pt}}
\multiput(249.17,303.00)(1.000,1.000){2}{\rule{0.400pt}{0.241pt}}
\put(250.67,302){\rule{0.400pt}{0.723pt}}
\multiput(250.17,303.50)(1.000,-1.500){2}{\rule{0.400pt}{0.361pt}}
\put(250.0,303.0){\usebox{\plotpoint}}
\put(252,301.67){\rule{0.241pt}{0.400pt}}
\multiput(252.00,302.17)(0.500,-1.000){2}{\rule{0.120pt}{0.400pt}}
\put(252.0,302.0){\usebox{\plotpoint}}
\put(253.0,302.0){\usebox{\plotpoint}}
\put(254.0,301.0){\usebox{\plotpoint}}
\put(256,300.67){\rule{0.241pt}{0.400pt}}
\multiput(256.00,300.17)(0.500,1.000){2}{\rule{0.120pt}{0.400pt}}
\put(256.67,300){\rule{0.400pt}{0.482pt}}
\multiput(256.17,301.00)(1.000,-1.000){2}{\rule{0.400pt}{0.241pt}}
\put(254.0,301.0){\rule[-0.200pt]{0.482pt}{0.400pt}}
\put(258,300){\usebox{\plotpoint}}
\put(258,298.67){\rule{0.241pt}{0.400pt}}
\multiput(258.00,299.17)(0.500,-1.000){2}{\rule{0.120pt}{0.400pt}}
\put(259.67,297){\rule{0.400pt}{0.482pt}}
\multiput(259.17,298.00)(1.000,-1.000){2}{\rule{0.400pt}{0.241pt}}
\put(260.67,297){\rule{0.400pt}{0.482pt}}
\multiput(260.17,297.00)(1.000,1.000){2}{\rule{0.400pt}{0.241pt}}
\put(259.0,299.0){\usebox{\plotpoint}}
\put(262,299){\usebox{\plotpoint}}
\put(262.67,297){\rule{0.400pt}{0.482pt}}
\multiput(262.17,298.00)(1.000,-1.000){2}{\rule{0.400pt}{0.241pt}}
\put(262.0,299.0){\usebox{\plotpoint}}
\put(264.0,297.0){\usebox{\plotpoint}}
\put(266,296.67){\rule{0.241pt}{0.400pt}}
\multiput(266.00,297.17)(0.500,-1.000){2}{\rule{0.120pt}{0.400pt}}
\put(264.0,298.0){\rule[-0.200pt]{0.482pt}{0.400pt}}
\put(268,295.67){\rule{0.241pt}{0.400pt}}
\multiput(268.00,296.17)(0.500,-1.000){2}{\rule{0.120pt}{0.400pt}}
\put(269,295.67){\rule{0.241pt}{0.400pt}}
\multiput(269.00,295.17)(0.500,1.000){2}{\rule{0.120pt}{0.400pt}}
\put(267.0,297.0){\usebox{\plotpoint}}
\put(270.0,296.0){\usebox{\plotpoint}}
\put(271,294.67){\rule{0.241pt}{0.400pt}}
\multiput(271.00,295.17)(0.500,-1.000){2}{\rule{0.120pt}{0.400pt}}
\put(270.0,296.0){\usebox{\plotpoint}}
\put(272.0,295.0){\usebox{\plotpoint}}
\put(273,295.67){\rule{0.241pt}{0.400pt}}
\multiput(273.00,295.17)(0.500,1.000){2}{\rule{0.120pt}{0.400pt}}
\put(272.0,296.0){\usebox{\plotpoint}}
\put(274,294.67){\rule{0.241pt}{0.400pt}}
\multiput(274.00,295.17)(0.500,-1.000){2}{\rule{0.120pt}{0.400pt}}
\put(274.0,296.0){\usebox{\plotpoint}}
\put(276,294.67){\rule{0.241pt}{0.400pt}}
\multiput(276.00,294.17)(0.500,1.000){2}{\rule{0.120pt}{0.400pt}}
\put(275.0,295.0){\usebox{\plotpoint}}
\put(278,294.67){\rule{0.241pt}{0.400pt}}
\multiput(278.00,295.17)(0.500,-1.000){2}{\rule{0.120pt}{0.400pt}}
\put(279,294.67){\rule{0.241pt}{0.400pt}}
\multiput(279.00,294.17)(0.500,1.000){2}{\rule{0.120pt}{0.400pt}}
\put(277.0,296.0){\usebox{\plotpoint}}
\put(280,296){\usebox{\plotpoint}}
\put(281,294.67){\rule{0.241pt}{0.400pt}}
\multiput(281.00,295.17)(0.500,-1.000){2}{\rule{0.120pt}{0.400pt}}
\put(280.0,296.0){\usebox{\plotpoint}}
\put(282,295){\usebox{\plotpoint}}
\put(282,294.67){\rule{0.241pt}{0.400pt}}
\multiput(282.00,294.17)(0.500,1.000){2}{\rule{0.120pt}{0.400pt}}
\put(286.67,296){\rule{0.400pt}{0.482pt}}
\multiput(286.17,296.00)(1.000,1.000){2}{\rule{0.400pt}{0.241pt}}
\put(283.0,296.0){\rule[-0.200pt]{0.964pt}{0.400pt}}
\put(288,298){\usebox{\plotpoint}}
\put(288,296.67){\rule{0.241pt}{0.400pt}}
\multiput(288.00,297.17)(0.500,-1.000){2}{\rule{0.120pt}{0.400pt}}
\put(288.67,297){\rule{0.400pt}{0.482pt}}
\multiput(288.17,297.00)(1.000,1.000){2}{\rule{0.400pt}{0.241pt}}
\put(290,296.67){\rule{0.241pt}{0.400pt}}
\multiput(290.00,296.17)(0.500,1.000){2}{\rule{0.120pt}{0.400pt}}
\put(290.67,298){\rule{0.400pt}{0.482pt}}
\multiput(290.17,298.00)(1.000,1.000){2}{\rule{0.400pt}{0.241pt}}
\put(290.0,297.0){\rule[-0.200pt]{0.400pt}{0.482pt}}
\put(292.0,299.0){\usebox{\plotpoint}}
\put(293,298.67){\rule{0.241pt}{0.400pt}}
\multiput(293.00,298.17)(0.500,1.000){2}{\rule{0.120pt}{0.400pt}}
\put(292.0,299.0){\usebox{\plotpoint}}
\put(294,298.67){\rule{0.241pt}{0.400pt}}
\multiput(294.00,298.17)(0.500,1.000){2}{\rule{0.120pt}{0.400pt}}
\put(294.67,300){\rule{0.400pt}{0.482pt}}
\multiput(294.17,300.00)(1.000,1.000){2}{\rule{0.400pt}{0.241pt}}
\put(294.0,299.0){\usebox{\plotpoint}}
\put(295.67,300){\rule{0.400pt}{0.482pt}}
\multiput(295.17,300.00)(1.000,1.000){2}{\rule{0.400pt}{0.241pt}}
\put(296.0,300.0){\rule[-0.200pt]{0.400pt}{0.482pt}}
\put(297.0,302.0){\usebox{\plotpoint}}
\put(298,300.67){\rule{0.241pt}{0.400pt}}
\multiput(298.00,300.17)(0.500,1.000){2}{\rule{0.120pt}{0.400pt}}
\put(299,301.67){\rule{0.241pt}{0.400pt}}
\multiput(299.00,301.17)(0.500,1.000){2}{\rule{0.120pt}{0.400pt}}
\put(298.0,301.0){\usebox{\plotpoint}}
\put(300,302.67){\rule{0.241pt}{0.400pt}}
\multiput(300.00,303.17)(0.500,-1.000){2}{\rule{0.120pt}{0.400pt}}
\put(300.67,303){\rule{0.400pt}{0.723pt}}
\multiput(300.17,303.00)(1.000,1.500){2}{\rule{0.400pt}{0.361pt}}
\put(300.0,303.0){\usebox{\plotpoint}}
\put(302.0,305.0){\usebox{\plotpoint}}
\put(303,304.67){\rule{0.241pt}{0.400pt}}
\multiput(303.00,304.17)(0.500,1.000){2}{\rule{0.120pt}{0.400pt}}
\put(302.0,305.0){\usebox{\plotpoint}}
\put(304,306.67){\rule{0.241pt}{0.400pt}}
\multiput(304.00,306.17)(0.500,1.000){2}{\rule{0.120pt}{0.400pt}}
\put(304.0,306.0){\usebox{\plotpoint}}
\put(306,307.67){\rule{0.241pt}{0.400pt}}
\multiput(306.00,307.17)(0.500,1.000){2}{\rule{0.120pt}{0.400pt}}
\put(307,308.67){\rule{0.241pt}{0.400pt}}
\multiput(307.00,308.17)(0.500,1.000){2}{\rule{0.120pt}{0.400pt}}
\put(305.0,308.0){\usebox{\plotpoint}}
\put(308.0,310.0){\usebox{\plotpoint}}
\put(309,310.67){\rule{0.241pt}{0.400pt}}
\multiput(309.00,310.17)(0.500,1.000){2}{\rule{0.120pt}{0.400pt}}
\put(308.0,311.0){\usebox{\plotpoint}}
\put(309.67,313){\rule{0.400pt}{0.482pt}}
\multiput(309.17,313.00)(1.000,1.000){2}{\rule{0.400pt}{0.241pt}}
\put(310.0,312.0){\usebox{\plotpoint}}
\put(311.0,315.0){\usebox{\plotpoint}}
\put(311.67,314){\rule{0.400pt}{0.482pt}}
\multiput(311.17,314.00)(1.000,1.000){2}{\rule{0.400pt}{0.241pt}}
\put(313,315.67){\rule{0.241pt}{0.400pt}}
\multiput(313.00,315.17)(0.500,1.000){2}{\rule{0.120pt}{0.400pt}}
\put(312.0,314.0){\usebox{\plotpoint}}
\put(314,317.67){\rule{0.241pt}{0.400pt}}
\multiput(314.00,317.17)(0.500,1.000){2}{\rule{0.120pt}{0.400pt}}
\put(315,318.67){\rule{0.241pt}{0.400pt}}
\multiput(315.00,318.17)(0.500,1.000){2}{\rule{0.120pt}{0.400pt}}
\put(314.0,317.0){\usebox{\plotpoint}}
\put(316.0,320.0){\usebox{\plotpoint}}
\put(316.67,321){\rule{0.400pt}{0.482pt}}
\multiput(316.17,321.00)(1.000,1.000){2}{\rule{0.400pt}{0.241pt}}
\put(316.0,321.0){\usebox{\plotpoint}}
\put(318,323){\usebox{\plotpoint}}
\put(317.67,323){\rule{0.400pt}{0.723pt}}
\multiput(317.17,323.00)(1.000,1.500){2}{\rule{0.400pt}{0.361pt}}
\put(318.67,324){\rule{0.400pt}{0.482pt}}
\multiput(318.17,325.00)(1.000,-1.000){2}{\rule{0.400pt}{0.241pt}}
\put(320,325.67){\rule{0.241pt}{0.400pt}}
\multiput(320.00,326.17)(0.500,-1.000){2}{\rule{0.120pt}{0.400pt}}
\put(321,325.67){\rule{0.241pt}{0.400pt}}
\multiput(321.00,325.17)(0.500,1.000){2}{\rule{0.120pt}{0.400pt}}
\put(320.0,324.0){\rule[-0.200pt]{0.400pt}{0.723pt}}
\put(322,328.67){\rule{0.241pt}{0.400pt}}
\multiput(322.00,328.17)(0.500,1.000){2}{\rule{0.120pt}{0.400pt}}
\put(323,329.67){\rule{0.241pt}{0.400pt}}
\multiput(323.00,329.17)(0.500,1.000){2}{\rule{0.120pt}{0.400pt}}
\put(322.0,327.0){\rule[-0.200pt]{0.400pt}{0.482pt}}
\put(324,331.67){\rule{0.241pt}{0.400pt}}
\multiput(324.00,331.17)(0.500,1.000){2}{\rule{0.120pt}{0.400pt}}
\put(325,332.67){\rule{0.241pt}{0.400pt}}
\multiput(325.00,332.17)(0.500,1.000){2}{\rule{0.120pt}{0.400pt}}
\put(324.0,331.0){\usebox{\plotpoint}}
\put(326.0,334.0){\rule[-0.200pt]{0.400pt}{0.482pt}}
\put(327,335.67){\rule{0.241pt}{0.400pt}}
\multiput(327.00,335.17)(0.500,1.000){2}{\rule{0.120pt}{0.400pt}}
\put(326.0,336.0){\usebox{\plotpoint}}
\put(327.67,338){\rule{0.400pt}{0.723pt}}
\multiput(327.17,338.00)(1.000,1.500){2}{\rule{0.400pt}{0.361pt}}
\put(329,339.67){\rule{0.241pt}{0.400pt}}
\multiput(329.00,340.17)(0.500,-1.000){2}{\rule{0.120pt}{0.400pt}}
\put(328.0,337.0){\usebox{\plotpoint}}
\put(329.67,341){\rule{0.400pt}{0.482pt}}
\multiput(329.17,341.00)(1.000,1.000){2}{\rule{0.400pt}{0.241pt}}
\put(330.67,343){\rule{0.400pt}{0.723pt}}
\multiput(330.17,343.00)(1.000,1.500){2}{\rule{0.400pt}{0.361pt}}
\put(330.0,340.0){\usebox{\plotpoint}}
\put(332,344.67){\rule{0.241pt}{0.400pt}}
\multiput(332.00,344.17)(0.500,1.000){2}{\rule{0.120pt}{0.400pt}}
\put(332.67,346){\rule{0.400pt}{0.482pt}}
\multiput(332.17,346.00)(1.000,1.000){2}{\rule{0.400pt}{0.241pt}}
\put(332.0,345.0){\usebox{\plotpoint}}
\put(334,348.67){\rule{0.241pt}{0.400pt}}
\multiput(334.00,348.17)(0.500,1.000){2}{\rule{0.120pt}{0.400pt}}
\put(335,349.67){\rule{0.241pt}{0.400pt}}
\multiput(335.00,349.17)(0.500,1.000){2}{\rule{0.120pt}{0.400pt}}
\put(334.0,348.0){\usebox{\plotpoint}}
\put(336,351){\usebox{\plotpoint}}
\put(335.67,351){\rule{0.400pt}{0.482pt}}
\multiput(335.17,351.00)(1.000,1.000){2}{\rule{0.400pt}{0.241pt}}
\put(336.67,353){\rule{0.400pt}{0.482pt}}
\multiput(336.17,353.00)(1.000,1.000){2}{\rule{0.400pt}{0.241pt}}
\put(338,356.67){\rule{0.241pt}{0.400pt}}
\multiput(338.00,356.17)(0.500,1.000){2}{\rule{0.120pt}{0.400pt}}
\put(339,357.67){\rule{0.241pt}{0.400pt}}
\multiput(339.00,357.17)(0.500,1.000){2}{\rule{0.120pt}{0.400pt}}
\put(338.0,355.0){\rule[-0.200pt]{0.400pt}{0.482pt}}
\put(340.0,359.0){\usebox{\plotpoint}}
\put(340.67,360){\rule{0.400pt}{0.482pt}}
\multiput(340.17,360.00)(1.000,1.000){2}{\rule{0.400pt}{0.241pt}}
\put(340.0,360.0){\usebox{\plotpoint}}
\put(341.67,363){\rule{0.400pt}{0.723pt}}
\multiput(341.17,363.00)(1.000,1.500){2}{\rule{0.400pt}{0.361pt}}
\put(342.0,362.0){\usebox{\plotpoint}}
\put(343.0,366.0){\usebox{\plotpoint}}
\put(343.67,368){\rule{0.400pt}{0.723pt}}
\multiput(343.17,368.00)(1.000,1.500){2}{\rule{0.400pt}{0.361pt}}
\put(345,370.67){\rule{0.241pt}{0.400pt}}
\multiput(345.00,370.17)(0.500,1.000){2}{\rule{0.120pt}{0.400pt}}
\put(344.0,366.0){\rule[-0.200pt]{0.400pt}{0.482pt}}
\put(346,372){\usebox{\plotpoint}}
\put(345.67,372){\rule{0.400pt}{0.482pt}}
\multiput(345.17,372.00)(1.000,1.000){2}{\rule{0.400pt}{0.241pt}}
\put(347,373.67){\rule{0.241pt}{0.400pt}}
\multiput(347.00,373.17)(0.500,1.000){2}{\rule{0.120pt}{0.400pt}}
\put(348,375){\usebox{\plotpoint}}
\put(347.67,375){\rule{0.400pt}{0.482pt}}
\multiput(347.17,375.00)(1.000,1.000){2}{\rule{0.400pt}{0.241pt}}
\put(348.67,377){\rule{0.400pt}{0.723pt}}
\multiput(348.17,377.00)(1.000,1.500){2}{\rule{0.400pt}{0.361pt}}
\put(349.67,379){\rule{0.400pt}{0.723pt}}
\multiput(349.17,379.00)(1.000,1.500){2}{\rule{0.400pt}{0.361pt}}
\put(350.67,382){\rule{0.400pt}{0.723pt}}
\multiput(350.17,382.00)(1.000,1.500){2}{\rule{0.400pt}{0.361pt}}
\put(350.0,379.0){\usebox{\plotpoint}}
\put(351.67,383){\rule{0.400pt}{0.723pt}}
\multiput(351.17,383.00)(1.000,1.500){2}{\rule{0.400pt}{0.361pt}}
\put(352.67,386){\rule{0.400pt}{0.482pt}}
\multiput(352.17,386.00)(1.000,1.000){2}{\rule{0.400pt}{0.241pt}}
\put(352.0,383.0){\rule[-0.200pt]{0.400pt}{0.482pt}}
\put(353.67,389){\rule{0.400pt}{0.482pt}}
\multiput(353.17,389.00)(1.000,1.000){2}{\rule{0.400pt}{0.241pt}}
\put(354.0,388.0){\usebox{\plotpoint}}
\put(355.0,391.0){\usebox{\plotpoint}}
\put(355.67,394){\rule{0.400pt}{0.482pt}}
\multiput(355.17,394.00)(1.000,1.000){2}{\rule{0.400pt}{0.241pt}}
\put(356.67,394){\rule{0.400pt}{0.482pt}}
\multiput(356.17,395.00)(1.000,-1.000){2}{\rule{0.400pt}{0.241pt}}
\put(356.0,391.0){\rule[-0.200pt]{0.400pt}{0.723pt}}
\put(358,397.67){\rule{0.241pt}{0.400pt}}
\multiput(358.00,397.17)(0.500,1.000){2}{\rule{0.120pt}{0.400pt}}
\put(358.0,394.0){\rule[-0.200pt]{0.400pt}{0.964pt}}
\put(358.67,400){\rule{0.400pt}{0.723pt}}
\multiput(358.17,400.00)(1.000,1.500){2}{\rule{0.400pt}{0.361pt}}
\put(359.0,399.0){\usebox{\plotpoint}}
\put(360.0,403.0){\usebox{\plotpoint}}
\put(360.67,404){\rule{0.400pt}{0.964pt}}
\multiput(360.17,404.00)(1.000,2.000){2}{\rule{0.400pt}{0.482pt}}
\put(362,407.67){\rule{0.241pt}{0.400pt}}
\multiput(362.00,407.17)(0.500,1.000){2}{\rule{0.120pt}{0.400pt}}
\put(361.0,403.0){\usebox{\plotpoint}}
\put(363.0,409.0){\usebox{\plotpoint}}
\put(363.67,410){\rule{0.400pt}{0.964pt}}
\multiput(363.17,410.00)(1.000,2.000){2}{\rule{0.400pt}{0.482pt}}
\put(363.0,410.0){\usebox{\plotpoint}}
\put(365,414.67){\rule{0.241pt}{0.400pt}}
\multiput(365.00,414.17)(0.500,1.000){2}{\rule{0.120pt}{0.400pt}}
\put(365.0,414.0){\usebox{\plotpoint}}
\put(366.0,416.0){\usebox{\plotpoint}}
\put(366.67,418){\rule{0.400pt}{0.723pt}}
\multiput(366.17,418.00)(1.000,1.500){2}{\rule{0.400pt}{0.361pt}}
\put(368,420.67){\rule{0.241pt}{0.400pt}}
\multiput(368.00,420.17)(0.500,1.000){2}{\rule{0.120pt}{0.400pt}}
\put(367.0,416.0){\rule[-0.200pt]{0.400pt}{0.482pt}}
\put(369.0,422.0){\rule[-0.200pt]{0.400pt}{0.482pt}}
\put(369.67,424){\rule{0.400pt}{0.482pt}}
\multiput(369.17,424.00)(1.000,1.000){2}{\rule{0.400pt}{0.241pt}}
\put(369.0,424.0){\usebox{\plotpoint}}
\put(370.67,427){\rule{0.400pt}{0.482pt}}
\multiput(370.17,427.00)(1.000,1.000){2}{\rule{0.400pt}{0.241pt}}
\put(371.67,429){\rule{0.400pt}{0.482pt}}
\multiput(371.17,429.00)(1.000,1.000){2}{\rule{0.400pt}{0.241pt}}
\put(371.0,426.0){\usebox{\plotpoint}}
\put(372.67,432){\rule{0.400pt}{0.723pt}}
\multiput(372.17,432.00)(1.000,1.500){2}{\rule{0.400pt}{0.361pt}}
\put(373.0,431.0){\usebox{\plotpoint}}
\put(374.0,435.0){\usebox{\plotpoint}}
\put(374.67,436){\rule{0.400pt}{1.204pt}}
\multiput(374.17,436.00)(1.000,2.500){2}{\rule{0.400pt}{0.602pt}}
\put(376,439.67){\rule{0.241pt}{0.400pt}}
\multiput(376.00,440.17)(0.500,-1.000){2}{\rule{0.120pt}{0.400pt}}
\put(375.0,435.0){\usebox{\plotpoint}}
\put(377,440){\usebox{\plotpoint}}
\put(376.67,440){\rule{0.400pt}{1.204pt}}
\multiput(376.17,440.00)(1.000,2.500){2}{\rule{0.400pt}{0.602pt}}
\put(377.67,443){\rule{0.400pt}{0.482pt}}
\multiput(377.17,444.00)(1.000,-1.000){2}{\rule{0.400pt}{0.241pt}}
\put(378.67,445){\rule{0.400pt}{0.482pt}}
\multiput(378.17,446.00)(1.000,-1.000){2}{\rule{0.400pt}{0.241pt}}
\put(379.67,445){\rule{0.400pt}{1.204pt}}
\multiput(379.17,445.00)(1.000,2.500){2}{\rule{0.400pt}{0.602pt}}
\put(379.0,443.0){\rule[-0.200pt]{0.400pt}{0.964pt}}
\put(381,450){\usebox{\plotpoint}}
\put(380.67,450){\rule{0.400pt}{0.482pt}}
\multiput(380.17,450.00)(1.000,1.000){2}{\rule{0.400pt}{0.241pt}}
\put(381.67,452){\rule{0.400pt}{0.482pt}}
\multiput(381.17,452.00)(1.000,1.000){2}{\rule{0.400pt}{0.241pt}}
\put(383.0,454.0){\rule[-0.200pt]{0.400pt}{0.482pt}}
\put(383.67,456){\rule{0.400pt}{0.964pt}}
\multiput(383.17,456.00)(1.000,2.000){2}{\rule{0.400pt}{0.482pt}}
\put(383.0,456.0){\usebox{\plotpoint}}
\put(385,460.67){\rule{0.241pt}{0.400pt}}
\multiput(385.00,461.17)(0.500,-1.000){2}{\rule{0.120pt}{0.400pt}}
\put(385.67,461){\rule{0.400pt}{0.482pt}}
\multiput(385.17,461.00)(1.000,1.000){2}{\rule{0.400pt}{0.241pt}}
\put(385.0,460.0){\rule[-0.200pt]{0.400pt}{0.482pt}}
\put(386.67,463){\rule{0.400pt}{0.482pt}}
\multiput(386.17,464.00)(1.000,-1.000){2}{\rule{0.400pt}{0.241pt}}
\put(387.67,463){\rule{0.400pt}{0.964pt}}
\multiput(387.17,463.00)(1.000,2.000){2}{\rule{0.400pt}{0.482pt}}
\put(387.0,463.0){\rule[-0.200pt]{0.400pt}{0.482pt}}
\put(389,469.67){\rule{0.241pt}{0.400pt}}
\multiput(389.00,469.17)(0.500,1.000){2}{\rule{0.120pt}{0.400pt}}
\put(389.67,471){\rule{0.400pt}{0.482pt}}
\multiput(389.17,471.00)(1.000,1.000){2}{\rule{0.400pt}{0.241pt}}
\put(389.0,467.0){\rule[-0.200pt]{0.400pt}{0.723pt}}
\put(391,474.67){\rule{0.241pt}{0.400pt}}
\multiput(391.00,474.17)(0.500,1.000){2}{\rule{0.120pt}{0.400pt}}
\put(392,475.67){\rule{0.241pt}{0.400pt}}
\multiput(392.00,475.17)(0.500,1.000){2}{\rule{0.120pt}{0.400pt}}
\put(391.0,473.0){\rule[-0.200pt]{0.400pt}{0.482pt}}
\put(393,481.67){\rule{0.241pt}{0.400pt}}
\multiput(393.00,481.17)(0.500,1.000){2}{\rule{0.120pt}{0.400pt}}
\put(394,481.67){\rule{0.241pt}{0.400pt}}
\multiput(394.00,482.17)(0.500,-1.000){2}{\rule{0.120pt}{0.400pt}}
\put(393.0,477.0){\rule[-0.200pt]{0.400pt}{1.204pt}}
\put(395,483.67){\rule{0.241pt}{0.400pt}}
\multiput(395.00,484.17)(0.500,-1.000){2}{\rule{0.120pt}{0.400pt}}
\put(395.67,484){\rule{0.400pt}{0.723pt}}
\multiput(395.17,484.00)(1.000,1.500){2}{\rule{0.400pt}{0.361pt}}
\put(395.0,482.0){\rule[-0.200pt]{0.400pt}{0.723pt}}
\put(397,489.67){\rule{0.241pt}{0.400pt}}
\multiput(397.00,489.17)(0.500,1.000){2}{\rule{0.120pt}{0.400pt}}
\put(397.0,487.0){\rule[-0.200pt]{0.400pt}{0.723pt}}
\put(398.0,491.0){\usebox{\plotpoint}}
\put(399,494.67){\rule{0.241pt}{0.400pt}}
\multiput(399.00,494.17)(0.500,1.000){2}{\rule{0.120pt}{0.400pt}}
\put(400,495.67){\rule{0.241pt}{0.400pt}}
\multiput(400.00,495.17)(0.500,1.000){2}{\rule{0.120pt}{0.400pt}}
\put(399.0,491.0){\rule[-0.200pt]{0.400pt}{0.964pt}}
\put(400.67,498){\rule{0.400pt}{0.482pt}}
\multiput(400.17,498.00)(1.000,1.000){2}{\rule{0.400pt}{0.241pt}}
\put(401.67,500){\rule{0.400pt}{0.723pt}}
\multiput(401.17,500.00)(1.000,1.500){2}{\rule{0.400pt}{0.361pt}}
\put(401.0,497.0){\usebox{\plotpoint}}
\put(402.67,502){\rule{0.400pt}{0.723pt}}
\multiput(402.17,502.00)(1.000,1.500){2}{\rule{0.400pt}{0.361pt}}
\put(403.0,502.0){\usebox{\plotpoint}}
\put(403.67,503){\rule{0.400pt}{0.723pt}}
\multiput(403.17,504.50)(1.000,-1.500){2}{\rule{0.400pt}{0.361pt}}
\put(404.67,503){\rule{0.400pt}{2.409pt}}
\multiput(404.17,503.00)(1.000,5.000){2}{\rule{0.400pt}{1.204pt}}
\put(404.0,505.0){\usebox{\plotpoint}}
\put(406,513){\usebox{\plotpoint}}
\put(406,511.67){\rule{0.241pt}{0.400pt}}
\multiput(406.00,512.17)(0.500,-1.000){2}{\rule{0.120pt}{0.400pt}}
\put(407,511.67){\rule{0.241pt}{0.400pt}}
\multiput(407.00,511.17)(0.500,1.000){2}{\rule{0.120pt}{0.400pt}}
\put(407.67,515){\rule{0.400pt}{0.723pt}}
\multiput(407.17,515.00)(1.000,1.500){2}{\rule{0.400pt}{0.361pt}}
\put(408.67,518){\rule{0.400pt}{0.964pt}}
\multiput(408.17,518.00)(1.000,2.000){2}{\rule{0.400pt}{0.482pt}}
\put(408.0,513.0){\rule[-0.200pt]{0.400pt}{0.482pt}}
\put(410,522.67){\rule{0.241pt}{0.400pt}}
\multiput(410.00,522.17)(0.500,1.000){2}{\rule{0.120pt}{0.400pt}}
\put(410.0,522.0){\usebox{\plotpoint}}
\put(411.67,524){\rule{0.400pt}{1.204pt}}
\multiput(411.17,524.00)(1.000,2.500){2}{\rule{0.400pt}{0.602pt}}
\put(413,528.67){\rule{0.241pt}{0.400pt}}
\multiput(413.00,528.17)(0.500,1.000){2}{\rule{0.120pt}{0.400pt}}
\put(411.0,524.0){\usebox{\plotpoint}}
\put(413.67,531){\rule{0.400pt}{0.482pt}}
\multiput(413.17,531.00)(1.000,1.000){2}{\rule{0.400pt}{0.241pt}}
\put(414.67,533){\rule{0.400pt}{0.723pt}}
\multiput(414.17,533.00)(1.000,1.500){2}{\rule{0.400pt}{0.361pt}}
\put(414.0,530.0){\usebox{\plotpoint}}
\put(415.67,535){\rule{0.400pt}{0.482pt}}
\multiput(415.17,535.00)(1.000,1.000){2}{\rule{0.400pt}{0.241pt}}
\put(416.67,535){\rule{0.400pt}{0.482pt}}
\multiput(416.17,536.00)(1.000,-1.000){2}{\rule{0.400pt}{0.241pt}}
\put(416.0,535.0){\usebox{\plotpoint}}
\put(417.67,539){\rule{0.400pt}{0.723pt}}
\multiput(417.17,539.00)(1.000,1.500){2}{\rule{0.400pt}{0.361pt}}
\put(418.67,542){\rule{0.400pt}{0.723pt}}
\multiput(418.17,542.00)(1.000,1.500){2}{\rule{0.400pt}{0.361pt}}
\put(418.0,535.0){\rule[-0.200pt]{0.400pt}{0.964pt}}
\put(419.67,544){\rule{0.400pt}{0.723pt}}
\multiput(419.17,545.50)(1.000,-1.500){2}{\rule{0.400pt}{0.361pt}}
\put(420.67,544){\rule{0.400pt}{1.445pt}}
\multiput(420.17,544.00)(1.000,3.000){2}{\rule{0.400pt}{0.723pt}}
\put(420.0,545.0){\rule[-0.200pt]{0.400pt}{0.482pt}}
\put(422,550.67){\rule{0.241pt}{0.400pt}}
\multiput(422.00,550.17)(0.500,1.000){2}{\rule{0.120pt}{0.400pt}}
\put(422.67,552){\rule{0.400pt}{0.482pt}}
\multiput(422.17,552.00)(1.000,1.000){2}{\rule{0.400pt}{0.241pt}}
\put(422.0,550.0){\usebox{\plotpoint}}
\put(423.67,553){\rule{0.400pt}{0.723pt}}
\multiput(423.17,554.50)(1.000,-1.500){2}{\rule{0.400pt}{0.361pt}}
\put(424.67,553){\rule{0.400pt}{1.686pt}}
\multiput(424.17,553.00)(1.000,3.500){2}{\rule{0.400pt}{0.843pt}}
\put(424.0,554.0){\rule[-0.200pt]{0.400pt}{0.482pt}}
\put(425.67,557){\rule{0.400pt}{0.723pt}}
\multiput(425.17,557.00)(1.000,1.500){2}{\rule{0.400pt}{0.361pt}}
\put(426.67,560){\rule{0.400pt}{0.482pt}}
\multiput(426.17,560.00)(1.000,1.000){2}{\rule{0.400pt}{0.241pt}}
\put(426.0,557.0){\rule[-0.200pt]{0.400pt}{0.723pt}}
\put(428,564.67){\rule{0.241pt}{0.400pt}}
\multiput(428.00,564.17)(0.500,1.000){2}{\rule{0.120pt}{0.400pt}}
\put(428.67,566){\rule{0.400pt}{0.964pt}}
\multiput(428.17,566.00)(1.000,2.000){2}{\rule{0.400pt}{0.482pt}}
\put(428.0,562.0){\rule[-0.200pt]{0.400pt}{0.723pt}}
\put(430,570.67){\rule{0.241pt}{0.400pt}}
\multiput(430.00,570.17)(0.500,1.000){2}{\rule{0.120pt}{0.400pt}}
\put(430.67,568){\rule{0.400pt}{0.964pt}}
\multiput(430.17,570.00)(1.000,-2.000){2}{\rule{0.400pt}{0.482pt}}
\put(430.0,570.0){\usebox{\plotpoint}}
\put(431.67,574){\rule{0.400pt}{0.482pt}}
\multiput(431.17,574.00)(1.000,1.000){2}{\rule{0.400pt}{0.241pt}}
\put(432.0,568.0){\rule[-0.200pt]{0.400pt}{1.445pt}}
\put(432.67,578){\rule{0.400pt}{0.723pt}}
\multiput(432.17,578.00)(1.000,1.500){2}{\rule{0.400pt}{0.361pt}}
\put(433.0,576.0){\rule[-0.200pt]{0.400pt}{0.482pt}}
\put(434.0,581.0){\usebox{\plotpoint}}
\put(435.0,581.0){\rule[-0.200pt]{0.400pt}{0.964pt}}
\put(435.67,585){\rule{0.400pt}{0.964pt}}
\multiput(435.17,585.00)(1.000,2.000){2}{\rule{0.400pt}{0.482pt}}
\put(435.0,585.0){\usebox{\plotpoint}}
\put(436.67,586){\rule{0.400pt}{0.723pt}}
\multiput(436.17,586.00)(1.000,1.500){2}{\rule{0.400pt}{0.361pt}}
\put(437.67,589){\rule{0.400pt}{0.482pt}}
\multiput(437.17,589.00)(1.000,1.000){2}{\rule{0.400pt}{0.241pt}}
\put(437.0,586.0){\rule[-0.200pt]{0.400pt}{0.723pt}}
\put(438.67,589){\rule{0.400pt}{0.964pt}}
\multiput(438.17,589.00)(1.000,2.000){2}{\rule{0.400pt}{0.482pt}}
\put(439.0,589.0){\rule[-0.200pt]{0.400pt}{0.482pt}}
\put(440.0,593.0){\usebox{\plotpoint}}
\put(440.67,595){\rule{0.400pt}{1.927pt}}
\multiput(440.17,595.00)(1.000,4.000){2}{\rule{0.400pt}{0.964pt}}
\put(441.67,600){\rule{0.400pt}{0.723pt}}
\multiput(441.17,601.50)(1.000,-1.500){2}{\rule{0.400pt}{0.361pt}}
\put(441.0,593.0){\rule[-0.200pt]{0.400pt}{0.482pt}}
\put(442.67,599){\rule{0.400pt}{0.482pt}}
\multiput(442.17,599.00)(1.000,1.000){2}{\rule{0.400pt}{0.241pt}}
\put(443.67,598){\rule{0.400pt}{0.723pt}}
\multiput(443.17,599.50)(1.000,-1.500){2}{\rule{0.400pt}{0.361pt}}
\put(443.0,599.0){\usebox{\plotpoint}}
\put(445.0,598.0){\rule[-0.200pt]{0.400pt}{1.927pt}}
\put(445.67,606){\rule{0.400pt}{0.723pt}}
\multiput(445.17,606.00)(1.000,1.500){2}{\rule{0.400pt}{0.361pt}}
\put(445.0,606.0){\usebox{\plotpoint}}
\put(447.0,607.0){\rule[-0.200pt]{0.400pt}{0.482pt}}
\put(447.67,607){\rule{0.400pt}{0.964pt}}
\multiput(447.17,607.00)(1.000,2.000){2}{\rule{0.400pt}{0.482pt}}
\put(447.0,607.0){\usebox{\plotpoint}}
\put(448.67,612){\rule{0.400pt}{0.723pt}}
\multiput(448.17,613.50)(1.000,-1.500){2}{\rule{0.400pt}{0.361pt}}
\put(449.67,612){\rule{0.400pt}{1.686pt}}
\multiput(449.17,612.00)(1.000,3.500){2}{\rule{0.400pt}{0.843pt}}
\put(449.0,611.0){\rule[-0.200pt]{0.400pt}{0.964pt}}
\put(451,619){\usebox{\plotpoint}}
\put(450.67,615){\rule{0.400pt}{0.964pt}}
\multiput(450.17,617.00)(1.000,-2.000){2}{\rule{0.400pt}{0.482pt}}
\put(451.67,615){\rule{0.400pt}{1.204pt}}
\multiput(451.17,615.00)(1.000,2.500){2}{\rule{0.400pt}{0.602pt}}
\put(452.67,619){\rule{0.400pt}{0.482pt}}
\multiput(452.17,620.00)(1.000,-1.000){2}{\rule{0.400pt}{0.241pt}}
\put(453.67,619){\rule{0.400pt}{0.723pt}}
\multiput(453.17,619.00)(1.000,1.500){2}{\rule{0.400pt}{0.361pt}}
\put(453.0,620.0){\usebox{\plotpoint}}
\put(454.67,623){\rule{0.400pt}{0.964pt}}
\multiput(454.17,623.00)(1.000,2.000){2}{\rule{0.400pt}{0.482pt}}
\put(455.67,625){\rule{0.400pt}{0.482pt}}
\multiput(455.17,626.00)(1.000,-1.000){2}{\rule{0.400pt}{0.241pt}}
\put(455.0,622.0){\usebox{\plotpoint}}
\put(457,627.67){\rule{0.241pt}{0.400pt}}
\multiput(457.00,627.17)(0.500,1.000){2}{\rule{0.120pt}{0.400pt}}
\put(457.0,625.0){\rule[-0.200pt]{0.400pt}{0.723pt}}
\put(458,629){\usebox{\plotpoint}}
\put(457.67,626){\rule{0.400pt}{0.723pt}}
\multiput(457.17,627.50)(1.000,-1.500){2}{\rule{0.400pt}{0.361pt}}
\put(458.67,626){\rule{0.400pt}{1.204pt}}
\multiput(458.17,626.00)(1.000,2.500){2}{\rule{0.400pt}{0.602pt}}
\put(459.67,632){\rule{0.400pt}{0.482pt}}
\multiput(459.17,632.00)(1.000,1.000){2}{\rule{0.400pt}{0.241pt}}
\put(460.67,634){\rule{0.400pt}{1.445pt}}
\multiput(460.17,634.00)(1.000,3.000){2}{\rule{0.400pt}{0.723pt}}
\put(460.0,631.0){\usebox{\plotpoint}}
\put(461.67,636){\rule{0.400pt}{0.723pt}}
\multiput(461.17,637.50)(1.000,-1.500){2}{\rule{0.400pt}{0.361pt}}
\put(462.67,636){\rule{0.400pt}{1.686pt}}
\multiput(462.17,636.00)(1.000,3.500){2}{\rule{0.400pt}{0.843pt}}
\put(462.0,639.0){\usebox{\plotpoint}}
\put(463.67,638){\rule{0.400pt}{0.482pt}}
\multiput(463.17,639.00)(1.000,-1.000){2}{\rule{0.400pt}{0.241pt}}
\put(464.67,638){\rule{0.400pt}{1.204pt}}
\multiput(464.17,638.00)(1.000,2.500){2}{\rule{0.400pt}{0.602pt}}
\put(464.0,640.0){\rule[-0.200pt]{0.400pt}{0.723pt}}
\put(466,641.67){\rule{0.241pt}{0.400pt}}
\multiput(466.00,641.17)(0.500,1.000){2}{\rule{0.120pt}{0.400pt}}
\put(466.67,643){\rule{0.400pt}{0.723pt}}
\multiput(466.17,643.00)(1.000,1.500){2}{\rule{0.400pt}{0.361pt}}
\put(466.0,642.0){\usebox{\plotpoint}}
\put(468,646){\usebox{\plotpoint}}
\put(467.67,641){\rule{0.400pt}{1.204pt}}
\multiput(467.17,643.50)(1.000,-2.500){2}{\rule{0.400pt}{0.602pt}}
\put(468.67,641){\rule{0.400pt}{1.445pt}}
\multiput(468.17,641.00)(1.000,3.000){2}{\rule{0.400pt}{0.723pt}}
\put(469.67,645){\rule{0.400pt}{0.723pt}}
\multiput(469.17,646.50)(1.000,-1.500){2}{\rule{0.400pt}{0.361pt}}
\put(470.67,645){\rule{0.400pt}{0.482pt}}
\multiput(470.17,645.00)(1.000,1.000){2}{\rule{0.400pt}{0.241pt}}
\put(470.0,647.0){\usebox{\plotpoint}}
\put(472,650.67){\rule{0.241pt}{0.400pt}}
\multiput(472.00,650.17)(0.500,1.000){2}{\rule{0.120pt}{0.400pt}}
\put(473,650.67){\rule{0.241pt}{0.400pt}}
\multiput(473.00,651.17)(0.500,-1.000){2}{\rule{0.120pt}{0.400pt}}
\put(472.0,647.0){\rule[-0.200pt]{0.400pt}{0.964pt}}
\put(473.67,649){\rule{0.400pt}{1.445pt}}
\multiput(473.17,649.00)(1.000,3.000){2}{\rule{0.400pt}{0.723pt}}
\put(474.67,651){\rule{0.400pt}{0.964pt}}
\multiput(474.17,653.00)(1.000,-2.000){2}{\rule{0.400pt}{0.482pt}}
\put(474.0,649.0){\rule[-0.200pt]{0.400pt}{0.482pt}}
\put(476,651.67){\rule{0.241pt}{0.400pt}}
\multiput(476.00,651.17)(0.500,1.000){2}{\rule{0.120pt}{0.400pt}}
\put(476.0,651.0){\usebox{\plotpoint}}
\put(477.67,653){\rule{0.400pt}{0.723pt}}
\multiput(477.17,653.00)(1.000,1.500){2}{\rule{0.400pt}{0.361pt}}
\put(477.0,653.0){\usebox{\plotpoint}}
\put(478.67,657){\rule{0.400pt}{0.482pt}}
\multiput(478.17,658.00)(1.000,-1.000){2}{\rule{0.400pt}{0.241pt}}
\put(479.67,653){\rule{0.400pt}{0.964pt}}
\multiput(479.17,655.00)(1.000,-2.000){2}{\rule{0.400pt}{0.482pt}}
\put(479.0,656.0){\rule[-0.200pt]{0.400pt}{0.723pt}}
\put(480.67,657){\rule{0.400pt}{0.482pt}}
\multiput(480.17,658.00)(1.000,-1.000){2}{\rule{0.400pt}{0.241pt}}
\put(482,655.67){\rule{0.241pt}{0.400pt}}
\multiput(482.00,656.17)(0.500,-1.000){2}{\rule{0.120pt}{0.400pt}}
\put(481.0,653.0){\rule[-0.200pt]{0.400pt}{1.445pt}}
\put(482.67,656){\rule{0.400pt}{0.723pt}}
\multiput(482.17,657.50)(1.000,-1.500){2}{\rule{0.400pt}{0.361pt}}
\put(483.67,656){\rule{0.400pt}{0.723pt}}
\multiput(483.17,656.00)(1.000,1.500){2}{\rule{0.400pt}{0.361pt}}
\put(483.0,656.0){\rule[-0.200pt]{0.400pt}{0.723pt}}
\put(485.0,658.0){\usebox{\plotpoint}}
\put(485.0,658.0){\rule[-0.200pt]{0.482pt}{0.400pt}}
\put(486.67,655){\rule{0.400pt}{1.445pt}}
\multiput(486.17,658.00)(1.000,-3.000){2}{\rule{0.400pt}{0.723pt}}
\put(487.67,647){\rule{0.400pt}{1.927pt}}
\multiput(487.17,651.00)(1.000,-4.000){2}{\rule{0.400pt}{0.964pt}}
\put(487.0,658.0){\rule[-0.200pt]{0.400pt}{0.723pt}}
\put(488.67,660){\rule{0.400pt}{0.964pt}}
\multiput(488.17,662.00)(1.000,-2.000){2}{\rule{0.400pt}{0.482pt}}
\put(489.67,660){\rule{0.400pt}{0.723pt}}
\multiput(489.17,660.00)(1.000,1.500){2}{\rule{0.400pt}{0.361pt}}
\put(489.0,647.0){\rule[-0.200pt]{0.400pt}{4.095pt}}
\put(490.67,656){\rule{0.400pt}{1.204pt}}
\multiput(490.17,658.50)(1.000,-2.500){2}{\rule{0.400pt}{0.602pt}}
\put(491.67,652){\rule{0.400pt}{0.964pt}}
\multiput(491.17,654.00)(1.000,-2.000){2}{\rule{0.400pt}{0.482pt}}
\put(491.0,661.0){\rule[-0.200pt]{0.400pt}{0.482pt}}
\put(492.67,653){\rule{0.400pt}{2.650pt}}
\multiput(492.17,653.00)(1.000,5.500){2}{\rule{0.400pt}{1.325pt}}
\put(493.67,661){\rule{0.400pt}{0.723pt}}
\multiput(493.17,662.50)(1.000,-1.500){2}{\rule{0.400pt}{0.361pt}}
\put(493.0,652.0){\usebox{\plotpoint}}
\put(494.67,657){\rule{0.400pt}{0.964pt}}
\multiput(494.17,657.00)(1.000,2.000){2}{\rule{0.400pt}{0.482pt}}
\put(495.67,659){\rule{0.400pt}{0.482pt}}
\multiput(495.17,660.00)(1.000,-1.000){2}{\rule{0.400pt}{0.241pt}}
\put(495.0,657.0){\rule[-0.200pt]{0.400pt}{0.964pt}}
\put(497,659.67){\rule{0.241pt}{0.400pt}}
\multiput(497.00,659.17)(0.500,1.000){2}{\rule{0.120pt}{0.400pt}}
\put(497.0,659.0){\usebox{\plotpoint}}
\put(497.67,654){\rule{0.400pt}{2.409pt}}
\multiput(497.17,659.00)(1.000,-5.000){2}{\rule{0.400pt}{1.204pt}}
\put(498.67,654){\rule{0.400pt}{0.964pt}}
\multiput(498.17,654.00)(1.000,2.000){2}{\rule{0.400pt}{0.482pt}}
\put(498.0,661.0){\rule[-0.200pt]{0.400pt}{0.723pt}}
\put(499.67,657){\rule{0.400pt}{0.964pt}}
\multiput(499.17,657.00)(1.000,2.000){2}{\rule{0.400pt}{0.482pt}}
\put(500.67,658){\rule{0.400pt}{0.723pt}}
\multiput(500.17,659.50)(1.000,-1.500){2}{\rule{0.400pt}{0.361pt}}
\put(500.0,657.0){\usebox{\plotpoint}}
\put(501.67,652){\rule{0.400pt}{0.964pt}}
\multiput(501.17,652.00)(1.000,2.000){2}{\rule{0.400pt}{0.482pt}}
\put(503,655.67){\rule{0.241pt}{0.400pt}}
\multiput(503.00,655.17)(0.500,1.000){2}{\rule{0.120pt}{0.400pt}}
\put(502.0,652.0){\rule[-0.200pt]{0.400pt}{1.445pt}}
\put(504,656.67){\rule{0.241pt}{0.400pt}}
\multiput(504.00,657.17)(0.500,-1.000){2}{\rule{0.120pt}{0.400pt}}
\put(504.67,657){\rule{0.400pt}{0.964pt}}
\multiput(504.17,657.00)(1.000,2.000){2}{\rule{0.400pt}{0.482pt}}
\put(504.0,657.0){\usebox{\plotpoint}}
\put(506,656.67){\rule{0.241pt}{0.400pt}}
\multiput(506.00,657.17)(0.500,-1.000){2}{\rule{0.120pt}{0.400pt}}
\put(507,656.67){\rule{0.241pt}{0.400pt}}
\multiput(507.00,656.17)(0.500,1.000){2}{\rule{0.120pt}{0.400pt}}
\put(506.0,658.0){\rule[-0.200pt]{0.400pt}{0.723pt}}
\put(507.67,656){\rule{0.400pt}{1.204pt}}
\multiput(507.17,658.50)(1.000,-2.500){2}{\rule{0.400pt}{0.602pt}}
\put(508.67,656){\rule{0.400pt}{0.964pt}}
\multiput(508.17,656.00)(1.000,2.000){2}{\rule{0.400pt}{0.482pt}}
\put(508.0,658.0){\rule[-0.200pt]{0.400pt}{0.723pt}}
\put(509.67,654){\rule{0.400pt}{1.445pt}}
\multiput(509.17,654.00)(1.000,3.000){2}{\rule{0.400pt}{0.723pt}}
\put(510.67,654){\rule{0.400pt}{1.445pt}}
\multiput(510.17,657.00)(1.000,-3.000){2}{\rule{0.400pt}{0.723pt}}
\put(510.0,654.0){\rule[-0.200pt]{0.400pt}{1.445pt}}
\put(511.67,655){\rule{0.400pt}{0.482pt}}
\multiput(511.17,655.00)(1.000,1.000){2}{\rule{0.400pt}{0.241pt}}
\put(512.67,657){\rule{0.400pt}{0.964pt}}
\multiput(512.17,657.00)(1.000,2.000){2}{\rule{0.400pt}{0.482pt}}
\put(512.0,654.0){\usebox{\plotpoint}}
\put(514,655.67){\rule{0.241pt}{0.400pt}}
\multiput(514.00,655.17)(0.500,1.000){2}{\rule{0.120pt}{0.400pt}}
\put(514.0,656.0){\rule[-0.200pt]{0.400pt}{1.204pt}}
\put(514.67,652){\rule{0.400pt}{2.168pt}}
\multiput(514.17,656.50)(1.000,-4.500){2}{\rule{0.400pt}{1.084pt}}
\put(515.67,650){\rule{0.400pt}{0.482pt}}
\multiput(515.17,651.00)(1.000,-1.000){2}{\rule{0.400pt}{0.241pt}}
\put(515.0,657.0){\rule[-0.200pt]{0.400pt}{0.964pt}}
\put(517,650){\usebox{\plotpoint}}
\put(516.67,646){\rule{0.400pt}{0.964pt}}
\multiput(516.17,648.00)(1.000,-2.000){2}{\rule{0.400pt}{0.482pt}}
\put(518.0,646.0){\usebox{\plotpoint}}
\put(518.67,649){\rule{0.400pt}{2.168pt}}
\multiput(518.17,653.50)(1.000,-4.500){2}{\rule{0.400pt}{1.084pt}}
\put(519.67,646){\rule{0.400pt}{0.723pt}}
\multiput(519.17,647.50)(1.000,-1.500){2}{\rule{0.400pt}{0.361pt}}
\put(519.0,646.0){\rule[-0.200pt]{0.400pt}{2.891pt}}
\put(521,652.67){\rule{0.241pt}{0.400pt}}
\multiput(521.00,653.17)(0.500,-1.000){2}{\rule{0.120pt}{0.400pt}}
\put(521.67,648){\rule{0.400pt}{1.204pt}}
\multiput(521.17,650.50)(1.000,-2.500){2}{\rule{0.400pt}{0.602pt}}
\put(521.0,646.0){\rule[-0.200pt]{0.400pt}{1.927pt}}
\put(523,644.67){\rule{0.241pt}{0.400pt}}
\multiput(523.00,644.17)(0.500,1.000){2}{\rule{0.120pt}{0.400pt}}
\put(524,645.67){\rule{0.241pt}{0.400pt}}
\multiput(524.00,645.17)(0.500,1.000){2}{\rule{0.120pt}{0.400pt}}
\put(523.0,645.0){\rule[-0.200pt]{0.400pt}{0.723pt}}
\put(524.67,644){\rule{0.400pt}{0.482pt}}
\multiput(524.17,644.00)(1.000,1.000){2}{\rule{0.400pt}{0.241pt}}
\put(526,644.67){\rule{0.241pt}{0.400pt}}
\multiput(526.00,645.17)(0.500,-1.000){2}{\rule{0.120pt}{0.400pt}}
\put(525.0,644.0){\rule[-0.200pt]{0.400pt}{0.723pt}}
\put(526.67,647){\rule{0.400pt}{0.482pt}}
\multiput(526.17,648.00)(1.000,-1.000){2}{\rule{0.400pt}{0.241pt}}
\put(527.67,643){\rule{0.400pt}{0.964pt}}
\multiput(527.17,645.00)(1.000,-2.000){2}{\rule{0.400pt}{0.482pt}}
\put(527.0,645.0){\rule[-0.200pt]{0.400pt}{0.964pt}}
\put(528.67,638){\rule{0.400pt}{0.723pt}}
\multiput(528.17,639.50)(1.000,-1.500){2}{\rule{0.400pt}{0.361pt}}
\put(529.0,641.0){\rule[-0.200pt]{0.400pt}{0.482pt}}
\put(529.67,642){\rule{0.400pt}{1.686pt}}
\multiput(529.17,642.00)(1.000,3.500){2}{\rule{0.400pt}{0.843pt}}
\put(530.67,641){\rule{0.400pt}{1.927pt}}
\multiput(530.17,645.00)(1.000,-4.000){2}{\rule{0.400pt}{0.964pt}}
\put(530.0,638.0){\rule[-0.200pt]{0.400pt}{0.964pt}}
\put(531.67,635){\rule{0.400pt}{2.168pt}}
\multiput(531.17,635.00)(1.000,4.500){2}{\rule{0.400pt}{1.084pt}}
\put(532.67,632){\rule{0.400pt}{2.891pt}}
\multiput(532.17,638.00)(1.000,-6.000){2}{\rule{0.400pt}{1.445pt}}
\put(532.0,635.0){\rule[-0.200pt]{0.400pt}{1.445pt}}
\put(533.67,636){\rule{0.400pt}{0.723pt}}
\multiput(533.17,636.00)(1.000,1.500){2}{\rule{0.400pt}{0.361pt}}
\put(534.0,632.0){\rule[-0.200pt]{0.400pt}{0.964pt}}
\put(535.0,639.0){\usebox{\plotpoint}}
\put(535.67,634){\rule{0.400pt}{2.891pt}}
\multiput(535.17,634.00)(1.000,6.000){2}{\rule{0.400pt}{1.445pt}}
\put(536.67,639){\rule{0.400pt}{1.686pt}}
\multiput(536.17,642.50)(1.000,-3.500){2}{\rule{0.400pt}{0.843pt}}
\put(536.0,634.0){\rule[-0.200pt]{0.400pt}{1.204pt}}
\put(537.67,631){\rule{0.400pt}{1.204pt}}
\multiput(537.17,633.50)(1.000,-2.500){2}{\rule{0.400pt}{0.602pt}}
\put(539,629.67){\rule{0.241pt}{0.400pt}}
\multiput(539.00,630.17)(0.500,-1.000){2}{\rule{0.120pt}{0.400pt}}
\put(538.0,636.0){\rule[-0.200pt]{0.400pt}{0.723pt}}
\put(540,631.67){\rule{0.241pt}{0.400pt}}
\multiput(540.00,631.17)(0.500,1.000){2}{\rule{0.120pt}{0.400pt}}
\put(540.67,630){\rule{0.400pt}{0.723pt}}
\multiput(540.17,631.50)(1.000,-1.500){2}{\rule{0.400pt}{0.361pt}}
\put(540.0,630.0){\rule[-0.200pt]{0.400pt}{0.482pt}}
\put(541.67,630){\rule{0.400pt}{1.686pt}}
\multiput(541.17,633.50)(1.000,-3.500){2}{\rule{0.400pt}{0.843pt}}
\put(542.67,630){\rule{0.400pt}{0.964pt}}
\multiput(542.17,630.00)(1.000,2.000){2}{\rule{0.400pt}{0.482pt}}
\put(542.0,630.0){\rule[-0.200pt]{0.400pt}{1.686pt}}
\put(543.67,627){\rule{0.400pt}{1.204pt}}
\multiput(543.17,627.00)(1.000,2.500){2}{\rule{0.400pt}{0.602pt}}
\put(544.0,627.0){\rule[-0.200pt]{0.400pt}{1.686pt}}
\put(544.67,628){\rule{0.400pt}{0.482pt}}
\multiput(544.17,629.00)(1.000,-1.000){2}{\rule{0.400pt}{0.241pt}}
\put(545.67,628){\rule{0.400pt}{0.723pt}}
\multiput(545.17,628.00)(1.000,1.500){2}{\rule{0.400pt}{0.361pt}}
\put(545.0,630.0){\rule[-0.200pt]{0.400pt}{0.482pt}}
\put(546.67,630){\rule{0.400pt}{0.723pt}}
\multiput(546.17,631.50)(1.000,-1.500){2}{\rule{0.400pt}{0.361pt}}
\put(547.0,631.0){\rule[-0.200pt]{0.400pt}{0.482pt}}
\put(548.0,630.0){\usebox{\plotpoint}}
\put(548.67,625){\rule{0.400pt}{2.409pt}}
\multiput(548.17,630.00)(1.000,-5.000){2}{\rule{0.400pt}{1.204pt}}
\put(549.67,625){\rule{0.400pt}{0.964pt}}
\multiput(549.17,625.00)(1.000,2.000){2}{\rule{0.400pt}{0.482pt}}
\put(549.0,630.0){\rule[-0.200pt]{0.400pt}{1.204pt}}
\put(551,623.67){\rule{0.241pt}{0.400pt}}
\multiput(551.00,624.17)(0.500,-1.000){2}{\rule{0.120pt}{0.400pt}}
\put(552,622.67){\rule{0.241pt}{0.400pt}}
\multiput(552.00,623.17)(0.500,-1.000){2}{\rule{0.120pt}{0.400pt}}
\put(551.0,625.0){\rule[-0.200pt]{0.400pt}{0.964pt}}
\put(552.67,618){\rule{0.400pt}{1.445pt}}
\multiput(552.17,621.00)(1.000,-3.000){2}{\rule{0.400pt}{0.723pt}}
\put(553.67,612){\rule{0.400pt}{1.445pt}}
\multiput(553.17,615.00)(1.000,-3.000){2}{\rule{0.400pt}{0.723pt}}
\put(553.0,623.0){\usebox{\plotpoint}}
\put(554.67,620){\rule{0.400pt}{0.723pt}}
\multiput(554.17,620.00)(1.000,1.500){2}{\rule{0.400pt}{0.361pt}}
\put(555.67,620){\rule{0.400pt}{0.723pt}}
\multiput(555.17,621.50)(1.000,-1.500){2}{\rule{0.400pt}{0.361pt}}
\put(555.0,612.0){\rule[-0.200pt]{0.400pt}{1.927pt}}
\put(557,621.67){\rule{0.241pt}{0.400pt}}
\multiput(557.00,621.17)(0.500,1.000){2}{\rule{0.120pt}{0.400pt}}
\put(557.67,620){\rule{0.400pt}{0.723pt}}
\multiput(557.17,621.50)(1.000,-1.500){2}{\rule{0.400pt}{0.361pt}}
\put(557.0,620.0){\rule[-0.200pt]{0.400pt}{0.482pt}}
\put(559,614.67){\rule{0.241pt}{0.400pt}}
\multiput(559.00,615.17)(0.500,-1.000){2}{\rule{0.120pt}{0.400pt}}
\put(559.0,616.0){\rule[-0.200pt]{0.400pt}{0.964pt}}
\put(559.67,614){\rule{0.400pt}{0.964pt}}
\multiput(559.17,616.00)(1.000,-2.000){2}{\rule{0.400pt}{0.482pt}}
\put(560.0,615.0){\rule[-0.200pt]{0.400pt}{0.723pt}}
\put(561.0,614.0){\usebox{\plotpoint}}
\put(561.67,613){\rule{0.400pt}{1.927pt}}
\multiput(561.17,613.00)(1.000,4.000){2}{\rule{0.400pt}{0.964pt}}
\put(562.67,619){\rule{0.400pt}{0.482pt}}
\multiput(562.17,620.00)(1.000,-1.000){2}{\rule{0.400pt}{0.241pt}}
\put(562.0,613.0){\usebox{\plotpoint}}
\put(563.67,607){\rule{0.400pt}{2.409pt}}
\multiput(563.17,607.00)(1.000,5.000){2}{\rule{0.400pt}{1.204pt}}
\put(564.67,612){\rule{0.400pt}{1.204pt}}
\multiput(564.17,614.50)(1.000,-2.500){2}{\rule{0.400pt}{0.602pt}}
\put(564.0,607.0){\rule[-0.200pt]{0.400pt}{2.891pt}}
\put(566,608.67){\rule{0.241pt}{0.400pt}}
\multiput(566.00,608.17)(0.500,1.000){2}{\rule{0.120pt}{0.400pt}}
\put(566.67,607){\rule{0.400pt}{0.723pt}}
\multiput(566.17,608.50)(1.000,-1.500){2}{\rule{0.400pt}{0.361pt}}
\put(566.0,609.0){\rule[-0.200pt]{0.400pt}{0.723pt}}
\put(568.0,607.0){\rule[-0.200pt]{0.400pt}{0.723pt}}
\put(568.67,607){\rule{0.400pt}{0.723pt}}
\multiput(568.17,608.50)(1.000,-1.500){2}{\rule{0.400pt}{0.361pt}}
\put(568.0,610.0){\usebox{\plotpoint}}
\put(569.67,605){\rule{0.400pt}{0.964pt}}
\multiput(569.17,607.00)(1.000,-2.000){2}{\rule{0.400pt}{0.482pt}}
\put(570.67,599){\rule{0.400pt}{1.445pt}}
\multiput(570.17,602.00)(1.000,-3.000){2}{\rule{0.400pt}{0.723pt}}
\put(570.0,607.0){\rule[-0.200pt]{0.400pt}{0.482pt}}
\put(572,603.67){\rule{0.241pt}{0.400pt}}
\multiput(572.00,604.17)(0.500,-1.000){2}{\rule{0.120pt}{0.400pt}}
\put(572.0,599.0){\rule[-0.200pt]{0.400pt}{1.445pt}}
\put(573,604){\usebox{\plotpoint}}
\put(572.67,604){\rule{0.400pt}{0.723pt}}
\multiput(572.17,604.00)(1.000,1.500){2}{\rule{0.400pt}{0.361pt}}
\put(573.67,596){\rule{0.400pt}{2.650pt}}
\multiput(573.17,601.50)(1.000,-5.500){2}{\rule{0.400pt}{1.325pt}}
\put(575.0,596.0){\rule[-0.200pt]{0.400pt}{1.686pt}}
\put(575.67,603){\rule{0.400pt}{0.723pt}}
\multiput(575.17,603.00)(1.000,1.500){2}{\rule{0.400pt}{0.361pt}}
\put(575.0,603.0){\usebox{\plotpoint}}
\put(577.0,596.0){\rule[-0.200pt]{0.400pt}{2.409pt}}
\put(577.0,596.0){\rule[-0.200pt]{0.482pt}{0.400pt}}
\put(578.67,595){\rule{0.400pt}{1.204pt}}
\multiput(578.17,595.00)(1.000,2.500){2}{\rule{0.400pt}{0.602pt}}
\put(579.67,589){\rule{0.400pt}{2.650pt}}
\multiput(579.17,594.50)(1.000,-5.500){2}{\rule{0.400pt}{1.325pt}}
\put(579.0,595.0){\usebox{\plotpoint}}
\put(580.67,584){\rule{0.400pt}{2.650pt}}
\multiput(580.17,584.00)(1.000,5.500){2}{\rule{0.400pt}{1.325pt}}
\put(581.67,595){\rule{0.400pt}{0.482pt}}
\multiput(581.17,595.00)(1.000,1.000){2}{\rule{0.400pt}{0.241pt}}
\put(581.0,584.0){\rule[-0.200pt]{0.400pt}{1.204pt}}
\put(582.67,589){\rule{0.400pt}{0.964pt}}
\multiput(582.17,589.00)(1.000,2.000){2}{\rule{0.400pt}{0.482pt}}
\put(583.0,589.0){\rule[-0.200pt]{0.400pt}{1.927pt}}
\put(584,590.67){\rule{0.241pt}{0.400pt}}
\multiput(584.00,590.17)(0.500,1.000){2}{\rule{0.120pt}{0.400pt}}
\put(584.67,589){\rule{0.400pt}{0.723pt}}
\multiput(584.17,590.50)(1.000,-1.500){2}{\rule{0.400pt}{0.361pt}}
\put(584.0,591.0){\rule[-0.200pt]{0.400pt}{0.482pt}}
\put(585.67,583){\rule{0.400pt}{0.482pt}}
\multiput(585.17,583.00)(1.000,1.000){2}{\rule{0.400pt}{0.241pt}}
\put(587,584.67){\rule{0.241pt}{0.400pt}}
\multiput(587.00,584.17)(0.500,1.000){2}{\rule{0.120pt}{0.400pt}}
\put(586.0,583.0){\rule[-0.200pt]{0.400pt}{1.445pt}}
\put(587.67,582){\rule{0.400pt}{1.927pt}}
\multiput(587.17,582.00)(1.000,4.000){2}{\rule{0.400pt}{0.964pt}}
\put(588.67,581){\rule{0.400pt}{2.168pt}}
\multiput(588.17,585.50)(1.000,-4.500){2}{\rule{0.400pt}{1.084pt}}
\put(588.0,582.0){\rule[-0.200pt]{0.400pt}{0.964pt}}
\put(590.0,579.0){\rule[-0.200pt]{0.400pt}{0.482pt}}
\put(590.67,579){\rule{0.400pt}{1.204pt}}
\multiput(590.17,579.00)(1.000,2.500){2}{\rule{0.400pt}{0.602pt}}
\put(590.0,579.0){\usebox{\plotpoint}}
\put(591.67,576){\rule{0.400pt}{1.204pt}}
\multiput(591.17,578.50)(1.000,-2.500){2}{\rule{0.400pt}{0.602pt}}
\put(592.67,573){\rule{0.400pt}{0.723pt}}
\multiput(592.17,574.50)(1.000,-1.500){2}{\rule{0.400pt}{0.361pt}}
\put(592.0,581.0){\rule[-0.200pt]{0.400pt}{0.723pt}}
\put(593.67,572){\rule{0.400pt}{0.964pt}}
\multiput(593.17,572.00)(1.000,2.000){2}{\rule{0.400pt}{0.482pt}}
\put(594.67,568){\rule{0.400pt}{1.927pt}}
\multiput(594.17,572.00)(1.000,-4.000){2}{\rule{0.400pt}{0.964pt}}
\put(594.0,572.0){\usebox{\plotpoint}}
\put(596,570.67){\rule{0.241pt}{0.400pt}}
\multiput(596.00,571.17)(0.500,-1.000){2}{\rule{0.120pt}{0.400pt}}
\put(596.0,568.0){\rule[-0.200pt]{0.400pt}{0.964pt}}
\put(596.67,570){\rule{0.400pt}{0.964pt}}
\multiput(596.17,572.00)(1.000,-2.000){2}{\rule{0.400pt}{0.482pt}}
\put(597.67,566){\rule{0.400pt}{0.964pt}}
\multiput(597.17,568.00)(1.000,-2.000){2}{\rule{0.400pt}{0.482pt}}
\put(597.0,571.0){\rule[-0.200pt]{0.400pt}{0.723pt}}
\put(598.67,556){\rule{0.400pt}{1.927pt}}
\multiput(598.17,560.00)(1.000,-4.000){2}{\rule{0.400pt}{0.964pt}}
\put(599.67,556){\rule{0.400pt}{3.373pt}}
\multiput(599.17,556.00)(1.000,7.000){2}{\rule{0.400pt}{1.686pt}}
\put(599.0,564.0){\rule[-0.200pt]{0.400pt}{0.482pt}}
\put(600.67,558){\rule{0.400pt}{0.482pt}}
\multiput(600.17,559.00)(1.000,-1.000){2}{\rule{0.400pt}{0.241pt}}
\put(602,557.67){\rule{0.241pt}{0.400pt}}
\multiput(602.00,557.17)(0.500,1.000){2}{\rule{0.120pt}{0.400pt}}
\put(601.0,560.0){\rule[-0.200pt]{0.400pt}{2.409pt}}
\put(602.67,557){\rule{0.400pt}{0.964pt}}
\multiput(602.17,559.00)(1.000,-2.000){2}{\rule{0.400pt}{0.482pt}}
\put(603.67,551){\rule{0.400pt}{1.445pt}}
\multiput(603.17,554.00)(1.000,-3.000){2}{\rule{0.400pt}{0.723pt}}
\put(603.0,559.0){\rule[-0.200pt]{0.400pt}{0.482pt}}
\put(604.67,550){\rule{0.400pt}{1.686pt}}
\multiput(604.17,553.50)(1.000,-3.500){2}{\rule{0.400pt}{0.843pt}}
\put(605.0,551.0){\rule[-0.200pt]{0.400pt}{1.445pt}}
\put(606.0,550.0){\usebox{\plotpoint}}
\put(606.67,546){\rule{0.400pt}{1.686pt}}
\multiput(606.17,549.50)(1.000,-3.500){2}{\rule{0.400pt}{0.843pt}}
\put(607.0,550.0){\rule[-0.200pt]{0.400pt}{0.723pt}}
\put(607.67,542){\rule{0.400pt}{1.204pt}}
\multiput(607.17,544.50)(1.000,-2.500){2}{\rule{0.400pt}{0.602pt}}
\put(608.67,542){\rule{0.400pt}{1.204pt}}
\multiput(608.17,542.00)(1.000,2.500){2}{\rule{0.400pt}{0.602pt}}
\put(608.0,546.0){\usebox{\plotpoint}}
\put(609.67,541){\rule{0.400pt}{0.964pt}}
\multiput(609.17,541.00)(1.000,2.000){2}{\rule{0.400pt}{0.482pt}}
\put(610.67,539){\rule{0.400pt}{1.445pt}}
\multiput(610.17,542.00)(1.000,-3.000){2}{\rule{0.400pt}{0.723pt}}
\put(610.0,541.0){\rule[-0.200pt]{0.400pt}{1.445pt}}
\put(612,539){\usebox{\plotpoint}}
\put(611.67,533){\rule{0.400pt}{1.445pt}}
\multiput(611.17,536.00)(1.000,-3.000){2}{\rule{0.400pt}{0.723pt}}
\put(612.67,533){\rule{0.400pt}{0.964pt}}
\multiput(612.17,533.00)(1.000,2.000){2}{\rule{0.400pt}{0.482pt}}
\put(613.67,529){\rule{0.400pt}{1.204pt}}
\multiput(613.17,531.50)(1.000,-2.500){2}{\rule{0.400pt}{0.602pt}}
\put(615,527.67){\rule{0.241pt}{0.400pt}}
\multiput(615.00,528.17)(0.500,-1.000){2}{\rule{0.120pt}{0.400pt}}
\put(614.0,534.0){\rule[-0.200pt]{0.400pt}{0.723pt}}
\put(616,528){\usebox{\plotpoint}}
\put(616,526.67){\rule{0.241pt}{0.400pt}}
\multiput(616.00,527.17)(0.500,-1.000){2}{\rule{0.120pt}{0.400pt}}
\put(617,526.67){\rule{0.241pt}{0.400pt}}
\multiput(617.00,526.17)(0.500,1.000){2}{\rule{0.120pt}{0.400pt}}
\put(617.67,521){\rule{0.400pt}{0.723pt}}
\multiput(617.17,522.50)(1.000,-1.500){2}{\rule{0.400pt}{0.361pt}}
\put(618.0,524.0){\rule[-0.200pt]{0.400pt}{0.964pt}}
\put(618.67,515){\rule{0.400pt}{2.168pt}}
\multiput(618.17,519.50)(1.000,-4.500){2}{\rule{0.400pt}{1.084pt}}
\put(619.0,521.0){\rule[-0.200pt]{0.400pt}{0.723pt}}
\put(620.0,515.0){\usebox{\plotpoint}}
\put(620.67,514){\rule{0.400pt}{0.723pt}}
\multiput(620.17,515.50)(1.000,-1.500){2}{\rule{0.400pt}{0.361pt}}
\put(621.0,515.0){\rule[-0.200pt]{0.400pt}{0.482pt}}
\put(622.0,514.0){\usebox{\plotpoint}}
\put(622.67,508){\rule{0.400pt}{0.723pt}}
\multiput(622.17,509.50)(1.000,-1.500){2}{\rule{0.400pt}{0.361pt}}
\put(624,506.67){\rule{0.241pt}{0.400pt}}
\multiput(624.00,507.17)(0.500,-1.000){2}{\rule{0.120pt}{0.400pt}}
\put(623.0,511.0){\rule[-0.200pt]{0.400pt}{0.723pt}}
\put(625,507){\usebox{\plotpoint}}
\put(624.67,501){\rule{0.400pt}{1.445pt}}
\multiput(624.17,504.00)(1.000,-3.000){2}{\rule{0.400pt}{0.723pt}}
\put(625.67,501){\rule{0.400pt}{1.445pt}}
\multiput(625.17,501.00)(1.000,3.000){2}{\rule{0.400pt}{0.723pt}}
\put(627,500.67){\rule{0.241pt}{0.400pt}}
\multiput(627.00,501.17)(0.500,-1.000){2}{\rule{0.120pt}{0.400pt}}
\put(627.67,495){\rule{0.400pt}{1.445pt}}
\multiput(627.17,498.00)(1.000,-3.000){2}{\rule{0.400pt}{0.723pt}}
\put(627.0,502.0){\rule[-0.200pt]{0.400pt}{1.204pt}}
\put(629,491.67){\rule{0.241pt}{0.400pt}}
\multiput(629.00,491.17)(0.500,1.000){2}{\rule{0.120pt}{0.400pt}}
\put(629.0,492.0){\rule[-0.200pt]{0.400pt}{0.723pt}}
\put(629.67,487){\rule{0.400pt}{1.204pt}}
\multiput(629.17,489.50)(1.000,-2.500){2}{\rule{0.400pt}{0.602pt}}
\put(631,485.67){\rule{0.241pt}{0.400pt}}
\multiput(631.00,486.17)(0.500,-1.000){2}{\rule{0.120pt}{0.400pt}}
\put(630.0,492.0){\usebox{\plotpoint}}
\put(631.67,481){\rule{0.400pt}{0.482pt}}
\multiput(631.17,481.00)(1.000,1.000){2}{\rule{0.400pt}{0.241pt}}
\put(632.67,481){\rule{0.400pt}{0.482pt}}
\multiput(632.17,482.00)(1.000,-1.000){2}{\rule{0.400pt}{0.241pt}}
\put(632.0,481.0){\rule[-0.200pt]{0.400pt}{1.204pt}}
\put(634,475.67){\rule{0.241pt}{0.400pt}}
\multiput(634.00,476.17)(0.500,-1.000){2}{\rule{0.120pt}{0.400pt}}
\put(634.0,477.0){\rule[-0.200pt]{0.400pt}{0.964pt}}
\put(635.0,476.0){\usebox{\plotpoint}}
\put(636.0,472.0){\rule[-0.200pt]{0.400pt}{0.964pt}}
\put(636.67,470){\rule{0.400pt}{0.482pt}}
\multiput(636.17,471.00)(1.000,-1.000){2}{\rule{0.400pt}{0.241pt}}
\put(636.0,472.0){\usebox{\plotpoint}}
\put(637.67,465){\rule{0.400pt}{0.964pt}}
\multiput(637.17,467.00)(1.000,-2.000){2}{\rule{0.400pt}{0.482pt}}
\put(638.0,469.0){\usebox{\plotpoint}}
\put(639,465){\usebox{\plotpoint}}
\put(638.67,462){\rule{0.400pt}{0.723pt}}
\multiput(638.17,463.50)(1.000,-1.500){2}{\rule{0.400pt}{0.361pt}}
\put(640,460.67){\rule{0.241pt}{0.400pt}}
\multiput(640.00,461.17)(0.500,-1.000){2}{\rule{0.120pt}{0.400pt}}
\put(640.67,454){\rule{0.400pt}{0.482pt}}
\multiput(640.17,454.00)(1.000,1.000){2}{\rule{0.400pt}{0.241pt}}
\put(641.67,454){\rule{0.400pt}{0.482pt}}
\multiput(641.17,455.00)(1.000,-1.000){2}{\rule{0.400pt}{0.241pt}}
\put(641.0,454.0){\rule[-0.200pt]{0.400pt}{1.686pt}}
\put(642.67,446){\rule{0.400pt}{0.723pt}}
\multiput(642.17,447.50)(1.000,-1.500){2}{\rule{0.400pt}{0.361pt}}
\put(644,445.67){\rule{0.241pt}{0.400pt}}
\multiput(644.00,445.17)(0.500,1.000){2}{\rule{0.120pt}{0.400pt}}
\put(643.0,449.0){\rule[-0.200pt]{0.400pt}{1.204pt}}
\put(645,441.67){\rule{0.241pt}{0.400pt}}
\multiput(645.00,442.17)(0.500,-1.000){2}{\rule{0.120pt}{0.400pt}}
\put(645.67,438){\rule{0.400pt}{0.964pt}}
\multiput(645.17,440.00)(1.000,-2.000){2}{\rule{0.400pt}{0.482pt}}
\put(645.0,443.0){\rule[-0.200pt]{0.400pt}{0.964pt}}
\put(646.67,436){\rule{0.400pt}{0.723pt}}
\multiput(646.17,437.50)(1.000,-1.500){2}{\rule{0.400pt}{0.361pt}}
\put(648,434.67){\rule{0.241pt}{0.400pt}}
\multiput(648.00,435.17)(0.500,-1.000){2}{\rule{0.120pt}{0.400pt}}
\put(647.0,438.0){\usebox{\plotpoint}}
\put(648.67,428){\rule{0.400pt}{0.482pt}}
\multiput(648.17,428.00)(1.000,1.000){2}{\rule{0.400pt}{0.241pt}}
\put(649.0,428.0){\rule[-0.200pt]{0.400pt}{1.686pt}}
\put(649.67,425){\rule{0.400pt}{0.482pt}}
\multiput(649.17,426.00)(1.000,-1.000){2}{\rule{0.400pt}{0.241pt}}
\put(650.67,422){\rule{0.400pt}{0.723pt}}
\multiput(650.17,423.50)(1.000,-1.500){2}{\rule{0.400pt}{0.361pt}}
\put(650.0,427.0){\rule[-0.200pt]{0.400pt}{0.723pt}}
\put(651.67,418){\rule{0.400pt}{1.445pt}}
\multiput(651.17,421.00)(1.000,-3.000){2}{\rule{0.400pt}{0.723pt}}
\put(652.0,422.0){\rule[-0.200pt]{0.400pt}{0.482pt}}
\put(653.0,418.0){\usebox{\plotpoint}}
\put(654,412.67){\rule{0.241pt}{0.400pt}}
\multiput(654.00,413.17)(0.500,-1.000){2}{\rule{0.120pt}{0.400pt}}
\put(654.67,407){\rule{0.400pt}{1.445pt}}
\multiput(654.17,410.00)(1.000,-3.000){2}{\rule{0.400pt}{0.723pt}}
\put(654.0,414.0){\rule[-0.200pt]{0.400pt}{0.964pt}}
\put(655.67,404){\rule{0.400pt}{1.204pt}}
\multiput(655.17,406.50)(1.000,-2.500){2}{\rule{0.400pt}{0.602pt}}
\put(657,403.67){\rule{0.241pt}{0.400pt}}
\multiput(657.00,403.17)(0.500,1.000){2}{\rule{0.120pt}{0.400pt}}
\put(656.0,407.0){\rule[-0.200pt]{0.400pt}{0.482pt}}
\put(657.67,400){\rule{0.400pt}{0.482pt}}
\multiput(657.17,401.00)(1.000,-1.000){2}{\rule{0.400pt}{0.241pt}}
\put(658.0,402.0){\rule[-0.200pt]{0.400pt}{0.723pt}}
\put(658.67,396){\rule{0.400pt}{0.482pt}}
\multiput(658.17,397.00)(1.000,-1.000){2}{\rule{0.400pt}{0.241pt}}
\put(659.67,393){\rule{0.400pt}{0.723pt}}
\multiput(659.17,394.50)(1.000,-1.500){2}{\rule{0.400pt}{0.361pt}}
\put(659.0,398.0){\rule[-0.200pt]{0.400pt}{0.482pt}}
\put(660.67,390){\rule{0.400pt}{0.482pt}}
\multiput(660.17,391.00)(1.000,-1.000){2}{\rule{0.400pt}{0.241pt}}
\put(662,389.67){\rule{0.241pt}{0.400pt}}
\multiput(662.00,389.17)(0.500,1.000){2}{\rule{0.120pt}{0.400pt}}
\put(661.0,392.0){\usebox{\plotpoint}}
\put(663,384.67){\rule{0.241pt}{0.400pt}}
\multiput(663.00,384.17)(0.500,1.000){2}{\rule{0.120pt}{0.400pt}}
\put(663.67,382){\rule{0.400pt}{0.964pt}}
\multiput(663.17,384.00)(1.000,-2.000){2}{\rule{0.400pt}{0.482pt}}
\put(663.0,385.0){\rule[-0.200pt]{0.400pt}{1.445pt}}
\put(665,378.67){\rule{0.241pt}{0.400pt}}
\multiput(665.00,378.17)(0.500,1.000){2}{\rule{0.120pt}{0.400pt}}
\put(665.67,377){\rule{0.400pt}{0.723pt}}
\multiput(665.17,378.50)(1.000,-1.500){2}{\rule{0.400pt}{0.361pt}}
\put(665.0,379.0){\rule[-0.200pt]{0.400pt}{0.723pt}}
\put(667.0,374.0){\rule[-0.200pt]{0.400pt}{0.723pt}}
\put(667.0,374.0){\usebox{\plotpoint}}
\put(667.67,369){\rule{0.400pt}{0.482pt}}
\multiput(667.17,370.00)(1.000,-1.000){2}{\rule{0.400pt}{0.241pt}}
\put(668.67,366){\rule{0.400pt}{0.723pt}}
\multiput(668.17,367.50)(1.000,-1.500){2}{\rule{0.400pt}{0.361pt}}
\put(668.0,371.0){\rule[-0.200pt]{0.400pt}{0.723pt}}
\put(670.0,364.0){\rule[-0.200pt]{0.400pt}{0.482pt}}
\put(671,362.67){\rule{0.241pt}{0.400pt}}
\multiput(671.00,363.17)(0.500,-1.000){2}{\rule{0.120pt}{0.400pt}}
\put(670.0,364.0){\usebox{\plotpoint}}
\put(671.67,358){\rule{0.400pt}{0.723pt}}
\multiput(671.17,359.50)(1.000,-1.500){2}{\rule{0.400pt}{0.361pt}}
\put(673,356.67){\rule{0.241pt}{0.400pt}}
\multiput(673.00,357.17)(0.500,-1.000){2}{\rule{0.120pt}{0.400pt}}
\put(672.0,361.0){\rule[-0.200pt]{0.400pt}{0.482pt}}
\put(674,353.67){\rule{0.241pt}{0.400pt}}
\multiput(674.00,353.17)(0.500,1.000){2}{\rule{0.120pt}{0.400pt}}
\put(674.0,354.0){\rule[-0.200pt]{0.400pt}{0.723pt}}
\put(675.0,355.0){\usebox{\plotpoint}}
\put(675.67,349){\rule{0.400pt}{0.964pt}}
\multiput(675.17,351.00)(1.000,-2.000){2}{\rule{0.400pt}{0.482pt}}
\put(676.0,353.0){\rule[-0.200pt]{0.400pt}{0.482pt}}
\put(676.67,345){\rule{0.400pt}{0.723pt}}
\multiput(676.17,346.50)(1.000,-1.500){2}{\rule{0.400pt}{0.361pt}}
\put(677.67,343){\rule{0.400pt}{0.482pt}}
\multiput(677.17,344.00)(1.000,-1.000){2}{\rule{0.400pt}{0.241pt}}
\put(677.0,348.0){\usebox{\plotpoint}}
\put(678.67,342){\rule{0.400pt}{0.482pt}}
\multiput(678.17,343.00)(1.000,-1.000){2}{\rule{0.400pt}{0.241pt}}
\put(679.67,340){\rule{0.400pt}{0.482pt}}
\multiput(679.17,341.00)(1.000,-1.000){2}{\rule{0.400pt}{0.241pt}}
\put(679.0,343.0){\usebox{\plotpoint}}
\put(681,340){\usebox{\plotpoint}}
\put(681,338.67){\rule{0.241pt}{0.400pt}}
\multiput(681.00,339.17)(0.500,-1.000){2}{\rule{0.120pt}{0.400pt}}
\put(681.67,335){\rule{0.400pt}{0.964pt}}
\multiput(681.17,337.00)(1.000,-2.000){2}{\rule{0.400pt}{0.482pt}}
\put(683,335){\usebox{\plotpoint}}
\put(683,334.67){\rule{0.241pt}{0.400pt}}
\multiput(683.00,334.17)(0.500,1.000){2}{\rule{0.120pt}{0.400pt}}
\put(683.67,333){\rule{0.400pt}{0.723pt}}
\multiput(683.17,334.50)(1.000,-1.500){2}{\rule{0.400pt}{0.361pt}}
\put(685.0,331.0){\rule[-0.200pt]{0.400pt}{0.482pt}}
\put(685.0,331.0){\usebox{\plotpoint}}
\put(685.67,327){\rule{0.400pt}{0.482pt}}
\multiput(685.17,328.00)(1.000,-1.000){2}{\rule{0.400pt}{0.241pt}}
\put(687,325.67){\rule{0.241pt}{0.400pt}}
\multiput(687.00,326.17)(0.500,-1.000){2}{\rule{0.120pt}{0.400pt}}
\put(686.0,329.0){\rule[-0.200pt]{0.400pt}{0.482pt}}
\put(688,323.67){\rule{0.241pt}{0.400pt}}
\multiput(688.00,324.17)(0.500,-1.000){2}{\rule{0.120pt}{0.400pt}}
\put(689,322.67){\rule{0.241pt}{0.400pt}}
\multiput(689.00,323.17)(0.500,-1.000){2}{\rule{0.120pt}{0.400pt}}
\put(688.0,325.0){\usebox{\plotpoint}}
\put(690,323){\usebox{\plotpoint}}
\put(689.67,319){\rule{0.400pt}{0.964pt}}
\multiput(689.17,321.00)(1.000,-2.000){2}{\rule{0.400pt}{0.482pt}}
\put(691,318.67){\rule{0.241pt}{0.400pt}}
\multiput(691.00,318.17)(0.500,1.000){2}{\rule{0.120pt}{0.400pt}}
\put(692,320){\usebox{\plotpoint}}
\put(691.67,318){\rule{0.400pt}{0.482pt}}
\multiput(691.17,319.00)(1.000,-1.000){2}{\rule{0.400pt}{0.241pt}}
\put(693.0,318.0){\usebox{\plotpoint}}
\put(694.0,316.0){\rule[-0.200pt]{0.400pt}{0.482pt}}
\put(694.0,316.0){\usebox{\plotpoint}}
\put(694.67,312){\rule{0.400pt}{0.482pt}}
\multiput(694.17,313.00)(1.000,-1.000){2}{\rule{0.400pt}{0.241pt}}
\put(695.67,312){\rule{0.400pt}{0.482pt}}
\multiput(695.17,312.00)(1.000,1.000){2}{\rule{0.400pt}{0.241pt}}
\put(695.0,314.0){\rule[-0.200pt]{0.400pt}{0.482pt}}
\put(697,310.67){\rule{0.241pt}{0.400pt}}
\multiput(697.00,310.17)(0.500,1.000){2}{\rule{0.120pt}{0.400pt}}
\put(697.67,309){\rule{0.400pt}{0.723pt}}
\multiput(697.17,310.50)(1.000,-1.500){2}{\rule{0.400pt}{0.361pt}}
\put(697.0,311.0){\rule[-0.200pt]{0.400pt}{0.723pt}}
\put(698.67,307){\rule{0.400pt}{0.964pt}}
\multiput(698.17,309.00)(1.000,-2.000){2}{\rule{0.400pt}{0.482pt}}
\put(699.0,309.0){\rule[-0.200pt]{0.400pt}{0.482pt}}
\put(700.0,307.0){\usebox{\plotpoint}}
\put(700.67,307){\rule{0.400pt}{0.482pt}}
\multiput(700.17,308.00)(1.000,-1.000){2}{\rule{0.400pt}{0.241pt}}
\put(702,305.67){\rule{0.241pt}{0.400pt}}
\multiput(702.00,306.17)(0.500,-1.000){2}{\rule{0.120pt}{0.400pt}}
\put(701.0,307.0){\rule[-0.200pt]{0.400pt}{0.482pt}}
\put(703,305.67){\rule{0.241pt}{0.400pt}}
\multiput(703.00,306.17)(0.500,-1.000){2}{\rule{0.120pt}{0.400pt}}
\put(703.0,306.0){\usebox{\plotpoint}}
\put(704,302.67){\rule{0.241pt}{0.400pt}}
\multiput(704.00,303.17)(0.500,-1.000){2}{\rule{0.120pt}{0.400pt}}
\put(705,302.67){\rule{0.241pt}{0.400pt}}
\multiput(705.00,302.17)(0.500,1.000){2}{\rule{0.120pt}{0.400pt}}
\put(704.0,304.0){\rule[-0.200pt]{0.400pt}{0.482pt}}
\put(706,301.67){\rule{0.241pt}{0.400pt}}
\multiput(706.00,302.17)(0.500,-1.000){2}{\rule{0.120pt}{0.400pt}}
\put(707,300.67){\rule{0.241pt}{0.400pt}}
\multiput(707.00,301.17)(0.500,-1.000){2}{\rule{0.120pt}{0.400pt}}
\put(706.0,303.0){\usebox{\plotpoint}}
\put(708,300.67){\rule{0.241pt}{0.400pt}}
\multiput(708.00,301.17)(0.500,-1.000){2}{\rule{0.120pt}{0.400pt}}
\put(708.0,301.0){\usebox{\plotpoint}}
\put(709.0,301.0){\usebox{\plotpoint}}
\put(710,298.67){\rule{0.241pt}{0.400pt}}
\multiput(710.00,299.17)(0.500,-1.000){2}{\rule{0.120pt}{0.400pt}}
\put(710.0,300.0){\usebox{\plotpoint}}
\put(711,299){\usebox{\plotpoint}}
\put(711.0,299.0){\rule[-0.200pt]{0.482pt}{0.400pt}}
\put(713.0,298.0){\usebox{\plotpoint}}
\put(714,296.67){\rule{0.241pt}{0.400pt}}
\multiput(714.00,297.17)(0.500,-1.000){2}{\rule{0.120pt}{0.400pt}}
\put(713.0,298.0){\usebox{\plotpoint}}
\put(715,297.67){\rule{0.241pt}{0.400pt}}
\multiput(715.00,298.17)(0.500,-1.000){2}{\rule{0.120pt}{0.400pt}}
\put(716,296.67){\rule{0.241pt}{0.400pt}}
\multiput(716.00,297.17)(0.500,-1.000){2}{\rule{0.120pt}{0.400pt}}
\put(715.0,297.0){\rule[-0.200pt]{0.400pt}{0.482pt}}
\put(717,297){\usebox{\plotpoint}}
\put(718,295.67){\rule{0.241pt}{0.400pt}}
\multiput(718.00,296.17)(0.500,-1.000){2}{\rule{0.120pt}{0.400pt}}
\put(717.0,297.0){\usebox{\plotpoint}}
\put(719,296){\usebox{\plotpoint}}
\put(721,294.67){\rule{0.241pt}{0.400pt}}
\multiput(721.00,295.17)(0.500,-1.000){2}{\rule{0.120pt}{0.400pt}}
\put(719.0,296.0){\rule[-0.200pt]{0.482pt}{0.400pt}}
\put(721.67,293){\rule{0.400pt}{0.723pt}}
\multiput(721.17,294.50)(1.000,-1.500){2}{\rule{0.400pt}{0.361pt}}
\put(722.67,293){\rule{0.400pt}{0.482pt}}
\multiput(722.17,293.00)(1.000,1.000){2}{\rule{0.400pt}{0.241pt}}
\put(722.0,295.0){\usebox{\plotpoint}}
\put(723.67,294){\rule{0.400pt}{0.723pt}}
\multiput(723.17,294.00)(1.000,1.500){2}{\rule{0.400pt}{0.361pt}}
\put(725,295.67){\rule{0.241pt}{0.400pt}}
\multiput(725.00,296.17)(0.500,-1.000){2}{\rule{0.120pt}{0.400pt}}
\put(724.0,294.0){\usebox{\plotpoint}}
\put(725.67,295){\rule{0.400pt}{0.482pt}}
\multiput(725.17,295.00)(1.000,1.000){2}{\rule{0.400pt}{0.241pt}}
\put(726.0,295.0){\usebox{\plotpoint}}
\put(727,297){\usebox{\plotpoint}}
\put(727,295.67){\rule{0.241pt}{0.400pt}}
\multiput(727.00,296.17)(0.500,-1.000){2}{\rule{0.120pt}{0.400pt}}
\put(728,294.67){\rule{0.241pt}{0.400pt}}
\multiput(728.00,295.17)(0.500,-1.000){2}{\rule{0.120pt}{0.400pt}}
\put(729,295){\usebox{\plotpoint}}
\put(730,294.67){\rule{0.241pt}{0.400pt}}
\multiput(730.00,294.17)(0.500,1.000){2}{\rule{0.120pt}{0.400pt}}
\put(729.0,295.0){\usebox{\plotpoint}}
\put(731,294.67){\rule{0.241pt}{0.400pt}}
\multiput(731.00,294.17)(0.500,1.000){2}{\rule{0.120pt}{0.400pt}}
\put(732,294.67){\rule{0.241pt}{0.400pt}}
\multiput(732.00,295.17)(0.500,-1.000){2}{\rule{0.120pt}{0.400pt}}
\put(731.0,295.0){\usebox{\plotpoint}}
\put(733,295){\usebox{\plotpoint}}
\put(734,293.67){\rule{0.241pt}{0.400pt}}
\multiput(734.00,294.17)(0.500,-1.000){2}{\rule{0.120pt}{0.400pt}}
\put(733.0,295.0){\usebox{\plotpoint}}
\put(735,295.67){\rule{0.241pt}{0.400pt}}
\multiput(735.00,295.17)(0.500,1.000){2}{\rule{0.120pt}{0.400pt}}
\put(735.0,294.0){\rule[-0.200pt]{0.400pt}{0.482pt}}
\put(736,297){\usebox{\plotpoint}}
\put(736,295.67){\rule{0.241pt}{0.400pt}}
\multiput(736.00,296.17)(0.500,-1.000){2}{\rule{0.120pt}{0.400pt}}
\put(737.0,296.0){\usebox{\plotpoint}}
\put(738,296.67){\rule{0.241pt}{0.400pt}}
\multiput(738.00,296.17)(0.500,1.000){2}{\rule{0.120pt}{0.400pt}}
\put(738.0,296.0){\usebox{\plotpoint}}
\put(740,297.67){\rule{0.241pt}{0.400pt}}
\multiput(740.00,297.17)(0.500,1.000){2}{\rule{0.120pt}{0.400pt}}
\put(741,298.67){\rule{0.241pt}{0.400pt}}
\multiput(741.00,298.17)(0.500,1.000){2}{\rule{0.120pt}{0.400pt}}
\put(739.0,298.0){\usebox{\plotpoint}}
\put(741.67,297){\rule{0.400pt}{0.482pt}}
\multiput(741.17,297.00)(1.000,1.000){2}{\rule{0.400pt}{0.241pt}}
\put(742.0,297.0){\rule[-0.200pt]{0.400pt}{0.723pt}}
\put(743.0,298.0){\usebox{\plotpoint}}
\put(743.67,298){\rule{0.400pt}{0.723pt}}
\multiput(743.17,298.00)(1.000,1.500){2}{\rule{0.400pt}{0.361pt}}
\put(743.0,298.0){\usebox{\plotpoint}}
\put(745,298.67){\rule{0.241pt}{0.400pt}}
\multiput(745.00,298.17)(0.500,1.000){2}{\rule{0.120pt}{0.400pt}}
\put(746,298.67){\rule{0.241pt}{0.400pt}}
\multiput(746.00,299.17)(0.500,-1.000){2}{\rule{0.120pt}{0.400pt}}
\put(745.0,299.0){\rule[-0.200pt]{0.400pt}{0.482pt}}
\put(747,297.67){\rule{0.241pt}{0.400pt}}
\multiput(747.00,297.17)(0.500,1.000){2}{\rule{0.120pt}{0.400pt}}
\put(747.67,299){\rule{0.400pt}{0.482pt}}
\multiput(747.17,299.00)(1.000,1.000){2}{\rule{0.400pt}{0.241pt}}
\put(747.0,298.0){\usebox{\plotpoint}}
\put(748.67,299){\rule{0.400pt}{0.723pt}}
\multiput(748.17,300.50)(1.000,-1.500){2}{\rule{0.400pt}{0.361pt}}
\put(749.0,301.0){\usebox{\plotpoint}}
\put(750.0,299.0){\rule[-0.200pt]{0.400pt}{0.723pt}}
\put(750.0,302.0){\rule[-0.200pt]{0.482pt}{0.400pt}}
\put(751.67,300){\rule{0.400pt}{0.482pt}}
\multiput(751.17,300.00)(1.000,1.000){2}{\rule{0.400pt}{0.241pt}}
\put(752.0,300.0){\rule[-0.200pt]{0.400pt}{0.482pt}}
\put(753.0,302.0){\usebox{\plotpoint}}
\put(754.0,302.0){\rule[-0.200pt]{0.400pt}{0.482pt}}
\put(755,302.67){\rule{0.241pt}{0.400pt}}
\multiput(755.00,303.17)(0.500,-1.000){2}{\rule{0.120pt}{0.400pt}}
\put(754.0,304.0){\usebox{\plotpoint}}
\put(756,303.67){\rule{0.241pt}{0.400pt}}
\multiput(756.00,304.17)(0.500,-1.000){2}{\rule{0.120pt}{0.400pt}}
\put(756.0,303.0){\rule[-0.200pt]{0.400pt}{0.482pt}}
\put(757,304){\usebox{\plotpoint}}
\put(757,303.67){\rule{0.241pt}{0.400pt}}
\multiput(757.00,303.17)(0.500,1.000){2}{\rule{0.120pt}{0.400pt}}
\put(758,303.67){\rule{0.241pt}{0.400pt}}
\multiput(758.00,304.17)(0.500,-1.000){2}{\rule{0.120pt}{0.400pt}}
\put(759,304.67){\rule{0.241pt}{0.400pt}}
\multiput(759.00,305.17)(0.500,-1.000){2}{\rule{0.120pt}{0.400pt}}
\put(759.0,304.0){\rule[-0.200pt]{0.400pt}{0.482pt}}
\put(761,304.67){\rule{0.241pt}{0.400pt}}
\multiput(761.00,304.17)(0.500,1.000){2}{\rule{0.120pt}{0.400pt}}
\put(762,305.67){\rule{0.241pt}{0.400pt}}
\multiput(762.00,305.17)(0.500,1.000){2}{\rule{0.120pt}{0.400pt}}
\put(760.0,305.0){\usebox{\plotpoint}}
\put(763,307){\usebox{\plotpoint}}
\put(763,306.67){\rule{0.241pt}{0.400pt}}
\multiput(763.00,306.17)(0.500,1.000){2}{\rule{0.120pt}{0.400pt}}
\put(764,307.67){\rule{0.241pt}{0.400pt}}
\multiput(764.00,307.17)(0.500,1.000){2}{\rule{0.120pt}{0.400pt}}
\put(765,309){\usebox{\plotpoint}}
\put(765.0,309.0){\usebox{\plotpoint}}
\put(766.0,309.0){\usebox{\plotpoint}}
\put(766.67,310){\rule{0.400pt}{0.723pt}}
\multiput(766.17,310.00)(1.000,1.500){2}{\rule{0.400pt}{0.361pt}}
\put(766.0,310.0){\usebox{\plotpoint}}
\put(768,309.67){\rule{0.241pt}{0.400pt}}
\multiput(768.00,310.17)(0.500,-1.000){2}{\rule{0.120pt}{0.400pt}}
\put(769,309.67){\rule{0.241pt}{0.400pt}}
\multiput(769.00,309.17)(0.500,1.000){2}{\rule{0.120pt}{0.400pt}}
\put(768.0,311.0){\rule[-0.200pt]{0.400pt}{0.482pt}}
\put(769.67,310){\rule{0.400pt}{0.482pt}}
\multiput(769.17,311.00)(1.000,-1.000){2}{\rule{0.400pt}{0.241pt}}
\put(770.67,310){\rule{0.400pt}{0.723pt}}
\multiput(770.17,310.00)(1.000,1.500){2}{\rule{0.400pt}{0.361pt}}
\put(770.0,311.0){\usebox{\plotpoint}}
\put(771.67,312){\rule{0.400pt}{0.482pt}}
\multiput(771.17,312.00)(1.000,1.000){2}{\rule{0.400pt}{0.241pt}}
\put(772.0,312.0){\usebox{\plotpoint}}
\put(773,314){\usebox{\plotpoint}}
\put(774,313.67){\rule{0.241pt}{0.400pt}}
\multiput(774.00,313.17)(0.500,1.000){2}{\rule{0.120pt}{0.400pt}}
\put(773.0,314.0){\usebox{\plotpoint}}
\put(775,314.67){\rule{0.241pt}{0.400pt}}
\multiput(775.00,315.17)(0.500,-1.000){2}{\rule{0.120pt}{0.400pt}}
\put(776,314.67){\rule{0.241pt}{0.400pt}}
\multiput(776.00,314.17)(0.500,1.000){2}{\rule{0.120pt}{0.400pt}}
\put(775.0,315.0){\usebox{\plotpoint}}
\put(777,317.67){\rule{0.241pt}{0.400pt}}
\multiput(777.00,317.17)(0.500,1.000){2}{\rule{0.120pt}{0.400pt}}
\put(777.0,316.0){\rule[-0.200pt]{0.400pt}{0.482pt}}
\put(778.67,319){\rule{0.400pt}{0.482pt}}
\multiput(778.17,319.00)(1.000,1.000){2}{\rule{0.400pt}{0.241pt}}
\put(778.0,319.0){\usebox{\plotpoint}}
\put(780,320.67){\rule{0.241pt}{0.400pt}}
\multiput(780.00,321.17)(0.500,-1.000){2}{\rule{0.120pt}{0.400pt}}
\put(780.67,321){\rule{0.400pt}{0.482pt}}
\multiput(780.17,321.00)(1.000,1.000){2}{\rule{0.400pt}{0.241pt}}
\put(780.0,321.0){\usebox{\plotpoint}}
\put(782,322.67){\rule{0.241pt}{0.400pt}}
\multiput(782.00,323.17)(0.500,-1.000){2}{\rule{0.120pt}{0.400pt}}
\put(782.0,323.0){\usebox{\plotpoint}}
\put(783.0,323.0){\usebox{\plotpoint}}
\put(784,324.67){\rule{0.241pt}{0.400pt}}
\multiput(784.00,324.17)(0.500,1.000){2}{\rule{0.120pt}{0.400pt}}
\put(784.0,323.0){\rule[-0.200pt]{0.400pt}{0.482pt}}
\put(785,326){\usebox{\plotpoint}}
\put(785,325.67){\rule{0.241pt}{0.400pt}}
\multiput(785.00,325.17)(0.500,1.000){2}{\rule{0.120pt}{0.400pt}}
\put(786,326.67){\rule{0.241pt}{0.400pt}}
\multiput(786.00,326.17)(0.500,1.000){2}{\rule{0.120pt}{0.400pt}}
\put(787,328){\usebox{\plotpoint}}
\put(787,327.67){\rule{0.241pt}{0.400pt}}
\multiput(787.00,327.17)(0.500,1.000){2}{\rule{0.120pt}{0.400pt}}
\put(787.67,329){\rule{0.400pt}{0.482pt}}
\multiput(787.17,329.00)(1.000,1.000){2}{\rule{0.400pt}{0.241pt}}
\put(788.67,329){\rule{0.400pt}{0.482pt}}
\multiput(788.17,329.00)(1.000,1.000){2}{\rule{0.400pt}{0.241pt}}
\put(789.0,329.0){\rule[-0.200pt]{0.400pt}{0.482pt}}
\put(790.67,331){\rule{0.400pt}{0.723pt}}
\multiput(790.17,331.00)(1.000,1.500){2}{\rule{0.400pt}{0.361pt}}
\put(790.0,331.0){\usebox{\plotpoint}}
\put(791.67,333){\rule{0.400pt}{0.964pt}}
\multiput(791.17,333.00)(1.000,2.000){2}{\rule{0.400pt}{0.482pt}}
\put(793,335.67){\rule{0.241pt}{0.400pt}}
\multiput(793.00,336.17)(0.500,-1.000){2}{\rule{0.120pt}{0.400pt}}
\put(792.0,333.0){\usebox{\plotpoint}}
\put(794,337.67){\rule{0.241pt}{0.400pt}}
\multiput(794.00,338.17)(0.500,-1.000){2}{\rule{0.120pt}{0.400pt}}
\put(795,337.67){\rule{0.241pt}{0.400pt}}
\multiput(795.00,337.17)(0.500,1.000){2}{\rule{0.120pt}{0.400pt}}
\put(794.0,336.0){\rule[-0.200pt]{0.400pt}{0.723pt}}
\put(796,340.67){\rule{0.241pt}{0.400pt}}
\multiput(796.00,340.17)(0.500,1.000){2}{\rule{0.120pt}{0.400pt}}
\put(796.0,339.0){\rule[-0.200pt]{0.400pt}{0.482pt}}
\put(798,341.67){\rule{0.241pt}{0.400pt}}
\multiput(798.00,341.17)(0.500,1.000){2}{\rule{0.120pt}{0.400pt}}
\put(797.0,342.0){\usebox{\plotpoint}}
\put(799.0,343.0){\rule[-0.200pt]{0.400pt}{0.482pt}}
\put(799.67,345){\rule{0.400pt}{0.723pt}}
\multiput(799.17,345.00)(1.000,1.500){2}{\rule{0.400pt}{0.361pt}}
\put(799.0,345.0){\usebox{\plotpoint}}
\put(801.0,348.0){\rule[-0.200pt]{0.400pt}{0.723pt}}
\put(801.0,351.0){\rule[-0.200pt]{0.482pt}{0.400pt}}
\put(802.67,352){\rule{0.400pt}{0.482pt}}
\multiput(802.17,352.00)(1.000,1.000){2}{\rule{0.400pt}{0.241pt}}
\put(803.0,351.0){\usebox{\plotpoint}}
\put(804.0,354.0){\usebox{\plotpoint}}
\put(805.0,354.0){\rule[-0.200pt]{0.400pt}{0.723pt}}
\put(805.0,357.0){\usebox{\plotpoint}}
\put(806.0,357.0){\rule[-0.200pt]{0.400pt}{0.723pt}}
\put(806.0,360.0){\rule[-0.200pt]{0.482pt}{0.400pt}}
\put(808,362.67){\rule{0.241pt}{0.400pt}}
\multiput(808.00,363.17)(0.500,-1.000){2}{\rule{0.120pt}{0.400pt}}
\put(809,362.67){\rule{0.241pt}{0.400pt}}
\multiput(809.00,362.17)(0.500,1.000){2}{\rule{0.120pt}{0.400pt}}
\put(808.0,360.0){\rule[-0.200pt]{0.400pt}{0.964pt}}
\put(809.67,366){\rule{0.400pt}{0.723pt}}
\multiput(809.17,366.00)(1.000,1.500){2}{\rule{0.400pt}{0.361pt}}
\put(810.0,364.0){\rule[-0.200pt]{0.400pt}{0.482pt}}
\put(811.0,369.0){\usebox{\plotpoint}}
\put(811.67,371){\rule{0.400pt}{0.482pt}}
\multiput(811.17,371.00)(1.000,1.000){2}{\rule{0.400pt}{0.241pt}}
\put(812.0,369.0){\rule[-0.200pt]{0.400pt}{0.482pt}}
\put(813,373){\usebox{\plotpoint}}
\put(812.67,373){\rule{0.400pt}{0.964pt}}
\multiput(812.17,373.00)(1.000,2.000){2}{\rule{0.400pt}{0.482pt}}
\put(814.0,377.0){\usebox{\plotpoint}}
\put(814.67,380){\rule{0.400pt}{0.482pt}}
\multiput(814.17,380.00)(1.000,1.000){2}{\rule{0.400pt}{0.241pt}}
\put(815.0,377.0){\rule[-0.200pt]{0.400pt}{0.723pt}}
\put(816.0,382.0){\usebox{\plotpoint}}
\put(817.0,382.0){\rule[-0.200pt]{0.400pt}{0.723pt}}
\put(817.0,385.0){\usebox{\plotpoint}}
\put(817.67,386){\rule{0.400pt}{0.964pt}}
\multiput(817.17,386.00)(1.000,2.000){2}{\rule{0.400pt}{0.482pt}}
\put(818.0,385.0){\usebox{\plotpoint}}
\put(819.0,390.0){\usebox{\plotpoint}}
\put(820,391.67){\rule{0.241pt}{0.400pt}}
\multiput(820.00,391.17)(0.500,1.000){2}{\rule{0.120pt}{0.400pt}}
\put(820.67,393){\rule{0.400pt}{0.964pt}}
\multiput(820.17,393.00)(1.000,2.000){2}{\rule{0.400pt}{0.482pt}}
\put(820.0,390.0){\rule[-0.200pt]{0.400pt}{0.482pt}}
\put(821.67,398){\rule{0.400pt}{0.723pt}}
\multiput(821.17,398.00)(1.000,1.500){2}{\rule{0.400pt}{0.361pt}}
\put(823,400.67){\rule{0.241pt}{0.400pt}}
\multiput(823.00,400.17)(0.500,1.000){2}{\rule{0.120pt}{0.400pt}}
\put(822.0,397.0){\usebox{\plotpoint}}
\put(824,402.67){\rule{0.241pt}{0.400pt}}
\multiput(824.00,403.17)(0.500,-1.000){2}{\rule{0.120pt}{0.400pt}}
\put(824.0,402.0){\rule[-0.200pt]{0.400pt}{0.482pt}}
\put(824.67,407){\rule{0.400pt}{0.482pt}}
\multiput(824.17,407.00)(1.000,1.000){2}{\rule{0.400pt}{0.241pt}}
\put(825.67,409){\rule{0.400pt}{0.723pt}}
\multiput(825.17,409.00)(1.000,1.500){2}{\rule{0.400pt}{0.361pt}}
\put(825.0,403.0){\rule[-0.200pt]{0.400pt}{0.964pt}}
\put(826.67,413){\rule{0.400pt}{0.482pt}}
\multiput(826.17,413.00)(1.000,1.000){2}{\rule{0.400pt}{0.241pt}}
\put(827.67,415){\rule{0.400pt}{0.482pt}}
\multiput(827.17,415.00)(1.000,1.000){2}{\rule{0.400pt}{0.241pt}}
\put(827.0,412.0){\usebox{\plotpoint}}
\put(829.0,417.0){\rule[-0.200pt]{0.400pt}{0.723pt}}
\put(829.0,420.0){\usebox{\plotpoint}}
\put(829.67,423){\rule{0.400pt}{0.964pt}}
\multiput(829.17,423.00)(1.000,2.000){2}{\rule{0.400pt}{0.482pt}}
\put(831,426.67){\rule{0.241pt}{0.400pt}}
\multiput(831.00,426.17)(0.500,1.000){2}{\rule{0.120pt}{0.400pt}}
\put(830.0,420.0){\rule[-0.200pt]{0.400pt}{0.723pt}}
\put(832,428.67){\rule{0.241pt}{0.400pt}}
\multiput(832.00,428.17)(0.500,1.000){2}{\rule{0.120pt}{0.400pt}}
\put(832.67,430){\rule{0.400pt}{1.445pt}}
\multiput(832.17,430.00)(1.000,3.000){2}{\rule{0.400pt}{0.723pt}}
\put(832.0,428.0){\usebox{\plotpoint}}
\put(833.67,437){\rule{0.400pt}{0.723pt}}
\multiput(833.17,437.00)(1.000,1.500){2}{\rule{0.400pt}{0.361pt}}
\put(835,439.67){\rule{0.241pt}{0.400pt}}
\multiput(835.00,439.17)(0.500,1.000){2}{\rule{0.120pt}{0.400pt}}
\put(834.0,436.0){\usebox{\plotpoint}}
\put(835.67,445){\rule{0.400pt}{0.482pt}}
\multiput(835.17,445.00)(1.000,1.000){2}{\rule{0.400pt}{0.241pt}}
\put(836.0,441.0){\rule[-0.200pt]{0.400pt}{0.964pt}}
\put(836.67,449){\rule{0.400pt}{0.482pt}}
\multiput(836.17,449.00)(1.000,1.000){2}{\rule{0.400pt}{0.241pt}}
\put(838,450.67){\rule{0.241pt}{0.400pt}}
\multiput(838.00,450.17)(0.500,1.000){2}{\rule{0.120pt}{0.400pt}}
\put(837.0,447.0){\rule[-0.200pt]{0.400pt}{0.482pt}}
\put(838.67,455){\rule{0.400pt}{0.723pt}}
\multiput(838.17,455.00)(1.000,1.500){2}{\rule{0.400pt}{0.361pt}}
\put(839.67,458){\rule{0.400pt}{0.482pt}}
\multiput(839.17,458.00)(1.000,1.000){2}{\rule{0.400pt}{0.241pt}}
\put(839.0,452.0){\rule[-0.200pt]{0.400pt}{0.723pt}}
\put(840.67,461){\rule{0.400pt}{0.723pt}}
\multiput(840.17,461.00)(1.000,1.500){2}{\rule{0.400pt}{0.361pt}}
\put(841.0,460.0){\usebox{\plotpoint}}
\put(841.67,468){\rule{0.400pt}{0.964pt}}
\multiput(841.17,468.00)(1.000,2.000){2}{\rule{0.400pt}{0.482pt}}
\put(842.67,472){\rule{0.400pt}{0.482pt}}
\multiput(842.17,472.00)(1.000,1.000){2}{\rule{0.400pt}{0.241pt}}
\put(842.0,464.0){\rule[-0.200pt]{0.400pt}{0.964pt}}
\put(844,476.67){\rule{0.241pt}{0.400pt}}
\multiput(844.00,476.17)(0.500,1.000){2}{\rule{0.120pt}{0.400pt}}
\put(845,477.67){\rule{0.241pt}{0.400pt}}
\multiput(845.00,477.17)(0.500,1.000){2}{\rule{0.120pt}{0.400pt}}
\put(844.0,474.0){\rule[-0.200pt]{0.400pt}{0.723pt}}
\put(845.67,481){\rule{0.400pt}{0.964pt}}
\multiput(845.17,481.00)(1.000,2.000){2}{\rule{0.400pt}{0.482pt}}
\put(846.67,485){\rule{0.400pt}{0.723pt}}
\multiput(846.17,485.00)(1.000,1.500){2}{\rule{0.400pt}{0.361pt}}
\put(846.0,479.0){\rule[-0.200pt]{0.400pt}{0.482pt}}
\put(847.67,490){\rule{0.400pt}{0.723pt}}
\multiput(847.17,490.00)(1.000,1.500){2}{\rule{0.400pt}{0.361pt}}
\put(848.0,488.0){\rule[-0.200pt]{0.400pt}{0.482pt}}
\put(848.67,495){\rule{0.400pt}{0.964pt}}
\multiput(848.17,495.00)(1.000,2.000){2}{\rule{0.400pt}{0.482pt}}
\put(849.67,499){\rule{0.400pt}{0.723pt}}
\multiput(849.17,499.00)(1.000,1.500){2}{\rule{0.400pt}{0.361pt}}
\put(849.0,493.0){\rule[-0.200pt]{0.400pt}{0.482pt}}
\put(851,502.67){\rule{0.241pt}{0.400pt}}
\multiput(851.00,502.17)(0.500,1.000){2}{\rule{0.120pt}{0.400pt}}
\put(851.67,504){\rule{0.400pt}{0.723pt}}
\multiput(851.17,504.00)(1.000,1.500){2}{\rule{0.400pt}{0.361pt}}
\put(851.0,502.0){\usebox{\plotpoint}}
\put(852.67,512){\rule{0.400pt}{0.723pt}}
\multiput(852.17,512.00)(1.000,1.500){2}{\rule{0.400pt}{0.361pt}}
\put(853.0,507.0){\rule[-0.200pt]{0.400pt}{1.204pt}}
\put(854,515){\usebox{\plotpoint}}
\put(853.67,515){\rule{0.400pt}{1.204pt}}
\multiput(853.17,515.00)(1.000,2.500){2}{\rule{0.400pt}{0.602pt}}
\put(855,519.67){\rule{0.241pt}{0.400pt}}
\multiput(855.00,519.17)(0.500,1.000){2}{\rule{0.120pt}{0.400pt}}
\put(855.67,523){\rule{0.400pt}{1.204pt}}
\multiput(855.17,523.00)(1.000,2.500){2}{\rule{0.400pt}{0.602pt}}
\put(856.67,526){\rule{0.400pt}{0.482pt}}
\multiput(856.17,527.00)(1.000,-1.000){2}{\rule{0.400pt}{0.241pt}}
\put(856.0,521.0){\rule[-0.200pt]{0.400pt}{0.482pt}}
\put(857.67,534){\rule{0.400pt}{0.723pt}}
\multiput(857.17,534.00)(1.000,1.500){2}{\rule{0.400pt}{0.361pt}}
\put(859,535.67){\rule{0.241pt}{0.400pt}}
\multiput(859.00,536.17)(0.500,-1.000){2}{\rule{0.120pt}{0.400pt}}
\put(858.0,526.0){\rule[-0.200pt]{0.400pt}{1.927pt}}
\put(860,539.67){\rule{0.241pt}{0.400pt}}
\multiput(860.00,540.17)(0.500,-1.000){2}{\rule{0.120pt}{0.400pt}}
\put(860.0,536.0){\rule[-0.200pt]{0.400pt}{1.204pt}}
\put(860.67,544){\rule{0.400pt}{0.723pt}}
\multiput(860.17,544.00)(1.000,1.500){2}{\rule{0.400pt}{0.361pt}}
\put(861.67,547){\rule{0.400pt}{0.723pt}}
\multiput(861.17,547.00)(1.000,1.500){2}{\rule{0.400pt}{0.361pt}}
\put(861.0,540.0){\rule[-0.200pt]{0.400pt}{0.964pt}}
\put(863.0,550.0){\rule[-0.200pt]{0.400pt}{1.445pt}}
\put(864,555.67){\rule{0.241pt}{0.400pt}}
\multiput(864.00,555.17)(0.500,1.000){2}{\rule{0.120pt}{0.400pt}}
\put(863.0,556.0){\usebox{\plotpoint}}
\put(865,562.67){\rule{0.241pt}{0.400pt}}
\multiput(865.00,562.17)(0.500,1.000){2}{\rule{0.120pt}{0.400pt}}
\put(865.0,557.0){\rule[-0.200pt]{0.400pt}{1.445pt}}
\put(865.67,563){\rule{0.400pt}{1.204pt}}
\multiput(865.17,563.00)(1.000,2.500){2}{\rule{0.400pt}{0.602pt}}
\put(866.67,568){\rule{0.400pt}{0.964pt}}
\multiput(866.17,568.00)(1.000,2.000){2}{\rule{0.400pt}{0.482pt}}
\put(866.0,563.0){\usebox{\plotpoint}}
\put(868,572){\usebox{\plotpoint}}
\put(867.67,572){\rule{0.400pt}{0.964pt}}
\multiput(867.17,572.00)(1.000,2.000){2}{\rule{0.400pt}{0.482pt}}
\put(868.67,576){\rule{0.400pt}{1.204pt}}
\multiput(868.17,576.00)(1.000,2.500){2}{\rule{0.400pt}{0.602pt}}
\put(870,581){\usebox{\plotpoint}}
\put(869.67,579){\rule{0.400pt}{0.482pt}}
\multiput(869.17,580.00)(1.000,-1.000){2}{\rule{0.400pt}{0.241pt}}
\put(870.67,583){\rule{0.400pt}{0.723pt}}
\multiput(870.17,583.00)(1.000,1.500){2}{\rule{0.400pt}{0.361pt}}
\put(871.67,586){\rule{0.400pt}{0.964pt}}
\multiput(871.17,586.00)(1.000,2.000){2}{\rule{0.400pt}{0.482pt}}
\put(871.0,579.0){\rule[-0.200pt]{0.400pt}{0.964pt}}
\put(873,590){\usebox{\plotpoint}}
\put(872.67,590){\rule{0.400pt}{0.723pt}}
\multiput(872.17,590.00)(1.000,1.500){2}{\rule{0.400pt}{0.361pt}}
\put(874,591.67){\rule{0.241pt}{0.400pt}}
\multiput(874.00,592.17)(0.500,-1.000){2}{\rule{0.120pt}{0.400pt}}
\put(874.67,596){\rule{0.400pt}{0.482pt}}
\multiput(874.17,596.00)(1.000,1.000){2}{\rule{0.400pt}{0.241pt}}
\put(876,596.67){\rule{0.241pt}{0.400pt}}
\multiput(876.00,597.17)(0.500,-1.000){2}{\rule{0.120pt}{0.400pt}}
\put(875.0,592.0){\rule[-0.200pt]{0.400pt}{0.964pt}}
\put(876.67,602){\rule{0.400pt}{0.723pt}}
\multiput(876.17,602.00)(1.000,1.500){2}{\rule{0.400pt}{0.361pt}}
\put(877.0,597.0){\rule[-0.200pt]{0.400pt}{1.204pt}}
\put(877.67,605){\rule{0.400pt}{0.723pt}}
\multiput(877.17,606.50)(1.000,-1.500){2}{\rule{0.400pt}{0.361pt}}
\put(878.67,605){\rule{0.400pt}{1.927pt}}
\multiput(878.17,605.00)(1.000,4.000){2}{\rule{0.400pt}{0.964pt}}
\put(878.0,605.0){\rule[-0.200pt]{0.400pt}{0.723pt}}
\put(879.67,611){\rule{0.400pt}{1.445pt}}
\multiput(879.17,611.00)(1.000,3.000){2}{\rule{0.400pt}{0.723pt}}
\put(881,615.67){\rule{0.241pt}{0.400pt}}
\multiput(881.00,616.17)(0.500,-1.000){2}{\rule{0.120pt}{0.400pt}}
\put(880.0,611.0){\rule[-0.200pt]{0.400pt}{0.482pt}}
\put(882,617.67){\rule{0.241pt}{0.400pt}}
\multiput(882.00,618.17)(0.500,-1.000){2}{\rule{0.120pt}{0.400pt}}
\put(882.0,616.0){\rule[-0.200pt]{0.400pt}{0.723pt}}
\put(883,618){\usebox{\plotpoint}}
\put(882.67,618){\rule{0.400pt}{0.482pt}}
\multiput(882.17,618.00)(1.000,1.000){2}{\rule{0.400pt}{0.241pt}}
\put(883.67,620){\rule{0.400pt}{0.482pt}}
\multiput(883.17,620.00)(1.000,1.000){2}{\rule{0.400pt}{0.241pt}}
\put(884.67,622){\rule{0.400pt}{1.686pt}}
\multiput(884.17,625.50)(1.000,-3.500){2}{\rule{0.400pt}{0.843pt}}
\put(885.67,622){\rule{0.400pt}{1.927pt}}
\multiput(885.17,622.00)(1.000,4.000){2}{\rule{0.400pt}{0.964pt}}
\put(885.0,622.0){\rule[-0.200pt]{0.400pt}{1.686pt}}
\put(886.67,629){\rule{0.400pt}{0.964pt}}
\multiput(886.17,631.00)(1.000,-2.000){2}{\rule{0.400pt}{0.482pt}}
\put(887.0,630.0){\rule[-0.200pt]{0.400pt}{0.723pt}}
\put(887.67,628){\rule{0.400pt}{1.686pt}}
\multiput(887.17,628.00)(1.000,3.500){2}{\rule{0.400pt}{0.843pt}}
\put(888.67,633){\rule{0.400pt}{0.482pt}}
\multiput(888.17,634.00)(1.000,-1.000){2}{\rule{0.400pt}{0.241pt}}
\put(888.0,628.0){\usebox{\plotpoint}}
\put(889.67,634){\rule{0.400pt}{0.964pt}}
\multiput(889.17,634.00)(1.000,2.000){2}{\rule{0.400pt}{0.482pt}}
\put(890.67,634){\rule{0.400pt}{0.964pt}}
\multiput(890.17,636.00)(1.000,-2.000){2}{\rule{0.400pt}{0.482pt}}
\put(890.0,633.0){\usebox{\plotpoint}}
\put(892,634){\usebox{\plotpoint}}
\put(891.67,634){\rule{0.400pt}{0.964pt}}
\multiput(891.17,634.00)(1.000,2.000){2}{\rule{0.400pt}{0.482pt}}
\put(892.67,634){\rule{0.400pt}{2.891pt}}
\multiput(892.17,634.00)(1.000,6.000){2}{\rule{0.400pt}{1.445pt}}
\put(893.67,636){\rule{0.400pt}{2.409pt}}
\multiput(893.17,641.00)(1.000,-5.000){2}{\rule{0.400pt}{1.204pt}}
\put(893.0,634.0){\rule[-0.200pt]{0.400pt}{0.964pt}}
\put(895,639.67){\rule{0.241pt}{0.400pt}}
\multiput(895.00,639.17)(0.500,1.000){2}{\rule{0.120pt}{0.400pt}}
\put(896,639.67){\rule{0.241pt}{0.400pt}}
\multiput(896.00,640.17)(0.500,-1.000){2}{\rule{0.120pt}{0.400pt}}
\put(895.0,636.0){\rule[-0.200pt]{0.400pt}{0.964pt}}
\put(896.67,642){\rule{0.400pt}{0.482pt}}
\multiput(896.17,642.00)(1.000,1.000){2}{\rule{0.400pt}{0.241pt}}
\put(898,642.67){\rule{0.241pt}{0.400pt}}
\multiput(898.00,643.17)(0.500,-1.000){2}{\rule{0.120pt}{0.400pt}}
\put(897.0,640.0){\rule[-0.200pt]{0.400pt}{0.482pt}}
\put(899,639.67){\rule{0.241pt}{0.400pt}}
\multiput(899.00,640.17)(0.500,-1.000){2}{\rule{0.120pt}{0.400pt}}
\put(899.0,641.0){\rule[-0.200pt]{0.400pt}{0.482pt}}
\put(900,639.67){\rule{0.241pt}{0.400pt}}
\multiput(900.00,640.17)(0.500,-1.000){2}{\rule{0.120pt}{0.400pt}}
\put(900.67,638){\rule{0.400pt}{0.482pt}}
\multiput(900.17,639.00)(1.000,-1.000){2}{\rule{0.400pt}{0.241pt}}
\put(900.0,640.0){\usebox{\plotpoint}}
\put(901.67,638){\rule{0.400pt}{0.482pt}}
\multiput(901.17,639.00)(1.000,-1.000){2}{\rule{0.400pt}{0.241pt}}
\put(903,636.67){\rule{0.241pt}{0.400pt}}
\multiput(903.00,637.17)(0.500,-1.000){2}{\rule{0.120pt}{0.400pt}}
\put(902.0,638.0){\rule[-0.200pt]{0.400pt}{0.482pt}}
\put(904,634.67){\rule{0.241pt}{0.400pt}}
\multiput(904.00,634.17)(0.500,1.000){2}{\rule{0.120pt}{0.400pt}}
\put(904.0,635.0){\rule[-0.200pt]{0.400pt}{0.482pt}}
\put(904.67,633){\rule{0.400pt}{0.964pt}}
\multiput(904.17,635.00)(1.000,-2.000){2}{\rule{0.400pt}{0.482pt}}
\put(905.67,633){\rule{0.400pt}{0.482pt}}
\multiput(905.17,633.00)(1.000,1.000){2}{\rule{0.400pt}{0.241pt}}
\put(905.0,636.0){\usebox{\plotpoint}}
\put(906.67,628){\rule{0.400pt}{1.204pt}}
\multiput(906.17,628.00)(1.000,2.500){2}{\rule{0.400pt}{0.602pt}}
\put(907.67,633){\rule{0.400pt}{0.482pt}}
\multiput(907.17,633.00)(1.000,1.000){2}{\rule{0.400pt}{0.241pt}}
\put(907.0,628.0){\rule[-0.200pt]{0.400pt}{1.686pt}}
\put(909,628.67){\rule{0.241pt}{0.400pt}}
\multiput(909.00,629.17)(0.500,-1.000){2}{\rule{0.120pt}{0.400pt}}
\put(909.0,630.0){\rule[-0.200pt]{0.400pt}{1.204pt}}
\put(909.67,629){\rule{0.400pt}{0.723pt}}
\multiput(909.17,630.50)(1.000,-1.500){2}{\rule{0.400pt}{0.361pt}}
\put(910.67,625){\rule{0.400pt}{0.964pt}}
\multiput(910.17,627.00)(1.000,-2.000){2}{\rule{0.400pt}{0.482pt}}
\put(910.0,629.0){\rule[-0.200pt]{0.400pt}{0.723pt}}
\put(911.67,619){\rule{0.400pt}{0.964pt}}
\multiput(911.17,619.00)(1.000,2.000){2}{\rule{0.400pt}{0.482pt}}
\put(912.67,620){\rule{0.400pt}{0.723pt}}
\multiput(912.17,621.50)(1.000,-1.500){2}{\rule{0.400pt}{0.361pt}}
\put(912.0,619.0){\rule[-0.200pt]{0.400pt}{1.445pt}}
\put(913.67,614){\rule{0.400pt}{1.686pt}}
\multiput(913.17,617.50)(1.000,-3.500){2}{\rule{0.400pt}{0.843pt}}
\put(914.0,620.0){\usebox{\plotpoint}}
\put(914.67,611){\rule{0.400pt}{0.964pt}}
\multiput(914.17,613.00)(1.000,-2.000){2}{\rule{0.400pt}{0.482pt}}
\put(915.67,611){\rule{0.400pt}{0.964pt}}
\multiput(915.17,611.00)(1.000,2.000){2}{\rule{0.400pt}{0.482pt}}
\put(915.0,614.0){\usebox{\plotpoint}}
\put(916.67,608){\rule{0.400pt}{0.964pt}}
\multiput(916.17,610.00)(1.000,-2.000){2}{\rule{0.400pt}{0.482pt}}
\put(917.67,602){\rule{0.400pt}{1.445pt}}
\multiput(917.17,605.00)(1.000,-3.000){2}{\rule{0.400pt}{0.723pt}}
\put(917.0,612.0){\rule[-0.200pt]{0.400pt}{0.723pt}}
\put(919,603.67){\rule{0.241pt}{0.400pt}}
\multiput(919.00,604.17)(0.500,-1.000){2}{\rule{0.120pt}{0.400pt}}
\put(919.0,602.0){\rule[-0.200pt]{0.400pt}{0.723pt}}
\put(919.67,594){\rule{0.400pt}{0.964pt}}
\multiput(919.17,596.00)(1.000,-2.000){2}{\rule{0.400pt}{0.482pt}}
\put(921,593.67){\rule{0.241pt}{0.400pt}}
\multiput(921.00,593.17)(0.500,1.000){2}{\rule{0.120pt}{0.400pt}}
\put(920.0,598.0){\rule[-0.200pt]{0.400pt}{1.445pt}}
\put(922,595){\usebox{\plotpoint}}
\put(921.67,588){\rule{0.400pt}{1.686pt}}
\multiput(921.17,591.50)(1.000,-3.500){2}{\rule{0.400pt}{0.843pt}}
\put(923,587.67){\rule{0.241pt}{0.400pt}}
\multiput(923.00,587.17)(0.500,1.000){2}{\rule{0.120pt}{0.400pt}}
\put(923.67,583){\rule{0.400pt}{0.482pt}}
\multiput(923.17,584.00)(1.000,-1.000){2}{\rule{0.400pt}{0.241pt}}
\put(924.0,585.0){\rule[-0.200pt]{0.400pt}{0.964pt}}
\put(925,583){\usebox{\plotpoint}}
\put(924.67,577){\rule{0.400pt}{1.445pt}}
\multiput(924.17,580.00)(1.000,-3.000){2}{\rule{0.400pt}{0.723pt}}
\put(925.67,573){\rule{0.400pt}{0.964pt}}
\multiput(925.17,575.00)(1.000,-2.000){2}{\rule{0.400pt}{0.482pt}}
\put(927,567.67){\rule{0.241pt}{0.400pt}}
\multiput(927.00,568.17)(0.500,-1.000){2}{\rule{0.120pt}{0.400pt}}
\put(927.67,565){\rule{0.400pt}{0.723pt}}
\multiput(927.17,566.50)(1.000,-1.500){2}{\rule{0.400pt}{0.361pt}}
\put(927.0,569.0){\rule[-0.200pt]{0.400pt}{0.964pt}}
\put(929.0,560.0){\rule[-0.200pt]{0.400pt}{1.204pt}}
\put(929.0,560.0){\usebox{\plotpoint}}
\put(929.67,552){\rule{0.400pt}{1.204pt}}
\multiput(929.17,554.50)(1.000,-2.500){2}{\rule{0.400pt}{0.602pt}}
\put(930.0,557.0){\rule[-0.200pt]{0.400pt}{0.723pt}}
\put(931.0,552.0){\usebox{\plotpoint}}
\put(931.67,542){\rule{0.400pt}{0.723pt}}
\multiput(931.17,543.50)(1.000,-1.500){2}{\rule{0.400pt}{0.361pt}}
\put(933,540.67){\rule{0.241pt}{0.400pt}}
\multiput(933.00,541.17)(0.500,-1.000){2}{\rule{0.120pt}{0.400pt}}
\put(932.0,545.0){\rule[-0.200pt]{0.400pt}{1.686pt}}
\put(933.67,533){\rule{0.400pt}{0.482pt}}
\multiput(933.17,533.00)(1.000,1.000){2}{\rule{0.400pt}{0.241pt}}
\put(934.0,533.0){\rule[-0.200pt]{0.400pt}{1.927pt}}
\put(934.67,525){\rule{0.400pt}{1.204pt}}
\multiput(934.17,527.50)(1.000,-2.500){2}{\rule{0.400pt}{0.602pt}}
\put(935.67,522){\rule{0.400pt}{0.723pt}}
\multiput(935.17,523.50)(1.000,-1.500){2}{\rule{0.400pt}{0.361pt}}
\put(935.0,530.0){\rule[-0.200pt]{0.400pt}{1.204pt}}
\put(937.0,518.0){\rule[-0.200pt]{0.400pt}{0.964pt}}
\put(937.67,510){\rule{0.400pt}{1.927pt}}
\multiput(937.17,514.00)(1.000,-4.000){2}{\rule{0.400pt}{0.964pt}}
\put(937.0,518.0){\usebox{\plotpoint}}
\put(938.67,503){\rule{0.400pt}{1.204pt}}
\multiput(938.17,505.50)(1.000,-2.500){2}{\rule{0.400pt}{0.602pt}}
\put(939.0,508.0){\rule[-0.200pt]{0.400pt}{0.482pt}}
\put(939.67,501){\rule{0.400pt}{0.723pt}}
\multiput(939.17,502.50)(1.000,-1.500){2}{\rule{0.400pt}{0.361pt}}
\put(940.67,493){\rule{0.400pt}{1.927pt}}
\multiput(940.17,497.00)(1.000,-4.000){2}{\rule{0.400pt}{0.964pt}}
\put(940.0,503.0){\usebox{\plotpoint}}
\put(941.67,485){\rule{0.400pt}{1.204pt}}
\multiput(941.17,487.50)(1.000,-2.500){2}{\rule{0.400pt}{0.602pt}}
\put(942.67,483){\rule{0.400pt}{0.482pt}}
\multiput(942.17,484.00)(1.000,-1.000){2}{\rule{0.400pt}{0.241pt}}
\put(942.0,490.0){\rule[-0.200pt]{0.400pt}{0.723pt}}
\put(943.67,475){\rule{0.400pt}{0.964pt}}
\multiput(943.17,477.00)(1.000,-2.000){2}{\rule{0.400pt}{0.482pt}}
\put(944.0,479.0){\rule[-0.200pt]{0.400pt}{0.964pt}}
\put(944.67,468){\rule{0.400pt}{1.204pt}}
\multiput(944.17,470.50)(1.000,-2.500){2}{\rule{0.400pt}{0.602pt}}
\put(945.67,465){\rule{0.400pt}{0.723pt}}
\multiput(945.17,466.50)(1.000,-1.500){2}{\rule{0.400pt}{0.361pt}}
\put(945.0,473.0){\rule[-0.200pt]{0.400pt}{0.482pt}}
\put(946.67,458){\rule{0.400pt}{0.482pt}}
\multiput(946.17,459.00)(1.000,-1.000){2}{\rule{0.400pt}{0.241pt}}
\put(947.67,453){\rule{0.400pt}{1.204pt}}
\multiput(947.17,455.50)(1.000,-2.500){2}{\rule{0.400pt}{0.602pt}}
\put(947.0,460.0){\rule[-0.200pt]{0.400pt}{1.204pt}}
\put(948.67,447){\rule{0.400pt}{0.482pt}}
\multiput(948.17,448.00)(1.000,-1.000){2}{\rule{0.400pt}{0.241pt}}
\put(949.0,449.0){\rule[-0.200pt]{0.400pt}{0.964pt}}
\put(949.67,441){\rule{0.400pt}{0.482pt}}
\multiput(949.17,442.00)(1.000,-1.000){2}{\rule{0.400pt}{0.241pt}}
\put(950.67,437){\rule{0.400pt}{0.964pt}}
\multiput(950.17,439.00)(1.000,-2.000){2}{\rule{0.400pt}{0.482pt}}
\put(950.0,443.0){\rule[-0.200pt]{0.400pt}{0.964pt}}
\put(951.67,428){\rule{0.400pt}{1.445pt}}
\multiput(951.17,431.00)(1.000,-3.000){2}{\rule{0.400pt}{0.723pt}}
\put(952.67,425){\rule{0.400pt}{0.723pt}}
\multiput(952.17,426.50)(1.000,-1.500){2}{\rule{0.400pt}{0.361pt}}
\put(952.0,434.0){\rule[-0.200pt]{0.400pt}{0.723pt}}
\put(953.67,418){\rule{0.400pt}{0.964pt}}
\multiput(953.17,420.00)(1.000,-2.000){2}{\rule{0.400pt}{0.482pt}}
\put(954.0,422.0){\rule[-0.200pt]{0.400pt}{0.723pt}}
\put(954.67,409){\rule{0.400pt}{1.204pt}}
\multiput(954.17,411.50)(1.000,-2.500){2}{\rule{0.400pt}{0.602pt}}
\put(955.67,407){\rule{0.400pt}{0.482pt}}
\multiput(955.17,408.00)(1.000,-1.000){2}{\rule{0.400pt}{0.241pt}}
\put(955.0,414.0){\rule[-0.200pt]{0.400pt}{0.964pt}}
\put(956.67,400){\rule{0.400pt}{0.723pt}}
\multiput(956.17,401.50)(1.000,-1.500){2}{\rule{0.400pt}{0.361pt}}
\put(957.67,394){\rule{0.400pt}{1.445pt}}
\multiput(957.17,397.00)(1.000,-3.000){2}{\rule{0.400pt}{0.723pt}}
\put(957.0,403.0){\rule[-0.200pt]{0.400pt}{0.964pt}}
\put(958.67,389){\rule{0.400pt}{0.964pt}}
\multiput(958.17,391.00)(1.000,-2.000){2}{\rule{0.400pt}{0.482pt}}
\put(959.0,393.0){\usebox{\plotpoint}}
\put(959.67,382){\rule{0.400pt}{0.964pt}}
\multiput(959.17,384.00)(1.000,-2.000){2}{\rule{0.400pt}{0.482pt}}
\put(960.67,379){\rule{0.400pt}{0.723pt}}
\multiput(960.17,380.50)(1.000,-1.500){2}{\rule{0.400pt}{0.361pt}}
\put(960.0,386.0){\rule[-0.200pt]{0.400pt}{0.723pt}}
\put(961.67,372){\rule{0.400pt}{0.723pt}}
\multiput(961.17,373.50)(1.000,-1.500){2}{\rule{0.400pt}{0.361pt}}
\put(962.67,369){\rule{0.400pt}{0.723pt}}
\multiput(962.17,370.50)(1.000,-1.500){2}{\rule{0.400pt}{0.361pt}}
\put(962.0,375.0){\rule[-0.200pt]{0.400pt}{0.964pt}}
\put(963.67,363){\rule{0.400pt}{0.482pt}}
\multiput(963.17,364.00)(1.000,-1.000){2}{\rule{0.400pt}{0.241pt}}
\put(964.0,365.0){\rule[-0.200pt]{0.400pt}{0.964pt}}
\put(964.67,355){\rule{0.400pt}{0.482pt}}
\multiput(964.17,356.00)(1.000,-1.000){2}{\rule{0.400pt}{0.241pt}}
\put(965.67,351){\rule{0.400pt}{0.964pt}}
\multiput(965.17,353.00)(1.000,-2.000){2}{\rule{0.400pt}{0.482pt}}
\put(965.0,357.0){\rule[-0.200pt]{0.400pt}{1.445pt}}
\put(967,344.67){\rule{0.241pt}{0.400pt}}
\multiput(967.00,345.17)(0.500,-1.000){2}{\rule{0.120pt}{0.400pt}}
\put(967.0,346.0){\rule[-0.200pt]{0.400pt}{1.204pt}}
\put(967.67,337){\rule{0.400pt}{0.964pt}}
\multiput(967.17,339.00)(1.000,-2.000){2}{\rule{0.400pt}{0.482pt}}
\put(969,335.67){\rule{0.241pt}{0.400pt}}
\multiput(969.00,336.17)(0.500,-1.000){2}{\rule{0.120pt}{0.400pt}}
\put(968.0,341.0){\rule[-0.200pt]{0.400pt}{0.964pt}}
\put(969.67,330){\rule{0.400pt}{0.482pt}}
\multiput(969.17,331.00)(1.000,-1.000){2}{\rule{0.400pt}{0.241pt}}
\put(970.67,326){\rule{0.400pt}{0.964pt}}
\multiput(970.17,328.00)(1.000,-2.000){2}{\rule{0.400pt}{0.482pt}}
\put(970.0,332.0){\rule[-0.200pt]{0.400pt}{0.964pt}}
\put(971.67,321){\rule{0.400pt}{0.723pt}}
\multiput(971.17,322.50)(1.000,-1.500){2}{\rule{0.400pt}{0.361pt}}
\put(972.0,324.0){\rule[-0.200pt]{0.400pt}{0.482pt}}
\put(972.67,315){\rule{0.400pt}{0.482pt}}
\multiput(972.17,316.00)(1.000,-1.000){2}{\rule{0.400pt}{0.241pt}}
\put(973.67,311){\rule{0.400pt}{0.964pt}}
\multiput(973.17,313.00)(1.000,-2.000){2}{\rule{0.400pt}{0.482pt}}
\put(973.0,317.0){\rule[-0.200pt]{0.400pt}{0.964pt}}
\put(974.67,305){\rule{0.400pt}{0.964pt}}
\multiput(974.17,307.00)(1.000,-2.000){2}{\rule{0.400pt}{0.482pt}}
\put(975.67,303){\rule{0.400pt}{0.482pt}}
\multiput(975.17,304.00)(1.000,-1.000){2}{\rule{0.400pt}{0.241pt}}
\put(975.0,309.0){\rule[-0.200pt]{0.400pt}{0.482pt}}
\put(976.67,297){\rule{0.400pt}{0.964pt}}
\multiput(976.17,299.00)(1.000,-2.000){2}{\rule{0.400pt}{0.482pt}}
\put(977.0,301.0){\rule[-0.200pt]{0.400pt}{0.482pt}}
\put(977.67,291){\rule{0.400pt}{0.964pt}}
\multiput(977.17,293.00)(1.000,-2.000){2}{\rule{0.400pt}{0.482pt}}
\put(978.67,289){\rule{0.400pt}{0.482pt}}
\multiput(978.17,290.00)(1.000,-1.000){2}{\rule{0.400pt}{0.241pt}}
\put(978.0,295.0){\rule[-0.200pt]{0.400pt}{0.482pt}}
\put(979.67,284){\rule{0.400pt}{0.482pt}}
\multiput(979.17,285.00)(1.000,-1.000){2}{\rule{0.400pt}{0.241pt}}
\put(980.67,281){\rule{0.400pt}{0.723pt}}
\multiput(980.17,282.50)(1.000,-1.500){2}{\rule{0.400pt}{0.361pt}}
\put(980.0,286.0){\rule[-0.200pt]{0.400pt}{0.723pt}}
\put(982,276.67){\rule{0.241pt}{0.400pt}}
\multiput(982.00,277.17)(0.500,-1.000){2}{\rule{0.120pt}{0.400pt}}
\put(982.0,278.0){\rule[-0.200pt]{0.400pt}{0.723pt}}
\put(982.67,271){\rule{0.400pt}{0.723pt}}
\multiput(982.17,272.50)(1.000,-1.500){2}{\rule{0.400pt}{0.361pt}}
\put(983.67,269){\rule{0.400pt}{0.482pt}}
\multiput(983.17,270.00)(1.000,-1.000){2}{\rule{0.400pt}{0.241pt}}
\put(983.0,274.0){\rule[-0.200pt]{0.400pt}{0.723pt}}
\put(984.67,264){\rule{0.400pt}{0.723pt}}
\multiput(984.17,265.50)(1.000,-1.500){2}{\rule{0.400pt}{0.361pt}}
\put(986,262.67){\rule{0.241pt}{0.400pt}}
\multiput(986.00,263.17)(0.500,-1.000){2}{\rule{0.120pt}{0.400pt}}
\put(985.0,267.0){\rule[-0.200pt]{0.400pt}{0.482pt}}
\put(986.67,258){\rule{0.400pt}{0.482pt}}
\multiput(986.17,259.00)(1.000,-1.000){2}{\rule{0.400pt}{0.241pt}}
\put(987.0,260.0){\rule[-0.200pt]{0.400pt}{0.723pt}}
\put(987.67,253){\rule{0.400pt}{0.482pt}}
\multiput(987.17,254.00)(1.000,-1.000){2}{\rule{0.400pt}{0.241pt}}
\put(988.67,251){\rule{0.400pt}{0.482pt}}
\multiput(988.17,252.00)(1.000,-1.000){2}{\rule{0.400pt}{0.241pt}}
\put(988.0,255.0){\rule[-0.200pt]{0.400pt}{0.723pt}}
\put(990,246.67){\rule{0.241pt}{0.400pt}}
\multiput(990.00,247.17)(0.500,-1.000){2}{\rule{0.120pt}{0.400pt}}
\put(990.0,248.0){\rule[-0.200pt]{0.400pt}{0.723pt}}
\put(991,242.67){\rule{0.241pt}{0.400pt}}
\multiput(991.00,243.17)(0.500,-1.000){2}{\rule{0.120pt}{0.400pt}}
\put(991.67,240){\rule{0.400pt}{0.723pt}}
\multiput(991.17,241.50)(1.000,-1.500){2}{\rule{0.400pt}{0.361pt}}
\put(991.0,244.0){\rule[-0.200pt]{0.400pt}{0.723pt}}
\put(993,240){\usebox{\plotpoint}}
\put(992.67,237){\rule{0.400pt}{0.723pt}}
\multiput(992.17,238.50)(1.000,-1.500){2}{\rule{0.400pt}{0.361pt}}
\put(993.67,234){\rule{0.400pt}{0.723pt}}
\multiput(993.17,235.50)(1.000,-1.500){2}{\rule{0.400pt}{0.361pt}}
\put(994.67,231){\rule{0.400pt}{0.482pt}}
\multiput(994.17,232.00)(1.000,-1.000){2}{\rule{0.400pt}{0.241pt}}
\put(995.0,233.0){\usebox{\plotpoint}}
\put(996,227.67){\rule{0.241pt}{0.400pt}}
\multiput(996.00,228.17)(0.500,-1.000){2}{\rule{0.120pt}{0.400pt}}
\put(996.67,225){\rule{0.400pt}{0.723pt}}
\multiput(996.17,226.50)(1.000,-1.500){2}{\rule{0.400pt}{0.361pt}}
\put(996.0,229.0){\rule[-0.200pt]{0.400pt}{0.482pt}}
\put(998.0,223.0){\rule[-0.200pt]{0.400pt}{0.482pt}}
\put(998.67,220){\rule{0.400pt}{0.723pt}}
\multiput(998.17,221.50)(1.000,-1.500){2}{\rule{0.400pt}{0.361pt}}
\put(998.0,223.0){\usebox{\plotpoint}}
\put(999.67,217){\rule{0.400pt}{0.482pt}}
\multiput(999.17,218.00)(1.000,-1.000){2}{\rule{0.400pt}{0.241pt}}
\put(1000.0,219.0){\usebox{\plotpoint}}
\put(1001.0,215.0){\rule[-0.200pt]{0.400pt}{0.482pt}}
\put(1001.67,213){\rule{0.400pt}{0.482pt}}
\multiput(1001.17,214.00)(1.000,-1.000){2}{\rule{0.400pt}{0.241pt}}
\put(1001.0,215.0){\usebox{\plotpoint}}
\put(1003.0,210.0){\rule[-0.200pt]{0.400pt}{0.723pt}}
\put(1003.0,210.0){\usebox{\plotpoint}}
\put(1004,206.67){\rule{0.241pt}{0.400pt}}
\multiput(1004.00,207.17)(0.500,-1.000){2}{\rule{0.120pt}{0.400pt}}
\put(1004.67,205){\rule{0.400pt}{0.482pt}}
\multiput(1004.17,206.00)(1.000,-1.000){2}{\rule{0.400pt}{0.241pt}}
\put(1004.0,208.0){\rule[-0.200pt]{0.400pt}{0.482pt}}
\put(1005.67,202){\rule{0.400pt}{0.482pt}}
\multiput(1005.17,203.00)(1.000,-1.000){2}{\rule{0.400pt}{0.241pt}}
\put(1007,200.67){\rule{0.241pt}{0.400pt}}
\multiput(1007.00,201.17)(0.500,-1.000){2}{\rule{0.120pt}{0.400pt}}
\put(1006.0,204.0){\usebox{\plotpoint}}
\put(1008.0,199.0){\rule[-0.200pt]{0.400pt}{0.482pt}}
\put(1008.0,199.0){\usebox{\plotpoint}}
\put(1009,194.67){\rule{0.241pt}{0.400pt}}
\multiput(1009.00,194.17)(0.500,1.000){2}{\rule{0.120pt}{0.400pt}}
\put(1009.67,194){\rule{0.400pt}{0.482pt}}
\multiput(1009.17,195.00)(1.000,-1.000){2}{\rule{0.400pt}{0.241pt}}
\put(1009.0,195.0){\rule[-0.200pt]{0.400pt}{0.964pt}}
\put(1011,191.67){\rule{0.241pt}{0.400pt}}
\multiput(1011.00,192.17)(0.500,-1.000){2}{\rule{0.120pt}{0.400pt}}
\put(1011.67,190){\rule{0.400pt}{0.482pt}}
\multiput(1011.17,191.00)(1.000,-1.000){2}{\rule{0.400pt}{0.241pt}}
\put(1011.0,193.0){\usebox{\plotpoint}}
\put(1013,187.67){\rule{0.241pt}{0.400pt}}
\multiput(1013.00,188.17)(0.500,-1.000){2}{\rule{0.120pt}{0.400pt}}
\put(1013.0,189.0){\usebox{\plotpoint}}
\put(1014,185.67){\rule{0.241pt}{0.400pt}}
\multiput(1014.00,186.17)(0.500,-1.000){2}{\rule{0.120pt}{0.400pt}}
\put(1015,184.67){\rule{0.241pt}{0.400pt}}
\multiput(1015.00,185.17)(0.500,-1.000){2}{\rule{0.120pt}{0.400pt}}
\put(1014.0,187.0){\usebox{\plotpoint}}
\put(1016.0,183.0){\rule[-0.200pt]{0.400pt}{0.482pt}}
\put(1016.0,183.0){\usebox{\plotpoint}}
\put(1016.67,180){\rule{0.400pt}{0.482pt}}
\multiput(1016.17,181.00)(1.000,-1.000){2}{\rule{0.400pt}{0.241pt}}
\put(1018,178.67){\rule{0.241pt}{0.400pt}}
\multiput(1018.00,179.17)(0.500,-1.000){2}{\rule{0.120pt}{0.400pt}}
\put(1017.0,182.0){\usebox{\plotpoint}}
\put(1019,179){\usebox{\plotpoint}}
\put(1019,177.67){\rule{0.241pt}{0.400pt}}
\multiput(1019.00,178.17)(0.500,-1.000){2}{\rule{0.120pt}{0.400pt}}
\put(1020,176.67){\rule{0.241pt}{0.400pt}}
\multiput(1020.00,177.17)(0.500,-1.000){2}{\rule{0.120pt}{0.400pt}}
\put(1021.0,176.0){\usebox{\plotpoint}}
\put(1021.0,176.0){\usebox{\plotpoint}}
\put(1022,172.67){\rule{0.241pt}{0.400pt}}
\multiput(1022.00,173.17)(0.500,-1.000){2}{\rule{0.120pt}{0.400pt}}
\put(1023,171.67){\rule{0.241pt}{0.400pt}}
\multiput(1023.00,172.17)(0.500,-1.000){2}{\rule{0.120pt}{0.400pt}}
\put(1022.0,174.0){\rule[-0.200pt]{0.400pt}{0.482pt}}
\put(1024,172){\usebox{\plotpoint}}
\put(1023.67,170){\rule{0.400pt}{0.482pt}}
\multiput(1023.17,171.00)(1.000,-1.000){2}{\rule{0.400pt}{0.241pt}}
\put(1025.0,170.0){\usebox{\plotpoint}}
\put(1026,167.67){\rule{0.241pt}{0.400pt}}
\multiput(1026.00,168.17)(0.500,-1.000){2}{\rule{0.120pt}{0.400pt}}
\put(1026.0,169.0){\usebox{\plotpoint}}
\put(1027.0,167.0){\usebox{\plotpoint}}
\put(1028,165.67){\rule{0.241pt}{0.400pt}}
\multiput(1028.00,166.17)(0.500,-1.000){2}{\rule{0.120pt}{0.400pt}}
\put(1027.0,167.0){\usebox{\plotpoint}}
\put(1029.0,164.0){\rule[-0.200pt]{0.400pt}{0.482pt}}
\put(1029.0,164.0){\usebox{\plotpoint}}
\put(1030.0,163.0){\usebox{\plotpoint}}
\put(1031,161.67){\rule{0.241pt}{0.400pt}}
\multiput(1031.00,162.17)(0.500,-1.000){2}{\rule{0.120pt}{0.400pt}}
\put(1030.0,163.0){\usebox{\plotpoint}}
\put(1032,159.67){\rule{0.241pt}{0.400pt}}
\multiput(1032.00,160.17)(0.500,-1.000){2}{\rule{0.120pt}{0.400pt}}
\put(1032.0,161.0){\usebox{\plotpoint}}
\put(1033.0,160.0){\usebox{\plotpoint}}
\put(1034.0,159.0){\usebox{\plotpoint}}
\put(1034.0,159.0){\usebox{\plotpoint}}
\put(1035,156.67){\rule{0.241pt}{0.400pt}}
\multiput(1035.00,157.17)(0.500,-1.000){2}{\rule{0.120pt}{0.400pt}}
\put(1035.0,158.0){\usebox{\plotpoint}}
\put(1037,155.67){\rule{0.241pt}{0.400pt}}
\multiput(1037.00,156.17)(0.500,-1.000){2}{\rule{0.120pt}{0.400pt}}
\put(1038,154.67){\rule{0.241pt}{0.400pt}}
\multiput(1038.00,155.17)(0.500,-1.000){2}{\rule{0.120pt}{0.400pt}}
\put(1036.0,157.0){\usebox{\plotpoint}}
\put(1039,155){\usebox{\plotpoint}}
\put(1039.0,155.0){\usebox{\plotpoint}}
\put(1040,152.67){\rule{0.241pt}{0.400pt}}
\multiput(1040.00,153.17)(0.500,-1.000){2}{\rule{0.120pt}{0.400pt}}
\put(1041,151.67){\rule{0.241pt}{0.400pt}}
\multiput(1041.00,152.17)(0.500,-1.000){2}{\rule{0.120pt}{0.400pt}}
\put(1040.0,154.0){\usebox{\plotpoint}}
\put(1042,151.67){\rule{0.241pt}{0.400pt}}
\multiput(1042.00,152.17)(0.500,-1.000){2}{\rule{0.120pt}{0.400pt}}
\put(1042.0,152.0){\usebox{\plotpoint}}
\put(1043.0,151.0){\usebox{\plotpoint}}
\put(1043.0,151.0){\rule[-0.200pt]{0.482pt}{0.400pt}}
\put(1045,148.67){\rule{0.241pt}{0.400pt}}
\multiput(1045.00,149.17)(0.500,-1.000){2}{\rule{0.120pt}{0.400pt}}
\put(1046,148.67){\rule{0.241pt}{0.400pt}}
\multiput(1046.00,148.17)(0.500,1.000){2}{\rule{0.120pt}{0.400pt}}
\put(1045.0,150.0){\usebox{\plotpoint}}
\put(1047,147.67){\rule{0.241pt}{0.400pt}}
\multiput(1047.00,148.17)(0.500,-1.000){2}{\rule{0.120pt}{0.400pt}}
\put(1047.0,149.0){\usebox{\plotpoint}}
\put(1048,148){\usebox{\plotpoint}}
\put(1048,146.67){\rule{0.241pt}{0.400pt}}
\multiput(1048.00,147.17)(0.500,-1.000){2}{\rule{0.120pt}{0.400pt}}
\put(1049.0,147.0){\usebox{\plotpoint}}
\put(1050.0,146.0){\usebox{\plotpoint}}
\put(1050.0,146.0){\usebox{\plotpoint}}
\put(1051.0,145.0){\usebox{\plotpoint}}
\put(1053,143.67){\rule{0.241pt}{0.400pt}}
\multiput(1053.00,144.17)(0.500,-1.000){2}{\rule{0.120pt}{0.400pt}}
\put(1054,142.67){\rule{0.241pt}{0.400pt}}
\multiput(1054.00,143.17)(0.500,-1.000){2}{\rule{0.120pt}{0.400pt}}
\put(1051.0,145.0){\rule[-0.200pt]{0.482pt}{0.400pt}}
\put(1055.0,143.0){\usebox{\plotpoint}}
\put(1055.0,144.0){\usebox{\plotpoint}}
\put(1056,141.67){\rule{0.241pt}{0.400pt}}
\multiput(1056.00,142.17)(0.500,-1.000){2}{\rule{0.120pt}{0.400pt}}
\put(1057,141.67){\rule{0.241pt}{0.400pt}}
\multiput(1057.00,141.17)(0.500,1.000){2}{\rule{0.120pt}{0.400pt}}
\put(1056.0,143.0){\usebox{\plotpoint}}
\put(1058,140.67){\rule{0.241pt}{0.400pt}}
\multiput(1058.00,141.17)(0.500,-1.000){2}{\rule{0.120pt}{0.400pt}}
\put(1058.0,142.0){\usebox{\plotpoint}}
\put(1059,141){\usebox{\plotpoint}}
\put(1061,139.67){\rule{0.241pt}{0.400pt}}
\multiput(1061.00,140.17)(0.500,-1.000){2}{\rule{0.120pt}{0.400pt}}
\put(1059.0,141.0){\rule[-0.200pt]{0.482pt}{0.400pt}}
\put(1062.0,140.0){\usebox{\plotpoint}}
\put(1063.0,139.0){\usebox{\plotpoint}}
\put(1063.0,139.0){\rule[-0.200pt]{0.964pt}{0.400pt}}
\put(1067,136.67){\rule{0.241pt}{0.400pt}}
\multiput(1067.00,137.17)(0.500,-1.000){2}{\rule{0.120pt}{0.400pt}}
\put(1067.0,138.0){\usebox{\plotpoint}}
\put(1068.0,137.0){\rule[-0.200pt]{0.723pt}{0.400pt}}
\put(1070.67,136){\rule{0.400pt}{0.482pt}}
\multiput(1070.17,137.00)(1.000,-1.000){2}{\rule{0.400pt}{0.241pt}}
\put(1071.0,137.0){\usebox{\plotpoint}}
\put(1072,136){\usebox{\plotpoint}}
\put(1072,135.67){\rule{0.241pt}{0.400pt}}
\multiput(1072.00,135.17)(0.500,1.000){2}{\rule{0.120pt}{0.400pt}}
\put(1073,135.67){\rule{0.241pt}{0.400pt}}
\multiput(1073.00,136.17)(0.500,-1.000){2}{\rule{0.120pt}{0.400pt}}
\put(1074,134.67){\rule{0.241pt}{0.400pt}}
\multiput(1074.00,134.17)(0.500,1.000){2}{\rule{0.120pt}{0.400pt}}
\put(1074.0,135.0){\usebox{\plotpoint}}
\put(1075.0,135.0){\usebox{\plotpoint}}
\put(1079,133.67){\rule{0.241pt}{0.400pt}}
\multiput(1079.00,134.17)(0.500,-1.000){2}{\rule{0.120pt}{0.400pt}}
\put(1075.0,135.0){\rule[-0.200pt]{0.964pt}{0.400pt}}
\put(1080,134){\usebox{\plotpoint}}
\put(1080.0,134.0){\rule[-0.200pt]{0.482pt}{0.400pt}}
\put(1082.0,133.0){\usebox{\plotpoint}}
\put(1082.0,133.0){\usebox{\plotpoint}}
\put(1083,132.67){\rule{0.241pt}{0.400pt}}
\multiput(1083.00,133.17)(0.500,-1.000){2}{\rule{0.120pt}{0.400pt}}
\put(1084,132.67){\rule{0.241pt}{0.400pt}}
\multiput(1084.00,132.17)(0.500,1.000){2}{\rule{0.120pt}{0.400pt}}
\put(1083.0,133.0){\usebox{\plotpoint}}
\put(1085.0,133.0){\usebox{\plotpoint}}
\put(1085.0,133.0){\usebox{\plotpoint}}
\put(1086,132.67){\rule{0.241pt}{0.400pt}}
\multiput(1086.00,133.17)(0.500,-1.000){2}{\rule{0.120pt}{0.400pt}}
\put(1086.0,133.0){\usebox{\plotpoint}}
\put(1087.0,133.0){\usebox{\plotpoint}}
\put(1088.0,132.0){\usebox{\plotpoint}}
\put(1089,131.67){\rule{0.241pt}{0.400pt}}
\multiput(1089.00,131.17)(0.500,1.000){2}{\rule{0.120pt}{0.400pt}}
\put(1088.0,132.0){\usebox{\plotpoint}}
\put(1090.0,132.0){\usebox{\plotpoint}}
\put(1090.0,132.0){\usebox{\plotpoint}}
\put(1091,130.67){\rule{0.241pt}{0.400pt}}
\multiput(1091.00,130.17)(0.500,1.000){2}{\rule{0.120pt}{0.400pt}}
\put(1091.0,131.0){\usebox{\plotpoint}}
\put(1093,130.67){\rule{0.241pt}{0.400pt}}
\multiput(1093.00,131.17)(0.500,-1.000){2}{\rule{0.120pt}{0.400pt}}
\put(1092.0,132.0){\usebox{\plotpoint}}
\put(1094.0,131.0){\usebox{\plotpoint}}
\put(1095,130.67){\rule{0.241pt}{0.400pt}}
\multiput(1095.00,131.17)(0.500,-1.000){2}{\rule{0.120pt}{0.400pt}}
\put(1094.0,132.0){\usebox{\plotpoint}}
\put(1096,131){\usebox{\plotpoint}}
\put(1096.0,131.0){\rule[-0.200pt]{0.482pt}{0.400pt}}
\put(171,313.67){\rule{0.241pt}{0.400pt}}
\multiput(171.00,313.17)(0.500,1.000){2}{\rule{0.120pt}{0.400pt}}
\put(1098.0,131.0){\rule[-0.200pt]{0.482pt}{0.400pt}}
\put(171.67,317){\rule{0.400pt}{0.723pt}}
\multiput(171.17,317.00)(1.000,1.500){2}{\rule{0.400pt}{0.361pt}}
\put(173,319.67){\rule{0.241pt}{0.400pt}}
\multiput(173.00,319.17)(0.500,1.000){2}{\rule{0.120pt}{0.400pt}}
\put(172.0,315.0){\rule[-0.200pt]{0.400pt}{0.482pt}}
\put(174,321){\usebox{\plotpoint}}
\put(174,319.67){\rule{0.241pt}{0.400pt}}
\multiput(174.00,320.17)(0.500,-1.000){2}{\rule{0.120pt}{0.400pt}}
\put(174.67,318){\rule{0.400pt}{0.482pt}}
\multiput(174.17,319.00)(1.000,-1.000){2}{\rule{0.400pt}{0.241pt}}
\put(175.67,321){\rule{0.400pt}{0.482pt}}
\multiput(175.17,321.00)(1.000,1.000){2}{\rule{0.400pt}{0.241pt}}
\put(177,321.67){\rule{0.241pt}{0.400pt}}
\multiput(177.00,322.17)(0.500,-1.000){2}{\rule{0.120pt}{0.400pt}}
\put(176.0,318.0){\rule[-0.200pt]{0.400pt}{0.723pt}}
\put(178,322){\usebox{\plotpoint}}
\put(177.67,322){\rule{0.400pt}{0.482pt}}
\multiput(177.17,322.00)(1.000,1.000){2}{\rule{0.400pt}{0.241pt}}
\put(179,322.67){\rule{0.241pt}{0.400pt}}
\multiput(179.00,323.17)(0.500,-1.000){2}{\rule{0.120pt}{0.400pt}}
\put(179.67,324){\rule{0.400pt}{0.482pt}}
\multiput(179.17,325.00)(1.000,-1.000){2}{\rule{0.400pt}{0.241pt}}
\put(180.67,324){\rule{0.400pt}{0.482pt}}
\multiput(180.17,324.00)(1.000,1.000){2}{\rule{0.400pt}{0.241pt}}
\put(180.0,323.0){\rule[-0.200pt]{0.400pt}{0.723pt}}
\put(182,326){\usebox{\plotpoint}}
\put(181.67,324){\rule{0.400pt}{0.482pt}}
\multiput(181.17,325.00)(1.000,-1.000){2}{\rule{0.400pt}{0.241pt}}
\put(182.67,324){\rule{0.400pt}{0.964pt}}
\multiput(182.17,324.00)(1.000,2.000){2}{\rule{0.400pt}{0.482pt}}
\put(184,328){\usebox{\plotpoint}}
\put(184,326.67){\rule{0.241pt}{0.400pt}}
\multiput(184.00,327.17)(0.500,-1.000){2}{\rule{0.120pt}{0.400pt}}
\put(185.0,327.0){\usebox{\plotpoint}}
\put(186,326.67){\rule{0.241pt}{0.400pt}}
\multiput(186.00,327.17)(0.500,-1.000){2}{\rule{0.120pt}{0.400pt}}
\put(186.67,327){\rule{0.400pt}{0.964pt}}
\multiput(186.17,327.00)(1.000,2.000){2}{\rule{0.400pt}{0.482pt}}
\put(186.0,327.0){\usebox{\plotpoint}}
\put(188,328.67){\rule{0.241pt}{0.400pt}}
\multiput(188.00,329.17)(0.500,-1.000){2}{\rule{0.120pt}{0.400pt}}
\put(188.67,329){\rule{0.400pt}{1.204pt}}
\multiput(188.17,329.00)(1.000,2.500){2}{\rule{0.400pt}{0.602pt}}
\put(189.67,331){\rule{0.400pt}{0.723pt}}
\multiput(189.17,332.50)(1.000,-1.500){2}{\rule{0.400pt}{0.361pt}}
\put(188.0,330.0){\usebox{\plotpoint}}
\put(191,328.67){\rule{0.241pt}{0.400pt}}
\multiput(191.00,329.17)(0.500,-1.000){2}{\rule{0.120pt}{0.400pt}}
\put(191.67,329){\rule{0.400pt}{0.723pt}}
\multiput(191.17,329.00)(1.000,1.500){2}{\rule{0.400pt}{0.361pt}}
\put(191.0,330.0){\usebox{\plotpoint}}
\put(193.0,332.0){\usebox{\plotpoint}}
\put(193.0,333.0){\rule[-0.200pt]{0.482pt}{0.400pt}}
\put(194.67,332){\rule{0.400pt}{0.482pt}}
\multiput(194.17,333.00)(1.000,-1.000){2}{\rule{0.400pt}{0.241pt}}
\put(195.67,332){\rule{0.400pt}{0.482pt}}
\multiput(195.17,332.00)(1.000,1.000){2}{\rule{0.400pt}{0.241pt}}
\put(195.0,333.0){\usebox{\plotpoint}}
\put(197,333.67){\rule{0.241pt}{0.400pt}}
\multiput(197.00,334.17)(0.500,-1.000){2}{\rule{0.120pt}{0.400pt}}
\put(197.67,332){\rule{0.400pt}{0.482pt}}
\multiput(197.17,333.00)(1.000,-1.000){2}{\rule{0.400pt}{0.241pt}}
\put(197.0,334.0){\usebox{\plotpoint}}
\put(199,333.67){\rule{0.241pt}{0.400pt}}
\multiput(199.00,334.17)(0.500,-1.000){2}{\rule{0.120pt}{0.400pt}}
\put(199.0,332.0){\rule[-0.200pt]{0.400pt}{0.723pt}}
\put(200.0,334.0){\usebox{\plotpoint}}
\put(201,334.67){\rule{0.241pt}{0.400pt}}
\multiput(201.00,334.17)(0.500,1.000){2}{\rule{0.120pt}{0.400pt}}
\put(201.67,334){\rule{0.400pt}{0.482pt}}
\multiput(201.17,335.00)(1.000,-1.000){2}{\rule{0.400pt}{0.241pt}}
\put(201.0,334.0){\usebox{\plotpoint}}
\put(203,334){\usebox{\plotpoint}}
\put(203,332.67){\rule{0.241pt}{0.400pt}}
\multiput(203.00,333.17)(0.500,-1.000){2}{\rule{0.120pt}{0.400pt}}
\put(203.67,333){\rule{0.400pt}{0.482pt}}
\multiput(203.17,333.00)(1.000,1.000){2}{\rule{0.400pt}{0.241pt}}
\put(205.0,334.0){\usebox{\plotpoint}}
\put(208,333.67){\rule{0.241pt}{0.400pt}}
\multiput(208.00,333.17)(0.500,1.000){2}{\rule{0.120pt}{0.400pt}}
\put(205.0,334.0){\rule[-0.200pt]{0.723pt}{0.400pt}}
\put(209,331.67){\rule{0.241pt}{0.400pt}}
\multiput(209.00,331.17)(0.500,1.000){2}{\rule{0.120pt}{0.400pt}}
\put(209.0,332.0){\rule[-0.200pt]{0.400pt}{0.723pt}}
\put(210.0,333.0){\usebox{\plotpoint}}
\put(210.67,332){\rule{0.400pt}{0.482pt}}
\multiput(210.17,333.00)(1.000,-1.000){2}{\rule{0.400pt}{0.241pt}}
\put(212,330.67){\rule{0.241pt}{0.400pt}}
\multiput(212.00,331.17)(0.500,-1.000){2}{\rule{0.120pt}{0.400pt}}
\put(211.0,333.0){\usebox{\plotpoint}}
\put(213,331){\usebox{\plotpoint}}
\put(213,330.67){\rule{0.241pt}{0.400pt}}
\multiput(213.00,330.17)(0.500,1.000){2}{\rule{0.120pt}{0.400pt}}
\put(214,330.67){\rule{0.241pt}{0.400pt}}
\multiput(214.00,331.17)(0.500,-1.000){2}{\rule{0.120pt}{0.400pt}}
\put(215,331){\usebox{\plotpoint}}
\put(215,329.67){\rule{0.241pt}{0.400pt}}
\multiput(215.00,330.17)(0.500,-1.000){2}{\rule{0.120pt}{0.400pt}}
\put(216.0,330.0){\rule[-0.200pt]{0.723pt}{0.400pt}}
\put(219,328.67){\rule{0.241pt}{0.400pt}}
\multiput(219.00,328.17)(0.500,1.000){2}{\rule{0.120pt}{0.400pt}}
\put(220,328.67){\rule{0.241pt}{0.400pt}}
\multiput(220.00,329.17)(0.500,-1.000){2}{\rule{0.120pt}{0.400pt}}
\put(219.0,329.0){\usebox{\plotpoint}}
\put(220.67,327){\rule{0.400pt}{0.482pt}}
\multiput(220.17,327.00)(1.000,1.000){2}{\rule{0.400pt}{0.241pt}}
\put(221.67,327){\rule{0.400pt}{0.482pt}}
\multiput(221.17,328.00)(1.000,-1.000){2}{\rule{0.400pt}{0.241pt}}
\put(221.0,327.0){\rule[-0.200pt]{0.400pt}{0.482pt}}
\put(223,327){\usebox{\plotpoint}}
\put(224,325.67){\rule{0.241pt}{0.400pt}}
\multiput(224.00,326.17)(0.500,-1.000){2}{\rule{0.120pt}{0.400pt}}
\put(223.0,327.0){\usebox{\plotpoint}}
\put(225.0,325.0){\usebox{\plotpoint}}
\put(226,323.67){\rule{0.241pt}{0.400pt}}
\multiput(226.00,324.17)(0.500,-1.000){2}{\rule{0.120pt}{0.400pt}}
\put(225.0,325.0){\usebox{\plotpoint}}
\put(227,324){\usebox{\plotpoint}}
\put(227.0,324.0){\rule[-0.200pt]{0.482pt}{0.400pt}}
\put(229,321.67){\rule{0.241pt}{0.400pt}}
\multiput(229.00,321.17)(0.500,1.000){2}{\rule{0.120pt}{0.400pt}}
\put(230,322.67){\rule{0.241pt}{0.400pt}}
\multiput(230.00,322.17)(0.500,1.000){2}{\rule{0.120pt}{0.400pt}}
\put(229.0,322.0){\rule[-0.200pt]{0.400pt}{0.482pt}}
\put(231.0,321.0){\rule[-0.200pt]{0.400pt}{0.723pt}}
\put(231.0,321.0){\rule[-0.200pt]{0.482pt}{0.400pt}}
\put(233,318.67){\rule{0.241pt}{0.400pt}}
\multiput(233.00,319.17)(0.500,-1.000){2}{\rule{0.120pt}{0.400pt}}
\put(233.0,320.0){\usebox{\plotpoint}}
\put(235,317.67){\rule{0.241pt}{0.400pt}}
\multiput(235.00,318.17)(0.500,-1.000){2}{\rule{0.120pt}{0.400pt}}
\put(234.0,319.0){\usebox{\plotpoint}}
\put(236,318){\usebox{\plotpoint}}
\put(236,317.67){\rule{0.241pt}{0.400pt}}
\multiput(236.00,317.17)(0.500,1.000){2}{\rule{0.120pt}{0.400pt}}
\put(236.67,317){\rule{0.400pt}{0.482pt}}
\multiput(236.17,318.00)(1.000,-1.000){2}{\rule{0.400pt}{0.241pt}}
\put(238,317){\usebox{\plotpoint}}
\put(238,315.67){\rule{0.241pt}{0.400pt}}
\multiput(238.00,316.17)(0.500,-1.000){2}{\rule{0.120pt}{0.400pt}}
\put(239,314.67){\rule{0.241pt}{0.400pt}}
\multiput(239.00,315.17)(0.500,-1.000){2}{\rule{0.120pt}{0.400pt}}
\put(240,315){\usebox{\plotpoint}}
\put(240,314.67){\rule{0.241pt}{0.400pt}}
\multiput(240.00,314.17)(0.500,1.000){2}{\rule{0.120pt}{0.400pt}}
\put(241,314.67){\rule{0.241pt}{0.400pt}}
\multiput(241.00,315.17)(0.500,-1.000){2}{\rule{0.120pt}{0.400pt}}
\put(242,315){\usebox{\plotpoint}}
\put(241.67,313){\rule{0.400pt}{0.482pt}}
\multiput(241.17,314.00)(1.000,-1.000){2}{\rule{0.400pt}{0.241pt}}
\put(243,312.67){\rule{0.241pt}{0.400pt}}
\multiput(243.00,312.17)(0.500,1.000){2}{\rule{0.120pt}{0.400pt}}
\put(244.0,313.0){\usebox{\plotpoint}}
\put(244.67,311){\rule{0.400pt}{0.482pt}}
\multiput(244.17,312.00)(1.000,-1.000){2}{\rule{0.400pt}{0.241pt}}
\put(244.0,313.0){\usebox{\plotpoint}}
\put(246,311){\usebox{\plotpoint}}
\put(246,310.67){\rule{0.241pt}{0.400pt}}
\multiput(246.00,310.17)(0.500,1.000){2}{\rule{0.120pt}{0.400pt}}
\put(247,310.67){\rule{0.241pt}{0.400pt}}
\multiput(247.00,311.17)(0.500,-1.000){2}{\rule{0.120pt}{0.400pt}}
\put(248,311){\usebox{\plotpoint}}
\put(248,309.67){\rule{0.241pt}{0.400pt}}
\multiput(248.00,310.17)(0.500,-1.000){2}{\rule{0.120pt}{0.400pt}}
\put(249.0,310.0){\usebox{\plotpoint}}
\put(250.0,309.0){\usebox{\plotpoint}}
\put(251.67,307){\rule{0.400pt}{0.482pt}}
\multiput(251.17,308.00)(1.000,-1.000){2}{\rule{0.400pt}{0.241pt}}
\put(253,306.67){\rule{0.241pt}{0.400pt}}
\multiput(253.00,306.17)(0.500,1.000){2}{\rule{0.120pt}{0.400pt}}
\put(250.0,309.0){\rule[-0.200pt]{0.482pt}{0.400pt}}
\put(254.0,307.0){\usebox{\plotpoint}}
\put(255,305.67){\rule{0.241pt}{0.400pt}}
\multiput(255.00,306.17)(0.500,-1.000){2}{\rule{0.120pt}{0.400pt}}
\put(254.0,307.0){\usebox{\plotpoint}}
\put(256,305.67){\rule{0.241pt}{0.400pt}}
\multiput(256.00,306.17)(0.500,-1.000){2}{\rule{0.120pt}{0.400pt}}
\put(256.0,306.0){\usebox{\plotpoint}}
\put(258,304.67){\rule{0.241pt}{0.400pt}}
\multiput(258.00,305.17)(0.500,-1.000){2}{\rule{0.120pt}{0.400pt}}
\put(259,303.67){\rule{0.241pt}{0.400pt}}
\multiput(259.00,304.17)(0.500,-1.000){2}{\rule{0.120pt}{0.400pt}}
\put(257.0,306.0){\usebox{\plotpoint}}
\put(259.67,303){\rule{0.400pt}{0.482pt}}
\multiput(259.17,304.00)(1.000,-1.000){2}{\rule{0.400pt}{0.241pt}}
\put(260.67,303){\rule{0.400pt}{0.482pt}}
\multiput(260.17,303.00)(1.000,1.000){2}{\rule{0.400pt}{0.241pt}}
\put(260.0,304.0){\usebox{\plotpoint}}
\put(262,305){\usebox{\plotpoint}}
\put(262,303.67){\rule{0.241pt}{0.400pt}}
\multiput(262.00,304.17)(0.500,-1.000){2}{\rule{0.120pt}{0.400pt}}
\put(263,302.67){\rule{0.241pt}{0.400pt}}
\multiput(263.00,303.17)(0.500,-1.000){2}{\rule{0.120pt}{0.400pt}}
\put(264,303){\usebox{\plotpoint}}
\put(266,301.67){\rule{0.241pt}{0.400pt}}
\multiput(266.00,302.17)(0.500,-1.000){2}{\rule{0.120pt}{0.400pt}}
\put(264.0,303.0){\rule[-0.200pt]{0.482pt}{0.400pt}}
\put(267.0,302.0){\usebox{\plotpoint}}
\put(267.67,301){\rule{0.400pt}{0.482pt}}
\multiput(267.17,302.00)(1.000,-1.000){2}{\rule{0.400pt}{0.241pt}}
\put(269,300.67){\rule{0.241pt}{0.400pt}}
\multiput(269.00,300.17)(0.500,1.000){2}{\rule{0.120pt}{0.400pt}}
\put(268.0,302.0){\usebox{\plotpoint}}
\put(270,302){\usebox{\plotpoint}}
\put(270,300.67){\rule{0.241pt}{0.400pt}}
\multiput(270.00,301.17)(0.500,-1.000){2}{\rule{0.120pt}{0.400pt}}
\put(271.0,301.0){\usebox{\plotpoint}}
\put(272,299.67){\rule{0.241pt}{0.400pt}}
\multiput(272.00,299.17)(0.500,1.000){2}{\rule{0.120pt}{0.400pt}}
\put(272.67,301){\rule{0.400pt}{0.482pt}}
\multiput(272.17,301.00)(1.000,1.000){2}{\rule{0.400pt}{0.241pt}}
\put(272.0,300.0){\usebox{\plotpoint}}
\put(274,299.67){\rule{0.241pt}{0.400pt}}
\multiput(274.00,300.17)(0.500,-1.000){2}{\rule{0.120pt}{0.400pt}}
\put(274.67,300){\rule{0.400pt}{0.482pt}}
\multiput(274.17,300.00)(1.000,1.000){2}{\rule{0.400pt}{0.241pt}}
\put(274.0,301.0){\rule[-0.200pt]{0.400pt}{0.482pt}}
\put(276,299.67){\rule{0.241pt}{0.400pt}}
\multiput(276.00,299.17)(0.500,1.000){2}{\rule{0.120pt}{0.400pt}}
\put(276.0,300.0){\rule[-0.200pt]{0.400pt}{0.482pt}}
\put(277.0,301.0){\usebox{\plotpoint}}
\put(277.67,300){\rule{0.400pt}{0.482pt}}
\multiput(277.17,301.00)(1.000,-1.000){2}{\rule{0.400pt}{0.241pt}}
\put(279,299.67){\rule{0.241pt}{0.400pt}}
\multiput(279.00,299.17)(0.500,1.000){2}{\rule{0.120pt}{0.400pt}}
\put(278.0,301.0){\usebox{\plotpoint}}
\put(280,301){\usebox{\plotpoint}}
\put(280.0,301.0){\rule[-0.200pt]{0.964pt}{0.400pt}}
\put(283.67,300){\rule{0.400pt}{0.482pt}}
\multiput(283.17,300.00)(1.000,1.000){2}{\rule{0.400pt}{0.241pt}}
\put(285,300.67){\rule{0.241pt}{0.400pt}}
\multiput(285.00,301.17)(0.500,-1.000){2}{\rule{0.120pt}{0.400pt}}
\put(284.0,300.0){\usebox{\plotpoint}}
\put(286.0,301.0){\usebox{\plotpoint}}
\put(286.0,302.0){\rule[-0.200pt]{0.482pt}{0.400pt}}
\put(287.67,301){\rule{0.400pt}{0.482pt}}
\multiput(287.17,302.00)(1.000,-1.000){2}{\rule{0.400pt}{0.241pt}}
\put(288.67,301){\rule{0.400pt}{0.482pt}}
\multiput(288.17,301.00)(1.000,1.000){2}{\rule{0.400pt}{0.241pt}}
\put(288.0,302.0){\usebox{\plotpoint}}
\put(289.67,302){\rule{0.400pt}{0.482pt}}
\multiput(289.17,303.00)(1.000,-1.000){2}{\rule{0.400pt}{0.241pt}}
\put(290.67,302){\rule{0.400pt}{0.482pt}}
\multiput(290.17,302.00)(1.000,1.000){2}{\rule{0.400pt}{0.241pt}}
\put(290.0,303.0){\usebox{\plotpoint}}
\put(292,304){\usebox{\plotpoint}}
\put(292.0,304.0){\rule[-0.200pt]{0.482pt}{0.400pt}}
\put(294.0,304.0){\usebox{\plotpoint}}
\put(294.67,305){\rule{0.400pt}{0.482pt}}
\multiput(294.17,305.00)(1.000,1.000){2}{\rule{0.400pt}{0.241pt}}
\put(294.0,305.0){\usebox{\plotpoint}}
\put(295.67,305){\rule{0.400pt}{0.482pt}}
\multiput(295.17,305.00)(1.000,1.000){2}{\rule{0.400pt}{0.241pt}}
\put(297,306.67){\rule{0.241pt}{0.400pt}}
\multiput(297.00,306.17)(0.500,1.000){2}{\rule{0.120pt}{0.400pt}}
\put(296.0,305.0){\rule[-0.200pt]{0.400pt}{0.482pt}}
\put(298.0,307.0){\usebox{\plotpoint}}
\put(298.0,307.0){\rule[-0.200pt]{0.482pt}{0.400pt}}
\put(300,307.67){\rule{0.241pt}{0.400pt}}
\multiput(300.00,308.17)(0.500,-1.000){2}{\rule{0.120pt}{0.400pt}}
\put(300.67,308){\rule{0.400pt}{0.723pt}}
\multiput(300.17,308.00)(1.000,1.500){2}{\rule{0.400pt}{0.361pt}}
\put(300.0,307.0){\rule[-0.200pt]{0.400pt}{0.482pt}}
\put(302,311){\usebox{\plotpoint}}
\put(301.67,309){\rule{0.400pt}{0.482pt}}
\multiput(301.17,310.00)(1.000,-1.000){2}{\rule{0.400pt}{0.241pt}}
\put(302.67,309){\rule{0.400pt}{0.482pt}}
\multiput(302.17,309.00)(1.000,1.000){2}{\rule{0.400pt}{0.241pt}}
\put(304,311.67){\rule{0.241pt}{0.400pt}}
\multiput(304.00,311.17)(0.500,1.000){2}{\rule{0.120pt}{0.400pt}}
\put(305,312.67){\rule{0.241pt}{0.400pt}}
\multiput(305.00,312.17)(0.500,1.000){2}{\rule{0.120pt}{0.400pt}}
\put(304.0,311.0){\usebox{\plotpoint}}
\put(306,312.67){\rule{0.241pt}{0.400pt}}
\multiput(306.00,312.17)(0.500,1.000){2}{\rule{0.120pt}{0.400pt}}
\put(307,313.67){\rule{0.241pt}{0.400pt}}
\multiput(307.00,313.17)(0.500,1.000){2}{\rule{0.120pt}{0.400pt}}
\put(306.0,313.0){\usebox{\plotpoint}}
\put(308,315.67){\rule{0.241pt}{0.400pt}}
\multiput(308.00,315.17)(0.500,1.000){2}{\rule{0.120pt}{0.400pt}}
\put(309,316.67){\rule{0.241pt}{0.400pt}}
\multiput(309.00,316.17)(0.500,1.000){2}{\rule{0.120pt}{0.400pt}}
\put(308.0,315.0){\usebox{\plotpoint}}
\put(309.67,317){\rule{0.400pt}{0.482pt}}
\multiput(309.17,317.00)(1.000,1.000){2}{\rule{0.400pt}{0.241pt}}
\put(310.0,317.0){\usebox{\plotpoint}}
\put(311.0,319.0){\usebox{\plotpoint}}
\put(312.0,319.0){\usebox{\plotpoint}}
\put(312.67,320){\rule{0.400pt}{0.723pt}}
\multiput(312.17,320.00)(1.000,1.500){2}{\rule{0.400pt}{0.361pt}}
\put(312.0,320.0){\usebox{\plotpoint}}
\put(313.67,322){\rule{0.400pt}{0.482pt}}
\multiput(313.17,322.00)(1.000,1.000){2}{\rule{0.400pt}{0.241pt}}
\put(314.67,324){\rule{0.400pt}{0.482pt}}
\multiput(314.17,324.00)(1.000,1.000){2}{\rule{0.400pt}{0.241pt}}
\put(314.0,322.0){\usebox{\plotpoint}}
\put(315.67,325){\rule{0.400pt}{0.482pt}}
\multiput(315.17,325.00)(1.000,1.000){2}{\rule{0.400pt}{0.241pt}}
\put(317,326.67){\rule{0.241pt}{0.400pt}}
\multiput(317.00,326.17)(0.500,1.000){2}{\rule{0.120pt}{0.400pt}}
\put(316.0,325.0){\usebox{\plotpoint}}
\put(318,328.67){\rule{0.241pt}{0.400pt}}
\multiput(318.00,328.17)(0.500,1.000){2}{\rule{0.120pt}{0.400pt}}
\put(318.0,328.0){\usebox{\plotpoint}}
\put(319.67,330){\rule{0.400pt}{0.482pt}}
\multiput(319.17,330.00)(1.000,1.000){2}{\rule{0.400pt}{0.241pt}}
\put(319.0,330.0){\usebox{\plotpoint}}
\put(321.0,332.0){\usebox{\plotpoint}}
\put(321.67,333){\rule{0.400pt}{0.482pt}}
\multiput(321.17,333.00)(1.000,1.000){2}{\rule{0.400pt}{0.241pt}}
\put(322.67,335){\rule{0.400pt}{0.482pt}}
\multiput(322.17,335.00)(1.000,1.000){2}{\rule{0.400pt}{0.241pt}}
\put(322.0,332.0){\usebox{\plotpoint}}
\put(324,337){\usebox{\plotpoint}}
\put(324.67,337){\rule{0.400pt}{0.482pt}}
\multiput(324.17,337.00)(1.000,1.000){2}{\rule{0.400pt}{0.241pt}}
\put(324.0,337.0){\usebox{\plotpoint}}
\put(326,339.67){\rule{0.241pt}{0.400pt}}
\multiput(326.00,339.17)(0.500,1.000){2}{\rule{0.120pt}{0.400pt}}
\put(326.0,339.0){\usebox{\plotpoint}}
\put(327.0,341.0){\usebox{\plotpoint}}
\put(328,343.67){\rule{0.241pt}{0.400pt}}
\multiput(328.00,343.17)(0.500,1.000){2}{\rule{0.120pt}{0.400pt}}
\put(328.0,341.0){\rule[-0.200pt]{0.400pt}{0.723pt}}
\put(329.0,345.0){\usebox{\plotpoint}}
\put(329.67,346){\rule{0.400pt}{0.482pt}}
\multiput(329.17,346.00)(1.000,1.000){2}{\rule{0.400pt}{0.241pt}}
\put(330.67,348){\rule{0.400pt}{0.482pt}}
\multiput(330.17,348.00)(1.000,1.000){2}{\rule{0.400pt}{0.241pt}}
\put(330.0,345.0){\usebox{\plotpoint}}
\put(332,350){\usebox{\plotpoint}}
\put(332,349.67){\rule{0.241pt}{0.400pt}}
\multiput(332.00,349.17)(0.500,1.000){2}{\rule{0.120pt}{0.400pt}}
\put(332.67,351){\rule{0.400pt}{0.482pt}}
\multiput(332.17,351.00)(1.000,1.000){2}{\rule{0.400pt}{0.241pt}}
\put(334,353){\usebox{\plotpoint}}
\put(333.67,353){\rule{0.400pt}{0.482pt}}
\multiput(333.17,353.00)(1.000,1.000){2}{\rule{0.400pt}{0.241pt}}
\put(335,354.67){\rule{0.241pt}{0.400pt}}
\multiput(335.00,354.17)(0.500,1.000){2}{\rule{0.120pt}{0.400pt}}
\put(335.67,357){\rule{0.400pt}{0.723pt}}
\multiput(335.17,357.00)(1.000,1.500){2}{\rule{0.400pt}{0.361pt}}
\put(336.0,356.0){\usebox{\plotpoint}}
\put(337.0,360.0){\usebox{\plotpoint}}
\put(338,361.67){\rule{0.241pt}{0.400pt}}
\multiput(338.00,361.17)(0.500,1.000){2}{\rule{0.120pt}{0.400pt}}
\put(338.0,360.0){\rule[-0.200pt]{0.400pt}{0.482pt}}
\put(339.0,363.0){\usebox{\plotpoint}}
\put(340,364.67){\rule{0.241pt}{0.400pt}}
\multiput(340.00,364.17)(0.500,1.000){2}{\rule{0.120pt}{0.400pt}}
\put(341,365.67){\rule{0.241pt}{0.400pt}}
\multiput(341.00,365.17)(0.500,1.000){2}{\rule{0.120pt}{0.400pt}}
\put(340.0,363.0){\rule[-0.200pt]{0.400pt}{0.482pt}}
\put(341.67,368){\rule{0.400pt}{0.723pt}}
\multiput(341.17,368.00)(1.000,1.500){2}{\rule{0.400pt}{0.361pt}}
\put(343,370.67){\rule{0.241pt}{0.400pt}}
\multiput(343.00,370.17)(0.500,1.000){2}{\rule{0.120pt}{0.400pt}}
\put(342.0,367.0){\usebox{\plotpoint}}
\put(343.67,373){\rule{0.400pt}{0.723pt}}
\multiput(343.17,373.00)(1.000,1.500){2}{\rule{0.400pt}{0.361pt}}
\put(345,374.67){\rule{0.241pt}{0.400pt}}
\multiput(345.00,375.17)(0.500,-1.000){2}{\rule{0.120pt}{0.400pt}}
\put(344.0,372.0){\usebox{\plotpoint}}
\put(345.67,376){\rule{0.400pt}{0.723pt}}
\multiput(345.17,376.00)(1.000,1.500){2}{\rule{0.400pt}{0.361pt}}
\put(346.0,375.0){\usebox{\plotpoint}}
\put(347.0,379.0){\usebox{\plotpoint}}
\put(348,380.67){\rule{0.241pt}{0.400pt}}
\multiput(348.00,380.17)(0.500,1.000){2}{\rule{0.120pt}{0.400pt}}
\put(348.67,382){\rule{0.400pt}{0.482pt}}
\multiput(348.17,382.00)(1.000,1.000){2}{\rule{0.400pt}{0.241pt}}
\put(348.0,379.0){\rule[-0.200pt]{0.400pt}{0.482pt}}
\put(349.67,383){\rule{0.400pt}{0.964pt}}
\multiput(349.17,383.00)(1.000,2.000){2}{\rule{0.400pt}{0.482pt}}
\put(351,386.67){\rule{0.241pt}{0.400pt}}
\multiput(351.00,386.17)(0.500,1.000){2}{\rule{0.120pt}{0.400pt}}
\put(350.0,383.0){\usebox{\plotpoint}}
\put(352,388.67){\rule{0.241pt}{0.400pt}}
\multiput(352.00,388.17)(0.500,1.000){2}{\rule{0.120pt}{0.400pt}}
\put(352.67,390){\rule{0.400pt}{0.482pt}}
\multiput(352.17,390.00)(1.000,1.000){2}{\rule{0.400pt}{0.241pt}}
\put(352.0,388.0){\usebox{\plotpoint}}
\put(354,393.67){\rule{0.241pt}{0.400pt}}
\multiput(354.00,393.17)(0.500,1.000){2}{\rule{0.120pt}{0.400pt}}
\put(354.67,393){\rule{0.400pt}{0.482pt}}
\multiput(354.17,394.00)(1.000,-1.000){2}{\rule{0.400pt}{0.241pt}}
\put(354.0,392.0){\rule[-0.200pt]{0.400pt}{0.482pt}}
\put(355.67,397){\rule{0.400pt}{0.723pt}}
\multiput(355.17,397.00)(1.000,1.500){2}{\rule{0.400pt}{0.361pt}}
\put(356.0,393.0){\rule[-0.200pt]{0.400pt}{0.964pt}}
\put(357.0,400.0){\usebox{\plotpoint}}
\put(358,401.67){\rule{0.241pt}{0.400pt}}
\multiput(358.00,401.17)(0.500,1.000){2}{\rule{0.120pt}{0.400pt}}
\put(358.0,400.0){\rule[-0.200pt]{0.400pt}{0.482pt}}
\put(359,403){\usebox{\plotpoint}}
\put(358.67,403){\rule{0.400pt}{0.964pt}}
\multiput(358.17,403.00)(1.000,2.000){2}{\rule{0.400pt}{0.482pt}}
\put(359.67,407){\rule{0.400pt}{0.482pt}}
\multiput(359.17,407.00)(1.000,1.000){2}{\rule{0.400pt}{0.241pt}}
\put(361,409){\usebox{\plotpoint}}
\put(360.67,409){\rule{0.400pt}{0.723pt}}
\multiput(360.17,409.00)(1.000,1.500){2}{\rule{0.400pt}{0.361pt}}
\put(362,411.67){\rule{0.241pt}{0.400pt}}
\multiput(362.00,411.17)(0.500,1.000){2}{\rule{0.120pt}{0.400pt}}
\put(363,413){\usebox{\plotpoint}}
\put(362.67,413){\rule{0.400pt}{0.482pt}}
\multiput(362.17,413.00)(1.000,1.000){2}{\rule{0.400pt}{0.241pt}}
\put(363.67,415){\rule{0.400pt}{0.723pt}}
\multiput(363.17,415.00)(1.000,1.500){2}{\rule{0.400pt}{0.361pt}}
\put(365,418){\usebox{\plotpoint}}
\put(364.67,418){\rule{0.400pt}{0.723pt}}
\multiput(364.17,418.00)(1.000,1.500){2}{\rule{0.400pt}{0.361pt}}
\put(365.67,419){\rule{0.400pt}{0.482pt}}
\multiput(365.17,420.00)(1.000,-1.000){2}{\rule{0.400pt}{0.241pt}}
\put(367,422.67){\rule{0.241pt}{0.400pt}}
\multiput(367.00,422.17)(0.500,1.000){2}{\rule{0.120pt}{0.400pt}}
\put(367.67,424){\rule{0.400pt}{0.723pt}}
\multiput(367.17,424.00)(1.000,1.500){2}{\rule{0.400pt}{0.361pt}}
\put(367.0,419.0){\rule[-0.200pt]{0.400pt}{0.964pt}}
\put(369,427.67){\rule{0.241pt}{0.400pt}}
\multiput(369.00,427.17)(0.500,1.000){2}{\rule{0.120pt}{0.400pt}}
\put(369.67,429){\rule{0.400pt}{0.723pt}}
\multiput(369.17,429.00)(1.000,1.500){2}{\rule{0.400pt}{0.361pt}}
\put(369.0,427.0){\usebox{\plotpoint}}
\put(371,431.67){\rule{0.241pt}{0.400pt}}
\multiput(371.00,432.17)(0.500,-1.000){2}{\rule{0.120pt}{0.400pt}}
\put(371.67,432){\rule{0.400pt}{0.482pt}}
\multiput(371.17,432.00)(1.000,1.000){2}{\rule{0.400pt}{0.241pt}}
\put(371.0,432.0){\usebox{\plotpoint}}
\put(372.67,438){\rule{0.400pt}{0.482pt}}
\multiput(372.17,438.00)(1.000,1.000){2}{\rule{0.400pt}{0.241pt}}
\put(374,438.67){\rule{0.241pt}{0.400pt}}
\multiput(374.00,439.17)(0.500,-1.000){2}{\rule{0.120pt}{0.400pt}}
\put(373.0,434.0){\rule[-0.200pt]{0.400pt}{0.964pt}}
\put(374.67,442){\rule{0.400pt}{0.723pt}}
\multiput(374.17,442.00)(1.000,1.500){2}{\rule{0.400pt}{0.361pt}}
\put(375.0,439.0){\rule[-0.200pt]{0.400pt}{0.723pt}}
\put(376.0,445.0){\usebox{\plotpoint}}
\put(376.67,446){\rule{0.400pt}{0.482pt}}
\multiput(376.17,446.00)(1.000,1.000){2}{\rule{0.400pt}{0.241pt}}
\put(378,446.67){\rule{0.241pt}{0.400pt}}
\multiput(378.00,447.17)(0.500,-1.000){2}{\rule{0.120pt}{0.400pt}}
\put(377.0,445.0){\usebox{\plotpoint}}
\put(379,450.67){\rule{0.241pt}{0.400pt}}
\multiput(379.00,451.17)(0.500,-1.000){2}{\rule{0.120pt}{0.400pt}}
\put(379.67,451){\rule{0.400pt}{0.482pt}}
\multiput(379.17,451.00)(1.000,1.000){2}{\rule{0.400pt}{0.241pt}}
\put(379.0,447.0){\rule[-0.200pt]{0.400pt}{1.204pt}}
\put(381,455.67){\rule{0.241pt}{0.400pt}}
\multiput(381.00,455.17)(0.500,1.000){2}{\rule{0.120pt}{0.400pt}}
\put(381.67,457){\rule{0.400pt}{0.482pt}}
\multiput(381.17,457.00)(1.000,1.000){2}{\rule{0.400pt}{0.241pt}}
\put(381.0,453.0){\rule[-0.200pt]{0.400pt}{0.723pt}}
\put(383,459){\usebox{\plotpoint}}
\put(383,458.67){\rule{0.241pt}{0.400pt}}
\multiput(383.00,458.17)(0.500,1.000){2}{\rule{0.120pt}{0.400pt}}
\put(383.67,460){\rule{0.400pt}{0.964pt}}
\multiput(383.17,460.00)(1.000,2.000){2}{\rule{0.400pt}{0.482pt}}
\put(384.67,465){\rule{0.400pt}{0.723pt}}
\multiput(384.17,465.00)(1.000,1.500){2}{\rule{0.400pt}{0.361pt}}
\put(385.67,468){\rule{0.400pt}{0.482pt}}
\multiput(385.17,468.00)(1.000,1.000){2}{\rule{0.400pt}{0.241pt}}
\put(385.0,464.0){\usebox{\plotpoint}}
\put(387.0,468.0){\rule[-0.200pt]{0.400pt}{0.482pt}}
\put(387.67,468){\rule{0.400pt}{1.204pt}}
\multiput(387.17,468.00)(1.000,2.500){2}{\rule{0.400pt}{0.602pt}}
\put(387.0,468.0){\usebox{\plotpoint}}
\put(388.67,474){\rule{0.400pt}{0.482pt}}
\multiput(388.17,474.00)(1.000,1.000){2}{\rule{0.400pt}{0.241pt}}
\put(389.67,476){\rule{0.400pt}{0.482pt}}
\multiput(389.17,476.00)(1.000,1.000){2}{\rule{0.400pt}{0.241pt}}
\put(389.0,473.0){\usebox{\plotpoint}}
\put(391,478){\usebox{\plotpoint}}
\put(390.67,478){\rule{0.400pt}{0.723pt}}
\multiput(390.17,478.00)(1.000,1.500){2}{\rule{0.400pt}{0.361pt}}
\put(392,480.67){\rule{0.241pt}{0.400pt}}
\multiput(392.00,480.17)(0.500,1.000){2}{\rule{0.120pt}{0.400pt}}
\put(392.67,486){\rule{0.400pt}{0.482pt}}
\multiput(392.17,486.00)(1.000,1.000){2}{\rule{0.400pt}{0.241pt}}
\put(394,486.67){\rule{0.241pt}{0.400pt}}
\multiput(394.00,487.17)(0.500,-1.000){2}{\rule{0.120pt}{0.400pt}}
\put(393.0,482.0){\rule[-0.200pt]{0.400pt}{0.964pt}}
\put(395,487.67){\rule{0.241pt}{0.400pt}}
\multiput(395.00,488.17)(0.500,-1.000){2}{\rule{0.120pt}{0.400pt}}
\put(395.67,488){\rule{0.400pt}{1.204pt}}
\multiput(395.17,488.00)(1.000,2.500){2}{\rule{0.400pt}{0.602pt}}
\put(395.0,487.0){\rule[-0.200pt]{0.400pt}{0.482pt}}
\put(397,493.67){\rule{0.241pt}{0.400pt}}
\multiput(397.00,493.17)(0.500,1.000){2}{\rule{0.120pt}{0.400pt}}
\put(398,494.67){\rule{0.241pt}{0.400pt}}
\multiput(398.00,494.17)(0.500,1.000){2}{\rule{0.120pt}{0.400pt}}
\put(397.0,493.0){\usebox{\plotpoint}}
\put(399.0,496.0){\rule[-0.200pt]{0.400pt}{0.964pt}}
\put(400,499.67){\rule{0.241pt}{0.400pt}}
\multiput(400.00,499.17)(0.500,1.000){2}{\rule{0.120pt}{0.400pt}}
\put(399.0,500.0){\usebox{\plotpoint}}
\put(401,502.67){\rule{0.241pt}{0.400pt}}
\multiput(401.00,502.17)(0.500,1.000){2}{\rule{0.120pt}{0.400pt}}
\put(401.67,504){\rule{0.400pt}{0.482pt}}
\multiput(401.17,504.00)(1.000,1.000){2}{\rule{0.400pt}{0.241pt}}
\put(401.0,501.0){\rule[-0.200pt]{0.400pt}{0.482pt}}
\put(403,507.67){\rule{0.241pt}{0.400pt}}
\multiput(403.00,507.17)(0.500,1.000){2}{\rule{0.120pt}{0.400pt}}
\put(403.0,506.0){\rule[-0.200pt]{0.400pt}{0.482pt}}
\put(404,509){\usebox{\plotpoint}}
\put(404,507.67){\rule{0.241pt}{0.400pt}}
\multiput(404.00,508.17)(0.500,-1.000){2}{\rule{0.120pt}{0.400pt}}
\put(404.67,508){\rule{0.400pt}{1.686pt}}
\multiput(404.17,508.00)(1.000,3.500){2}{\rule{0.400pt}{0.843pt}}
\put(406,515.67){\rule{0.241pt}{0.400pt}}
\multiput(406.00,516.17)(0.500,-1.000){2}{\rule{0.120pt}{0.400pt}}
\put(406.67,516){\rule{0.400pt}{0.482pt}}
\multiput(406.17,516.00)(1.000,1.000){2}{\rule{0.400pt}{0.241pt}}
\put(406.0,515.0){\rule[-0.200pt]{0.400pt}{0.482pt}}
\put(408.0,518.0){\rule[-0.200pt]{0.400pt}{0.964pt}}
\put(408.67,522){\rule{0.400pt}{0.723pt}}
\multiput(408.17,522.00)(1.000,1.500){2}{\rule{0.400pt}{0.361pt}}
\put(408.0,522.0){\usebox{\plotpoint}}
\put(410.0,525.0){\rule[-0.200pt]{0.400pt}{0.964pt}}
\put(411.67,529){\rule{0.400pt}{0.964pt}}
\multiput(411.17,529.00)(1.000,2.000){2}{\rule{0.400pt}{0.482pt}}
\put(410.0,529.0){\rule[-0.200pt]{0.482pt}{0.400pt}}
\put(413.0,533.0){\usebox{\plotpoint}}
\put(413.67,536){\rule{0.400pt}{0.482pt}}
\multiput(413.17,536.00)(1.000,1.000){2}{\rule{0.400pt}{0.241pt}}
\put(415,537.67){\rule{0.241pt}{0.400pt}}
\multiput(415.00,537.17)(0.500,1.000){2}{\rule{0.120pt}{0.400pt}}
\put(414.0,533.0){\rule[-0.200pt]{0.400pt}{0.723pt}}
\put(416,540.67){\rule{0.241pt}{0.400pt}}
\multiput(416.00,540.17)(0.500,1.000){2}{\rule{0.120pt}{0.400pt}}
\put(416.67,540){\rule{0.400pt}{0.482pt}}
\multiput(416.17,541.00)(1.000,-1.000){2}{\rule{0.400pt}{0.241pt}}
\put(416.0,539.0){\rule[-0.200pt]{0.400pt}{0.482pt}}
\put(417.67,542){\rule{0.400pt}{0.723pt}}
\multiput(417.17,542.00)(1.000,1.500){2}{\rule{0.400pt}{0.361pt}}
\put(418.67,545){\rule{0.400pt}{0.723pt}}
\multiput(418.17,545.00)(1.000,1.500){2}{\rule{0.400pt}{0.361pt}}
\put(418.0,540.0){\rule[-0.200pt]{0.400pt}{0.482pt}}
\put(420,549.67){\rule{0.241pt}{0.400pt}}
\multiput(420.00,550.17)(0.500,-1.000){2}{\rule{0.120pt}{0.400pt}}
\put(420.67,550){\rule{0.400pt}{0.964pt}}
\multiput(420.17,550.00)(1.000,2.000){2}{\rule{0.400pt}{0.482pt}}
\put(420.0,548.0){\rule[-0.200pt]{0.400pt}{0.723pt}}
\put(421.67,554){\rule{0.400pt}{0.723pt}}
\multiput(421.17,555.50)(1.000,-1.500){2}{\rule{0.400pt}{0.361pt}}
\put(422.67,554){\rule{0.400pt}{0.482pt}}
\multiput(422.17,554.00)(1.000,1.000){2}{\rule{0.400pt}{0.241pt}}
\put(422.0,554.0){\rule[-0.200pt]{0.400pt}{0.723pt}}
\put(424,555.67){\rule{0.241pt}{0.400pt}}
\multiput(424.00,556.17)(0.500,-1.000){2}{\rule{0.120pt}{0.400pt}}
\put(424.67,556){\rule{0.400pt}{1.927pt}}
\multiput(424.17,556.00)(1.000,4.000){2}{\rule{0.400pt}{0.964pt}}
\put(424.0,556.0){\usebox{\plotpoint}}
\put(425.67,562){\rule{0.400pt}{0.964pt}}
\multiput(425.17,562.00)(1.000,2.000){2}{\rule{0.400pt}{0.482pt}}
\put(427,565.67){\rule{0.241pt}{0.400pt}}
\multiput(427.00,565.17)(0.500,1.000){2}{\rule{0.120pt}{0.400pt}}
\put(426.0,562.0){\rule[-0.200pt]{0.400pt}{0.482pt}}
\put(428,566.67){\rule{0.241pt}{0.400pt}}
\multiput(428.00,567.17)(0.500,-1.000){2}{\rule{0.120pt}{0.400pt}}
\put(428.67,567){\rule{0.400pt}{1.445pt}}
\multiput(428.17,567.00)(1.000,3.000){2}{\rule{0.400pt}{0.723pt}}
\put(428.0,567.0){\usebox{\plotpoint}}
\put(430.0,573.0){\rule[-0.200pt]{0.400pt}{0.482pt}}
\put(430.67,571){\rule{0.400pt}{0.964pt}}
\multiput(430.17,573.00)(1.000,-2.000){2}{\rule{0.400pt}{0.482pt}}
\put(430.0,575.0){\usebox{\plotpoint}}
\put(431.67,577){\rule{0.400pt}{0.482pt}}
\multiput(431.17,577.00)(1.000,1.000){2}{\rule{0.400pt}{0.241pt}}
\put(432.0,571.0){\rule[-0.200pt]{0.400pt}{1.445pt}}
\put(432.67,581){\rule{0.400pt}{0.723pt}}
\multiput(432.17,581.00)(1.000,1.500){2}{\rule{0.400pt}{0.361pt}}
\put(433.67,584){\rule{0.400pt}{0.964pt}}
\multiput(433.17,584.00)(1.000,2.000){2}{\rule{0.400pt}{0.482pt}}
\put(433.0,579.0){\rule[-0.200pt]{0.400pt}{0.482pt}}
\put(434.67,588){\rule{0.400pt}{0.723pt}}
\multiput(434.17,589.50)(1.000,-1.500){2}{\rule{0.400pt}{0.361pt}}
\put(435.67,588){\rule{0.400pt}{0.723pt}}
\multiput(435.17,588.00)(1.000,1.500){2}{\rule{0.400pt}{0.361pt}}
\put(435.0,588.0){\rule[-0.200pt]{0.400pt}{0.723pt}}
\put(436.67,586){\rule{0.400pt}{1.686pt}}
\multiput(436.17,586.00)(1.000,3.500){2}{\rule{0.400pt}{0.843pt}}
\put(437.67,593){\rule{0.400pt}{0.723pt}}
\multiput(437.17,593.00)(1.000,1.500){2}{\rule{0.400pt}{0.361pt}}
\put(437.0,586.0){\rule[-0.200pt]{0.400pt}{1.204pt}}
\put(438.67,594){\rule{0.400pt}{0.482pt}}
\multiput(438.17,594.00)(1.000,1.000){2}{\rule{0.400pt}{0.241pt}}
\put(439.67,596){\rule{0.400pt}{0.482pt}}
\multiput(439.17,596.00)(1.000,1.000){2}{\rule{0.400pt}{0.241pt}}
\put(439.0,594.0){\rule[-0.200pt]{0.400pt}{0.482pt}}
\put(440.67,600){\rule{0.400pt}{0.964pt}}
\multiput(440.17,600.00)(1.000,2.000){2}{\rule{0.400pt}{0.482pt}}
\put(441.0,598.0){\rule[-0.200pt]{0.400pt}{0.482pt}}
\put(442.0,604.0){\usebox{\plotpoint}}
\put(442.67,601){\rule{0.400pt}{0.482pt}}
\multiput(442.17,601.00)(1.000,1.000){2}{\rule{0.400pt}{0.241pt}}
\put(443.0,601.0){\rule[-0.200pt]{0.400pt}{0.723pt}}
\put(444.0,603.0){\usebox{\plotpoint}}
\put(444.67,607){\rule{0.400pt}{1.445pt}}
\multiput(444.17,607.00)(1.000,3.000){2}{\rule{0.400pt}{0.723pt}}
\put(445.67,611){\rule{0.400pt}{0.482pt}}
\multiput(445.17,612.00)(1.000,-1.000){2}{\rule{0.400pt}{0.241pt}}
\put(445.0,603.0){\rule[-0.200pt]{0.400pt}{0.964pt}}
\put(447,611){\usebox{\plotpoint}}
\put(447,610.67){\rule{0.241pt}{0.400pt}}
\multiput(447.00,610.17)(0.500,1.000){2}{\rule{0.120pt}{0.400pt}}
\put(448,611.67){\rule{0.241pt}{0.400pt}}
\multiput(448.00,611.17)(0.500,1.000){2}{\rule{0.120pt}{0.400pt}}
\put(449,614.67){\rule{0.241pt}{0.400pt}}
\multiput(449.00,615.17)(0.500,-1.000){2}{\rule{0.120pt}{0.400pt}}
\put(449.67,615){\rule{0.400pt}{1.204pt}}
\multiput(449.17,615.00)(1.000,2.500){2}{\rule{0.400pt}{0.602pt}}
\put(449.0,613.0){\rule[-0.200pt]{0.400pt}{0.723pt}}
\put(450.67,619){\rule{0.400pt}{0.964pt}}
\multiput(450.17,621.00)(1.000,-2.000){2}{\rule{0.400pt}{0.482pt}}
\put(451.67,619){\rule{0.400pt}{1.445pt}}
\multiput(451.17,619.00)(1.000,3.000){2}{\rule{0.400pt}{0.723pt}}
\put(451.0,620.0){\rule[-0.200pt]{0.400pt}{0.723pt}}
\put(453,625){\usebox{\plotpoint}}
\put(453,623.67){\rule{0.241pt}{0.400pt}}
\multiput(453.00,624.17)(0.500,-1.000){2}{\rule{0.120pt}{0.400pt}}
\put(453.67,624){\rule{0.400pt}{0.482pt}}
\multiput(453.17,624.00)(1.000,1.000){2}{\rule{0.400pt}{0.241pt}}
\put(455,626.67){\rule{0.241pt}{0.400pt}}
\multiput(455.00,626.17)(0.500,1.000){2}{\rule{0.120pt}{0.400pt}}
\put(455.0,626.0){\usebox{\plotpoint}}
\put(456.0,628.0){\usebox{\plotpoint}}
\put(457,630.67){\rule{0.241pt}{0.400pt}}
\multiput(457.00,631.17)(0.500,-1.000){2}{\rule{0.120pt}{0.400pt}}
\put(457.0,628.0){\rule[-0.200pt]{0.400pt}{0.964pt}}
\put(457.67,631){\rule{0.400pt}{0.482pt}}
\multiput(457.17,632.00)(1.000,-1.000){2}{\rule{0.400pt}{0.241pt}}
\put(458.67,631){\rule{0.400pt}{1.204pt}}
\multiput(458.17,631.00)(1.000,2.500){2}{\rule{0.400pt}{0.602pt}}
\put(458.0,631.0){\rule[-0.200pt]{0.400pt}{0.482pt}}
\put(459.67,635){\rule{0.400pt}{0.964pt}}
\multiput(459.17,635.00)(1.000,2.000){2}{\rule{0.400pt}{0.482pt}}
\put(460.67,639){\rule{0.400pt}{0.723pt}}
\multiput(460.17,639.00)(1.000,1.500){2}{\rule{0.400pt}{0.361pt}}
\put(460.0,635.0){\usebox{\plotpoint}}
\put(462,638.67){\rule{0.241pt}{0.400pt}}
\multiput(462.00,639.17)(0.500,-1.000){2}{\rule{0.120pt}{0.400pt}}
\put(462.67,639){\rule{0.400pt}{2.168pt}}
\multiput(462.17,639.00)(1.000,4.500){2}{\rule{0.400pt}{1.084pt}}
\put(462.0,640.0){\rule[-0.200pt]{0.400pt}{0.482pt}}
\put(463.67,642){\rule{0.400pt}{0.482pt}}
\multiput(463.17,643.00)(1.000,-1.000){2}{\rule{0.400pt}{0.241pt}}
\put(464.67,642){\rule{0.400pt}{0.964pt}}
\multiput(464.17,642.00)(1.000,2.000){2}{\rule{0.400pt}{0.482pt}}
\put(464.0,644.0){\rule[-0.200pt]{0.400pt}{0.964pt}}
\put(465.67,644){\rule{0.400pt}{0.964pt}}
\multiput(465.17,646.00)(1.000,-2.000){2}{\rule{0.400pt}{0.482pt}}
\put(466.67,644){\rule{0.400pt}{1.927pt}}
\multiput(466.17,644.00)(1.000,4.000){2}{\rule{0.400pt}{0.964pt}}
\put(466.0,646.0){\rule[-0.200pt]{0.400pt}{0.482pt}}
\put(467.67,646){\rule{0.400pt}{0.964pt}}
\multiput(467.17,648.00)(1.000,-2.000){2}{\rule{0.400pt}{0.482pt}}
\put(468.67,646){\rule{0.400pt}{0.723pt}}
\multiput(468.17,646.00)(1.000,1.500){2}{\rule{0.400pt}{0.361pt}}
\put(468.0,650.0){\rule[-0.200pt]{0.400pt}{0.482pt}}
\put(470,648.67){\rule{0.241pt}{0.400pt}}
\multiput(470.00,649.17)(0.500,-1.000){2}{\rule{0.120pt}{0.400pt}}
\put(471,648.67){\rule{0.241pt}{0.400pt}}
\multiput(471.00,648.17)(0.500,1.000){2}{\rule{0.120pt}{0.400pt}}
\put(470.0,649.0){\usebox{\plotpoint}}
\put(472,654.67){\rule{0.241pt}{0.400pt}}
\multiput(472.00,654.17)(0.500,1.000){2}{\rule{0.120pt}{0.400pt}}
\put(472.67,654){\rule{0.400pt}{0.482pt}}
\multiput(472.17,655.00)(1.000,-1.000){2}{\rule{0.400pt}{0.241pt}}
\put(472.0,650.0){\rule[-0.200pt]{0.400pt}{1.204pt}}
\put(474,654){\usebox{\plotpoint}}
\put(473.67,654){\rule{0.400pt}{0.723pt}}
\multiput(473.17,654.00)(1.000,1.500){2}{\rule{0.400pt}{0.361pt}}
\put(475,655.67){\rule{0.241pt}{0.400pt}}
\multiput(475.00,656.17)(0.500,-1.000){2}{\rule{0.120pt}{0.400pt}}
\put(475.67,654){\rule{0.400pt}{0.964pt}}
\multiput(475.17,654.00)(1.000,2.000){2}{\rule{0.400pt}{0.482pt}}
\put(476.0,654.0){\rule[-0.200pt]{0.400pt}{0.482pt}}
\put(478,656.67){\rule{0.241pt}{0.400pt}}
\multiput(478.00,657.17)(0.500,-1.000){2}{\rule{0.120pt}{0.400pt}}
\put(477.0,658.0){\usebox{\plotpoint}}
\put(478.67,658){\rule{0.400pt}{0.723pt}}
\multiput(478.17,659.50)(1.000,-1.500){2}{\rule{0.400pt}{0.361pt}}
\put(480,656.67){\rule{0.241pt}{0.400pt}}
\multiput(480.00,657.17)(0.500,-1.000){2}{\rule{0.120pt}{0.400pt}}
\put(479.0,657.0){\rule[-0.200pt]{0.400pt}{0.964pt}}
\put(480.67,661){\rule{0.400pt}{0.482pt}}
\multiput(480.17,662.00)(1.000,-1.000){2}{\rule{0.400pt}{0.241pt}}
\put(481.0,657.0){\rule[-0.200pt]{0.400pt}{1.445pt}}
\put(482.0,661.0){\usebox{\plotpoint}}
\put(482.67,659){\rule{0.400pt}{1.204pt}}
\multiput(482.17,661.50)(1.000,-2.500){2}{\rule{0.400pt}{0.602pt}}
\put(483.67,659){\rule{0.400pt}{0.964pt}}
\multiput(483.17,659.00)(1.000,2.000){2}{\rule{0.400pt}{0.482pt}}
\put(483.0,661.0){\rule[-0.200pt]{0.400pt}{0.723pt}}
\put(484.67,660){\rule{0.400pt}{0.482pt}}
\multiput(484.17,660.00)(1.000,1.000){2}{\rule{0.400pt}{0.241pt}}
\put(486,661.67){\rule{0.241pt}{0.400pt}}
\multiput(486.00,661.17)(0.500,1.000){2}{\rule{0.120pt}{0.400pt}}
\put(485.0,660.0){\rule[-0.200pt]{0.400pt}{0.723pt}}
\put(487,663){\usebox{\plotpoint}}
\put(486.67,656){\rule{0.400pt}{1.686pt}}
\multiput(486.17,659.50)(1.000,-3.500){2}{\rule{0.400pt}{0.843pt}}
\put(487.67,647){\rule{0.400pt}{2.168pt}}
\multiput(487.17,651.50)(1.000,-4.500){2}{\rule{0.400pt}{1.084pt}}
\put(488.67,667){\rule{0.400pt}{0.482pt}}
\multiput(488.17,668.00)(1.000,-1.000){2}{\rule{0.400pt}{0.241pt}}
\put(489.0,647.0){\rule[-0.200pt]{0.400pt}{5.300pt}}
\put(490.67,657){\rule{0.400pt}{2.409pt}}
\multiput(490.17,662.00)(1.000,-5.000){2}{\rule{0.400pt}{1.204pt}}
\put(492,655.67){\rule{0.241pt}{0.400pt}}
\multiput(492.00,656.17)(0.500,-1.000){2}{\rule{0.120pt}{0.400pt}}
\put(490.0,667.0){\usebox{\plotpoint}}
\put(493,656){\usebox{\plotpoint}}
\put(492.67,656){\rule{0.400pt}{2.409pt}}
\multiput(492.17,656.00)(1.000,5.000){2}{\rule{0.400pt}{1.204pt}}
\put(494,665.67){\rule{0.241pt}{0.400pt}}
\multiput(494.00,665.17)(0.500,1.000){2}{\rule{0.120pt}{0.400pt}}
\put(494.67,661){\rule{0.400pt}{0.964pt}}
\multiput(494.17,661.00)(1.000,2.000){2}{\rule{0.400pt}{0.482pt}}
\put(495.0,661.0){\rule[-0.200pt]{0.400pt}{1.445pt}}
\put(496.0,665.0){\usebox{\plotpoint}}
\put(496.67,664){\rule{0.400pt}{0.964pt}}
\multiput(496.17,664.00)(1.000,2.000){2}{\rule{0.400pt}{0.482pt}}
\put(497.0,664.0){\usebox{\plotpoint}}
\put(498,668){\usebox{\plotpoint}}
\put(497.67,658){\rule{0.400pt}{2.409pt}}
\multiput(497.17,663.00)(1.000,-5.000){2}{\rule{0.400pt}{1.204pt}}
\put(499.0,658.0){\usebox{\plotpoint}}
\put(499.67,661){\rule{0.400pt}{1.204pt}}
\multiput(499.17,661.00)(1.000,2.500){2}{\rule{0.400pt}{0.602pt}}
\put(500.67,662){\rule{0.400pt}{0.964pt}}
\multiput(500.17,664.00)(1.000,-2.000){2}{\rule{0.400pt}{0.482pt}}
\put(500.0,658.0){\rule[-0.200pt]{0.400pt}{0.723pt}}
\put(502.0,660.0){\rule[-0.200pt]{0.400pt}{0.482pt}}
\put(502.67,660){\rule{0.400pt}{1.204pt}}
\multiput(502.17,660.00)(1.000,2.500){2}{\rule{0.400pt}{0.602pt}}
\put(502.0,660.0){\usebox{\plotpoint}}
\put(503.67,661){\rule{0.400pt}{0.723pt}}
\multiput(503.17,662.50)(1.000,-1.500){2}{\rule{0.400pt}{0.361pt}}
\put(504.67,661){\rule{0.400pt}{0.482pt}}
\multiput(504.17,661.00)(1.000,1.000){2}{\rule{0.400pt}{0.241pt}}
\put(504.0,664.0){\usebox{\plotpoint}}
\put(506,663.67){\rule{0.241pt}{0.400pt}}
\multiput(506.00,663.17)(0.500,1.000){2}{\rule{0.120pt}{0.400pt}}
\put(506.67,661){\rule{0.400pt}{0.964pt}}
\multiput(506.17,663.00)(1.000,-2.000){2}{\rule{0.400pt}{0.482pt}}
\put(506.0,663.0){\usebox{\plotpoint}}
\put(507.67,659){\rule{0.400pt}{1.204pt}}
\multiput(507.17,661.50)(1.000,-2.500){2}{\rule{0.400pt}{0.602pt}}
\put(508.67,659){\rule{0.400pt}{0.964pt}}
\multiput(508.17,659.00)(1.000,2.000){2}{\rule{0.400pt}{0.482pt}}
\put(508.0,661.0){\rule[-0.200pt]{0.400pt}{0.723pt}}
\put(509.67,658){\rule{0.400pt}{1.445pt}}
\multiput(509.17,658.00)(1.000,3.000){2}{\rule{0.400pt}{0.723pt}}
\put(510.67,656){\rule{0.400pt}{1.927pt}}
\multiput(510.17,660.00)(1.000,-4.000){2}{\rule{0.400pt}{0.964pt}}
\put(510.0,658.0){\rule[-0.200pt]{0.400pt}{1.204pt}}
\put(512,659.67){\rule{0.241pt}{0.400pt}}
\multiput(512.00,660.17)(0.500,-1.000){2}{\rule{0.120pt}{0.400pt}}
\put(512.67,660){\rule{0.400pt}{0.723pt}}
\multiput(512.17,660.00)(1.000,1.500){2}{\rule{0.400pt}{0.361pt}}
\put(512.0,656.0){\rule[-0.200pt]{0.400pt}{1.204pt}}
\put(513.67,659){\rule{0.400pt}{0.964pt}}
\multiput(513.17,659.00)(1.000,2.000){2}{\rule{0.400pt}{0.482pt}}
\put(514.0,659.0){\rule[-0.200pt]{0.400pt}{0.964pt}}
\put(514.67,656){\rule{0.400pt}{0.723pt}}
\multiput(514.17,657.50)(1.000,-1.500){2}{\rule{0.400pt}{0.361pt}}
\put(515.67,652){\rule{0.400pt}{0.964pt}}
\multiput(515.17,654.00)(1.000,-2.000){2}{\rule{0.400pt}{0.482pt}}
\put(515.0,659.0){\rule[-0.200pt]{0.400pt}{0.964pt}}
\put(516.67,653){\rule{0.400pt}{1.204pt}}
\multiput(516.17,655.50)(1.000,-2.500){2}{\rule{0.400pt}{0.602pt}}
\put(517.0,652.0){\rule[-0.200pt]{0.400pt}{1.445pt}}
\put(518.0,653.0){\usebox{\plotpoint}}
\put(518.67,648){\rule{0.400pt}{2.168pt}}
\multiput(518.17,652.50)(1.000,-4.500){2}{\rule{0.400pt}{1.084pt}}
\put(519.67,648){\rule{0.400pt}{1.204pt}}
\multiput(519.17,648.00)(1.000,2.500){2}{\rule{0.400pt}{0.602pt}}
\put(519.0,653.0){\rule[-0.200pt]{0.400pt}{0.964pt}}
\put(520.67,659){\rule{0.400pt}{0.482pt}}
\multiput(520.17,659.00)(1.000,1.000){2}{\rule{0.400pt}{0.241pt}}
\put(521.67,650){\rule{0.400pt}{2.650pt}}
\multiput(521.17,655.50)(1.000,-5.500){2}{\rule{0.400pt}{1.325pt}}
\put(521.0,653.0){\rule[-0.200pt]{0.400pt}{1.445pt}}
\put(522.67,646){\rule{0.400pt}{0.964pt}}
\multiput(522.17,646.00)(1.000,2.000){2}{\rule{0.400pt}{0.482pt}}
\put(523.0,646.0){\rule[-0.200pt]{0.400pt}{0.964pt}}
\put(524.0,650.0){\usebox{\plotpoint}}
\put(525.0,647.0){\rule[-0.200pt]{0.400pt}{0.723pt}}
\put(525.67,647){\rule{0.400pt}{1.445pt}}
\multiput(525.17,647.00)(1.000,3.000){2}{\rule{0.400pt}{0.723pt}}
\put(525.0,647.0){\usebox{\plotpoint}}
\put(526.67,649){\rule{0.400pt}{0.482pt}}
\multiput(526.17,650.00)(1.000,-1.000){2}{\rule{0.400pt}{0.241pt}}
\put(527.0,651.0){\rule[-0.200pt]{0.400pt}{0.482pt}}
\put(528.0,649.0){\usebox{\plotpoint}}
\put(528.67,642){\rule{0.400pt}{0.723pt}}
\multiput(528.17,643.50)(1.000,-1.500){2}{\rule{0.400pt}{0.361pt}}
\put(529.0,645.0){\rule[-0.200pt]{0.400pt}{0.964pt}}
\put(529.67,650){\rule{0.400pt}{0.723pt}}
\multiput(529.17,650.00)(1.000,1.500){2}{\rule{0.400pt}{0.361pt}}
\put(530.67,646){\rule{0.400pt}{1.686pt}}
\multiput(530.17,649.50)(1.000,-3.500){2}{\rule{0.400pt}{0.843pt}}
\put(530.0,642.0){\rule[-0.200pt]{0.400pt}{1.927pt}}
\put(531.67,640){\rule{0.400pt}{1.445pt}}
\multiput(531.17,640.00)(1.000,3.000){2}{\rule{0.400pt}{0.723pt}}
\put(532.67,640){\rule{0.400pt}{1.445pt}}
\multiput(532.17,643.00)(1.000,-3.000){2}{\rule{0.400pt}{0.723pt}}
\put(532.0,640.0){\rule[-0.200pt]{0.400pt}{1.445pt}}
\put(533.67,641){\rule{0.400pt}{0.482pt}}
\multiput(533.17,641.00)(1.000,1.000){2}{\rule{0.400pt}{0.241pt}}
\put(534.67,637){\rule{0.400pt}{1.445pt}}
\multiput(534.17,640.00)(1.000,-3.000){2}{\rule{0.400pt}{0.723pt}}
\put(534.0,640.0){\usebox{\plotpoint}}
\put(535.67,638){\rule{0.400pt}{2.168pt}}
\multiput(535.17,638.00)(1.000,4.500){2}{\rule{0.400pt}{1.084pt}}
\put(536.67,644){\rule{0.400pt}{0.723pt}}
\multiput(536.17,645.50)(1.000,-1.500){2}{\rule{0.400pt}{0.361pt}}
\put(536.0,637.0){\usebox{\plotpoint}}
\put(537.67,634){\rule{0.400pt}{0.964pt}}
\multiput(537.17,636.00)(1.000,-2.000){2}{\rule{0.400pt}{0.482pt}}
\put(539,633.67){\rule{0.241pt}{0.400pt}}
\multiput(539.00,633.17)(0.500,1.000){2}{\rule{0.120pt}{0.400pt}}
\put(538.0,638.0){\rule[-0.200pt]{0.400pt}{1.445pt}}
\put(540,635){\usebox{\plotpoint}}
\put(540,634.67){\rule{0.241pt}{0.400pt}}
\multiput(540.00,634.17)(0.500,1.000){2}{\rule{0.120pt}{0.400pt}}
\put(540.67,631){\rule{0.400pt}{1.204pt}}
\multiput(540.17,633.50)(1.000,-2.500){2}{\rule{0.400pt}{0.602pt}}
\put(541.67,635){\rule{0.400pt}{2.168pt}}
\multiput(541.17,639.50)(1.000,-4.500){2}{\rule{0.400pt}{1.084pt}}
\put(542.67,635){\rule{0.400pt}{0.964pt}}
\multiput(542.17,635.00)(1.000,2.000){2}{\rule{0.400pt}{0.482pt}}
\put(542.0,631.0){\rule[-0.200pt]{0.400pt}{3.132pt}}
\put(544,633.67){\rule{0.241pt}{0.400pt}}
\multiput(544.00,633.17)(0.500,1.000){2}{\rule{0.120pt}{0.400pt}}
\put(544.0,634.0){\rule[-0.200pt]{0.400pt}{1.204pt}}
\put(545,635){\usebox{\plotpoint}}
\put(544.67,630){\rule{0.400pt}{1.204pt}}
\multiput(544.17,632.50)(1.000,-2.500){2}{\rule{0.400pt}{0.602pt}}
\put(545.67,630){\rule{0.400pt}{1.686pt}}
\multiput(545.17,630.00)(1.000,3.500){2}{\rule{0.400pt}{0.843pt}}
\put(546.67,632){\rule{0.400pt}{0.723pt}}
\multiput(546.17,633.50)(1.000,-1.500){2}{\rule{0.400pt}{0.361pt}}
\put(547.0,635.0){\rule[-0.200pt]{0.400pt}{0.482pt}}
\put(548.0,632.0){\usebox{\plotpoint}}
\put(548.67,629){\rule{0.400pt}{2.409pt}}
\multiput(548.17,634.00)(1.000,-5.000){2}{\rule{0.400pt}{1.204pt}}
\put(549.67,629){\rule{0.400pt}{0.964pt}}
\multiput(549.17,629.00)(1.000,2.000){2}{\rule{0.400pt}{0.482pt}}
\put(549.0,632.0){\rule[-0.200pt]{0.400pt}{1.686pt}}
\put(551.0,627.0){\rule[-0.200pt]{0.400pt}{1.445pt}}
\put(551.67,627){\rule{0.400pt}{0.723pt}}
\multiput(551.17,627.00)(1.000,1.500){2}{\rule{0.400pt}{0.361pt}}
\put(551.0,627.0){\usebox{\plotpoint}}
\put(553.0,627.0){\rule[-0.200pt]{0.400pt}{0.723pt}}
\put(553.67,617){\rule{0.400pt}{2.409pt}}
\multiput(553.17,622.00)(1.000,-5.000){2}{\rule{0.400pt}{1.204pt}}
\put(553.0,627.0){\usebox{\plotpoint}}
\put(555,623.67){\rule{0.241pt}{0.400pt}}
\multiput(555.00,624.17)(0.500,-1.000){2}{\rule{0.120pt}{0.400pt}}
\put(555.0,617.0){\rule[-0.200pt]{0.400pt}{1.927pt}}
\put(556.0,624.0){\usebox{\plotpoint}}
\put(556.67,625){\rule{0.400pt}{0.482pt}}
\multiput(556.17,626.00)(1.000,-1.000){2}{\rule{0.400pt}{0.241pt}}
\put(558,624.67){\rule{0.241pt}{0.400pt}}
\multiput(558.00,624.17)(0.500,1.000){2}{\rule{0.120pt}{0.400pt}}
\put(557.0,624.0){\rule[-0.200pt]{0.400pt}{0.723pt}}
\put(558.67,619){\rule{0.400pt}{0.482pt}}
\multiput(558.17,620.00)(1.000,-1.000){2}{\rule{0.400pt}{0.241pt}}
\put(559.0,621.0){\rule[-0.200pt]{0.400pt}{1.204pt}}
\put(560,619){\usebox{\plotpoint}}
\put(559.67,619){\rule{0.400pt}{0.482pt}}
\multiput(559.17,619.00)(1.000,1.000){2}{\rule{0.400pt}{0.241pt}}
\put(560.67,619){\rule{0.400pt}{0.482pt}}
\multiput(560.17,620.00)(1.000,-1.000){2}{\rule{0.400pt}{0.241pt}}
\put(562,619){\usebox{\plotpoint}}
\put(562.67,619){\rule{0.400pt}{1.686pt}}
\multiput(562.17,619.00)(1.000,3.500){2}{\rule{0.400pt}{0.843pt}}
\put(562.0,619.0){\usebox{\plotpoint}}
\put(563.67,607){\rule{0.400pt}{2.891pt}}
\multiput(563.17,607.00)(1.000,6.000){2}{\rule{0.400pt}{1.445pt}}
\put(564.67,613){\rule{0.400pt}{1.445pt}}
\multiput(564.17,616.00)(1.000,-3.000){2}{\rule{0.400pt}{0.723pt}}
\put(564.0,607.0){\rule[-0.200pt]{0.400pt}{4.577pt}}
\put(566.0,613.0){\rule[-0.200pt]{0.400pt}{0.482pt}}
\put(566.67,610){\rule{0.400pt}{1.204pt}}
\multiput(566.17,612.50)(1.000,-2.500){2}{\rule{0.400pt}{0.602pt}}
\put(566.0,615.0){\usebox{\plotpoint}}
\put(567.67,611){\rule{0.400pt}{1.204pt}}
\multiput(567.17,611.00)(1.000,2.500){2}{\rule{0.400pt}{0.602pt}}
\put(568.67,610){\rule{0.400pt}{1.445pt}}
\multiput(568.17,613.00)(1.000,-3.000){2}{\rule{0.400pt}{0.723pt}}
\put(568.0,610.0){\usebox{\plotpoint}}
\put(569.67,611){\rule{0.400pt}{0.964pt}}
\multiput(569.17,611.00)(1.000,2.000){2}{\rule{0.400pt}{0.482pt}}
\put(570.67,602){\rule{0.400pt}{3.132pt}}
\multiput(570.17,608.50)(1.000,-6.500){2}{\rule{0.400pt}{1.566pt}}
\put(570.0,610.0){\usebox{\plotpoint}}
\put(572.0,602.0){\rule[-0.200pt]{0.400pt}{1.927pt}}
\put(572.0,610.0){\usebox{\plotpoint}}
\put(572.67,606){\rule{0.400pt}{0.723pt}}
\multiput(572.17,606.00)(1.000,1.500){2}{\rule{0.400pt}{0.361pt}}
\put(573.67,602){\rule{0.400pt}{1.686pt}}
\multiput(573.17,605.50)(1.000,-3.500){2}{\rule{0.400pt}{0.843pt}}
\put(573.0,606.0){\rule[-0.200pt]{0.400pt}{0.964pt}}
\put(574.67,605){\rule{0.400pt}{0.723pt}}
\multiput(574.17,606.50)(1.000,-1.500){2}{\rule{0.400pt}{0.361pt}}
\put(575.67,605){\rule{0.400pt}{0.723pt}}
\multiput(575.17,605.00)(1.000,1.500){2}{\rule{0.400pt}{0.361pt}}
\put(575.0,602.0){\rule[-0.200pt]{0.400pt}{1.445pt}}
\put(576.67,600){\rule{0.400pt}{0.723pt}}
\multiput(576.17,600.00)(1.000,1.500){2}{\rule{0.400pt}{0.361pt}}
\put(577.67,596){\rule{0.400pt}{1.686pt}}
\multiput(577.17,599.50)(1.000,-3.500){2}{\rule{0.400pt}{0.843pt}}
\put(577.0,600.0){\rule[-0.200pt]{0.400pt}{1.927pt}}
\put(579,596){\usebox{\plotpoint}}
\put(578.67,596){\rule{0.400pt}{1.686pt}}
\multiput(578.17,596.00)(1.000,3.500){2}{\rule{0.400pt}{0.843pt}}
\put(579.67,596){\rule{0.400pt}{1.686pt}}
\multiput(579.17,599.50)(1.000,-3.500){2}{\rule{0.400pt}{0.843pt}}
\put(580.67,589){\rule{0.400pt}{1.445pt}}
\multiput(580.17,589.00)(1.000,3.000){2}{\rule{0.400pt}{0.723pt}}
\put(581.67,595){\rule{0.400pt}{2.168pt}}
\multiput(581.17,595.00)(1.000,4.500){2}{\rule{0.400pt}{1.084pt}}
\put(581.0,589.0){\rule[-0.200pt]{0.400pt}{1.686pt}}
\put(582.67,596){\rule{0.400pt}{0.482pt}}
\multiput(582.17,596.00)(1.000,1.000){2}{\rule{0.400pt}{0.241pt}}
\put(583.0,596.0){\rule[-0.200pt]{0.400pt}{1.927pt}}
\put(583.67,595){\rule{0.400pt}{0.482pt}}
\multiput(583.17,595.00)(1.000,1.000){2}{\rule{0.400pt}{0.241pt}}
\put(584.67,591){\rule{0.400pt}{1.445pt}}
\multiput(584.17,594.00)(1.000,-3.000){2}{\rule{0.400pt}{0.723pt}}
\put(584.0,595.0){\rule[-0.200pt]{0.400pt}{0.723pt}}
\put(586,588.67){\rule{0.241pt}{0.400pt}}
\multiput(586.00,588.17)(0.500,1.000){2}{\rule{0.120pt}{0.400pt}}
\put(586.0,589.0){\rule[-0.200pt]{0.400pt}{0.482pt}}
\put(587.0,590.0){\usebox{\plotpoint}}
\put(587.67,584){\rule{0.400pt}{0.723pt}}
\multiput(587.17,584.00)(1.000,1.500){2}{\rule{0.400pt}{0.361pt}}
\put(589,585.67){\rule{0.241pt}{0.400pt}}
\multiput(589.00,586.17)(0.500,-1.000){2}{\rule{0.120pt}{0.400pt}}
\put(588.0,584.0){\rule[-0.200pt]{0.400pt}{1.445pt}}
\put(590,579.67){\rule{0.241pt}{0.400pt}}
\multiput(590.00,580.17)(0.500,-1.000){2}{\rule{0.120pt}{0.400pt}}
\put(590.67,580){\rule{0.400pt}{1.204pt}}
\multiput(590.17,580.00)(1.000,2.500){2}{\rule{0.400pt}{0.602pt}}
\put(590.0,581.0){\rule[-0.200pt]{0.400pt}{1.204pt}}
\put(591.67,582){\rule{0.400pt}{0.964pt}}
\multiput(591.17,584.00)(1.000,-2.000){2}{\rule{0.400pt}{0.482pt}}
\put(592.67,578){\rule{0.400pt}{0.964pt}}
\multiput(592.17,580.00)(1.000,-2.000){2}{\rule{0.400pt}{0.482pt}}
\put(592.0,585.0){\usebox{\plotpoint}}
\put(594,578){\usebox{\plotpoint}}
\put(594.67,574){\rule{0.400pt}{0.964pt}}
\multiput(594.17,576.00)(1.000,-2.000){2}{\rule{0.400pt}{0.482pt}}
\put(594.0,578.0){\usebox{\plotpoint}}
\put(596.0,574.0){\usebox{\plotpoint}}
\put(597,573.67){\rule{0.241pt}{0.400pt}}
\multiput(597.00,574.17)(0.500,-1.000){2}{\rule{0.120pt}{0.400pt}}
\put(597.67,570){\rule{0.400pt}{0.964pt}}
\multiput(597.17,572.00)(1.000,-2.000){2}{\rule{0.400pt}{0.482pt}}
\put(596.0,575.0){\usebox{\plotpoint}}
\put(598.67,562){\rule{0.400pt}{1.204pt}}
\multiput(598.17,564.50)(1.000,-2.500){2}{\rule{0.400pt}{0.602pt}}
\put(599.67,562){\rule{0.400pt}{2.409pt}}
\multiput(599.17,562.00)(1.000,5.000){2}{\rule{0.400pt}{1.204pt}}
\put(599.0,567.0){\rule[-0.200pt]{0.400pt}{0.723pt}}
\put(600.67,562){\rule{0.400pt}{0.482pt}}
\multiput(600.17,563.00)(1.000,-1.000){2}{\rule{0.400pt}{0.241pt}}
\put(601.0,564.0){\rule[-0.200pt]{0.400pt}{1.927pt}}
\put(602.0,562.0){\usebox{\plotpoint}}
\put(602.67,562){\rule{0.400pt}{0.964pt}}
\multiput(602.17,564.00)(1.000,-2.000){2}{\rule{0.400pt}{0.482pt}}
\put(603.67,555){\rule{0.400pt}{1.686pt}}
\multiput(603.17,558.50)(1.000,-3.500){2}{\rule{0.400pt}{0.843pt}}
\put(603.0,562.0){\rule[-0.200pt]{0.400pt}{0.964pt}}
\put(604.67,556){\rule{0.400pt}{0.723pt}}
\multiput(604.17,557.50)(1.000,-1.500){2}{\rule{0.400pt}{0.361pt}}
\put(605.67,551){\rule{0.400pt}{1.204pt}}
\multiput(605.17,553.50)(1.000,-2.500){2}{\rule{0.400pt}{0.602pt}}
\put(605.0,555.0){\rule[-0.200pt]{0.400pt}{0.964pt}}
\put(606.67,547){\rule{0.400pt}{2.168pt}}
\multiput(606.17,551.50)(1.000,-4.500){2}{\rule{0.400pt}{1.084pt}}
\put(607.0,551.0){\rule[-0.200pt]{0.400pt}{1.204pt}}
\put(607.67,545){\rule{0.400pt}{1.204pt}}
\multiput(607.17,547.50)(1.000,-2.500){2}{\rule{0.400pt}{0.602pt}}
\put(608.67,545){\rule{0.400pt}{0.723pt}}
\multiput(608.17,545.00)(1.000,1.500){2}{\rule{0.400pt}{0.361pt}}
\put(608.0,547.0){\rule[-0.200pt]{0.400pt}{0.723pt}}
\put(610,544.67){\rule{0.241pt}{0.400pt}}
\multiput(610.00,544.17)(0.500,1.000){2}{\rule{0.120pt}{0.400pt}}
\put(610.67,543){\rule{0.400pt}{0.723pt}}
\multiput(610.17,544.50)(1.000,-1.500){2}{\rule{0.400pt}{0.361pt}}
\put(610.0,545.0){\rule[-0.200pt]{0.400pt}{0.723pt}}
\put(611.67,536){\rule{0.400pt}{2.409pt}}
\multiput(611.17,541.00)(1.000,-5.000){2}{\rule{0.400pt}{1.204pt}}
\put(612.67,536){\rule{0.400pt}{0.723pt}}
\multiput(612.17,536.00)(1.000,1.500){2}{\rule{0.400pt}{0.361pt}}
\put(612.0,543.0){\rule[-0.200pt]{0.400pt}{0.723pt}}
\put(613.67,529){\rule{0.400pt}{1.927pt}}
\multiput(613.17,533.00)(1.000,-4.000){2}{\rule{0.400pt}{0.964pt}}
\put(614.67,529){\rule{0.400pt}{0.723pt}}
\multiput(614.17,529.00)(1.000,1.500){2}{\rule{0.400pt}{0.361pt}}
\put(614.0,537.0){\rule[-0.200pt]{0.400pt}{0.482pt}}
\put(616.0,531.0){\usebox{\plotpoint}}
\put(616.67,528){\rule{0.400pt}{0.723pt}}
\multiput(616.17,529.50)(1.000,-1.500){2}{\rule{0.400pt}{0.361pt}}
\put(616.0,531.0){\usebox{\plotpoint}}
\put(617.67,523){\rule{0.400pt}{1.686pt}}
\multiput(617.17,526.50)(1.000,-3.500){2}{\rule{0.400pt}{0.843pt}}
\put(618.0,528.0){\rule[-0.200pt]{0.400pt}{0.482pt}}
\put(618.67,519){\rule{0.400pt}{1.204pt}}
\multiput(618.17,521.50)(1.000,-2.500){2}{\rule{0.400pt}{0.602pt}}
\put(619.67,516){\rule{0.400pt}{0.723pt}}
\multiput(619.17,517.50)(1.000,-1.500){2}{\rule{0.400pt}{0.361pt}}
\put(619.0,523.0){\usebox{\plotpoint}}
\put(620.67,515){\rule{0.400pt}{0.723pt}}
\multiput(620.17,516.50)(1.000,-1.500){2}{\rule{0.400pt}{0.361pt}}
\put(622,514.67){\rule{0.241pt}{0.400pt}}
\multiput(622.00,514.17)(0.500,1.000){2}{\rule{0.120pt}{0.400pt}}
\put(621.0,516.0){\rule[-0.200pt]{0.400pt}{0.482pt}}
\put(623.0,512.0){\rule[-0.200pt]{0.400pt}{0.964pt}}
\put(623.67,507){\rule{0.400pt}{1.204pt}}
\multiput(623.17,509.50)(1.000,-2.500){2}{\rule{0.400pt}{0.602pt}}
\put(623.0,512.0){\usebox{\plotpoint}}
\put(624.67,501){\rule{0.400pt}{1.204pt}}
\multiput(624.17,503.50)(1.000,-2.500){2}{\rule{0.400pt}{0.602pt}}
\put(625.67,501){\rule{0.400pt}{1.445pt}}
\multiput(625.17,501.00)(1.000,3.000){2}{\rule{0.400pt}{0.723pt}}
\put(625.0,506.0){\usebox{\plotpoint}}
\put(627,500.67){\rule{0.241pt}{0.400pt}}
\multiput(627.00,501.17)(0.500,-1.000){2}{\rule{0.120pt}{0.400pt}}
\put(627.67,497){\rule{0.400pt}{0.964pt}}
\multiput(627.17,499.00)(1.000,-2.000){2}{\rule{0.400pt}{0.482pt}}
\put(627.0,502.0){\rule[-0.200pt]{0.400pt}{1.204pt}}
\put(629,492.67){\rule{0.241pt}{0.400pt}}
\multiput(629.00,493.17)(0.500,-1.000){2}{\rule{0.120pt}{0.400pt}}
\put(629.0,494.0){\rule[-0.200pt]{0.400pt}{0.723pt}}
\put(630,493){\usebox{\plotpoint}}
\put(629.67,489){\rule{0.400pt}{0.964pt}}
\multiput(629.17,491.00)(1.000,-2.000){2}{\rule{0.400pt}{0.482pt}}
\put(630.67,486){\rule{0.400pt}{0.723pt}}
\multiput(630.17,487.50)(1.000,-1.500){2}{\rule{0.400pt}{0.361pt}}
\put(632,486){\usebox{\plotpoint}}
\put(631.67,484){\rule{0.400pt}{0.482pt}}
\multiput(631.17,485.00)(1.000,-1.000){2}{\rule{0.400pt}{0.241pt}}
\put(632.67,481){\rule{0.400pt}{0.723pt}}
\multiput(632.17,482.50)(1.000,-1.500){2}{\rule{0.400pt}{0.361pt}}
\put(634.0,479.0){\rule[-0.200pt]{0.400pt}{0.482pt}}
\put(635,477.67){\rule{0.241pt}{0.400pt}}
\multiput(635.00,478.17)(0.500,-1.000){2}{\rule{0.120pt}{0.400pt}}
\put(634.0,479.0){\usebox{\plotpoint}}
\put(636.0,472.0){\rule[-0.200pt]{0.400pt}{1.445pt}}
\put(636.0,472.0){\rule[-0.200pt]{0.482pt}{0.400pt}}
\put(637.67,467){\rule{0.400pt}{0.482pt}}
\multiput(637.17,468.00)(1.000,-1.000){2}{\rule{0.400pt}{0.241pt}}
\put(638.0,469.0){\rule[-0.200pt]{0.400pt}{0.723pt}}
\put(638.67,461){\rule{0.400pt}{0.964pt}}
\multiput(638.17,463.00)(1.000,-2.000){2}{\rule{0.400pt}{0.482pt}}
\put(639.0,465.0){\rule[-0.200pt]{0.400pt}{0.482pt}}
\put(640.0,461.0){\usebox{\plotpoint}}
\put(641,454.67){\rule{0.241pt}{0.400pt}}
\multiput(641.00,454.17)(0.500,1.000){2}{\rule{0.120pt}{0.400pt}}
\put(641.67,454){\rule{0.400pt}{0.482pt}}
\multiput(641.17,455.00)(1.000,-1.000){2}{\rule{0.400pt}{0.241pt}}
\put(641.0,455.0){\rule[-0.200pt]{0.400pt}{1.445pt}}
\put(642.67,448){\rule{0.400pt}{0.482pt}}
\multiput(642.17,449.00)(1.000,-1.000){2}{\rule{0.400pt}{0.241pt}}
\put(643.67,446){\rule{0.400pt}{0.482pt}}
\multiput(643.17,447.00)(1.000,-1.000){2}{\rule{0.400pt}{0.241pt}}
\put(643.0,450.0){\rule[-0.200pt]{0.400pt}{0.964pt}}
\put(644.67,443){\rule{0.400pt}{0.482pt}}
\multiput(644.17,444.00)(1.000,-1.000){2}{\rule{0.400pt}{0.241pt}}
\put(645.67,438){\rule{0.400pt}{1.204pt}}
\multiput(645.17,440.50)(1.000,-2.500){2}{\rule{0.400pt}{0.602pt}}
\put(645.0,445.0){\usebox{\plotpoint}}
\put(647,438){\usebox{\plotpoint}}
\put(646.67,436){\rule{0.400pt}{0.482pt}}
\multiput(646.17,437.00)(1.000,-1.000){2}{\rule{0.400pt}{0.241pt}}
\put(648.0,436.0){\usebox{\plotpoint}}
\put(649.0,430.0){\rule[-0.200pt]{0.400pt}{1.445pt}}
\put(649.0,430.0){\usebox{\plotpoint}}
\put(649.67,423){\rule{0.400pt}{0.964pt}}
\multiput(649.17,425.00)(1.000,-2.000){2}{\rule{0.400pt}{0.482pt}}
\put(650.67,423){\rule{0.400pt}{0.482pt}}
\multiput(650.17,423.00)(1.000,1.000){2}{\rule{0.400pt}{0.241pt}}
\put(650.0,427.0){\rule[-0.200pt]{0.400pt}{0.723pt}}
\put(652,425){\usebox{\plotpoint}}
\put(651.67,421){\rule{0.400pt}{0.964pt}}
\multiput(651.17,423.00)(1.000,-2.000){2}{\rule{0.400pt}{0.482pt}}
\put(653,419.67){\rule{0.241pt}{0.400pt}}
\multiput(653.00,420.17)(0.500,-1.000){2}{\rule{0.120pt}{0.400pt}}
\put(653.67,414){\rule{0.400pt}{0.482pt}}
\multiput(653.17,415.00)(1.000,-1.000){2}{\rule{0.400pt}{0.241pt}}
\put(654.67,410){\rule{0.400pt}{0.964pt}}
\multiput(654.17,412.00)(1.000,-2.000){2}{\rule{0.400pt}{0.482pt}}
\put(654.0,416.0){\rule[-0.200pt]{0.400pt}{0.964pt}}
\put(655.67,405){\rule{0.400pt}{0.964pt}}
\multiput(655.17,407.00)(1.000,-2.000){2}{\rule{0.400pt}{0.482pt}}
\put(657,403.67){\rule{0.241pt}{0.400pt}}
\multiput(657.00,404.17)(0.500,-1.000){2}{\rule{0.120pt}{0.400pt}}
\put(656.0,409.0){\usebox{\plotpoint}}
\put(657.67,400){\rule{0.400pt}{0.723pt}}
\multiput(657.17,401.50)(1.000,-1.500){2}{\rule{0.400pt}{0.361pt}}
\put(658.0,403.0){\usebox{\plotpoint}}
\put(658.67,397){\rule{0.400pt}{0.964pt}}
\multiput(658.17,399.00)(1.000,-2.000){2}{\rule{0.400pt}{0.482pt}}
\put(659.67,395){\rule{0.400pt}{0.482pt}}
\multiput(659.17,396.00)(1.000,-1.000){2}{\rule{0.400pt}{0.241pt}}
\put(659.0,400.0){\usebox{\plotpoint}}
\put(660.67,391){\rule{0.400pt}{0.723pt}}
\multiput(660.17,392.50)(1.000,-1.500){2}{\rule{0.400pt}{0.361pt}}
\put(661.0,394.0){\usebox{\plotpoint}}
\put(662.0,391.0){\usebox{\plotpoint}}
\put(662.67,386){\rule{0.400pt}{0.482pt}}
\multiput(662.17,387.00)(1.000,-1.000){2}{\rule{0.400pt}{0.241pt}}
\put(663.67,381){\rule{0.400pt}{1.204pt}}
\multiput(663.17,383.50)(1.000,-2.500){2}{\rule{0.400pt}{0.602pt}}
\put(663.0,388.0){\rule[-0.200pt]{0.400pt}{0.723pt}}
\put(664.67,380){\rule{0.400pt}{0.482pt}}
\multiput(664.17,381.00)(1.000,-1.000){2}{\rule{0.400pt}{0.241pt}}
\put(666,378.67){\rule{0.241pt}{0.400pt}}
\multiput(666.00,379.17)(0.500,-1.000){2}{\rule{0.120pt}{0.400pt}}
\put(665.0,381.0){\usebox{\plotpoint}}
\put(666.67,374){\rule{0.400pt}{0.482pt}}
\multiput(666.17,375.00)(1.000,-1.000){2}{\rule{0.400pt}{0.241pt}}
\put(667.0,376.0){\rule[-0.200pt]{0.400pt}{0.723pt}}
\put(667.67,370){\rule{0.400pt}{0.482pt}}
\multiput(667.17,371.00)(1.000,-1.000){2}{\rule{0.400pt}{0.241pt}}
\put(668.67,367){\rule{0.400pt}{0.723pt}}
\multiput(668.17,368.50)(1.000,-1.500){2}{\rule{0.400pt}{0.361pt}}
\put(668.0,372.0){\rule[-0.200pt]{0.400pt}{0.482pt}}
\put(669.67,365){\rule{0.400pt}{0.482pt}}
\multiput(669.17,365.00)(1.000,1.000){2}{\rule{0.400pt}{0.241pt}}
\put(670.67,363){\rule{0.400pt}{0.964pt}}
\multiput(670.17,365.00)(1.000,-2.000){2}{\rule{0.400pt}{0.482pt}}
\put(670.0,365.0){\rule[-0.200pt]{0.400pt}{0.482pt}}
\put(671.67,359){\rule{0.400pt}{0.723pt}}
\multiput(671.17,360.50)(1.000,-1.500){2}{\rule{0.400pt}{0.361pt}}
\put(672.0,362.0){\usebox{\plotpoint}}
\put(673.0,359.0){\usebox{\plotpoint}}
\put(674,355.67){\rule{0.241pt}{0.400pt}}
\multiput(674.00,355.17)(0.500,1.000){2}{\rule{0.120pt}{0.400pt}}
\put(674.67,354){\rule{0.400pt}{0.723pt}}
\multiput(674.17,355.50)(1.000,-1.500){2}{\rule{0.400pt}{0.361pt}}
\put(674.0,356.0){\rule[-0.200pt]{0.400pt}{0.723pt}}
\put(675.67,351){\rule{0.400pt}{0.482pt}}
\multiput(675.17,352.00)(1.000,-1.000){2}{\rule{0.400pt}{0.241pt}}
\put(676.0,353.0){\usebox{\plotpoint}}
\put(676.67,346){\rule{0.400pt}{0.723pt}}
\multiput(676.17,347.50)(1.000,-1.500){2}{\rule{0.400pt}{0.361pt}}
\put(677.67,344){\rule{0.400pt}{0.482pt}}
\multiput(677.17,345.00)(1.000,-1.000){2}{\rule{0.400pt}{0.241pt}}
\put(677.0,349.0){\rule[-0.200pt]{0.400pt}{0.482pt}}
\put(678.67,343){\rule{0.400pt}{0.482pt}}
\multiput(678.17,344.00)(1.000,-1.000){2}{\rule{0.400pt}{0.241pt}}
\put(680,341.67){\rule{0.241pt}{0.400pt}}
\multiput(680.00,342.17)(0.500,-1.000){2}{\rule{0.120pt}{0.400pt}}
\put(679.0,344.0){\usebox{\plotpoint}}
\put(681.0,338.0){\rule[-0.200pt]{0.400pt}{0.964pt}}
\put(681.67,335){\rule{0.400pt}{0.723pt}}
\multiput(681.17,336.50)(1.000,-1.500){2}{\rule{0.400pt}{0.361pt}}
\put(681.0,338.0){\usebox{\plotpoint}}
\put(683,335){\usebox{\plotpoint}}
\put(682.67,335){\rule{0.400pt}{0.482pt}}
\multiput(682.17,335.00)(1.000,1.000){2}{\rule{0.400pt}{0.241pt}}
\put(683.67,334){\rule{0.400pt}{0.723pt}}
\multiput(683.17,335.50)(1.000,-1.500){2}{\rule{0.400pt}{0.361pt}}
\put(685.0,331.0){\rule[-0.200pt]{0.400pt}{0.723pt}}
\put(685.0,331.0){\usebox{\plotpoint}}
\put(686,327.67){\rule{0.241pt}{0.400pt}}
\multiput(686.00,328.17)(0.500,-1.000){2}{\rule{0.120pt}{0.400pt}}
\put(687,326.67){\rule{0.241pt}{0.400pt}}
\multiput(687.00,327.17)(0.500,-1.000){2}{\rule{0.120pt}{0.400pt}}
\put(686.0,329.0){\rule[-0.200pt]{0.400pt}{0.482pt}}
\put(688.0,326.0){\usebox{\plotpoint}}
\put(688.67,323){\rule{0.400pt}{0.723pt}}
\multiput(688.17,324.50)(1.000,-1.500){2}{\rule{0.400pt}{0.361pt}}
\put(688.0,326.0){\usebox{\plotpoint}}
\put(690,323){\usebox{\plotpoint}}
\put(689.67,320){\rule{0.400pt}{0.723pt}}
\multiput(689.17,321.50)(1.000,-1.500){2}{\rule{0.400pt}{0.361pt}}
\put(691,319.67){\rule{0.241pt}{0.400pt}}
\multiput(691.00,319.17)(0.500,1.000){2}{\rule{0.120pt}{0.400pt}}
\put(692,321){\usebox{\plotpoint}}
\put(691.67,319){\rule{0.400pt}{0.482pt}}
\multiput(691.17,320.00)(1.000,-1.000){2}{\rule{0.400pt}{0.241pt}}
\put(692.67,317){\rule{0.400pt}{0.482pt}}
\multiput(692.17,318.00)(1.000,-1.000){2}{\rule{0.400pt}{0.241pt}}
\put(694,317){\usebox{\plotpoint}}
\put(694,315.67){\rule{0.241pt}{0.400pt}}
\multiput(694.00,316.17)(0.500,-1.000){2}{\rule{0.120pt}{0.400pt}}
\put(694.67,313){\rule{0.400pt}{0.964pt}}
\multiput(694.17,315.00)(1.000,-2.000){2}{\rule{0.400pt}{0.482pt}}
\put(696,312.67){\rule{0.241pt}{0.400pt}}
\multiput(696.00,312.17)(0.500,1.000){2}{\rule{0.120pt}{0.400pt}}
\put(695.0,316.0){\usebox{\plotpoint}}
\put(697,311.67){\rule{0.241pt}{0.400pt}}
\multiput(697.00,312.17)(0.500,-1.000){2}{\rule{0.120pt}{0.400pt}}
\put(697.67,310){\rule{0.400pt}{0.482pt}}
\multiput(697.17,311.00)(1.000,-1.000){2}{\rule{0.400pt}{0.241pt}}
\put(697.0,313.0){\usebox{\plotpoint}}
\put(699,310){\usebox{\plotpoint}}
\put(698.67,307){\rule{0.400pt}{0.723pt}}
\multiput(698.17,308.50)(1.000,-1.500){2}{\rule{0.400pt}{0.361pt}}
\put(700,306.67){\rule{0.241pt}{0.400pt}}
\multiput(700.00,306.17)(0.500,1.000){2}{\rule{0.120pt}{0.400pt}}
\put(700.67,307){\rule{0.400pt}{0.482pt}}
\multiput(700.17,308.00)(1.000,-1.000){2}{\rule{0.400pt}{0.241pt}}
\put(702,305.67){\rule{0.241pt}{0.400pt}}
\multiput(702.00,306.17)(0.500,-1.000){2}{\rule{0.120pt}{0.400pt}}
\put(701.0,308.0){\usebox{\plotpoint}}
\put(703,306){\usebox{\plotpoint}}
\put(703.0,306.0){\usebox{\plotpoint}}
\put(704,303.67){\rule{0.241pt}{0.400pt}}
\multiput(704.00,304.17)(0.500,-1.000){2}{\rule{0.120pt}{0.400pt}}
\put(705,302.67){\rule{0.241pt}{0.400pt}}
\multiput(705.00,303.17)(0.500,-1.000){2}{\rule{0.120pt}{0.400pt}}
\put(704.0,305.0){\usebox{\plotpoint}}
\put(706,303){\usebox{\plotpoint}}
\put(707,301.67){\rule{0.241pt}{0.400pt}}
\multiput(707.00,302.17)(0.500,-1.000){2}{\rule{0.120pt}{0.400pt}}
\put(706.0,303.0){\usebox{\plotpoint}}
\put(708,302){\usebox{\plotpoint}}
\put(709,300.67){\rule{0.241pt}{0.400pt}}
\multiput(709.00,301.17)(0.500,-1.000){2}{\rule{0.120pt}{0.400pt}}
\put(708.0,302.0){\usebox{\plotpoint}}
\put(710,298.67){\rule{0.241pt}{0.400pt}}
\multiput(710.00,298.17)(0.500,1.000){2}{\rule{0.120pt}{0.400pt}}
\put(710.0,299.0){\rule[-0.200pt]{0.400pt}{0.482pt}}
\put(711.0,299.0){\usebox{\plotpoint}}
\put(711.0,299.0){\rule[-0.200pt]{0.482pt}{0.400pt}}
\put(713,296.67){\rule{0.241pt}{0.400pt}}
\multiput(713.00,296.17)(0.500,1.000){2}{\rule{0.120pt}{0.400pt}}
\put(713.0,297.0){\rule[-0.200pt]{0.400pt}{0.482pt}}
\put(715,296.67){\rule{0.241pt}{0.400pt}}
\multiput(715.00,297.17)(0.500,-1.000){2}{\rule{0.120pt}{0.400pt}}
\put(716,296.67){\rule{0.241pt}{0.400pt}}
\multiput(716.00,296.17)(0.500,1.000){2}{\rule{0.120pt}{0.400pt}}
\put(714.0,298.0){\usebox{\plotpoint}}
\put(717,295.67){\rule{0.241pt}{0.400pt}}
\multiput(717.00,295.17)(0.500,1.000){2}{\rule{0.120pt}{0.400pt}}
\put(717.0,296.0){\rule[-0.200pt]{0.400pt}{0.482pt}}
\put(718.0,297.0){\usebox{\plotpoint}}
\put(718.67,296){\rule{0.400pt}{0.482pt}}
\multiput(718.17,296.00)(1.000,1.000){2}{\rule{0.400pt}{0.241pt}}
\put(719.0,296.0){\usebox{\plotpoint}}
\put(720,294.67){\rule{0.241pt}{0.400pt}}
\multiput(720.00,295.17)(0.500,-1.000){2}{\rule{0.120pt}{0.400pt}}
\put(721,294.67){\rule{0.241pt}{0.400pt}}
\multiput(721.00,294.17)(0.500,1.000){2}{\rule{0.120pt}{0.400pt}}
\put(720.0,296.0){\rule[-0.200pt]{0.400pt}{0.482pt}}
\put(722,296){\usebox{\plotpoint}}
\put(722,294.67){\rule{0.241pt}{0.400pt}}
\multiput(722.00,295.17)(0.500,-1.000){2}{\rule{0.120pt}{0.400pt}}
\put(723.0,295.0){\usebox{\plotpoint}}
\put(723.67,294){\rule{0.400pt}{0.723pt}}
\multiput(723.17,294.00)(1.000,1.500){2}{\rule{0.400pt}{0.361pt}}
\put(725,295.67){\rule{0.241pt}{0.400pt}}
\multiput(725.00,296.17)(0.500,-1.000){2}{\rule{0.120pt}{0.400pt}}
\put(724.0,294.0){\usebox{\plotpoint}}
\put(726,296){\usebox{\plotpoint}}
\put(726,295.67){\rule{0.241pt}{0.400pt}}
\multiput(726.00,295.17)(0.500,1.000){2}{\rule{0.120pt}{0.400pt}}
\put(727.0,295.0){\rule[-0.200pt]{0.400pt}{0.482pt}}
\put(727.67,295){\rule{0.400pt}{0.482pt}}
\multiput(727.17,295.00)(1.000,1.000){2}{\rule{0.400pt}{0.241pt}}
\put(727.0,295.0){\usebox{\plotpoint}}
\put(729,294.67){\rule{0.241pt}{0.400pt}}
\multiput(729.00,295.17)(0.500,-1.000){2}{\rule{0.120pt}{0.400pt}}
\put(730,294.67){\rule{0.241pt}{0.400pt}}
\multiput(730.00,294.17)(0.500,1.000){2}{\rule{0.120pt}{0.400pt}}
\put(729.0,296.0){\usebox{\plotpoint}}
\put(731.0,295.0){\usebox{\plotpoint}}
\put(731.0,295.0){\rule[-0.200pt]{0.482pt}{0.400pt}}
\put(733.0,295.0){\usebox{\plotpoint}}
\put(733.0,296.0){\rule[-0.200pt]{0.482pt}{0.400pt}}
\put(734.67,295){\rule{0.400pt}{0.482pt}}
\multiput(734.17,296.00)(1.000,-1.000){2}{\rule{0.400pt}{0.241pt}}
\put(735.0,296.0){\usebox{\plotpoint}}
\put(736.0,295.0){\rule[-0.200pt]{0.400pt}{0.482pt}}
\put(736.0,297.0){\rule[-0.200pt]{0.482pt}{0.400pt}}
\put(738,295.67){\rule{0.241pt}{0.400pt}}
\multiput(738.00,295.17)(0.500,1.000){2}{\rule{0.120pt}{0.400pt}}
\put(738.0,296.0){\usebox{\plotpoint}}
\put(739.0,297.0){\usebox{\plotpoint}}
\put(740.0,297.0){\usebox{\plotpoint}}
\put(740.0,298.0){\rule[-0.200pt]{0.482pt}{0.400pt}}
\put(742,297.67){\rule{0.241pt}{0.400pt}}
\multiput(742.00,298.17)(0.500,-1.000){2}{\rule{0.120pt}{0.400pt}}
\put(742.0,298.0){\usebox{\plotpoint}}
\put(743,298){\usebox{\plotpoint}}
\put(743.67,298){\rule{0.400pt}{0.482pt}}
\multiput(743.17,298.00)(1.000,1.000){2}{\rule{0.400pt}{0.241pt}}
\put(743.0,298.0){\usebox{\plotpoint}}
\put(745,300){\usebox{\plotpoint}}
\put(746,298.67){\rule{0.241pt}{0.400pt}}
\multiput(746.00,299.17)(0.500,-1.000){2}{\rule{0.120pt}{0.400pt}}
\put(745.0,300.0){\usebox{\plotpoint}}
\put(747,298.67){\rule{0.241pt}{0.400pt}}
\multiput(747.00,299.17)(0.500,-1.000){2}{\rule{0.120pt}{0.400pt}}
\put(747.67,299){\rule{0.400pt}{0.482pt}}
\multiput(747.17,299.00)(1.000,1.000){2}{\rule{0.400pt}{0.241pt}}
\put(747.0,299.0){\usebox{\plotpoint}}
\put(749,300.67){\rule{0.241pt}{0.400pt}}
\multiput(749.00,301.17)(0.500,-1.000){2}{\rule{0.120pt}{0.400pt}}
\put(749.0,301.0){\usebox{\plotpoint}}
\put(750.0,301.0){\usebox{\plotpoint}}
\put(750.0,302.0){\rule[-0.200pt]{0.482pt}{0.400pt}}
\put(751.67,301){\rule{0.400pt}{0.482pt}}
\multiput(751.17,301.00)(1.000,1.000){2}{\rule{0.400pt}{0.241pt}}
\put(753,301.67){\rule{0.241pt}{0.400pt}}
\multiput(753.00,302.17)(0.500,-1.000){2}{\rule{0.120pt}{0.400pt}}
\put(752.0,301.0){\usebox{\plotpoint}}
\put(753.67,303){\rule{0.400pt}{0.482pt}}
\multiput(753.17,303.00)(1.000,1.000){2}{\rule{0.400pt}{0.241pt}}
\put(755,303.67){\rule{0.241pt}{0.400pt}}
\multiput(755.00,304.17)(0.500,-1.000){2}{\rule{0.120pt}{0.400pt}}
\put(754.0,302.0){\usebox{\plotpoint}}
\put(755.67,303){\rule{0.400pt}{0.482pt}}
\multiput(755.17,304.00)(1.000,-1.000){2}{\rule{0.400pt}{0.241pt}}
\put(756.0,304.0){\usebox{\plotpoint}}
\put(757.0,303.0){\usebox{\plotpoint}}
\put(758,303.67){\rule{0.241pt}{0.400pt}}
\multiput(758.00,303.17)(0.500,1.000){2}{\rule{0.120pt}{0.400pt}}
\put(757.0,304.0){\usebox{\plotpoint}}
\put(759.0,305.0){\usebox{\plotpoint}}
\put(760,304.67){\rule{0.241pt}{0.400pt}}
\multiput(760.00,305.17)(0.500,-1.000){2}{\rule{0.120pt}{0.400pt}}
\put(759.0,306.0){\usebox{\plotpoint}}
\put(761.0,305.0){\rule[-0.200pt]{0.400pt}{0.482pt}}
\put(762,306.67){\rule{0.241pt}{0.400pt}}
\multiput(762.00,306.17)(0.500,1.000){2}{\rule{0.120pt}{0.400pt}}
\put(761.0,307.0){\usebox{\plotpoint}}
\put(763,308){\usebox{\plotpoint}}
\put(763.0,308.0){\rule[-0.200pt]{0.723pt}{0.400pt}}
\put(766,309.67){\rule{0.241pt}{0.400pt}}
\multiput(766.00,310.17)(0.500,-1.000){2}{\rule{0.120pt}{0.400pt}}
\put(766.67,310){\rule{0.400pt}{0.482pt}}
\multiput(766.17,310.00)(1.000,1.000){2}{\rule{0.400pt}{0.241pt}}
\put(766.0,308.0){\rule[-0.200pt]{0.400pt}{0.723pt}}
\put(768,310.67){\rule{0.241pt}{0.400pt}}
\multiput(768.00,310.17)(0.500,1.000){2}{\rule{0.120pt}{0.400pt}}
\put(768.0,311.0){\usebox{\plotpoint}}
\put(769.0,312.0){\usebox{\plotpoint}}
\put(770,311.67){\rule{0.241pt}{0.400pt}}
\multiput(770.00,312.17)(0.500,-1.000){2}{\rule{0.120pt}{0.400pt}}
\put(770.67,312){\rule{0.400pt}{0.482pt}}
\multiput(770.17,312.00)(1.000,1.000){2}{\rule{0.400pt}{0.241pt}}
\put(770.0,312.0){\usebox{\plotpoint}}
\put(772,311.67){\rule{0.241pt}{0.400pt}}
\multiput(772.00,312.17)(0.500,-1.000){2}{\rule{0.120pt}{0.400pt}}
\put(772.0,313.0){\usebox{\plotpoint}}
\put(773.0,312.0){\rule[-0.200pt]{0.400pt}{0.723pt}}
\put(773.0,315.0){\rule[-0.200pt]{0.482pt}{0.400pt}}
\put(775,315.67){\rule{0.241pt}{0.400pt}}
\multiput(775.00,316.17)(0.500,-1.000){2}{\rule{0.120pt}{0.400pt}}
\put(776,314.67){\rule{0.241pt}{0.400pt}}
\multiput(776.00,315.17)(0.500,-1.000){2}{\rule{0.120pt}{0.400pt}}
\put(775.0,315.0){\rule[-0.200pt]{0.400pt}{0.482pt}}
\put(777,316.67){\rule{0.241pt}{0.400pt}}
\multiput(777.00,317.17)(0.500,-1.000){2}{\rule{0.120pt}{0.400pt}}
\put(778,316.67){\rule{0.241pt}{0.400pt}}
\multiput(778.00,316.17)(0.500,1.000){2}{\rule{0.120pt}{0.400pt}}
\put(777.0,315.0){\rule[-0.200pt]{0.400pt}{0.723pt}}
\put(779,318){\usebox{\plotpoint}}
\put(778.67,318){\rule{0.400pt}{0.964pt}}
\multiput(778.17,318.00)(1.000,2.000){2}{\rule{0.400pt}{0.482pt}}
\put(780,319.67){\rule{0.241pt}{0.400pt}}
\multiput(780.00,319.17)(0.500,1.000){2}{\rule{0.120pt}{0.400pt}}
\put(780.67,321){\rule{0.400pt}{0.482pt}}
\multiput(780.17,321.00)(1.000,1.000){2}{\rule{0.400pt}{0.241pt}}
\put(780.0,320.0){\rule[-0.200pt]{0.400pt}{0.482pt}}
\put(782,323){\usebox{\plotpoint}}
\put(782.0,323.0){\rule[-0.200pt]{0.482pt}{0.400pt}}
\put(783.67,324){\rule{0.400pt}{0.482pt}}
\multiput(783.17,324.00)(1.000,1.000){2}{\rule{0.400pt}{0.241pt}}
\put(784.0,323.0){\usebox{\plotpoint}}
\put(785,326){\usebox{\plotpoint}}
\put(785,325.67){\rule{0.241pt}{0.400pt}}
\multiput(785.00,325.17)(0.500,1.000){2}{\rule{0.120pt}{0.400pt}}
\put(786,326.67){\rule{0.241pt}{0.400pt}}
\multiput(786.00,326.17)(0.500,1.000){2}{\rule{0.120pt}{0.400pt}}
\put(787,328){\usebox{\plotpoint}}
\put(787,327.67){\rule{0.241pt}{0.400pt}}
\multiput(787.00,327.17)(0.500,1.000){2}{\rule{0.120pt}{0.400pt}}
\put(787.67,329){\rule{0.400pt}{0.482pt}}
\multiput(787.17,329.00)(1.000,1.000){2}{\rule{0.400pt}{0.241pt}}
\put(789.0,331.0){\usebox{\plotpoint}}
\put(790,330.67){\rule{0.241pt}{0.400pt}}
\multiput(790.00,331.17)(0.500,-1.000){2}{\rule{0.120pt}{0.400pt}}
\put(789.0,332.0){\usebox{\plotpoint}}
\put(790.67,332){\rule{0.400pt}{0.482pt}}
\multiput(790.17,332.00)(1.000,1.000){2}{\rule{0.400pt}{0.241pt}}
\put(791.0,331.0){\usebox{\plotpoint}}
\put(792,334){\usebox{\plotpoint}}
\put(791.67,334){\rule{0.400pt}{0.723pt}}
\multiput(791.17,334.00)(1.000,1.500){2}{\rule{0.400pt}{0.361pt}}
\put(793,335.67){\rule{0.241pt}{0.400pt}}
\multiput(793.00,336.17)(0.500,-1.000){2}{\rule{0.120pt}{0.400pt}}
\put(794.0,336.0){\rule[-0.200pt]{0.400pt}{0.482pt}}
\put(794.67,338){\rule{0.400pt}{0.482pt}}
\multiput(794.17,338.00)(1.000,1.000){2}{\rule{0.400pt}{0.241pt}}
\put(794.0,338.0){\usebox{\plotpoint}}
\put(795.67,341){\rule{0.400pt}{0.482pt}}
\multiput(795.17,341.00)(1.000,1.000){2}{\rule{0.400pt}{0.241pt}}
\put(796.0,340.0){\usebox{\plotpoint}}
\put(797.0,343.0){\usebox{\plotpoint}}
\put(798.0,343.0){\usebox{\plotpoint}}
\put(798.0,344.0){\usebox{\plotpoint}}
\put(799.0,344.0){\usebox{\plotpoint}}
\put(799.67,345){\rule{0.400pt}{0.723pt}}
\multiput(799.17,345.00)(1.000,1.500){2}{\rule{0.400pt}{0.361pt}}
\put(799.0,345.0){\usebox{\plotpoint}}
\put(800.67,349){\rule{0.400pt}{0.723pt}}
\multiput(800.17,349.00)(1.000,1.500){2}{\rule{0.400pt}{0.361pt}}
\put(801.0,348.0){\usebox{\plotpoint}}
\put(802.0,352.0){\usebox{\plotpoint}}
\put(803,352.67){\rule{0.241pt}{0.400pt}}
\multiput(803.00,352.17)(0.500,1.000){2}{\rule{0.120pt}{0.400pt}}
\put(803.67,354){\rule{0.400pt}{0.482pt}}
\multiput(803.17,354.00)(1.000,1.000){2}{\rule{0.400pt}{0.241pt}}
\put(803.0,352.0){\usebox{\plotpoint}}
\put(805,356.67){\rule{0.241pt}{0.400pt}}
\multiput(805.00,357.17)(0.500,-1.000){2}{\rule{0.120pt}{0.400pt}}
\put(805.0,356.0){\rule[-0.200pt]{0.400pt}{0.482pt}}
\put(806,358.67){\rule{0.241pt}{0.400pt}}
\multiput(806.00,358.17)(0.500,1.000){2}{\rule{0.120pt}{0.400pt}}
\put(807,359.67){\rule{0.241pt}{0.400pt}}
\multiput(807.00,359.17)(0.500,1.000){2}{\rule{0.120pt}{0.400pt}}
\put(806.0,357.0){\rule[-0.200pt]{0.400pt}{0.482pt}}
\put(808,362.67){\rule{0.241pt}{0.400pt}}
\multiput(808.00,363.17)(0.500,-1.000){2}{\rule{0.120pt}{0.400pt}}
\put(809,362.67){\rule{0.241pt}{0.400pt}}
\multiput(809.00,362.17)(0.500,1.000){2}{\rule{0.120pt}{0.400pt}}
\put(808.0,361.0){\rule[-0.200pt]{0.400pt}{0.723pt}}
\put(809.67,366){\rule{0.400pt}{0.482pt}}
\multiput(809.17,366.00)(1.000,1.000){2}{\rule{0.400pt}{0.241pt}}
\put(811,367.67){\rule{0.241pt}{0.400pt}}
\multiput(811.00,367.17)(0.500,1.000){2}{\rule{0.120pt}{0.400pt}}
\put(810.0,364.0){\rule[-0.200pt]{0.400pt}{0.482pt}}
\put(811.67,371){\rule{0.400pt}{0.482pt}}
\multiput(811.17,371.00)(1.000,1.000){2}{\rule{0.400pt}{0.241pt}}
\put(812.0,369.0){\rule[-0.200pt]{0.400pt}{0.482pt}}
\put(813,373){\usebox{\plotpoint}}
\put(812.67,373){\rule{0.400pt}{0.482pt}}
\multiput(812.17,373.00)(1.000,1.000){2}{\rule{0.400pt}{0.241pt}}
\put(813.67,375){\rule{0.400pt}{0.723pt}}
\multiput(813.17,375.00)(1.000,1.500){2}{\rule{0.400pt}{0.361pt}}
\put(814.67,380){\rule{0.400pt}{0.482pt}}
\multiput(814.17,380.00)(1.000,1.000){2}{\rule{0.400pt}{0.241pt}}
\put(816,380.67){\rule{0.241pt}{0.400pt}}
\multiput(816.00,381.17)(0.500,-1.000){2}{\rule{0.120pt}{0.400pt}}
\put(815.0,378.0){\rule[-0.200pt]{0.400pt}{0.482pt}}
\put(816.67,383){\rule{0.400pt}{0.482pt}}
\multiput(816.17,383.00)(1.000,1.000){2}{\rule{0.400pt}{0.241pt}}
\put(817.0,381.0){\rule[-0.200pt]{0.400pt}{0.482pt}}
\put(818,386.67){\rule{0.241pt}{0.400pt}}
\multiput(818.00,386.17)(0.500,1.000){2}{\rule{0.120pt}{0.400pt}}
\put(818.67,388){\rule{0.400pt}{0.723pt}}
\multiput(818.17,388.00)(1.000,1.500){2}{\rule{0.400pt}{0.361pt}}
\put(818.0,385.0){\rule[-0.200pt]{0.400pt}{0.482pt}}
\put(820,391){\usebox{\plotpoint}}
\put(819.67,391){\rule{0.400pt}{0.482pt}}
\multiput(819.17,391.00)(1.000,1.000){2}{\rule{0.400pt}{0.241pt}}
\put(820.67,393){\rule{0.400pt}{0.482pt}}
\multiput(820.17,393.00)(1.000,1.000){2}{\rule{0.400pt}{0.241pt}}
\put(822,399.67){\rule{0.241pt}{0.400pt}}
\multiput(822.00,399.17)(0.500,1.000){2}{\rule{0.120pt}{0.400pt}}
\put(823,400.67){\rule{0.241pt}{0.400pt}}
\multiput(823.00,400.17)(0.500,1.000){2}{\rule{0.120pt}{0.400pt}}
\put(822.0,395.0){\rule[-0.200pt]{0.400pt}{1.204pt}}
\put(824,403.67){\rule{0.241pt}{0.400pt}}
\multiput(824.00,403.17)(0.500,1.000){2}{\rule{0.120pt}{0.400pt}}
\put(824.0,402.0){\rule[-0.200pt]{0.400pt}{0.482pt}}
\put(824.67,407){\rule{0.400pt}{0.723pt}}
\multiput(824.17,407.00)(1.000,1.500){2}{\rule{0.400pt}{0.361pt}}
\put(825.67,410){\rule{0.400pt}{0.723pt}}
\multiput(825.17,410.00)(1.000,1.500){2}{\rule{0.400pt}{0.361pt}}
\put(825.0,405.0){\rule[-0.200pt]{0.400pt}{0.482pt}}
\put(827,413){\usebox{\plotpoint}}
\put(826.67,413){\rule{0.400pt}{0.482pt}}
\multiput(826.17,413.00)(1.000,1.000){2}{\rule{0.400pt}{0.241pt}}
\put(827.67,415){\rule{0.400pt}{0.482pt}}
\multiput(827.17,415.00)(1.000,1.000){2}{\rule{0.400pt}{0.241pt}}
\put(829,419.67){\rule{0.241pt}{0.400pt}}
\multiput(829.00,419.17)(0.500,1.000){2}{\rule{0.120pt}{0.400pt}}
\put(829.0,417.0){\rule[-0.200pt]{0.400pt}{0.723pt}}
\put(830,424.67){\rule{0.241pt}{0.400pt}}
\multiput(830.00,424.17)(0.500,1.000){2}{\rule{0.120pt}{0.400pt}}
\put(830.67,426){\rule{0.400pt}{0.723pt}}
\multiput(830.17,426.00)(1.000,1.500){2}{\rule{0.400pt}{0.361pt}}
\put(830.0,421.0){\rule[-0.200pt]{0.400pt}{0.964pt}}
\put(832.0,429.0){\usebox{\plotpoint}}
\put(832.67,430){\rule{0.400pt}{1.445pt}}
\multiput(832.17,430.00)(1.000,3.000){2}{\rule{0.400pt}{0.723pt}}
\put(832.0,430.0){\usebox{\plotpoint}}
\put(833.67,437){\rule{0.400pt}{0.723pt}}
\multiput(833.17,437.00)(1.000,1.500){2}{\rule{0.400pt}{0.361pt}}
\put(834.67,440){\rule{0.400pt}{0.482pt}}
\multiput(834.17,440.00)(1.000,1.000){2}{\rule{0.400pt}{0.241pt}}
\put(834.0,436.0){\usebox{\plotpoint}}
\put(835.67,444){\rule{0.400pt}{0.482pt}}
\multiput(835.17,444.00)(1.000,1.000){2}{\rule{0.400pt}{0.241pt}}
\put(836.0,442.0){\rule[-0.200pt]{0.400pt}{0.482pt}}
\put(836.67,450){\rule{0.400pt}{0.482pt}}
\multiput(836.17,450.00)(1.000,1.000){2}{\rule{0.400pt}{0.241pt}}
\put(837.0,446.0){\rule[-0.200pt]{0.400pt}{0.964pt}}
\put(838.67,452){\rule{0.400pt}{1.686pt}}
\multiput(838.17,452.00)(1.000,3.500){2}{\rule{0.400pt}{0.843pt}}
\put(838.0,452.0){\usebox{\plotpoint}}
\put(840.0,459.0){\usebox{\plotpoint}}
\put(840.67,460){\rule{0.400pt}{0.964pt}}
\multiput(840.17,460.00)(1.000,2.000){2}{\rule{0.400pt}{0.482pt}}
\put(841.0,459.0){\usebox{\plotpoint}}
\put(841.67,468){\rule{0.400pt}{0.482pt}}
\multiput(841.17,468.00)(1.000,1.000){2}{\rule{0.400pt}{0.241pt}}
\put(842.67,470){\rule{0.400pt}{0.482pt}}
\multiput(842.17,470.00)(1.000,1.000){2}{\rule{0.400pt}{0.241pt}}
\put(842.0,464.0){\rule[-0.200pt]{0.400pt}{0.964pt}}
\put(844.0,472.0){\rule[-0.200pt]{0.400pt}{0.964pt}}
\put(844.67,476){\rule{0.400pt}{0.723pt}}
\multiput(844.17,476.00)(1.000,1.500){2}{\rule{0.400pt}{0.361pt}}
\put(844.0,476.0){\usebox{\plotpoint}}
\put(845.67,482){\rule{0.400pt}{0.482pt}}
\multiput(845.17,482.00)(1.000,1.000){2}{\rule{0.400pt}{0.241pt}}
\put(846.67,484){\rule{0.400pt}{1.204pt}}
\multiput(846.17,484.00)(1.000,2.500){2}{\rule{0.400pt}{0.602pt}}
\put(846.0,479.0){\rule[-0.200pt]{0.400pt}{0.723pt}}
\put(848,491.67){\rule{0.241pt}{0.400pt}}
\multiput(848.00,491.17)(0.500,1.000){2}{\rule{0.120pt}{0.400pt}}
\put(848.0,489.0){\rule[-0.200pt]{0.400pt}{0.723pt}}
\put(848.67,496){\rule{0.400pt}{0.482pt}}
\multiput(848.17,496.00)(1.000,1.000){2}{\rule{0.400pt}{0.241pt}}
\put(849.67,498){\rule{0.400pt}{0.723pt}}
\multiput(849.17,498.00)(1.000,1.500){2}{\rule{0.400pt}{0.361pt}}
\put(849.0,493.0){\rule[-0.200pt]{0.400pt}{0.723pt}}
\put(850.67,504){\rule{0.400pt}{0.482pt}}
\multiput(850.17,504.00)(1.000,1.000){2}{\rule{0.400pt}{0.241pt}}
\put(851.0,501.0){\rule[-0.200pt]{0.400pt}{0.723pt}}
\put(852.0,506.0){\usebox{\plotpoint}}
\put(853,511.67){\rule{0.241pt}{0.400pt}}
\multiput(853.00,511.17)(0.500,1.000){2}{\rule{0.120pt}{0.400pt}}
\put(853.0,506.0){\rule[-0.200pt]{0.400pt}{1.445pt}}
\put(853.67,515){\rule{0.400pt}{0.964pt}}
\multiput(853.17,515.00)(1.000,2.000){2}{\rule{0.400pt}{0.482pt}}
\put(854.67,519){\rule{0.400pt}{0.723pt}}
\multiput(854.17,519.00)(1.000,1.500){2}{\rule{0.400pt}{0.361pt}}
\put(854.0,513.0){\rule[-0.200pt]{0.400pt}{0.482pt}}
\put(856,522){\usebox{\plotpoint}}
\put(855.67,522){\rule{0.400pt}{1.445pt}}
\multiput(855.17,522.00)(1.000,3.000){2}{\rule{0.400pt}{0.723pt}}
\put(856.67,525){\rule{0.400pt}{0.723pt}}
\multiput(856.17,526.50)(1.000,-1.500){2}{\rule{0.400pt}{0.361pt}}
\put(857.67,532){\rule{0.400pt}{0.723pt}}
\multiput(857.17,532.00)(1.000,1.500){2}{\rule{0.400pt}{0.361pt}}
\put(859,534.67){\rule{0.241pt}{0.400pt}}
\multiput(859.00,534.17)(0.500,1.000){2}{\rule{0.120pt}{0.400pt}}
\put(858.0,525.0){\rule[-0.200pt]{0.400pt}{1.686pt}}
\put(859.67,541){\rule{0.400pt}{0.723pt}}
\multiput(859.17,542.50)(1.000,-1.500){2}{\rule{0.400pt}{0.361pt}}
\put(860.0,536.0){\rule[-0.200pt]{0.400pt}{1.927pt}}
\put(860.67,543){\rule{0.400pt}{0.723pt}}
\multiput(860.17,543.00)(1.000,1.500){2}{\rule{0.400pt}{0.361pt}}
\put(861.67,546){\rule{0.400pt}{1.445pt}}
\multiput(861.17,546.00)(1.000,3.000){2}{\rule{0.400pt}{0.723pt}}
\put(861.0,541.0){\rule[-0.200pt]{0.400pt}{0.482pt}}
\put(863.0,552.0){\rule[-0.200pt]{0.400pt}{0.723pt}}
\put(863.67,555){\rule{0.400pt}{0.482pt}}
\multiput(863.17,555.00)(1.000,1.000){2}{\rule{0.400pt}{0.241pt}}
\put(863.0,555.0){\usebox{\plotpoint}}
\put(865.0,557.0){\rule[-0.200pt]{0.400pt}{0.964pt}}
\put(865.0,561.0){\usebox{\plotpoint}}
\put(865.67,564){\rule{0.400pt}{0.482pt}}
\multiput(865.17,564.00)(1.000,1.000){2}{\rule{0.400pt}{0.241pt}}
\put(866.67,566){\rule{0.400pt}{1.445pt}}
\multiput(866.17,566.00)(1.000,3.000){2}{\rule{0.400pt}{0.723pt}}
\put(866.0,561.0){\rule[-0.200pt]{0.400pt}{0.723pt}}
\put(867.67,573){\rule{0.400pt}{0.482pt}}
\multiput(867.17,573.00)(1.000,1.000){2}{\rule{0.400pt}{0.241pt}}
\put(868.67,575){\rule{0.400pt}{0.723pt}}
\multiput(868.17,575.00)(1.000,1.500){2}{\rule{0.400pt}{0.361pt}}
\put(868.0,572.0){\usebox{\plotpoint}}
\put(870,578.67){\rule{0.241pt}{0.400pt}}
\multiput(870.00,579.17)(0.500,-1.000){2}{\rule{0.120pt}{0.400pt}}
\put(870.0,578.0){\rule[-0.200pt]{0.400pt}{0.482pt}}
\put(870.67,582){\rule{0.400pt}{0.482pt}}
\multiput(870.17,583.00)(1.000,-1.000){2}{\rule{0.400pt}{0.241pt}}
\put(871.67,582){\rule{0.400pt}{1.204pt}}
\multiput(871.17,582.00)(1.000,2.500){2}{\rule{0.400pt}{0.602pt}}
\put(871.0,579.0){\rule[-0.200pt]{0.400pt}{1.204pt}}
\put(872.67,590){\rule{0.400pt}{0.723pt}}
\multiput(872.17,590.00)(1.000,1.500){2}{\rule{0.400pt}{0.361pt}}
\put(874,592.67){\rule{0.241pt}{0.400pt}}
\multiput(874.00,592.17)(0.500,1.000){2}{\rule{0.120pt}{0.400pt}}
\put(873.0,587.0){\rule[-0.200pt]{0.400pt}{0.723pt}}
\put(874.67,596){\rule{0.400pt}{0.723pt}}
\multiput(874.17,596.00)(1.000,1.500){2}{\rule{0.400pt}{0.361pt}}
\put(876,597.67){\rule{0.241pt}{0.400pt}}
\multiput(876.00,598.17)(0.500,-1.000){2}{\rule{0.120pt}{0.400pt}}
\put(875.0,594.0){\rule[-0.200pt]{0.400pt}{0.482pt}}
\put(876.67,602){\rule{0.400pt}{0.964pt}}
\multiput(876.17,602.00)(1.000,2.000){2}{\rule{0.400pt}{0.482pt}}
\put(877.0,598.0){\rule[-0.200pt]{0.400pt}{0.964pt}}
\put(877.67,607){\rule{0.400pt}{0.482pt}}
\multiput(877.17,608.00)(1.000,-1.000){2}{\rule{0.400pt}{0.241pt}}
\put(878.67,607){\rule{0.400pt}{0.482pt}}
\multiput(878.17,607.00)(1.000,1.000){2}{\rule{0.400pt}{0.241pt}}
\put(878.0,606.0){\rule[-0.200pt]{0.400pt}{0.723pt}}
\put(879.67,610){\rule{0.400pt}{1.686pt}}
\multiput(879.17,610.00)(1.000,3.500){2}{\rule{0.400pt}{0.843pt}}
\put(880.67,614){\rule{0.400pt}{0.723pt}}
\multiput(880.17,615.50)(1.000,-1.500){2}{\rule{0.400pt}{0.361pt}}
\put(880.0,609.0){\usebox{\plotpoint}}
\put(882.0,614.0){\rule[-0.200pt]{0.400pt}{0.482pt}}
\put(882.0,616.0){\usebox{\plotpoint}}
\put(882.67,619){\rule{0.400pt}{0.482pt}}
\multiput(882.17,619.00)(1.000,1.000){2}{\rule{0.400pt}{0.241pt}}
\put(883.67,618){\rule{0.400pt}{0.723pt}}
\multiput(883.17,619.50)(1.000,-1.500){2}{\rule{0.400pt}{0.361pt}}
\put(883.0,616.0){\rule[-0.200pt]{0.400pt}{0.723pt}}
\put(884.67,622){\rule{0.400pt}{1.445pt}}
\multiput(884.17,625.00)(1.000,-3.000){2}{\rule{0.400pt}{0.723pt}}
\put(885.67,622){\rule{0.400pt}{1.686pt}}
\multiput(885.17,622.00)(1.000,3.500){2}{\rule{0.400pt}{0.843pt}}
\put(885.0,618.0){\rule[-0.200pt]{0.400pt}{2.409pt}}
\put(887,628.67){\rule{0.241pt}{0.400pt}}
\multiput(887.00,629.17)(0.500,-1.000){2}{\rule{0.120pt}{0.400pt}}
\put(887.0,629.0){\usebox{\plotpoint}}
\put(887.67,627){\rule{0.400pt}{0.964pt}}
\multiput(887.17,627.00)(1.000,2.000){2}{\rule{0.400pt}{0.482pt}}
\put(888.67,631){\rule{0.400pt}{0.482pt}}
\multiput(888.17,631.00)(1.000,1.000){2}{\rule{0.400pt}{0.241pt}}
\put(888.0,627.0){\rule[-0.200pt]{0.400pt}{0.482pt}}
\put(890,633.67){\rule{0.241pt}{0.400pt}}
\multiput(890.00,633.17)(0.500,1.000){2}{\rule{0.120pt}{0.400pt}}
\put(890.0,633.0){\usebox{\plotpoint}}
\put(891.67,635){\rule{0.400pt}{0.482pt}}
\multiput(891.17,635.00)(1.000,1.000){2}{\rule{0.400pt}{0.241pt}}
\put(891.0,635.0){\usebox{\plotpoint}}
\put(892.67,634){\rule{0.400pt}{1.686pt}}
\multiput(892.17,634.00)(1.000,3.500){2}{\rule{0.400pt}{0.843pt}}
\put(893.67,635){\rule{0.400pt}{1.445pt}}
\multiput(893.17,638.00)(1.000,-3.000){2}{\rule{0.400pt}{0.723pt}}
\put(893.0,634.0){\rule[-0.200pt]{0.400pt}{0.723pt}}
\put(894.67,637){\rule{0.400pt}{1.204pt}}
\multiput(894.17,637.00)(1.000,2.500){2}{\rule{0.400pt}{0.602pt}}
\put(895.67,639){\rule{0.400pt}{0.723pt}}
\multiput(895.17,640.50)(1.000,-1.500){2}{\rule{0.400pt}{0.361pt}}
\put(895.0,635.0){\rule[-0.200pt]{0.400pt}{0.482pt}}
\put(896.67,640){\rule{0.400pt}{0.723pt}}
\multiput(896.17,640.00)(1.000,1.500){2}{\rule{0.400pt}{0.361pt}}
\put(897.67,641){\rule{0.400pt}{0.482pt}}
\multiput(897.17,642.00)(1.000,-1.000){2}{\rule{0.400pt}{0.241pt}}
\put(897.0,639.0){\usebox{\plotpoint}}
\put(898.67,636){\rule{0.400pt}{0.964pt}}
\multiput(898.17,638.00)(1.000,-2.000){2}{\rule{0.400pt}{0.482pt}}
\put(899.0,640.0){\usebox{\plotpoint}}
\put(900.0,636.0){\rule[-0.200pt]{0.400pt}{1.204pt}}
\put(900.67,633){\rule{0.400pt}{1.927pt}}
\multiput(900.17,637.00)(1.000,-4.000){2}{\rule{0.400pt}{0.964pt}}
\put(900.0,641.0){\usebox{\plotpoint}}
\put(901.67,638){\rule{0.400pt}{0.482pt}}
\multiput(901.17,638.00)(1.000,1.000){2}{\rule{0.400pt}{0.241pt}}
\put(902.67,636){\rule{0.400pt}{0.964pt}}
\multiput(902.17,638.00)(1.000,-2.000){2}{\rule{0.400pt}{0.482pt}}
\put(902.0,633.0){\rule[-0.200pt]{0.400pt}{1.204pt}}
\put(903.67,634){\rule{0.400pt}{0.482pt}}
\multiput(903.17,634.00)(1.000,1.000){2}{\rule{0.400pt}{0.241pt}}
\put(904.0,634.0){\rule[-0.200pt]{0.400pt}{0.482pt}}
\put(904.67,631){\rule{0.400pt}{1.445pt}}
\multiput(904.17,634.00)(1.000,-3.000){2}{\rule{0.400pt}{0.723pt}}
\put(906,630.67){\rule{0.241pt}{0.400pt}}
\multiput(906.00,630.17)(0.500,1.000){2}{\rule{0.120pt}{0.400pt}}
\put(905.0,636.0){\usebox{\plotpoint}}
\put(907,629.67){\rule{0.241pt}{0.400pt}}
\multiput(907.00,629.17)(0.500,1.000){2}{\rule{0.120pt}{0.400pt}}
\put(907.67,631){\rule{0.400pt}{0.482pt}}
\multiput(907.17,631.00)(1.000,1.000){2}{\rule{0.400pt}{0.241pt}}
\put(907.0,630.0){\rule[-0.200pt]{0.400pt}{0.482pt}}
\put(909,627.67){\rule{0.241pt}{0.400pt}}
\multiput(909.00,627.17)(0.500,1.000){2}{\rule{0.120pt}{0.400pt}}
\put(909.0,628.0){\rule[-0.200pt]{0.400pt}{1.204pt}}
\put(909.67,628){\rule{0.400pt}{0.482pt}}
\multiput(909.17,629.00)(1.000,-1.000){2}{\rule{0.400pt}{0.241pt}}
\put(910.67,625){\rule{0.400pt}{0.723pt}}
\multiput(910.17,626.50)(1.000,-1.500){2}{\rule{0.400pt}{0.361pt}}
\put(910.0,629.0){\usebox{\plotpoint}}
\put(911.67,616){\rule{0.400pt}{1.927pt}}
\multiput(911.17,616.00)(1.000,4.000){2}{\rule{0.400pt}{0.964pt}}
\put(912.67,618){\rule{0.400pt}{1.445pt}}
\multiput(912.17,621.00)(1.000,-3.000){2}{\rule{0.400pt}{0.723pt}}
\put(912.0,616.0){\rule[-0.200pt]{0.400pt}{2.168pt}}
\put(913.67,613){\rule{0.400pt}{1.686pt}}
\multiput(913.17,616.50)(1.000,-3.500){2}{\rule{0.400pt}{0.843pt}}
\put(914.0,618.0){\rule[-0.200pt]{0.400pt}{0.482pt}}
\put(914.67,611){\rule{0.400pt}{0.964pt}}
\multiput(914.17,613.00)(1.000,-2.000){2}{\rule{0.400pt}{0.482pt}}
\put(915.67,611){\rule{0.400pt}{0.723pt}}
\multiput(915.17,611.00)(1.000,1.500){2}{\rule{0.400pt}{0.361pt}}
\put(915.0,613.0){\rule[-0.200pt]{0.400pt}{0.482pt}}
\put(917,608.67){\rule{0.241pt}{0.400pt}}
\multiput(917.00,609.17)(0.500,-1.000){2}{\rule{0.120pt}{0.400pt}}
\put(917.67,600){\rule{0.400pt}{2.168pt}}
\multiput(917.17,604.50)(1.000,-4.500){2}{\rule{0.400pt}{1.084pt}}
\put(917.0,610.0){\rule[-0.200pt]{0.400pt}{0.964pt}}
\put(918.67,602){\rule{0.400pt}{0.964pt}}
\multiput(918.17,604.00)(1.000,-2.000){2}{\rule{0.400pt}{0.482pt}}
\put(919.0,600.0){\rule[-0.200pt]{0.400pt}{1.445pt}}
\put(919.67,595){\rule{0.400pt}{0.964pt}}
\multiput(919.17,597.00)(1.000,-2.000){2}{\rule{0.400pt}{0.482pt}}
\put(921,593.67){\rule{0.241pt}{0.400pt}}
\multiput(921.00,594.17)(0.500,-1.000){2}{\rule{0.120pt}{0.400pt}}
\put(920.0,599.0){\rule[-0.200pt]{0.400pt}{0.723pt}}
\put(921.67,587){\rule{0.400pt}{1.927pt}}
\multiput(921.17,591.00)(1.000,-4.000){2}{\rule{0.400pt}{0.964pt}}
\put(923,585.67){\rule{0.241pt}{0.400pt}}
\multiput(923.00,586.17)(0.500,-1.000){2}{\rule{0.120pt}{0.400pt}}
\put(922.0,594.0){\usebox{\plotpoint}}
\put(923.67,581){\rule{0.400pt}{0.723pt}}
\multiput(923.17,582.50)(1.000,-1.500){2}{\rule{0.400pt}{0.361pt}}
\put(924.0,584.0){\rule[-0.200pt]{0.400pt}{0.482pt}}
\put(924.67,575){\rule{0.400pt}{1.204pt}}
\multiput(924.17,577.50)(1.000,-2.500){2}{\rule{0.400pt}{0.602pt}}
\put(925.67,572){\rule{0.400pt}{0.723pt}}
\multiput(925.17,573.50)(1.000,-1.500){2}{\rule{0.400pt}{0.361pt}}
\put(925.0,580.0){\usebox{\plotpoint}}
\put(927,567.67){\rule{0.241pt}{0.400pt}}
\multiput(927.00,568.17)(0.500,-1.000){2}{\rule{0.120pt}{0.400pt}}
\put(927.67,564){\rule{0.400pt}{0.964pt}}
\multiput(927.17,566.00)(1.000,-2.000){2}{\rule{0.400pt}{0.482pt}}
\put(927.0,569.0){\rule[-0.200pt]{0.400pt}{0.723pt}}
\put(929,555.67){\rule{0.241pt}{0.400pt}}
\multiput(929.00,555.17)(0.500,1.000){2}{\rule{0.120pt}{0.400pt}}
\put(929.0,556.0){\rule[-0.200pt]{0.400pt}{1.927pt}}
\put(929.67,551){\rule{0.400pt}{0.964pt}}
\multiput(929.17,553.00)(1.000,-2.000){2}{\rule{0.400pt}{0.482pt}}
\put(930.67,549){\rule{0.400pt}{0.482pt}}
\multiput(930.17,550.00)(1.000,-1.000){2}{\rule{0.400pt}{0.241pt}}
\put(930.0,555.0){\rule[-0.200pt]{0.400pt}{0.482pt}}
\put(931.67,542){\rule{0.400pt}{0.723pt}}
\multiput(931.17,543.50)(1.000,-1.500){2}{\rule{0.400pt}{0.361pt}}
\put(932.67,538){\rule{0.400pt}{0.964pt}}
\multiput(932.17,540.00)(1.000,-2.000){2}{\rule{0.400pt}{0.482pt}}
\put(932.0,545.0){\rule[-0.200pt]{0.400pt}{0.964pt}}
\put(934.0,533.0){\rule[-0.200pt]{0.400pt}{1.204pt}}
\put(934.0,533.0){\usebox{\plotpoint}}
\put(935,525.67){\rule{0.241pt}{0.400pt}}
\multiput(935.00,526.17)(0.500,-1.000){2}{\rule{0.120pt}{0.400pt}}
\put(935.67,521){\rule{0.400pt}{1.204pt}}
\multiput(935.17,523.50)(1.000,-2.500){2}{\rule{0.400pt}{0.602pt}}
\put(935.0,527.0){\rule[-0.200pt]{0.400pt}{1.445pt}}
\put(936.67,516){\rule{0.400pt}{0.482pt}}
\multiput(936.17,517.00)(1.000,-1.000){2}{\rule{0.400pt}{0.241pt}}
\put(937.67,509){\rule{0.400pt}{1.686pt}}
\multiput(937.17,512.50)(1.000,-3.500){2}{\rule{0.400pt}{0.843pt}}
\put(937.0,518.0){\rule[-0.200pt]{0.400pt}{0.723pt}}
\put(938.67,502){\rule{0.400pt}{1.204pt}}
\multiput(938.17,504.50)(1.000,-2.500){2}{\rule{0.400pt}{0.602pt}}
\put(939.0,507.0){\rule[-0.200pt]{0.400pt}{0.482pt}}
\put(940,502){\usebox{\plotpoint}}
\put(939.67,497){\rule{0.400pt}{1.204pt}}
\multiput(939.17,499.50)(1.000,-2.500){2}{\rule{0.400pt}{0.602pt}}
\put(940.67,493){\rule{0.400pt}{0.964pt}}
\multiput(940.17,495.00)(1.000,-2.000){2}{\rule{0.400pt}{0.482pt}}
\put(941.67,483){\rule{0.400pt}{1.686pt}}
\multiput(941.17,486.50)(1.000,-3.500){2}{\rule{0.400pt}{0.843pt}}
\put(942.0,490.0){\rule[-0.200pt]{0.400pt}{0.723pt}}
\put(943.0,483.0){\usebox{\plotpoint}}
\put(944,476.67){\rule{0.241pt}{0.400pt}}
\multiput(944.00,477.17)(0.500,-1.000){2}{\rule{0.120pt}{0.400pt}}
\put(944.0,478.0){\rule[-0.200pt]{0.400pt}{1.204pt}}
\put(944.67,469){\rule{0.400pt}{0.723pt}}
\multiput(944.17,470.50)(1.000,-1.500){2}{\rule{0.400pt}{0.361pt}}
\put(945.67,465){\rule{0.400pt}{0.964pt}}
\multiput(945.17,467.00)(1.000,-2.000){2}{\rule{0.400pt}{0.482pt}}
\put(945.0,472.0){\rule[-0.200pt]{0.400pt}{1.204pt}}
\put(947,457.67){\rule{0.241pt}{0.400pt}}
\multiput(947.00,458.17)(0.500,-1.000){2}{\rule{0.120pt}{0.400pt}}
\put(947.67,452){\rule{0.400pt}{1.445pt}}
\multiput(947.17,455.00)(1.000,-3.000){2}{\rule{0.400pt}{0.723pt}}
\put(947.0,459.0){\rule[-0.200pt]{0.400pt}{1.445pt}}
\put(948.67,447){\rule{0.400pt}{0.723pt}}
\multiput(948.17,448.50)(1.000,-1.500){2}{\rule{0.400pt}{0.361pt}}
\put(949.0,450.0){\rule[-0.200pt]{0.400pt}{0.482pt}}
\put(949.67,441){\rule{0.400pt}{0.482pt}}
\multiput(949.17,442.00)(1.000,-1.000){2}{\rule{0.400pt}{0.241pt}}
\put(950.67,437){\rule{0.400pt}{0.964pt}}
\multiput(950.17,439.00)(1.000,-2.000){2}{\rule{0.400pt}{0.482pt}}
\put(950.0,443.0){\rule[-0.200pt]{0.400pt}{0.964pt}}
\put(951.67,428){\rule{0.400pt}{1.445pt}}
\multiput(951.17,431.00)(1.000,-3.000){2}{\rule{0.400pt}{0.723pt}}
\put(952.67,424){\rule{0.400pt}{0.964pt}}
\multiput(952.17,426.00)(1.000,-2.000){2}{\rule{0.400pt}{0.482pt}}
\put(952.0,434.0){\rule[-0.200pt]{0.400pt}{0.723pt}}
\put(953.67,416){\rule{0.400pt}{0.964pt}}
\multiput(953.17,418.00)(1.000,-2.000){2}{\rule{0.400pt}{0.482pt}}
\put(954.0,420.0){\rule[-0.200pt]{0.400pt}{0.964pt}}
\put(954.67,409){\rule{0.400pt}{1.204pt}}
\multiput(954.17,411.50)(1.000,-2.500){2}{\rule{0.400pt}{0.602pt}}
\put(955.67,407){\rule{0.400pt}{0.482pt}}
\multiput(955.17,408.00)(1.000,-1.000){2}{\rule{0.400pt}{0.241pt}}
\put(955.0,414.0){\rule[-0.200pt]{0.400pt}{0.482pt}}
\put(957,401.67){\rule{0.241pt}{0.400pt}}
\multiput(957.00,402.17)(0.500,-1.000){2}{\rule{0.120pt}{0.400pt}}
\put(957.67,394){\rule{0.400pt}{1.927pt}}
\multiput(957.17,398.00)(1.000,-4.000){2}{\rule{0.400pt}{0.964pt}}
\put(957.0,403.0){\rule[-0.200pt]{0.400pt}{0.964pt}}
\put(958.67,389){\rule{0.400pt}{0.482pt}}
\multiput(958.17,390.00)(1.000,-1.000){2}{\rule{0.400pt}{0.241pt}}
\put(959.0,391.0){\rule[-0.200pt]{0.400pt}{0.723pt}}
\put(959.67,381){\rule{0.400pt}{0.964pt}}
\multiput(959.17,383.00)(1.000,-2.000){2}{\rule{0.400pt}{0.482pt}}
\put(960.0,385.0){\rule[-0.200pt]{0.400pt}{0.964pt}}
\put(961.0,381.0){\usebox{\plotpoint}}
\put(961.67,372){\rule{0.400pt}{0.964pt}}
\multiput(961.17,374.00)(1.000,-2.000){2}{\rule{0.400pt}{0.482pt}}
\put(962.67,369){\rule{0.400pt}{0.723pt}}
\multiput(962.17,370.50)(1.000,-1.500){2}{\rule{0.400pt}{0.361pt}}
\put(962.0,376.0){\rule[-0.200pt]{0.400pt}{1.204pt}}
\put(963.67,362){\rule{0.400pt}{0.723pt}}
\multiput(963.17,363.50)(1.000,-1.500){2}{\rule{0.400pt}{0.361pt}}
\put(964.0,365.0){\rule[-0.200pt]{0.400pt}{0.964pt}}
\put(964.67,355){\rule{0.400pt}{0.964pt}}
\multiput(964.17,357.00)(1.000,-2.000){2}{\rule{0.400pt}{0.482pt}}
\put(965.67,352){\rule{0.400pt}{0.723pt}}
\multiput(965.17,353.50)(1.000,-1.500){2}{\rule{0.400pt}{0.361pt}}
\put(965.0,359.0){\rule[-0.200pt]{0.400pt}{0.723pt}}
\put(967,345.67){\rule{0.241pt}{0.400pt}}
\multiput(967.00,346.17)(0.500,-1.000){2}{\rule{0.120pt}{0.400pt}}
\put(967.0,347.0){\rule[-0.200pt]{0.400pt}{1.204pt}}
\put(967.67,338){\rule{0.400pt}{0.723pt}}
\multiput(967.17,339.50)(1.000,-1.500){2}{\rule{0.400pt}{0.361pt}}
\put(968.67,336){\rule{0.400pt}{0.482pt}}
\multiput(968.17,337.00)(1.000,-1.000){2}{\rule{0.400pt}{0.241pt}}
\put(968.0,341.0){\rule[-0.200pt]{0.400pt}{1.204pt}}
\put(969.67,330){\rule{0.400pt}{0.723pt}}
\multiput(969.17,331.50)(1.000,-1.500){2}{\rule{0.400pt}{0.361pt}}
\put(970.67,328){\rule{0.400pt}{0.482pt}}
\multiput(970.17,329.00)(1.000,-1.000){2}{\rule{0.400pt}{0.241pt}}
\put(970.0,333.0){\rule[-0.200pt]{0.400pt}{0.723pt}}
\put(971.67,319){\rule{0.400pt}{1.204pt}}
\multiput(971.17,321.50)(1.000,-2.500){2}{\rule{0.400pt}{0.602pt}}
\put(972.0,324.0){\rule[-0.200pt]{0.400pt}{0.964pt}}
\put(972.67,315){\rule{0.400pt}{0.723pt}}
\multiput(972.17,316.50)(1.000,-1.500){2}{\rule{0.400pt}{0.361pt}}
\put(973.67,312){\rule{0.400pt}{0.723pt}}
\multiput(973.17,313.50)(1.000,-1.500){2}{\rule{0.400pt}{0.361pt}}
\put(973.0,318.0){\usebox{\plotpoint}}
\put(974.67,305){\rule{0.400pt}{1.204pt}}
\multiput(974.17,307.50)(1.000,-2.500){2}{\rule{0.400pt}{0.602pt}}
\put(975.67,303){\rule{0.400pt}{0.482pt}}
\multiput(975.17,304.00)(1.000,-1.000){2}{\rule{0.400pt}{0.241pt}}
\put(975.0,310.0){\rule[-0.200pt]{0.400pt}{0.482pt}}
\put(976.67,297){\rule{0.400pt}{0.964pt}}
\multiput(976.17,299.00)(1.000,-2.000){2}{\rule{0.400pt}{0.482pt}}
\put(977.0,301.0){\rule[-0.200pt]{0.400pt}{0.482pt}}
\put(977.67,292){\rule{0.400pt}{0.723pt}}
\multiput(977.17,293.50)(1.000,-1.500){2}{\rule{0.400pt}{0.361pt}}
\put(978.67,290){\rule{0.400pt}{0.482pt}}
\multiput(978.17,291.00)(1.000,-1.000){2}{\rule{0.400pt}{0.241pt}}
\put(978.0,295.0){\rule[-0.200pt]{0.400pt}{0.482pt}}
\put(979.67,285){\rule{0.400pt}{0.482pt}}
\multiput(979.17,286.00)(1.000,-1.000){2}{\rule{0.400pt}{0.241pt}}
\put(980.67,283){\rule{0.400pt}{0.482pt}}
\multiput(980.17,284.00)(1.000,-1.000){2}{\rule{0.400pt}{0.241pt}}
\put(980.0,287.0){\rule[-0.200pt]{0.400pt}{0.723pt}}
\put(981.67,277){\rule{0.400pt}{0.482pt}}
\multiput(981.17,278.00)(1.000,-1.000){2}{\rule{0.400pt}{0.241pt}}
\put(982.0,279.0){\rule[-0.200pt]{0.400pt}{0.964pt}}
\put(983,271.67){\rule{0.241pt}{0.400pt}}
\multiput(983.00,272.17)(0.500,-1.000){2}{\rule{0.120pt}{0.400pt}}
\put(983.67,270){\rule{0.400pt}{0.482pt}}
\multiput(983.17,271.00)(1.000,-1.000){2}{\rule{0.400pt}{0.241pt}}
\put(983.0,273.0){\rule[-0.200pt]{0.400pt}{0.964pt}}
\put(985,264.67){\rule{0.241pt}{0.400pt}}
\multiput(985.00,265.17)(0.500,-1.000){2}{\rule{0.120pt}{0.400pt}}
\put(985.67,262){\rule{0.400pt}{0.723pt}}
\multiput(985.17,263.50)(1.000,-1.500){2}{\rule{0.400pt}{0.361pt}}
\put(985.0,266.0){\rule[-0.200pt]{0.400pt}{0.964pt}}
\put(986.67,258){\rule{0.400pt}{0.723pt}}
\multiput(986.17,259.50)(1.000,-1.500){2}{\rule{0.400pt}{0.361pt}}
\put(987.0,261.0){\usebox{\plotpoint}}
\put(988,253.67){\rule{0.241pt}{0.400pt}}
\multiput(988.00,254.17)(0.500,-1.000){2}{\rule{0.120pt}{0.400pt}}
\put(988.67,250){\rule{0.400pt}{0.964pt}}
\multiput(988.17,252.00)(1.000,-2.000){2}{\rule{0.400pt}{0.482pt}}
\put(988.0,255.0){\rule[-0.200pt]{0.400pt}{0.723pt}}
\put(989.67,247){\rule{0.400pt}{0.482pt}}
\multiput(989.17,248.00)(1.000,-1.000){2}{\rule{0.400pt}{0.241pt}}
\put(990.0,249.0){\usebox{\plotpoint}}
\put(991,243.67){\rule{0.241pt}{0.400pt}}
\multiput(991.00,244.17)(0.500,-1.000){2}{\rule{0.120pt}{0.400pt}}
\put(991.67,241){\rule{0.400pt}{0.723pt}}
\multiput(991.17,242.50)(1.000,-1.500){2}{\rule{0.400pt}{0.361pt}}
\put(991.0,245.0){\rule[-0.200pt]{0.400pt}{0.482pt}}
\put(992.67,237){\rule{0.400pt}{0.482pt}}
\multiput(992.17,238.00)(1.000,-1.000){2}{\rule{0.400pt}{0.241pt}}
\put(993.67,234){\rule{0.400pt}{0.723pt}}
\multiput(993.17,235.50)(1.000,-1.500){2}{\rule{0.400pt}{0.361pt}}
\put(993.0,239.0){\rule[-0.200pt]{0.400pt}{0.482pt}}
\put(995,231.67){\rule{0.241pt}{0.400pt}}
\multiput(995.00,232.17)(0.500,-1.000){2}{\rule{0.120pt}{0.400pt}}
\put(995.0,233.0){\usebox{\plotpoint}}
\put(996,228.67){\rule{0.241pt}{0.400pt}}
\multiput(996.00,229.17)(0.500,-1.000){2}{\rule{0.120pt}{0.400pt}}
\put(996.67,226){\rule{0.400pt}{0.723pt}}
\multiput(996.17,227.50)(1.000,-1.500){2}{\rule{0.400pt}{0.361pt}}
\put(996.0,230.0){\rule[-0.200pt]{0.400pt}{0.482pt}}
\put(997.67,222){\rule{0.400pt}{0.482pt}}
\multiput(997.17,223.00)(1.000,-1.000){2}{\rule{0.400pt}{0.241pt}}
\put(999,220.67){\rule{0.241pt}{0.400pt}}
\multiput(999.00,221.17)(0.500,-1.000){2}{\rule{0.120pt}{0.400pt}}
\put(998.0,224.0){\rule[-0.200pt]{0.400pt}{0.482pt}}
\put(1000.0,218.0){\rule[-0.200pt]{0.400pt}{0.723pt}}
\put(1000.0,218.0){\usebox{\plotpoint}}
\put(1001,213.67){\rule{0.241pt}{0.400pt}}
\multiput(1001.00,214.17)(0.500,-1.000){2}{\rule{0.120pt}{0.400pt}}
\put(1001.67,212){\rule{0.400pt}{0.482pt}}
\multiput(1001.17,213.00)(1.000,-1.000){2}{\rule{0.400pt}{0.241pt}}
\put(1001.0,215.0){\rule[-0.200pt]{0.400pt}{0.723pt}}
\put(1003,208.67){\rule{0.241pt}{0.400pt}}
\multiput(1003.00,209.17)(0.500,-1.000){2}{\rule{0.120pt}{0.400pt}}
\put(1003.0,210.0){\rule[-0.200pt]{0.400pt}{0.482pt}}
\put(1004,206.67){\rule{0.241pt}{0.400pt}}
\multiput(1004.00,207.17)(0.500,-1.000){2}{\rule{0.120pt}{0.400pt}}
\put(1004.67,205){\rule{0.400pt}{0.482pt}}
\multiput(1004.17,206.00)(1.000,-1.000){2}{\rule{0.400pt}{0.241pt}}
\put(1004.0,208.0){\usebox{\plotpoint}}
\put(1005.67,202){\rule{0.400pt}{0.482pt}}
\multiput(1005.17,203.00)(1.000,-1.000){2}{\rule{0.400pt}{0.241pt}}
\put(1006.0,204.0){\usebox{\plotpoint}}
\put(1007.0,202.0){\usebox{\plotpoint}}
\put(1008.0,199.0){\rule[-0.200pt]{0.400pt}{0.723pt}}
\put(1008.0,199.0){\usebox{\plotpoint}}
\put(1009,195.67){\rule{0.241pt}{0.400pt}}
\multiput(1009.00,196.17)(0.500,-1.000){2}{\rule{0.120pt}{0.400pt}}
\put(1009.67,194){\rule{0.400pt}{0.482pt}}
\multiput(1009.17,195.00)(1.000,-1.000){2}{\rule{0.400pt}{0.241pt}}
\put(1009.0,197.0){\rule[-0.200pt]{0.400pt}{0.482pt}}
\put(1011,191.67){\rule{0.241pt}{0.400pt}}
\multiput(1011.00,192.17)(0.500,-1.000){2}{\rule{0.120pt}{0.400pt}}
\put(1012,190.67){\rule{0.241pt}{0.400pt}}
\multiput(1012.00,191.17)(0.500,-1.000){2}{\rule{0.120pt}{0.400pt}}
\put(1011.0,193.0){\usebox{\plotpoint}}
\put(1013,187.67){\rule{0.241pt}{0.400pt}}
\multiput(1013.00,188.17)(0.500,-1.000){2}{\rule{0.120pt}{0.400pt}}
\put(1013.0,189.0){\rule[-0.200pt]{0.400pt}{0.482pt}}
\put(1014.0,186.0){\rule[-0.200pt]{0.400pt}{0.482pt}}
\put(1015,184.67){\rule{0.241pt}{0.400pt}}
\multiput(1015.00,185.17)(0.500,-1.000){2}{\rule{0.120pt}{0.400pt}}
\put(1014.0,186.0){\usebox{\plotpoint}}
\put(1016,182.67){\rule{0.241pt}{0.400pt}}
\multiput(1016.00,183.17)(0.500,-1.000){2}{\rule{0.120pt}{0.400pt}}
\put(1016.0,184.0){\usebox{\plotpoint}}
\put(1016.67,180){\rule{0.400pt}{0.482pt}}
\multiput(1016.17,181.00)(1.000,-1.000){2}{\rule{0.400pt}{0.241pt}}
\put(1018,178.67){\rule{0.241pt}{0.400pt}}
\multiput(1018.00,179.17)(0.500,-1.000){2}{\rule{0.120pt}{0.400pt}}
\put(1017.0,182.0){\usebox{\plotpoint}}
\put(1019.0,178.0){\usebox{\plotpoint}}
\put(1020,176.67){\rule{0.241pt}{0.400pt}}
\multiput(1020.00,177.17)(0.500,-1.000){2}{\rule{0.120pt}{0.400pt}}
\put(1019.0,178.0){\usebox{\plotpoint}}
\put(1021,177){\usebox{\plotpoint}}
\put(1021,175.67){\rule{0.241pt}{0.400pt}}
\multiput(1021.00,176.17)(0.500,-1.000){2}{\rule{0.120pt}{0.400pt}}
\put(1021.67,172){\rule{0.400pt}{0.482pt}}
\multiput(1021.17,173.00)(1.000,-1.000){2}{\rule{0.400pt}{0.241pt}}
\put(1023,171.67){\rule{0.241pt}{0.400pt}}
\multiput(1023.00,171.17)(0.500,1.000){2}{\rule{0.120pt}{0.400pt}}
\put(1022.0,174.0){\rule[-0.200pt]{0.400pt}{0.482pt}}
\put(1023.67,170){\rule{0.400pt}{0.482pt}}
\multiput(1023.17,171.00)(1.000,-1.000){2}{\rule{0.400pt}{0.241pt}}
\put(1025,168.67){\rule{0.241pt}{0.400pt}}
\multiput(1025.00,169.17)(0.500,-1.000){2}{\rule{0.120pt}{0.400pt}}
\put(1024.0,172.0){\usebox{\plotpoint}}
\put(1025.67,168){\rule{0.400pt}{0.482pt}}
\multiput(1025.17,169.00)(1.000,-1.000){2}{\rule{0.400pt}{0.241pt}}
\put(1026.0,169.0){\usebox{\plotpoint}}
\put(1027.0,167.0){\usebox{\plotpoint}}
\put(1027.67,165){\rule{0.400pt}{0.482pt}}
\multiput(1027.17,166.00)(1.000,-1.000){2}{\rule{0.400pt}{0.241pt}}
\put(1027.0,167.0){\usebox{\plotpoint}}
\put(1029,165){\usebox{\plotpoint}}
\put(1029,163.67){\rule{0.241pt}{0.400pt}}
\multiput(1029.00,164.17)(0.500,-1.000){2}{\rule{0.120pt}{0.400pt}}
\put(1030.0,163.0){\usebox{\plotpoint}}
\put(1031,161.67){\rule{0.241pt}{0.400pt}}
\multiput(1031.00,162.17)(0.500,-1.000){2}{\rule{0.120pt}{0.400pt}}
\put(1030.0,163.0){\usebox{\plotpoint}}
\put(1032.0,161.0){\usebox{\plotpoint}}
\put(1033,159.67){\rule{0.241pt}{0.400pt}}
\multiput(1033.00,160.17)(0.500,-1.000){2}{\rule{0.120pt}{0.400pt}}
\put(1032.0,161.0){\usebox{\plotpoint}}
\put(1034,158.67){\rule{0.241pt}{0.400pt}}
\multiput(1034.00,158.17)(0.500,1.000){2}{\rule{0.120pt}{0.400pt}}
\put(1034.0,159.0){\usebox{\plotpoint}}
\put(1035,156.67){\rule{0.241pt}{0.400pt}}
\multiput(1035.00,157.17)(0.500,-1.000){2}{\rule{0.120pt}{0.400pt}}
\put(1036,155.67){\rule{0.241pt}{0.400pt}}
\multiput(1036.00,156.17)(0.500,-1.000){2}{\rule{0.120pt}{0.400pt}}
\put(1035.0,158.0){\rule[-0.200pt]{0.400pt}{0.482pt}}
\put(1037,156){\usebox{\plotpoint}}
\put(1037,155.67){\rule{0.241pt}{0.400pt}}
\multiput(1037.00,155.17)(0.500,1.000){2}{\rule{0.120pt}{0.400pt}}
\put(1037.67,155){\rule{0.400pt}{0.482pt}}
\multiput(1037.17,156.00)(1.000,-1.000){2}{\rule{0.400pt}{0.241pt}}
\put(1038.67,154){\rule{0.400pt}{0.482pt}}
\multiput(1038.17,155.00)(1.000,-1.000){2}{\rule{0.400pt}{0.241pt}}
\put(1039.0,155.0){\usebox{\plotpoint}}
\put(1040.0,153.0){\usebox{\plotpoint}}
\put(1042,151.67){\rule{0.241pt}{0.400pt}}
\multiput(1042.00,152.17)(0.500,-1.000){2}{\rule{0.120pt}{0.400pt}}
\put(1040.0,153.0){\rule[-0.200pt]{0.482pt}{0.400pt}}
\put(1043,152){\usebox{\plotpoint}}
\put(1043,150.67){\rule{0.241pt}{0.400pt}}
\multiput(1043.00,151.17)(0.500,-1.000){2}{\rule{0.120pt}{0.400pt}}
\put(1044,149.67){\rule{0.241pt}{0.400pt}}
\multiput(1044.00,150.17)(0.500,-1.000){2}{\rule{0.120pt}{0.400pt}}
\put(1044.67,149){\rule{0.400pt}{0.482pt}}
\multiput(1044.17,150.00)(1.000,-1.000){2}{\rule{0.400pt}{0.241pt}}
\put(1045.0,150.0){\usebox{\plotpoint}}
\put(1047,147.67){\rule{0.241pt}{0.400pt}}
\multiput(1047.00,148.17)(0.500,-1.000){2}{\rule{0.120pt}{0.400pt}}
\put(1046.0,149.0){\usebox{\plotpoint}}
\put(1048,148){\usebox{\plotpoint}}
\put(1048,146.67){\rule{0.241pt}{0.400pt}}
\multiput(1048.00,147.17)(0.500,-1.000){2}{\rule{0.120pt}{0.400pt}}
\put(1049,146.67){\rule{0.241pt}{0.400pt}}
\multiput(1049.00,146.17)(0.500,1.000){2}{\rule{0.120pt}{0.400pt}}
\put(1050.0,146.0){\rule[-0.200pt]{0.400pt}{0.482pt}}
\put(1051,144.67){\rule{0.241pt}{0.400pt}}
\multiput(1051.00,145.17)(0.500,-1.000){2}{\rule{0.120pt}{0.400pt}}
\put(1050.0,146.0){\usebox{\plotpoint}}
\put(1053,143.67){\rule{0.241pt}{0.400pt}}
\multiput(1053.00,144.17)(0.500,-1.000){2}{\rule{0.120pt}{0.400pt}}
\put(1054,142.67){\rule{0.241pt}{0.400pt}}
\multiput(1054.00,143.17)(0.500,-1.000){2}{\rule{0.120pt}{0.400pt}}
\put(1052.0,145.0){\usebox{\plotpoint}}
\put(1055,142.67){\rule{0.241pt}{0.400pt}}
\multiput(1055.00,143.17)(0.500,-1.000){2}{\rule{0.120pt}{0.400pt}}
\put(1055.0,143.0){\usebox{\plotpoint}}
\put(1056,143){\usebox{\plotpoint}}
\put(1056,141.67){\rule{0.241pt}{0.400pt}}
\multiput(1056.00,142.17)(0.500,-1.000){2}{\rule{0.120pt}{0.400pt}}
\put(1057.0,142.0){\rule[-0.200pt]{0.482pt}{0.400pt}}
\put(1059.0,141.0){\usebox{\plotpoint}}
\put(1061,139.67){\rule{0.241pt}{0.400pt}}
\multiput(1061.00,140.17)(0.500,-1.000){2}{\rule{0.120pt}{0.400pt}}
\put(1059.0,141.0){\rule[-0.200pt]{0.482pt}{0.400pt}}
\put(1063,138.67){\rule{0.241pt}{0.400pt}}
\multiput(1063.00,139.17)(0.500,-1.000){2}{\rule{0.120pt}{0.400pt}}
\put(1062.0,140.0){\usebox{\plotpoint}}
\put(1064,139){\usebox{\plotpoint}}
\put(1065,137.67){\rule{0.241pt}{0.400pt}}
\multiput(1065.00,138.17)(0.500,-1.000){2}{\rule{0.120pt}{0.400pt}}
\put(1064.0,139.0){\usebox{\plotpoint}}
\put(1066,138){\usebox{\plotpoint}}
\put(1067,136.67){\rule{0.241pt}{0.400pt}}
\multiput(1067.00,137.17)(0.500,-1.000){2}{\rule{0.120pt}{0.400pt}}
\put(1066.0,138.0){\usebox{\plotpoint}}
\put(1068.0,137.0){\usebox{\plotpoint}}
\put(1069,136.67){\rule{0.241pt}{0.400pt}}
\multiput(1069.00,137.17)(0.500,-1.000){2}{\rule{0.120pt}{0.400pt}}
\put(1069.0,137.0){\usebox{\plotpoint}}
\put(1072,135.67){\rule{0.241pt}{0.400pt}}
\multiput(1072.00,136.17)(0.500,-1.000){2}{\rule{0.120pt}{0.400pt}}
\put(1070.0,137.0){\rule[-0.200pt]{0.482pt}{0.400pt}}
\put(1073.0,136.0){\usebox{\plotpoint}}
\put(1074,134.67){\rule{0.241pt}{0.400pt}}
\multiput(1074.00,134.17)(0.500,1.000){2}{\rule{0.120pt}{0.400pt}}
\put(1074.0,135.0){\usebox{\plotpoint}}
\put(1075,134.67){\rule{0.241pt}{0.400pt}}
\multiput(1075.00,134.17)(0.500,1.000){2}{\rule{0.120pt}{0.400pt}}
\put(1076,134.67){\rule{0.241pt}{0.400pt}}
\multiput(1076.00,135.17)(0.500,-1.000){2}{\rule{0.120pt}{0.400pt}}
\put(1075.0,135.0){\usebox{\plotpoint}}
\put(1077,135){\usebox{\plotpoint}}
\put(1077.0,135.0){\rule[-0.200pt]{0.482pt}{0.400pt}}
\put(1079.0,134.0){\usebox{\plotpoint}}
\put(1081,132.67){\rule{0.241pt}{0.400pt}}
\multiput(1081.00,133.17)(0.500,-1.000){2}{\rule{0.120pt}{0.400pt}}
\put(1079.0,134.0){\rule[-0.200pt]{0.482pt}{0.400pt}}
\put(1082,132.67){\rule{0.241pt}{0.400pt}}
\multiput(1082.00,133.17)(0.500,-1.000){2}{\rule{0.120pt}{0.400pt}}
\put(1082.0,133.0){\usebox{\plotpoint}}
\put(1083,133){\usebox{\plotpoint}}
\put(1086,131.67){\rule{0.241pt}{0.400pt}}
\multiput(1086.00,132.17)(0.500,-1.000){2}{\rule{0.120pt}{0.400pt}}
\put(1083.0,133.0){\rule[-0.200pt]{0.723pt}{0.400pt}}
\put(1087.0,132.0){\rule[-0.200pt]{0.964pt}{0.400pt}}
\put(1091.0,131.0){\usebox{\plotpoint}}
\put(1092,130.67){\rule{0.241pt}{0.400pt}}
\multiput(1092.00,130.17)(0.500,1.000){2}{\rule{0.120pt}{0.400pt}}
\put(1091.0,131.0){\usebox{\plotpoint}}
\put(1093,132){\usebox{\plotpoint}}
\put(1093,130.67){\rule{0.241pt}{0.400pt}}
\multiput(1093.00,131.17)(0.500,-1.000){2}{\rule{0.120pt}{0.400pt}}
\put(1094.0,131.0){\usebox{\plotpoint}}
\put(1094.0,132.0){\rule[-0.200pt]{0.482pt}{0.400pt}}
\put(1096.0,131.0){\usebox{\plotpoint}}
\put(1097,131){\usebox{\plotpoint}}
\put(1097.0,131.0){\usebox{\plotpoint}}
\put(171.0,131.0){\rule[-0.200pt]{0.400pt}{155.380pt}}
\put(171.0,131.0){\rule[-0.200pt]{305.461pt}{0.400pt}}
\put(1439.0,131.0){\rule[-0.200pt]{0.400pt}{155.380pt}}
\put(171.0,776.0){\rule[-0.200pt]{305.461pt}{0.400pt}}
\end{picture}
}
  \end	{center}
\end {figure}\\
Visual inspection of \textit{Figure 5} shows that the two curves respond nearly identically to wavelength (again showing the dyes to be correct), just with different amplitude of absorption. In this case, the replicated drink has greater absorption throughout all wavelengths, perhaps indicating it to be over-concentrated. This results in a positive error for the replicated drink's absorption. This error may be attributed to several sources, one being the presence of chemicals other than the dye in the grape drink. These extra chemicals could potentially alter the absorption of the original drink in an unknown manner, leading to uncertainty in its measurements that may skew the results of the experiment. Another potential error is due to drift in the spectrometers calibration. In the experiment, the spectrometer was calibrated only once at the start of each experimental session. This leads to the possibility that the spectrometer drifted out of calibration over the session, causing the measured spectra to be off and yielding uncertainty. There is also a small level of uncertainty associated with the glassware (discussed in the glassware accuracy project), however these are so small (on the order of hundredths of a percent) that it is improbable they caused the observed error. A final uncertainty is that different spectrometers were used in the two sessions, and they could likely had different accuracy and precision, which would have produced a difference between the values measured in each session, possibly explaining the discrepancy. 
\vspace{12pt}\\
A complementary technique to the spectrometry-based approach used in this lab would be column chromatography$^{3}$. Column chromatography is a technique that separates and purifies chemicals from a solution. Column chromatography can be used to separate the individual dyes from the solution based off their properties, such as how polar they are. Once they are separated, they can be quantitatively judged in their pure form using a spectrometer, giving the absorption and wavelength without uncertainties due to other chemicals. This allows for a more accurate quantitative measurements to be performed on the drink. The previously discussed paper chromatography method discussed in the \textit{Introduction} section could also be used, however this approach seems limited in that it offers mostly qualitative as opposed to quantitative results.
%-----------------------------------------------------------------------------------------
%  Conculusion
%-----------------------------------------------------------------------------------------
\section{Conclusion}
It was found that the purple grape-flavored Kool-Aid contained Red 40 and Blue 1 dyes, with $4.64 \times 10^{-5}$ M and $8.67 \times 10^{-6}$ M concentrations respectively. A 50 mL replicated solution was produced, and it was found that the solution matched the original given a maximum level of uncertainty of about +8.7\% for absorption and -0.54\% for wavelength at the peak of absorption of the absorption spectrum. This agreement confirmed the found dye concentrations of the original grape drink, showing them to be a formula for replicating the drink. These results also confirm the accuracy of Beer's law as it was used to successfully determine the concentrations of the grape drink and to replicate it.\par
\vspace{6pt}
The methodology and results of this experiment hold significance in that they bring forth an approach to identifying the contents and concentration of a solution or material via comparison of the spectral absorption of the material to other known materials. This holds value for example to government bodies as it allows them to establish means to test and verify compliance of food and pharmaceutical to standards such as acceptable food dye concentration.
%-----------------------------------------------------------------------------------------
%  References
%-----------------------------------------------------------------------------------------
\section*{References}
[1]\hspace{4ex} Visible and Ultraviolet Spectroscopy. http://www2.chemistry.msu.edu

\hspace{12ex}(accessed March 30, 2015).
\vspace{6pt}\newline
[2]\hspace{4ex}Yanuka, Y.; Shalon, Y.; Weissenberg, E.; Nir-Grosfeld, I. A Paper-chromatographic Method 

\hspace{12ex}for the Identification of Food Dyes. \textit{Analyst.} \textbf{1962}, 87, 791-796\newline
[3]\hspace{4ex} Separation of Food Dyes Via Column Chromatography. http://www.chem.umn.edu

\hspace{12ex}(accessed March 29, 2015).\newline
[4]\hspace{4ex} Colorimetric Analysis. http://www.chem.ucla.edu (accessed April 6, 2015).
\end{document}