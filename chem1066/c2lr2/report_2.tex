\documentclass[12pt, letterpaper]{article}

\usepackage{setspace}
\usepackage[version=3]{mhchem} % Package for chemical equation typesetting
\usepackage{siunitx} % Provides the \SI{}{} and \si{} command for typesetting SI units
\usepackage{graphicx} % Required for the inclusion of images
\usepackage{amsmath} % Required for some math elements 
\usepackage{subcaption}
\setlength\parindent{0pt} % Removes all indentation from paragraphs
\usepackage[top=0.8in, bottom= 0.8in, left= 0.8in, right= 0.8in]{geometry}
\usepackage{fancyhdr}
\usepackage{array}
\usepackage{adforn}
\usepackage{textcomp}
\usepackage{placeins}
\newcolumntype{L}{>{\centering\arraybackslash}m{3cm}}

\pagestyle{fancy}
\renewcommand{\labelenumi}{\alph{enumi}.} % Make numbering in the enumerate environment by letter rather than number (e.g. section 6)

%----------------------------------------------------------------------------------------
%	Cover Page
%----------------------------------------------------------------------------------------

\title{Determining Rate Law and Activation Energy of Magnesium Powder and Hydrochloric Acid Reaction Using Pressure Probe Based Rate Measurements \\ \adforn{21} \\  Chemistry 1066} % Title

\author{Author: Cole \textsc{Nielsen}} % Author name
\date{Fall 2015} % Date for the report


\begin{document}

\maketitle % Insert the title, author and date

\begin{center}
 \begin{tabular}{l r}
  Dates Performed: & Oct. 22 \& 29, 2015 \\ % Date the experiment was performed
  Partners: & Maile Anthony \\  
  &Joshua Widman \\ % Partner names
  Instructor: & Subhasree Kal % Instructor/supervisor
 \end{tabular}
\end{center}

\pagebreak
%-----------------------------------------------------------------------------------------
%   Abstract
%-----------------------------------------------------------------------------------------
\begin{abstract}\doublespacing
\noindent The rate law and activation energy of the reaction between hydrochloric acid and magnesium powder was determined experimentally using pH and pressure probe based methods. This was done using these methods by finding the initial rate and then using the rate law equation and Arrhenius equation to find the desired values. The pH test was performed initially by monitoring the pH of the reaction solution (that is the HCl concentration) to find the initial rate. This method was found to be unsuitable in this experiment, as the reaction was too turbulent and resulted in data with too much uncertainty to produce a meaningful initial rate. Therefore, a pressure based method where the reaction was performed in a sealed beaker with a pressure probe measuring the increase in pressure (\ce{H2} generation) was used to find the initial reaction rates. From the pressure based data, the rate law was successfully found to be 
\begin{equation}
\textnormal{rate}  = 1.378 (\textnormal{grams Mg})^{1.029} [\textnormal{HCl}]^{1.853}
\end{equation}
The experimental orders of reaction were found to be close to the expected 1 and 2 respective for Mg and HCl. Finally, the activation energy (E$_a$) was found to be 18.9 kJ/mol, also similar to the expected value of 28.5 kJ/mol \cite{mg_hcl}. 
\end{abstract}
\hrulefill

%-----------------------------------------------------------------------------------------
%  Introduction
%-----------------------------------------------------------------------------------------
\section{Introduction}\doublespacing
The objective of this experiment was to determine the rate law equation and activation energy for the reaction between magnesium powder and hydrochloric acid. The interest of this experiment is largely with the methods it uses, as they lay ground work for techniques to determine rate laws and activation energy experimentally. Rate law equations and activation energy are of specific interest and value because they allow for reactions to quantitatively be understood in terms of how fast a reaction will occur and the energy cost to begin a reaction. These factors are crucial in understanding how many reactions work, such as the iodine clock reaction \cite{clock}, which is highly dependent on reaction rates of the chemicals used. These concepts are also of great interest in chemical engineering, as they allow processes to be designed to take a minimal amount of time (using rates) and with minimal energy (activation energy) to reduce cost of production.\\\par

Similar work has been done before to this experiment using a related pressure based method and strips of magnesium metal \cite{mg_hcl} to find the reaction rate and activation energy. Those results will be compared with this experiments results as a form of validation. It is predicted for this experiment that the rate law will take the following form:
\begin{equation}
rate = k'(\textnormal{grams Mg})^1 [\textnormal{HCl}]^2
\label{eq:rl}
\end{equation}
This is because the balanced chemical equation for the reaction is \ce{Mg(s) + 2HCl(aq) -> MgCl2(aq) + H2(g)}, which correlates to the rate law of the left side being k$ [\textnormal{Mg}]^1 [\textnormal{HCl}]^2$, which is reaffirmed by \cite{mg_hcl}. Since the magnesium is a solid and therefore its concentration is unknown, we replace concentration with surface area, which is proportional to mass of the powder. Since Mg concentration is not used, the reaction rate coefficient will be changed, and thus we used k' instead, denoting the rate coefficient for a given mass of Mg instead of concentration. This leads to equation \ref{eq:rl}. The values for k' and E$_a$ are not so easily predictable and can only be found experimentally.\\\par

This prediction will be tested experimentally given by the procedures in the \textit{Experimental} section to produce values given in the \textit{Results} section. These results will later be discussed in detail, spanning theory, derivation and implications in the \textit{Discussion } section. Final remarks on the overall outcomes and results of the experiment will be made last in the \textit{Conclusions} section.


%-----------------------------------------------------------------------------------------
%  Experimental
%-----------------------------------------------------------------------------------------
\section{Experimental}
The experiment was performed and will be discussed in two major parts: a pH based test and a pressure based test.
\subsection{pH Test}
The rate equation was found using a pH test by the following procedure. Initially samples of Mg powder and HCl were prepared for three reactions, with the following pairs used (Mg then HCl): 0.04g and 10 mL 1M, 0.02g and 10mL 1M, 0.04g and 10mL 0.5M. These values were chosen so that between the sets of trials there were two where Mg was held constant and HCl and varied, and vice versa to determine the reaction order. The volume of HCl was chosen to be arbitrarily in excess for the reaction. A Vernier pH probe, being a USB pH meter that connects to a PC for pH monitoring, was set up with LoggerPro to record the experimental data. The probe was calibrated before the experiment using two buffer solutions of pH 7 and pH 4. The first trial was performed by placing 10 mL of 1M HCl into a beaker and placing the clean pH probe into the solution until the measured pH stabilized. Data was then set to record, and the Mg was added into the acid and stirred with a glass stir-rod until the pH stabilized again. The data sampling was then stopped, saved, and the probe cleaned. This process was then repeated for the other two sets of samples.
\subsection{Pressure Test}
The rate equation was found using a pressure probe by the following procedure. Samples of Mg and HCl were again prepared for four reactions, with the following pairs used: two sets of 0.04g Mg and 10 mL 1M HCl, 0.02g Mg and 10mL 1M HCl, 0.04g Mg and 10mL 0.5M HCl. These were again chosen so that one value was held constant while the other varied so both rate orders of the reactants could be found. Additionally, the temperature of the solutions were recorded this time, and one of the 10 mL 1M HCl solutions of the duplicated pair was chilled in a ice bath to allow for calculation of E$_a$. A Vernier Gas Pressure sensor, being a USB pressure meter that connects to a PC was set up with LoggerPro to monitor pressure in atmospheres. The pressure probe was used on a cork-sealed 250 mL Erlenmeyer Flask, which acted as a pressure vessel. The amounts of reactants were chosen such that the pressure would never exceed 1.3 atm, leaving little possibility for a vessel rupture. The first trial was performed by loading 10 mL of 1M HCl into a syringe, and 0.04g of Mg powder into the flask. Data recording was started, the HCl was injected quickly into the unsealed flask, after which the flask was quickly sealed. The flask was unsealed during injection to avoid fluctuations in pressure due to the added liquid volume. After the pressure stabilized, data sampling was stopped, the pressure vessel was cleaned and then the process was repeated for the other samples.

%-----------------------------------------------------------------------------------------
%  Results
%-----------------------------------------------------------------------------------------
\section{Results}%\singlespacing
\FloatBarrier
\subsection{pH Test}
Below in figure \ref{fig:pH_plot} is an example of the plotted (unusable) data from the pH trials. The data was found to be too turbulent at the start of every trial (seen as the initial rise and fall) to determine an accurate initial rate.
\begin{figure}[h!]
  \begin{center}
    	\resizebox{0.6\textwidth}{!}{% GNUPLOT: LaTeX picture
\setlength{\unitlength}{0.240900pt}
\ifx\plotpoint\undefined\newsavebox{\plotpoint}\fi
\sbox{\plotpoint}{\rule[-0.200pt]{0.400pt}{0.400pt}}%
\begin{picture}(1500,900)(0,0)
\sbox{\plotpoint}{\rule[-0.200pt]{0.400pt}{0.400pt}}%
\put(191.0,131.0){\rule[-0.200pt]{4.818pt}{0.400pt}}
\put(171,131){\makebox(0,0)[r]{ 0.2}}
\put(1419.0,131.0){\rule[-0.200pt]{4.818pt}{0.400pt}}
\put(191.0,238.0){\rule[-0.200pt]{4.818pt}{0.400pt}}
\put(171,238){\makebox(0,0)[r]{ 0.25}}
\put(1419.0,238.0){\rule[-0.200pt]{4.818pt}{0.400pt}}
\put(191.0,346.0){\rule[-0.200pt]{4.818pt}{0.400pt}}
\put(171,346){\makebox(0,0)[r]{ 0.3}}
\put(1419.0,346.0){\rule[-0.200pt]{4.818pt}{0.400pt}}
\put(191.0,453.0){\rule[-0.200pt]{4.818pt}{0.400pt}}
\put(171,453){\makebox(0,0)[r]{ 0.35}}
\put(1419.0,453.0){\rule[-0.200pt]{4.818pt}{0.400pt}}
\put(191.0,561.0){\rule[-0.200pt]{4.818pt}{0.400pt}}
\put(171,561){\makebox(0,0)[r]{ 0.4}}
\put(1419.0,561.0){\rule[-0.200pt]{4.818pt}{0.400pt}}
\put(191.0,668.0){\rule[-0.200pt]{4.818pt}{0.400pt}}
\put(171,668){\makebox(0,0)[r]{ 0.45}}
\put(1419.0,668.0){\rule[-0.200pt]{4.818pt}{0.400pt}}
\put(191.0,776.0){\rule[-0.200pt]{4.818pt}{0.400pt}}
\put(171,776){\makebox(0,0)[r]{ 0.5}}
\put(1419.0,776.0){\rule[-0.200pt]{4.818pt}{0.400pt}}
\put(191.0,131.0){\rule[-0.200pt]{0.400pt}{4.818pt}}
\put(191,90){\makebox(0,0){ 0}}
\put(191.0,756.0){\rule[-0.200pt]{0.400pt}{4.818pt}}
\put(441.0,131.0){\rule[-0.200pt]{0.400pt}{4.818pt}}
\put(441,90){\makebox(0,0){ 50}}
\put(441.0,756.0){\rule[-0.200pt]{0.400pt}{4.818pt}}
\put(690.0,131.0){\rule[-0.200pt]{0.400pt}{4.818pt}}
\put(690,90){\makebox(0,0){ 100}}
\put(690.0,756.0){\rule[-0.200pt]{0.400pt}{4.818pt}}
\put(940.0,131.0){\rule[-0.200pt]{0.400pt}{4.818pt}}
\put(940,90){\makebox(0,0){ 150}}
\put(940.0,756.0){\rule[-0.200pt]{0.400pt}{4.818pt}}
\put(1189.0,131.0){\rule[-0.200pt]{0.400pt}{4.818pt}}
\put(1189,90){\makebox(0,0){ 200}}
\put(1189.0,756.0){\rule[-0.200pt]{0.400pt}{4.818pt}}
\put(1439.0,131.0){\rule[-0.200pt]{0.400pt}{4.818pt}}
\put(1439,90){\makebox(0,0){ 250}}
\put(1439.0,756.0){\rule[-0.200pt]{0.400pt}{4.818pt}}
\put(191.0,131.0){\rule[-0.200pt]{0.400pt}{155.380pt}}
\put(191.0,131.0){\rule[-0.200pt]{300.643pt}{0.400pt}}
\put(1439.0,131.0){\rule[-0.200pt]{0.400pt}{155.380pt}}
\put(191.0,776.0){\rule[-0.200pt]{300.643pt}{0.400pt}}
\put(30,453){\makebox(0,0){pH}}
\put(815,29){\makebox(0,0){Time (seconds)}}
\put(815,838){\makebox(0,0){0.04g Mg Powder - 10mL 0.5M HCl Reaction pH v. Time}}
\put(191,317){\rule{1pt}{1pt}}
\put(201,317){\rule{1pt}{1pt}}
\put(211,315){\rule{1pt}{1pt}}
\put(221,309){\rule{1pt}{1pt}}
\put(231,307){\rule{1pt}{1pt}}
\put(241,304){\rule{1pt}{1pt}}
\put(251,303){\rule{1pt}{1pt}}
\put(261,298){\rule{1pt}{1pt}}
\put(271,294){\rule{1pt}{1pt}}
\put(281,296){\rule{1pt}{1pt}}
\put(291,293){\rule{1pt}{1pt}}
\put(301,292){\rule{1pt}{1pt}}
\put(311,324){\rule{1pt}{1pt}}
\put(321,326){\rule{1pt}{1pt}}
\put(331,342){\rule{1pt}{1pt}}
\put(341,308){\rule{1pt}{1pt}}
\put(351,283){\rule{1pt}{1pt}}
\put(361,252){\rule{1pt}{1pt}}
\put(371,233){\rule{1pt}{1pt}}
\put(381,242){\rule{1pt}{1pt}}
\put(391,246){\rule{1pt}{1pt}}
\put(401,260){\rule{1pt}{1pt}}
\put(411,281){\rule{1pt}{1pt}}
\put(421,273){\rule{1pt}{1pt}}
\put(431,283){\rule{1pt}{1pt}}
\put(441,304){\rule{1pt}{1pt}}
\put(451,319){\rule{1pt}{1pt}}
\put(461,274){\rule{1pt}{1pt}}
\put(471,280){\rule{1pt}{1pt}}
\put(481,296){\rule{1pt}{1pt}}
\put(491,315){\rule{1pt}{1pt}}
\put(501,328){\rule{1pt}{1pt}}
\put(510,343){\rule{1pt}{1pt}}
\put(520,357){\rule{1pt}{1pt}}
\put(530,368){\rule{1pt}{1pt}}
\put(540,373){\rule{1pt}{1pt}}
\put(550,387){\rule{1pt}{1pt}}
\put(560,394){\rule{1pt}{1pt}}
\put(570,412){\rule{1pt}{1pt}}
\put(580,421){\rule{1pt}{1pt}}
\put(590,441){\rule{1pt}{1pt}}
\put(600,447){\rule{1pt}{1pt}}
\put(610,462){\rule{1pt}{1pt}}
\put(620,474){\rule{1pt}{1pt}}
\put(630,484){\rule{1pt}{1pt}}
\put(640,493){\rule{1pt}{1pt}}
\put(650,505){\rule{1pt}{1pt}}
\put(660,511){\rule{1pt}{1pt}}
\put(670,520){\rule{1pt}{1pt}}
\put(680,526){\rule{1pt}{1pt}}
\put(690,543){\rule{1pt}{1pt}}
\put(700,552){\rule{1pt}{1pt}}
\put(710,568){\rule{1pt}{1pt}}
\put(720,569){\rule{1pt}{1pt}}
\put(730,576){\rule{1pt}{1pt}}
\put(740,588){\rule{1pt}{1pt}}
\put(750,589){\rule{1pt}{1pt}}
\put(760,599){\rule{1pt}{1pt}}
\put(770,600){\rule{1pt}{1pt}}
\put(780,613){\rule{1pt}{1pt}}
\put(790,622){\rule{1pt}{1pt}}
\put(930,685){\rule{1pt}{1pt}}
\put(940,692){\rule{1pt}{1pt}}
\put(950,691){\rule{1pt}{1pt}}
\put(960,697){\rule{1pt}{1pt}}
\put(970,699){\rule{1pt}{1pt}}
\put(980,699){\rule{1pt}{1pt}}
\put(990,704){\rule{1pt}{1pt}}
\put(1000,706){\rule{1pt}{1pt}}
\put(1010,710){\rule{1pt}{1pt}}
\put(1020,710){\rule{1pt}{1pt}}
\put(1030,711){\rule{1pt}{1pt}}
\put(1040,712){\rule{1pt}{1pt}}
\put(1050,712){\rule{1pt}{1pt}}
\put(1060,723){\rule{1pt}{1pt}}
\put(1070,729){\rule{1pt}{1pt}}
\put(1080,723){\rule{1pt}{1pt}}
\put(1090,726){\rule{1pt}{1pt}}
\put(1100,733){\rule{1pt}{1pt}}
\put(1110,730){\rule{1pt}{1pt}}
\put(1120,726){\rule{1pt}{1pt}}
\put(1129,733){\rule{1pt}{1pt}}
\put(1139,736){\rule{1pt}{1pt}}
\put(1149,736){\rule{1pt}{1pt}}
\put(1159,740){\rule{1pt}{1pt}}
\put(1169,741){\rule{1pt}{1pt}}
\put(1179,745){\rule{1pt}{1pt}}
\put(1189,743){\rule{1pt}{1pt}}
\put(1199,751){\rule{1pt}{1pt}}
\put(1209,746){\rule{1pt}{1pt}}
\put(1219,747){\rule{1pt}{1pt}}
\put(1229,742){\rule{1pt}{1pt}}
\put(1239,754){\rule{1pt}{1pt}}
\put(1249,754){\rule{1pt}{1pt}}
\put(1259,756){\rule{1pt}{1pt}}
\put(1269,755){\rule{1pt}{1pt}}
\put(1279,756){\rule{1pt}{1pt}}
\put(1289,760){\rule{1pt}{1pt}}
\put(1299,756){\rule{1pt}{1pt}}
\put(1309,758){\rule{1pt}{1pt}}
\put(1319,762){\rule{1pt}{1pt}}
\put(1329,762){\rule{1pt}{1pt}}
\put(1339,762){\rule{1pt}{1pt}}
\put(1349,764){\rule{1pt}{1pt}}
\put(191,317){\usebox{\plotpoint}}
\put(201,315.17){\rule{2.100pt}{0.400pt}}
\multiput(201.00,316.17)(5.641,-2.000){2}{\rule{1.050pt}{0.400pt}}
\multiput(211.00,313.93)(0.852,-0.482){9}{\rule{0.767pt}{0.116pt}}
\multiput(211.00,314.17)(8.409,-6.000){2}{\rule{0.383pt}{0.400pt}}
\put(221,307.17){\rule{2.100pt}{0.400pt}}
\multiput(221.00,308.17)(5.641,-2.000){2}{\rule{1.050pt}{0.400pt}}
\multiput(231.00,305.95)(2.025,-0.447){3}{\rule{1.433pt}{0.108pt}}
\multiput(231.00,306.17)(7.025,-3.000){2}{\rule{0.717pt}{0.400pt}}
\put(241,302.67){\rule{2.409pt}{0.400pt}}
\multiput(241.00,303.17)(5.000,-1.000){2}{\rule{1.204pt}{0.400pt}}
\multiput(251.00,301.93)(1.044,-0.477){7}{\rule{0.900pt}{0.115pt}}
\multiput(251.00,302.17)(8.132,-5.000){2}{\rule{0.450pt}{0.400pt}}
\multiput(261.00,296.94)(1.358,-0.468){5}{\rule{1.100pt}{0.113pt}}
\multiput(261.00,297.17)(7.717,-4.000){2}{\rule{0.550pt}{0.400pt}}
\put(271,294.17){\rule{2.100pt}{0.400pt}}
\multiput(271.00,293.17)(5.641,2.000){2}{\rule{1.050pt}{0.400pt}}
\multiput(281.00,294.95)(2.025,-0.447){3}{\rule{1.433pt}{0.108pt}}
\multiput(281.00,295.17)(7.025,-3.000){2}{\rule{0.717pt}{0.400pt}}
\put(291,291.67){\rule{2.409pt}{0.400pt}}
\multiput(291.00,292.17)(5.000,-1.000){2}{\rule{1.204pt}{0.400pt}}
\multiput(301.58,292.00)(0.491,1.642){17}{\rule{0.118pt}{1.380pt}}
\multiput(300.17,292.00)(10.000,29.136){2}{\rule{0.400pt}{0.690pt}}
\put(311,324.17){\rule{2.100pt}{0.400pt}}
\multiput(311.00,323.17)(5.641,2.000){2}{\rule{1.050pt}{0.400pt}}
\multiput(321.58,326.00)(0.491,0.808){17}{\rule{0.118pt}{0.740pt}}
\multiput(320.17,326.00)(10.000,14.464){2}{\rule{0.400pt}{0.370pt}}
\multiput(331.58,335.94)(0.491,-1.746){17}{\rule{0.118pt}{1.460pt}}
\multiput(330.17,338.97)(10.000,-30.970){2}{\rule{0.400pt}{0.730pt}}
\multiput(341.58,303.43)(0.491,-1.277){17}{\rule{0.118pt}{1.100pt}}
\multiput(340.17,305.72)(10.000,-22.717){2}{\rule{0.400pt}{0.550pt}}
\multiput(351.58,277.44)(0.491,-1.590){17}{\rule{0.118pt}{1.340pt}}
\multiput(350.17,280.22)(10.000,-28.219){2}{\rule{0.400pt}{0.670pt}}
\multiput(361.58,248.43)(0.491,-0.964){17}{\rule{0.118pt}{0.860pt}}
\multiput(360.17,250.22)(10.000,-17.215){2}{\rule{0.400pt}{0.430pt}}
\multiput(371.00,233.59)(0.553,0.489){15}{\rule{0.544pt}{0.118pt}}
\multiput(371.00,232.17)(8.870,9.000){2}{\rule{0.272pt}{0.400pt}}
\multiput(381.00,242.60)(1.358,0.468){5}{\rule{1.100pt}{0.113pt}}
\multiput(381.00,241.17)(7.717,4.000){2}{\rule{0.550pt}{0.400pt}}
\multiput(391.58,246.00)(0.491,0.704){17}{\rule{0.118pt}{0.660pt}}
\multiput(390.17,246.00)(10.000,12.630){2}{\rule{0.400pt}{0.330pt}}
\multiput(401.58,260.00)(0.491,1.069){17}{\rule{0.118pt}{0.940pt}}
\multiput(400.17,260.00)(10.000,19.049){2}{\rule{0.400pt}{0.470pt}}
\multiput(411.00,279.93)(0.626,-0.488){13}{\rule{0.600pt}{0.117pt}}
\multiput(411.00,280.17)(8.755,-8.000){2}{\rule{0.300pt}{0.400pt}}
\multiput(421.00,273.58)(0.495,0.491){17}{\rule{0.500pt}{0.118pt}}
\multiput(421.00,272.17)(8.962,10.000){2}{\rule{0.250pt}{0.400pt}}
\multiput(431.58,283.00)(0.491,1.069){17}{\rule{0.118pt}{0.940pt}}
\multiput(430.17,283.00)(10.000,19.049){2}{\rule{0.400pt}{0.470pt}}
\multiput(441.58,304.00)(0.491,0.756){17}{\rule{0.118pt}{0.700pt}}
\multiput(440.17,304.00)(10.000,13.547){2}{\rule{0.400pt}{0.350pt}}
\multiput(451.58,311.11)(0.491,-2.320){17}{\rule{0.118pt}{1.900pt}}
\multiput(450.17,315.06)(10.000,-41.056){2}{\rule{0.400pt}{0.950pt}}
\multiput(461.00,274.59)(0.852,0.482){9}{\rule{0.767pt}{0.116pt}}
\multiput(461.00,273.17)(8.409,6.000){2}{\rule{0.383pt}{0.400pt}}
\multiput(471.58,280.00)(0.491,0.808){17}{\rule{0.118pt}{0.740pt}}
\multiput(470.17,280.00)(10.000,14.464){2}{\rule{0.400pt}{0.370pt}}
\multiput(481.58,296.00)(0.491,0.964){17}{\rule{0.118pt}{0.860pt}}
\multiput(480.17,296.00)(10.000,17.215){2}{\rule{0.400pt}{0.430pt}}
\multiput(491.58,315.00)(0.491,0.652){17}{\rule{0.118pt}{0.620pt}}
\multiput(490.17,315.00)(10.000,11.713){2}{\rule{0.400pt}{0.310pt}}
\multiput(501.59,328.00)(0.489,0.844){15}{\rule{0.118pt}{0.767pt}}
\multiput(500.17,328.00)(9.000,13.409){2}{\rule{0.400pt}{0.383pt}}
\multiput(510.58,343.00)(0.491,0.704){17}{\rule{0.118pt}{0.660pt}}
\multiput(509.17,343.00)(10.000,12.630){2}{\rule{0.400pt}{0.330pt}}
\multiput(520.58,357.00)(0.491,0.547){17}{\rule{0.118pt}{0.540pt}}
\multiput(519.17,357.00)(10.000,9.879){2}{\rule{0.400pt}{0.270pt}}
\multiput(530.00,368.59)(1.044,0.477){7}{\rule{0.900pt}{0.115pt}}
\multiput(530.00,367.17)(8.132,5.000){2}{\rule{0.450pt}{0.400pt}}
\multiput(540.58,373.00)(0.491,0.704){17}{\rule{0.118pt}{0.660pt}}
\multiput(539.17,373.00)(10.000,12.630){2}{\rule{0.400pt}{0.330pt}}
\multiput(550.00,387.59)(0.721,0.485){11}{\rule{0.671pt}{0.117pt}}
\multiput(550.00,386.17)(8.606,7.000){2}{\rule{0.336pt}{0.400pt}}
\multiput(560.58,394.00)(0.491,0.912){17}{\rule{0.118pt}{0.820pt}}
\multiput(559.17,394.00)(10.000,16.298){2}{\rule{0.400pt}{0.410pt}}
\multiput(570.00,412.59)(0.553,0.489){15}{\rule{0.544pt}{0.118pt}}
\multiput(570.00,411.17)(8.870,9.000){2}{\rule{0.272pt}{0.400pt}}
\multiput(580.58,421.00)(0.491,1.017){17}{\rule{0.118pt}{0.900pt}}
\multiput(579.17,421.00)(10.000,18.132){2}{\rule{0.400pt}{0.450pt}}
\multiput(590.00,441.59)(0.852,0.482){9}{\rule{0.767pt}{0.116pt}}
\multiput(590.00,440.17)(8.409,6.000){2}{\rule{0.383pt}{0.400pt}}
\multiput(600.58,447.00)(0.491,0.756){17}{\rule{0.118pt}{0.700pt}}
\multiput(599.17,447.00)(10.000,13.547){2}{\rule{0.400pt}{0.350pt}}
\multiput(610.58,462.00)(0.491,0.600){17}{\rule{0.118pt}{0.580pt}}
\multiput(609.17,462.00)(10.000,10.796){2}{\rule{0.400pt}{0.290pt}}
\multiput(620.00,474.58)(0.495,0.491){17}{\rule{0.500pt}{0.118pt}}
\multiput(620.00,473.17)(8.962,10.000){2}{\rule{0.250pt}{0.400pt}}
\multiput(630.00,484.59)(0.553,0.489){15}{\rule{0.544pt}{0.118pt}}
\multiput(630.00,483.17)(8.870,9.000){2}{\rule{0.272pt}{0.400pt}}
\multiput(640.58,493.00)(0.491,0.600){17}{\rule{0.118pt}{0.580pt}}
\multiput(639.17,493.00)(10.000,10.796){2}{\rule{0.400pt}{0.290pt}}
\multiput(650.00,505.59)(0.852,0.482){9}{\rule{0.767pt}{0.116pt}}
\multiput(650.00,504.17)(8.409,6.000){2}{\rule{0.383pt}{0.400pt}}
\multiput(660.00,511.59)(0.553,0.489){15}{\rule{0.544pt}{0.118pt}}
\multiput(660.00,510.17)(8.870,9.000){2}{\rule{0.272pt}{0.400pt}}
\multiput(670.00,520.59)(0.852,0.482){9}{\rule{0.767pt}{0.116pt}}
\multiput(670.00,519.17)(8.409,6.000){2}{\rule{0.383pt}{0.400pt}}
\multiput(680.58,526.00)(0.491,0.860){17}{\rule{0.118pt}{0.780pt}}
\multiput(679.17,526.00)(10.000,15.381){2}{\rule{0.400pt}{0.390pt}}
\multiput(690.00,543.59)(0.553,0.489){15}{\rule{0.544pt}{0.118pt}}
\multiput(690.00,542.17)(8.870,9.000){2}{\rule{0.272pt}{0.400pt}}
\multiput(700.58,552.00)(0.491,0.808){17}{\rule{0.118pt}{0.740pt}}
\multiput(699.17,552.00)(10.000,14.464){2}{\rule{0.400pt}{0.370pt}}
\put(710,567.67){\rule{2.409pt}{0.400pt}}
\multiput(710.00,567.17)(5.000,1.000){2}{\rule{1.204pt}{0.400pt}}
\multiput(720.00,569.59)(0.721,0.485){11}{\rule{0.671pt}{0.117pt}}
\multiput(720.00,568.17)(8.606,7.000){2}{\rule{0.336pt}{0.400pt}}
\multiput(730.58,576.00)(0.491,0.600){17}{\rule{0.118pt}{0.580pt}}
\multiput(729.17,576.00)(10.000,10.796){2}{\rule{0.400pt}{0.290pt}}
\put(740,587.67){\rule{2.409pt}{0.400pt}}
\multiput(740.00,587.17)(5.000,1.000){2}{\rule{1.204pt}{0.400pt}}
\multiput(750.00,589.58)(0.495,0.491){17}{\rule{0.500pt}{0.118pt}}
\multiput(750.00,588.17)(8.962,10.000){2}{\rule{0.250pt}{0.400pt}}
\put(760,598.67){\rule{2.409pt}{0.400pt}}
\multiput(760.00,598.17)(5.000,1.000){2}{\rule{1.204pt}{0.400pt}}
\multiput(770.58,600.00)(0.491,0.652){17}{\rule{0.118pt}{0.620pt}}
\multiput(769.17,600.00)(10.000,11.713){2}{\rule{0.400pt}{0.310pt}}
\multiput(780.00,613.59)(0.553,0.489){15}{\rule{0.544pt}{0.118pt}}
\multiput(780.00,612.17)(8.870,9.000){2}{\rule{0.272pt}{0.400pt}}
\multiput(790.00,622.58)(1.114,0.499){123}{\rule{0.989pt}{0.120pt}}
\multiput(790.00,621.17)(137.948,63.000){2}{\rule{0.494pt}{0.400pt}}
\multiput(930.00,685.59)(0.721,0.485){11}{\rule{0.671pt}{0.117pt}}
\multiput(930.00,684.17)(8.606,7.000){2}{\rule{0.336pt}{0.400pt}}
\put(940,690.67){\rule{2.409pt}{0.400pt}}
\multiput(940.00,691.17)(5.000,-1.000){2}{\rule{1.204pt}{0.400pt}}
\multiput(950.00,691.59)(0.852,0.482){9}{\rule{0.767pt}{0.116pt}}
\multiput(950.00,690.17)(8.409,6.000){2}{\rule{0.383pt}{0.400pt}}
\put(960,697.17){\rule{2.100pt}{0.400pt}}
\multiput(960.00,696.17)(5.641,2.000){2}{\rule{1.050pt}{0.400pt}}
\put(191.0,317.0){\rule[-0.200pt]{2.409pt}{0.400pt}}
\multiput(980.00,699.59)(1.044,0.477){7}{\rule{0.900pt}{0.115pt}}
\multiput(980.00,698.17)(8.132,5.000){2}{\rule{0.450pt}{0.400pt}}
\put(990,704.17){\rule{2.100pt}{0.400pt}}
\multiput(990.00,703.17)(5.641,2.000){2}{\rule{1.050pt}{0.400pt}}
\multiput(1000.00,706.60)(1.358,0.468){5}{\rule{1.100pt}{0.113pt}}
\multiput(1000.00,705.17)(7.717,4.000){2}{\rule{0.550pt}{0.400pt}}
\put(970.0,699.0){\rule[-0.200pt]{2.409pt}{0.400pt}}
\put(1020,709.67){\rule{2.409pt}{0.400pt}}
\multiput(1020.00,709.17)(5.000,1.000){2}{\rule{1.204pt}{0.400pt}}
\put(1030,710.67){\rule{2.409pt}{0.400pt}}
\multiput(1030.00,710.17)(5.000,1.000){2}{\rule{1.204pt}{0.400pt}}
\put(1010.0,710.0){\rule[-0.200pt]{2.409pt}{0.400pt}}
\multiput(1050.58,712.00)(0.491,0.547){17}{\rule{0.118pt}{0.540pt}}
\multiput(1049.17,712.00)(10.000,9.879){2}{\rule{0.400pt}{0.270pt}}
\multiput(1060.00,723.59)(0.852,0.482){9}{\rule{0.767pt}{0.116pt}}
\multiput(1060.00,722.17)(8.409,6.000){2}{\rule{0.383pt}{0.400pt}}
\multiput(1070.00,727.93)(0.852,-0.482){9}{\rule{0.767pt}{0.116pt}}
\multiput(1070.00,728.17)(8.409,-6.000){2}{\rule{0.383pt}{0.400pt}}
\multiput(1080.00,723.61)(2.025,0.447){3}{\rule{1.433pt}{0.108pt}}
\multiput(1080.00,722.17)(7.025,3.000){2}{\rule{0.717pt}{0.400pt}}
\multiput(1090.00,726.59)(0.721,0.485){11}{\rule{0.671pt}{0.117pt}}
\multiput(1090.00,725.17)(8.606,7.000){2}{\rule{0.336pt}{0.400pt}}
\multiput(1100.00,731.95)(2.025,-0.447){3}{\rule{1.433pt}{0.108pt}}
\multiput(1100.00,732.17)(7.025,-3.000){2}{\rule{0.717pt}{0.400pt}}
\multiput(1110.00,728.94)(1.358,-0.468){5}{\rule{1.100pt}{0.113pt}}
\multiput(1110.00,729.17)(7.717,-4.000){2}{\rule{0.550pt}{0.400pt}}
\multiput(1120.00,726.59)(0.645,0.485){11}{\rule{0.614pt}{0.117pt}}
\multiput(1120.00,725.17)(7.725,7.000){2}{\rule{0.307pt}{0.400pt}}
\multiput(1129.00,733.61)(2.025,0.447){3}{\rule{1.433pt}{0.108pt}}
\multiput(1129.00,732.17)(7.025,3.000){2}{\rule{0.717pt}{0.400pt}}
\put(1040.0,712.0){\rule[-0.200pt]{2.409pt}{0.400pt}}
\multiput(1149.00,736.60)(1.358,0.468){5}{\rule{1.100pt}{0.113pt}}
\multiput(1149.00,735.17)(7.717,4.000){2}{\rule{0.550pt}{0.400pt}}
\put(1159,739.67){\rule{2.409pt}{0.400pt}}
\multiput(1159.00,739.17)(5.000,1.000){2}{\rule{1.204pt}{0.400pt}}
\multiput(1169.00,741.60)(1.358,0.468){5}{\rule{1.100pt}{0.113pt}}
\multiput(1169.00,740.17)(7.717,4.000){2}{\rule{0.550pt}{0.400pt}}
\put(1179,743.17){\rule{2.100pt}{0.400pt}}
\multiput(1179.00,744.17)(5.641,-2.000){2}{\rule{1.050pt}{0.400pt}}
\multiput(1189.00,743.59)(0.626,0.488){13}{\rule{0.600pt}{0.117pt}}
\multiput(1189.00,742.17)(8.755,8.000){2}{\rule{0.300pt}{0.400pt}}
\multiput(1199.00,749.93)(1.044,-0.477){7}{\rule{0.900pt}{0.115pt}}
\multiput(1199.00,750.17)(8.132,-5.000){2}{\rule{0.450pt}{0.400pt}}
\put(1209,745.67){\rule{2.409pt}{0.400pt}}
\multiput(1209.00,745.17)(5.000,1.000){2}{\rule{1.204pt}{0.400pt}}
\multiput(1219.00,745.93)(1.044,-0.477){7}{\rule{0.900pt}{0.115pt}}
\multiput(1219.00,746.17)(8.132,-5.000){2}{\rule{0.450pt}{0.400pt}}
\multiput(1229.58,742.00)(0.491,0.600){17}{\rule{0.118pt}{0.580pt}}
\multiput(1228.17,742.00)(10.000,10.796){2}{\rule{0.400pt}{0.290pt}}
\put(1139.0,736.0){\rule[-0.200pt]{2.409pt}{0.400pt}}
\put(1249,754.17){\rule{2.100pt}{0.400pt}}
\multiput(1249.00,753.17)(5.641,2.000){2}{\rule{1.050pt}{0.400pt}}
\put(1259,754.67){\rule{2.409pt}{0.400pt}}
\multiput(1259.00,755.17)(5.000,-1.000){2}{\rule{1.204pt}{0.400pt}}
\put(1269,754.67){\rule{2.409pt}{0.400pt}}
\multiput(1269.00,754.17)(5.000,1.000){2}{\rule{1.204pt}{0.400pt}}
\multiput(1279.00,756.60)(1.358,0.468){5}{\rule{1.100pt}{0.113pt}}
\multiput(1279.00,755.17)(7.717,4.000){2}{\rule{0.550pt}{0.400pt}}
\multiput(1289.00,758.94)(1.358,-0.468){5}{\rule{1.100pt}{0.113pt}}
\multiput(1289.00,759.17)(7.717,-4.000){2}{\rule{0.550pt}{0.400pt}}
\put(1299,756.17){\rule{2.100pt}{0.400pt}}
\multiput(1299.00,755.17)(5.641,2.000){2}{\rule{1.050pt}{0.400pt}}
\multiput(1309.00,758.60)(1.358,0.468){5}{\rule{1.100pt}{0.113pt}}
\multiput(1309.00,757.17)(7.717,4.000){2}{\rule{0.550pt}{0.400pt}}
\put(1239.0,754.0){\rule[-0.200pt]{2.409pt}{0.400pt}}
\put(1339,762.17){\rule{2.100pt}{0.400pt}}
\multiput(1339.00,761.17)(5.641,2.000){2}{\rule{1.050pt}{0.400pt}}
\put(1319.0,762.0){\rule[-0.200pt]{4.818pt}{0.400pt}}
\put(191.0,131.0){\rule[-0.200pt]{0.400pt}{155.380pt}}
\put(191.0,131.0){\rule[-0.200pt]{300.643pt}{0.400pt}}
\put(1439.0,131.0){\rule[-0.200pt]{0.400pt}{155.380pt}}
\put(191.0,776.0){\rule[-0.200pt]{300.643pt}{0.400pt}}
\end{picture}
}
  \end	{center}
  \caption{Example plot of pH test data}
\label{fig:pH_plot}
\end {figure}
\FloatBarrier

The following table contains the (erroneous) initial rates extrapolated from all the performed pH trials. Below the table is the calculated rate equation from this data.
\FloatBarrier
\begin{figure}[h!]
	\begin{center}
	\renewcommand\arraystretch{1.5}
	\renewcommand\tabcolsep{12pt}
		\begin{tabular}{|c|c|c|c|}
			\hline 
			Trial & Mg Mass & HCl M & Initial Rate $\frac{-\Delta pH}{2 \Delta t}$\\
			\hline 
			1 & 0.04g & 1M & -0.0025\\
	 		\hline
			2 & 0.02g & 1M & -0.00167\\
	 		\hline
			3 & 0.04g & 0.5M & -0.00167\\
	 		\hline			 			 		
		\end{tabular}
	\end{center}
	\label{fig:pHtab}
	\caption{Results of trials including initial rates}
\end{figure}
\FloatBarrier

\begin{equation}
\textnormal{rate}  = 0.0025 (\textnormal{grams Mg})^0 [\textnormal{HCl}]^{0.58}
\label{eq:pHrate}
\end{equation}
It should be noted that this rate law is far from expected for the rate constants, where Mg was seen to be negligible with an order of 0. 


\subsection{Pressure Test}
Below are the plots for pressure versus time for the four pressure probe trials, each varying in Mg masses, HCl concentrations and temperatures. This data is notably less turbulent than the pH trials, making the initial rate easier to extrapolate.
\FloatBarrier
\begin{figure}[h!]
\centering
\begin{minipage}{0.5\textwidth}
		\raggedright
	   	\resizebox{0.9\textwidth}{!}{% GNUPLOT: LaTeX picture
\setlength{\unitlength}{0.240900pt}
\ifx\plotpoint\undefined\newsavebox{\plotpoint}\fi
\sbox{\plotpoint}{\rule[-0.200pt]{0.400pt}{0.400pt}}%
\begin{picture}(1500,900)(0,0)
\sbox{\plotpoint}{\rule[-0.200pt]{0.400pt}{0.400pt}}%
\put(191.0,131.0){\rule[-0.200pt]{4.818pt}{0.400pt}}
\put(171,131){\makebox(0,0)[r]{ 0.94}}
\put(1419.0,131.0){\rule[-0.200pt]{4.818pt}{0.400pt}}
\put(191.0,196.0){\rule[-0.200pt]{4.818pt}{0.400pt}}
\put(171,196){\makebox(0,0)[r]{ 0.96}}
\put(1419.0,196.0){\rule[-0.200pt]{4.818pt}{0.400pt}}
\put(191.0,260.0){\rule[-0.200pt]{4.818pt}{0.400pt}}
\put(171,260){\makebox(0,0)[r]{ 0.98}}
\put(1419.0,260.0){\rule[-0.200pt]{4.818pt}{0.400pt}}
\put(191.0,325.0){\rule[-0.200pt]{4.818pt}{0.400pt}}
\put(171,325){\makebox(0,0)[r]{ 1}}
\put(1419.0,325.0){\rule[-0.200pt]{4.818pt}{0.400pt}}
\put(191.0,389.0){\rule[-0.200pt]{4.818pt}{0.400pt}}
\put(171,389){\makebox(0,0)[r]{ 1.02}}
\put(1419.0,389.0){\rule[-0.200pt]{4.818pt}{0.400pt}}
\put(191.0,454.0){\rule[-0.200pt]{4.818pt}{0.400pt}}
\put(171,454){\makebox(0,0)[r]{ 1.04}}
\put(1419.0,454.0){\rule[-0.200pt]{4.818pt}{0.400pt}}
\put(191.0,518.0){\rule[-0.200pt]{4.818pt}{0.400pt}}
\put(171,518){\makebox(0,0)[r]{ 1.06}}
\put(1419.0,518.0){\rule[-0.200pt]{4.818pt}{0.400pt}}
\put(191.0,583.0){\rule[-0.200pt]{4.818pt}{0.400pt}}
\put(171,583){\makebox(0,0)[r]{ 1.08}}
\put(1419.0,583.0){\rule[-0.200pt]{4.818pt}{0.400pt}}
\put(191.0,647.0){\rule[-0.200pt]{4.818pt}{0.400pt}}
\put(171,647){\makebox(0,0)[r]{ 1.1}}
\put(1419.0,647.0){\rule[-0.200pt]{4.818pt}{0.400pt}}
\put(191.0,712.0){\rule[-0.200pt]{4.818pt}{0.400pt}}
\put(171,712){\makebox(0,0)[r]{ 1.12}}
\put(1419.0,712.0){\rule[-0.200pt]{4.818pt}{0.400pt}}
\put(191.0,776.0){\rule[-0.200pt]{4.818pt}{0.400pt}}
\put(171,776){\makebox(0,0)[r]{ 1.14}}
\put(1419.0,776.0){\rule[-0.200pt]{4.818pt}{0.400pt}}
\put(191.0,131.0){\rule[-0.200pt]{0.400pt}{4.818pt}}
\put(191,90){\makebox(0,0){ 0}}
\put(191.0,756.0){\rule[-0.200pt]{0.400pt}{4.818pt}}
\put(347.0,131.0){\rule[-0.200pt]{0.400pt}{4.818pt}}
\put(347,90){\makebox(0,0){ 5}}
\put(347.0,756.0){\rule[-0.200pt]{0.400pt}{4.818pt}}
\put(503.0,131.0){\rule[-0.200pt]{0.400pt}{4.818pt}}
\put(503,90){\makebox(0,0){ 10}}
\put(503.0,756.0){\rule[-0.200pt]{0.400pt}{4.818pt}}
\put(659.0,131.0){\rule[-0.200pt]{0.400pt}{4.818pt}}
\put(659,90){\makebox(0,0){ 15}}
\put(659.0,756.0){\rule[-0.200pt]{0.400pt}{4.818pt}}
\put(815.0,131.0){\rule[-0.200pt]{0.400pt}{4.818pt}}
\put(815,90){\makebox(0,0){ 20}}
\put(815.0,756.0){\rule[-0.200pt]{0.400pt}{4.818pt}}
\put(971.0,131.0){\rule[-0.200pt]{0.400pt}{4.818pt}}
\put(971,90){\makebox(0,0){ 25}}
\put(971.0,756.0){\rule[-0.200pt]{0.400pt}{4.818pt}}
\put(1127.0,131.0){\rule[-0.200pt]{0.400pt}{4.818pt}}
\put(1127,90){\makebox(0,0){ 30}}
\put(1127.0,756.0){\rule[-0.200pt]{0.400pt}{4.818pt}}
\put(1283.0,131.0){\rule[-0.200pt]{0.400pt}{4.818pt}}
\put(1283,90){\makebox(0,0){ 35}}
\put(1283.0,756.0){\rule[-0.200pt]{0.400pt}{4.818pt}}
\put(1439.0,131.0){\rule[-0.200pt]{0.400pt}{4.818pt}}
\put(1439,90){\makebox(0,0){ 40}}
\put(1439.0,756.0){\rule[-0.200pt]{0.400pt}{4.818pt}}
\put(191.0,131.0){\rule[-0.200pt]{0.400pt}{155.380pt}}
\put(191.0,131.0){\rule[-0.200pt]{300.643pt}{0.400pt}}
\put(1439.0,131.0){\rule[-0.200pt]{0.400pt}{155.380pt}}
\put(191.0,776.0){\rule[-0.200pt]{300.643pt}{0.400pt}}
\put(30,453){\makebox(0,0){\hspace{-72pt} Pressure (atm)}}
\put(815,29){\makebox(0,0){Time (seconds)}}
\put(815,838){\makebox(0,0){0.04g Mg Powder - 10mL 1M HCl Reaction Pressure v. Time}}
\put(191,192){\rule{1pt}{1pt}}
\put(199,194){\rule{1pt}{1pt}}
\put(207,194){\rule{1pt}{1pt}}
\put(214,194){\rule{1pt}{1pt}}
\put(222,194){\rule{1pt}{1pt}}
\put(230,192){\rule{1pt}{1pt}}
\put(238,194){\rule{1pt}{1pt}}
\put(246,194){\rule{1pt}{1pt}}
\put(253,194){\rule{1pt}{1pt}}
\put(261,194){\rule{1pt}{1pt}}
\put(269,194){\rule{1pt}{1pt}}
\put(277,194){\rule{1pt}{1pt}}
\put(285,194){\rule{1pt}{1pt}}
\put(292,194){\rule{1pt}{1pt}}
\put(300,194){\rule{1pt}{1pt}}
\put(308,194){\rule{1pt}{1pt}}
\put(316,194){\rule{1pt}{1pt}}
\put(324,194){\rule{1pt}{1pt}}
\put(331,192){\rule{1pt}{1pt}}
\put(339,194){\rule{1pt}{1pt}}
\put(347,194){\rule{1pt}{1pt}}
\put(355,194){\rule{1pt}{1pt}}
\put(363,245){\rule{1pt}{1pt}}
\put(370,275){\rule{1pt}{1pt}}
\put(378,301){\rule{1pt}{1pt}}
\put(386,318){\rule{1pt}{1pt}}
\put(394,337){\rule{1pt}{1pt}}
\put(402,352){\rule{1pt}{1pt}}
\put(409,363){\rule{1pt}{1pt}}
\put(417,378){\rule{1pt}{1pt}}
\put(425,393){\rule{1pt}{1pt}}
\put(433,410){\rule{1pt}{1pt}}
\put(441,429){\rule{1pt}{1pt}}
\put(448,448){\rule{1pt}{1pt}}
\put(456,467){\rule{1pt}{1pt}}
\put(464,482){\rule{1pt}{1pt}}
\put(472,497){\rule{1pt}{1pt}}
\put(480,508){\rule{1pt}{1pt}}
\put(487,517){\rule{1pt}{1pt}}
\put(495,525){\rule{1pt}{1pt}}
\put(503,533){\rule{1pt}{1pt}}
\put(511,536){\rule{1pt}{1pt}}
\put(519,544){\rule{1pt}{1pt}}
\put(526,548){\rule{1pt}{1pt}}
\put(534,551){\rule{1pt}{1pt}}
\put(542,559){\rule{1pt}{1pt}}
\put(550,568){\rule{1pt}{1pt}}
\put(558,580){\rule{1pt}{1pt}}
\put(565,587){\rule{1pt}{1pt}}
\put(573,595){\rule{1pt}{1pt}}
\put(581,600){\rule{1pt}{1pt}}
\put(589,608){\rule{1pt}{1pt}}
\put(597,613){\rule{1pt}{1pt}}
\put(604,617){\rule{1pt}{1pt}}
\put(612,621){\rule{1pt}{1pt}}
\put(620,625){\rule{1pt}{1pt}}
\put(628,630){\rule{1pt}{1pt}}
\put(636,632){\rule{1pt}{1pt}}
\put(643,634){\rule{1pt}{1pt}}
\put(651,636){\rule{1pt}{1pt}}
\put(659,640){\rule{1pt}{1pt}}
\put(667,643){\rule{1pt}{1pt}}
\put(675,645){\rule{1pt}{1pt}}
\put(682,649){\rule{1pt}{1pt}}
\put(690,651){\rule{1pt}{1pt}}
\put(698,655){\rule{1pt}{1pt}}
\put(706,655){\rule{1pt}{1pt}}
\put(714,657){\rule{1pt}{1pt}}
\put(721,659){\rule{1pt}{1pt}}
\put(729,660){\rule{1pt}{1pt}}
\put(737,664){\rule{1pt}{1pt}}
\put(745,664){\rule{1pt}{1pt}}
\put(753,666){\rule{1pt}{1pt}}
\put(760,666){\rule{1pt}{1pt}}
\put(768,670){\rule{1pt}{1pt}}
\put(776,670){\rule{1pt}{1pt}}
\put(784,670){\rule{1pt}{1pt}}
\put(792,670){\rule{1pt}{1pt}}
\put(799,672){\rule{1pt}{1pt}}
\put(807,674){\rule{1pt}{1pt}}
\put(815,675){\rule{1pt}{1pt}}
\put(823,677){\rule{1pt}{1pt}}
\put(831,679){\rule{1pt}{1pt}}
\put(838,681){\rule{1pt}{1pt}}
\put(846,681){\rule{1pt}{1pt}}
\put(854,683){\rule{1pt}{1pt}}
\put(862,683){\rule{1pt}{1pt}}
\put(870,685){\rule{1pt}{1pt}}
\put(877,685){\rule{1pt}{1pt}}
\put(885,687){\rule{1pt}{1pt}}
\put(893,687){\rule{1pt}{1pt}}
\put(901,689){\rule{1pt}{1pt}}
\put(909,690){\rule{1pt}{1pt}}
\put(916,692){\rule{1pt}{1pt}}
\put(924,692){\rule{1pt}{1pt}}
\put(932,694){\rule{1pt}{1pt}}
\put(940,694){\rule{1pt}{1pt}}
\put(948,694){\rule{1pt}{1pt}}
\put(955,696){\rule{1pt}{1pt}}
\put(963,696){\rule{1pt}{1pt}}
\put(971,696){\rule{1pt}{1pt}}
\put(979,698){\rule{1pt}{1pt}}
\put(987,696){\rule{1pt}{1pt}}
\put(994,698){\rule{1pt}{1pt}}
\put(1002,698){\rule{1pt}{1pt}}
\put(1010,700){\rule{1pt}{1pt}}
\put(1018,700){\rule{1pt}{1pt}}
\put(1026,700){\rule{1pt}{1pt}}
\put(1033,702){\rule{1pt}{1pt}}
\put(1041,702){\rule{1pt}{1pt}}
\put(1049,704){\rule{1pt}{1pt}}
\put(1057,704){\rule{1pt}{1pt}}
\put(1065,704){\rule{1pt}{1pt}}
\put(1072,705){\rule{1pt}{1pt}}
\put(1080,705){\rule{1pt}{1pt}}
\put(1088,705){\rule{1pt}{1pt}}
\put(1096,705){\rule{1pt}{1pt}}
\put(1104,705){\rule{1pt}{1pt}}
\put(1111,707){\rule{1pt}{1pt}}
\put(1119,705){\rule{1pt}{1pt}}
\put(1127,707){\rule{1pt}{1pt}}
\put(1135,707){\rule{1pt}{1pt}}
\put(1143,707){\rule{1pt}{1pt}}
\put(1150,709){\rule{1pt}{1pt}}
\put(1158,707){\rule{1pt}{1pt}}
\put(1166,709){\rule{1pt}{1pt}}
\put(1174,711){\rule{1pt}{1pt}}
\put(1182,711){\rule{1pt}{1pt}}
\put(1189,709){\rule{1pt}{1pt}}
\put(1197,711){\rule{1pt}{1pt}}
\put(1205,713){\rule{1pt}{1pt}}
\put(1213,711){\rule{1pt}{1pt}}
\put(1221,711){\rule{1pt}{1pt}}
\put(1228,711){\rule{1pt}{1pt}}
\put(1236,713){\rule{1pt}{1pt}}
\put(1244,711){\rule{1pt}{1pt}}
\put(1252,713){\rule{1pt}{1pt}}
\put(1260,713){\rule{1pt}{1pt}}
\put(1267,713){\rule{1pt}{1pt}}
\put(1275,713){\rule{1pt}{1pt}}
\put(1283,715){\rule{1pt}{1pt}}
\put(1291,715){\rule{1pt}{1pt}}
\put(1299,717){\rule{1pt}{1pt}}
\put(1306,717){\rule{1pt}{1pt}}
\put(1314,717){\rule{1pt}{1pt}}
\put(1322,719){\rule{1pt}{1pt}}
\put(1330,717){\rule{1pt}{1pt}}
\put(1338,717){\rule{1pt}{1pt}}
\put(1345,719){\rule{1pt}{1pt}}
\put(1353,721){\rule{1pt}{1pt}}
\put(1361,719){\rule{1pt}{1pt}}
\put(191,192){\usebox{\plotpoint}}
\put(191,192.17){\rule{1.700pt}{0.400pt}}
\multiput(191.00,191.17)(4.472,2.000){2}{\rule{0.850pt}{0.400pt}}
\put(222,192.17){\rule{1.700pt}{0.400pt}}
\multiput(222.00,193.17)(4.472,-2.000){2}{\rule{0.850pt}{0.400pt}}
\put(230,192.17){\rule{1.700pt}{0.400pt}}
\multiput(230.00,191.17)(4.472,2.000){2}{\rule{0.850pt}{0.400pt}}
\put(199.0,194.0){\rule[-0.200pt]{5.541pt}{0.400pt}}
\put(324,192.17){\rule{1.500pt}{0.400pt}}
\multiput(324.00,193.17)(3.887,-2.000){2}{\rule{0.750pt}{0.400pt}}
\put(331,192.17){\rule{1.700pt}{0.400pt}}
\multiput(331.00,191.17)(4.472,2.000){2}{\rule{0.850pt}{0.400pt}}
\put(238.0,194.0){\rule[-0.200pt]{20.717pt}{0.400pt}}
\multiput(355.59,194.00)(0.488,3.333){13}{\rule{0.117pt}{2.650pt}}
\multiput(354.17,194.00)(8.000,45.500){2}{\rule{0.400pt}{1.325pt}}
\multiput(363.59,245.00)(0.485,2.247){11}{\rule{0.117pt}{1.814pt}}
\multiput(362.17,245.00)(7.000,26.234){2}{\rule{0.400pt}{0.907pt}}
\multiput(370.59,275.00)(0.488,1.682){13}{\rule{0.117pt}{1.400pt}}
\multiput(369.17,275.00)(8.000,23.094){2}{\rule{0.400pt}{0.700pt}}
\multiput(378.59,301.00)(0.488,1.088){13}{\rule{0.117pt}{0.950pt}}
\multiput(377.17,301.00)(8.000,15.028){2}{\rule{0.400pt}{0.475pt}}
\multiput(386.59,318.00)(0.488,1.220){13}{\rule{0.117pt}{1.050pt}}
\multiput(385.17,318.00)(8.000,16.821){2}{\rule{0.400pt}{0.525pt}}
\multiput(394.59,337.00)(0.488,0.956){13}{\rule{0.117pt}{0.850pt}}
\multiput(393.17,337.00)(8.000,13.236){2}{\rule{0.400pt}{0.425pt}}
\multiput(402.59,352.00)(0.485,0.798){11}{\rule{0.117pt}{0.729pt}}
\multiput(401.17,352.00)(7.000,9.488){2}{\rule{0.400pt}{0.364pt}}
\multiput(409.59,363.00)(0.488,0.956){13}{\rule{0.117pt}{0.850pt}}
\multiput(408.17,363.00)(8.000,13.236){2}{\rule{0.400pt}{0.425pt}}
\multiput(417.59,378.00)(0.488,0.956){13}{\rule{0.117pt}{0.850pt}}
\multiput(416.17,378.00)(8.000,13.236){2}{\rule{0.400pt}{0.425pt}}
\multiput(425.59,393.00)(0.488,1.088){13}{\rule{0.117pt}{0.950pt}}
\multiput(424.17,393.00)(8.000,15.028){2}{\rule{0.400pt}{0.475pt}}
\multiput(433.59,410.00)(0.488,1.220){13}{\rule{0.117pt}{1.050pt}}
\multiput(432.17,410.00)(8.000,16.821){2}{\rule{0.400pt}{0.525pt}}
\multiput(441.59,429.00)(0.485,1.408){11}{\rule{0.117pt}{1.186pt}}
\multiput(440.17,429.00)(7.000,16.539){2}{\rule{0.400pt}{0.593pt}}
\multiput(448.59,448.00)(0.488,1.220){13}{\rule{0.117pt}{1.050pt}}
\multiput(447.17,448.00)(8.000,16.821){2}{\rule{0.400pt}{0.525pt}}
\multiput(456.59,467.00)(0.488,0.956){13}{\rule{0.117pt}{0.850pt}}
\multiput(455.17,467.00)(8.000,13.236){2}{\rule{0.400pt}{0.425pt}}
\multiput(464.59,482.00)(0.488,0.956){13}{\rule{0.117pt}{0.850pt}}
\multiput(463.17,482.00)(8.000,13.236){2}{\rule{0.400pt}{0.425pt}}
\multiput(472.59,497.00)(0.488,0.692){13}{\rule{0.117pt}{0.650pt}}
\multiput(471.17,497.00)(8.000,9.651){2}{\rule{0.400pt}{0.325pt}}
\multiput(480.59,508.00)(0.485,0.645){11}{\rule{0.117pt}{0.614pt}}
\multiput(479.17,508.00)(7.000,7.725){2}{\rule{0.400pt}{0.307pt}}
\multiput(487.00,517.59)(0.494,0.488){13}{\rule{0.500pt}{0.117pt}}
\multiput(487.00,516.17)(6.962,8.000){2}{\rule{0.250pt}{0.400pt}}
\multiput(495.00,525.59)(0.494,0.488){13}{\rule{0.500pt}{0.117pt}}
\multiput(495.00,524.17)(6.962,8.000){2}{\rule{0.250pt}{0.400pt}}
\multiput(503.00,533.61)(1.579,0.447){3}{\rule{1.167pt}{0.108pt}}
\multiput(503.00,532.17)(5.579,3.000){2}{\rule{0.583pt}{0.400pt}}
\multiput(511.00,536.59)(0.494,0.488){13}{\rule{0.500pt}{0.117pt}}
\multiput(511.00,535.17)(6.962,8.000){2}{\rule{0.250pt}{0.400pt}}
\multiput(519.00,544.60)(0.920,0.468){5}{\rule{0.800pt}{0.113pt}}
\multiput(519.00,543.17)(5.340,4.000){2}{\rule{0.400pt}{0.400pt}}
\multiput(526.00,548.61)(1.579,0.447){3}{\rule{1.167pt}{0.108pt}}
\multiput(526.00,547.17)(5.579,3.000){2}{\rule{0.583pt}{0.400pt}}
\multiput(534.00,551.59)(0.494,0.488){13}{\rule{0.500pt}{0.117pt}}
\multiput(534.00,550.17)(6.962,8.000){2}{\rule{0.250pt}{0.400pt}}
\multiput(542.59,559.00)(0.488,0.560){13}{\rule{0.117pt}{0.550pt}}
\multiput(541.17,559.00)(8.000,7.858){2}{\rule{0.400pt}{0.275pt}}
\multiput(550.59,568.00)(0.488,0.758){13}{\rule{0.117pt}{0.700pt}}
\multiput(549.17,568.00)(8.000,10.547){2}{\rule{0.400pt}{0.350pt}}
\multiput(558.00,580.59)(0.492,0.485){11}{\rule{0.500pt}{0.117pt}}
\multiput(558.00,579.17)(5.962,7.000){2}{\rule{0.250pt}{0.400pt}}
\multiput(565.00,587.59)(0.494,0.488){13}{\rule{0.500pt}{0.117pt}}
\multiput(565.00,586.17)(6.962,8.000){2}{\rule{0.250pt}{0.400pt}}
\multiput(573.00,595.59)(0.821,0.477){7}{\rule{0.740pt}{0.115pt}}
\multiput(573.00,594.17)(6.464,5.000){2}{\rule{0.370pt}{0.400pt}}
\multiput(581.00,600.59)(0.494,0.488){13}{\rule{0.500pt}{0.117pt}}
\multiput(581.00,599.17)(6.962,8.000){2}{\rule{0.250pt}{0.400pt}}
\multiput(589.00,608.59)(0.821,0.477){7}{\rule{0.740pt}{0.115pt}}
\multiput(589.00,607.17)(6.464,5.000){2}{\rule{0.370pt}{0.400pt}}
\multiput(597.00,613.60)(0.920,0.468){5}{\rule{0.800pt}{0.113pt}}
\multiput(597.00,612.17)(5.340,4.000){2}{\rule{0.400pt}{0.400pt}}
\multiput(604.00,617.60)(1.066,0.468){5}{\rule{0.900pt}{0.113pt}}
\multiput(604.00,616.17)(6.132,4.000){2}{\rule{0.450pt}{0.400pt}}
\multiput(612.00,621.60)(1.066,0.468){5}{\rule{0.900pt}{0.113pt}}
\multiput(612.00,620.17)(6.132,4.000){2}{\rule{0.450pt}{0.400pt}}
\multiput(620.00,625.59)(0.821,0.477){7}{\rule{0.740pt}{0.115pt}}
\multiput(620.00,624.17)(6.464,5.000){2}{\rule{0.370pt}{0.400pt}}
\put(628,630.17){\rule{1.700pt}{0.400pt}}
\multiput(628.00,629.17)(4.472,2.000){2}{\rule{0.850pt}{0.400pt}}
\put(636,632.17){\rule{1.500pt}{0.400pt}}
\multiput(636.00,631.17)(3.887,2.000){2}{\rule{0.750pt}{0.400pt}}
\put(643,634.17){\rule{1.700pt}{0.400pt}}
\multiput(643.00,633.17)(4.472,2.000){2}{\rule{0.850pt}{0.400pt}}
\multiput(651.00,636.60)(1.066,0.468){5}{\rule{0.900pt}{0.113pt}}
\multiput(651.00,635.17)(6.132,4.000){2}{\rule{0.450pt}{0.400pt}}
\multiput(659.00,640.61)(1.579,0.447){3}{\rule{1.167pt}{0.108pt}}
\multiput(659.00,639.17)(5.579,3.000){2}{\rule{0.583pt}{0.400pt}}
\put(667,643.17){\rule{1.700pt}{0.400pt}}
\multiput(667.00,642.17)(4.472,2.000){2}{\rule{0.850pt}{0.400pt}}
\multiput(675.00,645.60)(0.920,0.468){5}{\rule{0.800pt}{0.113pt}}
\multiput(675.00,644.17)(5.340,4.000){2}{\rule{0.400pt}{0.400pt}}
\put(682,649.17){\rule{1.700pt}{0.400pt}}
\multiput(682.00,648.17)(4.472,2.000){2}{\rule{0.850pt}{0.400pt}}
\multiput(690.00,651.60)(1.066,0.468){5}{\rule{0.900pt}{0.113pt}}
\multiput(690.00,650.17)(6.132,4.000){2}{\rule{0.450pt}{0.400pt}}
\put(339.0,194.0){\rule[-0.200pt]{3.854pt}{0.400pt}}
\put(706,655.17){\rule{1.700pt}{0.400pt}}
\multiput(706.00,654.17)(4.472,2.000){2}{\rule{0.850pt}{0.400pt}}
\put(714,657.17){\rule{1.500pt}{0.400pt}}
\multiput(714.00,656.17)(3.887,2.000){2}{\rule{0.750pt}{0.400pt}}
\put(721,658.67){\rule{1.927pt}{0.400pt}}
\multiput(721.00,658.17)(4.000,1.000){2}{\rule{0.964pt}{0.400pt}}
\multiput(729.00,660.60)(1.066,0.468){5}{\rule{0.900pt}{0.113pt}}
\multiput(729.00,659.17)(6.132,4.000){2}{\rule{0.450pt}{0.400pt}}
\put(698.0,655.0){\rule[-0.200pt]{1.927pt}{0.400pt}}
\put(745,664.17){\rule{1.700pt}{0.400pt}}
\multiput(745.00,663.17)(4.472,2.000){2}{\rule{0.850pt}{0.400pt}}
\put(737.0,664.0){\rule[-0.200pt]{1.927pt}{0.400pt}}
\multiput(760.00,666.60)(1.066,0.468){5}{\rule{0.900pt}{0.113pt}}
\multiput(760.00,665.17)(6.132,4.000){2}{\rule{0.450pt}{0.400pt}}
\put(753.0,666.0){\rule[-0.200pt]{1.686pt}{0.400pt}}
\put(792,670.17){\rule{1.500pt}{0.400pt}}
\multiput(792.00,669.17)(3.887,2.000){2}{\rule{0.750pt}{0.400pt}}
\put(799,672.17){\rule{1.700pt}{0.400pt}}
\multiput(799.00,671.17)(4.472,2.000){2}{\rule{0.850pt}{0.400pt}}
\put(807,673.67){\rule{1.927pt}{0.400pt}}
\multiput(807.00,673.17)(4.000,1.000){2}{\rule{0.964pt}{0.400pt}}
\put(815,675.17){\rule{1.700pt}{0.400pt}}
\multiput(815.00,674.17)(4.472,2.000){2}{\rule{0.850pt}{0.400pt}}
\put(823,677.17){\rule{1.700pt}{0.400pt}}
\multiput(823.00,676.17)(4.472,2.000){2}{\rule{0.850pt}{0.400pt}}
\put(831,679.17){\rule{1.500pt}{0.400pt}}
\multiput(831.00,678.17)(3.887,2.000){2}{\rule{0.750pt}{0.400pt}}
\put(768.0,670.0){\rule[-0.200pt]{5.782pt}{0.400pt}}
\put(846,681.17){\rule{1.700pt}{0.400pt}}
\multiput(846.00,680.17)(4.472,2.000){2}{\rule{0.850pt}{0.400pt}}
\put(838.0,681.0){\rule[-0.200pt]{1.927pt}{0.400pt}}
\put(862,683.17){\rule{1.700pt}{0.400pt}}
\multiput(862.00,682.17)(4.472,2.000){2}{\rule{0.850pt}{0.400pt}}
\put(854.0,683.0){\rule[-0.200pt]{1.927pt}{0.400pt}}
\put(877,685.17){\rule{1.700pt}{0.400pt}}
\multiput(877.00,684.17)(4.472,2.000){2}{\rule{0.850pt}{0.400pt}}
\put(870.0,685.0){\rule[-0.200pt]{1.686pt}{0.400pt}}
\put(893,687.17){\rule{1.700pt}{0.400pt}}
\multiput(893.00,686.17)(4.472,2.000){2}{\rule{0.850pt}{0.400pt}}
\put(901,688.67){\rule{1.927pt}{0.400pt}}
\multiput(901.00,688.17)(4.000,1.000){2}{\rule{0.964pt}{0.400pt}}
\put(909,690.17){\rule{1.500pt}{0.400pt}}
\multiput(909.00,689.17)(3.887,2.000){2}{\rule{0.750pt}{0.400pt}}
\put(885.0,687.0){\rule[-0.200pt]{1.927pt}{0.400pt}}
\put(924,692.17){\rule{1.700pt}{0.400pt}}
\multiput(924.00,691.17)(4.472,2.000){2}{\rule{0.850pt}{0.400pt}}
\put(916.0,692.0){\rule[-0.200pt]{1.927pt}{0.400pt}}
\put(948,694.17){\rule{1.500pt}{0.400pt}}
\multiput(948.00,693.17)(3.887,2.000){2}{\rule{0.750pt}{0.400pt}}
\put(932.0,694.0){\rule[-0.200pt]{3.854pt}{0.400pt}}
\put(971,696.17){\rule{1.700pt}{0.400pt}}
\multiput(971.00,695.17)(4.472,2.000){2}{\rule{0.850pt}{0.400pt}}
\put(979,696.17){\rule{1.700pt}{0.400pt}}
\multiput(979.00,697.17)(4.472,-2.000){2}{\rule{0.850pt}{0.400pt}}
\put(987,696.17){\rule{1.500pt}{0.400pt}}
\multiput(987.00,695.17)(3.887,2.000){2}{\rule{0.750pt}{0.400pt}}
\put(955.0,696.0){\rule[-0.200pt]{3.854pt}{0.400pt}}
\put(1002,698.17){\rule{1.700pt}{0.400pt}}
\multiput(1002.00,697.17)(4.472,2.000){2}{\rule{0.850pt}{0.400pt}}
\put(994.0,698.0){\rule[-0.200pt]{1.927pt}{0.400pt}}
\put(1026,700.17){\rule{1.500pt}{0.400pt}}
\multiput(1026.00,699.17)(3.887,2.000){2}{\rule{0.750pt}{0.400pt}}
\put(1010.0,700.0){\rule[-0.200pt]{3.854pt}{0.400pt}}
\put(1041,702.17){\rule{1.700pt}{0.400pt}}
\multiput(1041.00,701.17)(4.472,2.000){2}{\rule{0.850pt}{0.400pt}}
\put(1033.0,702.0){\rule[-0.200pt]{1.927pt}{0.400pt}}
\put(1065,703.67){\rule{1.686pt}{0.400pt}}
\multiput(1065.00,703.17)(3.500,1.000){2}{\rule{0.843pt}{0.400pt}}
\put(1049.0,704.0){\rule[-0.200pt]{3.854pt}{0.400pt}}
\put(1104,705.17){\rule{1.500pt}{0.400pt}}
\multiput(1104.00,704.17)(3.887,2.000){2}{\rule{0.750pt}{0.400pt}}
\put(1111,705.17){\rule{1.700pt}{0.400pt}}
\multiput(1111.00,706.17)(4.472,-2.000){2}{\rule{0.850pt}{0.400pt}}
\put(1119,705.17){\rule{1.700pt}{0.400pt}}
\multiput(1119.00,704.17)(4.472,2.000){2}{\rule{0.850pt}{0.400pt}}
\put(1072.0,705.0){\rule[-0.200pt]{7.709pt}{0.400pt}}
\put(1143,707.17){\rule{1.500pt}{0.400pt}}
\multiput(1143.00,706.17)(3.887,2.000){2}{\rule{0.750pt}{0.400pt}}
\put(1150,707.17){\rule{1.700pt}{0.400pt}}
\multiput(1150.00,708.17)(4.472,-2.000){2}{\rule{0.850pt}{0.400pt}}
\put(1158,707.17){\rule{1.700pt}{0.400pt}}
\multiput(1158.00,706.17)(4.472,2.000){2}{\rule{0.850pt}{0.400pt}}
\put(1166,709.17){\rule{1.700pt}{0.400pt}}
\multiput(1166.00,708.17)(4.472,2.000){2}{\rule{0.850pt}{0.400pt}}
\put(1127.0,707.0){\rule[-0.200pt]{3.854pt}{0.400pt}}
\put(1182,709.17){\rule{1.500pt}{0.400pt}}
\multiput(1182.00,710.17)(3.887,-2.000){2}{\rule{0.750pt}{0.400pt}}
\put(1189,709.17){\rule{1.700pt}{0.400pt}}
\multiput(1189.00,708.17)(4.472,2.000){2}{\rule{0.850pt}{0.400pt}}
\put(1197,711.17){\rule{1.700pt}{0.400pt}}
\multiput(1197.00,710.17)(4.472,2.000){2}{\rule{0.850pt}{0.400pt}}
\put(1205,711.17){\rule{1.700pt}{0.400pt}}
\multiput(1205.00,712.17)(4.472,-2.000){2}{\rule{0.850pt}{0.400pt}}
\put(1174.0,711.0){\rule[-0.200pt]{1.927pt}{0.400pt}}
\put(1228,711.17){\rule{1.700pt}{0.400pt}}
\multiput(1228.00,710.17)(4.472,2.000){2}{\rule{0.850pt}{0.400pt}}
\put(1236,711.17){\rule{1.700pt}{0.400pt}}
\multiput(1236.00,712.17)(4.472,-2.000){2}{\rule{0.850pt}{0.400pt}}
\put(1244,711.17){\rule{1.700pt}{0.400pt}}
\multiput(1244.00,710.17)(4.472,2.000){2}{\rule{0.850pt}{0.400pt}}
\put(1213.0,711.0){\rule[-0.200pt]{3.613pt}{0.400pt}}
\put(1275,713.17){\rule{1.700pt}{0.400pt}}
\multiput(1275.00,712.17)(4.472,2.000){2}{\rule{0.850pt}{0.400pt}}
\put(1252.0,713.0){\rule[-0.200pt]{5.541pt}{0.400pt}}
\put(1291,715.17){\rule{1.700pt}{0.400pt}}
\multiput(1291.00,714.17)(4.472,2.000){2}{\rule{0.850pt}{0.400pt}}
\put(1283.0,715.0){\rule[-0.200pt]{1.927pt}{0.400pt}}
\put(1314,717.17){\rule{1.700pt}{0.400pt}}
\multiput(1314.00,716.17)(4.472,2.000){2}{\rule{0.850pt}{0.400pt}}
\put(1322,717.17){\rule{1.700pt}{0.400pt}}
\multiput(1322.00,718.17)(4.472,-2.000){2}{\rule{0.850pt}{0.400pt}}
\put(1299.0,717.0){\rule[-0.200pt]{3.613pt}{0.400pt}}
\put(1338,717.17){\rule{1.500pt}{0.400pt}}
\multiput(1338.00,716.17)(3.887,2.000){2}{\rule{0.750pt}{0.400pt}}
\put(1345,719.17){\rule{1.700pt}{0.400pt}}
\multiput(1345.00,718.17)(4.472,2.000){2}{\rule{0.850pt}{0.400pt}}
\put(1353,719.17){\rule{1.700pt}{0.400pt}}
\multiput(1353.00,720.17)(4.472,-2.000){2}{\rule{0.850pt}{0.400pt}}
\put(1330.0,717.0){\rule[-0.200pt]{1.927pt}{0.400pt}}
\put(191.0,131.0){\rule[-0.200pt]{0.400pt}{155.380pt}}
\put(191.0,131.0){\rule[-0.200pt]{300.643pt}{0.400pt}}
\put(1439.0,131.0){\rule[-0.200pt]{0.400pt}{155.380pt}}
\put(191.0,776.0){\rule[-0.200pt]{300.643pt}{0.400pt}}
\end{picture}
}
  \caption{Trial 1, temperature = 22.4 \textdegree C}%
\label{fig:1m4}%
\end{minipage}%
\hfill%
\begin{minipage}{0.5\textwidth}%
		\raggedleft
    	\resizebox{0.9\textwidth}{!}{% GNUPLOT: LaTeX picture
\setlength{\unitlength}{0.240900pt}
\ifx\plotpoint\undefined\newsavebox{\plotpoint}\fi
\sbox{\plotpoint}{\rule[-0.200pt]{0.400pt}{0.400pt}}%
\begin{picture}(1500,900)(0,0)
\sbox{\plotpoint}{\rule[-0.200pt]{0.400pt}{0.400pt}}%
\put(191.0,131.0){\rule[-0.200pt]{4.818pt}{0.400pt}}
\put(171,131){\makebox(0,0)[r]{ 0.95}}
\put(1419.0,131.0){\rule[-0.200pt]{4.818pt}{0.400pt}}
\put(191.0,196.0){\rule[-0.200pt]{4.818pt}{0.400pt}}
\put(171,196){\makebox(0,0)[r]{ 0.96}}
\put(1419.0,196.0){\rule[-0.200pt]{4.818pt}{0.400pt}}
\put(191.0,260.0){\rule[-0.200pt]{4.818pt}{0.400pt}}
\put(171,260){\makebox(0,0)[r]{ 0.97}}
\put(1419.0,260.0){\rule[-0.200pt]{4.818pt}{0.400pt}}
\put(191.0,325.0){\rule[-0.200pt]{4.818pt}{0.400pt}}
\put(171,325){\makebox(0,0)[r]{ 0.98}}
\put(1419.0,325.0){\rule[-0.200pt]{4.818pt}{0.400pt}}
\put(191.0,389.0){\rule[-0.200pt]{4.818pt}{0.400pt}}
\put(171,389){\makebox(0,0)[r]{ 0.99}}
\put(1419.0,389.0){\rule[-0.200pt]{4.818pt}{0.400pt}}
\put(191.0,453.0){\rule[-0.200pt]{4.818pt}{0.400pt}}
\put(171,453){\makebox(0,0)[r]{ 1}}
\put(1419.0,453.0){\rule[-0.200pt]{4.818pt}{0.400pt}}
\put(191.0,518.0){\rule[-0.200pt]{4.818pt}{0.400pt}}
\put(171,518){\makebox(0,0)[r]{ 1.01}}
\put(1419.0,518.0){\rule[-0.200pt]{4.818pt}{0.400pt}}
\put(191.0,582.0){\rule[-0.200pt]{4.818pt}{0.400pt}}
\put(171,582){\makebox(0,0)[r]{ 1.02}}
\put(1419.0,582.0){\rule[-0.200pt]{4.818pt}{0.400pt}}
\put(191.0,647.0){\rule[-0.200pt]{4.818pt}{0.400pt}}
\put(171,647){\makebox(0,0)[r]{ 1.03}}
\put(1419.0,647.0){\rule[-0.200pt]{4.818pt}{0.400pt}}
\put(191.0,712.0){\rule[-0.200pt]{4.818pt}{0.400pt}}
\put(171,712){\makebox(0,0)[r]{ 1.04}}
\put(1419.0,712.0){\rule[-0.200pt]{4.818pt}{0.400pt}}
\put(191.0,776.0){\rule[-0.200pt]{4.818pt}{0.400pt}}
\put(171,776){\makebox(0,0)[r]{ 1.05}}
\put(1419.0,776.0){\rule[-0.200pt]{4.818pt}{0.400pt}}
\put(191.0,131.0){\rule[-0.200pt]{0.400pt}{4.818pt}}
\put(191,90){\makebox(0,0){ 0}}
\put(191.0,756.0){\rule[-0.200pt]{0.400pt}{4.818pt}}
\put(399.0,131.0){\rule[-0.200pt]{0.400pt}{4.818pt}}
\put(399,90){\makebox(0,0){ 5}}
\put(399.0,756.0){\rule[-0.200pt]{0.400pt}{4.818pt}}
\put(607.0,131.0){\rule[-0.200pt]{0.400pt}{4.818pt}}
\put(607,90){\makebox(0,0){ 10}}
\put(607.0,756.0){\rule[-0.200pt]{0.400pt}{4.818pt}}
\put(815.0,131.0){\rule[-0.200pt]{0.400pt}{4.818pt}}
\put(815,90){\makebox(0,0){ 15}}
\put(815.0,756.0){\rule[-0.200pt]{0.400pt}{4.818pt}}
\put(1023.0,131.0){\rule[-0.200pt]{0.400pt}{4.818pt}}
\put(1023,90){\makebox(0,0){ 20}}
\put(1023.0,756.0){\rule[-0.200pt]{0.400pt}{4.818pt}}
\put(1231.0,131.0){\rule[-0.200pt]{0.400pt}{4.818pt}}
\put(1231,90){\makebox(0,0){ 25}}
\put(1231.0,756.0){\rule[-0.200pt]{0.400pt}{4.818pt}}
\put(1439.0,131.0){\rule[-0.200pt]{0.400pt}{4.818pt}}
\put(1439,90){\makebox(0,0){ 30}}
\put(1439.0,756.0){\rule[-0.200pt]{0.400pt}{4.818pt}}
\put(191.0,131.0){\rule[-0.200pt]{0.400pt}{155.380pt}}
\put(191.0,131.0){\rule[-0.200pt]{300.643pt}{0.400pt}}
\put(1439.0,131.0){\rule[-0.200pt]{0.400pt}{155.380pt}}
\put(191.0,776.0){\rule[-0.200pt]{300.643pt}{0.400pt}}
\put(30,453){\makebox(0,0){\hspace{-72pt} Pressure (atm)}}
\put(815,29){\makebox(0,0){Time (seconds)}}
\put(815,838){\makebox(0,0){0.02g Mg Powder - 10mL 1M HCl Reaction Pressure v. Time}}
\put(191,193){\rule{1pt}{1pt}}
\put(201,193){\rule{1pt}{1pt}}
\put(212,193){\rule{1pt}{1pt}}
\put(222,189){\rule{1pt}{1pt}}
\put(233,193){\rule{1pt}{1pt}}
\put(243,193){\rule{1pt}{1pt}}
\put(253,193){\rule{1pt}{1pt}}
\put(264,193){\rule{1pt}{1pt}}
\put(274,193){\rule{1pt}{1pt}}
\put(285,193){\rule{1pt}{1pt}}
\put(295,193){\rule{1pt}{1pt}}
\put(305,193){\rule{1pt}{1pt}}
\put(316,193){\rule{1pt}{1pt}}
\put(326,189){\rule{1pt}{1pt}}
\put(337,193){\rule{1pt}{1pt}}
\put(347,193){\rule{1pt}{1pt}}
\put(357,193){\rule{1pt}{1pt}}
\put(368,193){\rule{1pt}{1pt}}
\put(378,193){\rule{1pt}{1pt}}
\put(389,189){\rule{1pt}{1pt}}
\put(399,193){\rule{1pt}{1pt}}
\put(409,204){\rule{1pt}{1pt}}
\put(420,260){\rule{1pt}{1pt}}
\put(430,283){\rule{1pt}{1pt}}
\put(441,306){\rule{1pt}{1pt}}
\put(451,324){\rule{1pt}{1pt}}
\put(461,343){\rule{1pt}{1pt}}
\put(472,362){\rule{1pt}{1pt}}
\put(482,385){\rule{1pt}{1pt}}
\put(493,399){\rule{1pt}{1pt}}
\put(503,418){\rule{1pt}{1pt}}
\put(513,433){\rule{1pt}{1pt}}
\put(524,452){\rule{1pt}{1pt}}
\put(534,467){\rule{1pt}{1pt}}
\put(545,486){\rule{1pt}{1pt}}
\put(555,501){\rule{1pt}{1pt}}
\put(565,512){\rule{1pt}{1pt}}
\put(576,527){\rule{1pt}{1pt}}
\put(586,539){\rule{1pt}{1pt}}
\put(597,550){\rule{1pt}{1pt}}
\put(607,565){\rule{1pt}{1pt}}
\put(617,573){\rule{1pt}{1pt}}
\put(628,580){\rule{1pt}{1pt}}
\put(638,595){\rule{1pt}{1pt}}
\put(649,599){\rule{1pt}{1pt}}
\put(659,606){\rule{1pt}{1pt}}
\put(669,614){\rule{1pt}{1pt}}
\put(680,625){\rule{1pt}{1pt}}
\put(690,633){\rule{1pt}{1pt}}
\put(701,640){\rule{1pt}{1pt}}
\put(711,644){\rule{1pt}{1pt}}
\put(721,648){\rule{1pt}{1pt}}
\put(732,655){\rule{1pt}{1pt}}
\put(742,655){\rule{1pt}{1pt}}
\put(753,666){\rule{1pt}{1pt}}
\put(763,670){\rule{1pt}{1pt}}
\put(773,670){\rule{1pt}{1pt}}
\put(784,674){\rule{1pt}{1pt}}
\put(794,678){\rule{1pt}{1pt}}
\put(805,682){\rule{1pt}{1pt}}
\put(815,685){\rule{1pt}{1pt}}
\put(825,685){\rule{1pt}{1pt}}
\put(836,689){\rule{1pt}{1pt}}
\put(846,693){\rule{1pt}{1pt}}
\put(857,693){\rule{1pt}{1pt}}
\put(867,697){\rule{1pt}{1pt}}
\put(877,697){\rule{1pt}{1pt}}
\put(888,700){\rule{1pt}{1pt}}
\put(898,700){\rule{1pt}{1pt}}
\put(909,704){\rule{1pt}{1pt}}
\put(919,704){\rule{1pt}{1pt}}
\put(929,704){\rule{1pt}{1pt}}
\put(940,704){\rule{1pt}{1pt}}
\put(950,704){\rule{1pt}{1pt}}
\put(961,708){\rule{1pt}{1pt}}
\put(971,712){\rule{1pt}{1pt}}
\put(981,712){\rule{1pt}{1pt}}
\put(992,712){\rule{1pt}{1pt}}
\put(1002,715){\rule{1pt}{1pt}}
\put(1013,712){\rule{1pt}{1pt}}
\put(1023,715){\rule{1pt}{1pt}}
\put(1033,712){\rule{1pt}{1pt}}
\put(1044,715){\rule{1pt}{1pt}}
\put(1054,715){\rule{1pt}{1pt}}
\put(1065,715){\rule{1pt}{1pt}}
\put(1075,719){\rule{1pt}{1pt}}
\put(1085,719){\rule{1pt}{1pt}}
\put(1096,719){\rule{1pt}{1pt}}
\put(1106,719){\rule{1pt}{1pt}}
\put(1117,719){\rule{1pt}{1pt}}
\put(1127,719){\rule{1pt}{1pt}}
\put(1137,719){\rule{1pt}{1pt}}
\put(1148,719){\rule{1pt}{1pt}}
\put(1158,723){\rule{1pt}{1pt}}
\put(1169,719){\rule{1pt}{1pt}}
\put(1179,719){\rule{1pt}{1pt}}
\put(1189,723){\rule{1pt}{1pt}}
\put(1200,719){\rule{1pt}{1pt}}
\put(1210,727){\rule{1pt}{1pt}}
\put(1221,723){\rule{1pt}{1pt}}
\put(1231,719){\rule{1pt}{1pt}}
\put(1241,723){\rule{1pt}{1pt}}
\put(1252,723){\rule{1pt}{1pt}}
\put(1262,719){\rule{1pt}{1pt}}
\put(1273,719){\rule{1pt}{1pt}}
\put(1283,723){\rule{1pt}{1pt}}
\put(1293,719){\rule{1pt}{1pt}}
\put(1304,723){\rule{1pt}{1pt}}
\put(1314,723){\rule{1pt}{1pt}}
\put(1325,723){\rule{1pt}{1pt}}
\put(1335,723){\rule{1pt}{1pt}}
\put(1345,723){\rule{1pt}{1pt}}
\put(1356,723){\rule{1pt}{1pt}}
\put(1366,723){\rule{1pt}{1pt}}
\put(1377,723){\rule{1pt}{1pt}}
\put(1387,723){\rule{1pt}{1pt}}
\put(1397,723){\rule{1pt}{1pt}}
\put(1408,727){\rule{1pt}{1pt}}
\put(191,193){\usebox{\plotpoint}}
\multiput(212.00,191.94)(1.358,-0.468){5}{\rule{1.100pt}{0.113pt}}
\multiput(212.00,192.17)(7.717,-4.000){2}{\rule{0.550pt}{0.400pt}}
\multiput(222.00,189.60)(1.505,0.468){5}{\rule{1.200pt}{0.113pt}}
\multiput(222.00,188.17)(8.509,4.000){2}{\rule{0.600pt}{0.400pt}}
\put(191.0,193.0){\rule[-0.200pt]{5.059pt}{0.400pt}}
\multiput(316.00,191.94)(1.358,-0.468){5}{\rule{1.100pt}{0.113pt}}
\multiput(316.00,192.17)(7.717,-4.000){2}{\rule{0.550pt}{0.400pt}}
\multiput(326.00,189.60)(1.505,0.468){5}{\rule{1.200pt}{0.113pt}}
\multiput(326.00,188.17)(8.509,4.000){2}{\rule{0.600pt}{0.400pt}}
\put(233.0,193.0){\rule[-0.200pt]{19.995pt}{0.400pt}}
\multiput(378.00,191.94)(1.505,-0.468){5}{\rule{1.200pt}{0.113pt}}
\multiput(378.00,192.17)(8.509,-4.000){2}{\rule{0.600pt}{0.400pt}}
\multiput(389.00,189.60)(1.358,0.468){5}{\rule{1.100pt}{0.113pt}}
\multiput(389.00,188.17)(7.717,4.000){2}{\rule{0.550pt}{0.400pt}}
\multiput(399.58,193.00)(0.491,0.547){17}{\rule{0.118pt}{0.540pt}}
\multiput(398.17,193.00)(10.000,9.879){2}{\rule{0.400pt}{0.270pt}}
\multiput(409.58,204.00)(0.492,2.618){19}{\rule{0.118pt}{2.136pt}}
\multiput(408.17,204.00)(11.000,51.566){2}{\rule{0.400pt}{1.068pt}}
\multiput(420.58,260.00)(0.491,1.173){17}{\rule{0.118pt}{1.020pt}}
\multiput(419.17,260.00)(10.000,20.883){2}{\rule{0.400pt}{0.510pt}}
\multiput(430.58,283.00)(0.492,1.062){19}{\rule{0.118pt}{0.936pt}}
\multiput(429.17,283.00)(11.000,21.057){2}{\rule{0.400pt}{0.468pt}}
\multiput(441.58,306.00)(0.491,0.912){17}{\rule{0.118pt}{0.820pt}}
\multiput(440.17,306.00)(10.000,16.298){2}{\rule{0.400pt}{0.410pt}}
\multiput(451.58,324.00)(0.491,0.964){17}{\rule{0.118pt}{0.860pt}}
\multiput(450.17,324.00)(10.000,17.215){2}{\rule{0.400pt}{0.430pt}}
\multiput(461.58,343.00)(0.492,0.873){19}{\rule{0.118pt}{0.791pt}}
\multiput(460.17,343.00)(11.000,17.358){2}{\rule{0.400pt}{0.395pt}}
\multiput(472.58,362.00)(0.491,1.173){17}{\rule{0.118pt}{1.020pt}}
\multiput(471.17,362.00)(10.000,20.883){2}{\rule{0.400pt}{0.510pt}}
\multiput(482.58,385.00)(0.492,0.637){19}{\rule{0.118pt}{0.609pt}}
\multiput(481.17,385.00)(11.000,12.736){2}{\rule{0.400pt}{0.305pt}}
\multiput(493.58,399.00)(0.491,0.964){17}{\rule{0.118pt}{0.860pt}}
\multiput(492.17,399.00)(10.000,17.215){2}{\rule{0.400pt}{0.430pt}}
\multiput(503.58,418.00)(0.491,0.756){17}{\rule{0.118pt}{0.700pt}}
\multiput(502.17,418.00)(10.000,13.547){2}{\rule{0.400pt}{0.350pt}}
\multiput(513.58,433.00)(0.492,0.873){19}{\rule{0.118pt}{0.791pt}}
\multiput(512.17,433.00)(11.000,17.358){2}{\rule{0.400pt}{0.395pt}}
\multiput(524.58,452.00)(0.491,0.756){17}{\rule{0.118pt}{0.700pt}}
\multiput(523.17,452.00)(10.000,13.547){2}{\rule{0.400pt}{0.350pt}}
\multiput(534.58,467.00)(0.492,0.873){19}{\rule{0.118pt}{0.791pt}}
\multiput(533.17,467.00)(11.000,17.358){2}{\rule{0.400pt}{0.395pt}}
\multiput(545.58,486.00)(0.491,0.756){17}{\rule{0.118pt}{0.700pt}}
\multiput(544.17,486.00)(10.000,13.547){2}{\rule{0.400pt}{0.350pt}}
\multiput(555.58,501.00)(0.491,0.547){17}{\rule{0.118pt}{0.540pt}}
\multiput(554.17,501.00)(10.000,9.879){2}{\rule{0.400pt}{0.270pt}}
\multiput(565.58,512.00)(0.492,0.684){19}{\rule{0.118pt}{0.645pt}}
\multiput(564.17,512.00)(11.000,13.660){2}{\rule{0.400pt}{0.323pt}}
\multiput(576.58,527.00)(0.491,0.600){17}{\rule{0.118pt}{0.580pt}}
\multiput(575.17,527.00)(10.000,10.796){2}{\rule{0.400pt}{0.290pt}}
\multiput(586.00,539.58)(0.496,0.492){19}{\rule{0.500pt}{0.118pt}}
\multiput(586.00,538.17)(9.962,11.000){2}{\rule{0.250pt}{0.400pt}}
\multiput(597.58,550.00)(0.491,0.756){17}{\rule{0.118pt}{0.700pt}}
\multiput(596.17,550.00)(10.000,13.547){2}{\rule{0.400pt}{0.350pt}}
\multiput(607.00,565.59)(0.626,0.488){13}{\rule{0.600pt}{0.117pt}}
\multiput(607.00,564.17)(8.755,8.000){2}{\rule{0.300pt}{0.400pt}}
\multiput(617.00,573.59)(0.798,0.485){11}{\rule{0.729pt}{0.117pt}}
\multiput(617.00,572.17)(9.488,7.000){2}{\rule{0.364pt}{0.400pt}}
\multiput(628.58,580.00)(0.491,0.756){17}{\rule{0.118pt}{0.700pt}}
\multiput(627.17,580.00)(10.000,13.547){2}{\rule{0.400pt}{0.350pt}}
\multiput(638.00,595.60)(1.505,0.468){5}{\rule{1.200pt}{0.113pt}}
\multiput(638.00,594.17)(8.509,4.000){2}{\rule{0.600pt}{0.400pt}}
\multiput(649.00,599.59)(0.721,0.485){11}{\rule{0.671pt}{0.117pt}}
\multiput(649.00,598.17)(8.606,7.000){2}{\rule{0.336pt}{0.400pt}}
\multiput(659.00,606.59)(0.626,0.488){13}{\rule{0.600pt}{0.117pt}}
\multiput(659.00,605.17)(8.755,8.000){2}{\rule{0.300pt}{0.400pt}}
\multiput(669.00,614.58)(0.496,0.492){19}{\rule{0.500pt}{0.118pt}}
\multiput(669.00,613.17)(9.962,11.000){2}{\rule{0.250pt}{0.400pt}}
\multiput(680.00,625.59)(0.626,0.488){13}{\rule{0.600pt}{0.117pt}}
\multiput(680.00,624.17)(8.755,8.000){2}{\rule{0.300pt}{0.400pt}}
\multiput(690.00,633.59)(0.798,0.485){11}{\rule{0.729pt}{0.117pt}}
\multiput(690.00,632.17)(9.488,7.000){2}{\rule{0.364pt}{0.400pt}}
\multiput(701.00,640.60)(1.358,0.468){5}{\rule{1.100pt}{0.113pt}}
\multiput(701.00,639.17)(7.717,4.000){2}{\rule{0.550pt}{0.400pt}}
\multiput(711.00,644.60)(1.358,0.468){5}{\rule{1.100pt}{0.113pt}}
\multiput(711.00,643.17)(7.717,4.000){2}{\rule{0.550pt}{0.400pt}}
\multiput(721.00,648.59)(0.798,0.485){11}{\rule{0.729pt}{0.117pt}}
\multiput(721.00,647.17)(9.488,7.000){2}{\rule{0.364pt}{0.400pt}}
\put(337.0,193.0){\rule[-0.200pt]{9.877pt}{0.400pt}}
\multiput(742.00,655.58)(0.496,0.492){19}{\rule{0.500pt}{0.118pt}}
\multiput(742.00,654.17)(9.962,11.000){2}{\rule{0.250pt}{0.400pt}}
\multiput(753.00,666.60)(1.358,0.468){5}{\rule{1.100pt}{0.113pt}}
\multiput(753.00,665.17)(7.717,4.000){2}{\rule{0.550pt}{0.400pt}}
\put(732.0,655.0){\rule[-0.200pt]{2.409pt}{0.400pt}}
\multiput(773.00,670.60)(1.505,0.468){5}{\rule{1.200pt}{0.113pt}}
\multiput(773.00,669.17)(8.509,4.000){2}{\rule{0.600pt}{0.400pt}}
\multiput(784.00,674.60)(1.358,0.468){5}{\rule{1.100pt}{0.113pt}}
\multiput(784.00,673.17)(7.717,4.000){2}{\rule{0.550pt}{0.400pt}}
\multiput(794.00,678.60)(1.505,0.468){5}{\rule{1.200pt}{0.113pt}}
\multiput(794.00,677.17)(8.509,4.000){2}{\rule{0.600pt}{0.400pt}}
\multiput(805.00,682.61)(2.025,0.447){3}{\rule{1.433pt}{0.108pt}}
\multiput(805.00,681.17)(7.025,3.000){2}{\rule{0.717pt}{0.400pt}}
\put(763.0,670.0){\rule[-0.200pt]{2.409pt}{0.400pt}}
\multiput(825.00,685.60)(1.505,0.468){5}{\rule{1.200pt}{0.113pt}}
\multiput(825.00,684.17)(8.509,4.000){2}{\rule{0.600pt}{0.400pt}}
\multiput(836.00,689.60)(1.358,0.468){5}{\rule{1.100pt}{0.113pt}}
\multiput(836.00,688.17)(7.717,4.000){2}{\rule{0.550pt}{0.400pt}}
\put(815.0,685.0){\rule[-0.200pt]{2.409pt}{0.400pt}}
\multiput(857.00,693.60)(1.358,0.468){5}{\rule{1.100pt}{0.113pt}}
\multiput(857.00,692.17)(7.717,4.000){2}{\rule{0.550pt}{0.400pt}}
\put(846.0,693.0){\rule[-0.200pt]{2.650pt}{0.400pt}}
\multiput(877.00,697.61)(2.248,0.447){3}{\rule{1.567pt}{0.108pt}}
\multiput(877.00,696.17)(7.748,3.000){2}{\rule{0.783pt}{0.400pt}}
\put(867.0,697.0){\rule[-0.200pt]{2.409pt}{0.400pt}}
\multiput(898.00,700.60)(1.505,0.468){5}{\rule{1.200pt}{0.113pt}}
\multiput(898.00,699.17)(8.509,4.000){2}{\rule{0.600pt}{0.400pt}}
\put(888.0,700.0){\rule[-0.200pt]{2.409pt}{0.400pt}}
\multiput(950.00,704.60)(1.505,0.468){5}{\rule{1.200pt}{0.113pt}}
\multiput(950.00,703.17)(8.509,4.000){2}{\rule{0.600pt}{0.400pt}}
\multiput(961.00,708.60)(1.358,0.468){5}{\rule{1.100pt}{0.113pt}}
\multiput(961.00,707.17)(7.717,4.000){2}{\rule{0.550pt}{0.400pt}}
\put(909.0,704.0){\rule[-0.200pt]{9.877pt}{0.400pt}}
\multiput(992.00,712.61)(2.025,0.447){3}{\rule{1.433pt}{0.108pt}}
\multiput(992.00,711.17)(7.025,3.000){2}{\rule{0.717pt}{0.400pt}}
\multiput(1002.00,713.95)(2.248,-0.447){3}{\rule{1.567pt}{0.108pt}}
\multiput(1002.00,714.17)(7.748,-3.000){2}{\rule{0.783pt}{0.400pt}}
\multiput(1013.00,712.61)(2.025,0.447){3}{\rule{1.433pt}{0.108pt}}
\multiput(1013.00,711.17)(7.025,3.000){2}{\rule{0.717pt}{0.400pt}}
\multiput(1023.00,713.95)(2.025,-0.447){3}{\rule{1.433pt}{0.108pt}}
\multiput(1023.00,714.17)(7.025,-3.000){2}{\rule{0.717pt}{0.400pt}}
\multiput(1033.00,712.61)(2.248,0.447){3}{\rule{1.567pt}{0.108pt}}
\multiput(1033.00,711.17)(7.748,3.000){2}{\rule{0.783pt}{0.400pt}}
\put(971.0,712.0){\rule[-0.200pt]{5.059pt}{0.400pt}}
\multiput(1065.00,715.60)(1.358,0.468){5}{\rule{1.100pt}{0.113pt}}
\multiput(1065.00,714.17)(7.717,4.000){2}{\rule{0.550pt}{0.400pt}}
\put(1044.0,715.0){\rule[-0.200pt]{5.059pt}{0.400pt}}
\multiput(1148.00,719.60)(1.358,0.468){5}{\rule{1.100pt}{0.113pt}}
\multiput(1148.00,718.17)(7.717,4.000){2}{\rule{0.550pt}{0.400pt}}
\multiput(1158.00,721.94)(1.505,-0.468){5}{\rule{1.200pt}{0.113pt}}
\multiput(1158.00,722.17)(8.509,-4.000){2}{\rule{0.600pt}{0.400pt}}
\put(1075.0,719.0){\rule[-0.200pt]{17.586pt}{0.400pt}}
\multiput(1179.00,719.60)(1.358,0.468){5}{\rule{1.100pt}{0.113pt}}
\multiput(1179.00,718.17)(7.717,4.000){2}{\rule{0.550pt}{0.400pt}}
\multiput(1189.00,721.94)(1.505,-0.468){5}{\rule{1.200pt}{0.113pt}}
\multiput(1189.00,722.17)(8.509,-4.000){2}{\rule{0.600pt}{0.400pt}}
\multiput(1200.00,719.59)(0.626,0.488){13}{\rule{0.600pt}{0.117pt}}
\multiput(1200.00,718.17)(8.755,8.000){2}{\rule{0.300pt}{0.400pt}}
\multiput(1210.00,725.94)(1.505,-0.468){5}{\rule{1.200pt}{0.113pt}}
\multiput(1210.00,726.17)(8.509,-4.000){2}{\rule{0.600pt}{0.400pt}}
\multiput(1221.00,721.94)(1.358,-0.468){5}{\rule{1.100pt}{0.113pt}}
\multiput(1221.00,722.17)(7.717,-4.000){2}{\rule{0.550pt}{0.400pt}}
\multiput(1231.00,719.60)(1.358,0.468){5}{\rule{1.100pt}{0.113pt}}
\multiput(1231.00,718.17)(7.717,4.000){2}{\rule{0.550pt}{0.400pt}}
\put(1169.0,719.0){\rule[-0.200pt]{2.409pt}{0.400pt}}
\multiput(1252.00,721.94)(1.358,-0.468){5}{\rule{1.100pt}{0.113pt}}
\multiput(1252.00,722.17)(7.717,-4.000){2}{\rule{0.550pt}{0.400pt}}
\put(1241.0,723.0){\rule[-0.200pt]{2.650pt}{0.400pt}}
\multiput(1273.00,719.60)(1.358,0.468){5}{\rule{1.100pt}{0.113pt}}
\multiput(1273.00,718.17)(7.717,4.000){2}{\rule{0.550pt}{0.400pt}}
\multiput(1283.00,721.94)(1.358,-0.468){5}{\rule{1.100pt}{0.113pt}}
\multiput(1283.00,722.17)(7.717,-4.000){2}{\rule{0.550pt}{0.400pt}}
\multiput(1293.00,719.60)(1.505,0.468){5}{\rule{1.200pt}{0.113pt}}
\multiput(1293.00,718.17)(8.509,4.000){2}{\rule{0.600pt}{0.400pt}}
\put(1262.0,719.0){\rule[-0.200pt]{2.650pt}{0.400pt}}
\multiput(1397.00,723.60)(1.505,0.468){5}{\rule{1.200pt}{0.113pt}}
\multiput(1397.00,722.17)(8.509,4.000){2}{\rule{0.600pt}{0.400pt}}
\put(1304.0,723.0){\rule[-0.200pt]{22.404pt}{0.400pt}}
\put(191.0,131.0){\rule[-0.200pt]{0.400pt}{155.380pt}}
\put(191.0,131.0){\rule[-0.200pt]{300.643pt}{0.400pt}}
\put(1439.0,131.0){\rule[-0.200pt]{0.400pt}{155.380pt}}
\put(191.0,776.0){\rule[-0.200pt]{300.643pt}{0.400pt}}
\end{picture}
}
  \caption{Trial 2, temperature = 22.4 \textdegree C }
\label{fig:1m2}
\end {minipage}
\end {figure}
\FloatBarrier

\begin{figure}[h!]
\centering
\begin{minipage}{0.5\textwidth}%
		\raggedright
    	\resizebox{0.9\textwidth}{!}{% GNUPLOT: LaTeX picture
\setlength{\unitlength}{0.240900pt}
\ifx\plotpoint\undefined\newsavebox{\plotpoint}\fi
\sbox{\plotpoint}{\rule[-0.200pt]{0.400pt}{0.400pt}}%
\begin{picture}(1500,900)(0,0)
\sbox{\plotpoint}{\rule[-0.200pt]{0.400pt}{0.400pt}}%
\put(191.0,131.0){\rule[-0.200pt]{4.818pt}{0.400pt}}
\put(171,131){\makebox(0,0)[r]{ 0.94}}
\put(1419.0,131.0){\rule[-0.200pt]{4.818pt}{0.400pt}}
\put(191.0,196.0){\rule[-0.200pt]{4.818pt}{0.400pt}}
\put(171,196){\makebox(0,0)[r]{ 0.96}}
\put(1419.0,196.0){\rule[-0.200pt]{4.818pt}{0.400pt}}
\put(191.0,260.0){\rule[-0.200pt]{4.818pt}{0.400pt}}
\put(171,260){\makebox(0,0)[r]{ 0.98}}
\put(1419.0,260.0){\rule[-0.200pt]{4.818pt}{0.400pt}}
\put(191.0,325.0){\rule[-0.200pt]{4.818pt}{0.400pt}}
\put(171,325){\makebox(0,0)[r]{ 1}}
\put(1419.0,325.0){\rule[-0.200pt]{4.818pt}{0.400pt}}
\put(191.0,389.0){\rule[-0.200pt]{4.818pt}{0.400pt}}
\put(171,389){\makebox(0,0)[r]{ 1.02}}
\put(1419.0,389.0){\rule[-0.200pt]{4.818pt}{0.400pt}}
\put(191.0,454.0){\rule[-0.200pt]{4.818pt}{0.400pt}}
\put(171,454){\makebox(0,0)[r]{ 1.04}}
\put(1419.0,454.0){\rule[-0.200pt]{4.818pt}{0.400pt}}
\put(191.0,518.0){\rule[-0.200pt]{4.818pt}{0.400pt}}
\put(171,518){\makebox(0,0)[r]{ 1.06}}
\put(1419.0,518.0){\rule[-0.200pt]{4.818pt}{0.400pt}}
\put(191.0,583.0){\rule[-0.200pt]{4.818pt}{0.400pt}}
\put(171,583){\makebox(0,0)[r]{ 1.08}}
\put(1419.0,583.0){\rule[-0.200pt]{4.818pt}{0.400pt}}
\put(191.0,647.0){\rule[-0.200pt]{4.818pt}{0.400pt}}
\put(171,647){\makebox(0,0)[r]{ 1.1}}
\put(1419.0,647.0){\rule[-0.200pt]{4.818pt}{0.400pt}}
\put(191.0,712.0){\rule[-0.200pt]{4.818pt}{0.400pt}}
\put(171,712){\makebox(0,0)[r]{ 1.12}}
\put(1419.0,712.0){\rule[-0.200pt]{4.818pt}{0.400pt}}
\put(191.0,776.0){\rule[-0.200pt]{4.818pt}{0.400pt}}
\put(171,776){\makebox(0,0)[r]{ 1.14}}
\put(1419.0,776.0){\rule[-0.200pt]{4.818pt}{0.400pt}}
\put(191.0,131.0){\rule[-0.200pt]{0.400pt}{4.818pt}}
\put(191,90){\makebox(0,0){ 0}}
\put(191.0,756.0){\rule[-0.200pt]{0.400pt}{4.818pt}}
\put(347.0,131.0){\rule[-0.200pt]{0.400pt}{4.818pt}}
\put(347,90){\makebox(0,0){ 10}}
\put(347.0,756.0){\rule[-0.200pt]{0.400pt}{4.818pt}}
\put(503.0,131.0){\rule[-0.200pt]{0.400pt}{4.818pt}}
\put(503,90){\makebox(0,0){ 20}}
\put(503.0,756.0){\rule[-0.200pt]{0.400pt}{4.818pt}}
\put(659.0,131.0){\rule[-0.200pt]{0.400pt}{4.818pt}}
\put(659,90){\makebox(0,0){ 30}}
\put(659.0,756.0){\rule[-0.200pt]{0.400pt}{4.818pt}}
\put(815.0,131.0){\rule[-0.200pt]{0.400pt}{4.818pt}}
\put(815,90){\makebox(0,0){ 40}}
\put(815.0,756.0){\rule[-0.200pt]{0.400pt}{4.818pt}}
\put(971.0,131.0){\rule[-0.200pt]{0.400pt}{4.818pt}}
\put(971,90){\makebox(0,0){ 50}}
\put(971.0,756.0){\rule[-0.200pt]{0.400pt}{4.818pt}}
\put(1127.0,131.0){\rule[-0.200pt]{0.400pt}{4.818pt}}
\put(1127,90){\makebox(0,0){ 60}}
\put(1127.0,756.0){\rule[-0.200pt]{0.400pt}{4.818pt}}
\put(1283.0,131.0){\rule[-0.200pt]{0.400pt}{4.818pt}}
\put(1283,90){\makebox(0,0){ 70}}
\put(1283.0,756.0){\rule[-0.200pt]{0.400pt}{4.818pt}}
\put(1439.0,131.0){\rule[-0.200pt]{0.400pt}{4.818pt}}
\put(1439,90){\makebox(0,0){ 80}}
\put(1439.0,756.0){\rule[-0.200pt]{0.400pt}{4.818pt}}
\put(191.0,131.0){\rule[-0.200pt]{0.400pt}{155.380pt}}
\put(191.0,131.0){\rule[-0.200pt]{300.643pt}{0.400pt}}
\put(1439.0,131.0){\rule[-0.200pt]{0.400pt}{155.380pt}}
\put(191.0,776.0){\rule[-0.200pt]{300.643pt}{0.400pt}}
\put(30,453){\makebox(0,0){\hspace{-72pt} Pressure (atm)}}
\put(815,29){\makebox(0,0){Time (seconds)}}
\put(815,838){\makebox(0,0){0.04g Mg Powder - 10mL 0.5M HCl Reaction Pressure v. Time}}
\put(191,194){\rule{1pt}{1pt}}
\put(195,194){\rule{1pt}{1pt}}
\put(199,192){\rule{1pt}{1pt}}
\put(203,192){\rule{1pt}{1pt}}
\put(207,192){\rule{1pt}{1pt}}
\put(211,194){\rule{1pt}{1pt}}
\put(214,192){\rule{1pt}{1pt}}
\put(218,194){\rule{1pt}{1pt}}
\put(222,192){\rule{1pt}{1pt}}
\put(226,194){\rule{1pt}{1pt}}
\put(230,192){\rule{1pt}{1pt}}
\put(234,194){\rule{1pt}{1pt}}
\put(238,192){\rule{1pt}{1pt}}
\put(242,192){\rule{1pt}{1pt}}
\put(246,194){\rule{1pt}{1pt}}
\put(250,192){\rule{1pt}{1pt}}
\put(253,194){\rule{1pt}{1pt}}
\put(257,192){\rule{1pt}{1pt}}
\put(261,192){\rule{1pt}{1pt}}
\put(265,194){\rule{1pt}{1pt}}
\put(269,194){\rule{1pt}{1pt}}
\put(273,192){\rule{1pt}{1pt}}
\put(277,194){\rule{1pt}{1pt}}
\put(281,194){\rule{1pt}{1pt}}
\put(285,194){\rule{1pt}{1pt}}
\put(289,192){\rule{1pt}{1pt}}
\put(292,192){\rule{1pt}{1pt}}
\put(296,192){\rule{1pt}{1pt}}
\put(300,194){\rule{1pt}{1pt}}
\put(304,209){\rule{1pt}{1pt}}
\put(308,234){\rule{1pt}{1pt}}
\put(312,239){\rule{1pt}{1pt}}
\put(316,247){\rule{1pt}{1pt}}
\put(320,254){\rule{1pt}{1pt}}
\put(324,262){\rule{1pt}{1pt}}
\put(328,269){\rule{1pt}{1pt}}
\put(331,277){\rule{1pt}{1pt}}
\put(335,282){\rule{1pt}{1pt}}
\put(339,288){\rule{1pt}{1pt}}
\put(343,294){\rule{1pt}{1pt}}
\put(347,299){\rule{1pt}{1pt}}
\put(351,305){\rule{1pt}{1pt}}
\put(355,311){\rule{1pt}{1pt}}
\put(359,314){\rule{1pt}{1pt}}
\put(363,322){\rule{1pt}{1pt}}
\put(367,328){\rule{1pt}{1pt}}
\put(370,335){\rule{1pt}{1pt}}
\put(374,341){\rule{1pt}{1pt}}
\put(378,348){\rule{1pt}{1pt}}
\put(382,354){\rule{1pt}{1pt}}
\put(386,361){\rule{1pt}{1pt}}
\put(390,367){\rule{1pt}{1pt}}
\put(394,375){\rule{1pt}{1pt}}
\put(398,382){\rule{1pt}{1pt}}
\put(402,388){\rule{1pt}{1pt}}
\put(406,393){\rule{1pt}{1pt}}
\put(409,401){\rule{1pt}{1pt}}
\put(413,407){\rule{1pt}{1pt}}
\put(417,410){\rule{1pt}{1pt}}
\put(421,416){\rule{1pt}{1pt}}
\put(425,424){\rule{1pt}{1pt}}
\put(429,429){\rule{1pt}{1pt}}
\put(433,435){\rule{1pt}{1pt}}
\put(437,440){\rule{1pt}{1pt}}
\put(441,444){\rule{1pt}{1pt}}
\put(445,450){\rule{1pt}{1pt}}
\put(448,457){\rule{1pt}{1pt}}
\put(452,461){\rule{1pt}{1pt}}
\put(456,467){\rule{1pt}{1pt}}
\put(460,470){\rule{1pt}{1pt}}
\put(464,474){\rule{1pt}{1pt}}
\put(468,480){\rule{1pt}{1pt}}
\put(472,484){\rule{1pt}{1pt}}
\put(476,487){\rule{1pt}{1pt}}
\put(480,491){\rule{1pt}{1pt}}
\put(484,497){\rule{1pt}{1pt}}
\put(487,501){\rule{1pt}{1pt}}
\put(491,506){\rule{1pt}{1pt}}
\put(495,510){\rule{1pt}{1pt}}
\put(499,514){\rule{1pt}{1pt}}
\put(503,517){\rule{1pt}{1pt}}
\put(507,521){\rule{1pt}{1pt}}
\put(511,525){\rule{1pt}{1pt}}
\put(515,529){\rule{1pt}{1pt}}
\put(519,533){\rule{1pt}{1pt}}
\put(523,536){\rule{1pt}{1pt}}
\put(526,538){\rule{1pt}{1pt}}
\put(530,542){\rule{1pt}{1pt}}
\put(534,546){\rule{1pt}{1pt}}
\put(538,548){\rule{1pt}{1pt}}
\put(542,551){\rule{1pt}{1pt}}
\put(546,555){\rule{1pt}{1pt}}
\put(550,557){\rule{1pt}{1pt}}
\put(554,561){\rule{1pt}{1pt}}
\put(558,565){\rule{1pt}{1pt}}
\put(562,566){\rule{1pt}{1pt}}
\put(565,570){\rule{1pt}{1pt}}
\put(569,572){\rule{1pt}{1pt}}
\put(573,574){\rule{1pt}{1pt}}
\put(577,578){\rule{1pt}{1pt}}
\put(581,580){\rule{1pt}{1pt}}
\put(585,581){\rule{1pt}{1pt}}
\put(589,583){\rule{1pt}{1pt}}
\put(593,587){\rule{1pt}{1pt}}
\put(597,587){\rule{1pt}{1pt}}
\put(601,591){\rule{1pt}{1pt}}
\put(604,593){\rule{1pt}{1pt}}
\put(608,595){\rule{1pt}{1pt}}
\put(612,598){\rule{1pt}{1pt}}
\put(616,600){\rule{1pt}{1pt}}
\put(620,602){\rule{1pt}{1pt}}
\put(624,602){\rule{1pt}{1pt}}
\put(628,604){\rule{1pt}{1pt}}
\put(632,608){\rule{1pt}{1pt}}
\put(636,608){\rule{1pt}{1pt}}
\put(640,610){\rule{1pt}{1pt}}
\put(643,613){\rule{1pt}{1pt}}
\put(647,615){\rule{1pt}{1pt}}
\put(651,615){\rule{1pt}{1pt}}
\put(655,617){\rule{1pt}{1pt}}
\put(659,621){\rule{1pt}{1pt}}
\put(663,621){\rule{1pt}{1pt}}
\put(667,623){\rule{1pt}{1pt}}
\put(671,625){\rule{1pt}{1pt}}
\put(675,627){\rule{1pt}{1pt}}
\put(679,628){\rule{1pt}{1pt}}
\put(682,628){\rule{1pt}{1pt}}
\put(686,630){\rule{1pt}{1pt}}
\put(690,632){\rule{1pt}{1pt}}
\put(694,634){\rule{1pt}{1pt}}
\put(698,634){\rule{1pt}{1pt}}
\put(702,638){\rule{1pt}{1pt}}
\put(706,638){\rule{1pt}{1pt}}
\put(710,640){\rule{1pt}{1pt}}
\put(714,642){\rule{1pt}{1pt}}
\put(718,643){\rule{1pt}{1pt}}
\put(721,643){\rule{1pt}{1pt}}
\put(725,645){\rule{1pt}{1pt}}
\put(729,647){\rule{1pt}{1pt}}
\put(733,649){\rule{1pt}{1pt}}
\put(737,651){\rule{1pt}{1pt}}
\put(741,651){\rule{1pt}{1pt}}
\put(745,653){\rule{1pt}{1pt}}
\put(749,655){\rule{1pt}{1pt}}
\put(753,655){\rule{1pt}{1pt}}
\put(757,657){\rule{1pt}{1pt}}
\put(760,659){\rule{1pt}{1pt}}
\put(764,659){\rule{1pt}{1pt}}
\put(768,660){\rule{1pt}{1pt}}
\put(772,662){\rule{1pt}{1pt}}
\put(776,662){\rule{1pt}{1pt}}
\put(780,666){\rule{1pt}{1pt}}
\put(784,664){\rule{1pt}{1pt}}
\put(788,666){\rule{1pt}{1pt}}
\put(792,668){\rule{1pt}{1pt}}
\put(796,668){\rule{1pt}{1pt}}
\put(799,670){\rule{1pt}{1pt}}
\put(803,672){\rule{1pt}{1pt}}
\put(807,672){\rule{1pt}{1pt}}
\put(811,674){\rule{1pt}{1pt}}
\put(815,674){\rule{1pt}{1pt}}
\put(819,675){\rule{1pt}{1pt}}
\put(823,677){\rule{1pt}{1pt}}
\put(827,677){\rule{1pt}{1pt}}
\put(831,679){\rule{1pt}{1pt}}
\put(835,679){\rule{1pt}{1pt}}
\put(838,681){\rule{1pt}{1pt}}
\put(842,681){\rule{1pt}{1pt}}
\put(846,683){\rule{1pt}{1pt}}
\put(850,683){\rule{1pt}{1pt}}
\put(854,685){\rule{1pt}{1pt}}
\put(858,687){\rule{1pt}{1pt}}
\put(862,687){\rule{1pt}{1pt}}
\put(866,689){\rule{1pt}{1pt}}
\put(870,689){\rule{1pt}{1pt}}
\put(874,689){\rule{1pt}{1pt}}
\put(877,690){\rule{1pt}{1pt}}
\put(881,690){\rule{1pt}{1pt}}
\put(885,692){\rule{1pt}{1pt}}
\put(889,692){\rule{1pt}{1pt}}
\put(893,694){\rule{1pt}{1pt}}
\put(897,694){\rule{1pt}{1pt}}
\put(901,696){\rule{1pt}{1pt}}
\put(905,696){\rule{1pt}{1pt}}
\put(909,698){\rule{1pt}{1pt}}
\put(913,696){\rule{1pt}{1pt}}
\put(916,698){\rule{1pt}{1pt}}
\put(920,700){\rule{1pt}{1pt}}
\put(924,700){\rule{1pt}{1pt}}
\put(928,700){\rule{1pt}{1pt}}
\put(932,702){\rule{1pt}{1pt}}
\put(936,702){\rule{1pt}{1pt}}
\put(940,702){\rule{1pt}{1pt}}
\put(944,704){\rule{1pt}{1pt}}
\put(948,704){\rule{1pt}{1pt}}
\put(952,704){\rule{1pt}{1pt}}
\put(955,705){\rule{1pt}{1pt}}
\put(959,707){\rule{1pt}{1pt}}
\put(963,705){\rule{1pt}{1pt}}
\put(967,705){\rule{1pt}{1pt}}
\put(971,707){\rule{1pt}{1pt}}
\put(975,709){\rule{1pt}{1pt}}
\put(979,709){\rule{1pt}{1pt}}
\put(983,709){\rule{1pt}{1pt}}
\put(987,711){\rule{1pt}{1pt}}
\put(991,709){\rule{1pt}{1pt}}
\put(994,711){\rule{1pt}{1pt}}
\put(998,713){\rule{1pt}{1pt}}
\put(1002,713){\rule{1pt}{1pt}}
\put(1006,713){\rule{1pt}{1pt}}
\put(1010,715){\rule{1pt}{1pt}}
\put(1014,717){\rule{1pt}{1pt}}
\put(1018,717){\rule{1pt}{1pt}}
\put(1022,719){\rule{1pt}{1pt}}
\put(1026,719){\rule{1pt}{1pt}}
\put(1030,719){\rule{1pt}{1pt}}
\put(1033,719){\rule{1pt}{1pt}}
\put(1037,719){\rule{1pt}{1pt}}
\put(1041,719){\rule{1pt}{1pt}}
\put(1045,721){\rule{1pt}{1pt}}
\put(1049,719){\rule{1pt}{1pt}}
\put(1053,721){\rule{1pt}{1pt}}
\put(1057,721){\rule{1pt}{1pt}}
\put(1061,721){\rule{1pt}{1pt}}
\put(1065,722){\rule{1pt}{1pt}}
\put(1069,721){\rule{1pt}{1pt}}
\put(1072,722){\rule{1pt}{1pt}}
\put(1076,724){\rule{1pt}{1pt}}
\put(1080,724){\rule{1pt}{1pt}}
\put(1084,724){\rule{1pt}{1pt}}
\put(1088,726){\rule{1pt}{1pt}}
\put(1092,724){\rule{1pt}{1pt}}
\put(1096,726){\rule{1pt}{1pt}}
\put(1100,726){\rule{1pt}{1pt}}
\put(1104,728){\rule{1pt}{1pt}}
\put(1108,728){\rule{1pt}{1pt}}
\put(1111,728){\rule{1pt}{1pt}}
\put(1115,728){\rule{1pt}{1pt}}
\put(1119,730){\rule{1pt}{1pt}}
\put(1123,728){\rule{1pt}{1pt}}
\put(1127,730){\rule{1pt}{1pt}}
\put(1131,730){\rule{1pt}{1pt}}
\put(1135,730){\rule{1pt}{1pt}}
\put(1139,732){\rule{1pt}{1pt}}
\put(1143,730){\rule{1pt}{1pt}}
\put(1147,730){\rule{1pt}{1pt}}
\put(1150,732){\rule{1pt}{1pt}}
\put(1154,732){\rule{1pt}{1pt}}
\put(1158,732){\rule{1pt}{1pt}}
\put(1162,732){\rule{1pt}{1pt}}
\put(1166,734){\rule{1pt}{1pt}}
\put(1170,732){\rule{1pt}{1pt}}
\put(1174,734){\rule{1pt}{1pt}}
\put(1178,734){\rule{1pt}{1pt}}
\put(1182,736){\rule{1pt}{1pt}}
\put(1186,736){\rule{1pt}{1pt}}
\put(1189,736){\rule{1pt}{1pt}}
\put(1193,737){\rule{1pt}{1pt}}
\put(1197,737){\rule{1pt}{1pt}}
\put(1201,737){\rule{1pt}{1pt}}
\put(1205,737){\rule{1pt}{1pt}}
\put(1209,739){\rule{1pt}{1pt}}
\put(1213,737){\rule{1pt}{1pt}}
\put(1217,739){\rule{1pt}{1pt}}
\put(1221,739){\rule{1pt}{1pt}}
\put(1225,739){\rule{1pt}{1pt}}
\put(1228,739){\rule{1pt}{1pt}}
\put(1232,739){\rule{1pt}{1pt}}
\put(1236,741){\rule{1pt}{1pt}}
\put(1240,739){\rule{1pt}{1pt}}
\put(1244,741){\rule{1pt}{1pt}}
\put(1248,741){\rule{1pt}{1pt}}
\put(1252,741){\rule{1pt}{1pt}}
\put(1256,743){\rule{1pt}{1pt}}
\put(1260,741){\rule{1pt}{1pt}}
\put(1264,743){\rule{1pt}{1pt}}
\put(1267,741){\rule{1pt}{1pt}}
\put(1271,743){\rule{1pt}{1pt}}
\put(1275,743){\rule{1pt}{1pt}}
\put(1279,743){\rule{1pt}{1pt}}
\put(1283,745){\rule{1pt}{1pt}}
\put(1287,743){\rule{1pt}{1pt}}
\put(1291,743){\rule{1pt}{1pt}}
\put(1295,745){\rule{1pt}{1pt}}
\put(1299,745){\rule{1pt}{1pt}}
\put(1303,745){\rule{1pt}{1pt}}
\put(1306,745){\rule{1pt}{1pt}}
\put(1310,745){\rule{1pt}{1pt}}
\put(1314,747){\rule{1pt}{1pt}}
\put(1318,745){\rule{1pt}{1pt}}
\put(1322,747){\rule{1pt}{1pt}}
\put(1326,747){\rule{1pt}{1pt}}
\put(1330,747){\rule{1pt}{1pt}}
\put(1334,747){\rule{1pt}{1pt}}
\put(1338,747){\rule{1pt}{1pt}}
\put(1342,747){\rule{1pt}{1pt}}
\put(1345,749){\rule{1pt}{1pt}}
\put(1349,749){\rule{1pt}{1pt}}
\put(1353,749){\rule{1pt}{1pt}}
\put(1357,747){\rule{1pt}{1pt}}
\put(1361,749){\rule{1pt}{1pt}}
\put(1365,749){\rule{1pt}{1pt}}
\put(1369,749){\rule{1pt}{1pt}}
\put(1373,749){\rule{1pt}{1pt}}
\put(1377,749){\rule{1pt}{1pt}}
\put(1381,749){\rule{1pt}{1pt}}
\put(1384,749){\rule{1pt}{1pt}}
\put(1388,749){\rule{1pt}{1pt}}
\put(1392,749){\rule{1pt}{1pt}}
\put(191,194){\usebox{\plotpoint}}
\put(195,192.17){\rule{0.900pt}{0.400pt}}
\multiput(195.00,193.17)(2.132,-2.000){2}{\rule{0.450pt}{0.400pt}}
\put(191.0,194.0){\rule[-0.200pt]{0.964pt}{0.400pt}}
\put(207,192.17){\rule{0.900pt}{0.400pt}}
\multiput(207.00,191.17)(2.132,2.000){2}{\rule{0.450pt}{0.400pt}}
\put(211,192.17){\rule{0.700pt}{0.400pt}}
\multiput(211.00,193.17)(1.547,-2.000){2}{\rule{0.350pt}{0.400pt}}
\put(214,192.17){\rule{0.900pt}{0.400pt}}
\multiput(214.00,191.17)(2.132,2.000){2}{\rule{0.450pt}{0.400pt}}
\put(218,192.17){\rule{0.900pt}{0.400pt}}
\multiput(218.00,193.17)(2.132,-2.000){2}{\rule{0.450pt}{0.400pt}}
\put(222,192.17){\rule{0.900pt}{0.400pt}}
\multiput(222.00,191.17)(2.132,2.000){2}{\rule{0.450pt}{0.400pt}}
\put(226,192.17){\rule{0.900pt}{0.400pt}}
\multiput(226.00,193.17)(2.132,-2.000){2}{\rule{0.450pt}{0.400pt}}
\put(230,192.17){\rule{0.900pt}{0.400pt}}
\multiput(230.00,191.17)(2.132,2.000){2}{\rule{0.450pt}{0.400pt}}
\put(234,192.17){\rule{0.900pt}{0.400pt}}
\multiput(234.00,193.17)(2.132,-2.000){2}{\rule{0.450pt}{0.400pt}}
\put(199.0,192.0){\rule[-0.200pt]{1.927pt}{0.400pt}}
\put(242,192.17){\rule{0.900pt}{0.400pt}}
\multiput(242.00,191.17)(2.132,2.000){2}{\rule{0.450pt}{0.400pt}}
\put(246,192.17){\rule{0.900pt}{0.400pt}}
\multiput(246.00,193.17)(2.132,-2.000){2}{\rule{0.450pt}{0.400pt}}
\put(250,192.17){\rule{0.700pt}{0.400pt}}
\multiput(250.00,191.17)(1.547,2.000){2}{\rule{0.350pt}{0.400pt}}
\put(253,192.17){\rule{0.900pt}{0.400pt}}
\multiput(253.00,193.17)(2.132,-2.000){2}{\rule{0.450pt}{0.400pt}}
\put(238.0,192.0){\rule[-0.200pt]{0.964pt}{0.400pt}}
\put(261,192.17){\rule{0.900pt}{0.400pt}}
\multiput(261.00,191.17)(2.132,2.000){2}{\rule{0.450pt}{0.400pt}}
\put(257.0,192.0){\rule[-0.200pt]{0.964pt}{0.400pt}}
\put(269,192.17){\rule{0.900pt}{0.400pt}}
\multiput(269.00,193.17)(2.132,-2.000){2}{\rule{0.450pt}{0.400pt}}
\put(273,192.17){\rule{0.900pt}{0.400pt}}
\multiput(273.00,191.17)(2.132,2.000){2}{\rule{0.450pt}{0.400pt}}
\put(265.0,194.0){\rule[-0.200pt]{0.964pt}{0.400pt}}
\put(285,192.17){\rule{0.900pt}{0.400pt}}
\multiput(285.00,193.17)(2.132,-2.000){2}{\rule{0.450pt}{0.400pt}}
\put(277.0,194.0){\rule[-0.200pt]{1.927pt}{0.400pt}}
\put(296,192.17){\rule{0.900pt}{0.400pt}}
\multiput(296.00,191.17)(2.132,2.000){2}{\rule{0.450pt}{0.400pt}}
\multiput(300.60,194.00)(0.468,2.090){5}{\rule{0.113pt}{1.600pt}}
\multiput(299.17,194.00)(4.000,11.679){2}{\rule{0.400pt}{0.800pt}}
\multiput(304.60,209.00)(0.468,3.552){5}{\rule{0.113pt}{2.600pt}}
\multiput(303.17,209.00)(4.000,19.604){2}{\rule{0.400pt}{1.300pt}}
\multiput(308.60,234.00)(0.468,0.627){5}{\rule{0.113pt}{0.600pt}}
\multiput(307.17,234.00)(4.000,3.755){2}{\rule{0.400pt}{0.300pt}}
\multiput(312.60,239.00)(0.468,1.066){5}{\rule{0.113pt}{0.900pt}}
\multiput(311.17,239.00)(4.000,6.132){2}{\rule{0.400pt}{0.450pt}}
\multiput(316.60,247.00)(0.468,0.920){5}{\rule{0.113pt}{0.800pt}}
\multiput(315.17,247.00)(4.000,5.340){2}{\rule{0.400pt}{0.400pt}}
\multiput(320.60,254.00)(0.468,1.066){5}{\rule{0.113pt}{0.900pt}}
\multiput(319.17,254.00)(4.000,6.132){2}{\rule{0.400pt}{0.450pt}}
\multiput(324.60,262.00)(0.468,0.920){5}{\rule{0.113pt}{0.800pt}}
\multiput(323.17,262.00)(4.000,5.340){2}{\rule{0.400pt}{0.400pt}}
\multiput(328.61,269.00)(0.447,1.579){3}{\rule{0.108pt}{1.167pt}}
\multiput(327.17,269.00)(3.000,5.579){2}{\rule{0.400pt}{0.583pt}}
\multiput(331.60,277.00)(0.468,0.627){5}{\rule{0.113pt}{0.600pt}}
\multiput(330.17,277.00)(4.000,3.755){2}{\rule{0.400pt}{0.300pt}}
\multiput(335.60,282.00)(0.468,0.774){5}{\rule{0.113pt}{0.700pt}}
\multiput(334.17,282.00)(4.000,4.547){2}{\rule{0.400pt}{0.350pt}}
\multiput(339.60,288.00)(0.468,0.774){5}{\rule{0.113pt}{0.700pt}}
\multiput(338.17,288.00)(4.000,4.547){2}{\rule{0.400pt}{0.350pt}}
\multiput(343.60,294.00)(0.468,0.627){5}{\rule{0.113pt}{0.600pt}}
\multiput(342.17,294.00)(4.000,3.755){2}{\rule{0.400pt}{0.300pt}}
\multiput(347.60,299.00)(0.468,0.774){5}{\rule{0.113pt}{0.700pt}}
\multiput(346.17,299.00)(4.000,4.547){2}{\rule{0.400pt}{0.350pt}}
\multiput(351.60,305.00)(0.468,0.774){5}{\rule{0.113pt}{0.700pt}}
\multiput(350.17,305.00)(4.000,4.547){2}{\rule{0.400pt}{0.350pt}}
\multiput(355.00,311.61)(0.685,0.447){3}{\rule{0.633pt}{0.108pt}}
\multiput(355.00,310.17)(2.685,3.000){2}{\rule{0.317pt}{0.400pt}}
\multiput(359.60,314.00)(0.468,1.066){5}{\rule{0.113pt}{0.900pt}}
\multiput(358.17,314.00)(4.000,6.132){2}{\rule{0.400pt}{0.450pt}}
\multiput(363.60,322.00)(0.468,0.774){5}{\rule{0.113pt}{0.700pt}}
\multiput(362.17,322.00)(4.000,4.547){2}{\rule{0.400pt}{0.350pt}}
\multiput(367.61,328.00)(0.447,1.355){3}{\rule{0.108pt}{1.033pt}}
\multiput(366.17,328.00)(3.000,4.855){2}{\rule{0.400pt}{0.517pt}}
\multiput(370.60,335.00)(0.468,0.774){5}{\rule{0.113pt}{0.700pt}}
\multiput(369.17,335.00)(4.000,4.547){2}{\rule{0.400pt}{0.350pt}}
\multiput(374.60,341.00)(0.468,0.920){5}{\rule{0.113pt}{0.800pt}}
\multiput(373.17,341.00)(4.000,5.340){2}{\rule{0.400pt}{0.400pt}}
\multiput(378.60,348.00)(0.468,0.774){5}{\rule{0.113pt}{0.700pt}}
\multiput(377.17,348.00)(4.000,4.547){2}{\rule{0.400pt}{0.350pt}}
\multiput(382.60,354.00)(0.468,0.920){5}{\rule{0.113pt}{0.800pt}}
\multiput(381.17,354.00)(4.000,5.340){2}{\rule{0.400pt}{0.400pt}}
\multiput(386.60,361.00)(0.468,0.774){5}{\rule{0.113pt}{0.700pt}}
\multiput(385.17,361.00)(4.000,4.547){2}{\rule{0.400pt}{0.350pt}}
\multiput(390.60,367.00)(0.468,1.066){5}{\rule{0.113pt}{0.900pt}}
\multiput(389.17,367.00)(4.000,6.132){2}{\rule{0.400pt}{0.450pt}}
\multiput(394.60,375.00)(0.468,0.920){5}{\rule{0.113pt}{0.800pt}}
\multiput(393.17,375.00)(4.000,5.340){2}{\rule{0.400pt}{0.400pt}}
\multiput(398.60,382.00)(0.468,0.774){5}{\rule{0.113pt}{0.700pt}}
\multiput(397.17,382.00)(4.000,4.547){2}{\rule{0.400pt}{0.350pt}}
\multiput(402.60,388.00)(0.468,0.627){5}{\rule{0.113pt}{0.600pt}}
\multiput(401.17,388.00)(4.000,3.755){2}{\rule{0.400pt}{0.300pt}}
\multiput(406.61,393.00)(0.447,1.579){3}{\rule{0.108pt}{1.167pt}}
\multiput(405.17,393.00)(3.000,5.579){2}{\rule{0.400pt}{0.583pt}}
\multiput(409.60,401.00)(0.468,0.774){5}{\rule{0.113pt}{0.700pt}}
\multiput(408.17,401.00)(4.000,4.547){2}{\rule{0.400pt}{0.350pt}}
\multiput(413.00,407.61)(0.685,0.447){3}{\rule{0.633pt}{0.108pt}}
\multiput(413.00,406.17)(2.685,3.000){2}{\rule{0.317pt}{0.400pt}}
\multiput(417.60,410.00)(0.468,0.774){5}{\rule{0.113pt}{0.700pt}}
\multiput(416.17,410.00)(4.000,4.547){2}{\rule{0.400pt}{0.350pt}}
\multiput(421.60,416.00)(0.468,1.066){5}{\rule{0.113pt}{0.900pt}}
\multiput(420.17,416.00)(4.000,6.132){2}{\rule{0.400pt}{0.450pt}}
\multiput(425.60,424.00)(0.468,0.627){5}{\rule{0.113pt}{0.600pt}}
\multiput(424.17,424.00)(4.000,3.755){2}{\rule{0.400pt}{0.300pt}}
\multiput(429.60,429.00)(0.468,0.774){5}{\rule{0.113pt}{0.700pt}}
\multiput(428.17,429.00)(4.000,4.547){2}{\rule{0.400pt}{0.350pt}}
\multiput(433.60,435.00)(0.468,0.627){5}{\rule{0.113pt}{0.600pt}}
\multiput(432.17,435.00)(4.000,3.755){2}{\rule{0.400pt}{0.300pt}}
\multiput(437.00,440.60)(0.481,0.468){5}{\rule{0.500pt}{0.113pt}}
\multiput(437.00,439.17)(2.962,4.000){2}{\rule{0.250pt}{0.400pt}}
\multiput(441.60,444.00)(0.468,0.774){5}{\rule{0.113pt}{0.700pt}}
\multiput(440.17,444.00)(4.000,4.547){2}{\rule{0.400pt}{0.350pt}}
\multiput(445.61,450.00)(0.447,1.355){3}{\rule{0.108pt}{1.033pt}}
\multiput(444.17,450.00)(3.000,4.855){2}{\rule{0.400pt}{0.517pt}}
\multiput(448.00,457.60)(0.481,0.468){5}{\rule{0.500pt}{0.113pt}}
\multiput(448.00,456.17)(2.962,4.000){2}{\rule{0.250pt}{0.400pt}}
\multiput(452.60,461.00)(0.468,0.774){5}{\rule{0.113pt}{0.700pt}}
\multiput(451.17,461.00)(4.000,4.547){2}{\rule{0.400pt}{0.350pt}}
\multiput(456.00,467.61)(0.685,0.447){3}{\rule{0.633pt}{0.108pt}}
\multiput(456.00,466.17)(2.685,3.000){2}{\rule{0.317pt}{0.400pt}}
\multiput(460.00,470.60)(0.481,0.468){5}{\rule{0.500pt}{0.113pt}}
\multiput(460.00,469.17)(2.962,4.000){2}{\rule{0.250pt}{0.400pt}}
\multiput(464.60,474.00)(0.468,0.774){5}{\rule{0.113pt}{0.700pt}}
\multiput(463.17,474.00)(4.000,4.547){2}{\rule{0.400pt}{0.350pt}}
\multiput(468.00,480.60)(0.481,0.468){5}{\rule{0.500pt}{0.113pt}}
\multiput(468.00,479.17)(2.962,4.000){2}{\rule{0.250pt}{0.400pt}}
\multiput(472.00,484.61)(0.685,0.447){3}{\rule{0.633pt}{0.108pt}}
\multiput(472.00,483.17)(2.685,3.000){2}{\rule{0.317pt}{0.400pt}}
\multiput(476.00,487.60)(0.481,0.468){5}{\rule{0.500pt}{0.113pt}}
\multiput(476.00,486.17)(2.962,4.000){2}{\rule{0.250pt}{0.400pt}}
\multiput(480.60,491.00)(0.468,0.774){5}{\rule{0.113pt}{0.700pt}}
\multiput(479.17,491.00)(4.000,4.547){2}{\rule{0.400pt}{0.350pt}}
\multiput(484.61,497.00)(0.447,0.685){3}{\rule{0.108pt}{0.633pt}}
\multiput(483.17,497.00)(3.000,2.685){2}{\rule{0.400pt}{0.317pt}}
\multiput(487.60,501.00)(0.468,0.627){5}{\rule{0.113pt}{0.600pt}}
\multiput(486.17,501.00)(4.000,3.755){2}{\rule{0.400pt}{0.300pt}}
\multiput(491.00,506.60)(0.481,0.468){5}{\rule{0.500pt}{0.113pt}}
\multiput(491.00,505.17)(2.962,4.000){2}{\rule{0.250pt}{0.400pt}}
\multiput(495.00,510.60)(0.481,0.468){5}{\rule{0.500pt}{0.113pt}}
\multiput(495.00,509.17)(2.962,4.000){2}{\rule{0.250pt}{0.400pt}}
\multiput(499.00,514.61)(0.685,0.447){3}{\rule{0.633pt}{0.108pt}}
\multiput(499.00,513.17)(2.685,3.000){2}{\rule{0.317pt}{0.400pt}}
\multiput(503.00,517.60)(0.481,0.468){5}{\rule{0.500pt}{0.113pt}}
\multiput(503.00,516.17)(2.962,4.000){2}{\rule{0.250pt}{0.400pt}}
\multiput(507.00,521.60)(0.481,0.468){5}{\rule{0.500pt}{0.113pt}}
\multiput(507.00,520.17)(2.962,4.000){2}{\rule{0.250pt}{0.400pt}}
\multiput(511.00,525.60)(0.481,0.468){5}{\rule{0.500pt}{0.113pt}}
\multiput(511.00,524.17)(2.962,4.000){2}{\rule{0.250pt}{0.400pt}}
\multiput(515.00,529.60)(0.481,0.468){5}{\rule{0.500pt}{0.113pt}}
\multiput(515.00,528.17)(2.962,4.000){2}{\rule{0.250pt}{0.400pt}}
\multiput(519.00,533.61)(0.685,0.447){3}{\rule{0.633pt}{0.108pt}}
\multiput(519.00,532.17)(2.685,3.000){2}{\rule{0.317pt}{0.400pt}}
\put(523,536.17){\rule{0.700pt}{0.400pt}}
\multiput(523.00,535.17)(1.547,2.000){2}{\rule{0.350pt}{0.400pt}}
\multiput(526.00,538.60)(0.481,0.468){5}{\rule{0.500pt}{0.113pt}}
\multiput(526.00,537.17)(2.962,4.000){2}{\rule{0.250pt}{0.400pt}}
\multiput(530.00,542.60)(0.481,0.468){5}{\rule{0.500pt}{0.113pt}}
\multiput(530.00,541.17)(2.962,4.000){2}{\rule{0.250pt}{0.400pt}}
\put(534,546.17){\rule{0.900pt}{0.400pt}}
\multiput(534.00,545.17)(2.132,2.000){2}{\rule{0.450pt}{0.400pt}}
\multiput(538.00,548.61)(0.685,0.447){3}{\rule{0.633pt}{0.108pt}}
\multiput(538.00,547.17)(2.685,3.000){2}{\rule{0.317pt}{0.400pt}}
\multiput(542.00,551.60)(0.481,0.468){5}{\rule{0.500pt}{0.113pt}}
\multiput(542.00,550.17)(2.962,4.000){2}{\rule{0.250pt}{0.400pt}}
\put(546,555.17){\rule{0.900pt}{0.400pt}}
\multiput(546.00,554.17)(2.132,2.000){2}{\rule{0.450pt}{0.400pt}}
\multiput(550.00,557.60)(0.481,0.468){5}{\rule{0.500pt}{0.113pt}}
\multiput(550.00,556.17)(2.962,4.000){2}{\rule{0.250pt}{0.400pt}}
\multiput(554.00,561.60)(0.481,0.468){5}{\rule{0.500pt}{0.113pt}}
\multiput(554.00,560.17)(2.962,4.000){2}{\rule{0.250pt}{0.400pt}}
\put(558,564.67){\rule{0.964pt}{0.400pt}}
\multiput(558.00,564.17)(2.000,1.000){2}{\rule{0.482pt}{0.400pt}}
\multiput(562.61,566.00)(0.447,0.685){3}{\rule{0.108pt}{0.633pt}}
\multiput(561.17,566.00)(3.000,2.685){2}{\rule{0.400pt}{0.317pt}}
\put(565,570.17){\rule{0.900pt}{0.400pt}}
\multiput(565.00,569.17)(2.132,2.000){2}{\rule{0.450pt}{0.400pt}}
\put(569,572.17){\rule{0.900pt}{0.400pt}}
\multiput(569.00,571.17)(2.132,2.000){2}{\rule{0.450pt}{0.400pt}}
\multiput(573.00,574.60)(0.481,0.468){5}{\rule{0.500pt}{0.113pt}}
\multiput(573.00,573.17)(2.962,4.000){2}{\rule{0.250pt}{0.400pt}}
\put(577,578.17){\rule{0.900pt}{0.400pt}}
\multiput(577.00,577.17)(2.132,2.000){2}{\rule{0.450pt}{0.400pt}}
\put(581,579.67){\rule{0.964pt}{0.400pt}}
\multiput(581.00,579.17)(2.000,1.000){2}{\rule{0.482pt}{0.400pt}}
\put(585,581.17){\rule{0.900pt}{0.400pt}}
\multiput(585.00,580.17)(2.132,2.000){2}{\rule{0.450pt}{0.400pt}}
\multiput(589.00,583.60)(0.481,0.468){5}{\rule{0.500pt}{0.113pt}}
\multiput(589.00,582.17)(2.962,4.000){2}{\rule{0.250pt}{0.400pt}}
\put(289.0,192.0){\rule[-0.200pt]{1.686pt}{0.400pt}}
\multiput(597.00,587.60)(0.481,0.468){5}{\rule{0.500pt}{0.113pt}}
\multiput(597.00,586.17)(2.962,4.000){2}{\rule{0.250pt}{0.400pt}}
\put(601,591.17){\rule{0.700pt}{0.400pt}}
\multiput(601.00,590.17)(1.547,2.000){2}{\rule{0.350pt}{0.400pt}}
\put(604,593.17){\rule{0.900pt}{0.400pt}}
\multiput(604.00,592.17)(2.132,2.000){2}{\rule{0.450pt}{0.400pt}}
\multiput(608.00,595.61)(0.685,0.447){3}{\rule{0.633pt}{0.108pt}}
\multiput(608.00,594.17)(2.685,3.000){2}{\rule{0.317pt}{0.400pt}}
\put(612,598.17){\rule{0.900pt}{0.400pt}}
\multiput(612.00,597.17)(2.132,2.000){2}{\rule{0.450pt}{0.400pt}}
\put(616,600.17){\rule{0.900pt}{0.400pt}}
\multiput(616.00,599.17)(2.132,2.000){2}{\rule{0.450pt}{0.400pt}}
\put(593.0,587.0){\rule[-0.200pt]{0.964pt}{0.400pt}}
\put(624,602.17){\rule{0.900pt}{0.400pt}}
\multiput(624.00,601.17)(2.132,2.000){2}{\rule{0.450pt}{0.400pt}}
\multiput(628.00,604.60)(0.481,0.468){5}{\rule{0.500pt}{0.113pt}}
\multiput(628.00,603.17)(2.962,4.000){2}{\rule{0.250pt}{0.400pt}}
\put(620.0,602.0){\rule[-0.200pt]{0.964pt}{0.400pt}}
\put(636,608.17){\rule{0.900pt}{0.400pt}}
\multiput(636.00,607.17)(2.132,2.000){2}{\rule{0.450pt}{0.400pt}}
\multiput(640.00,610.61)(0.462,0.447){3}{\rule{0.500pt}{0.108pt}}
\multiput(640.00,609.17)(1.962,3.000){2}{\rule{0.250pt}{0.400pt}}
\put(643,613.17){\rule{0.900pt}{0.400pt}}
\multiput(643.00,612.17)(2.132,2.000){2}{\rule{0.450pt}{0.400pt}}
\put(632.0,608.0){\rule[-0.200pt]{0.964pt}{0.400pt}}
\put(651,615.17){\rule{0.900pt}{0.400pt}}
\multiput(651.00,614.17)(2.132,2.000){2}{\rule{0.450pt}{0.400pt}}
\multiput(655.00,617.60)(0.481,0.468){5}{\rule{0.500pt}{0.113pt}}
\multiput(655.00,616.17)(2.962,4.000){2}{\rule{0.250pt}{0.400pt}}
\put(647.0,615.0){\rule[-0.200pt]{0.964pt}{0.400pt}}
\put(663,621.17){\rule{0.900pt}{0.400pt}}
\multiput(663.00,620.17)(2.132,2.000){2}{\rule{0.450pt}{0.400pt}}
\put(667,623.17){\rule{0.900pt}{0.400pt}}
\multiput(667.00,622.17)(2.132,2.000){2}{\rule{0.450pt}{0.400pt}}
\put(671,625.17){\rule{0.900pt}{0.400pt}}
\multiput(671.00,624.17)(2.132,2.000){2}{\rule{0.450pt}{0.400pt}}
\put(675,626.67){\rule{0.964pt}{0.400pt}}
\multiput(675.00,626.17)(2.000,1.000){2}{\rule{0.482pt}{0.400pt}}
\put(659.0,621.0){\rule[-0.200pt]{0.964pt}{0.400pt}}
\put(682,628.17){\rule{0.900pt}{0.400pt}}
\multiput(682.00,627.17)(2.132,2.000){2}{\rule{0.450pt}{0.400pt}}
\put(686,630.17){\rule{0.900pt}{0.400pt}}
\multiput(686.00,629.17)(2.132,2.000){2}{\rule{0.450pt}{0.400pt}}
\put(690,632.17){\rule{0.900pt}{0.400pt}}
\multiput(690.00,631.17)(2.132,2.000){2}{\rule{0.450pt}{0.400pt}}
\put(679.0,628.0){\rule[-0.200pt]{0.723pt}{0.400pt}}
\multiput(698.00,634.60)(0.481,0.468){5}{\rule{0.500pt}{0.113pt}}
\multiput(698.00,633.17)(2.962,4.000){2}{\rule{0.250pt}{0.400pt}}
\put(694.0,634.0){\rule[-0.200pt]{0.964pt}{0.400pt}}
\put(706,638.17){\rule{0.900pt}{0.400pt}}
\multiput(706.00,637.17)(2.132,2.000){2}{\rule{0.450pt}{0.400pt}}
\put(710,640.17){\rule{0.900pt}{0.400pt}}
\multiput(710.00,639.17)(2.132,2.000){2}{\rule{0.450pt}{0.400pt}}
\put(714,641.67){\rule{0.964pt}{0.400pt}}
\multiput(714.00,641.17)(2.000,1.000){2}{\rule{0.482pt}{0.400pt}}
\put(702.0,638.0){\rule[-0.200pt]{0.964pt}{0.400pt}}
\put(721,643.17){\rule{0.900pt}{0.400pt}}
\multiput(721.00,642.17)(2.132,2.000){2}{\rule{0.450pt}{0.400pt}}
\put(725,645.17){\rule{0.900pt}{0.400pt}}
\multiput(725.00,644.17)(2.132,2.000){2}{\rule{0.450pt}{0.400pt}}
\put(729,647.17){\rule{0.900pt}{0.400pt}}
\multiput(729.00,646.17)(2.132,2.000){2}{\rule{0.450pt}{0.400pt}}
\put(733,649.17){\rule{0.900pt}{0.400pt}}
\multiput(733.00,648.17)(2.132,2.000){2}{\rule{0.450pt}{0.400pt}}
\put(718.0,643.0){\rule[-0.200pt]{0.723pt}{0.400pt}}
\put(741,651.17){\rule{0.900pt}{0.400pt}}
\multiput(741.00,650.17)(2.132,2.000){2}{\rule{0.450pt}{0.400pt}}
\put(745,653.17){\rule{0.900pt}{0.400pt}}
\multiput(745.00,652.17)(2.132,2.000){2}{\rule{0.450pt}{0.400pt}}
\put(737.0,651.0){\rule[-0.200pt]{0.964pt}{0.400pt}}
\put(753,655.17){\rule{0.900pt}{0.400pt}}
\multiput(753.00,654.17)(2.132,2.000){2}{\rule{0.450pt}{0.400pt}}
\put(757,657.17){\rule{0.700pt}{0.400pt}}
\multiput(757.00,656.17)(1.547,2.000){2}{\rule{0.350pt}{0.400pt}}
\put(749.0,655.0){\rule[-0.200pt]{0.964pt}{0.400pt}}
\put(764,658.67){\rule{0.964pt}{0.400pt}}
\multiput(764.00,658.17)(2.000,1.000){2}{\rule{0.482pt}{0.400pt}}
\put(768,660.17){\rule{0.900pt}{0.400pt}}
\multiput(768.00,659.17)(2.132,2.000){2}{\rule{0.450pt}{0.400pt}}
\put(760.0,659.0){\rule[-0.200pt]{0.964pt}{0.400pt}}
\multiput(776.00,662.60)(0.481,0.468){5}{\rule{0.500pt}{0.113pt}}
\multiput(776.00,661.17)(2.962,4.000){2}{\rule{0.250pt}{0.400pt}}
\put(780,664.17){\rule{0.900pt}{0.400pt}}
\multiput(780.00,665.17)(2.132,-2.000){2}{\rule{0.450pt}{0.400pt}}
\put(784,664.17){\rule{0.900pt}{0.400pt}}
\multiput(784.00,663.17)(2.132,2.000){2}{\rule{0.450pt}{0.400pt}}
\put(788,666.17){\rule{0.900pt}{0.400pt}}
\multiput(788.00,665.17)(2.132,2.000){2}{\rule{0.450pt}{0.400pt}}
\put(772.0,662.0){\rule[-0.200pt]{0.964pt}{0.400pt}}
\put(796,668.17){\rule{0.700pt}{0.400pt}}
\multiput(796.00,667.17)(1.547,2.000){2}{\rule{0.350pt}{0.400pt}}
\put(799,670.17){\rule{0.900pt}{0.400pt}}
\multiput(799.00,669.17)(2.132,2.000){2}{\rule{0.450pt}{0.400pt}}
\put(792.0,668.0){\rule[-0.200pt]{0.964pt}{0.400pt}}
\put(807,672.17){\rule{0.900pt}{0.400pt}}
\multiput(807.00,671.17)(2.132,2.000){2}{\rule{0.450pt}{0.400pt}}
\put(803.0,672.0){\rule[-0.200pt]{0.964pt}{0.400pt}}
\put(815,673.67){\rule{0.964pt}{0.400pt}}
\multiput(815.00,673.17)(2.000,1.000){2}{\rule{0.482pt}{0.400pt}}
\put(819,675.17){\rule{0.900pt}{0.400pt}}
\multiput(819.00,674.17)(2.132,2.000){2}{\rule{0.450pt}{0.400pt}}
\put(811.0,674.0){\rule[-0.200pt]{0.964pt}{0.400pt}}
\put(827,677.17){\rule{0.900pt}{0.400pt}}
\multiput(827.00,676.17)(2.132,2.000){2}{\rule{0.450pt}{0.400pt}}
\put(823.0,677.0){\rule[-0.200pt]{0.964pt}{0.400pt}}
\put(835,679.17){\rule{0.700pt}{0.400pt}}
\multiput(835.00,678.17)(1.547,2.000){2}{\rule{0.350pt}{0.400pt}}
\put(831.0,679.0){\rule[-0.200pt]{0.964pt}{0.400pt}}
\put(842,681.17){\rule{0.900pt}{0.400pt}}
\multiput(842.00,680.17)(2.132,2.000){2}{\rule{0.450pt}{0.400pt}}
\put(838.0,681.0){\rule[-0.200pt]{0.964pt}{0.400pt}}
\put(850,683.17){\rule{0.900pt}{0.400pt}}
\multiput(850.00,682.17)(2.132,2.000){2}{\rule{0.450pt}{0.400pt}}
\put(854,685.17){\rule{0.900pt}{0.400pt}}
\multiput(854.00,684.17)(2.132,2.000){2}{\rule{0.450pt}{0.400pt}}
\put(846.0,683.0){\rule[-0.200pt]{0.964pt}{0.400pt}}
\put(862,687.17){\rule{0.900pt}{0.400pt}}
\multiput(862.00,686.17)(2.132,2.000){2}{\rule{0.450pt}{0.400pt}}
\put(858.0,687.0){\rule[-0.200pt]{0.964pt}{0.400pt}}
\put(874,688.67){\rule{0.723pt}{0.400pt}}
\multiput(874.00,688.17)(1.500,1.000){2}{\rule{0.361pt}{0.400pt}}
\put(866.0,689.0){\rule[-0.200pt]{1.927pt}{0.400pt}}
\put(881,690.17){\rule{0.900pt}{0.400pt}}
\multiput(881.00,689.17)(2.132,2.000){2}{\rule{0.450pt}{0.400pt}}
\put(877.0,690.0){\rule[-0.200pt]{0.964pt}{0.400pt}}
\put(889,692.17){\rule{0.900pt}{0.400pt}}
\multiput(889.00,691.17)(2.132,2.000){2}{\rule{0.450pt}{0.400pt}}
\put(885.0,692.0){\rule[-0.200pt]{0.964pt}{0.400pt}}
\put(897,694.17){\rule{0.900pt}{0.400pt}}
\multiput(897.00,693.17)(2.132,2.000){2}{\rule{0.450pt}{0.400pt}}
\put(893.0,694.0){\rule[-0.200pt]{0.964pt}{0.400pt}}
\put(905,696.17){\rule{0.900pt}{0.400pt}}
\multiput(905.00,695.17)(2.132,2.000){2}{\rule{0.450pt}{0.400pt}}
\put(909,696.17){\rule{0.900pt}{0.400pt}}
\multiput(909.00,697.17)(2.132,-2.000){2}{\rule{0.450pt}{0.400pt}}
\put(913,696.17){\rule{0.700pt}{0.400pt}}
\multiput(913.00,695.17)(1.547,2.000){2}{\rule{0.350pt}{0.400pt}}
\put(916,698.17){\rule{0.900pt}{0.400pt}}
\multiput(916.00,697.17)(2.132,2.000){2}{\rule{0.450pt}{0.400pt}}
\put(901.0,696.0){\rule[-0.200pt]{0.964pt}{0.400pt}}
\put(928,700.17){\rule{0.900pt}{0.400pt}}
\multiput(928.00,699.17)(2.132,2.000){2}{\rule{0.450pt}{0.400pt}}
\put(920.0,700.0){\rule[-0.200pt]{1.927pt}{0.400pt}}
\put(940,702.17){\rule{0.900pt}{0.400pt}}
\multiput(940.00,701.17)(2.132,2.000){2}{\rule{0.450pt}{0.400pt}}
\put(932.0,702.0){\rule[-0.200pt]{1.927pt}{0.400pt}}
\put(952,703.67){\rule{0.723pt}{0.400pt}}
\multiput(952.00,703.17)(1.500,1.000){2}{\rule{0.361pt}{0.400pt}}
\put(955,705.17){\rule{0.900pt}{0.400pt}}
\multiput(955.00,704.17)(2.132,2.000){2}{\rule{0.450pt}{0.400pt}}
\put(959,705.17){\rule{0.900pt}{0.400pt}}
\multiput(959.00,706.17)(2.132,-2.000){2}{\rule{0.450pt}{0.400pt}}
\put(944.0,704.0){\rule[-0.200pt]{1.927pt}{0.400pt}}
\put(967,705.17){\rule{0.900pt}{0.400pt}}
\multiput(967.00,704.17)(2.132,2.000){2}{\rule{0.450pt}{0.400pt}}
\put(971,707.17){\rule{0.900pt}{0.400pt}}
\multiput(971.00,706.17)(2.132,2.000){2}{\rule{0.450pt}{0.400pt}}
\put(963.0,705.0){\rule[-0.200pt]{0.964pt}{0.400pt}}
\put(983,709.17){\rule{0.900pt}{0.400pt}}
\multiput(983.00,708.17)(2.132,2.000){2}{\rule{0.450pt}{0.400pt}}
\put(987,709.17){\rule{0.900pt}{0.400pt}}
\multiput(987.00,710.17)(2.132,-2.000){2}{\rule{0.450pt}{0.400pt}}
\put(991,709.17){\rule{0.700pt}{0.400pt}}
\multiput(991.00,708.17)(1.547,2.000){2}{\rule{0.350pt}{0.400pt}}
\put(994,711.17){\rule{0.900pt}{0.400pt}}
\multiput(994.00,710.17)(2.132,2.000){2}{\rule{0.450pt}{0.400pt}}
\put(975.0,709.0){\rule[-0.200pt]{1.927pt}{0.400pt}}
\put(1006,713.17){\rule{0.900pt}{0.400pt}}
\multiput(1006.00,712.17)(2.132,2.000){2}{\rule{0.450pt}{0.400pt}}
\put(1010,715.17){\rule{0.900pt}{0.400pt}}
\multiput(1010.00,714.17)(2.132,2.000){2}{\rule{0.450pt}{0.400pt}}
\put(998.0,713.0){\rule[-0.200pt]{1.927pt}{0.400pt}}
\put(1018,717.17){\rule{0.900pt}{0.400pt}}
\multiput(1018.00,716.17)(2.132,2.000){2}{\rule{0.450pt}{0.400pt}}
\put(1014.0,717.0){\rule[-0.200pt]{0.964pt}{0.400pt}}
\put(1041,719.17){\rule{0.900pt}{0.400pt}}
\multiput(1041.00,718.17)(2.132,2.000){2}{\rule{0.450pt}{0.400pt}}
\put(1045,719.17){\rule{0.900pt}{0.400pt}}
\multiput(1045.00,720.17)(2.132,-2.000){2}{\rule{0.450pt}{0.400pt}}
\put(1049,719.17){\rule{0.900pt}{0.400pt}}
\multiput(1049.00,718.17)(2.132,2.000){2}{\rule{0.450pt}{0.400pt}}
\put(1022.0,719.0){\rule[-0.200pt]{4.577pt}{0.400pt}}
\put(1061,720.67){\rule{0.964pt}{0.400pt}}
\multiput(1061.00,720.17)(2.000,1.000){2}{\rule{0.482pt}{0.400pt}}
\put(1065,720.67){\rule{0.964pt}{0.400pt}}
\multiput(1065.00,721.17)(2.000,-1.000){2}{\rule{0.482pt}{0.400pt}}
\put(1069,720.67){\rule{0.723pt}{0.400pt}}
\multiput(1069.00,720.17)(1.500,1.000){2}{\rule{0.361pt}{0.400pt}}
\put(1072,722.17){\rule{0.900pt}{0.400pt}}
\multiput(1072.00,721.17)(2.132,2.000){2}{\rule{0.450pt}{0.400pt}}
\put(1053.0,721.0){\rule[-0.200pt]{1.927pt}{0.400pt}}
\put(1084,724.17){\rule{0.900pt}{0.400pt}}
\multiput(1084.00,723.17)(2.132,2.000){2}{\rule{0.450pt}{0.400pt}}
\put(1088,724.17){\rule{0.900pt}{0.400pt}}
\multiput(1088.00,725.17)(2.132,-2.000){2}{\rule{0.450pt}{0.400pt}}
\put(1092,724.17){\rule{0.900pt}{0.400pt}}
\multiput(1092.00,723.17)(2.132,2.000){2}{\rule{0.450pt}{0.400pt}}
\put(1076.0,724.0){\rule[-0.200pt]{1.927pt}{0.400pt}}
\put(1100,726.17){\rule{0.900pt}{0.400pt}}
\multiput(1100.00,725.17)(2.132,2.000){2}{\rule{0.450pt}{0.400pt}}
\put(1096.0,726.0){\rule[-0.200pt]{0.964pt}{0.400pt}}
\put(1115,728.17){\rule{0.900pt}{0.400pt}}
\multiput(1115.00,727.17)(2.132,2.000){2}{\rule{0.450pt}{0.400pt}}
\put(1119,728.17){\rule{0.900pt}{0.400pt}}
\multiput(1119.00,729.17)(2.132,-2.000){2}{\rule{0.450pt}{0.400pt}}
\put(1123,728.17){\rule{0.900pt}{0.400pt}}
\multiput(1123.00,727.17)(2.132,2.000){2}{\rule{0.450pt}{0.400pt}}
\put(1104.0,728.0){\rule[-0.200pt]{2.650pt}{0.400pt}}
\put(1135,730.17){\rule{0.900pt}{0.400pt}}
\multiput(1135.00,729.17)(2.132,2.000){2}{\rule{0.450pt}{0.400pt}}
\put(1139,730.17){\rule{0.900pt}{0.400pt}}
\multiput(1139.00,731.17)(2.132,-2.000){2}{\rule{0.450pt}{0.400pt}}
\put(1127.0,730.0){\rule[-0.200pt]{1.927pt}{0.400pt}}
\put(1147,730.17){\rule{0.700pt}{0.400pt}}
\multiput(1147.00,729.17)(1.547,2.000){2}{\rule{0.350pt}{0.400pt}}
\put(1143.0,730.0){\rule[-0.200pt]{0.964pt}{0.400pt}}
\put(1162,732.17){\rule{0.900pt}{0.400pt}}
\multiput(1162.00,731.17)(2.132,2.000){2}{\rule{0.450pt}{0.400pt}}
\put(1166,732.17){\rule{0.900pt}{0.400pt}}
\multiput(1166.00,733.17)(2.132,-2.000){2}{\rule{0.450pt}{0.400pt}}
\put(1170,732.17){\rule{0.900pt}{0.400pt}}
\multiput(1170.00,731.17)(2.132,2.000){2}{\rule{0.450pt}{0.400pt}}
\put(1150.0,732.0){\rule[-0.200pt]{2.891pt}{0.400pt}}
\put(1178,734.17){\rule{0.900pt}{0.400pt}}
\multiput(1178.00,733.17)(2.132,2.000){2}{\rule{0.450pt}{0.400pt}}
\put(1174.0,734.0){\rule[-0.200pt]{0.964pt}{0.400pt}}
\put(1189,735.67){\rule{0.964pt}{0.400pt}}
\multiput(1189.00,735.17)(2.000,1.000){2}{\rule{0.482pt}{0.400pt}}
\put(1182.0,736.0){\rule[-0.200pt]{1.686pt}{0.400pt}}
\put(1205,737.17){\rule{0.900pt}{0.400pt}}
\multiput(1205.00,736.17)(2.132,2.000){2}{\rule{0.450pt}{0.400pt}}
\put(1209,737.17){\rule{0.900pt}{0.400pt}}
\multiput(1209.00,738.17)(2.132,-2.000){2}{\rule{0.450pt}{0.400pt}}
\put(1213,737.17){\rule{0.900pt}{0.400pt}}
\multiput(1213.00,736.17)(2.132,2.000){2}{\rule{0.450pt}{0.400pt}}
\put(1193.0,737.0){\rule[-0.200pt]{2.891pt}{0.400pt}}
\put(1232,739.17){\rule{0.900pt}{0.400pt}}
\multiput(1232.00,738.17)(2.132,2.000){2}{\rule{0.450pt}{0.400pt}}
\put(1236,739.17){\rule{0.900pt}{0.400pt}}
\multiput(1236.00,740.17)(2.132,-2.000){2}{\rule{0.450pt}{0.400pt}}
\put(1240,739.17){\rule{0.900pt}{0.400pt}}
\multiput(1240.00,738.17)(2.132,2.000){2}{\rule{0.450pt}{0.400pt}}
\put(1217.0,739.0){\rule[-0.200pt]{3.613pt}{0.400pt}}
\put(1252,741.17){\rule{0.900pt}{0.400pt}}
\multiput(1252.00,740.17)(2.132,2.000){2}{\rule{0.450pt}{0.400pt}}
\put(1256,741.17){\rule{0.900pt}{0.400pt}}
\multiput(1256.00,742.17)(2.132,-2.000){2}{\rule{0.450pt}{0.400pt}}
\put(1260,741.17){\rule{0.900pt}{0.400pt}}
\multiput(1260.00,740.17)(2.132,2.000){2}{\rule{0.450pt}{0.400pt}}
\put(1264,741.17){\rule{0.700pt}{0.400pt}}
\multiput(1264.00,742.17)(1.547,-2.000){2}{\rule{0.350pt}{0.400pt}}
\put(1267,741.17){\rule{0.900pt}{0.400pt}}
\multiput(1267.00,740.17)(2.132,2.000){2}{\rule{0.450pt}{0.400pt}}
\put(1244.0,741.0){\rule[-0.200pt]{1.927pt}{0.400pt}}
\put(1279,743.17){\rule{0.900pt}{0.400pt}}
\multiput(1279.00,742.17)(2.132,2.000){2}{\rule{0.450pt}{0.400pt}}
\put(1283,743.17){\rule{0.900pt}{0.400pt}}
\multiput(1283.00,744.17)(2.132,-2.000){2}{\rule{0.450pt}{0.400pt}}
\put(1271.0,743.0){\rule[-0.200pt]{1.927pt}{0.400pt}}
\put(1291,743.17){\rule{0.900pt}{0.400pt}}
\multiput(1291.00,742.17)(2.132,2.000){2}{\rule{0.450pt}{0.400pt}}
\put(1287.0,743.0){\rule[-0.200pt]{0.964pt}{0.400pt}}
\put(1310,745.17){\rule{0.900pt}{0.400pt}}
\multiput(1310.00,744.17)(2.132,2.000){2}{\rule{0.450pt}{0.400pt}}
\put(1314,745.17){\rule{0.900pt}{0.400pt}}
\multiput(1314.00,746.17)(2.132,-2.000){2}{\rule{0.450pt}{0.400pt}}
\put(1318,745.17){\rule{0.900pt}{0.400pt}}
\multiput(1318.00,744.17)(2.132,2.000){2}{\rule{0.450pt}{0.400pt}}
\put(1295.0,745.0){\rule[-0.200pt]{3.613pt}{0.400pt}}
\put(1342,747.17){\rule{0.700pt}{0.400pt}}
\multiput(1342.00,746.17)(1.547,2.000){2}{\rule{0.350pt}{0.400pt}}
\put(1322.0,747.0){\rule[-0.200pt]{4.818pt}{0.400pt}}
\put(1353,747.17){\rule{0.900pt}{0.400pt}}
\multiput(1353.00,748.17)(2.132,-2.000){2}{\rule{0.450pt}{0.400pt}}
\put(1357,747.17){\rule{0.900pt}{0.400pt}}
\multiput(1357.00,746.17)(2.132,2.000){2}{\rule{0.450pt}{0.400pt}}
\put(1345.0,749.0){\rule[-0.200pt]{1.927pt}{0.400pt}}
\put(1361.0,749.0){\rule[-0.200pt]{7.468pt}{0.400pt}}
\put(191.0,131.0){\rule[-0.200pt]{0.400pt}{155.380pt}}
\put(191.0,131.0){\rule[-0.200pt]{300.643pt}{0.400pt}}
\put(1439.0,131.0){\rule[-0.200pt]{0.400pt}{155.380pt}}
\put(191.0,776.0){\rule[-0.200pt]{300.643pt}{0.400pt}}
\end{picture}
}
  \caption{Trial 3, temperature = 22.4 \textdegree C}
\label{fig:05m4}
\end {minipage}%
\hfill%
\begin{minipage}{0.5\textwidth}%
  \raggedleft
    	\resizebox{0.9\textwidth}{!}{% GNUPLOT: LaTeX picture
\setlength{\unitlength}{0.240900pt}
\ifx\plotpoint\undefined\newsavebox{\plotpoint}\fi
\sbox{\plotpoint}{\rule[-0.200pt]{0.400pt}{0.400pt}}%
\begin{picture}(1500,900)(0,0)
\sbox{\plotpoint}{\rule[-0.200pt]{0.400pt}{0.400pt}}%
\put(191.0,131.0){\rule[-0.200pt]{4.818pt}{0.400pt}}
\put(171,131){\makebox(0,0)[r]{ 0.94}}
\put(1419.0,131.0){\rule[-0.200pt]{4.818pt}{0.400pt}}
\put(191.0,190.0){\rule[-0.200pt]{4.818pt}{0.400pt}}
\put(171,190){\makebox(0,0)[r]{ 0.96}}
\put(1419.0,190.0){\rule[-0.200pt]{4.818pt}{0.400pt}}
\put(191.0,248.0){\rule[-0.200pt]{4.818pt}{0.400pt}}
\put(171,248){\makebox(0,0)[r]{ 0.98}}
\put(1419.0,248.0){\rule[-0.200pt]{4.818pt}{0.400pt}}
\put(191.0,307.0){\rule[-0.200pt]{4.818pt}{0.400pt}}
\put(171,307){\makebox(0,0)[r]{ 1}}
\put(1419.0,307.0){\rule[-0.200pt]{4.818pt}{0.400pt}}
\put(191.0,366.0){\rule[-0.200pt]{4.818pt}{0.400pt}}
\put(171,366){\makebox(0,0)[r]{ 1.02}}
\put(1419.0,366.0){\rule[-0.200pt]{4.818pt}{0.400pt}}
\put(191.0,424.0){\rule[-0.200pt]{4.818pt}{0.400pt}}
\put(171,424){\makebox(0,0)[r]{ 1.04}}
\put(1419.0,424.0){\rule[-0.200pt]{4.818pt}{0.400pt}}
\put(191.0,483.0){\rule[-0.200pt]{4.818pt}{0.400pt}}
\put(171,483){\makebox(0,0)[r]{ 1.06}}
\put(1419.0,483.0){\rule[-0.200pt]{4.818pt}{0.400pt}}
\put(191.0,541.0){\rule[-0.200pt]{4.818pt}{0.400pt}}
\put(171,541){\makebox(0,0)[r]{ 1.08}}
\put(1419.0,541.0){\rule[-0.200pt]{4.818pt}{0.400pt}}
\put(191.0,600.0){\rule[-0.200pt]{4.818pt}{0.400pt}}
\put(171,600){\makebox(0,0)[r]{ 1.1}}
\put(1419.0,600.0){\rule[-0.200pt]{4.818pt}{0.400pt}}
\put(191.0,659.0){\rule[-0.200pt]{4.818pt}{0.400pt}}
\put(171,659){\makebox(0,0)[r]{ 1.12}}
\put(1419.0,659.0){\rule[-0.200pt]{4.818pt}{0.400pt}}
\put(191.0,717.0){\rule[-0.200pt]{4.818pt}{0.400pt}}
\put(171,717){\makebox(0,0)[r]{ 1.14}}
\put(1419.0,717.0){\rule[-0.200pt]{4.818pt}{0.400pt}}
\put(191.0,776.0){\rule[-0.200pt]{4.818pt}{0.400pt}}
\put(171,776){\makebox(0,0)[r]{ 1.16}}
\put(1419.0,776.0){\rule[-0.200pt]{4.818pt}{0.400pt}}
\put(191.0,131.0){\rule[-0.200pt]{0.400pt}{4.818pt}}
\put(191,90){\makebox(0,0){ 0}}
\put(191.0,756.0){\rule[-0.200pt]{0.400pt}{4.818pt}}
\put(330.0,131.0){\rule[-0.200pt]{0.400pt}{4.818pt}}
\put(330,90){\makebox(0,0){ 5}}
\put(330.0,756.0){\rule[-0.200pt]{0.400pt}{4.818pt}}
\put(468.0,131.0){\rule[-0.200pt]{0.400pt}{4.818pt}}
\put(468,90){\makebox(0,0){ 10}}
\put(468.0,756.0){\rule[-0.200pt]{0.400pt}{4.818pt}}
\put(607.0,131.0){\rule[-0.200pt]{0.400pt}{4.818pt}}
\put(607,90){\makebox(0,0){ 15}}
\put(607.0,756.0){\rule[-0.200pt]{0.400pt}{4.818pt}}
\put(746.0,131.0){\rule[-0.200pt]{0.400pt}{4.818pt}}
\put(746,90){\makebox(0,0){ 20}}
\put(746.0,756.0){\rule[-0.200pt]{0.400pt}{4.818pt}}
\put(884.0,131.0){\rule[-0.200pt]{0.400pt}{4.818pt}}
\put(884,90){\makebox(0,0){ 25}}
\put(884.0,756.0){\rule[-0.200pt]{0.400pt}{4.818pt}}
\put(1023.0,131.0){\rule[-0.200pt]{0.400pt}{4.818pt}}
\put(1023,90){\makebox(0,0){ 30}}
\put(1023.0,756.0){\rule[-0.200pt]{0.400pt}{4.818pt}}
\put(1162.0,131.0){\rule[-0.200pt]{0.400pt}{4.818pt}}
\put(1162,90){\makebox(0,0){ 35}}
\put(1162.0,756.0){\rule[-0.200pt]{0.400pt}{4.818pt}}
\put(1300.0,131.0){\rule[-0.200pt]{0.400pt}{4.818pt}}
\put(1300,90){\makebox(0,0){ 40}}
\put(1300.0,756.0){\rule[-0.200pt]{0.400pt}{4.818pt}}
\put(1439.0,131.0){\rule[-0.200pt]{0.400pt}{4.818pt}}
\put(1439,90){\makebox(0,0){ 45}}
\put(1439.0,756.0){\rule[-0.200pt]{0.400pt}{4.818pt}}
\put(191.0,131.0){\rule[-0.200pt]{0.400pt}{155.380pt}}
\put(191.0,131.0){\rule[-0.200pt]{300.643pt}{0.400pt}}
\put(1439.0,131.0){\rule[-0.200pt]{0.400pt}{155.380pt}}
\put(191.0,776.0){\rule[-0.200pt]{300.643pt}{0.400pt}}
\put(30,453){\makebox(0,0){\hspace{-72pt} Pressure (atm)}}
\put(815,29){\makebox(0,0){Time (seconds)}}
\put(815,838){\makebox(0,0){Cold Bath 0.04g Mg Powder - 10mL 1M HCl Reaction Pressure v. Time}}
\put(191,188){\rule{1pt}{1pt}}
\put(198,188){\rule{1pt}{1pt}}
\put(205,188){\rule{1pt}{1pt}}
\put(212,188){\rule{1pt}{1pt}}
\put(219,188){\rule{1pt}{1pt}}
\put(226,188){\rule{1pt}{1pt}}
\put(233,188){\rule{1pt}{1pt}}
\put(240,188){\rule{1pt}{1pt}}
\put(246,188){\rule{1pt}{1pt}}
\put(253,188){\rule{1pt}{1pt}}
\put(260,188){\rule{1pt}{1pt}}
\put(267,188){\rule{1pt}{1pt}}
\put(274,188){\rule{1pt}{1pt}}
\put(281,188){\rule{1pt}{1pt}}
\put(288,188){\rule{1pt}{1pt}}
\put(295,188){\rule{1pt}{1pt}}
\put(302,188){\rule{1pt}{1pt}}
\put(309,188){\rule{1pt}{1pt}}
\put(316,188){\rule{1pt}{1pt}}
\put(323,192){\rule{1pt}{1pt}}
\put(330,228){\rule{1pt}{1pt}}
\put(337,241){\rule{1pt}{1pt}}
\put(344,255){\rule{1pt}{1pt}}
\put(350,270){\rule{1pt}{1pt}}
\put(357,286){\rule{1pt}{1pt}}
\put(364,296){\rule{1pt}{1pt}}
\put(371,310){\rule{1pt}{1pt}}
\put(378,323){\rule{1pt}{1pt}}
\put(385,332){\rule{1pt}{1pt}}
\put(392,346){\rule{1pt}{1pt}}
\put(399,358){\rule{1pt}{1pt}}
\put(406,371){\rule{1pt}{1pt}}
\put(413,385){\rule{1pt}{1pt}}
\put(420,399){\rule{1pt}{1pt}}
\put(427,412){\rule{1pt}{1pt}}
\put(434,426){\rule{1pt}{1pt}}
\put(441,440){\rule{1pt}{1pt}}
\put(448,452){\rule{1pt}{1pt}}
\put(454,465){\rule{1pt}{1pt}}
\put(461,477){\rule{1pt}{1pt}}
\put(468,487){\rule{1pt}{1pt}}
\put(475,499){\rule{1pt}{1pt}}
\put(482,510){\rule{1pt}{1pt}}
\put(489,520){\rule{1pt}{1pt}}
\put(496,528){\rule{1pt}{1pt}}
\put(503,539){\rule{1pt}{1pt}}
\put(510,547){\rule{1pt}{1pt}}
\put(517,554){\rule{1pt}{1pt}}
\put(524,563){\rule{1pt}{1pt}}
\put(531,568){\rule{1pt}{1pt}}
\put(538,575){\rule{1pt}{1pt}}
\put(545,582){\rule{1pt}{1pt}}
\put(552,587){\rule{1pt}{1pt}}
\put(558,592){\rule{1pt}{1pt}}
\put(565,597){\rule{1pt}{1pt}}
\put(572,600){\rule{1pt}{1pt}}
\put(579,605){\rule{1pt}{1pt}}
\put(586,609){\rule{1pt}{1pt}}
\put(593,611){\rule{1pt}{1pt}}
\put(600,616){\rule{1pt}{1pt}}
\put(607,619){\rule{1pt}{1pt}}
\put(614,624){\rule{1pt}{1pt}}
\put(621,631){\rule{1pt}{1pt}}
\put(628,634){\rule{1pt}{1pt}}
\put(635,636){\rule{1pt}{1pt}}
\put(642,640){\rule{1pt}{1pt}}
\put(649,643){\rule{1pt}{1pt}}
\put(656,645){\rule{1pt}{1pt}}
\put(662,646){\rule{1pt}{1pt}}
\put(669,648){\rule{1pt}{1pt}}
\put(676,650){\rule{1pt}{1pt}}
\put(683,652){\rule{1pt}{1pt}}
\put(690,653){\rule{1pt}{1pt}}
\put(697,655){\rule{1pt}{1pt}}
\put(704,657){\rule{1pt}{1pt}}
\put(711,658){\rule{1pt}{1pt}}
\put(718,660){\rule{1pt}{1pt}}
\put(725,662){\rule{1pt}{1pt}}
\put(732,665){\rule{1pt}{1pt}}
\put(739,669){\rule{1pt}{1pt}}
\put(746,669){\rule{1pt}{1pt}}
\put(753,670){\rule{1pt}{1pt}}
\put(760,672){\rule{1pt}{1pt}}
\put(766,676){\rule{1pt}{1pt}}
\put(773,677){\rule{1pt}{1pt}}
\put(780,679){\rule{1pt}{1pt}}
\put(787,679){\rule{1pt}{1pt}}
\put(794,682){\rule{1pt}{1pt}}
\put(801,684){\rule{1pt}{1pt}}
\put(808,686){\rule{1pt}{1pt}}
\put(815,687){\rule{1pt}{1pt}}
\put(822,687){\rule{1pt}{1pt}}
\put(829,689){\rule{1pt}{1pt}}
\put(836,691){\rule{1pt}{1pt}}
\put(843,691){\rule{1pt}{1pt}}
\put(850,693){\rule{1pt}{1pt}}
\put(857,694){\rule{1pt}{1pt}}
\put(864,694){\rule{1pt}{1pt}}
\put(870,696){\rule{1pt}{1pt}}
\put(877,696){\rule{1pt}{1pt}}
\put(884,696){\rule{1pt}{1pt}}
\put(891,698){\rule{1pt}{1pt}}
\put(898,698){\rule{1pt}{1pt}}
\put(905,701){\rule{1pt}{1pt}}
\put(912,701){\rule{1pt}{1pt}}
\put(919,703){\rule{1pt}{1pt}}
\put(926,703){\rule{1pt}{1pt}}
\put(933,703){\rule{1pt}{1pt}}
\put(940,705){\rule{1pt}{1pt}}
\put(947,705){\rule{1pt}{1pt}}
\put(954,706){\rule{1pt}{1pt}}
\put(961,706){\rule{1pt}{1pt}}
\put(968,708){\rule{1pt}{1pt}}
\put(974,708){\rule{1pt}{1pt}}
\put(981,710){\rule{1pt}{1pt}}
\put(988,710){\rule{1pt}{1pt}}
\put(995,711){\rule{1pt}{1pt}}
\put(1002,713){\rule{1pt}{1pt}}
\put(1009,713){\rule{1pt}{1pt}}
\put(1016,715){\rule{1pt}{1pt}}
\put(1023,715){\rule{1pt}{1pt}}
\put(1030,715){\rule{1pt}{1pt}}
\put(1037,717){\rule{1pt}{1pt}}
\put(1044,715){\rule{1pt}{1pt}}
\put(1051,718){\rule{1pt}{1pt}}
\put(1058,717){\rule{1pt}{1pt}}
\put(1065,718){\rule{1pt}{1pt}}
\put(1072,720){\rule{1pt}{1pt}}
\put(1078,718){\rule{1pt}{1pt}}
\put(1085,720){\rule{1pt}{1pt}}
\put(1092,722){\rule{1pt}{1pt}}
\put(1099,722){\rule{1pt}{1pt}}
\put(1106,722){\rule{1pt}{1pt}}
\put(1113,723){\rule{1pt}{1pt}}
\put(1120,722){\rule{1pt}{1pt}}
\put(1127,723){\rule{1pt}{1pt}}
\put(1134,723){\rule{1pt}{1pt}}
\put(1141,723){\rule{1pt}{1pt}}
\put(1148,725){\rule{1pt}{1pt}}
\put(1155,723){\rule{1pt}{1pt}}
\put(1162,725){\rule{1pt}{1pt}}
\put(1169,725){\rule{1pt}{1pt}}
\put(1176,725){\rule{1pt}{1pt}}
\put(1182,727){\rule{1pt}{1pt}}
\put(1189,725){\rule{1pt}{1pt}}
\put(1196,725){\rule{1pt}{1pt}}
\put(1203,727){\rule{1pt}{1pt}}
\put(1210,728){\rule{1pt}{1pt}}
\put(1217,728){\rule{1pt}{1pt}}
\put(1224,727){\rule{1pt}{1pt}}
\put(1231,727){\rule{1pt}{1pt}}
\put(1238,728){\rule{1pt}{1pt}}
\put(1245,727){\rule{1pt}{1pt}}
\put(1252,727){\rule{1pt}{1pt}}
\put(1259,728){\rule{1pt}{1pt}}
\put(1266,730){\rule{1pt}{1pt}}
\put(1273,730){\rule{1pt}{1pt}}
\put(1280,732){\rule{1pt}{1pt}}
\put(1286,730){\rule{1pt}{1pt}}
\put(1293,732){\rule{1pt}{1pt}}
\put(1300,734){\rule{1pt}{1pt}}
\put(1307,732){\rule{1pt}{1pt}}
\put(1314,734){\rule{1pt}{1pt}}
\put(1321,734){\rule{1pt}{1pt}}
\put(1328,734){\rule{1pt}{1pt}}
\put(1335,735){\rule{1pt}{1pt}}
\put(1342,734){\rule{1pt}{1pt}}
\put(1349,735){\rule{1pt}{1pt}}
\put(1356,735){\rule{1pt}{1pt}}
\put(1363,735){\rule{1pt}{1pt}}
\put(191,188){\usebox{\plotpoint}}
\multiput(316.00,188.60)(0.920,0.468){5}{\rule{0.800pt}{0.113pt}}
\multiput(316.00,187.17)(5.340,4.000){2}{\rule{0.400pt}{0.400pt}}
\multiput(323.59,192.00)(0.485,2.705){11}{\rule{0.117pt}{2.157pt}}
\multiput(322.17,192.00)(7.000,31.523){2}{\rule{0.400pt}{1.079pt}}
\multiput(330.59,228.00)(0.485,0.950){11}{\rule{0.117pt}{0.843pt}}
\multiput(329.17,228.00)(7.000,11.251){2}{\rule{0.400pt}{0.421pt}}
\multiput(337.59,241.00)(0.485,1.026){11}{\rule{0.117pt}{0.900pt}}
\multiput(336.17,241.00)(7.000,12.132){2}{\rule{0.400pt}{0.450pt}}
\multiput(344.59,255.00)(0.482,1.304){9}{\rule{0.116pt}{1.100pt}}
\multiput(343.17,255.00)(6.000,12.717){2}{\rule{0.400pt}{0.550pt}}
\multiput(350.59,270.00)(0.485,1.179){11}{\rule{0.117pt}{1.014pt}}
\multiput(349.17,270.00)(7.000,13.895){2}{\rule{0.400pt}{0.507pt}}
\multiput(357.59,286.00)(0.485,0.721){11}{\rule{0.117pt}{0.671pt}}
\multiput(356.17,286.00)(7.000,8.606){2}{\rule{0.400pt}{0.336pt}}
\multiput(364.59,296.00)(0.485,1.026){11}{\rule{0.117pt}{0.900pt}}
\multiput(363.17,296.00)(7.000,12.132){2}{\rule{0.400pt}{0.450pt}}
\multiput(371.59,310.00)(0.485,0.950){11}{\rule{0.117pt}{0.843pt}}
\multiput(370.17,310.00)(7.000,11.251){2}{\rule{0.400pt}{0.421pt}}
\multiput(378.59,323.00)(0.485,0.645){11}{\rule{0.117pt}{0.614pt}}
\multiput(377.17,323.00)(7.000,7.725){2}{\rule{0.400pt}{0.307pt}}
\multiput(385.59,332.00)(0.485,1.026){11}{\rule{0.117pt}{0.900pt}}
\multiput(384.17,332.00)(7.000,12.132){2}{\rule{0.400pt}{0.450pt}}
\multiput(392.59,346.00)(0.485,0.874){11}{\rule{0.117pt}{0.786pt}}
\multiput(391.17,346.00)(7.000,10.369){2}{\rule{0.400pt}{0.393pt}}
\multiput(399.59,358.00)(0.485,0.950){11}{\rule{0.117pt}{0.843pt}}
\multiput(398.17,358.00)(7.000,11.251){2}{\rule{0.400pt}{0.421pt}}
\multiput(406.59,371.00)(0.485,1.026){11}{\rule{0.117pt}{0.900pt}}
\multiput(405.17,371.00)(7.000,12.132){2}{\rule{0.400pt}{0.450pt}}
\multiput(413.59,385.00)(0.485,1.026){11}{\rule{0.117pt}{0.900pt}}
\multiput(412.17,385.00)(7.000,12.132){2}{\rule{0.400pt}{0.450pt}}
\multiput(420.59,399.00)(0.485,0.950){11}{\rule{0.117pt}{0.843pt}}
\multiput(419.17,399.00)(7.000,11.251){2}{\rule{0.400pt}{0.421pt}}
\multiput(427.59,412.00)(0.485,1.026){11}{\rule{0.117pt}{0.900pt}}
\multiput(426.17,412.00)(7.000,12.132){2}{\rule{0.400pt}{0.450pt}}
\multiput(434.59,426.00)(0.485,1.026){11}{\rule{0.117pt}{0.900pt}}
\multiput(433.17,426.00)(7.000,12.132){2}{\rule{0.400pt}{0.450pt}}
\multiput(441.59,440.00)(0.485,0.874){11}{\rule{0.117pt}{0.786pt}}
\multiput(440.17,440.00)(7.000,10.369){2}{\rule{0.400pt}{0.393pt}}
\multiput(448.59,452.00)(0.482,1.123){9}{\rule{0.116pt}{0.967pt}}
\multiput(447.17,452.00)(6.000,10.994){2}{\rule{0.400pt}{0.483pt}}
\multiput(454.59,465.00)(0.485,0.874){11}{\rule{0.117pt}{0.786pt}}
\multiput(453.17,465.00)(7.000,10.369){2}{\rule{0.400pt}{0.393pt}}
\multiput(461.59,477.00)(0.485,0.721){11}{\rule{0.117pt}{0.671pt}}
\multiput(460.17,477.00)(7.000,8.606){2}{\rule{0.400pt}{0.336pt}}
\multiput(468.59,487.00)(0.485,0.874){11}{\rule{0.117pt}{0.786pt}}
\multiput(467.17,487.00)(7.000,10.369){2}{\rule{0.400pt}{0.393pt}}
\multiput(475.59,499.00)(0.485,0.798){11}{\rule{0.117pt}{0.729pt}}
\multiput(474.17,499.00)(7.000,9.488){2}{\rule{0.400pt}{0.364pt}}
\multiput(482.59,510.00)(0.485,0.721){11}{\rule{0.117pt}{0.671pt}}
\multiput(481.17,510.00)(7.000,8.606){2}{\rule{0.400pt}{0.336pt}}
\multiput(489.59,520.00)(0.485,0.569){11}{\rule{0.117pt}{0.557pt}}
\multiput(488.17,520.00)(7.000,6.844){2}{\rule{0.400pt}{0.279pt}}
\multiput(496.59,528.00)(0.485,0.798){11}{\rule{0.117pt}{0.729pt}}
\multiput(495.17,528.00)(7.000,9.488){2}{\rule{0.400pt}{0.364pt}}
\multiput(503.59,539.00)(0.485,0.569){11}{\rule{0.117pt}{0.557pt}}
\multiput(502.17,539.00)(7.000,6.844){2}{\rule{0.400pt}{0.279pt}}
\multiput(510.00,547.59)(0.492,0.485){11}{\rule{0.500pt}{0.117pt}}
\multiput(510.00,546.17)(5.962,7.000){2}{\rule{0.250pt}{0.400pt}}
\multiput(517.59,554.00)(0.485,0.645){11}{\rule{0.117pt}{0.614pt}}
\multiput(516.17,554.00)(7.000,7.725){2}{\rule{0.400pt}{0.307pt}}
\multiput(524.00,563.59)(0.710,0.477){7}{\rule{0.660pt}{0.115pt}}
\multiput(524.00,562.17)(5.630,5.000){2}{\rule{0.330pt}{0.400pt}}
\multiput(531.00,568.59)(0.492,0.485){11}{\rule{0.500pt}{0.117pt}}
\multiput(531.00,567.17)(5.962,7.000){2}{\rule{0.250pt}{0.400pt}}
\multiput(538.00,575.59)(0.492,0.485){11}{\rule{0.500pt}{0.117pt}}
\multiput(538.00,574.17)(5.962,7.000){2}{\rule{0.250pt}{0.400pt}}
\multiput(545.00,582.59)(0.710,0.477){7}{\rule{0.660pt}{0.115pt}}
\multiput(545.00,581.17)(5.630,5.000){2}{\rule{0.330pt}{0.400pt}}
\multiput(552.00,587.59)(0.599,0.477){7}{\rule{0.580pt}{0.115pt}}
\multiput(552.00,586.17)(4.796,5.000){2}{\rule{0.290pt}{0.400pt}}
\multiput(558.00,592.59)(0.710,0.477){7}{\rule{0.660pt}{0.115pt}}
\multiput(558.00,591.17)(5.630,5.000){2}{\rule{0.330pt}{0.400pt}}
\multiput(565.00,597.61)(1.355,0.447){3}{\rule{1.033pt}{0.108pt}}
\multiput(565.00,596.17)(4.855,3.000){2}{\rule{0.517pt}{0.400pt}}
\multiput(572.00,600.59)(0.710,0.477){7}{\rule{0.660pt}{0.115pt}}
\multiput(572.00,599.17)(5.630,5.000){2}{\rule{0.330pt}{0.400pt}}
\multiput(579.00,605.60)(0.920,0.468){5}{\rule{0.800pt}{0.113pt}}
\multiput(579.00,604.17)(5.340,4.000){2}{\rule{0.400pt}{0.400pt}}
\put(586,609.17){\rule{1.500pt}{0.400pt}}
\multiput(586.00,608.17)(3.887,2.000){2}{\rule{0.750pt}{0.400pt}}
\multiput(593.00,611.59)(0.710,0.477){7}{\rule{0.660pt}{0.115pt}}
\multiput(593.00,610.17)(5.630,5.000){2}{\rule{0.330pt}{0.400pt}}
\multiput(600.00,616.61)(1.355,0.447){3}{\rule{1.033pt}{0.108pt}}
\multiput(600.00,615.17)(4.855,3.000){2}{\rule{0.517pt}{0.400pt}}
\multiput(607.00,619.59)(0.710,0.477){7}{\rule{0.660pt}{0.115pt}}
\multiput(607.00,618.17)(5.630,5.000){2}{\rule{0.330pt}{0.400pt}}
\multiput(614.00,624.59)(0.492,0.485){11}{\rule{0.500pt}{0.117pt}}
\multiput(614.00,623.17)(5.962,7.000){2}{\rule{0.250pt}{0.400pt}}
\multiput(621.00,631.61)(1.355,0.447){3}{\rule{1.033pt}{0.108pt}}
\multiput(621.00,630.17)(4.855,3.000){2}{\rule{0.517pt}{0.400pt}}
\put(628,634.17){\rule{1.500pt}{0.400pt}}
\multiput(628.00,633.17)(3.887,2.000){2}{\rule{0.750pt}{0.400pt}}
\multiput(635.00,636.60)(0.920,0.468){5}{\rule{0.800pt}{0.113pt}}
\multiput(635.00,635.17)(5.340,4.000){2}{\rule{0.400pt}{0.400pt}}
\multiput(642.00,640.61)(1.355,0.447){3}{\rule{1.033pt}{0.108pt}}
\multiput(642.00,639.17)(4.855,3.000){2}{\rule{0.517pt}{0.400pt}}
\put(649,643.17){\rule{1.500pt}{0.400pt}}
\multiput(649.00,642.17)(3.887,2.000){2}{\rule{0.750pt}{0.400pt}}
\put(656,644.67){\rule{1.445pt}{0.400pt}}
\multiput(656.00,644.17)(3.000,1.000){2}{\rule{0.723pt}{0.400pt}}
\put(662,646.17){\rule{1.500pt}{0.400pt}}
\multiput(662.00,645.17)(3.887,2.000){2}{\rule{0.750pt}{0.400pt}}
\put(669,648.17){\rule{1.500pt}{0.400pt}}
\multiput(669.00,647.17)(3.887,2.000){2}{\rule{0.750pt}{0.400pt}}
\put(676,650.17){\rule{1.500pt}{0.400pt}}
\multiput(676.00,649.17)(3.887,2.000){2}{\rule{0.750pt}{0.400pt}}
\put(683,651.67){\rule{1.686pt}{0.400pt}}
\multiput(683.00,651.17)(3.500,1.000){2}{\rule{0.843pt}{0.400pt}}
\put(690,653.17){\rule{1.500pt}{0.400pt}}
\multiput(690.00,652.17)(3.887,2.000){2}{\rule{0.750pt}{0.400pt}}
\put(697,655.17){\rule{1.500pt}{0.400pt}}
\multiput(697.00,654.17)(3.887,2.000){2}{\rule{0.750pt}{0.400pt}}
\put(704,656.67){\rule{1.686pt}{0.400pt}}
\multiput(704.00,656.17)(3.500,1.000){2}{\rule{0.843pt}{0.400pt}}
\put(711,658.17){\rule{1.500pt}{0.400pt}}
\multiput(711.00,657.17)(3.887,2.000){2}{\rule{0.750pt}{0.400pt}}
\put(718,660.17){\rule{1.500pt}{0.400pt}}
\multiput(718.00,659.17)(3.887,2.000){2}{\rule{0.750pt}{0.400pt}}
\multiput(725.00,662.61)(1.355,0.447){3}{\rule{1.033pt}{0.108pt}}
\multiput(725.00,661.17)(4.855,3.000){2}{\rule{0.517pt}{0.400pt}}
\multiput(732.00,665.60)(0.920,0.468){5}{\rule{0.800pt}{0.113pt}}
\multiput(732.00,664.17)(5.340,4.000){2}{\rule{0.400pt}{0.400pt}}
\put(191.0,188.0){\rule[-0.200pt]{30.112pt}{0.400pt}}
\put(746,668.67){\rule{1.686pt}{0.400pt}}
\multiput(746.00,668.17)(3.500,1.000){2}{\rule{0.843pt}{0.400pt}}
\put(753,670.17){\rule{1.500pt}{0.400pt}}
\multiput(753.00,669.17)(3.887,2.000){2}{\rule{0.750pt}{0.400pt}}
\multiput(760.00,672.60)(0.774,0.468){5}{\rule{0.700pt}{0.113pt}}
\multiput(760.00,671.17)(4.547,4.000){2}{\rule{0.350pt}{0.400pt}}
\put(766,675.67){\rule{1.686pt}{0.400pt}}
\multiput(766.00,675.17)(3.500,1.000){2}{\rule{0.843pt}{0.400pt}}
\put(773,677.17){\rule{1.500pt}{0.400pt}}
\multiput(773.00,676.17)(3.887,2.000){2}{\rule{0.750pt}{0.400pt}}
\put(739.0,669.0){\rule[-0.200pt]{1.686pt}{0.400pt}}
\multiput(787.00,679.61)(1.355,0.447){3}{\rule{1.033pt}{0.108pt}}
\multiput(787.00,678.17)(4.855,3.000){2}{\rule{0.517pt}{0.400pt}}
\put(794,682.17){\rule{1.500pt}{0.400pt}}
\multiput(794.00,681.17)(3.887,2.000){2}{\rule{0.750pt}{0.400pt}}
\put(801,684.17){\rule{1.500pt}{0.400pt}}
\multiput(801.00,683.17)(3.887,2.000){2}{\rule{0.750pt}{0.400pt}}
\put(808,685.67){\rule{1.686pt}{0.400pt}}
\multiput(808.00,685.17)(3.500,1.000){2}{\rule{0.843pt}{0.400pt}}
\put(780.0,679.0){\rule[-0.200pt]{1.686pt}{0.400pt}}
\put(822,687.17){\rule{1.500pt}{0.400pt}}
\multiput(822.00,686.17)(3.887,2.000){2}{\rule{0.750pt}{0.400pt}}
\put(829,689.17){\rule{1.500pt}{0.400pt}}
\multiput(829.00,688.17)(3.887,2.000){2}{\rule{0.750pt}{0.400pt}}
\put(815.0,687.0){\rule[-0.200pt]{1.686pt}{0.400pt}}
\put(843,691.17){\rule{1.500pt}{0.400pt}}
\multiput(843.00,690.17)(3.887,2.000){2}{\rule{0.750pt}{0.400pt}}
\put(850,692.67){\rule{1.686pt}{0.400pt}}
\multiput(850.00,692.17)(3.500,1.000){2}{\rule{0.843pt}{0.400pt}}
\put(836.0,691.0){\rule[-0.200pt]{1.686pt}{0.400pt}}
\put(864,694.17){\rule{1.300pt}{0.400pt}}
\multiput(864.00,693.17)(3.302,2.000){2}{\rule{0.650pt}{0.400pt}}
\put(857.0,694.0){\rule[-0.200pt]{1.686pt}{0.400pt}}
\put(884,696.17){\rule{1.500pt}{0.400pt}}
\multiput(884.00,695.17)(3.887,2.000){2}{\rule{0.750pt}{0.400pt}}
\put(870.0,696.0){\rule[-0.200pt]{3.373pt}{0.400pt}}
\multiput(898.00,698.61)(1.355,0.447){3}{\rule{1.033pt}{0.108pt}}
\multiput(898.00,697.17)(4.855,3.000){2}{\rule{0.517pt}{0.400pt}}
\put(891.0,698.0){\rule[-0.200pt]{1.686pt}{0.400pt}}
\put(912,701.17){\rule{1.500pt}{0.400pt}}
\multiput(912.00,700.17)(3.887,2.000){2}{\rule{0.750pt}{0.400pt}}
\put(905.0,701.0){\rule[-0.200pt]{1.686pt}{0.400pt}}
\put(933,703.17){\rule{1.500pt}{0.400pt}}
\multiput(933.00,702.17)(3.887,2.000){2}{\rule{0.750pt}{0.400pt}}
\put(919.0,703.0){\rule[-0.200pt]{3.373pt}{0.400pt}}
\put(947,704.67){\rule{1.686pt}{0.400pt}}
\multiput(947.00,704.17)(3.500,1.000){2}{\rule{0.843pt}{0.400pt}}
\put(940.0,705.0){\rule[-0.200pt]{1.686pt}{0.400pt}}
\put(961,706.17){\rule{1.500pt}{0.400pt}}
\multiput(961.00,705.17)(3.887,2.000){2}{\rule{0.750pt}{0.400pt}}
\put(954.0,706.0){\rule[-0.200pt]{1.686pt}{0.400pt}}
\put(974,708.17){\rule{1.500pt}{0.400pt}}
\multiput(974.00,707.17)(3.887,2.000){2}{\rule{0.750pt}{0.400pt}}
\put(968.0,708.0){\rule[-0.200pt]{1.445pt}{0.400pt}}
\put(988,709.67){\rule{1.686pt}{0.400pt}}
\multiput(988.00,709.17)(3.500,1.000){2}{\rule{0.843pt}{0.400pt}}
\put(995,711.17){\rule{1.500pt}{0.400pt}}
\multiput(995.00,710.17)(3.887,2.000){2}{\rule{0.750pt}{0.400pt}}
\put(981.0,710.0){\rule[-0.200pt]{1.686pt}{0.400pt}}
\put(1009,713.17){\rule{1.500pt}{0.400pt}}
\multiput(1009.00,712.17)(3.887,2.000){2}{\rule{0.750pt}{0.400pt}}
\put(1002.0,713.0){\rule[-0.200pt]{1.686pt}{0.400pt}}
\put(1030,715.17){\rule{1.500pt}{0.400pt}}
\multiput(1030.00,714.17)(3.887,2.000){2}{\rule{0.750pt}{0.400pt}}
\put(1037,715.17){\rule{1.500pt}{0.400pt}}
\multiput(1037.00,716.17)(3.887,-2.000){2}{\rule{0.750pt}{0.400pt}}
\multiput(1044.00,715.61)(1.355,0.447){3}{\rule{1.033pt}{0.108pt}}
\multiput(1044.00,714.17)(4.855,3.000){2}{\rule{0.517pt}{0.400pt}}
\put(1051,716.67){\rule{1.686pt}{0.400pt}}
\multiput(1051.00,717.17)(3.500,-1.000){2}{\rule{0.843pt}{0.400pt}}
\put(1058,716.67){\rule{1.686pt}{0.400pt}}
\multiput(1058.00,716.17)(3.500,1.000){2}{\rule{0.843pt}{0.400pt}}
\put(1065,718.17){\rule{1.500pt}{0.400pt}}
\multiput(1065.00,717.17)(3.887,2.000){2}{\rule{0.750pt}{0.400pt}}
\put(1072,718.17){\rule{1.300pt}{0.400pt}}
\multiput(1072.00,719.17)(3.302,-2.000){2}{\rule{0.650pt}{0.400pt}}
\put(1078,718.17){\rule{1.500pt}{0.400pt}}
\multiput(1078.00,717.17)(3.887,2.000){2}{\rule{0.750pt}{0.400pt}}
\put(1085,720.17){\rule{1.500pt}{0.400pt}}
\multiput(1085.00,719.17)(3.887,2.000){2}{\rule{0.750pt}{0.400pt}}
\put(1016.0,715.0){\rule[-0.200pt]{3.373pt}{0.400pt}}
\put(1106,721.67){\rule{1.686pt}{0.400pt}}
\multiput(1106.00,721.17)(3.500,1.000){2}{\rule{0.843pt}{0.400pt}}
\put(1113,721.67){\rule{1.686pt}{0.400pt}}
\multiput(1113.00,722.17)(3.500,-1.000){2}{\rule{0.843pt}{0.400pt}}
\put(1120,721.67){\rule{1.686pt}{0.400pt}}
\multiput(1120.00,721.17)(3.500,1.000){2}{\rule{0.843pt}{0.400pt}}
\put(1092.0,722.0){\rule[-0.200pt]{3.373pt}{0.400pt}}
\put(1141,723.17){\rule{1.500pt}{0.400pt}}
\multiput(1141.00,722.17)(3.887,2.000){2}{\rule{0.750pt}{0.400pt}}
\put(1148,723.17){\rule{1.500pt}{0.400pt}}
\multiput(1148.00,724.17)(3.887,-2.000){2}{\rule{0.750pt}{0.400pt}}
\put(1155,723.17){\rule{1.500pt}{0.400pt}}
\multiput(1155.00,722.17)(3.887,2.000){2}{\rule{0.750pt}{0.400pt}}
\put(1127.0,723.0){\rule[-0.200pt]{3.373pt}{0.400pt}}
\put(1176,725.17){\rule{1.300pt}{0.400pt}}
\multiput(1176.00,724.17)(3.302,2.000){2}{\rule{0.650pt}{0.400pt}}
\put(1182,725.17){\rule{1.500pt}{0.400pt}}
\multiput(1182.00,726.17)(3.887,-2.000){2}{\rule{0.750pt}{0.400pt}}
\put(1162.0,725.0){\rule[-0.200pt]{3.373pt}{0.400pt}}
\put(1196,725.17){\rule{1.500pt}{0.400pt}}
\multiput(1196.00,724.17)(3.887,2.000){2}{\rule{0.750pt}{0.400pt}}
\put(1203,726.67){\rule{1.686pt}{0.400pt}}
\multiput(1203.00,726.17)(3.500,1.000){2}{\rule{0.843pt}{0.400pt}}
\put(1189.0,725.0){\rule[-0.200pt]{1.686pt}{0.400pt}}
\put(1217,726.67){\rule{1.686pt}{0.400pt}}
\multiput(1217.00,727.17)(3.500,-1.000){2}{\rule{0.843pt}{0.400pt}}
\put(1210.0,728.0){\rule[-0.200pt]{1.686pt}{0.400pt}}
\put(1231,726.67){\rule{1.686pt}{0.400pt}}
\multiput(1231.00,726.17)(3.500,1.000){2}{\rule{0.843pt}{0.400pt}}
\put(1238,726.67){\rule{1.686pt}{0.400pt}}
\multiput(1238.00,727.17)(3.500,-1.000){2}{\rule{0.843pt}{0.400pt}}
\put(1224.0,727.0){\rule[-0.200pt]{1.686pt}{0.400pt}}
\put(1252,726.67){\rule{1.686pt}{0.400pt}}
\multiput(1252.00,726.17)(3.500,1.000){2}{\rule{0.843pt}{0.400pt}}
\put(1259,728.17){\rule{1.500pt}{0.400pt}}
\multiput(1259.00,727.17)(3.887,2.000){2}{\rule{0.750pt}{0.400pt}}
\put(1245.0,727.0){\rule[-0.200pt]{1.686pt}{0.400pt}}
\put(1273,730.17){\rule{1.500pt}{0.400pt}}
\multiput(1273.00,729.17)(3.887,2.000){2}{\rule{0.750pt}{0.400pt}}
\put(1280,730.17){\rule{1.300pt}{0.400pt}}
\multiput(1280.00,731.17)(3.302,-2.000){2}{\rule{0.650pt}{0.400pt}}
\put(1286,730.17){\rule{1.500pt}{0.400pt}}
\multiput(1286.00,729.17)(3.887,2.000){2}{\rule{0.750pt}{0.400pt}}
\put(1293,732.17){\rule{1.500pt}{0.400pt}}
\multiput(1293.00,731.17)(3.887,2.000){2}{\rule{0.750pt}{0.400pt}}
\put(1300,732.17){\rule{1.500pt}{0.400pt}}
\multiput(1300.00,733.17)(3.887,-2.000){2}{\rule{0.750pt}{0.400pt}}
\put(1307,732.17){\rule{1.500pt}{0.400pt}}
\multiput(1307.00,731.17)(3.887,2.000){2}{\rule{0.750pt}{0.400pt}}
\put(1266.0,730.0){\rule[-0.200pt]{1.686pt}{0.400pt}}
\put(1328,733.67){\rule{1.686pt}{0.400pt}}
\multiput(1328.00,733.17)(3.500,1.000){2}{\rule{0.843pt}{0.400pt}}
\put(1335,733.67){\rule{1.686pt}{0.400pt}}
\multiput(1335.00,734.17)(3.500,-1.000){2}{\rule{0.843pt}{0.400pt}}
\put(1342,733.67){\rule{1.686pt}{0.400pt}}
\multiput(1342.00,733.17)(3.500,1.000){2}{\rule{0.843pt}{0.400pt}}
\put(1314.0,734.0){\rule[-0.200pt]{3.373pt}{0.400pt}}
\put(1349.0,735.0){\rule[-0.200pt]{3.373pt}{0.400pt}}
\put(191.0,131.0){\rule[-0.200pt]{0.400pt}{155.380pt}}
\put(191.0,131.0){\rule[-0.200pt]{300.643pt}{0.400pt}}
\put(1439.0,131.0){\rule[-0.200pt]{0.400pt}{155.380pt}}
\put(191.0,776.0){\rule[-0.200pt]{300.643pt}{0.400pt}}
\end{picture}
}
  \caption{Pressure trial 4, temperature = 8.5 \textdegree C }
\label{fig:1m4c}
\end {minipage}
\end {figure}
\FloatBarrier

Below is a table containing the initial rates ($\frac{\Delta P}{\Delta t}$) taken as the average slope of the first $\frac{1}{2}$ second of the pressure rise. Underneath the table is the calculated rate equation and activation energy (E$_a$) for the pressure trials.

\FloatBarrier
\begin{figure}[h!]
	\begin{center}
	\renewcommand\arraystretch{1.5}
	\renewcommand\tabcolsep{12pt}
		\begin{tabular}{|c|c|c|c|c|}
			\hline 
			Trial & Mg Mass & HCl M & Initial Rate $\frac{\Delta P}{\Delta t}$ & Temperature (\textdegree C)\\
			\hline 
			1 & 0.04g & 1M & 0.0502 & 22.4 \\
	 		\hline
			2 & 0.02g & 1M & 0.0139 & 22.4\\
	 		\hline
			3 & 0.04g & 0.5M & 0.0246 & 22.4\\
	 		\hline	
			4 & 0.04g & 1M & 0.0489 & 8.5\\
	 		\hline		 				 			 		
		\end{tabular}
	\end{center}
	\label{fig:prestab}
	\caption{Results of trials including initial rates}
\end{figure}
\FloatBarrier
\begin{equation}
\textnormal{rate}  = 1.378 (\textnormal{grams Mg})^{1.029} [\textnormal{HCl}]^{1.853}
\label{eq:presrate}
\end{equation}
\begin{equation}
\textnormal{E}_a = 18.9\textnormal{\space kJ/mol}
\label{Ea}
\end{equation}
It should be noted that the rate law using pressure data is far more reasonable, with the rate orders being on par with expectations.
%\begin{figure}[h!]
%  \begin{center}
%    	\resizebox{0.6\textwidth}{!}{% GNUPLOT: LaTeX picture
\setlength{\unitlength}{0.240900pt}
\ifx\plotpoint\undefined\newsavebox{\plotpoint}\fi
\sbox{\plotpoint}{\rule[-0.200pt]{0.400pt}{0.400pt}}%
\begin{picture}(1500,900)(0,0)
\sbox{\plotpoint}{\rule[-0.200pt]{0.400pt}{0.400pt}}%
\put(131.0,131.0){\rule[-0.200pt]{4.818pt}{0.400pt}}
\put(111,131){\makebox(0,0)[r]{ 0}}
\put(1419.0,131.0){\rule[-0.200pt]{4.818pt}{0.400pt}}
\put(131.0,239.0){\rule[-0.200pt]{4.818pt}{0.400pt}}
\put(111,239){\makebox(0,0)[r]{ 1}}
\put(1419.0,239.0){\rule[-0.200pt]{4.818pt}{0.400pt}}
\put(131.0,346.0){\rule[-0.200pt]{4.818pt}{0.400pt}}
\put(111,346){\makebox(0,0)[r]{ 2}}
\put(1419.0,346.0){\rule[-0.200pt]{4.818pt}{0.400pt}}
\put(131.0,454.0){\rule[-0.200pt]{4.818pt}{0.400pt}}
\put(111,454){\makebox(0,0)[r]{ 3}}
\put(1419.0,454.0){\rule[-0.200pt]{4.818pt}{0.400pt}}
\put(131.0,561.0){\rule[-0.200pt]{4.818pt}{0.400pt}}
\put(111,561){\makebox(0,0)[r]{ 4}}
\put(1419.0,561.0){\rule[-0.200pt]{4.818pt}{0.400pt}}
\put(131.0,669.0){\rule[-0.200pt]{4.818pt}{0.400pt}}
\put(111,669){\makebox(0,0)[r]{ 5}}
\put(1419.0,669.0){\rule[-0.200pt]{4.818pt}{0.400pt}}
\put(131.0,776.0){\rule[-0.200pt]{4.818pt}{0.400pt}}
\put(111,776){\makebox(0,0)[r]{ 6}}
\put(1419.0,776.0){\rule[-0.200pt]{4.818pt}{0.400pt}}
\put(410.0,131.0){\rule[-0.200pt]{0.400pt}{4.818pt}}
\put(410,90){\makebox(0,0){ 0.427}}
\put(410.0,756.0){\rule[-0.200pt]{0.400pt}{4.818pt}}
\put(684.0,131.0){\rule[-0.200pt]{0.400pt}{4.818pt}}
\put(684,90){\makebox(0,0){ 0.845}}
\put(684.0,756.0){\rule[-0.200pt]{0.400pt}{4.818pt}}
\put(968.0,131.0){\rule[-0.200pt]{0.400pt}{4.818pt}}
\put(968,90){\makebox(0,0){ 1.28}}
\put(968.0,756.0){\rule[-0.200pt]{0.400pt}{4.818pt}}
\put(1249.0,131.0){\rule[-0.200pt]{0.400pt}{4.818pt}}
\put(1249,90){\makebox(0,0){ 1.71}}
\put(1249.0,756.0){\rule[-0.200pt]{0.400pt}{4.818pt}}
\put(131.0,131.0){\rule[-0.200pt]{0.400pt}{155.380pt}}
\put(131.0,131.0){\rule[-0.200pt]{315.097pt}{0.400pt}}
\put(1439.0,131.0){\rule[-0.200pt]{0.400pt}{155.380pt}}
\put(131.0,776.0){\rule[-0.200pt]{315.097pt}{0.400pt}}
\put(30,453){\makebox(0,0){\hspace{-100pt} Freezing  Point}}
\put(30,400){\makebox(0,0){\hspace{-100pt}Depression $\Delta T_f$ (\textdegree C)}}
\put(785,20){\makebox(0,0){Molality ($\frac{mol\hspace{6pt} solute}{kg\hspace{6pt} solvent}$)}}
\put(785,838){\makebox(0,0){Sodium Chloride Freezing Point Depression vs. Molality}}
\put(1250,690){\rule{1pt}{1pt}\makebox(0,0){$+$}}
\put(970,572){\rule{1pt}{1pt}\makebox(0,0){$+$}}
\put(691,368){\rule{1pt}{1pt}\makebox(0,0){$+$}}
\put(411,217){\rule{1pt}{1pt}\makebox(0,0){$+$}}
\put(131,131){\usebox{\plotpoint}}
\multiput(131.00,131.59)(1.123,0.482){9}{\rule{0.967pt}{0.116pt}}
\multiput(131.00,130.17)(10.994,6.000){2}{\rule{0.483pt}{0.400pt}}
\multiput(144.00,137.59)(0.950,0.485){11}{\rule{0.843pt}{0.117pt}}
\multiput(144.00,136.17)(11.251,7.000){2}{\rule{0.421pt}{0.400pt}}
\multiput(157.00,144.59)(1.214,0.482){9}{\rule{1.033pt}{0.116pt}}
\multiput(157.00,143.17)(11.855,6.000){2}{\rule{0.517pt}{0.400pt}}
\multiput(171.00,150.59)(0.950,0.485){11}{\rule{0.843pt}{0.117pt}}
\multiput(171.00,149.17)(11.251,7.000){2}{\rule{0.421pt}{0.400pt}}
\multiput(184.00,157.59)(1.123,0.482){9}{\rule{0.967pt}{0.116pt}}
\multiput(184.00,156.17)(10.994,6.000){2}{\rule{0.483pt}{0.400pt}}
\multiput(197.00,163.59)(0.950,0.485){11}{\rule{0.843pt}{0.117pt}}
\multiput(197.00,162.17)(11.251,7.000){2}{\rule{0.421pt}{0.400pt}}
\multiput(210.00,170.59)(1.123,0.482){9}{\rule{0.967pt}{0.116pt}}
\multiput(210.00,169.17)(10.994,6.000){2}{\rule{0.483pt}{0.400pt}}
\multiput(223.00,176.59)(1.026,0.485){11}{\rule{0.900pt}{0.117pt}}
\multiput(223.00,175.17)(12.132,7.000){2}{\rule{0.450pt}{0.400pt}}
\multiput(237.00,183.59)(1.123,0.482){9}{\rule{0.967pt}{0.116pt}}
\multiput(237.00,182.17)(10.994,6.000){2}{\rule{0.483pt}{0.400pt}}
\multiput(250.00,189.59)(0.950,0.485){11}{\rule{0.843pt}{0.117pt}}
\multiput(250.00,188.17)(11.251,7.000){2}{\rule{0.421pt}{0.400pt}}
\multiput(263.00,196.59)(1.123,0.482){9}{\rule{0.967pt}{0.116pt}}
\multiput(263.00,195.17)(10.994,6.000){2}{\rule{0.483pt}{0.400pt}}
\multiput(276.00,202.59)(1.026,0.485){11}{\rule{0.900pt}{0.117pt}}
\multiput(276.00,201.17)(12.132,7.000){2}{\rule{0.450pt}{0.400pt}}
\multiput(290.00,209.59)(1.123,0.482){9}{\rule{0.967pt}{0.116pt}}
\multiput(290.00,208.17)(10.994,6.000){2}{\rule{0.483pt}{0.400pt}}
\multiput(303.00,215.59)(0.950,0.485){11}{\rule{0.843pt}{0.117pt}}
\multiput(303.00,214.17)(11.251,7.000){2}{\rule{0.421pt}{0.400pt}}
\multiput(316.00,222.59)(1.123,0.482){9}{\rule{0.967pt}{0.116pt}}
\multiput(316.00,221.17)(10.994,6.000){2}{\rule{0.483pt}{0.400pt}}
\multiput(329.00,228.59)(0.950,0.485){11}{\rule{0.843pt}{0.117pt}}
\multiput(329.00,227.17)(11.251,7.000){2}{\rule{0.421pt}{0.400pt}}
\multiput(342.00,235.59)(1.214,0.482){9}{\rule{1.033pt}{0.116pt}}
\multiput(342.00,234.17)(11.855,6.000){2}{\rule{0.517pt}{0.400pt}}
\multiput(356.00,241.59)(0.950,0.485){11}{\rule{0.843pt}{0.117pt}}
\multiput(356.00,240.17)(11.251,7.000){2}{\rule{0.421pt}{0.400pt}}
\multiput(369.00,248.59)(1.123,0.482){9}{\rule{0.967pt}{0.116pt}}
\multiput(369.00,247.17)(10.994,6.000){2}{\rule{0.483pt}{0.400pt}}
\multiput(382.00,254.59)(0.950,0.485){11}{\rule{0.843pt}{0.117pt}}
\multiput(382.00,253.17)(11.251,7.000){2}{\rule{0.421pt}{0.400pt}}
\multiput(395.00,261.59)(1.123,0.482){9}{\rule{0.967pt}{0.116pt}}
\multiput(395.00,260.17)(10.994,6.000){2}{\rule{0.483pt}{0.400pt}}
\multiput(408.00,267.59)(1.026,0.485){11}{\rule{0.900pt}{0.117pt}}
\multiput(408.00,266.17)(12.132,7.000){2}{\rule{0.450pt}{0.400pt}}
\multiput(422.00,274.59)(1.123,0.482){9}{\rule{0.967pt}{0.116pt}}
\multiput(422.00,273.17)(10.994,6.000){2}{\rule{0.483pt}{0.400pt}}
\multiput(435.00,280.59)(0.950,0.485){11}{\rule{0.843pt}{0.117pt}}
\multiput(435.00,279.17)(11.251,7.000){2}{\rule{0.421pt}{0.400pt}}
\multiput(448.00,287.59)(1.123,0.482){9}{\rule{0.967pt}{0.116pt}}
\multiput(448.00,286.17)(10.994,6.000){2}{\rule{0.483pt}{0.400pt}}
\multiput(461.00,293.59)(1.026,0.485){11}{\rule{0.900pt}{0.117pt}}
\multiput(461.00,292.17)(12.132,7.000){2}{\rule{0.450pt}{0.400pt}}
\multiput(475.00,300.59)(1.123,0.482){9}{\rule{0.967pt}{0.116pt}}
\multiput(475.00,299.17)(10.994,6.000){2}{\rule{0.483pt}{0.400pt}}
\multiput(488.00,306.59)(1.123,0.482){9}{\rule{0.967pt}{0.116pt}}
\multiput(488.00,305.17)(10.994,6.000){2}{\rule{0.483pt}{0.400pt}}
\multiput(501.00,312.59)(0.950,0.485){11}{\rule{0.843pt}{0.117pt}}
\multiput(501.00,311.17)(11.251,7.000){2}{\rule{0.421pt}{0.400pt}}
\multiput(514.00,319.59)(1.123,0.482){9}{\rule{0.967pt}{0.116pt}}
\multiput(514.00,318.17)(10.994,6.000){2}{\rule{0.483pt}{0.400pt}}
\multiput(527.00,325.59)(1.026,0.485){11}{\rule{0.900pt}{0.117pt}}
\multiput(527.00,324.17)(12.132,7.000){2}{\rule{0.450pt}{0.400pt}}
\multiput(541.00,332.59)(1.123,0.482){9}{\rule{0.967pt}{0.116pt}}
\multiput(541.00,331.17)(10.994,6.000){2}{\rule{0.483pt}{0.400pt}}
\multiput(554.00,338.59)(0.950,0.485){11}{\rule{0.843pt}{0.117pt}}
\multiput(554.00,337.17)(11.251,7.000){2}{\rule{0.421pt}{0.400pt}}
\multiput(567.00,345.59)(1.123,0.482){9}{\rule{0.967pt}{0.116pt}}
\multiput(567.00,344.17)(10.994,6.000){2}{\rule{0.483pt}{0.400pt}}
\multiput(580.00,351.59)(0.950,0.485){11}{\rule{0.843pt}{0.117pt}}
\multiput(580.00,350.17)(11.251,7.000){2}{\rule{0.421pt}{0.400pt}}
\multiput(593.00,358.59)(1.214,0.482){9}{\rule{1.033pt}{0.116pt}}
\multiput(593.00,357.17)(11.855,6.000){2}{\rule{0.517pt}{0.400pt}}
\multiput(607.00,364.59)(0.950,0.485){11}{\rule{0.843pt}{0.117pt}}
\multiput(607.00,363.17)(11.251,7.000){2}{\rule{0.421pt}{0.400pt}}
\multiput(620.00,371.59)(1.123,0.482){9}{\rule{0.967pt}{0.116pt}}
\multiput(620.00,370.17)(10.994,6.000){2}{\rule{0.483pt}{0.400pt}}
\multiput(633.00,377.59)(0.950,0.485){11}{\rule{0.843pt}{0.117pt}}
\multiput(633.00,376.17)(11.251,7.000){2}{\rule{0.421pt}{0.400pt}}
\multiput(646.00,384.59)(1.123,0.482){9}{\rule{0.967pt}{0.116pt}}
\multiput(646.00,383.17)(10.994,6.000){2}{\rule{0.483pt}{0.400pt}}
\multiput(659.00,390.59)(1.026,0.485){11}{\rule{0.900pt}{0.117pt}}
\multiput(659.00,389.17)(12.132,7.000){2}{\rule{0.450pt}{0.400pt}}
\multiput(673.00,397.59)(1.123,0.482){9}{\rule{0.967pt}{0.116pt}}
\multiput(673.00,396.17)(10.994,6.000){2}{\rule{0.483pt}{0.400pt}}
\multiput(686.00,403.59)(0.950,0.485){11}{\rule{0.843pt}{0.117pt}}
\multiput(686.00,402.17)(11.251,7.000){2}{\rule{0.421pt}{0.400pt}}
\multiput(699.00,410.59)(1.123,0.482){9}{\rule{0.967pt}{0.116pt}}
\multiput(699.00,409.17)(10.994,6.000){2}{\rule{0.483pt}{0.400pt}}
\multiput(712.00,416.59)(1.026,0.485){11}{\rule{0.900pt}{0.117pt}}
\multiput(712.00,415.17)(12.132,7.000){2}{\rule{0.450pt}{0.400pt}}
\multiput(726.00,423.59)(1.123,0.482){9}{\rule{0.967pt}{0.116pt}}
\multiput(726.00,422.17)(10.994,6.000){2}{\rule{0.483pt}{0.400pt}}
\multiput(739.00,429.59)(0.950,0.485){11}{\rule{0.843pt}{0.117pt}}
\multiput(739.00,428.17)(11.251,7.000){2}{\rule{0.421pt}{0.400pt}}
\multiput(752.00,436.59)(1.123,0.482){9}{\rule{0.967pt}{0.116pt}}
\multiput(752.00,435.17)(10.994,6.000){2}{\rule{0.483pt}{0.400pt}}
\multiput(765.00,442.59)(0.950,0.485){11}{\rule{0.843pt}{0.117pt}}
\multiput(765.00,441.17)(11.251,7.000){2}{\rule{0.421pt}{0.400pt}}
\multiput(778.00,449.59)(1.214,0.482){9}{\rule{1.033pt}{0.116pt}}
\multiput(778.00,448.17)(11.855,6.000){2}{\rule{0.517pt}{0.400pt}}
\multiput(792.00,455.59)(0.950,0.485){11}{\rule{0.843pt}{0.117pt}}
\multiput(792.00,454.17)(11.251,7.000){2}{\rule{0.421pt}{0.400pt}}
\multiput(805.00,462.59)(1.123,0.482){9}{\rule{0.967pt}{0.116pt}}
\multiput(805.00,461.17)(10.994,6.000){2}{\rule{0.483pt}{0.400pt}}
\multiput(818.00,468.59)(0.950,0.485){11}{\rule{0.843pt}{0.117pt}}
\multiput(818.00,467.17)(11.251,7.000){2}{\rule{0.421pt}{0.400pt}}
\multiput(831.00,475.59)(1.123,0.482){9}{\rule{0.967pt}{0.116pt}}
\multiput(831.00,474.17)(10.994,6.000){2}{\rule{0.483pt}{0.400pt}}
\multiput(844.00,481.59)(1.214,0.482){9}{\rule{1.033pt}{0.116pt}}
\multiput(844.00,480.17)(11.855,6.000){2}{\rule{0.517pt}{0.400pt}}
\multiput(858.00,487.59)(0.950,0.485){11}{\rule{0.843pt}{0.117pt}}
\multiput(858.00,486.17)(11.251,7.000){2}{\rule{0.421pt}{0.400pt}}
\multiput(871.00,494.59)(1.123,0.482){9}{\rule{0.967pt}{0.116pt}}
\multiput(871.00,493.17)(10.994,6.000){2}{\rule{0.483pt}{0.400pt}}
\multiput(884.00,500.59)(0.950,0.485){11}{\rule{0.843pt}{0.117pt}}
\multiput(884.00,499.17)(11.251,7.000){2}{\rule{0.421pt}{0.400pt}}
\multiput(897.00,507.59)(1.214,0.482){9}{\rule{1.033pt}{0.116pt}}
\multiput(897.00,506.17)(11.855,6.000){2}{\rule{0.517pt}{0.400pt}}
\multiput(911.00,513.59)(0.950,0.485){11}{\rule{0.843pt}{0.117pt}}
\multiput(911.00,512.17)(11.251,7.000){2}{\rule{0.421pt}{0.400pt}}
\multiput(924.00,520.59)(1.123,0.482){9}{\rule{0.967pt}{0.116pt}}
\multiput(924.00,519.17)(10.994,6.000){2}{\rule{0.483pt}{0.400pt}}
\multiput(937.00,526.59)(0.950,0.485){11}{\rule{0.843pt}{0.117pt}}
\multiput(937.00,525.17)(11.251,7.000){2}{\rule{0.421pt}{0.400pt}}
\multiput(950.00,533.59)(1.123,0.482){9}{\rule{0.967pt}{0.116pt}}
\multiput(950.00,532.17)(10.994,6.000){2}{\rule{0.483pt}{0.400pt}}
\multiput(963.00,539.59)(1.026,0.485){11}{\rule{0.900pt}{0.117pt}}
\multiput(963.00,538.17)(12.132,7.000){2}{\rule{0.450pt}{0.400pt}}
\multiput(977.00,546.59)(1.123,0.482){9}{\rule{0.967pt}{0.116pt}}
\multiput(977.00,545.17)(10.994,6.000){2}{\rule{0.483pt}{0.400pt}}
\multiput(990.00,552.59)(0.950,0.485){11}{\rule{0.843pt}{0.117pt}}
\multiput(990.00,551.17)(11.251,7.000){2}{\rule{0.421pt}{0.400pt}}
\multiput(1003.00,559.59)(1.123,0.482){9}{\rule{0.967pt}{0.116pt}}
\multiput(1003.00,558.17)(10.994,6.000){2}{\rule{0.483pt}{0.400pt}}
\multiput(1016.00,565.59)(0.950,0.485){11}{\rule{0.843pt}{0.117pt}}
\multiput(1016.00,564.17)(11.251,7.000){2}{\rule{0.421pt}{0.400pt}}
\multiput(1029.00,572.59)(1.214,0.482){9}{\rule{1.033pt}{0.116pt}}
\multiput(1029.00,571.17)(11.855,6.000){2}{\rule{0.517pt}{0.400pt}}
\multiput(1043.00,578.59)(0.950,0.485){11}{\rule{0.843pt}{0.117pt}}
\multiput(1043.00,577.17)(11.251,7.000){2}{\rule{0.421pt}{0.400pt}}
\multiput(1056.00,585.59)(1.123,0.482){9}{\rule{0.967pt}{0.116pt}}
\multiput(1056.00,584.17)(10.994,6.000){2}{\rule{0.483pt}{0.400pt}}
\multiput(1069.00,591.59)(0.950,0.485){11}{\rule{0.843pt}{0.117pt}}
\multiput(1069.00,590.17)(11.251,7.000){2}{\rule{0.421pt}{0.400pt}}
\multiput(1082.00,598.59)(1.123,0.482){9}{\rule{0.967pt}{0.116pt}}
\multiput(1082.00,597.17)(10.994,6.000){2}{\rule{0.483pt}{0.400pt}}
\multiput(1095.00,604.59)(1.026,0.485){11}{\rule{0.900pt}{0.117pt}}
\multiput(1095.00,603.17)(12.132,7.000){2}{\rule{0.450pt}{0.400pt}}
\multiput(1109.00,611.59)(1.123,0.482){9}{\rule{0.967pt}{0.116pt}}
\multiput(1109.00,610.17)(10.994,6.000){2}{\rule{0.483pt}{0.400pt}}
\multiput(1122.00,617.59)(0.950,0.485){11}{\rule{0.843pt}{0.117pt}}
\multiput(1122.00,616.17)(11.251,7.000){2}{\rule{0.421pt}{0.400pt}}
\multiput(1135.00,624.59)(1.123,0.482){9}{\rule{0.967pt}{0.116pt}}
\multiput(1135.00,623.17)(10.994,6.000){2}{\rule{0.483pt}{0.400pt}}
\multiput(1148.00,630.59)(1.026,0.485){11}{\rule{0.900pt}{0.117pt}}
\multiput(1148.00,629.17)(12.132,7.000){2}{\rule{0.450pt}{0.400pt}}
\multiput(1162.00,637.59)(1.123,0.482){9}{\rule{0.967pt}{0.116pt}}
\multiput(1162.00,636.17)(10.994,6.000){2}{\rule{0.483pt}{0.400pt}}
\multiput(1175.00,643.59)(1.123,0.482){9}{\rule{0.967pt}{0.116pt}}
\multiput(1175.00,642.17)(10.994,6.000){2}{\rule{0.483pt}{0.400pt}}
\multiput(1188.00,649.59)(0.950,0.485){11}{\rule{0.843pt}{0.117pt}}
\multiput(1188.00,648.17)(11.251,7.000){2}{\rule{0.421pt}{0.400pt}}
\multiput(1201.00,656.59)(1.123,0.482){9}{\rule{0.967pt}{0.116pt}}
\multiput(1201.00,655.17)(10.994,6.000){2}{\rule{0.483pt}{0.400pt}}
\multiput(1214.00,662.59)(1.026,0.485){11}{\rule{0.900pt}{0.117pt}}
\multiput(1214.00,661.17)(12.132,7.000){2}{\rule{0.450pt}{0.400pt}}
\multiput(1228.00,669.59)(1.123,0.482){9}{\rule{0.967pt}{0.116pt}}
\multiput(1228.00,668.17)(10.994,6.000){2}{\rule{0.483pt}{0.400pt}}
\multiput(1241.00,675.59)(0.950,0.485){11}{\rule{0.843pt}{0.117pt}}
\multiput(1241.00,674.17)(11.251,7.000){2}{\rule{0.421pt}{0.400pt}}
\multiput(1254.00,682.59)(1.123,0.482){9}{\rule{0.967pt}{0.116pt}}
\multiput(1254.00,681.17)(10.994,6.000){2}{\rule{0.483pt}{0.400pt}}
\multiput(1267.00,688.59)(0.950,0.485){11}{\rule{0.843pt}{0.117pt}}
\multiput(1267.00,687.17)(11.251,7.000){2}{\rule{0.421pt}{0.400pt}}
\multiput(1280.00,695.59)(1.214,0.482){9}{\rule{1.033pt}{0.116pt}}
\multiput(1280.00,694.17)(11.855,6.000){2}{\rule{0.517pt}{0.400pt}}
\multiput(1294.00,701.59)(0.950,0.485){11}{\rule{0.843pt}{0.117pt}}
\multiput(1294.00,700.17)(11.251,7.000){2}{\rule{0.421pt}{0.400pt}}
\multiput(1307.00,708.59)(1.123,0.482){9}{\rule{0.967pt}{0.116pt}}
\multiput(1307.00,707.17)(10.994,6.000){2}{\rule{0.483pt}{0.400pt}}
\multiput(1320.00,714.59)(0.950,0.485){11}{\rule{0.843pt}{0.117pt}}
\multiput(1320.00,713.17)(11.251,7.000){2}{\rule{0.421pt}{0.400pt}}
\multiput(1333.00,721.59)(1.214,0.482){9}{\rule{1.033pt}{0.116pt}}
\multiput(1333.00,720.17)(11.855,6.000){2}{\rule{0.517pt}{0.400pt}}
\multiput(1347.00,727.59)(0.950,0.485){11}{\rule{0.843pt}{0.117pt}}
\multiput(1347.00,726.17)(11.251,7.000){2}{\rule{0.421pt}{0.400pt}}
\multiput(1360.00,734.59)(1.123,0.482){9}{\rule{0.967pt}{0.116pt}}
\multiput(1360.00,733.17)(10.994,6.000){2}{\rule{0.483pt}{0.400pt}}
\multiput(1373.00,740.59)(0.950,0.485){11}{\rule{0.843pt}{0.117pt}}
\multiput(1373.00,739.17)(11.251,7.000){2}{\rule{0.421pt}{0.400pt}}
\multiput(1386.00,747.59)(1.123,0.482){9}{\rule{0.967pt}{0.116pt}}
\multiput(1386.00,746.17)(10.994,6.000){2}{\rule{0.483pt}{0.400pt}}
\multiput(1399.00,753.59)(1.026,0.485){11}{\rule{0.900pt}{0.117pt}}
\multiput(1399.00,752.17)(12.132,7.000){2}{\rule{0.450pt}{0.400pt}}
\multiput(1413.00,760.59)(1.123,0.482){9}{\rule{0.967pt}{0.116pt}}
\multiput(1413.00,759.17)(10.994,6.000){2}{\rule{0.483pt}{0.400pt}}
\multiput(1426.00,766.59)(0.950,0.485){11}{\rule{0.843pt}{0.117pt}}
\multiput(1426.00,765.17)(11.251,7.000){2}{\rule{0.421pt}{0.400pt}}
\put(131.0,131.0){\rule[-0.200pt]{0.400pt}{155.380pt}}
\put(131.0,131.0){\rule[-0.200pt]{315.097pt}{0.400pt}}
\put(1439.0,131.0){\rule[-0.200pt]{0.400pt}{155.380pt}}
\put(131.0,776.0){\rule[-0.200pt]{315.097pt}{0.400pt}}
\end{picture}
}
%  \end	{center}
%  \caption{NaCl freezing point depression versus molality}
%\end {figure}
%\FloatBarrier

%\FloatBarrier
%\begin{figure}[h!]
%	\begin{center}
%	\renewcommand\arraystretch{1.5}
%	\renewcommand\tabcolsep{12pt}
%		\begin{tabular}{|c|c|c|c|}
%			\hline 
%			Trial & 1 & 2& Average \\
%			\hline 
%			$C_{cal}$ & 18.22 $\frac{J}{\textnormal{\textdegree} C}$  & 20.80 $\frac{J}{\textnormal{\textdegree} C}$& 19.51 $\frac{J}{\textnormal{\textdegree} C}$ \\
%			\hline 
%		\end{tabular}
%	\end{center}
%	\caption{Measured Cup Calorimetry Constants (in [J]/[\textdegree\space C])}
%\end{figure}
%\FloatBarrier


\FloatBarrier
%-----------------------------------------------------------------------------------------
%  Discussion
%-----------------------------------------------------------------------------------------
\section{Discussion}\doublespacing
The rate law equation was solved using the initial rates found experimentally, given above in the \textit{Results} section. The rate law was first solved for by finding the reaction orders $m$ and $n$ in the rate equation for this reaction:
\begin{equation}
rate = k'(\textnormal{grams Mg})^m [\textnormal{HCl}]^n
\label{eq:ratel}
\end{equation}
Recall the values for Mg and HCl were chosen so that between the sets of trials there were two where Mg was held constant and HCl and varied, and vice versa. The parameter held constant can be solved for as follows, for example if [HCl] is constant:
\begin{equation}
[\textnormal{HCl}]^n = \frac{\textnormal{rate}}{k'(\textnormal{grams Mg)}^m}
\end{equation}
The initial amount of Mg is known for both trials, and the initial rate was found experimentally, so they can be plugged in as rate$_0$ and (grams Mg)$_0$. Also, since both trials have the same [HCl], the above equation can be set equal to each other for the two trials with initial values substituted in:
\begin{equation}
\frac{\textnormal{rate}_{a,0}}{k' (\textnormal{grams Mg})_{a,0}^m} = \frac{\textnormal{rate}_{b,0}}{k' (\textnormal{grams Mg})_{b,0}^m} 
\end{equation}
Rearanged and simplified:
\begin{equation}
\frac{\textnormal{rate}_{a,0}}{\textnormal{rate}_{b,0}} = \Big(\frac{\textnormal{grams Mg}_{a,0}}{\textnormal{grams Mg}_{b,0}}\Big)^m
\end{equation}
Solve for m:
\begin{equation}
X = \Big(\frac{\textnormal{grams Mg}_{a,0}}{\textnormal{grams Mg}_{b,0}}\Big), \hspace{48pt} m = \textnormal{log}_X\Big(\frac{\textnormal{rate}_{a,0}}{\textnormal{rate}_{b,0}}\Big)
\end{equation}
The equation for n can be found using the same method, or the above equation can be used by substituting in [HCl] for (grams Mg). Applying this formula, it was found that the data for the pH based trials was very poor. m was found to be 0 and and n was 0.58, far from the expected 1 and 2 respectively. These results are poor due to the unstable nature of the measurements made. Looking at Figure \ref{fig:pH_plot}, it can be seen that the pH violently varies initially (instead of monotonically increasing like expected), at the start of the reaction. This occurred in all trials performed. This made it near impossible to determine the real initial rate, and accordingly the rate was just taken at the start of the monotonically increasing region after the ripples, as an approximation. This results in high uncertainty, however, because the rate is taken after the start of the reaction, so the precise amount of reactants are unknown, meaning the values for k', m, n cannot be calculated accurately. It was decided that this method was infeasible for determining the sought after quantities for these reasons, and this method is therefore extremely limited because so. It is probable that this method worked so poorly because the reaction between powdered Mg and HCl is too fast and turbulent, that the solution doesn't mix evenly enough for the pH probe to measure, and the generated gas likely prevents the probe from properly being exposed to solution.\\\par
Applying the equations for m and n to the pressure based data yields much more favourable results of m = 1.029 and n = 1.853, close to the predictions. This is likely due to the data being far more stable from the pressure measurements as they were not done in solution as with the pH ones. Some error is still inherent in these results, however. A primary source would be due to the pressure vessel not being closed until a short time after the HCl was injected. This means that the measurements are not exactly when the reaction initiates, but they are quite close. Unsealed injection of HCl was preferred to sealed because adding the HCl to a sealed vessel will increase the pressure at the same time the reaction runs, making it hard to know how much each the addition of liquid and the reaction contribute to the pressure. It was deemed that open injection would have the least effect on the final results. This experimental method could be extended and improved by making sure the addition of the acid does not affect vessel pressure, making the initial rate measurements more accurate.\\\par 

The rate constant k' was next found by simply plugging in the initial values from each trial into equation \ref{eq:ratel} and solving for k'. Over each of the 22.4 \textdegree C trials, k' was found to be 1.378, which completes the final experimental rate law (equation \ref{eq:presrate}). \\\par

Last, the activation energy E$_a$ was found by using the rate data recorded for the two trials of the same reactant amounts (0.04g Mg and 10mL 1.0M HCl) at different temperatures. The value for k at each temperature was calculated using the previously described methods, but with m = 1 and n = 2. This resulted in k' at 22.4 \textdegree C equalling 1.255, and k' at 8.5 \textdegree C being 1.2225. E$_a$ is easily solved for by using Arrhenius' Equation, solved for E$_a$:
\begin{equation}
\textnormal{E}_a = \frac{\textnormal{ln}\big(\frac{k_1}{k_2}\big)}{\big(\frac{1}{T_2}-\frac{1}{T_1}\big)R}
= \frac{\textnormal{ln}\big(\frac{1.255}{1.2225}\big)}{\big(\frac{1}{8.5+273.15}-\frac{1}{22.4+273.15}\big)\times 8.314} = 18.9 \textnormal{ kJ/mol}
\end{equation}
Therefore, it is seen that E$_a$ is 18.9 kJ/mol, which is near the 28.5 kJ \cite{mg_hcl} found by the similar experiment discussed in the introduction. Since only one point of data was found in this experiment for activation energy, there is high, unquantifiable uncertainty associated with it. This could easily be remedied by performing more trials at different temperatures and different concentrations, and then calculate the average for a more statistically sound value for E$_a$.

%\begin{figure}[h!]
%  \begin{center}
%    	\resizebox{0.6\textwidth}{!}{% GNUPLOT: LaTeX picture
\setlength{\unitlength}{0.240900pt}
\ifx\plotpoint\undefined\newsavebox{\plotpoint}\fi
\sbox{\plotpoint}{\rule[-0.200pt]{0.400pt}{0.400pt}}%
\begin{picture}(1500,900)(0,0)
\sbox{\plotpoint}{\rule[-0.200pt]{0.400pt}{0.400pt}}%
\put(151.0,131.0){\rule[-0.200pt]{4.818pt}{0.400pt}}
\put(131,131){\makebox(0,0)[r]{-5}}
\put(1419.0,131.0){\rule[-0.200pt]{4.818pt}{0.400pt}}
\put(151.0,239.0){\rule[-0.200pt]{4.818pt}{0.400pt}}
\put(131,239){\makebox(0,0)[r]{ 0}}
\put(1419.0,239.0){\rule[-0.200pt]{4.818pt}{0.400pt}}
\put(151.0,346.0){\rule[-0.200pt]{4.818pt}{0.400pt}}
\put(131,346){\makebox(0,0)[r]{ 5}}
\put(1419.0,346.0){\rule[-0.200pt]{4.818pt}{0.400pt}}
\put(151.0,454.0){\rule[-0.200pt]{4.818pt}{0.400pt}}
\put(131,454){\makebox(0,0)[r]{ 10}}
\put(1419.0,454.0){\rule[-0.200pt]{4.818pt}{0.400pt}}
\put(151.0,561.0){\rule[-0.200pt]{4.818pt}{0.400pt}}
\put(131,561){\makebox(0,0)[r]{ 15}}
\put(1419.0,561.0){\rule[-0.200pt]{4.818pt}{0.400pt}}
\put(151.0,669.0){\rule[-0.200pt]{4.818pt}{0.400pt}}
\put(131,669){\makebox(0,0)[r]{ 20}}
\put(1419.0,669.0){\rule[-0.200pt]{4.818pt}{0.400pt}}
\put(151.0,776.0){\rule[-0.200pt]{4.818pt}{0.400pt}}
\put(131,776){\makebox(0,0)[r]{ 25}}
\put(1419.0,776.0){\rule[-0.200pt]{4.818pt}{0.400pt}}
\put(151.0,131.0){\rule[-0.200pt]{0.400pt}{4.818pt}}
\put(151,90){\makebox(0,0){ 0}}
\put(151.0,756.0){\rule[-0.200pt]{0.400pt}{4.818pt}}
\put(312.0,131.0){\rule[-0.200pt]{0.400pt}{4.818pt}}
\put(312,90){\makebox(0,0){ 20}}
\put(312.0,756.0){\rule[-0.200pt]{0.400pt}{4.818pt}}
\put(473.0,131.0){\rule[-0.200pt]{0.400pt}{4.818pt}}
\put(473,90){\makebox(0,0){ 40}}
\put(473.0,756.0){\rule[-0.200pt]{0.400pt}{4.818pt}}
\put(634.0,131.0){\rule[-0.200pt]{0.400pt}{4.818pt}}
\put(634,90){\makebox(0,0){ 60}}
\put(634.0,756.0){\rule[-0.200pt]{0.400pt}{4.818pt}}
\put(795.0,131.0){\rule[-0.200pt]{0.400pt}{4.818pt}}
\put(795,90){\makebox(0,0){ 80}}
\put(795.0,756.0){\rule[-0.200pt]{0.400pt}{4.818pt}}
\put(956.0,131.0){\rule[-0.200pt]{0.400pt}{4.818pt}}
\put(956,90){\makebox(0,0){ 100}}
\put(956.0,756.0){\rule[-0.200pt]{0.400pt}{4.818pt}}
\put(1117.0,131.0){\rule[-0.200pt]{0.400pt}{4.818pt}}
\put(1117,90){\makebox(0,0){ 120}}
\put(1117.0,756.0){\rule[-0.200pt]{0.400pt}{4.818pt}}
\put(1278.0,131.0){\rule[-0.200pt]{0.400pt}{4.818pt}}
\put(1278,90){\makebox(0,0){ 140}}
\put(1278.0,756.0){\rule[-0.200pt]{0.400pt}{4.818pt}}
\put(1439.0,131.0){\rule[-0.200pt]{0.400pt}{4.818pt}}
\put(1439,90){\makebox(0,0){ 160}}
\put(1439.0,756.0){\rule[-0.200pt]{0.400pt}{4.818pt}}
\put(151.0,131.0){\rule[-0.200pt]{0.400pt}{155.380pt}}
\put(151.0,131.0){\rule[-0.200pt]{310.279pt}{0.400pt}}
\put(1439.0,131.0){\rule[-0.200pt]{0.400pt}{155.380pt}}
\put(151.0,776.0){\rule[-0.200pt]{310.279pt}{0.400pt}}
\put(30,453){\makebox(0,0){\hspace{-72pt}Temperature (C)}}
\put(795,29){\makebox(0,0){Time (s)}}
\put(795,838){\makebox(0,0){0.428 m NaCl Solution Freezing Point Depression}}
\put(151,737){\usebox{\plotpoint}}
\put(151,737.17){\rule{0.900pt}{0.400pt}}
\multiput(151.00,736.17)(2.132,2.000){2}{\rule{0.450pt}{0.400pt}}
\put(155,739.17){\rule{0.900pt}{0.400pt}}
\multiput(155.00,738.17)(2.132,2.000){2}{\rule{0.450pt}{0.400pt}}
\put(159,739.67){\rule{0.964pt}{0.400pt}}
\multiput(159.00,740.17)(2.000,-1.000){2}{\rule{0.482pt}{0.400pt}}
\multiput(163.00,738.94)(0.481,-0.468){5}{\rule{0.500pt}{0.113pt}}
\multiput(163.00,739.17)(2.962,-4.000){2}{\rule{0.250pt}{0.400pt}}
\multiput(167.60,732.68)(0.468,-0.920){5}{\rule{0.113pt}{0.800pt}}
\multiput(166.17,734.34)(4.000,-5.340){2}{\rule{0.400pt}{0.400pt}}
\multiput(171.60,726.51)(0.468,-0.627){5}{\rule{0.113pt}{0.600pt}}
\multiput(170.17,727.75)(4.000,-3.755){2}{\rule{0.400pt}{0.300pt}}
\multiput(175.60,720.26)(0.468,-1.066){5}{\rule{0.113pt}{0.900pt}}
\multiput(174.17,722.13)(4.000,-6.132){2}{\rule{0.400pt}{0.450pt}}
\multiput(179.60,713.09)(0.468,-0.774){5}{\rule{0.113pt}{0.700pt}}
\multiput(178.17,714.55)(4.000,-4.547){2}{\rule{0.400pt}{0.350pt}}
\multiput(183.00,708.95)(0.685,-0.447){3}{\rule{0.633pt}{0.108pt}}
\multiput(183.00,709.17)(2.685,-3.000){2}{\rule{0.317pt}{0.400pt}}
\put(187,705.67){\rule{0.964pt}{0.400pt}}
\multiput(187.00,706.17)(2.000,-1.000){2}{\rule{0.482pt}{0.400pt}}
\put(191,704.17){\rule{0.900pt}{0.400pt}}
\multiput(191.00,705.17)(2.132,-2.000){2}{\rule{0.450pt}{0.400pt}}
\put(195,702.17){\rule{0.900pt}{0.400pt}}
\multiput(195.00,703.17)(2.132,-2.000){2}{\rule{0.450pt}{0.400pt}}
\multiput(199.00,700.94)(0.481,-0.468){5}{\rule{0.500pt}{0.113pt}}
\multiput(199.00,701.17)(2.962,-4.000){2}{\rule{0.250pt}{0.400pt}}
\multiput(203.60,695.09)(0.468,-0.774){5}{\rule{0.113pt}{0.700pt}}
\multiput(202.17,696.55)(4.000,-4.547){2}{\rule{0.400pt}{0.350pt}}
\multiput(207.60,689.51)(0.468,-0.627){5}{\rule{0.113pt}{0.600pt}}
\multiput(206.17,690.75)(4.000,-3.755){2}{\rule{0.400pt}{0.300pt}}
\put(211,685.67){\rule{0.964pt}{0.400pt}}
\multiput(211.00,686.17)(2.000,-1.000){2}{\rule{0.482pt}{0.400pt}}
\put(215,685.67){\rule{0.964pt}{0.400pt}}
\multiput(215.00,685.17)(2.000,1.000){2}{\rule{0.482pt}{0.400pt}}
\put(219,685.17){\rule{0.900pt}{0.400pt}}
\multiput(219.00,686.17)(2.132,-2.000){2}{\rule{0.450pt}{0.400pt}}
\multiput(223.60,681.68)(0.468,-0.920){5}{\rule{0.113pt}{0.800pt}}
\multiput(222.17,683.34)(4.000,-5.340){2}{\rule{0.400pt}{0.400pt}}
\multiput(227.59,674.93)(0.477,-0.821){7}{\rule{0.115pt}{0.740pt}}
\multiput(226.17,676.46)(5.000,-6.464){2}{\rule{0.400pt}{0.370pt}}
\multiput(232.60,666.68)(0.468,-0.920){5}{\rule{0.113pt}{0.800pt}}
\multiput(231.17,668.34)(4.000,-5.340){2}{\rule{0.400pt}{0.400pt}}
\multiput(236.60,659.26)(0.468,-1.066){5}{\rule{0.113pt}{0.900pt}}
\multiput(235.17,661.13)(4.000,-6.132){2}{\rule{0.400pt}{0.450pt}}
\multiput(240.60,651.26)(0.468,-1.066){5}{\rule{0.113pt}{0.900pt}}
\multiput(239.17,653.13)(4.000,-6.132){2}{\rule{0.400pt}{0.450pt}}
\multiput(244.60,642.85)(0.468,-1.212){5}{\rule{0.113pt}{1.000pt}}
\multiput(243.17,644.92)(4.000,-6.924){2}{\rule{0.400pt}{0.500pt}}
\multiput(248.60,634.68)(0.468,-0.920){5}{\rule{0.113pt}{0.800pt}}
\multiput(247.17,636.34)(4.000,-5.340){2}{\rule{0.400pt}{0.400pt}}
\multiput(252.60,626.02)(0.468,-1.505){5}{\rule{0.113pt}{1.200pt}}
\multiput(251.17,628.51)(4.000,-8.509){2}{\rule{0.400pt}{0.600pt}}
\multiput(256.60,615.43)(0.468,-1.358){5}{\rule{0.113pt}{1.100pt}}
\multiput(255.17,617.72)(4.000,-7.717){2}{\rule{0.400pt}{0.550pt}}
\multiput(260.60,605.43)(0.468,-1.358){5}{\rule{0.113pt}{1.100pt}}
\multiput(259.17,607.72)(4.000,-7.717){2}{\rule{0.400pt}{0.550pt}}
\multiput(264.60,595.02)(0.468,-1.505){5}{\rule{0.113pt}{1.200pt}}
\multiput(263.17,597.51)(4.000,-8.509){2}{\rule{0.400pt}{0.600pt}}
\multiput(268.60,585.68)(0.468,-0.920){5}{\rule{0.113pt}{0.800pt}}
\multiput(267.17,587.34)(4.000,-5.340){2}{\rule{0.400pt}{0.400pt}}
\multiput(272.60,578.26)(0.468,-1.066){5}{\rule{0.113pt}{0.900pt}}
\multiput(271.17,580.13)(4.000,-6.132){2}{\rule{0.400pt}{0.450pt}}
\multiput(276.60,571.09)(0.468,-0.774){5}{\rule{0.113pt}{0.700pt}}
\multiput(275.17,572.55)(4.000,-4.547){2}{\rule{0.400pt}{0.350pt}}
\multiput(280.60,565.09)(0.468,-0.774){5}{\rule{0.113pt}{0.700pt}}
\multiput(279.17,566.55)(4.000,-4.547){2}{\rule{0.400pt}{0.350pt}}
\multiput(284.60,558.68)(0.468,-0.920){5}{\rule{0.113pt}{0.800pt}}
\multiput(283.17,560.34)(4.000,-5.340){2}{\rule{0.400pt}{0.400pt}}
\multiput(288.60,551.68)(0.468,-0.920){5}{\rule{0.113pt}{0.800pt}}
\multiput(287.17,553.34)(4.000,-5.340){2}{\rule{0.400pt}{0.400pt}}
\multiput(292.60,544.68)(0.468,-0.920){5}{\rule{0.113pt}{0.800pt}}
\multiput(291.17,546.34)(4.000,-5.340){2}{\rule{0.400pt}{0.400pt}}
\multiput(296.60,537.68)(0.468,-0.920){5}{\rule{0.113pt}{0.800pt}}
\multiput(295.17,539.34)(4.000,-5.340){2}{\rule{0.400pt}{0.400pt}}
\multiput(300.60,531.09)(0.468,-0.774){5}{\rule{0.113pt}{0.700pt}}
\multiput(299.17,532.55)(4.000,-4.547){2}{\rule{0.400pt}{0.350pt}}
\multiput(304.60,524.26)(0.468,-1.066){5}{\rule{0.113pt}{0.900pt}}
\multiput(303.17,526.13)(4.000,-6.132){2}{\rule{0.400pt}{0.450pt}}
\multiput(308.60,515.85)(0.468,-1.212){5}{\rule{0.113pt}{1.000pt}}
\multiput(307.17,517.92)(4.000,-6.924){2}{\rule{0.400pt}{0.500pt}}
\multiput(312.60,507.26)(0.468,-1.066){5}{\rule{0.113pt}{0.900pt}}
\multiput(311.17,509.13)(4.000,-6.132){2}{\rule{0.400pt}{0.450pt}}
\multiput(316.00,501.94)(0.481,-0.468){5}{\rule{0.500pt}{0.113pt}}
\multiput(316.00,502.17)(2.962,-4.000){2}{\rule{0.250pt}{0.400pt}}
\multiput(320.60,496.51)(0.468,-0.627){5}{\rule{0.113pt}{0.600pt}}
\multiput(319.17,497.75)(4.000,-3.755){2}{\rule{0.400pt}{0.300pt}}
\multiput(324.60,488.60)(0.468,-1.651){5}{\rule{0.113pt}{1.300pt}}
\multiput(323.17,491.30)(4.000,-9.302){2}{\rule{0.400pt}{0.650pt}}
\multiput(328.60,478.68)(0.468,-0.920){5}{\rule{0.113pt}{0.800pt}}
\multiput(327.17,480.34)(4.000,-5.340){2}{\rule{0.400pt}{0.400pt}}
\multiput(332.00,473.94)(0.481,-0.468){5}{\rule{0.500pt}{0.113pt}}
\multiput(332.00,474.17)(2.962,-4.000){2}{\rule{0.250pt}{0.400pt}}
\multiput(336.60,468.51)(0.468,-0.627){5}{\rule{0.113pt}{0.600pt}}
\multiput(335.17,469.75)(4.000,-3.755){2}{\rule{0.400pt}{0.300pt}}
\multiput(340.00,464.95)(0.685,-0.447){3}{\rule{0.633pt}{0.108pt}}
\multiput(340.00,465.17)(2.685,-3.000){2}{\rule{0.317pt}{0.400pt}}
\put(344,461.17){\rule{0.900pt}{0.400pt}}
\multiput(344.00,462.17)(2.132,-2.000){2}{\rule{0.450pt}{0.400pt}}
\multiput(348.00,459.94)(0.481,-0.468){5}{\rule{0.500pt}{0.113pt}}
\multiput(348.00,460.17)(2.962,-4.000){2}{\rule{0.250pt}{0.400pt}}
\multiput(352.60,452.43)(0.468,-1.358){5}{\rule{0.113pt}{1.100pt}}
\multiput(351.17,454.72)(4.000,-7.717){2}{\rule{0.400pt}{0.550pt}}
\multiput(356.60,443.26)(0.468,-1.066){5}{\rule{0.113pt}{0.900pt}}
\multiput(355.17,445.13)(4.000,-6.132){2}{\rule{0.400pt}{0.450pt}}
\multiput(360.60,436.09)(0.468,-0.774){5}{\rule{0.113pt}{0.700pt}}
\multiput(359.17,437.55)(4.000,-4.547){2}{\rule{0.400pt}{0.350pt}}
\multiput(364.60,430.51)(0.468,-0.627){5}{\rule{0.113pt}{0.600pt}}
\multiput(363.17,431.75)(4.000,-3.755){2}{\rule{0.400pt}{0.300pt}}
\multiput(368.00,426.94)(0.481,-0.468){5}{\rule{0.500pt}{0.113pt}}
\multiput(368.00,427.17)(2.962,-4.000){2}{\rule{0.250pt}{0.400pt}}
\multiput(372.60,421.09)(0.468,-0.774){5}{\rule{0.113pt}{0.700pt}}
\multiput(371.17,422.55)(4.000,-4.547){2}{\rule{0.400pt}{0.350pt}}
\multiput(376.00,416.94)(0.481,-0.468){5}{\rule{0.500pt}{0.113pt}}
\multiput(376.00,417.17)(2.962,-4.000){2}{\rule{0.250pt}{0.400pt}}
\multiput(380.60,411.51)(0.468,-0.627){5}{\rule{0.113pt}{0.600pt}}
\multiput(379.17,412.75)(4.000,-3.755){2}{\rule{0.400pt}{0.300pt}}
\multiput(384.60,406.51)(0.468,-0.627){5}{\rule{0.113pt}{0.600pt}}
\multiput(383.17,407.75)(4.000,-3.755){2}{\rule{0.400pt}{0.300pt}}
\multiput(388.00,402.94)(0.627,-0.468){5}{\rule{0.600pt}{0.113pt}}
\multiput(388.00,403.17)(3.755,-4.000){2}{\rule{0.300pt}{0.400pt}}
\multiput(393.60,397.51)(0.468,-0.627){5}{\rule{0.113pt}{0.600pt}}
\multiput(392.17,398.75)(4.000,-3.755){2}{\rule{0.400pt}{0.300pt}}
\multiput(397.00,393.94)(0.481,-0.468){5}{\rule{0.500pt}{0.113pt}}
\multiput(397.00,394.17)(2.962,-4.000){2}{\rule{0.250pt}{0.400pt}}
\multiput(401.00,389.94)(0.481,-0.468){5}{\rule{0.500pt}{0.113pt}}
\multiput(401.00,390.17)(2.962,-4.000){2}{\rule{0.250pt}{0.400pt}}
\multiput(405.00,385.94)(0.481,-0.468){5}{\rule{0.500pt}{0.113pt}}
\multiput(405.00,386.17)(2.962,-4.000){2}{\rule{0.250pt}{0.400pt}}
\multiput(409.00,381.95)(0.685,-0.447){3}{\rule{0.633pt}{0.108pt}}
\multiput(409.00,382.17)(2.685,-3.000){2}{\rule{0.317pt}{0.400pt}}
\multiput(413.00,378.95)(0.685,-0.447){3}{\rule{0.633pt}{0.108pt}}
\multiput(413.00,379.17)(2.685,-3.000){2}{\rule{0.317pt}{0.400pt}}
\multiput(417.60,374.51)(0.468,-0.627){5}{\rule{0.113pt}{0.600pt}}
\multiput(416.17,375.75)(4.000,-3.755){2}{\rule{0.400pt}{0.300pt}}
\multiput(421.60,369.51)(0.468,-0.627){5}{\rule{0.113pt}{0.600pt}}
\multiput(420.17,370.75)(4.000,-3.755){2}{\rule{0.400pt}{0.300pt}}
\multiput(425.60,364.09)(0.468,-0.774){5}{\rule{0.113pt}{0.700pt}}
\multiput(424.17,365.55)(4.000,-4.547){2}{\rule{0.400pt}{0.350pt}}
\multiput(429.60,358.51)(0.468,-0.627){5}{\rule{0.113pt}{0.600pt}}
\multiput(428.17,359.75)(4.000,-3.755){2}{\rule{0.400pt}{0.300pt}}
\multiput(433.60,353.51)(0.468,-0.627){5}{\rule{0.113pt}{0.600pt}}
\multiput(432.17,354.75)(4.000,-3.755){2}{\rule{0.400pt}{0.300pt}}
\multiput(437.00,349.94)(0.481,-0.468){5}{\rule{0.500pt}{0.113pt}}
\multiput(437.00,350.17)(2.962,-4.000){2}{\rule{0.250pt}{0.400pt}}
\multiput(441.60,344.51)(0.468,-0.627){5}{\rule{0.113pt}{0.600pt}}
\multiput(440.17,345.75)(4.000,-3.755){2}{\rule{0.400pt}{0.300pt}}
\multiput(445.00,340.94)(0.481,-0.468){5}{\rule{0.500pt}{0.113pt}}
\multiput(445.00,341.17)(2.962,-4.000){2}{\rule{0.250pt}{0.400pt}}
\multiput(449.00,336.95)(0.685,-0.447){3}{\rule{0.633pt}{0.108pt}}
\multiput(449.00,337.17)(2.685,-3.000){2}{\rule{0.317pt}{0.400pt}}
\multiput(453.00,333.94)(0.481,-0.468){5}{\rule{0.500pt}{0.113pt}}
\multiput(453.00,334.17)(2.962,-4.000){2}{\rule{0.250pt}{0.400pt}}
\multiput(457.00,329.95)(0.685,-0.447){3}{\rule{0.633pt}{0.108pt}}
\multiput(457.00,330.17)(2.685,-3.000){2}{\rule{0.317pt}{0.400pt}}
\multiput(461.00,326.94)(0.481,-0.468){5}{\rule{0.500pt}{0.113pt}}
\multiput(461.00,327.17)(2.962,-4.000){2}{\rule{0.250pt}{0.400pt}}
\multiput(465.00,322.94)(0.481,-0.468){5}{\rule{0.500pt}{0.113pt}}
\multiput(465.00,323.17)(2.962,-4.000){2}{\rule{0.250pt}{0.400pt}}
\multiput(469.00,318.94)(0.481,-0.468){5}{\rule{0.500pt}{0.113pt}}
\multiput(469.00,319.17)(2.962,-4.000){2}{\rule{0.250pt}{0.400pt}}
\multiput(473.00,314.94)(0.481,-0.468){5}{\rule{0.500pt}{0.113pt}}
\multiput(473.00,315.17)(2.962,-4.000){2}{\rule{0.250pt}{0.400pt}}
\multiput(477.00,310.95)(0.685,-0.447){3}{\rule{0.633pt}{0.108pt}}
\multiput(477.00,311.17)(2.685,-3.000){2}{\rule{0.317pt}{0.400pt}}
\multiput(481.00,307.95)(0.685,-0.447){3}{\rule{0.633pt}{0.108pt}}
\multiput(481.00,308.17)(2.685,-3.000){2}{\rule{0.317pt}{0.400pt}}
\multiput(485.60,303.09)(0.468,-0.774){5}{\rule{0.113pt}{0.700pt}}
\multiput(484.17,304.55)(4.000,-4.547){2}{\rule{0.400pt}{0.350pt}}
\multiput(489.60,297.09)(0.468,-0.774){5}{\rule{0.113pt}{0.700pt}}
\multiput(488.17,298.55)(4.000,-4.547){2}{\rule{0.400pt}{0.350pt}}
\multiput(493.00,292.94)(0.481,-0.468){5}{\rule{0.500pt}{0.113pt}}
\multiput(493.00,293.17)(2.962,-4.000){2}{\rule{0.250pt}{0.400pt}}
\multiput(497.00,288.94)(0.481,-0.468){5}{\rule{0.500pt}{0.113pt}}
\multiput(497.00,289.17)(2.962,-4.000){2}{\rule{0.250pt}{0.400pt}}
\multiput(501.00,284.94)(0.481,-0.468){5}{\rule{0.500pt}{0.113pt}}
\multiput(501.00,285.17)(2.962,-4.000){2}{\rule{0.250pt}{0.400pt}}
\multiput(505.00,280.94)(0.481,-0.468){5}{\rule{0.500pt}{0.113pt}}
\multiput(505.00,281.17)(2.962,-4.000){2}{\rule{0.250pt}{0.400pt}}
\multiput(509.00,276.95)(0.685,-0.447){3}{\rule{0.633pt}{0.108pt}}
\multiput(509.00,277.17)(2.685,-3.000){2}{\rule{0.317pt}{0.400pt}}
\put(513,273.17){\rule{0.900pt}{0.400pt}}
\multiput(513.00,274.17)(2.132,-2.000){2}{\rule{0.450pt}{0.400pt}}
\put(517,271.17){\rule{0.900pt}{0.400pt}}
\multiput(517.00,272.17)(2.132,-2.000){2}{\rule{0.450pt}{0.400pt}}
\put(521,269.67){\rule{0.964pt}{0.400pt}}
\multiput(521.00,270.17)(2.000,-1.000){2}{\rule{0.482pt}{0.400pt}}
\multiput(525.00,268.94)(0.481,-0.468){5}{\rule{0.500pt}{0.113pt}}
\multiput(525.00,269.17)(2.962,-4.000){2}{\rule{0.250pt}{0.400pt}}
\multiput(529.00,264.94)(0.481,-0.468){5}{\rule{0.500pt}{0.113pt}}
\multiput(529.00,265.17)(2.962,-4.000){2}{\rule{0.250pt}{0.400pt}}
\put(533,260.17){\rule{0.900pt}{0.400pt}}
\multiput(533.00,261.17)(2.132,-2.000){2}{\rule{0.450pt}{0.400pt}}
\multiput(537.00,258.95)(0.685,-0.447){3}{\rule{0.633pt}{0.108pt}}
\multiput(537.00,259.17)(2.685,-3.000){2}{\rule{0.317pt}{0.400pt}}
\multiput(541.00,255.94)(0.481,-0.468){5}{\rule{0.500pt}{0.113pt}}
\multiput(541.00,256.17)(2.962,-4.000){2}{\rule{0.250pt}{0.400pt}}
\multiput(545.60,250.51)(0.468,-0.627){5}{\rule{0.113pt}{0.600pt}}
\multiput(544.17,251.75)(4.000,-3.755){2}{\rule{0.400pt}{0.300pt}}
\multiput(549.00,246.95)(0.909,-0.447){3}{\rule{0.767pt}{0.108pt}}
\multiput(549.00,247.17)(3.409,-3.000){2}{\rule{0.383pt}{0.400pt}}
\put(554,243.67){\rule{0.964pt}{0.400pt}}
\multiput(554.00,244.17)(2.000,-1.000){2}{\rule{0.482pt}{0.400pt}}
\multiput(558.00,242.95)(0.685,-0.447){3}{\rule{0.633pt}{0.108pt}}
\multiput(558.00,243.17)(2.685,-3.000){2}{\rule{0.317pt}{0.400pt}}
\put(562,239.17){\rule{0.900pt}{0.400pt}}
\multiput(562.00,240.17)(2.132,-2.000){2}{\rule{0.450pt}{0.400pt}}
\put(566,237.67){\rule{0.964pt}{0.400pt}}
\multiput(566.00,238.17)(2.000,-1.000){2}{\rule{0.482pt}{0.400pt}}
\put(570,236.17){\rule{0.900pt}{0.400pt}}
\multiput(570.00,237.17)(2.132,-2.000){2}{\rule{0.450pt}{0.400pt}}
\put(574,234.17){\rule{0.900pt}{0.400pt}}
\multiput(574.00,235.17)(2.132,-2.000){2}{\rule{0.450pt}{0.400pt}}
\put(578,232.17){\rule{0.900pt}{0.400pt}}
\multiput(578.00,233.17)(2.132,-2.000){2}{\rule{0.450pt}{0.400pt}}
\put(582,230.17){\rule{0.900pt}{0.400pt}}
\multiput(582.00,231.17)(2.132,-2.000){2}{\rule{0.450pt}{0.400pt}}
\put(586,228.67){\rule{0.964pt}{0.400pt}}
\multiput(586.00,229.17)(2.000,-1.000){2}{\rule{0.482pt}{0.400pt}}
\put(590,227.17){\rule{0.900pt}{0.400pt}}
\multiput(590.00,228.17)(2.132,-2.000){2}{\rule{0.450pt}{0.400pt}}
\multiput(594.00,225.95)(0.685,-0.447){3}{\rule{0.633pt}{0.108pt}}
\multiput(594.00,226.17)(2.685,-3.000){2}{\rule{0.317pt}{0.400pt}}
\put(598,222.17){\rule{0.900pt}{0.400pt}}
\multiput(598.00,223.17)(2.132,-2.000){2}{\rule{0.450pt}{0.400pt}}
\put(602,220.17){\rule{0.900pt}{0.400pt}}
\multiput(602.00,221.17)(2.132,-2.000){2}{\rule{0.450pt}{0.400pt}}
\multiput(606.00,218.95)(0.685,-0.447){3}{\rule{0.633pt}{0.108pt}}
\multiput(606.00,219.17)(2.685,-3.000){2}{\rule{0.317pt}{0.400pt}}
\put(610,215.17){\rule{0.900pt}{0.400pt}}
\multiput(610.00,216.17)(2.132,-2.000){2}{\rule{0.450pt}{0.400pt}}
\put(614,213.17){\rule{0.900pt}{0.400pt}}
\multiput(614.00,214.17)(2.132,-2.000){2}{\rule{0.450pt}{0.400pt}}
\put(618,211.67){\rule{0.964pt}{0.400pt}}
\multiput(618.00,212.17)(2.000,-1.000){2}{\rule{0.482pt}{0.400pt}}
\put(622,210.17){\rule{0.900pt}{0.400pt}}
\multiput(622.00,211.17)(2.132,-2.000){2}{\rule{0.450pt}{0.400pt}}
\multiput(626.00,208.95)(0.685,-0.447){3}{\rule{0.633pt}{0.108pt}}
\multiput(626.00,209.17)(2.685,-3.000){2}{\rule{0.317pt}{0.400pt}}
\put(630,205.17){\rule{0.900pt}{0.400pt}}
\multiput(630.00,206.17)(2.132,-2.000){2}{\rule{0.450pt}{0.400pt}}
\put(634,203.17){\rule{0.900pt}{0.400pt}}
\multiput(634.00,204.17)(2.132,-2.000){2}{\rule{0.450pt}{0.400pt}}
\put(638,201.67){\rule{0.964pt}{0.400pt}}
\multiput(638.00,202.17)(2.000,-1.000){2}{\rule{0.482pt}{0.400pt}}
\put(642,200.17){\rule{0.900pt}{0.400pt}}
\multiput(642.00,201.17)(2.132,-2.000){2}{\rule{0.450pt}{0.400pt}}
\put(646,198.67){\rule{0.964pt}{0.400pt}}
\multiput(646.00,199.17)(2.000,-1.000){2}{\rule{0.482pt}{0.400pt}}
\put(654,197.67){\rule{0.964pt}{0.400pt}}
\multiput(654.00,198.17)(2.000,-1.000){2}{\rule{0.482pt}{0.400pt}}
\put(650.0,199.0){\rule[-0.200pt]{0.964pt}{0.400pt}}
\put(662,196.17){\rule{0.900pt}{0.400pt}}
\multiput(662.00,197.17)(2.132,-2.000){2}{\rule{0.450pt}{0.400pt}}
\put(666,194.67){\rule{0.964pt}{0.400pt}}
\multiput(666.00,195.17)(2.000,-1.000){2}{\rule{0.482pt}{0.400pt}}
\multiput(670.00,193.95)(0.685,-0.447){3}{\rule{0.633pt}{0.108pt}}
\multiput(670.00,194.17)(2.685,-3.000){2}{\rule{0.317pt}{0.400pt}}
\put(674,190.17){\rule{0.900pt}{0.400pt}}
\multiput(674.00,191.17)(2.132,-2.000){2}{\rule{0.450pt}{0.400pt}}
\multiput(678.00,188.95)(0.685,-0.447){3}{\rule{0.633pt}{0.108pt}}
\multiput(678.00,189.17)(2.685,-3.000){2}{\rule{0.317pt}{0.400pt}}
\put(682,185.17){\rule{0.900pt}{0.400pt}}
\multiput(682.00,186.17)(2.132,-2.000){2}{\rule{0.450pt}{0.400pt}}
\multiput(686.00,183.95)(0.685,-0.447){3}{\rule{0.633pt}{0.108pt}}
\multiput(686.00,184.17)(2.685,-3.000){2}{\rule{0.317pt}{0.400pt}}
\put(690,180.67){\rule{0.964pt}{0.400pt}}
\multiput(690.00,181.17)(2.000,-1.000){2}{\rule{0.482pt}{0.400pt}}
\put(694,179.17){\rule{0.900pt}{0.400pt}}
\multiput(694.00,180.17)(2.132,-2.000){2}{\rule{0.450pt}{0.400pt}}
\put(698,177.67){\rule{0.964pt}{0.400pt}}
\multiput(698.00,178.17)(2.000,-1.000){2}{\rule{0.482pt}{0.400pt}}
\multiput(702.00,176.95)(0.685,-0.447){3}{\rule{0.633pt}{0.108pt}}
\multiput(702.00,177.17)(2.685,-3.000){2}{\rule{0.317pt}{0.400pt}}
\put(706,173.17){\rule{0.900pt}{0.400pt}}
\multiput(706.00,174.17)(2.132,-2.000){2}{\rule{0.450pt}{0.400pt}}
\multiput(710.00,171.95)(0.909,-0.447){3}{\rule{0.767pt}{0.108pt}}
\multiput(710.00,172.17)(3.409,-3.000){2}{\rule{0.383pt}{0.400pt}}
\put(715,168.67){\rule{0.964pt}{0.400pt}}
\multiput(715.00,169.17)(2.000,-1.000){2}{\rule{0.482pt}{0.400pt}}
\put(719,167.17){\rule{0.900pt}{0.400pt}}
\multiput(719.00,168.17)(2.132,-2.000){2}{\rule{0.450pt}{0.400pt}}
\put(723,166.67){\rule{0.964pt}{0.400pt}}
\multiput(723.00,166.17)(2.000,1.000){2}{\rule{0.482pt}{0.400pt}}
\multiput(727.60,168.00)(0.468,0.774){5}{\rule{0.113pt}{0.700pt}}
\multiput(726.17,168.00)(4.000,4.547){2}{\rule{0.400pt}{0.350pt}}
\multiput(731.60,174.00)(0.468,1.066){5}{\rule{0.113pt}{0.900pt}}
\multiput(730.17,174.00)(4.000,6.132){2}{\rule{0.400pt}{0.450pt}}
\multiput(735.60,182.00)(0.468,1.066){5}{\rule{0.113pt}{0.900pt}}
\multiput(734.17,182.00)(4.000,6.132){2}{\rule{0.400pt}{0.450pt}}
\multiput(739.60,190.00)(0.468,0.774){5}{\rule{0.113pt}{0.700pt}}
\multiput(738.17,190.00)(4.000,4.547){2}{\rule{0.400pt}{0.350pt}}
\multiput(743.60,196.00)(0.468,0.774){5}{\rule{0.113pt}{0.700pt}}
\multiput(742.17,196.00)(4.000,4.547){2}{\rule{0.400pt}{0.350pt}}
\multiput(747.00,202.60)(0.481,0.468){5}{\rule{0.500pt}{0.113pt}}
\multiput(747.00,201.17)(2.962,4.000){2}{\rule{0.250pt}{0.400pt}}
\multiput(751.00,206.61)(0.685,0.447){3}{\rule{0.633pt}{0.108pt}}
\multiput(751.00,205.17)(2.685,3.000){2}{\rule{0.317pt}{0.400pt}}
\put(755,209.17){\rule{0.900pt}{0.400pt}}
\multiput(755.00,208.17)(2.132,2.000){2}{\rule{0.450pt}{0.400pt}}
\put(759,211.17){\rule{0.900pt}{0.400pt}}
\multiput(759.00,210.17)(2.132,2.000){2}{\rule{0.450pt}{0.400pt}}
\put(763,213.17){\rule{0.900pt}{0.400pt}}
\multiput(763.00,212.17)(2.132,2.000){2}{\rule{0.450pt}{0.400pt}}
\put(767,214.67){\rule{0.964pt}{0.400pt}}
\multiput(767.00,214.17)(2.000,1.000){2}{\rule{0.482pt}{0.400pt}}
\put(658.0,198.0){\rule[-0.200pt]{0.964pt}{0.400pt}}
\put(775,216.17){\rule{0.900pt}{0.400pt}}
\multiput(775.00,215.17)(2.132,2.000){2}{\rule{0.450pt}{0.400pt}}
\put(771.0,216.0){\rule[-0.200pt]{0.964pt}{0.400pt}}
\put(783,217.67){\rule{0.964pt}{0.400pt}}
\multiput(783.00,217.17)(2.000,1.000){2}{\rule{0.482pt}{0.400pt}}
\put(779.0,218.0){\rule[-0.200pt]{0.964pt}{0.400pt}}
\put(791,218.67){\rule{0.964pt}{0.400pt}}
\multiput(791.00,218.17)(2.000,1.000){2}{\rule{0.482pt}{0.400pt}}
\put(787.0,219.0){\rule[-0.200pt]{0.964pt}{0.400pt}}
\put(811,219.67){\rule{0.964pt}{0.400pt}}
\multiput(811.00,219.17)(2.000,1.000){2}{\rule{0.482pt}{0.400pt}}
\put(795.0,220.0){\rule[-0.200pt]{3.854pt}{0.400pt}}
\put(847,219.67){\rule{0.964pt}{0.400pt}}
\multiput(847.00,220.17)(2.000,-1.000){2}{\rule{0.482pt}{0.400pt}}
\put(851,219.67){\rule{0.964pt}{0.400pt}}
\multiput(851.00,219.17)(2.000,1.000){2}{\rule{0.482pt}{0.400pt}}
\put(815.0,221.0){\rule[-0.200pt]{7.709pt}{0.400pt}}
\put(867,219.67){\rule{0.964pt}{0.400pt}}
\multiput(867.00,220.17)(2.000,-1.000){2}{\rule{0.482pt}{0.400pt}}
\put(855.0,221.0){\rule[-0.200pt]{2.891pt}{0.400pt}}
\put(904,219.67){\rule{0.964pt}{0.400pt}}
\multiput(904.00,219.17)(2.000,1.000){2}{\rule{0.482pt}{0.400pt}}
\put(871.0,220.0){\rule[-0.200pt]{7.950pt}{0.400pt}}
\put(960,219.67){\rule{0.964pt}{0.400pt}}
\multiput(960.00,220.17)(2.000,-1.000){2}{\rule{0.482pt}{0.400pt}}
\put(908.0,221.0){\rule[-0.200pt]{12.527pt}{0.400pt}}
\put(968,219.67){\rule{0.964pt}{0.400pt}}
\multiput(968.00,219.17)(2.000,1.000){2}{\rule{0.482pt}{0.400pt}}
\put(972,219.67){\rule{0.964pt}{0.400pt}}
\multiput(972.00,220.17)(2.000,-1.000){2}{\rule{0.482pt}{0.400pt}}
\put(964.0,220.0){\rule[-0.200pt]{0.964pt}{0.400pt}}
\put(992,219.67){\rule{0.964pt}{0.400pt}}
\multiput(992.00,219.17)(2.000,1.000){2}{\rule{0.482pt}{0.400pt}}
\put(996,219.67){\rule{0.964pt}{0.400pt}}
\multiput(996.00,220.17)(2.000,-1.000){2}{\rule{0.482pt}{0.400pt}}
\put(1000,219.67){\rule{0.964pt}{0.400pt}}
\multiput(1000.00,219.17)(2.000,1.000){2}{\rule{0.482pt}{0.400pt}}
\put(1004,219.67){\rule{0.964pt}{0.400pt}}
\multiput(1004.00,220.17)(2.000,-1.000){2}{\rule{0.482pt}{0.400pt}}
\put(1008,219.67){\rule{0.964pt}{0.400pt}}
\multiput(1008.00,219.17)(2.000,1.000){2}{\rule{0.482pt}{0.400pt}}
\put(976.0,220.0){\rule[-0.200pt]{3.854pt}{0.400pt}}
\put(1016,219.67){\rule{0.964pt}{0.400pt}}
\multiput(1016.00,220.17)(2.000,-1.000){2}{\rule{0.482pt}{0.400pt}}
\put(1012.0,221.0){\rule[-0.200pt]{0.964pt}{0.400pt}}
\put(1057,219.67){\rule{0.964pt}{0.400pt}}
\multiput(1057.00,219.17)(2.000,1.000){2}{\rule{0.482pt}{0.400pt}}
\put(1061,219.67){\rule{0.964pt}{0.400pt}}
\multiput(1061.00,220.17)(2.000,-1.000){2}{\rule{0.482pt}{0.400pt}}
\put(1020.0,220.0){\rule[-0.200pt]{8.913pt}{0.400pt}}
\put(1089,218.67){\rule{0.964pt}{0.400pt}}
\multiput(1089.00,219.17)(2.000,-1.000){2}{\rule{0.482pt}{0.400pt}}
\put(1093,217.67){\rule{0.964pt}{0.400pt}}
\multiput(1093.00,218.17)(2.000,-1.000){2}{\rule{0.482pt}{0.400pt}}
\put(1097,217.67){\rule{0.964pt}{0.400pt}}
\multiput(1097.00,217.17)(2.000,1.000){2}{\rule{0.482pt}{0.400pt}}
\put(1065.0,220.0){\rule[-0.200pt]{5.782pt}{0.400pt}}
\put(1105,217.67){\rule{0.964pt}{0.400pt}}
\multiput(1105.00,218.17)(2.000,-1.000){2}{\rule{0.482pt}{0.400pt}}
\put(1109,217.67){\rule{0.964pt}{0.400pt}}
\multiput(1109.00,217.17)(2.000,1.000){2}{\rule{0.482pt}{0.400pt}}
\put(1101.0,219.0){\rule[-0.200pt]{0.964pt}{0.400pt}}
\put(1133,218.67){\rule{0.964pt}{0.400pt}}
\multiput(1133.00,218.17)(2.000,1.000){2}{\rule{0.482pt}{0.400pt}}
\put(1137,218.67){\rule{0.964pt}{0.400pt}}
\multiput(1137.00,219.17)(2.000,-1.000){2}{\rule{0.482pt}{0.400pt}}
\put(1113.0,219.0){\rule[-0.200pt]{4.818pt}{0.400pt}}
\put(1181,217.67){\rule{0.964pt}{0.400pt}}
\multiput(1181.00,218.17)(2.000,-1.000){2}{\rule{0.482pt}{0.400pt}}
\put(1141.0,219.0){\rule[-0.200pt]{9.636pt}{0.400pt}}
\put(1214,216.67){\rule{0.964pt}{0.400pt}}
\multiput(1214.00,217.17)(2.000,-1.000){2}{\rule{0.482pt}{0.400pt}}
\put(1185.0,218.0){\rule[-0.200pt]{6.986pt}{0.400pt}}
\put(1226,216.67){\rule{0.964pt}{0.400pt}}
\multiput(1226.00,216.17)(2.000,1.000){2}{\rule{0.482pt}{0.400pt}}
\put(1218.0,217.0){\rule[-0.200pt]{1.927pt}{0.400pt}}
\put(1242,216.67){\rule{0.964pt}{0.400pt}}
\multiput(1242.00,217.17)(2.000,-1.000){2}{\rule{0.482pt}{0.400pt}}
\put(1230.0,218.0){\rule[-0.200pt]{2.891pt}{0.400pt}}
\put(1254,215.67){\rule{0.964pt}{0.400pt}}
\multiput(1254.00,216.17)(2.000,-1.000){2}{\rule{0.482pt}{0.400pt}}
\put(1246.0,217.0){\rule[-0.200pt]{1.927pt}{0.400pt}}
\put(1266,214.67){\rule{0.964pt}{0.400pt}}
\multiput(1266.00,215.17)(2.000,-1.000){2}{\rule{0.482pt}{0.400pt}}
\put(1258.0,216.0){\rule[-0.200pt]{1.927pt}{0.400pt}}
\put(1278,213.67){\rule{0.964pt}{0.400pt}}
\multiput(1278.00,214.17)(2.000,-1.000){2}{\rule{0.482pt}{0.400pt}}
\put(1270.0,215.0){\rule[-0.200pt]{1.927pt}{0.400pt}}
\put(1310,212.67){\rule{0.964pt}{0.400pt}}
\multiput(1310.00,213.17)(2.000,-1.000){2}{\rule{0.482pt}{0.400pt}}
\put(1282.0,214.0){\rule[-0.200pt]{6.745pt}{0.400pt}}
\put(1350,211.67){\rule{0.964pt}{0.400pt}}
\multiput(1350.00,212.17)(2.000,-1.000){2}{\rule{0.482pt}{0.400pt}}
\put(1314.0,213.0){\rule[-0.200pt]{8.672pt}{0.400pt}}
\put(1354.0,212.0){\rule[-0.200pt]{1.204pt}{0.400pt}}
\put(151.0,131.0){\rule[-0.200pt]{0.400pt}{155.380pt}}
\put(151.0,131.0){\rule[-0.200pt]{310.279pt}{0.400pt}}
\put(1439.0,131.0){\rule[-0.200pt]{0.400pt}{155.380pt}}
\put(151.0,776.0){\rule[-0.200pt]{310.279pt}{0.400pt}}
\end{picture}
}
%  \end	{center}
%  \caption{0.428 m NaCl solution freezing point depression}
%\end {figure}
%\FloatBarrier


%-----------------------------------------------------------------------------------------
%  Conculusion
%-----------------------------------------------------------------------------------------
\section{Conclusion}
The rate law of the reaction between HCl and Mg powder was found by measuring the rate of change of pressure (generated by \ce{H2} product) and knowing the initial amounts of reactants. This rate law is:
\begin{equation}
\textnormal{rate}  = 1.38 (\textnormal{grams Mg})^{1.03} [\textnormal{HCl}]^{1.85}
\end{equation}
The reaction orders were single point measurements, so there is no calculable uncertainty. Given three significant figures, the uncertainty in k' is statistically insignificant  at 1.86$\times 10^{-4}$, which is much smaller than the least significant digit of k'. The reaction orders are also very close to the theoretical values of 1 and 2 for Mg and HCl respectively, which implies that the model used for the rate law is valid according to experimental data.\\\par

The activation energy for powdered Mg and HCl reacting was determined to be 18.9 kJ/mol, and there is an unknown level of uncertainty due to being a single point measurement. This value agrees with an externally found value of 28.5 kJ/mol, being off only by -33.4\%. Activation energy is of importance because it quantifies how much energy it takes for a reaction to start. Thus it is important to know it when designing reactions to make sure they will occur (have enough energy to start). These results are significant in that they affirm the model used for the rate law to be true, which provides certainty in its accuracy. Also, this experiment lays out a basic and effective methodology to finding the rate law and activation energy for basic reactions, which may help others in performing similar work.

%  References
%-----------------------------------------------------------------------------------------
\vspace{-12pt}
\begin{thebibliography}{11}
\raggedright

 \bibitem{mg_hcl}
Birk, James P. ; 
Walters, David L.
Pressure measurements to determine the rate law of the magnesium-hydrochloric acid reaction.
Journal of Chemical Education, July, 1993, Vol.70(7), p.587(3)
\bibitem{clock}
Oliveira, André P ; Faria, Roberto B.
The chlorate-iodine clock reaction.
Journal of the American Chemical Society, 28 2005, Vol.127(51), pp.18022-3
	
	
	
\end{thebibliography}
\end{document}