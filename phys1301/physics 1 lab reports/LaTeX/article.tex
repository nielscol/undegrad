\documentclass[12pt, letterpaper]{article}
\usepackage{lipsum}
\usepackage{authblk}
\usepackage[top=1in, bottom=1in, left=1in, right=1in]{geometry}
\usepackage{fancyhdr}
%
\pagestyle{fancy}
%
\renewenvironment{abstract}{%
\hfill\begin{minipage}{0.95\textwidth}
\rule{\textwidth}{1pt}}
{\par\noindent\rule{\textwidth}{1pt}\end{minipage}}
%
\makeatletter
\renewcommand\@maketitle{%
\hfill
\begin{minipage}{0.95\textwidth}
\vskip 2em
\let\footnote\thanks 
{\LARGE \@title \par }
\vskip 1.5em
{\large \@author \par}
\end{minipage}
\vskip 1em \par
}
\makeatother
%
\begin{document}
%
%title and author details
\title{Laboratory I, Problem 7: Acceleration of a Ball with an Initial Velocity}
\author[]{Cole Nielsen}
\affil[]{Physics 1301.301 TA: Yao Meng}
%
\maketitle
%
\begin{abstract}
\textbf{Abstract:} The effect of the initial velocity of an object on its acceleration was tested. A ball was thrown straight down multiple times at different initial velocities and the acceleration was determined using a camera and motion analasys software. The independence of initial velocity and acceleration was observed throuhgh 
\end{abstract}
\section{Introduction}
\lipsum
\end{document}